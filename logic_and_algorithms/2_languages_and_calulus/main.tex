%! TeX root = ./main.tex
\documentclass[11pt, a5paper]{article}
\usepackage[width=10cm, top=0.5cm, bottom=2cm]{geometry}

\usepackage[T1,T2A]{fontenc}
\usepackage[utf8]{inputenc}
\usepackage[english,russian]{babel}
\usepackage{libertine}

\usepackage{amsmath}
\usepackage{amssymb}
\usepackage{amsthm}
\usepackage{mathrsfs}
\usepackage{framed}
\usepackage{xcolor}

\setlength{\parindent}{0pt}

% Force \pagebreak for every section
\let\oldsection\section
\renewcommand\section{\pagebreak\oldsection}

\renewcommand{\thesection}{}
\renewcommand{\thesubsection}{Note \arabic{subsection}}
\renewcommand{\thesubsubsection}{}
\renewcommand{\theparagraph}{}

\newenvironment{note}[1]{\goodbreak\par\subsection{\hfill \color{lightgray}\tiny #1}}{}
\newenvironment{cloze}[2][\ldots]{\begin{leftbar}}{\end{leftbar}}
\newenvironment{icloze}[2][\ldots]{%
  \ignorespaces\text{\tiny \color{lightgray}\{\{c#2::}\hspace{0pt}%
}{%
  \hspace{0pt}\text{\tiny\color{lightgray}\}\}}\unskip%
}


\begin{document}
\section{Логика высказываний}
\begin{note}{4105104fcc464b9f8c59bbd454c55018}
    Высказывания могут быть \begin{icloze}{1}истинными\end{icloze} или \begin{icloze}{1}ложными.\end{icloze}
\end{note}

\begin{note}{de1b54d7432f42a488aa49e8895875f4}
    Высказывания можно \begin{icloze}{2}соединять друг с другом\end{icloze} с помощью \begin{icloze}{1}<<логических связок>>.\end{icloze}
\end{note}

\begin{note}{efe0b8c83a7d42ba956a1aa0c2206d8b}
    Логическая связка \begin{icloze}{2}<<\({ A }\) и \({ B }\)>>\end{icloze} называется \begin{icloze}{1}конъюнкцией.\end{icloze}
\end{note}

\begin{note}{f832b96c89b645348cd5bce12659be41}
    Логическая связка \begin{icloze}{2}<<\({ A }\) и \({ B }\)>>\end{icloze} обозначается
    \[
        \begin{icloze}{1}A \& B\end{icloze} \qquad \begin{icloze}{1}A \land B.\end{icloze}
    \]
\end{note}

\begin{note}{a974b0e29e724094ae11458fec237466}
    Логическая связка \begin{icloze}{2}<<\({ A }\) или \({ B }\)>>\end{icloze} называется \begin{icloze}{1}дизъюнкцией.\end{icloze}
\end{note}

\begin{note}{6d584e09645042b781dbbfaac613fb5b}
    Логическая связка \begin{icloze}{2}<<\({ A }\) или \({ B }\)>>\end{icloze} обозначается
    \[
        \begin{icloze}{1}A \lor B.\end{icloze}
    \]
\end{note}

\begin{note}{71cf2d212d4e4515b861e0c05c459de9}
    Логическая связка \begin{icloze}{2}<<не \({ A }\)>>\end{icloze} называется \begin{icloze}{1}отрицанием.\end{icloze}
\end{note}

\begin{note}{63977dd5c58c492eb2b405f32658fe6e}
    Логическая связка \begin{icloze}{2}<<не \({ A }\)>>\end{icloze} обозначается
    \[
        \begin{icloze}{1}\lnot A\end{icloze} \qquad \begin{icloze}{1}\sim A\end{icloze} \qquad \begin{icloze}{1}\overline{A}.\end{icloze}
    \]
\end{note}

\begin{note}{7974978120c7467e8052614ab9497071}
    Логическая связка \begin{icloze}{2}<<из \({ A }\) следует \({ B }\)>>\end{icloze} называется \begin{icloze}{1}импликацией.\end{icloze}
\end{note}

\begin{note}{5332245f17ca45fc8a382edeafc3f5fd}
    Логическая связка \begin{icloze}{2}<<из \({ A }\) следует \({ B }\)>>\end{icloze} обозначается
    \[
        \begin{icloze}{1}A \to B\end{icloze} \qquad \begin{icloze}{1}A \Rightarrow B\end{icloze} \qquad \begin{icloze}{1}A \supset B.\end{icloze}
    \]
\end{note}

\begin{note}{b3f6c41f719544a3836ef8c752c02311}
    В импликации \({ A \to B }\) \begin{icloze}{2}высказывание \({ A }\)\end{icloze} называется \begin{icloze}{1}посылкой или антецедентом.\end{icloze}
\end{note}

\begin{note}{2f17f543a0fb4c1e872c5e894881a8d8}
    В импликации \({ A \to B }\) \begin{icloze}{2}высказывание \({ B }\)\end{icloze} называется \begin{icloze}{1}заключением или консеквентом.\end{icloze}
\end{note}

\begin{note}{7ffade67cdfc4ba1916748d4d24eb205}
    Значения \begin{icloze}{2}\textbf{И} (истина) и \textbf{Л} (ложь)\end{icloze} называют \begin{icloze}{1}истинностными значениями.\end{icloze}
\end{note}

\begin{note}{6104c0e89dc84b4082b7bf5b1034a41c}
    Истинностное значение \textbf{И} так же обозначают \begin{icloze}{1}\textbf{T} или \({ 1 }\).\end{icloze}
\end{note}

\begin{note}{1ffd99464a964a2d802713627284de05}
    Истинностное значение \textbf{Л} так же обозначают \begin{icloze}{1}\textbf{F} или \({ 0 }\).\end{icloze}
\end{note}

\begin{note}{3a7d57b46b5446cbb1879ca5a88f99f1}
    Говорят, что высказывание имеет значение \begin{icloze}{2}\textbf{И},\end{icloze} если \begin{icloze}{1}оно истинно.\end{icloze}
\end{note}

\begin{note}{39190b33ebc64ffea127054a389ba2fe}
    Говорят, что высказывание имеет значение \begin{icloze}{2}\textbf{Л},\end{icloze} если \begin{icloze}{1}оно ложно.\end{icloze}
\end{note}

\begin{note}{f195f78759bd493fabd6353285403ac6}
    Высказывание \begin{icloze}{2}\({ A \land B }\)\end{icloze} истинно, если \begin{icloze}{1}оба высказывания \({ A }\) и \({ B }\) истинны.\end{icloze}
\end{note}

\begin{note}{4d34320fa4e049b3b9d25a44d54c51d6}
    Высказывание \begin{icloze}{2}\({ A \lor B }\)\end{icloze} истинно, если \begin{icloze}{1}хотя бы одно из высказываний \({ A }\) и \({ B }\) истинно.\end{icloze}
\end{note}

\begin{note}{2dd24cf0c3d7432295ed7a72663db077}
    Высказывание \begin{icloze}{2}\({ A \to B }\)\end{icloze} истинно, если \begin{icloze}{1}\({ A }\) ложно или \({ B }\) истинно.\end{icloze}
\end{note}

\begin{note}{a4c9e67aa6cd4603b5c54c110f4aa726}
    Высказывание \begin{icloze}{2}\({ \lnot A }\)\end{icloze} истинно, если \begin{icloze}{1}\({ A }\) ложно.\end{icloze}
\end{note}

\begin{note}{62f4c5602b68482898cf2f25eee71998}
    \begin{icloze}{2}Элементарные высказывания, из которых составляются более сложные высказывания,\end{icloze} называется \begin{icloze}{1}пропозициональными переменными.\end{icloze}
\end{note}

\begin{note}{50f80c87e0ce459fa19869447ca5cf1c}
    \begin{icloze}{2}Пропозициональные переменные\end{icloze} будем обозначать \begin{icloze}{1}маленькими латинскими буквами.\end{icloze}
\end{note}

\begin{note}{e2ffb750b54545d9a36793717bd9f8c4}
    \begin{icloze}{3}Множество пропозициональных формул\end{icloze} есть \begin{icloze}{2}минимальное надмножество множества пропозициональных переменных,\end{icloze} \begin{icloze}{1}замкнутое относительно логических связок.\end{icloze}
\end{note}

\begin{note}{652a82f3a2c04744b06854eb9802db27}
    Замкнутость относительно каких связок требуется в определении множества пропозициональных формул?

    \begin{cloze}{1}
        <<не>>, <<и>>, <<или>> и импликация.
    \end{cloze}
\end{note}

\begin{note}{49604013e544418bb20cd89b7548c5d4}
    \[
        \begin{icloze}{2}\mathbb B\end{icloze} \overset{\text{def}}= \begin{icloze}{1}\left\{ 0, 1 \right\}.\end{icloze}
    \]
\end{note}

\begin{note}{4ce7564004a54b05a70415e8feffcdb9}
    \begin{icloze}{2}Отображения вида \({ \mathbb B^{n} \to \mathbb B }\)\end{icloze} называют \begin{icloze}{1}булевыми функциями \({ n }\) переменных.\end{icloze}
\end{note}

\begin{note}{366bad41e3824924af52e47078565094}
    Пусть \({ \varphi }\) --- пропозициональная формула, \begin{icloze}{2}содержащая не более \({ n }\) пропозициональных переменных.\end{icloze}
    Тогда \({ \varphi }\) задаёт \begin{icloze}{1}булеву функцию \({ n }\) переменных.\end{icloze}
\end{note}

\begin{note}{fcee92733caa4d679a755c4ee7b1d92d}
    Пусть \({ \varphi }\) --- пропозициональная формула.
    Как вычисляется значение соответствующей булевой функции?

    \begin{cloze}{1}
        Вместо пропозициональных переменных подставляются их истинностные значения.
    \end{cloze}
\end{note}

\begin{note}{fdf2b62e30ae48e2b0a0187aa9129100}
    \begin{icloze}{2}Пропозициональные формулы, истинные при всех значениях их переменных,\end{icloze} называют \begin{icloze}{1}тавтологиями.\end{icloze}
\end{note}

\begin{note}{671a63ca209a42e49d5d782d1da8ff77}
    Две пропозициональные формулы называют \begin{icloze}{2}эквивалентными,\end{icloze} если \begin{icloze}{1}они задают одну и ту же булеву функцию.\end{icloze}
\end{note}

\begin{note}{22583d6fdedc4b14b5ee74452624a641}
    Обязана ли пропозициональная формула содержать все переменные, от которых зависит порождённая ей булева функция?

    \begin{cloze}{1}
        Не обязана.
    \end{cloze}
\end{note}

\begin{note}{3f428c34bf344a849efa7109edd09021}
    Если пропозициональная формула содержит не все переменные, от которых зависит порождённая её булева функция, то \begin{icloze}{1}по некоторых аргументам эта функция постоянна.\end{icloze}
\end{note}

\begin{note}{0c628958376f4d8bb39bc8f93a28b81a}
    Две формулы \({ \varphi }\) и \({ \psi }\) \begin{icloze}{2}эквивалентны\end{icloze} тогда и только тогда, когда
    \begin{icloze}{1}
        \[
            (\varphi \to \psi) \land (\psi \to \varphi) \text{ --- тавтология}.
        \]
    \end{icloze}
\end{note}

\begin{note}{1aedce40881448628c79eedd50b0e6cc}
    \begin{icloze}{2}\({ (p \to q) \land (q \to p) }\)\end{icloze} сокращается как \begin{icloze}{1}\({ p \leftrightarrow q }\).\end{icloze}
\end{note}

\begin{note}{2e54de05c39b423989f92d177f66967d}
    Используя сокращение \({ p \leftrightarrow q }\) можно записывать утверждения об эквивалентности в виде \begin{icloze}{1}тавтологий.\end{icloze}
\end{note}

\begin{note}{4f385dabf08045f9afc40ca97d97d8d1}
    Первые три свойства конъюнкции и дизъюнкции.

    \begin{cloze}{1}
        Ассоциативность, коммутативность, дистрибутивность.
    \end{cloze}
\end{note}

\begin{note}{142ee54823b14d4bad73e4e09dcfac3c}
    Как утверждение о коммутативности конъюнкции записывается в виде тавтологии?

    \begin{cloze}{1}
        \({ (p \land q) \leftrightarrow (p \land p) }\).
    \end{cloze}
\end{note}

\begin{note}{c93d5a120f0341749d62a3705aae96ab}
    \[
        \begin{icloze}{2}\lnot (p \land q)\end{icloze} \leftrightarrow \begin{icloze}{1}(\lnot p \lor \lnot q).\end{icloze}
    \]
\end{note}

\begin{note}{577f39ecc78a42d498fd72413341bdeb}
    \[
        \begin{icloze}{2}\lnot (p \lor q)\end{icloze} \leftrightarrow \begin{icloze}{1}(\lnot p \land \lnot q).\end{icloze}
    \]
\end{note}

\begin{note}{dc0dd2fdc0fe4e3d927f9c9938b1f4ed}
    \[
        \begin{gathered}
            \lnot (p \land q) \leftrightarrow (\lnot p \lor \lnot q), \\
            \lnot (p \lor q) \leftrightarrow (\lnot p \land \lnot q).
        \end{gathered}
    \]

    \begin{center}
        \tiny
        <<\begin{icloze}{1}Законы Де Моргана\end{icloze}>>
    \end{center}
\end{note}

\begin{note}{fc8341e166fb4ed3a69456e055f57d8c}
    \[
        (p \lor (p \land q)) \leftrightarrow \begin{icloze}{1}p.\end{icloze}
    \]
\end{note}

\begin{note}{79971beeb5c749c0aa731ff75be7c184}
    \[
        (p \land (p \lor q)) \leftrightarrow \begin{icloze}{1}p.\end{icloze}
    \]
\end{note}

\begin{note}{74277840f9f74c64a3570075bf441dd3}
    \begin{gather*}
        (p \lor (p \land q)) \leftrightarrow p,
        (p \land (p \lor q)) \leftrightarrow p.
    \end{gather*}

    \begin{center}
        \tiny
        <<\begin{icloze}{1}Законы поглощения\end{icloze}>>
    \end{center}
\end{note}

\begin{note}{b0470e6ee152460e85f6765609b9d4ed}
    \[
        \begin{icloze}{2}(p \to q)\end{icloze} \leftrightarrow \begin{icloze}{1}(\lnot q \to \lnot p).\end{icloze}
    \]

    \begin{center}
        \tiny
        <<\begin{icloze}{3}Правило контрапозиции\end{icloze}>>
    \end{center}
\end{note}

\begin{note}{e7790b37774d4e06845510cd77932583}
    \[
        \lnot \lnot p \leftrightarrow \begin{icloze}{1}p.\end{icloze}
    \]
\end{note}

\begin{note}{cc36fc9a7d8945799fdc56d9124ce34b}
    \[
        \lnot \lnot p \leftrightarrow p.
    \]

    \begin{center}
        \tiny
        <<\begin{icloze}{1}Снятие двойного отрицания\end{icloze}>>
    \end{center}
\end{note}

\begin{note}{12216d5d3e55466898f2f6ea416667e2}
    Пусть две пропозициональные формулы эквивалентны. Что произойдёт, если заменить все \({ \land }\) на \({ \lor }\) и наоборот.

    \begin{cloze}{1}
        Они останутся эквивалентными.
    \end{cloze}
\end{note}

\begin{note}{b89ed5bfb55a4cd190ae423d6659bb18}
    Пусть две пропозициональные формулы эквивалентны. Если заменить все \({ \land }\) на \({ \lor }\) и наоборот, то формулы останутся эквивалентными.
    В чём ключевая идея доказательства?

    \begin{cloze}{1}
        Добавить везде отрицание через законы Де Моргана.
    \end{cloze}
\end{note}

\begin{note}{95d2cc9c35bc48638505c9c8969657e9}
    \begin{icloze}{2}Префиксная\end{icloze} форма записи пропозициональных формул называется \begin{icloze}{1}польской записью.\end{icloze}
\end{note}

\begin{note}{b83adc1541294949a122d22e679e1d56}
    \begin{icloze}{2}Постфиксная\end{icloze} форма записи пропозициональных формул называется \begin{icloze}{1}обратной польской записью.\end{icloze}
\end{note}

\begin{note}{7d133f61d12048fd912eb4388951b4c0}
    Порядок действий в польской нотации \begin{icloze}{1}восстанавливается однозначно.\end{icloze}
\end{note}

\begin{note}{06a383a050d64e979b6880df7567db76}
    Порядок действий в польской нотации восстанавливается однозначно.
    В чём ключевая идея доказательства?

    \begin{cloze}{1}
        Показать однозначность разделения аргументов индукцией по построению.
    \end{cloze}
\end{note}

\end{document}
