%! TeX root = ./main.tex
\documentclass[11pt, a5paper]{article}
\usepackage[width=10cm, top=0.5cm, bottom=2cm]{geometry}

\usepackage[T1,T2A]{fontenc}
\usepackage[utf8]{inputenc}
\usepackage[english,russian]{babel}
\usepackage{libertine}

\usepackage{amsmath}
\usepackage{amssymb}
\usepackage{amsthm}
\usepackage{mathrsfs}
\usepackage{framed}
\usepackage{xcolor}

\setlength{\parindent}{0pt}

% Force \pagebreak for every section
\let\oldsection\section
\renewcommand\section{\pagebreak\oldsection}

\renewcommand{\thesection}{}
\renewcommand{\thesubsection}{Note \arabic{subsection}}
\renewcommand{\thesubsubsection}{}
\renewcommand{\theparagraph}{}

\newenvironment{note}[1]{\goodbreak\par\subsection{\hfill \color{lightgray}\tiny #1}}{}
\newenvironment{cloze}[2][\ldots]{\begin{leftbar}}{\end{leftbar}}
\newenvironment{icloze}[2][\ldots]{%
  \ignorespaces\text{\tiny \color{lightgray}\{\{c#2::}\hspace{0pt}%
}{%
  \hspace{0pt}\text{\tiny\color{lightgray}\}\}}\unskip%
}


\begin{document}
\section{Эквивалентность и порядок}
\begin{note}{f30a93a82cb7436dbf01a4f27b739d36}
    Каким одним требованием можно заменить симметричность и транзитивность в определении отношения эквивалентности?

    \begin{cloze}{1}
        Евклидовость.
    \end{cloze}
\end{note}

\begin{note}{0eb76bd962504c05aac97e46fec59cf8}
    Всегда ли пересечение эквивалентностей есть эквивалентность?

    \begin{cloze}{1}
        Да.
    \end{cloze}
\end{note}

\begin{note}{dd29e40951124bd680e98b6254d5580a}
    Всегда ли объединение эквивалентностей есть эквивалентность?

    \begin{cloze}{1}
        Нет.
    \end{cloze}
\end{note}

\begin{note}{23b76ed0b5e7470fa4c94e0e7d49ec94}
    Как визуально представить фактор-множество по пересечению эквивалентностей?

    \begin{cloze}{1}
         Границы классов ``накладываются'' друг на друга.
    \end{cloze}
\end{note}

\begin{note}{1cc6c8f320a24c858acdeb80dc2e7662}
    Пусть \({ R, S \subseteq A \times A }\) --- отношения эквивалентности.
    Что представляет из себя \({ A / (R \cap S) }\)?

    \begin{cloze}{1}
        Множество всевозможных пересечений классов \({ R }\) и \({ S }\) соответственно.
    \end{cloze}
\end{note}

\begin{note}{509963a76e504adeb2bab8a65ed1c3f9}
    Какая структура рассматривается в теореме Рамсея для бесконечных множеств?

    \begin{cloze}{1}
        Множество \({ k }\)-подмножеств разбито на конечное число классов.
    \end{cloze}
\end{note}

\begin{note}{9d28ae71e4614fcb8fe1ce901861a29b}
    Что мы можем заключить из теоремы Рамсея для бесконечных множеств?

    \begin{cloze}{1}
        Найдётся бесконечное подмножество, все \({ k }\)-под\-мно\-же\-ства которого принадлежат одному классу.
    \end{cloze}
\end{note}

\begin{note}{03c961e533a443ef83dda0c0e73fc61c}
    Интерпретация теоремы Рамсея для бесконечного множества людей\ldots

    \begin{cloze}{1}
        Можно выбрать либо бесконечно много попарно знакомых, либо --- попарно незнакомых.
    \end{cloze}
\end{note}

\begin{note}{d8936dde76084fbfaa621700f57c7cd4}
    Пусть \({ R \subseteq A \times A }\) --- отношение эквивалентности.
    \begin{icloze}{2}Множество классов эквивалентности \({ R }\)\end{icloze} называется \begin{icloze}{1}фактор-множеством множества \({ A }\) по отношению \({ R }\).\end{icloze}
\end{note}

\begin{note}{e212c805b47c40c48f35bdbd5130db2b}
    Бинарное отношение \({ R \subseteq \begin{icloze}{3}A \times A\end{icloze} }\) называется \begin{icloze}{2}называется отношением частичного порядка,\end{icloze} если \begin{icloze}{1}оно рефлексивно, антисимметрично и транзитивно.\end{icloze}
\end{note}

\begin{note}{2a3a6e89d50d41068b22bfd1c595b39a}
    Отношение \begin{icloze}{2}частичного порядка\end{icloze} обычно обозначается символом \begin{icloze}{1}\({ \leqslant }\).\end{icloze}
\end{note}

\begin{note}{90faa1ffef764c7d808d6757d97dfa4b}
    Множество \({ A }\) с \begin{icloze}{2}заданным на нём отношением частичного порядка\end{icloze} называется \begin{icloze}{1}частично упорядоченным множеством.\end{icloze}
\end{note}

\begin{note}{4157aa1725c244a58f3e32a92a0937bb}
    Пусть \({ (A, \leqslant) }\) --- частично упорядоченное множество, \({ x, y \in A }\).
    Говорят, что \begin{icloze}{2}\({ x }\) и \({ y }\) сравнимы,\end{icloze} если \begin{icloze}{1}\({ x \leqslant y }\) или \({ y \leqslant x }\).\end{icloze}
\end{note}

\begin{note}{e75ca87d267f4673a53c15a0e7adcccb}
    Бинарное отношение \({ R \subseteq \begin{icloze}{3}A \times A\end{icloze} }\) называется \begin{icloze}{2}отношением линейного порядка,\end{icloze} если \begin{icloze}{1}\({ R }\) --- отношение частичного порядка и любые \({ x, y \in A }\) сравнимы.\end{icloze}
\end{note}

\begin{note}{79eba4d41c8b4aafa75c4a7c56268adb}
    Множество \({ A }\) с \begin{icloze}{2}заданным на нём отношением линейного порядка\end{icloze} называется \begin{icloze}{1}линейно упорядоченным множеством.\end{icloze}
\end{note}

\begin{note}{e914e0e523ee44139c021af45c63a712}
    Пусть \({ (A, \leqslant) }\) --- частично упорядоченное множество, \({ x, y \in A }\).
    Говорят, что \begin{icloze}{2}\({ x < y }\),\end{icloze} если \begin{icloze}{1}\({ x \leqslant y }\) и \({ x \neq y }\).\end{icloze}
\end{note}

\begin{note}{c264501d4458400e8b0073eac66b95fe}
    Пусть \({ (A, \leqslant) }\) --- частично упорядоченное множество.
    Во избежание путаницы, отношение \begin{icloze}{2}\({ < }\)\end{icloze} называют отношением \begin{icloze}{1}строгого\end{icloze} порядка.
\end{note}

\begin{note}{ec44ba694d2541deaae260221aaafdc5}
    Пусть \({ (A, \leqslant) }\) --- частично упорядоченное множество.
    Во избежание путаницы, отношение \begin{icloze}{2}\({ \leqslant }\)\end{icloze} называют отношением \begin{icloze}{1}нестрого\end{icloze} порядка.
\end{note}

\begin{note}{962a3744a3cc4153bd9317aab2cb46cb}
    Пусть \({ (A, \leqslant) }\) --- частично упорядоченное множество.
    Мы читаем знак \({ < }\) как \begin{icloze}{1}<<меньше>>.\end{icloze}
\end{note}

\begin{note}{850b05ff29334d869b6a9c7e96eef9a9}
    Пусть \({ (A, \leqslant) }\) --- частично упорядоченное множество.
    Мы читаем знак \({ \leqslant }\) как \begin{icloze}{1}<<меньше или равно>>.\end{icloze}
\end{note}

\begin{note}{0e5d3d3ef97541309f99f132d7d20073}
    Пусть \({ (A, \leqslant) }\) --- частично упорядоченное множество, \({ x, y \in A }\).
    Тогда \begin{icloze}{2}\({ x \leqslant y }\)\end{icloze} \begin{icloze}{3}тогда и только тогда, когда\end{icloze} \begin{icloze}{1}\({ x < y }\)\end{icloze} или \begin{icloze}{1}\({ x = y }\).\end{icloze}
\end{note}

\begin{note}{9b75255301e143ba94b347847852b33f}
    Пусть \({ (A, \leqslant) }\) --- частично упорядоченное множество.
    Является ли отношение \({ < }\) рефлексивным?

    \begin{cloze}{1}
        Нет.
    \end{cloze}
\end{note}

\begin{note}{fcc7c32a4ca7455dbcd3260a478ecd97}
    Пусть \({ (A, \leqslant) }\) --- частично упорядоченное множество.
    Является ли отношение \({ < }\) антирефлексивным?

   \begin{cloze}{1}
       Да.
   \end{cloze}
\end{note}

\begin{note}{2d5bf110950f42b4bc343f143b82dfc8}
    Пусть \({ (A, \leqslant) }\) --- частично упорядоченное множество.
    Является ли отношение \({ < }\) транзитивным?

    \begin{cloze}{1}
        Да.
    \end{cloze}
\end{note}

\begin{note}{378780d3b9d74367a71bdf0fb3f67e9f}
    Пусть \({ (A, \leqslant) }\) --- частично упорядоченное множество.
    Является ли отношение \({ < }\) асимметричным?

    \begin{cloze}{1}
        Да.
    \end{cloze}
\end{note}

\begin{note}{f4e2e2fe9c8140a6b8fcda896dd5da35}
    Пусть \({ (A, \leqslant) }\) --- частично упорядоченное множество, \({ x, y \in A }\).
    Тогда если \begin{icloze}{2}\({ x \leqslant y \leqslant x }\),\end{icloze} то \begin{icloze}{1}\({ x = y }\).\end{icloze}
\end{note}

\begin{note}{1ca369e310d2477782f82089ab512891}
    Пусть \({ (A, \leqslant) }\) --- частично упорядоченное множество, \({ x, y \in A }\).
    Тогда если \({ x \leqslant y \leqslant x }\), то \({ x = y }\).
    В чём ключевая идея доказательства?

    \begin{cloze}{1}
        Антисимметричность.
    \end{cloze}
\end{note}

\begin{note}{0af7ee8e9a5c4ad88db6ea371bee9527}
    Пусть \({ (A, \leqslant) }\) --- частично упорядоченное множество, \({ x, y \in A }\).
    Почему не стоит читать \({ x \leqslant y }\) как <<\({ x }\) не больше \({ y }\)>>?

    \begin{cloze}{1}
        \({ \overline{x \geqslant y} \:\:\not\!\!\!\implies x \leqslant y }\), если порядок не линеен.
    \end{cloze}
\end{note}

\begin{note}{414d948920404634bec1fec01bd9b0b2}
    Бинарное отношение \({ R \subseteq \begin{icloze}{3}A \times A\end{icloze} }\) называется \begin{icloze}{2}называется отношением предпорядка,\end{icloze} если \begin{icloze}{1}оно рефлексивно и транзитивно.\end{icloze}
\end{note}

\begin{note}{5a0d3dae2151442795c045fdb2e1ba7f}
    Пусть \({ \leqslant }\) --- \begin{icloze}{3}предпорядок\end{icloze} на множестве \({ A }\).
    Тогда \({ \leqslant }\) задаёт естественное \begin{icloze}{2}отношение частичного порядка\end{icloze}
    \begin{icloze}{1}
        на фактор множестве \({ A }\) по отношению
        \[
            x \leqslant y \text{ и } y \leqslant x.
        \]
    \end{icloze}
\end{note}

\begin{note}{ac3052b941bd4ae981f8d3559789c7e0}
    Пусть \({ (A, \leqslant) }\)--- частично упорядоченное множество, \begin{icloze}{3}\({ B \subseteq A }\).\end{icloze}
    \begin{icloze}{2}Частичный порядок \({ (\leqslant) \cap B^2 }\)\end{icloze} называется \begin{icloze}{1}частичным порядком на \({ B }\), индуцированным из \({ A }\).\end{icloze}
\end{note}

\begin{note}{01c0ab122f3d4940ab98f766e6b357c2}
    Пусть \({ (A, \leqslant) }\)--- частично упорядоченное множество, \({ B \subseteq A }\).
    \begin{icloze}{2}Частичный порядок на \({ B }\), индуцированный из \({ A }\),\end{icloze} обозначается \begin{icloze}{1}\({ \leqslant_{B} }\).\end{icloze}
\end{note}

\begin{note}{2a1949206b6843f8859d96feb5f3d640}
    Пусть \({ (A, \leqslant) }\)--- частично упорядоченное множество, \({ B \subseteq A }\).
    Если \begin{icloze}{2}\({ \leqslant }\) линеен,\end{icloze} то \begin{icloze}{1}и \({ \leqslant_{B} }\) линеен.\end{icloze}
\end{note}

\begin{note}{76f003723d594be4bb2f9df3ea565469}
    Пусть \({ X }\) и \({ Y }\) --- два множества.
    Что есть множество \({ X + Y }\)?

    \begin{cloze}{1}
        Объединение непересекающихся копий \({ X }\) и \({ Y }\).
    \end{cloze}
\end{note}

\begin{note}{c18350dfc5244bddb59d2f27a28a33ae}
    Пусть \({ X }\) и \({ Y }\) --- два множества.
    Если \({ X }\) и \({ Y }\) пересекаются, то как они разделяются в \({ X + Y }\)?

    \begin{cloze}{1}
        Элементы из \({ Y }\) записываются с чертой (как вариант).
    \end{cloze}
\end{note}

\begin{note}{b9a67a4ebd2542d6b6cf86b0d4505d81}
    Пусть \({ X }\) и \({ Y }\) --- два частично упорядоченных множества.
    Как задаётся порядок на \({ X + Y }\)?

    \begin{cloze}{1}
        Внутри \({ X }\) и \({ Y }\) порядок обычный и \({ x \leqslant \overline{y} }\).
    \end{cloze}
\end{note}

\begin{note}{caa0e9e0bd594f80aff6ab6d1711059e}
    Пусть \({ X }\) и \({ Y }\) --- два частично упорядоченных множества.
    При каком условии порядок на \({ X + Y }\) будет линейным?

    \begin{cloze}{1}
        Только если порядки на \({ X }\) и \({ Y }\) линейны.
    \end{cloze}
\end{note}

\begin{note}{2f6406e74c4443c191caa1532527294f}
    Пусть \({ X }\) и \({ Y }\) --- два частично упорядоченных множества.
    Как определятся покоординатное сравнение на \({ X \times Y }\)?

    \begin{cloze}{1}
        Первая координаты \({ \leqslant_{X} }\) и вторые \({ \leqslant_{Y} }\).
    \end{cloze}
\end{note}

\begin{note}{fec1ac2eb92d496e9677d8011742333e}
    Пусть \({ X }\) и \({ Y }\) --- два частично упорядоченных множества.
    В чём недостаток покоординатного сравнения на \({ X \times Y }\)?

    \begin{cloze}{1}
        Он не линеен.
    \end{cloze}
\end{note}

\begin{note}{f80cf291316d430489b6ed3f5ea1f116}
    Пусть \({ X }\) и \({ Y }\) --- два частично упорядоченных множества.
    Как определятся порядок на \({ X \times Y }\)?

    \begin{cloze}{1}
        Аналогично лексикографическому порядку.
    \end{cloze}
\end{note}

\begin{note}{a3950be00a93447ba48cc93e24fc1436}
    Сколько различных линейных порядков на множестве из \({ n }\) элементов?

    \begin{cloze}{1}
        \({ n! }\)
    \end{cloze}
\end{note}

\begin{note}{5dc17edbd7424921a6801b645ec91847}
    Всякий ли частичный порядок на конечном множестве можно продолжить до линейного?

    \begin{cloze}{1}
        Да.
    \end{cloze}
\end{note}

\begin{note}{8ee88fb845d94611a7bad34e78b447b0}
    Всякий ли частичный порядок на бесконечном множестве можно продолжить до линейного?

    \begin{cloze}{1}
        Да.
    \end{cloze}
\end{note}

\begin{note}{d5d18db2bf744c3f95cc4c390a2b63fa}
    Всякий частичный порядок на конечном множестве можно продолжить до линейного.
    В чём ключевая идея доказательства?

    \begin{cloze}{1}
        По индукции выбирать минимальный элемент.
    \end{cloze}
\end{note}

\begin{note}{971c66ef783547948c700b1551bcbf50}
    Пусть \({ X }\) --- \begin{icloze}{3}бесконечное\end{icloze} частично упорядоченное множество.
    Тогда найдётся \begin{icloze}{2}бесконечное подмножество \({ X }\),\end{icloze} элементы которого либо \begin{icloze}{1}все сравнимы,\end{icloze} либо \begin{icloze}{2}все несравнимы.\end{icloze}
\end{note}

\begin{note}{eb211955acee4c57a93149e9322d4c94}
    Пусть \({ X }\) --- бесконечное частично упорядоченное множество.
    Тогда найдётся бесконечное подмножество \({ X }\), элементы которого либо все сравнимы, либо все несравнимы.
    В чём ключевая идея доказательства?

    \begin{cloze}{1}
        Теорема Рамсея для разбиения множества пар по сравнимости.
    \end{cloze}
\end{note}

\begin{note}{53391ec9f26040328fc75be035ca15c7}
    Какой элемент частично упорядоченного множества называется наибольшим?

    \begin{cloze}{1}
        Тот, что больше любого другого элемента.
    \end{cloze}
\end{note}

\begin{note}{1fbc4c3d075344fb8889b64fc11d73cc}
    Какой элемент частично упорядоченного множества называется максимальным?

    \begin{cloze}{1}
        Тот, для которого не существует большего элемента.
    \end{cloze}
\end{note}

\begin{note}{ba81375ada8245dc86f019d40cfc73b2}
    При каком условии понятия наибольшего и максимального элемента совпадают?

    \begin{cloze}{1}
        Если порядок линеен.
    \end{cloze}
\end{note}

\begin{note}{df72779f7c5049dd82bb27f993c2a177}
    Сколько наибольших элементов может существовать у произвольного частично упорядоченного множества?

    \begin{cloze}{1}
        Не более одного.
    \end{cloze}
\end{note}

\begin{note}{dd861e24b4c246bc9769d4d9d8c1684a}
    Сколько максимальных элементов может существовать у произвольного частично упорядоченного множества?

    \begin{cloze}{1}
        Сколь угодно.
    \end{cloze}
\end{note}

\begin{note}{fe2b7f694c644bfbbab346e18477d765}
    Какой элемент частично упорядоченного множества называется наименьшим?

    \begin{cloze}{1}
        Тот, что меньше любого другого элемента.
    \end{cloze}
\end{note}

\begin{note}{f42b3c3570904de282e2a8bfe96a337a}
    Какой элемент частично упорядоченного множества называется минимальным?

    \begin{cloze}{1}
        Тот, для которого не существует меньшего элемента.
    \end{cloze}
\end{note}

\begin{note}{0d235255b51b40e2970c6f0c00ee671e}
    Любые два различных максимальных элемента \begin{icloze}{1}не сравнимы.\end{icloze}
\end{note}

\begin{note}{e6a3453de9ae404f80ca583e6ea036c5}
    Пусть \({ X }\) --- частично упорядоченное множество и \begin{icloze}{2}\({ X }\) конечно.\end{icloze}
    Для любого \({ x \in X }\) найдётся максимальный элемент \begin{icloze}{1}\({ \geqslant x }\).\end{icloze}
\end{note}

\section{Изоморфизмы}
\begin{note}{110d1d04fa2246daa69b785b7fd393fe}
    Пусть \({ A, B }\) --- частично упорядоченные множества, \({ f : A \to B }\).
    Отображение \({ f }\) называется \begin{icloze}{2}изоморфизмом,\end{icloze} если \begin{icloze}{1}оно биективно и сохраняет порядок.\end{icloze}
\end{note}

\begin{note}{13d514bd40fc4478a6ed3a4ab34ff195}
    Пусть \({ A, B }\) --- частично упорядоченные множества.
    Множества \({ A }\) и \({ B }\) называют \begin{icloze}{2}изоморфными,\end{icloze} если \begin{icloze}{1}существует изоморфизм \({ f : A \to B }\).\end{icloze}
\end{note}

\begin{note}{007687446bf044fb8b2643fb03f6dcd9}
    Все частично упорядоченные множества разбиваются на классы изоморфных, называемые \begin{icloze}{1}порядковыми типами.\end{icloze}
\end{note}

\begin{note}{b680b5e72c204aad8c291c056457dc4d}
    \begin{icloze}{1}Конечные линейно\end{icloze} упорядоченные множества \begin{icloze}{2}из одинакового числа элементов\end{icloze} \begin{icloze}{3}изоморфны.\end{icloze}
\end{note}

\begin{note}{5083f28c1c664d26a48aba53a9beae8c}
    Конечные линейно упорядоченные множества из одинакового числа элементов изоморфны.
    В чём ключевая идея доказательства?

    \begin{cloze}{1}
        Построить изоморфизм в \({ \left\{ 1, 2, \ldots, n \right\} }\), начиная с наименьшего элемента.
    \end{cloze}
\end{note}

\begin{note}{743cdca62182497686915562c74cf65d}
    Вещественная последовательность называется \begin{icloze}{2}финитной,\end{icloze} если \begin{icloze}{1}все её члены, кроме конечного числа, равны \({ 0 }\).\end{icloze}
\end{note}

\begin{note}{fea26e4dd50b4a68ba7f6f90da06332b}
    Множестве всех финитных последовательностей в \begin{icloze}{4}\({ \mathbb Z_+ }\)\end{icloze} с \begin{icloze}{3}заданным на нём покомпонентным порядком\end{icloze} изоморфно \begin{icloze}{2}\({ \mathbb N }\)\end{icloze} с отношением \begin{icloze}{1}<<быть делителем>>.\end{icloze}
\end{note}

\begin{note}{f5d322f891eb4f7797645163e5f45497}
    Как изоморфизм частично упорядоченных множеств действует на наибольший элемент?

    \begin{cloze}{1}
        Переводит его в наибольший элемент.
    \end{cloze}
\end{note}

\begin{note}{2793f2d5da5d4b4397c20d874b1fc14e}
    Пусть \({ A }\) --- частично упорядоченное множество.
    \begin{icloze}{2}Изоморфизм \({ A \to A }\)\end{icloze} называется \begin{icloze}{1}автоморфизмом \({ A }\).\end{icloze}
\end{note}

\begin{note}{5995cccf4e2848eebf953a18d27ff7cf}
    Любой автоморфизм частично упорядоченного множества \({ \mathbb N }\) \begin{icloze}{1}является тождественным отображением.\end{icloze}
\end{note}

\begin{note}{9a1b8384b96b4c1b80082eaa74d708ad}
    Любой автоморфизм частично упорядоченного множества \({ \mathbb N }\) является тождественным отображением.
    В чём ключевая идея доказательства?

    \begin{cloze}{1}
        По индукции \({ f(n) = n }\).
    \end{cloze}
\end{note}

\begin{note}{9ec39c6ae7d94216b153421659205a07}
    Пусть \({ A }\) --- \({ k }\)-элементное множество и \({ \mathcal P(A) }\) упорядоченно по включению.
    Тогда
    \[
        \left\lvert \operatorname{Aut} \mathcal P(A) \right\rvert = \begin{icloze}{1}k!.\end{icloze}
    \]
\end{note}

\begin{note}{b8bb9cd707674a4a8df6e43759264a4e}
    Пусть \({ A }\) --- \({ k }\)-элементное множество и \({ \mathcal P(A) }\) упорядоченно по включению.
    Тогда \({ \left\lvert \operatorname{Aut} \mathcal P(A) \right\rvert = k! }\).
    В чём ключевая идея доказательства?

    \begin{cloze}{1}
        Автоморфизм определяется его действием на одно\-э\-ле\-мен\-тных множествах.
    \end{cloze}
\end{note}

\begin{note}{0c815486f4924485acded03c5bcfcbdd}
    Пусть \({ \mathbb N }\) упорядоченно отношением <<быть делителем>>.
    Тогда
    \[
        \left\lvert \operatorname{Aut} \mathbb N \right\rvert = \begin{icloze}{1}\mathfrak{c}.\end{icloze}
    \]
\end{note}

\begin{note}{3e9541105db840b299859a32c0f7ceb9}
    Пусть \({ \mathbb N }\) упорядоченно отношением <<быть делителем>>.
    Тогда \({ \left\lvert \operatorname{Aut} \mathbb N \right\rvert = \mathfrak{c} }\).
    В чём ключевая идея доказательства?

    \begin{cloze}{1}
        Можно ``перемешать'' простые числа.
    \end{cloze}
\end{note}

\begin{note}{c0d051d05a0b4228902a6d0ea3506209}
    Изоморфен ли \({ ([0, 1], \leqslant) }\) множеству \({ (\mathbb R, \leqslant) }\)?

    \begin{cloze}{1}
        Нет.
    \end{cloze}
\end{note}

\begin{note}{6e4e63c5d18f46dd9e164c78f193c40e}
    Почему \({ ([0, 1], \leqslant) }\) не изоморфен \({ (\mathbb R, \leqslant) }\)?

    \begin{cloze}{1}
        В \({ \mathbb R }\) нет наибольшего элемента.
    \end{cloze}
\end{note}

\begin{note}{9a8f1a6138c949ba91eee96998f166b7}
    Изоморфно ли \({ (\mathbb Z, \leqslant) }\) множеству \({ (\mathbb Q, \leqslant) }\)?

    \begin{cloze}{1}
        Нет.
    \end{cloze}
\end{note}

\begin{note}{87d5dfed83174bec88c290c54882ec3a}
    Почему \({ (\mathbb Z, \leqslant) }\) не изоморфно \({ (\mathbb Q, \leqslant) }\)?

    \begin{cloze}{1}
        \({ \mathbb Q }\) плотно в \({ \mathbb R }\).
    \end{cloze}
\end{note}

\begin{note}{2b69523f72144db68161c333b4e5ec82}
    Изоморфны ли \({ (\mathbb Z, \leqslant) }\) и \({ (\mathbb Z + \mathbb Z, \leqslant) }\)?

    \begin{cloze}{1}
        Нет.
    \end{cloze}
\end{note}

\begin{note}{23d161737e134362b37a92013b77c5d8}
    Почему \({ (\mathbb Z, \leqslant) }\) и \({ (\mathbb Z + \mathbb Z, \leqslant) }\) не изоморфны?

    \begin{cloze}{1}
        Между \({ 0 }\) и \({ \overline{0} }\) бесконечно много элементов, что невозможно в \({ \mathbb Z }\).
    \end{cloze}
\end{note}

\begin{note}{360028f6f71143e99dc483ac687c9383}
    Изоморфны ли \({ (\mathbb N, \leqslant) }\) и \({ (\mathbb Z, \leqslant) }\)?

    \begin{cloze}{1}
        Нет.
    \end{cloze}
\end{note}

\begin{note}{7a70e9e2377c4257b8efb486b7967d89}
    Почему \({ (\mathbb N, \leqslant) }\) и \({ (\mathbb Z, \leqslant) }\) не изоморфны?

    \begin{cloze}{1}
        В \({ \mathbb N }\) есть наименьший элемент.
    \end{cloze}
\end{note}

\begin{note}{697e50108ea547c3a372fb441f7d1448}
    Как можно визуально представить \({ (\mathbb Z \times \mathbb N, \leqslant) }\)?

    \begin{cloze}{1}
        Последовательность непересекающихся ``столбцов''
        \[
            \mathbb Z \times \left\{ i \right\}.
        \]
    \end{cloze}
\end{note}

\begin{note}{0c16841754c3437c88d0e4c139976fac}
    Как представить \({ (\mathbb Z \times \mathbb N, \leqslant) }\) в виде суммы?

    \begin{cloze}{1}
        \({ \mathbb Z + \mathbb Z + \mathbb Z + \ldots }\)
    \end{cloze}
\end{note}

\begin{note}{1a67db9b47ea40bf95ec1d08c9e9c699}
    Изоморфны ли \({ (\mathbb Z \times \mathbb N, \leqslant) }\) и \({ (\mathbb Z \times \mathbb Z, \leqslant) }\)?

    \begin{cloze}{1}
        Нет.
    \end{cloze}
\end{note}

\begin{note}{19a444a43c154a90be9f954bb3b05480}
    Почему \({ (\mathbb Z \times \mathbb N, \leqslant) }\) и \({ (\mathbb Z \times \mathbb Z, \leqslant) }\) не изоморфны?

    \begin{cloze}{1}
        От обратного и каждому ``столбцу'' в \({ \mathbb Z \times \mathbb N }\) соответствует ``столбец'' в \({ \mathbb Z \times \mathbb Z }\).
    \end{cloze}
\end{note}

\begin{note}{8a9b0bca924d47f4b16ce8a629424644}
    Допустим, что \({ f }\) --- изоморфизм \({ (\mathbb Z \times \mathbb N, \leqslant) }\) и \({ (\mathbb Z \times \mathbb Z, \leqslant) }\).
    Как показать, что \({ f }\) сопоставляет ``столбцу'' в \({ \mathbb Z \times \mathbb N }\) ``столбец'' в \({ \mathbb Z \times \mathbb Z }\).

    \begin{cloze}{1}
        Между элементами одного столбца есть лишь конечное число других элементов.
    \end{cloze}
\end{note}

\begin{note}{100d9aff0d0a4aafa0d2cf77b58ef9e3}
    Изоморфны ли \({ (\mathbb N \times \mathbb Z, \leqslant) }\) и \({ (\mathbb Z \times \mathbb Z, \leqslant) }\)?

    \begin{cloze}{1}
        Нет.
    \end{cloze}
\end{note}

\begin{note}{422d3d1f24854d10abc217203fc5af30}
    Почему \({ (\mathbb N \times \mathbb Z, \leqslant) }\) и \({ (\mathbb Z \times \mathbb Z, \leqslant) }\) не изоморфны?

    \begin{cloze}{1}
        От обратного и любому ``столбцу'' в \({ \mathbb N \times \mathbb Z }\) соответствует ``столбец'' в \({ \mathbb Z \times \mathbb Z }\).
    \end{cloze}
\end{note}

\begin{note}{2554b1cb3c6c4f17a2aae72ec3237ec4}
    Изоморфны ли \({ (\mathbb Q \times \mathbb N, \leqslant) }\) и \({ (\mathbb Q \times \mathbb Z, \leqslant) }\)?

    \begin{cloze}{1}
        Да.
    \end{cloze}
\end{note}

\begin{note}{b81b5778d5b84d9692f1a7e9b2a7213e}
    Почему \({ (\mathbb Q \times \mathbb N, \leqslant) }\) и \({ (\mathbb Q \times \mathbb Z, \leqslant) }\) изоморфны?

    \begin{cloze}{1}
        Можно разделить \({ \mathbb Q \times \left\{ 0 \right\} }\) на интервалы с иррациональными границами.
    \end{cloze}
\end{note}

\begin{note}{38a9f34456fe49d897d1266a579abb34}
    Изоморфны ли упорядоченные множества рациональных точек интервалов \({ (0, 1) }\) и \({ (0, \sqrt{2}) }\)?

    \begin{cloze}{1}
        Да.
    \end{cloze}
\end{note}

\begin{note}{349e19c69cbf4e538962f2b5647c9ec8}
    Упорядоченные множества рациональных точек интервалов \({ (0, 1) }\) и \({ (0, \sqrt{2}) }\) изоморфны.
    В чём ключевая идея доказательства?

    \begin{cloze}{1}
        Выбрать строго возрастающие последовательности, сходящиеся к \({ 1 }\) и к \({ \sqrt{2} }\), и построить кусочно-линейную функцию.
    \end{cloze}
\end{note}

\begin{note}{9935ca5d5b984b1d8a3abdb6a136e45e}
    Для каких упорядоченных множеств вводят понятие соседних элементов?

    \begin{cloze}{1}
        Для линейно упорядоченных.
    \end{cloze}
\end{note}

\begin{note}{6c9efbfcd1f248499b3684e875f90187}
    Какие два элемента линейно упорядоченного множества называются соседними?

    \begin{cloze}{1}
        \({ x < y }\) и не существует элемента между ними.
    \end{cloze}
\end{note}

\begin{note}{5257d91e2bd644a394471f45f956b381}
    \begin{icloze}{3}Линейно\end{icloze} упорядоченное множество называется \begin{icloze}{2}плотным,\end{icloze} если \begin{icloze}{1}в нём нет соседних элементов.\end{icloze}
\end{note}

\begin{note}{fbb41cd3433e4f518aad0470091368f8}
    \begin{icloze}{4}Любые\end{icloze} два \begin{icloze}{3}счётных плотных линейно\end{icloze} упорядоченных множества \begin{icloze}{2}без наибольшего и наименьшего элементов\end{icloze} \begin{icloze}{1}изоморфны.\end{icloze}
\end{note}

\begin{note}{897a90582ea34029857e54219cdaae9b}
    Любые два счётных плотных линейно упорядоченных множества без наибольшего и наименьшего элементов изоморфны.
    В чём ключевая идея доказательства?

    \begin{cloze}{1}
        Построить изоморфизм по шагам.
    \end{cloze}
\end{note}

\begin{note}{05c3a6d6aaf14ee2ace8ac0a6f341ede}
    Любые два счётных плотных линейно упорядоченных множества без наибольшего и наименьшего элементов изоморфны.
    Что строится на \({ n }\)-м шаге доказательства?

    \begin{cloze}{1}
        Два изоморфных \({ n }\)-элементных подмножества.
    \end{cloze}
\end{note}

\begin{note}{81d3607a53a44a02adc761c77683cfcd}
    Любые два счётных плотных линейно упорядоченных множества без наибольшего и наименьшего элементов изоморфны.
    Как строятся изоморфные подмножества на \({ 0 }\)-м шаге?

    \begin{cloze}{1}
        Два пустых множества.
    \end{cloze}
\end{note}

\begin{note}{8c36dbd253ad4e49accbfda588771eba}
    Любые два счётных плотных линейно упорядоченных множества без наибольшего и наименьшего элементов изоморфны.
    Как строятся изоморфные подмножества на каждом следующем (\({ (n+1) }\)-м) шаге?

    \begin{cloze}{1}
        Выбирается <<неохваченный>> элемент из \({ X }\) и для его позиции выбирается элемент из \({ Y }\).
    \end{cloze}
\end{note}

\begin{note}{4c20edaf43894fa6aad8ff406ff45512}
    Любые два счётных плотных линейно упорядоченных множества без наибольшего и наименьшего элементов изоморфны.
    Как в доказательстве гарантировать, что все элементы обоих множеств будут охвачены?

    \begin{cloze}{1}
        Пронумеровать и поочерёдно выбирать <<неохваченный>> элемент с наименьшим индексом.
    \end{cloze}
\end{note}

\begin{note}{9046bf3a560244b1be2ff15bfb8fc1c5}
    Сколько существует неизоморфных счётных плотных линейных множеств?

    \begin{cloze}{1}
        Четыре.
    \end{cloze}
\end{note}

\begin{note}{00ea7e2d37ed47ebba2ca53baf9f132c}
    Существует только 4 неизоморфных счётных плотных линейных множеств.
    В чём ключевая идея доказательства?

    \begin{cloze}{1}
        Изоморфность зависит только от наличия наибольших и наименьших элементов.
    \end{cloze}
\end{note}

\begin{note}{888cf7317e604c62adeadeb29423837e}
    \begin{icloze}{4}Всякое\end{icloze} \begin{icloze}{3}счётное линейно\end{icloze} упорядоченное множество \begin{icloze}{2}изоморфно\end{icloze} \begin{icloze}{1}некоторому подмножеству \({ \mathbb Q }\).\end{icloze}
\end{note}

\begin{note}{3003b82fd8b342478147fe3f6da09ad4}
    Всякое счётное линейно упорядоченное множество изоморфно некоторому подмножеству \({ \mathbb Q }\).
    В чём ключевая идея доказательства?

    \begin{cloze}{1}
        По шагам <<охватывать>> элементы изоморфизмом.
    \end{cloze}
\end{note}

\begin{note}{5071d1ba54de4db0955aa3bc403cbc54}
    Всякое счётное линейно упорядоченное множество изоморфно некоторому подмножеству \({ \mathbb Q }\).
    Почему именно \({ \mathbb Q }\)?

    \begin{cloze}{1}
        Можно было взять любое счётное плотное линейно упорядоченное множество без наибольшего и наименьшего элементов.
    \end{cloze}
\end{note}

\end{document}
