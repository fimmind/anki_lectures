\documentclass[11pt, a5paper]{article}
\usepackage[width=10cm, top=0.5cm, bottom=2cm]{geometry}

\usepackage[T1,T2A]{fontenc}
\usepackage[utf8]{inputenc}
\usepackage[english,russian]{babel}
\usepackage{libertine}

\usepackage{amsmath}
\usepackage{amssymb}
\usepackage{amsthm}
\usepackage{mathrsfs}
\usepackage{framed}
\usepackage{xcolor}

\setlength{\parindent}{0pt}

% Force \pagebreak for every section
\let\oldsection\section
\renewcommand\section{\pagebreak\oldsection}

\renewcommand{\thesection}{}
\renewcommand{\thesubsection}{Note \arabic{subsection}}
\renewcommand{\thesubsubsection}{}
\renewcommand{\theparagraph}{}

\newenvironment{note}[1]{\goodbreak\par\subsection{\hfill \color{lightgray}\tiny #1}}{}
\newenvironment{cloze}[2][\ldots]{\begin{leftbar}}{\end{leftbar}}
\newenvironment{icloze}[2][\ldots]{%
  \text{\tiny \color{lightgray}\{\{c#2::}\hspace{0pt}\ignorespaces%
}{%
  \unskip\hspace{0pt}\text{\tiny\color{lightgray}\}\}}%
}


\begin{document}
\section{Prerequisites}
\begin{note}{621caffff9ce421bb4309fc0c1cf144c}
    A function is said to be \begin{icloze}{2}multilinear\end{icloze} if and only if it is \begin{icloze}{1}linear separately in each variable.\end{icloze}
\end{note}

\begin{note}{1a514ffb24744a278834d0048496a850}
    A function is said to be \begin{icloze}{2}bilinear\end{icloze} if and only if \begin{icloze}{1}it is a multilinear function of two argument.\end{icloze}
\end{note}

\section{1.1. The notion of Lie algebra}
\begin{note}{686bbc96abfb46e883a4acb108450cc1}
    At the first place a Lie algebra is \begin{icloze}{1}a vector space \( L \)
    over a field \( \mathrm F \)\end{icloze}.
\end{note}

\begin{note}{7a252531934f4c00829418ab1f3a1d01}
    What is the signature of the new operation in the definition of a Lie
    algebra?

    \begin{cloze}{1}
        \[
            L \times L \to L.
        \]
    \end{cloze}
\end{note}

\begin{note}{a1cc6426fa49471dad192df5295fb310}
    The operation \( L \times L \to L \) from the definition of a Lie algebra is
    denoted \begin{icloze}{1}\( (x, y) \mapsto [xy] \)\end{icloze}.
\end{note}

\begin{note}{8bb3c76247ab416a97f8f6e247a6c2a2}
    The operation \( (x, y) \mapsto [xy] \)  from the definition of a Lie
    algebra is called \begin{icloze}{1}the bracket or commutator of \( x \) and
    \( y \)\end{icloze}.
\end{note}

\begin{note}{6c529b4b819a45c3b91755b1280be2a2}
    How many axioms are there in the definition of a Lie algebra?

    \begin{cloze}{1}
        (\( L 1 \)), (\( L 2 \)), (\( L 3 \)).
    \end{cloze}
\end{note}

\begin{note}{f8d0434e7d3c404b8319bf527f96627c}
    What is the axiom (\( L 1 \)) from the definition of a lie algebra?

    \begin{cloze}{1}
        The bracket operation is bilinear.
    \end{cloze}
\end{note}

\begin{note}{807fbd0c878541998eb3be30e870652c}
    What is the axiom (\( L 2 \)) from the definition of a lie algebra?

    \begin{cloze}{1}
        \( [x x] = 0 \) \quad for all \( x \in L \).
    \end{cloze}
\end{note}

\begin{note}{d096a87546b14acfa601179c2ae323e8}
    What is the axiom (\( L 3 \)) from the definition of a Lie algebra?

    \begin{cloze}{1}
        \( [x[yz]] + [y[zx]] + [z[xy]] = 0 \) \quad for all \( x,  y, z \in L \).
    \end{cloze}
\end{note}

\begin{note}{db6289e2261549bcb58877ac4d6f36f7}
    \begin{icloze}{2}The axiom (\( L 3 \)) from the definition of a Lie algebra\end{icloze} is called
    \begin{icloze}{1}the Jacobi identity\end{icloze}.
\end{note}

\begin{note}{fdd2c3d3027e4a34be5e9540f148a9cd}
    Let \( L \), \( L' \) be two Lie algebras over \( \mathrm F \). \begin{icloze}{1}A vector space isomorphism \( \phi : L \to L' \) satisfying
    \[
        \phi([xy]) = [\phi(x) \phi(y)] \quad \forall x,y \in L
    \]\end{icloze}
    is called \begin{icloze}{2}an isomorphism of Lie algebras.\end{icloze}
\end{note}

\begin{note}{cbd7fbe2d06c41e29e72d8075fc10e5f}
    We say that two Lie algebras \( L \), \( L' \) over \( \mathrm F \) are \begin{icloze}{2}isomorphic\end{icloze} if \begin{icloze}{1}there exists a Lie algebra isomorphism \( \phi : L \to L' \).\end{icloze}
\end{note}

\begin{note}{fe5ab449614b4098a1e81d3d86903d64}
    Let \( L \) be a Lie algebra over \( \mathrm F \). \begin{icloze}{2}A subspace \( K \) of \( L \) satisfying
        \[
            [xy] \in K \quad \forall x,y \in K.
        \]
    \end{icloze}
    is called \begin{icloze}{1}a subalgebra of \( L \)\end{icloze}
\end{note}
\end{document}
