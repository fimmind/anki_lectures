%! TeX root = ./main.tex
\documentclass[11pt, a5paper]{article}
\usepackage[width=10cm, top=0.5cm, bottom=2cm]{geometry}

\usepackage[T1,T2A]{fontenc}
\usepackage[utf8]{inputenc}
\usepackage[english,russian]{babel}
\usepackage{libertine}

\usepackage{amsmath}
\usepackage{amssymb}
\usepackage{amsthm}
\usepackage{mathrsfs}
\usepackage{framed}
\usepackage{xcolor}

\setlength{\parindent}{0pt}

% Force \pagebreak for every section
\let\oldsection\section
\renewcommand\section{\pagebreak\oldsection}

\renewcommand{\thesection}{}
\renewcommand{\thesubsection}{Note \arabic{subsection}}
\renewcommand{\thesubsubsection}{}
\renewcommand{\theparagraph}{}

\newenvironment{note}[1]{\goodbreak\par\subsection{\hfill \color{lightgray}\tiny #1}}{}
\newenvironment{cloze}[2][\ldots]{\begin{leftbar}}{\end{leftbar}}
\newenvironment{icloze}[2][\ldots]{%
  \ignorespaces\text{\tiny \color{lightgray}\{\{c#2::}\hspace{0pt}%
}{%
  \hspace{0pt}\text{\tiny\color{lightgray}\}\}}\unskip%
}


\begin{document}
\section{Properties of Infinite Series} % 2.7
\begin{note}{51836e3c068e46888891ad60f449bd6}
    Let \({ \sum_{k=1}^{\infty} a_k = A }\) and \({ c \in \mathbf{R} }\).
    Under which condition does
    \[
        \sum_{k=1}^{\infty} ca_k
    \]
    converge?

    \begin{cloze}{1}
        Always.
    \end{cloze}
\end{note}

\begin{note}{548101004aba462b8e81b2c4f7cbd1b9}
    If \({ \sum_{k=1}^{\infty} a_k = A }\) and \({ c \in \mathbf{R} }\), then \({ \sum_{k=1}^{\infty} ca_k = \begin{icloze}{1}cA\end{icloze} }\).
\end{note}

\begin{note}{30607fca749d4ea9814ec7460a102865}
    Let \({ \sum_{k=1}^{\infty} a_k = A }\) and \({ \sum_{k=1}^{\infty} b_k = B }\).
    Under which condition does
    \[
        \sum_{k=1}^{\infty} a_k + b_k
    \]
    converge?

    \begin{cloze}{1}
        Always.
    \end{cloze}
\end{note}

\begin{note}{4f1064d2b18d4e889fa4e80010f532b1}
    If \({ \sum_{k=1}^{\infty} a_k = A }\) and \({ \sum_{k=1}^{\infty} b_k = B }\), then
    \[
        \sum_{k=1}^{\infty} a_k + b_k = \begin{icloze}{1}A + B.\end{icloze}
    \]
\end{note}

\begin{note}{6795efea2a204bfb90bf19f3ac01f60a}
    The series \({ \sum_{k=1}^{\infty} a_k }\) \begin{icloze}{5}converges\end{icloze} \begin{icloze}{4}if and only if,\end{icloze} given \begin{icloze}{3}\({ \epsilon > 0 }\),\end{icloze} there exists \begin{icloze}{2}an \({ N \in \mathbf{N} }\)\end{icloze} such that whenever \begin{icloze}{2}\({ n > m \geq N }\)\end{icloze} it follows that
    \begin{icloze}{1}
        \[
            \left\lvert a_{m + 1} + \cdots + a_n \right\rvert < \epsilon.
        \]
    \end{icloze}
\end{note}

\begin{note}{f83e35fa266b4b71ae674a5ae53196aa}
    The series \({ \sum_{k=1}^{\infty} a_k }\) converges if and only if, given \({ \epsilon > 0 }\), there exists an \({ N \in \mathbf{N} }\) such that whenever \({ n > m \geq N }\) it follows that
    \[
        \left\lvert a_{m + 1} + \cdots + a_n \right\rvert < \epsilon.
    \]

    \begin{center}
        \tiny
        <<\begin{icloze}{1}Cauchy Criterion\end{icloze}>>
    \end{center}
\end{note}

\begin{note}{255fd1a8d1ca40ddbe4706f396dcaad5}
    What is the key idea in the proof of the Cauchy Criterion for Series?

    \begin{cloze}{1}
        Cauchy Criterion for the sequence of partial sums.
    \end{cloze}
\end{note}

\begin{note}{2ccccd666d0d4025a48baaa6ac297e88}
    If the series \({ \sum_{k=1}^{\infty} a_k }\) \begin{icloze}{2}converges,\end{icloze} then \begin{icloze}{1}\({ (a_k) \to 0 }\).\end{icloze}
\end{note}

\begin{note}{e553a27c1b0240b4a08a2d2e1291a1c5}
    If the series \({ \sum_{k=1}^{\infty} a_k }\) converges, then \({ (a_k) \to 0 }\).
    What is the key idea in the proof?

    \begin{cloze}{1}
        Apply the Cauchy Criterion with \({ n = m + 1 }\).
    \end{cloze}
\end{note}

\begin{note}{0314d6d2761e4bd1b24b1b858e9c5086}
    Assume \({ (a_k) }\) and \({ (b_k) }\) are sequences satisfying \begin{icloze}{3}\({ 0 \leq a_k \leq b_k }\) for all \({ k \in \mathbf{N} }\).\end{icloze}
    If \({ \sum_{k=1}^{\infty} \begin{icloze}{1}b_k\end{icloze} }\) \begin{icloze}{2}converges,\end{icloze} then \({ \sum_{k=1}^{\infty} \begin{icloze}{1}a_k\end{icloze} }\) \begin{icloze}{2}converges.\end{icloze}
\end{note}

\begin{note}{03fddbcdb39340e0a421d24fe7298f2e}
    Assume \({ (a_k) }\) and \({ (b_k) }\) are sequences satisfying \({ 0 \leq a_k \leq b_k }\) for all \({ k \in \mathbf{N} }\).
    If \({ \sum_{k=1}^{\infty} \begin{icloze}{1}a_k\end{icloze} }\) \begin{icloze}{2}diverges,\end{icloze} then \({ \sum_{k=1}^{\infty} \begin{icloze}{1}b_k\end{icloze} }\) \begin{icloze}{2}diverges.\end{icloze}
\end{note}

\begin{note}{d6553b70220c4348a7c7692a58f91271}
    Assume \({ (a_k) }\) and \({ (b_k) }\) are sequences satisfying \({ 0 \leq a_k \leq b_k }\) for all \({ k \in \mathbf{N} }\).
    If \({ \sum_{k=1}^{\infty} b_k }\) converges, then \({ \sum_{k=1}^{\infty} a_k }\) converges.

    \begin{center}
        \tiny
        <<\begin{icloze}{1}Comparison Test\end{icloze}>>
    \end{center}
\end{note}

\begin{note}{7f40a1b03ff44e75af1465ca5e329e3e}
    What is the key idea in the proof of the Comparison Test for Series?

    \begin{cloze}{1}
        Use the Cauchy Criterion explicitly.
    \end{cloze}
\end{note}

\begin{note}{f49c77a313a747e9b024dd5189511f35}
    \[
        \sum_{k=1}^{\infty} \frac{1}{k} = \begin{icloze}{1}\infty.\end{icloze}
    \]
\end{note}

\begin{note}{184fe5e5e62b4c3f8a49c4ea6d26c240}
    \[
        \sum_{k=1}^{\infty} \frac{1}{k} = \infty.
    \]
    What is the key idea in the proof?

    \begin{cloze}{1}
        Observe \({ \frac{1}{k} \geqslant \frac{1}{2^{i}} }\) for every next \({ 2^{i - 1} }\) terms.
    \end{cloze}
\end{note}

\begin{note}{1f9364c8930f4fedbfb3501d9a92ee2e}
    Statements about \begin{icloze}{2}convergence\end{icloze} of sequences and series are immune to \begin{icloze}{1}changes in some finite number of initial terms.\end{icloze}
\end{note}

\begin{note}{89c3e03f687b4c4aa41185f6c668d327}
    A series is called \begin{icloze}{2}geometric\end{icloze} if it is of the form
    \begin{icloze}{1}
        \[
            \sum_{k=1}^{\infty} ar^{k}.
        \]
    \end{icloze}
\end{note}

\begin{note}{4d18a586f7754236bac47a23a54ede43}
    The series \({ \sum_{k=1}^{\infty} ar^{k} }\) \begin{icloze}{2}converges\end{icloze} \begin{icloze}{3}if and only if\end{icloze} \begin{icloze}{1}\({ \left\lvert r \right\rvert < 1 }\).\end{icloze}
\end{note}

\begin{note}{f7ab1e58f37b4580a558de06c51dc6f7}
    Given \({ \left\lvert r \right\rvert < 1 }\),
    \[
        \sum_{k=1}^{\infty} ar^{k} = \begin{icloze}{1}\frac{a}{1 - r}.\end{icloze}
    \]
\end{note}

\begin{note}{c409ec230f6741b796ea4ef3e8813d9c}
    Given \({ \left\lvert r \right\rvert < 1 }\), \({ \sum_{k=1}^{\infty} ar^{k} = \frac{a}{1 - r} }\).
    What is the key idea in the proof?

    \begin{cloze}{1}
        Rewrite partial sums.
    \end{cloze}
\end{note}

\begin{note}{28dc84fd3d384adea7a15102e07c644a}
    If \begin{icloze}{2}the series \({ \sum_{k=1}^{\infty} a_k }\) converges,\end{icloze} then \begin{icloze}{1}\({ \sum_{k=1}^{\infty} a_k }\) converges.\end{icloze}

    \begin{center}
        \tiny
        <<\begin{icloze}{3}Absolute Convergence Test\end{icloze}>>
    \end{center}
\end{note}

\begin{note}{fb10bc5e919347ffa66da221bf832aa3}
    What is the key idea in the proof of the Absolute Convergence Test?

    \begin{cloze}{1}
        The Cauchy Criterion and the Triangle Inequality.
    \end{cloze}
\end{note}

\begin{note}{998d23f7cbbb49ed885b7ef2f62bb629}
    Let \({ (a_k) }\) be \begin{icloze}{4}a decreasing sequence\end{icloze} and \begin{icloze}{3}\({ (a_k) \to 0 }\).\end{icloze}
    Then
    \[
        \sum_{k=1}^{\infty} \begin{icloze}{2}(-1)^{n + 1}a_k\end{icloze}
    \]
    \begin{icloze}{1}converges.\end{icloze}
\end{note}

\begin{note}{cb8249219a644a12b50a90701e47e548}
    We say \({ \sum_{k=1}^{\infty} a_k }\) \begin{icloze}{2}converges absolutely,\end{icloze} if \begin{icloze}{1}\({ \sum_{k=1}^{\infty} \left\lvert a_k \right\rvert }\) converges.\end{icloze}
\end{note}

\begin{note}{c07bf73c30a04766803b1c0fae6b38d9}
    We say \({ \sum_{k=1}^{\infty} a_k }\) \begin{icloze}{2}converges conditionally,\end{icloze} if \begin{icloze}{1}it converges and does not converge absolutely.\end{icloze}
\end{note}

% Lecture 05.09.22 {{{
\begin{note}{f54a6f91b89f42c7b548ace2e106608d}
    A series \({ \sum_{k=1}^{\infty} a_k }\) is said to be \begin{icloze}{2}positive\end{icloze} if \begin{icloze}{1}\({ a_k \geq 0 }\) for all \({ k \in \mathbf{N} }\).\end{icloze}
\end{note}

\begin{note}{c5acade4dde342f8b7ac4acec2278ac6}
    Any \begin{icloze}{2}positive\end{icloze} converges series must \begin{icloze}{1}converge absolutely.\end{icloze}
\end{note}

\begin{note}{e85b9eb09cfa4056b868f983703a571c}
    May a positive series diverge?

    \begin{cloze}{1}
        Only to \({ +\infty }\).
    \end{cloze}
\end{note}

\begin{note}{b65eba46e51c438e933833ad313a4cf8}
    A \begin{icloze}{2}positive\end{icloze} series converges \begin{icloze}{3}if and only if\end{icloze} \begin{icloze}{1}the sequence of partial sums \({ (s_n) }\) is bounded.\end{icloze}
\end{note}
% }}}

\begin{note}{4ef68f3ca3544ea98fd3c54340c65ce5}
    Let \({ \sum_{k=1}^{\infty} a_k }\) be a series and \begin{icloze}{3}\({ f : \mathbf{N} \to \mathbf{N} }\) be 1--1 and onto.\end{icloze}
    \begin{icloze}{2}The series \({ \sum_{k=1}^{\infty} a_{f(k)} }\)\end{icloze} is called \begin{icloze}{1}a rearrangement of \({ \sum_{k=1}^{\infty} a_k }\).\end{icloze}
\end{note}

\begin{note}{4071d910f5e6410cb2b01dfc73ae48da}
    If a series \begin{icloze}{2}converges absolutely,\end{icloze} then \begin{icloze}{3}any rearrangement of this series\end{icloze} \begin{icloze}{1}converges to the same limit.\end{icloze}
\end{note}

\begin{note}{057430cb21934da7ac9bc037ba169eb5}
    If a series converges absolutely, then any rearrangement of this series converges to the same limit.
    What is the key idea in the proof?

    \begin{cloze}{1}
        Substitute the original series' initial terms for the rearrangement's partial sum.
    \end{cloze}
\end{note}

\begin{note}{d572332d7e36407ab1531e824f794b4b}
    If a series converges absolutely, then any rearrangement of this series converges to the same limit.
    In the proof, how many of the original series' initial terms are substituted from the rearrangement's partial sum?

    \begin{cloze}{1}
        So as to use the definition of convergence and the Cauchy Criterion for absolute convergence.
    \end{cloze}
\end{note}

\begin{note}{574ee484bcf94971932baee731b90c95}
    If a series converges absolutely, then any rearrangement of this series converges to the same limit.
    In the proof, how many of the rearrangement's terms are taken for the partial sum?

    \begin{cloze}{1}
        So as to contain the initial terms of the sequence.
    \end{cloze}
\end{note}

\begin{note}{c50d4f3043cb4ca38411c1b1dc20ae26}
    If a series converges absolutely, then any rearrangement of this series converges to the same limit.
    In the proof we denote \begin{icloze}{2}\({ s_n }\)\end{icloze} to be \begin{icloze}{1}the original series' partial sum.\end{icloze}
\end{note}

\begin{note}{2f9195ab94ee4143800fc5300d10d80f}
    If a series converges absolutely, then any rearrangement of this series converges to the same limit.
    In the proof we denote \begin{icloze}{2}\({ t_n }\)\end{icloze} to be \begin{icloze}{1}the rearrangement' partial sum.\end{icloze}
\end{note}

\begin{note}{1bacf92272b04fc98d69ac25f5fcdfe2}
    If a series converges absolutely, then any rearrangement of this series converges to the same limit.
    In the proof, what do we show about \({ t_m - s_N }\)?

    \begin{cloze}{1}
        \({ \left\lvert t_m - s_N \right\rvert < \varepsilon }\) due to the Cauchy Criterion.
    \end{cloze}
\end{note}

\begin{note}{8ffac6aca55141b29861f55f5d1dd8fb}
    If a series converges absolutely, then any rearrangement of this series converges to the same limit.
    In the proof, how do you show \({ \left\lvert t_m - A \right\rvert < \varepsilon }\)?

    \begin{cloze}{1}
        \({ \left\lvert t_m - s_N + s_N - A \right\rvert }\).
    \end{cloze}
\end{note}

\end{document}
