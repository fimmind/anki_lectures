%! TeX root = ./main.tex
\documentclass[11pt, a5paper]{article}
\usepackage[width=10cm, top=0.5cm, bottom=2cm]{geometry}

\usepackage[T1,T2A]{fontenc}
\usepackage[utf8]{inputenc}
\usepackage[english,russian]{babel}
\usepackage{libertine}

\usepackage{amsmath}
\usepackage{amssymb}
\usepackage{amsthm}
\usepackage{mathrsfs}
\usepackage{framed}
\usepackage{xcolor}

\setlength{\parindent}{0pt}

% Force \pagebreak for every section
\let\oldsection\section
\renewcommand\section{\pagebreak\oldsection}

\renewcommand{\thesection}{}
\renewcommand{\thesubsection}{Note \arabic{subsection}}
\renewcommand{\thesubsubsection}{}
\renewcommand{\theparagraph}{}

\newenvironment{note}[1]{\goodbreak\par\subsection{\hfill \color{lightgray}\tiny #1}}{}
\newenvironment{cloze}[2][\ldots]{\begin{leftbar}}{\end{leftbar}}
\newenvironment{icloze}[2][\ldots]{%
  \ignorespaces\text{\tiny \color{lightgray}\{\{c#2::}\hspace{0pt}%
}{%
  \hspace{0pt}\text{\tiny\color{lightgray}\}\}}\unskip%
}


\begin{document}
\section{The Monotone Convergence Theorem and a First Look at Infinite Series} % 2.4
\begin{note}{7f744b7eecb54041a6e188d2283abcff}
    A sequence \({ (a_n) }\) is \begin{icloze}{2}increasing\end{icloze} if \begin{icloze}{1}\({ a_{n + 1} \geq a_n }\) for all \({ n \in \mathbf{N} }\).\end{icloze}
\end{note}

\begin{note}{cb73357863a14f808fcb79e9f2888e9d}
    A sequence \({ (a_n) }\) is \begin{icloze}{2}decreasing\end{icloze} if \begin{icloze}{1}\({ a_{n + 1} \leq a_n }\) for all \({ n \in \mathbf{N} }\).\end{icloze}
\end{note}

\begin{note}{428c29af1f87467cba4605f856da5dc0}
    A sequence \({ (a_n) }\) is \begin{icloze}{2}monotone\end{icloze} if \begin{icloze}{1}it is either increasing or decreasing.\end{icloze}
\end{note}

\begin{note}{f0effd26705b4fe2850675b4a8b69fa7}
    If a sequence is \begin{icloze}{3}monotone\end{icloze} and \begin{icloze}{2}bounded,\end{icloze} then \begin{icloze}{1}it converges.\end{icloze}
\end{note}

\begin{note}{f04966660a1d453499de164d33c3efd9}
    If a sequence is monotone and bounded, then it converges.

    \begin{center}
        \tiny
        <<\begin{icloze}{1}Monotone Convergence Theorem\end{icloze}>>
    \end{center}
\end{note}

\begin{note}{fe52926982cd479399d0e77cf6fbb8ae}
    What is the key idea in the proof of the Monotone Convergence Theorem?

    \begin{cloze}{1}
        The limit equals to \({ \sup \left\{ a_n \mid n \in \mathbb N \right\} }\)
    \end{cloze}
\end{note}

\begin{note}{b7b0d33916a74554bee0bb1e829b7a20}
    Let \begin{icloze}{3}\({ (a_n) }\) be a sequence.\end{icloze}
    \begin{icloze}{2}An infinite series\end{icloze} is
    \begin{icloze}{1}
        a formal expression of the form
        \[
            \sum_{n=1}^{\infty} a_n = a_1 + a_2 + a_3 + \cdots.
        \]
    \end{icloze}
\end{note}

\begin{note}{024782c9319a441f91dfd2c8e8aac542}
    Let \({ \sum_{n=1}^{\infty} a_n }\) be a series.
    We define the corresponding \begin{icloze}{2}sequence of partial sums\end{icloze} by
    \begin{icloze}{1}
        \[
            m \mapsto a_1 + a_2 + \cdots + a_m.
        \]
    \end{icloze}
\end{note}

\begin{note}{56563c7563df42c0a111a49ad4ae24ae}
    Let \({ \sum_{n=1}^{\infty} a_n }\) be a series.
    \begin{icloze}{2}The sequence of partial sums\end{icloze} is usually denoted \begin{icloze}{1}\({ (s_m) }\).\end{icloze}
\end{note}

\begin{note}{dc59f9b31fff4dcb9113d42da885c946}
    Let \({ \sum_{n=1}^{\infty} a_n }\) be a series.
    We say that \begin{icloze}{2}\({ \sum_{n=1}^{\infty} a_n }\) converges to \({ A }\)\end{icloze} if \begin{icloze}{1}the sequence of partial sums converges to \({ A }\).\end{icloze}
\end{note}

\begin{note}{356961ddcb85482c8155d43bd6d8061c}
    Let \({ \sum_{n=1}^{\infty} a_n }\) be a series.
    If \begin{icloze}{2}\({ \sum_{n=1}^{\infty} a_n }\) converges to \({ A }\),\end{icloze} we write
    \begin{icloze}{1}
        \[
            \sum_{n=1}^{\infty} a_n = A.
        \]
    \end{icloze}
\end{note}

\begin{note}{4819e0996d5d4eeb8ab8df01f58c8efe}
    Does \({ \sum_{n=1}^{\infty} \frac{1}{n^2} }\) converge?

    \begin{cloze}{1}
        Yes.
    \end{cloze}
\end{note}

\begin{note}{64c293a1a2f74541ba8e3ffa23fb54b2}
    \({ \sum_{n=1}^{\infty} \frac{1}{n^2} }\) converges.
    What is the key idea in the proof?

    \begin{cloze}{1}
        \({ \frac{1}{n^2} \leq \frac{1}{n(n - 1)} }\).
    \end{cloze}
\end{note}

\begin{note}{cd5ca73daf014641b49c5445adcd69b5}
    Does \({ \sum_{n=1}^{\infty} \frac{1}{n} }\) converge?

    \begin{cloze}{1}
        No.
    \end{cloze}
\end{note}

\begin{note}{184fe5e5e62b4c3f8a49c4ea6d26c240}
    \({ \sum_{n=1}^{\infty} \frac{1}{n} }\) diverges.
    What is the key idea in the proof?

    \begin{cloze}{1}
        Find a lower bound using powers of two.
    \end{cloze}
\end{note}

\begin{note}{4608dd8499934012aadc1209fb34ec1e}
    \begin{icloze}{2}\({ \sum_{n=1}^{\infty} \frac{1}{n} }\)\end{icloze} is called \begin{icloze}{1}the harmonic series.\end{icloze}
\end{note}

% Lecture 19.10.22 {{{
\begin{note}{c09166f03686451eabbc0fbfeff75b48}
    \begin{icloze}{2}The harmonic series' partial sums\end{icloze} are called \begin{icloze}{1}harmonic numbers.\end{icloze}
\end{note}

\begin{note}{967408ec06384fc5bcebcfe9d34754e3}
    \begin{icloze}{2}The \({ n }\)-th harmonic number\end{icloze} is denoted \begin{icloze}{1}\({ H_n }\).\end{icloze}
\end{note}

\begin{note}{b41eb87209464924a744a8142f77f9fc}
    What is the harmonic numbers' growth rate?

    \begin{cloze}{1}
        Logarithmic.
    \end{cloze}
\end{note}

\begin{note}{758809447e7f453ea7b35e206473125c}
    How is \({ H_n }\) approximated with \({ \ln n }\)?

    \begin{cloze}{1}
        \({ \ln n + \text{a constant} + \delta_n }\), \quad where \({ (\delta_n) \to 0 }\).
    \end{cloze}
\end{note}

\begin{note}{73dd17acc2ae4135a8b47403834cc4b6}
    What is the name of the constant term from the approximation of \({ H_n }\) with \({ \ln n }\)?

    \begin{cloze}{1}
        The Euler-Mascheroni constant.
    \end{cloze}
\end{note}

\begin{note}{867a8efe526b49388eea0f0a54e8ec20}
    What is the value of the Euler-Mascheroni constant?

    \begin{cloze}{1}
        \[
            \lim_{n \to \infty} (H_n - \ln n).
        \]
    \end{cloze}
\end{note}

\begin{note}{8361e3e94e624c89b3279fd526ece194}
    What is value of \({ \lim (H_n - \ln n) }\)?

    \begin{cloze}{1}
        The Euler-Mascheroni constant.
    \end{cloze}
\end{note}

\begin{note}{5af1c127b9d44469b37ca4390dbcc30f}
    The Euler-Mascheroni constant is usually denoted \begin{icloze}{1}\({ \gamma }\).\end{icloze}
\end{note}
% }}}


\begin{note}{ccea4c33507e4d5f9387c996a8bb13ad}
    Let \({ (a_n) }\) be \begin{icloze}{4}a positive decreasing sequence.\end{icloze}
    Then
    \[
        \begin{icloze}{2}\sum_{n=1}^{\infty} a_n \text{ converges}\end{icloze}
        \begin{icloze}{3}\iff\end{icloze}
        \begin{icloze}{1}\sum_{n=1}^{\infty} 2^{n} a_{2^{n}} \text{ converges.}\end{icloze}
    \]

    \begin{center}
        \tiny
        <<\begin{icloze}{6}Cauchy Condensation Test\end{icloze}>>
    \end{center}
\end{note}

\begin{note}{88287ba71bd545459ba16b4e2ca5cb69}
    Let \({ (a_n) }\) be a decreasing sequence and \({ a_n \leq 0 }\).
    Then
    \[
        \sum_{n=1}^{\infty} a_n \text{ converges} \iff \sum_{n=1}^{\infty} 2^{n} a_{2^{n}} \text{ converges.}
    \]
    What is the key idea in the proof?

    \begin{cloze}{1}
        Group the element of a partial sum in chunks of size \({ 2^{m} }\).
    \end{cloze}
\end{note}

\begin{note}{7dfc9afff8a045caa6549458d3264c8d}
    The series \({ \sum_{n=1}^{\infty} \frac{1}{n^{p}} }\) \begin{icloze}{2}converges\end{icloze} \begin{icloze}{3}if and only if\end{icloze} \begin{icloze}{1}\({ p > 1 }\).\end{icloze}
\end{note}

\begin{note}{66666197109243728959180963a362d4}
    The series \({ \sum_{n=1}^{\infty} \frac{1}{n^{p}} }\) converges if and only if \({ p > 1 }\).
    What is the key idea in the proof?

    \begin{cloze}{1}
        The Cauchy Condensation Test and the convergence of geometric series.
    \end{cloze}
\end{note}

\begin{note}{02c63840577e480888a118f988cd15f5}
    Let
    \[
        x_1 = 3  \quad \text{and} \quad  x_{n+1} = \frac{1}{4 - x_n}.
    \]
    How do you prove that \({ (x_n) }\) converges?

    \begin{cloze}{1}
        Use the Monotone Convergence Theorem.
    \end{cloze}
\end{note}

\begin{note}{c2af7f6b381941e4a6225dc675a8b580}
    Let
    \[
        x_1 = 3  \quad \text{and} \quad  x_{n+1} = \frac{1}{4 - x_n}.
    \]
    How do you proof that \({ x_n }\) is monotone?

    \begin{cloze}{1}
        By induction.
    \end{cloze}
\end{note}

\begin{note}{f3e64eda5bb1491ea87a57a0627b885c}
    Let
    \[
        x_1 = 3  \quad \text{and} \quad  x_{n+1} = \frac{1}{4 - x_n}.
    \]
    Given that \({ \lim x_n }\) exists, how do you find its value?

    \begin{cloze}{1}
        Take the limit of the equality.
    \end{cloze}
\end{note}

\begin{note}{c9b146dff86b4f93bb189517510996a0}
    What is the geometric mean of two real numbers \({ x }\) and \({ y }\)?

    \begin{cloze}{1}
        \[
            \sqrt{x \cdot y}
        \]
    \end{cloze}
\end{note}

\begin{note}{5a27b7e61ce148dbbd2ed59ebe4e62af}
    What is the arithmetic mean of two real numbers \({ x }\) and \({ y }\)?

    \begin{cloze}{1}
        \[
            \frac{x + y}{2}
        \]
    \end{cloze}
\end{note}

\begin{note}{f2819387c22a4fa0bb81aa43dc5713a1}
    How do the geometric mean and the arithmetic mean of two positive real number relate?

    \begin{cloze}{1}
        Geometric \({ \leq }\) arithmetic.
    \end{cloze}
\end{note}

\begin{note}{2139b7a22c4943d4a6a97cea13e2d4c7}
    What, vaguely, is the limit superior of a sequence?

    \begin{cloze}{1}
        The limit of consecutive supremums.
    \end{cloze}
\end{note}

\begin{note}{a0aaa8a84d2e45d4ab724b0acab3f929}
    For which sequences is the limit superior always well-defined?

    \begin{cloze}{1}
        Bounded above.
    \end{cloze}
\end{note}

\begin{note}{8f61f74e27fb429aa768dd1cd4d18e9f}
    Which set's supremum is taken in the definition of the limit superior of a sequence?

    \begin{cloze}{1}
        Of all the elements starting from the \({ n }\)-th.
    \end{cloze}
\end{note}

\begin{note}{fedfa9f4b6af42368cedc5eca7373c93}
    Let \({ (a_n) }\) be a sequence.
    \begin{icloze}{2}The limit superior of \({ (a_n) }\)\end{icloze} is denoted
    \begin{icloze}{1}
        \[
            \limsup a_n.
        \]
    \end{icloze}
\end{note}

\begin{note}{2139b7a22c4943d4a6a97cea13e2d4c7}
    What, vaguely, is the limit inferior of a sequence?

    \begin{cloze}{1}
        The limit of consecutive infimums.
    \end{cloze}
\end{note}

\begin{note}{72a567f6068242118de0a4d171564149}
    For which sequences is the limit inferior always well-defined?

    \begin{cloze}{1}
        Bounded below.
    \end{cloze}
\end{note}

\begin{note}{a392b9b00d884f018250ab615f16dcba}
    Let \({ (a_n) }\) be a sequence.
    \begin{icloze}{2}The limit inferior of \({ (a_n) }\)\end{icloze} is denoted
    \begin{icloze}{1}
        \[
            \liminf a_n.
        \]
    \end{icloze}
\end{note}

\begin{note}{6d07e6f3644649ab814c1c25ac7f8ea7}
    Let \({ (a_n) }\) be a bounded sequence.
    How do \({ \liminf a_n }\) and \({ \limsup a_n }\) relate?

    \begin{cloze}{1}
        \[
            \liminf a_n \leq \limsup a_n.
        \]
    \end{cloze}
\end{note}

\begin{note}{2de9f038606f4764a5ff29923042646c}
    Let \({ (a_n) }\) be a bounded sequence.
    \[
        \begin{icloze}{2}(a_n) \to a\end{icloze}
        \quad \begin{icloze}{3}\iff\end{icloze} \quad
        \liminf a_n = \begin{icloze}{1}\limsup a_n = a.\end{icloze}
    \]
\end{note}

\begin{note}{53b4f3cf3d6a41bfb0e016a3935e7891}
    Let \({ (a_n) }\) be \begin{icloze}{3}a sequence.\end{icloze}
    \begin{icloze}{2}An infinite product\end{icloze} is
    \begin{icloze}{1}
        a formal expression of the form
        \[
            \prod_{n = 1}^{\infty} a_n = a_1 a_2 a_3 \cdots
        \]
    \end{icloze}
\end{note}

\begin{note}{df2c87f7d21745eca396cc7544135480}
    \begin{icloze}{2}A partial product\end{icloze} of an infinite product is
    \begin{icloze}{1}
        the product of the first \({ m }\) terms.
    \end{icloze}
\end{note}

\begin{note}{9d34fc764e4440759885868ead946718}
    \begin{icloze}{2}The sequence of partial products\end{icloze} of an infinite product is usually denoted \begin{icloze}{1}\({ (p_m) }\).\end{icloze}
\end{note}

\begin{note}{d3c3280660df404ba970bb6ce71933e7}
    We say that an infinite product \begin{icloze}{2}converges to \({ A }\)\end{icloze} if \begin{icloze}{1}the corresponding sequence of partial products converges to \({ A }\).\end{icloze}
\end{note}

\begin{note}{707abc48979742b7b9a5a7c0646a03d7}
    Does \({ \displaystyle \prod_{n = 1}^{\infty} \left( 1 + \frac{1}{n} \right) }\) converge?

    \begin{cloze}{1}
        No.
    \end{cloze}
\end{note}

\begin{note}{d5d2d9e020c04d0aa59e356b7a11447c}
    \[
        \prod_{n = 1}^{\infty} \left( 1 + a_n \right), \quad \text{where } \begin{icloze}{2}a_n \geq 0,\end{icloze}
    \]
    \begin{icloze}{3}converges\end{icloze} if and only if \begin{icloze}{1}\({ \sum a_n }\) converges.\end{icloze}
\end{note}

\begin{note}{eb2761aa4e8e47008fbc04a66a05d627}
    Let \({ (a_n) }\) be a positive sequence.
    \[
        \prod \left( 1 + a_n \right) \text{ converges } \implies \sum a_n \text{ converges.}
    \]
    What is the key idea in the proof?

    \begin{cloze}{1}
        A partial product's expansion contains the partial sum.
    \end{cloze}
\end{note}

\begin{note}{51003bd899f44cd3ae8c7ceef7a14dc5}
    Let \({ (a_n) }\) be a positive sequence.
    \[
        \prod \left( 1 + a_n \right) \text{ converges } \impliedby \sum a_n \text{ converges.}
    \]
    What is the key idea in the proof?

    \begin{cloze}{1}
        Give an upper bound for \({ 1 + a_n }\) with an exponential.
    \end{cloze}
\end{note}

\begin{note}{000b398dbcc04190a753fdbee51f11f8}
    Given \({ x \geq 0 }\), provide an upper bound for \({ 1 + x }\) with an exponential.

    \begin{cloze}{1}
        \({ 1 + x \leq e^{x} }\).
    \end{cloze}
\end{note}

\begin{note}{523b13cecd6249faac6c5f8c39e0c77f}
    What is the visual representation behind the inequality
    \[
        1 + x \leq e^{x} \quad \text{?}
    \]

    \begin{cloze}{1}
        \({ y = 1 + x }\) is tangent to \({ y = e^{x} }\), which is convex.
    \end{cloze}
\end{note}

\section{Properties of Infinite Series} % 2.7
\begin{note}{51836e3c068e46888891ad60f449bd6}
    Let \({ \sum_{k=1}^{\infty} a_k = A }\) and \({ c \in \mathbf{R} }\).
    Under which condition does
    \[
        \sum_{k=1}^{\infty} ca_k
    \]
    converge?

    \begin{cloze}{1}
        Always.
    \end{cloze}
\end{note}

\begin{note}{548101004aba462b8e81b2c4f7cbd1b9}
    If \({ \sum_{k=1}^{\infty} a_k = A }\) and \({ c \in \mathbf{R} }\), then \({ \sum_{k=1}^{\infty} ca_k = \begin{icloze}{1}cA\end{icloze} }\).
\end{note}

\begin{note}{30607fca749d4ea9814ec7460a102865}
    Let \({ \sum_{k=1}^{\infty} a_k = A }\) and \({ \sum_{k=1}^{\infty} b_k = B }\).
    Under which condition does
    \[
        \sum_{k=1}^{\infty} a_k + b_k
    \]
    converge?

    \begin{cloze}{1}
        Always.
    \end{cloze}
\end{note}

\begin{note}{4f1064d2b18d4e889fa4e80010f532b1}
    If \({ \sum_{k=1}^{\infty} a_k = A }\) and \({ \sum_{k=1}^{\infty} b_k = B }\), then
    \[
        \sum_{k=1}^{\infty} a_k + b_k = \begin{icloze}{1}A + B.\end{icloze}
    \]
\end{note}

% Lecture 05.09.22 {{{
\begin{note}{7c0abecdaf8e4bc19ba89cb4fe114bd6}
    If \({ \sum_{k=1}^{\infty} a_k }\) \begin{icloze}{3}converges\end{icloze} and \({ \sum_{k=1}^{\infty} b_k }\) \begin{icloze}{3}diverges,\end{icloze} then
    \[
        \begin{icloze}{2}\sum_{k=1}^{\infty} a_k + b_k\end{icloze} \begin{icloze}{1}\text{ diverges.}\end{icloze}
    \]
\end{note}

\begin{note}{b5cdf0512d8d457cb2a379569c3be2d3}
    If \({ \sum_{k=1}^{\infty} a_k }\) converges and \({ \sum_{k=1}^{\infty} b_k }\) diverges, then
    \[
        \sum_{k=1}^{\infty} a_k + b_k \text{ diverges.}
    \]
    What is the key idea in the proof?

    \begin{cloze}{1}
        By contradiction and \({ \sum b_k }\) converges.
    \end{cloze}
\end{note}

\begin{note}{41d1b8798dd64b74a5b57efd33beaa27}
    The tail of a \begin{icloze}{2}convergent\end{icloze} series \begin{icloze}{1}tends to \({ 0 }\).\end{icloze}
\end{note}
% }}}

\begin{note}{6795efea2a204bfb90bf19f3ac01f60a}
    The series \({ \sum_{k=1}^{\infty} a_k }\) \begin{icloze}{5}converges\end{icloze} \begin{icloze}{4}if and only if,\end{icloze} given \begin{icloze}{3}\({ \epsilon > 0 }\),\end{icloze} there exists \begin{icloze}{2}an \({ N \in \mathbf{N} }\)\end{icloze} such that whenever \begin{icloze}{2}\({ n > m \geq N }\)\end{icloze} it follows that
    \begin{icloze}{1}
        \[
            \left\lvert a_{m + 1} + \cdots + a_n \right\rvert < \epsilon.
        \]
    \end{icloze}
\end{note}

\begin{note}{f83e35fa266b4b71ae674a5ae53196aa}
    The series \({ \sum_{k=1}^{\infty} a_k }\) converges if and only if, given \({ \epsilon > 0 }\), there exists an \({ N \in \mathbf{N} }\) such that whenever \({ n > m \geq N }\) it follows that
    \[
        \left\lvert a_{m + 1} + \cdots + a_n \right\rvert < \epsilon.
    \]

    \begin{center}
        \tiny
        <<\begin{icloze}{1}Cauchy Criterion\end{icloze}>>
    \end{center}
\end{note}

\begin{note}{255fd1a8d1ca40ddbe4706f396dcaad5}
    What is the key idea in the proof of the Cauchy Criterion for Series?

    \begin{cloze}{1}
        Cauchy Criterion for the sequence of partial sums.
    \end{cloze}
\end{note}

\begin{note}{2ccccd666d0d4025a48baaa6ac297e88}
    If the series \({ \sum_{k=1}^{\infty} a_k }\) \begin{icloze}{2}converges,\end{icloze} then \begin{icloze}{1}\({ (a_k) \to 0 }\).\end{icloze}
\end{note}

\begin{note}{e553a27c1b0240b4a08a2d2e1291a1c5}
    If the series \({ \sum_{k=1}^{\infty} a_k }\) converges, then \({ (a_k) \to 0 }\).
    What is the key idea in the proof?

    \begin{cloze}{1}
        Apply the Cauchy Criterion with \({ n = m + 1 }\).
    \end{cloze}
\end{note}

\begin{note}{0314d6d2761e4bd1b24b1b858e9c5086}
    Assume \({ (a_k) }\) and \({ (b_k) }\) are sequences satisfying \begin{icloze}{3}\({ 0 \leq a_k \leq b_k }\) for all \({ k \in \mathbf{N} }\).\end{icloze}
    If \({ \sum_{k=1}^{\infty} \begin{icloze}{1}b_k\end{icloze} }\) \begin{icloze}{2}converges,\end{icloze} then \({ \sum_{k=1}^{\infty} \begin{icloze}{1}a_k\end{icloze} }\) \begin{icloze}{2}converges.\end{icloze}
\end{note}

\begin{note}{03fddbcdb39340e0a421d24fe7298f2e}
    Assume \({ (a_k) }\) and \({ (b_k) }\) are sequences satisfying \({ 0 \leq a_k \leq b_k }\) for all \({ k \in \mathbf{N} }\).
    If \({ \sum_{k=1}^{\infty} \begin{icloze}{1}a_k\end{icloze} }\) \begin{icloze}{2}diverges,\end{icloze} then \({ \sum_{k=1}^{\infty} \begin{icloze}{1}b_k\end{icloze} }\) \begin{icloze}{2}diverges.\end{icloze}
\end{note}

\begin{note}{d6553b70220c4348a7c7692a58f91271}
    Assume \({ (a_k) }\) and \({ (b_k) }\) are sequences satisfying \({ 0 \leq a_k \leq b_k }\) for all \({ k \in \mathbf{N} }\).
    If \({ \sum_{k=1}^{\infty} b_k }\) converges, then \({ \sum_{k=1}^{\infty} a_k }\) converges.

    \begin{center}
        \tiny
        <<\begin{icloze}{1}Comparison Test\end{icloze}>>
    \end{center}
\end{note}

\begin{note}{7f40a1b03ff44e75af1465ca5e329e3e}
    What is the key idea in the proof of the Comparison Test for Series?

    \begin{cloze}{1}
        Use the Cauchy Criterion explicitly.
    \end{cloze}
\end{note}

% Lecture 12.09.22 {{{
\begin{note}{e02413e7068f47d28eab58d2542d2858}
    What series are considered in the Limit Comparison Test?

    \begin{cloze}{1}
        Positive and one containing no zeros.
    \end{cloze}
\end{note}

\begin{note}{cedc7eb1b2ac4f578caebcbaf4398f01}
    Which value is considered in the Limit Comparison Test?

    \begin{cloze}{1}
        The limit of the ratio of corresponding terms.
    \end{cloze}
\end{note}

\begin{note}{9ce4a06cfa6e42c7bae44e61649416d4}
    Which cases exist on the Limit Comparison Test?

    \begin{cloze}{1}
        The limit is finite or is nonzero.
    \end{cloze}
\end{note}

\begin{note}{bb6597f3ea41409da5895548c598ddae}
    What can we say from the Limit Comparison Test if the limit is finite?

    \begin{cloze}{1}
        The denominator's series convergence implies that of the numerator.
    \end{cloze}
\end{note}

\begin{note}{08f8caf25dd54ec8bd0b0a03a66d00f2}
    What can we say from the Limit Comparison Test if the limit is nonzero?

    \begin{cloze}{1}
        The numerator's series convergence implies that of the denominator.
    \end{cloze}
\end{note}

\begin{note}{8474f88f7b4140dabe637c96e7a5005d}
    What can we say from the Limit Comparison Test if the limit is finite and nonzero?

    \begin{cloze}{1}
        The two series's convergences are equivalent.
    \end{cloze}
\end{note}

\begin{note}{ca9aa1db61144f7e99c9c0ead13fed2f}
    What can we say from the Limit Comparison Test if the limit does not exist?

    \begin{cloze}{1}
        Nothing.
    \end{cloze}
\end{note}

\begin{note}{34848474b28a469dbb7bc1859e1ab612}
    What is the key idea in the proof of the Limit Comparison Test (finite limit)?

    \begin{cloze}{1}
        The set of ratios is bounded above + the Comparison Test.
    \end{cloze}
\end{note}

\begin{note}{6f66af55f5d042cb85559bf7718f0641}
    What is the key idea in the proof of the Limit Comparison Test (nonzero limit)?

    \begin{cloze}{1}
        Swap the numerator and the denominator.
    \end{cloze}
\end{note}
% }}}

\begin{note}{1f9364c8930f4fedbfb3501d9a92ee2e}
    Statements about \begin{icloze}{2}convergence\end{icloze} of sequences and series are immune to \begin{icloze}{1}changes in some finite number of initial terms.\end{icloze}
\end{note}

\begin{note}{89c3e03f687b4c4aa41185f6c668d327}
    A series is called \begin{icloze}{2}geometric\end{icloze} if it is of the form
    \begin{icloze}{1}
        \[
            \sum_{k=0}^{\infty} ar^{k}.
        \]
    \end{icloze}
\end{note}

\begin{note}{4d18a586f7754236bac47a23a54ede43}
    The series \({ \sum_{k=0}^{\infty} ar^{k} }\) \begin{icloze}{2}converges\end{icloze} \begin{icloze}{3}if and only if\end{icloze} \begin{icloze}{1}\({ \left\lvert r \right\rvert < 1 }\).\end{icloze}
\end{note}

\begin{note}{f7ab1e58f37b4580a558de06c51dc6f7}
    Given \({ \left\lvert r \right\rvert < 1 }\),
    \[
        \sum_{k=0}^{\infty} ar^{k} = \begin{icloze}{1}\frac{a}{1 - r}.\end{icloze}
    \]
\end{note}

\begin{note}{c409ec230f6741b796ea4ef3e8813d9c}
    Given \({ \left\lvert r \right\rvert < 1 }\), \({ \sum_{k=0}^{\infty} ar^{k} = \frac{a}{1 - r} }\).
    What is the key idea in the proof?

    \begin{cloze}{1}
        Rewrite the partial sums.
    \end{cloze}
\end{note}

\begin{note}{28dc84fd3d384adea7a15102e07c644a}
    If \begin{icloze}{2}the series \({ \sum_{k=1}^{\infty} \left\lvert a_k \right\rvert }\) converges,\end{icloze} then \begin{icloze}{1}\({ \sum_{k=1}^{\infty} a_k }\) converges.\end{icloze}

    \begin{center}
        \tiny
        <<\begin{icloze}{3}Absolute Convergence Test\end{icloze}>>
    \end{center}
\end{note}

\begin{note}{fb10bc5e919347ffa66da221bf832aa3}
    What is the key idea in the proof of the Absolute Convergence Test?

    \begin{cloze}{1}
        The Cauchy Criterion and the Triangle Inequality.
    \end{cloze}
\end{note}

% Lecture 19.10.22 {{{
\begin{note}{9e485bfc6cf2430e8c654c0404657fdf}
    Let \({ (a_k) }\) be a sequence.
    If \begin{icloze}{2}\({ (a_k) }\) is decreasing and approaches \({ A }\),\end{icloze} we say \begin{icloze}{1}\({ (a_k) }\) decreases to \({ A }\).\end{icloze}
\end{note}

\begin{note}{c25d4896df3146b68a046db8ad0db7b2}
    Let \({ (a_k) }\) be a sequence.
    If \begin{icloze}{2}\({ (a_k) }\) decreases to \({ A }\),\end{icloze} we write
    \begin{icloze}{1}
        \[
            (a_k) \searrow A.
        \]
    \end{icloze}
\end{note}

\begin{note}{b3913aa4697f4849ae2b0a876b7412ab}
    Let \({ (a_k) }\) be a sequence.
    If \begin{icloze}{2}\({ (a_k) }\) is increasing and approaches \({ A }\),\end{icloze} we say \begin{icloze}{1}\({ (a_k) }\) increases to \({ A }\).\end{icloze}
\end{note}

\begin{note}{e24175a89ff848fa93c82f0fc0830dd9}
    Let \({ (a_k) }\) be a sequence.
    If \begin{icloze}{2}\({ (a_k) }\) increases to \({ A }\),\end{icloze} we write
    \begin{icloze}{1}
        \[
            (a_k) \nearrow A.
        \]
    \end{icloze}
\end{note}
% }}}

\begin{note}{998d23f7cbbb49ed885b7ef2f62bb629}
    Let \({ (a_k) }\) be \begin{icloze}{3}a sequence decreasing to zero.\end{icloze}
    Then
    \begin{icloze}{2}
        \[
            \sum_{k=0}^{\infty} (-1)^{k}a_k
        \]
    \end{icloze}
    \begin{icloze}{1}converges.\end{icloze}
\end{note}

\begin{note}{df767d19abbf4031899b4a87577b2625}
    Let \({ (a_k) }\) be a sequence decreasing to zero.
    Then
    \[
        \sum_{k=0}^{\infty} (-1)^{k}a_k
    \]
    converges.

    \begin{center}
        \tiny
        <<\begin{icloze}{1}Alternating Series Test\end{icloze}>>
    \end{center}
\end{note}

\begin{note}{61711709fb284700a09065f04aedcf0d}
    What is the nominal name of the Alternating Series Test?

    \begin{cloze}{1}
        Leibniz's Test.
    \end{cloze}
\end{note}

\begin{note}{5023b0a2f0ca4300bfa09b61e0ec0a9c}
    \begin{icloze}{1}An alternating series\end{icloze} is a series of the form
    \begin{icloze}{2}
        \[
            \sum_{k=0}^{\infty} (-1)^{k}a_k,
        \]
    \end{icloze}
    where \begin{icloze}{3}all \({ a_k > 0 }\).\end{icloze}
\end{note}

\begin{note}{6fb766a68cd14aa395c223e4a0e95999}
    What is the key idea in the proof of the Alternating Series Test?

    \begin{cloze}{1}
        The Cauchy criterion for the sequence of partial sums.
    \end{cloze}
\end{note}

\begin{note}{9bfa24b4310b474db9705bceed02cc45}
    Which intervals are considered in the proof of the Alternating Series Test?

    \begin{cloze}{1}
        Those formed by successive partial sums.
    \end{cloze}
\end{note}

\begin{note}{a581365ace824e89ae7a397fe6d02f1d}
    In the proof of the Alternating Series Test, how to you choose \({ \Delta_{s_m, s_{m+1}} }\), given \({ \epsilon > 0 }\)?

    \begin{cloze}{1}
        So that its length is less then \({ \epsilon }\).
    \end{cloze}
\end{note}

\begin{note}{a77a5abf0f2a46e8af759deffbaeed9e}
    In the proof of the Alternating Series Test, what do you need to show about an interval \({ \Delta_{s_m, s_{m+1}} }\)?

    \begin{cloze}{1}
        It contains all of the following partial sums.
    \end{cloze}
\end{note}

% Lecture 19.10.22 {{{
\begin{note}{337566470e054b4cb38ea03a6a388ce0}
    Does the alternating harmonic series converge?

    \begin{cloze}{1}
        Yes.
    \end{cloze}
\end{note}

\begin{note}{ced51236176744dc901d6cd2463ed6fd}
    Why does the alternating harmonic series converge?

    \begin{cloze}{1}
        Due to the Alternating Series Test.
    \end{cloze}
\end{note}

\begin{note}{0e3cb5d839ba49f3aa704f2bfeffb052}
    What does the alternating harmonic series converge to?

    \begin{cloze}{1}
        \({ \ln 2 }\).
    \end{cloze}
\end{note}

\begin{note}{92cff082756243d9a9d2f060a0aec391}
    \({ \sum \frac{(-1)^{n-1}}{n} = \ln 2 }\).
    What is the key idea in the proof?

    \begin{cloze}{1}
        Use the logarithmic approximation of harmonic numbers.
    \end{cloze}
\end{note}

\begin{note}{be5d93836bcf452e9c9263d6206ce81b}
    \({ \sum \frac{(-1)^{n-1}}{n} = \ln 2 }\).
    In the proof, how do you transform the partial sums as to use the logarithmic approximation of \({ H_n }\).

    \begin{cloze}{1}
        Add and subtract negative terms as to make them positive.
    \end{cloze}
\end{note}
% }}}

\begin{note}{cb8249219a644a12b50a90701e47e548}
    We say \({ \sum_{k=1}^{\infty} a_k }\) \begin{icloze}{2}converges absolutely,\end{icloze} if \begin{icloze}{1}\({ \sum_{k=1}^{\infty} \left\lvert a_k \right\rvert }\) converges.\end{icloze}
\end{note}

\begin{note}{c07bf73c30a04766803b1c0fae6b38d9}
    We say \({ \sum_{k=1}^{\infty} a_k }\) \begin{icloze}{2}converges conditionally,\end{icloze} if \begin{icloze}{1}it converges and does not converge absolutely.\end{icloze}
\end{note}

% Lecture 05.09.22 {{{
\begin{note}{f54a6f91b89f42c7b548ace2e106608d}
    A series \({ \sum_{k=1}^{\infty} a_k }\) is said to be \begin{icloze}{2}positive\end{icloze} if \begin{icloze}{1}\({ a_k \geq 0 }\) for all \({ k \in \mathbf{N} }\).\end{icloze}
\end{note}

\begin{note}{c5acade4dde342f8b7ac4acec2278ac6}
    Any \begin{icloze}{2}positive\end{icloze} convergent series must \begin{icloze}{1}converge absolutely.\end{icloze}
\end{note}

\begin{note}{e85b9eb09cfa4056b868f983703a571c}
    May a positive series diverge?

    \begin{cloze}{1}
        Only to \({ +\infty }\).
    \end{cloze}
\end{note}

\begin{note}{b65eba46e51c438e933833ad313a4cf8}
    A \begin{icloze}{2}positive\end{icloze} series converges \begin{icloze}{3}if and only if\end{icloze} \begin{icloze}{1}the sequence of partial sums \({ (s_n) }\) is bounded.\end{icloze}
\end{note}
% }}}

\begin{note}{4ef68f3ca3544ea98fd3c54340c65ce5}
    Let \({ \sum_{k=1}^{\infty} a_k }\) be a series and \begin{icloze}{3}\({ f : \mathbf{N} \to \mathbf{N} }\) be 1--1 and onto.\end{icloze}
    \begin{icloze}{2}The series \({ \sum_{k=1}^{\infty} a_{f(k)} }\)\end{icloze} is called \begin{icloze}{1}a rearrangement of \({ \sum_{k=1}^{\infty} a_k }\).\end{icloze}
\end{note}

\begin{note}{4071d910f5e6410cb2b01dfc73ae48da}
    If a series \begin{icloze}{2}converges absolutely,\end{icloze} then \begin{icloze}{3}any rearrangement of this series\end{icloze} \begin{icloze}{1}converges to the same limit.\end{icloze}
\end{note}

\begin{note}{057430cb21934da7ac9bc037ba169eb5}
    If a series converges absolutely, then any rearrangement of this series converges to the same limit.
    What is the key idea in the proof?

    \begin{cloze}{1}
        Subtract the original series' initial terms for the re\-ar\-range\-ment's partial sum.
    \end{cloze}
\end{note}

\begin{note}{d572332d7e36407ab1531e824f794b4b}
    If a series converges absolutely, then any rearrangement of this series converges to the same limit.
    In the proof, how many of the original series' initial terms are subtracted from the rearrangement's partial sum?

    \begin{cloze}{1}
        So as to use the definition of convergence and the Cauchy Criterion for absolute convergence.
    \end{cloze}
\end{note}

\begin{note}{574ee484bcf94971932baee731b90c95}
    If a series converges absolutely, then any rearrangement of this series converges to the same limit.
    In the proof, how many of the rearrangement's terms are taken for the partial sum?

    \begin{cloze}{1}
        So as to contain the initial terms of the original sequence.
    \end{cloze}
\end{note}

\begin{note}{c50d4f3043cb4ca38411c1b1dc20ae26}
    If a series converges absolutely, then any rearrangement of this series converges to the same limit.
    In the proof we denote \begin{icloze}{2}\({ s_n }\)\end{icloze} to be \begin{icloze}{1}the original series' partial sum.\end{icloze}
\end{note}

\begin{note}{2f9195ab94ee4143800fc5300d10d80f}
    If a series converges absolutely, then any rearrangement of this series converges to the same limit.
    In the proof we denote \begin{icloze}{2}\({ t_n }\)\end{icloze} to be \begin{icloze}{1}the rearrangement' partial sum.\end{icloze}
\end{note}

\begin{note}{1bacf92272b04fc98d69ac25f5fcdfe2}
    If a series converges absolutely, then any rearrangement of this series converges to the same limit.
    In the proof, what do we show about \({ t_m - s_N }\)?

    \begin{cloze}{1}
        \({ \left\lvert t_m - s_N \right\rvert < \varepsilon }\)
    \end{cloze}
\end{note}

\begin{note}{6e8705bf5bd84118a85ac3eb8a1d5e28}
    If a series converges absolutely, then any rearrangement of this series converges to the same limit.
    In the proof, why is it that \({ \left\lvert t_m - s_N \right\rvert < \varepsilon }\)?

    \begin{cloze}{1}
        Due to the Cauchy Criterion.
    \end{cloze}
\end{note}

\begin{note}{8ffac6aca55141b29861f55f5d1dd8fb}
    If a series converges absolutely, then any rearrangement of this series converges to the same limit.
    In the proof, how do you show \({ \left\lvert t_m - A \right\rvert < \varepsilon }\)?

    \begin{cloze}{1}
        \({ \left\lvert t_m - s_N + s_N - A \right\rvert }\) and the triangle inequality.
    \end{cloze}
\end{note}

% Lecture 26.10.22 {{{
\begin{note}{d0ce809592604649888a354c618fd0ec}
    Are positive series immune to rearrangement?

    \begin{cloze}{1}
        Yes.
    \end{cloze}
\end{note}

\begin{note}{de28685020ea44d4998072ea240cb29c}
    Why are convergent positive series immune to rearrangement?

    \begin{cloze}{1}
        They must converge absolutely.
    \end{cloze}
\end{note}

\begin{note}{125d4c6fb0df43ac826a676d13ca67e8}
    Why are divergent positive series immune to rearrangement?

    \begin{cloze}{1}
        Large enough partial sums contain the original initial terms.
    \end{cloze}
\end{note}

\begin{note}{96c6d35aa4854b3781e1b3e4d59bfb49}
    What series is considered in the Riemann Series Theorem?

    \begin{cloze}{1}
        Conditionally convergent.
    \end{cloze}
\end{note}

\begin{note}{811acdda0a24480388e62f060d18d67e}
    What do we conclude from the Riemann Series Theorem?

    \begin{cloze}{1}
        Rearrangements may converge to any chosen value.
    \end{cloze}
\end{note}

\begin{note}{c7e493071a114013a18f4ff1b7bdf8a5}
    To which value can a conditionally convergent series' re\-ar\-range\-ment converge (due to the Riemann Series Theorem)?

    \begin{cloze}{1}
        Any real number, \({ \pm \infty }\) or nothing (i.e.\ it may also diverge).
    \end{cloze}
\end{note}

\begin{note}{4feb9c52558943278e61f143708d6d96}
    What do we conclude from the Riemann Series Theorem when the series is absolutely convergent?

    \begin{cloze}{1}
        This is out of the theorem's scope.
    \end{cloze}
\end{note}

\begin{note}{2d7decf83860424eb1ccfd16074bce4d}
    What is the first step in proving the Riemann Series Theorem?

    \begin{cloze}{1}
        Split for the limiting value being finite/infinite/nonexistent.
    \end{cloze}
\end{note}

\begin{note}{5d58155034074cd9a49eea0ec8af064e}
    What is the key idea in the proof of the Riemann Series Theorem (finite limit)?

    \begin{cloze}{1}
        Make the partial sums revolve around the given value.
    \end{cloze}
\end{note}

\begin{note}{1f5a84357034420c8a97d48d8c110ddd}
    What is the algorithm for building the partial sums in the proof of the Riemann Series Theorem?

    \begin{cloze}{1}
        Go up till you get above, then down to get below and repeat.
    \end{cloze}
\end{note}

\begin{note}{a865d0048c1d4f3fb5d1c37bab47fcc6}
    In the proof of the Riemann Series Theorem, why can we make the partial sums revolve around the given value?

    \begin{cloze}{1}
        Both ``up'' and ``down'' motions are unlimited.
    \end{cloze}
\end{note}

\begin{note}{2f65a75387fa4c6c8a5f52993a2512a8}
    In the proof of the Riemann Series Theorem, why are both ``up'' and ``down'' motions unlimited?

    \begin{cloze}{1}
        Due to convergence being conditional.
    \end{cloze}
\end{note}

\begin{note}{0fc52132c4e748a3919527215e6a9bae}
    In the proof of the Riemann Series Theorem, why are the revolving partial sums approaching the given value?

    \begin{cloze}{1}
        The terms must tend to zero.
    \end{cloze}
\end{note}

\begin{note}{5b7ee13ee0d44167bca803673ded00bf}
    What is the key idea in the proof on the Riemann Series Theorem (infinite limit)?

    \begin{cloze}{1}
        Go up two units, down one unit and repeat.
    \end{cloze}
\end{note}

\begin{note}{d13a96547a804e139bfd1a25a5f4d303}
    What is the key idea in the proof on the Riemann Series Theorem (no limit)?

    \begin{cloze}{1}
        Go up over 1, then down below 0 and repeat.
    \end{cloze}
\end{note}

\begin{note}{817e11b891694f8bb974b530bda3015a}
    \[
        1 - \frac{1}{2} - \frac{1}{4} + \frac{1}{3} - \frac{1}{6} - \frac{1}{8} + \frac{1}{5} - \frac{1}{10} - \frac{1}{12} + \cdots
    \]
    is \begin{icloze}{2}a rearrangement\end{icloze} of \begin{icloze}{1}the harmonic series.\end{icloze}
\end{note}

\begin{note}{3cb62d056cdf4830898fbdd672aef478}
    \[
        1 - \frac{1}{2} - \frac{1}{4} + \frac{1}{3} - \frac{1}{6} - \frac{1}{8} + \frac{1}{5} - \frac{1}{10} - \frac{1}{12} + \cdots = \begin{icloze}{1}\frac{1}{2} \ln 2.\end{icloze}
    \]
\end{note}

\begin{note}{a961182dd27140969e35373da31fdbc3}
    \[
        1 - \frac{1}{2} - \frac{1}{4} + \frac{1}{3} - \frac{1}{6} - \frac{1}{8} + \frac{1}{5} - \frac{1}{10} - \frac{1}{12} + \cdots = \frac{1}{2} \ln 2.
    \]
    What is the key idea in the proof (intuitively)?

    \begin{cloze}{1}
        Collapse \({ \frac{1}{k} - \frac{1}{2k} }\).
    \end{cloze}
\end{note}
% }}}

\begin{note}{e36370a244d444edb06b2037f16d05b0}
    If \({ \sum a_n }\) converges absolutely, then \({ \sum a_n^2 }\) converges absolutely.
    Is this true?

    \begin{cloze}{1}
        Yes, it is.
    \end{cloze}
\end{note}

\begin{note}{64d44699e8e445089178e11a54560668}
    Assume \({ \sum a_n }\) converges absolutely.
    What can we tell about \({ \sum a_n^2 }\)?

    \begin{cloze}{1}
        It converges absolutely.
    \end{cloze}
\end{note}

\begin{note}{aa0da9a453cf405dbd207a83925a030c}
    Assume \({ \sum a_n }\) converges absolutely. Then \({ \sum a_n^2 }\) converges absolutely.
    What is the key idea in the proof?

    \begin{cloze}{1}
        Absolute values are eventually \({ < 1 }\) + the Comparison Test.
    \end{cloze}
\end{note}

\begin{note}{9562860ae3544066a13fce0c8e105bff}
    If \({ \sum a_n }\) converges and \({ (b_n) }\) converges, then \({ \sum a_n b_n }\) converges.
    Is this true?

    \begin{cloze}{1}
        No, it's false.
    \end{cloze}
\end{note}

\begin{note}{b518cd6950614ffba71bcc13155cdf31}
    If \({ \sum a_n }\) converges and \({ (b_n) }\) converges, then \({ \sum a_n b_n }\) converges.
    Provide a counterexample.

    \begin{cloze}{1}
        Alternating harmonic series and alternating \({ \frac{1}{\ln n} }\).
    \end{cloze}
\end{note}

\begin{note}{60feb468c7a54a7fb244e9e0c8b61c47}
    If \({ \sum a_n }\) converges conditionally, then \({ \sum n^2 a_n }\) diverges.
    Is this true?

    \begin{cloze}{1}
        Yes, it is.
    \end{cloze}
\end{note}

\begin{note}{a3db2ba7fe3e4210a033c14756cda177}
    Assume \({ \sum a_n }\) converges conditionally.
    What do we know about \({ \sum n^2 a_n }\)?

    \begin{cloze}{1}
        It diverges.
    \end{cloze}
\end{note}

\begin{note}{0d6253382e994acdbb84a82dcfa1152b}
    If \({ \sum a_n }\) converges conditionally, then \({ \sum n^2 a_n }\) diverges.
    What is the key idea in the proof?

    \begin{cloze}{1}
        By contradiction; \({ (n^2 a_n) }\) is bounded.
    \end{cloze}
\end{note}

\begin{note}{918203417e3c433499de22e1f1e71b37}
    If \({ \sum a_n }\) converges conditionally, then \({ \sum n^2 a_n }\) diverges.
    In the proof (by contradiction), how do you show that \({ \sum \left\lvert a_n \right\rvert }\) converges?

    \begin{cloze}{1}
        \({ (n^2 a_n) }\) is bounded; the Comparison Test.
    \end{cloze}
\end{note}

\begin{note}{d67d12138d8741b2a9f636eaee48e7d4}
    If \({ \sum n^2 a_n }\) converges, then \({ \sum a_n }\) \begin{icloze}{1}converges absolutely.\end{icloze}
\end{note}

\begin{note}{b4e0eacc15f64559b6c255552fe3aadf}
    What series are considered in the Ratio Test?

    \begin{cloze}{1}
        Strictly positive.
    \end{cloze}
\end{note}

\begin{note}{dcfddd94a3304571a442fff1f7009cb8}
    What value is considered in the Ratio Test?

    \begin{cloze}{1}
        The limit of successive ratios.
    \end{cloze}
\end{note}

\begin{note}{d00eda65eafa4efabe918bfacc3ff819}
    Which term is placed to the numerator in the Ratio Test?

    \begin{cloze}{1}
        The next one.
    \end{cloze}
\end{note}

\begin{note}{605c64a7226c48eebe5ee34d51cd470b}
    When does the Ratio Test let us conclude something?

    \begin{cloze}{1}
        When the ratios approach a value other than \({ 1 }\).
    \end{cloze}
\end{note}

\begin{note}{a70e3ac68ab947fc8e389e85e5f54588}
    Which cases exist on the Ratio Test?

    \begin{cloze}{1}
        Ratios converge to less than, or greater than, \({ 1 }\).
    \end{cloze}
\end{note}

\begin{note}{de649e2ae5cc4b3b93aac925d3b37d4b}
    What do we conclude from the Ratio Test when the ratios converge to something less than 1?

    \begin{cloze}{1}
        The series converges.
    \end{cloze}
\end{note}

\begin{note}{3bcf7fb3ba4f4ace92b222a3c8af9174}
    What do we conclude from the Ratio Test when the ratios converge to something greater than 1?

    \begin{cloze}{1}
        The series diverges.
    \end{cloze}
\end{note}

\begin{note}{90519e5b985b4f97a25636a1473b500d}
    What do we conclude from the Ratio Test when the ratios converge to 1?

    \begin{cloze}{1}
        Nothing.
    \end{cloze}
\end{note}

\begin{note}{4bab403524b240cda38745c2324966c0}
    What do we conclude from the Ratio Test when the ratios do not converge?

    \begin{cloze}{1}
        Nothing.
    \end{cloze}
\end{note}

\begin{note}{1a0caf850c00432b93871e8c66f3397b}
    Give an example when the Ratio Test is inconclusive and the series diverges.

    \begin{cloze}{1}
        The harmonic series.
    \end{cloze}
\end{note}

\begin{note}{0c417f771ac54fa3ad89fb5d65d5f10d}
    Give an example when the Ratio Test is inconclusive and the series converges.

    \begin{cloze}{1}
        \({ \sum_{n=1}^{\infty} \frac{1}{n^2} }\).
    \end{cloze}
\end{note}

\begin{note}{0a54c42a8bd74ba883e310f36f865ca6}
    What is the nominal name of the Ratio Test?

    \begin{cloze}{1}
        The d'Alambert's Ratio Test.
    \end{cloze}
\end{note}

\begin{note}{f1e24cc124f84cf3a6d14e77ee23368b}
    What is the first step in proving the Ratio Test?

    \begin{cloze}{1}
        Split \({ r < 1 }\), \({ r > 1 }\).
    \end{cloze}
\end{note}

\begin{note}{127428f8805043978b16164456c8acf5}
    What is the key idea in the proof of the Ratio Test (\({ r > 1 }\))?

    \begin{cloze}{1}
        The terms are eventually increasing.
    \end{cloze}
\end{note}

\begin{note}{535154065a884eb7bf3e87e8d4b400e5}
    What is the first key idea in the proof of the Ratio Test (\({ r < 1 }\))?

    \begin{cloze}{1}
        For \({ r < r' < 1 }\) the ratios are eventually less than \({ r' }\).
    \end{cloze}
\end{note}

\begin{note}{5ac59226423b4b8fb84c087795e5ed6f}
    What is the second key idea in the proof of the Ratio Test (\({ r < 1 }\))?

    \begin{cloze}{1}
        Find an upper bound using a geometric series.
    \end{cloze}
\end{note}

% Lecture 12.09.22 {{{
\begin{note}{ce4c6aa5f15044a2a804f11a91d677b7}
    What series are considered in the Root Test?

    \begin{cloze}{1}
        Positive.
    \end{cloze}
\end{note}

\begin{note}{02964fce0fcd409cab46d91942e3f1c2}
    What value is considered in the Root Test?

    \begin{cloze}{1}
        The limit of \({ \sqrt[n]{a_n} }\).
    \end{cloze}
\end{note}

\begin{note}{06c9e889bae041afb32a8f2da431bbf9}
    Which cases exist on the Root Test?

    \begin{cloze}{1}
        \({ n }\)-th roots approach something less than, or greater than, \({ 1 }\).
    \end{cloze}
\end{note}

\begin{note}{562b1b6b74e24c73ad75d944ff17d581}
    When does the Root Test let us conclude something?

    \begin{cloze}{1}
        When \({ n }\)-th roots approach something other than \({ 1 }\).
    \end{cloze}
\end{note}

\begin{note}{687fe6a03e28430189cd57632f9bae0b}
    What do we conclude from the Root Test if the limit is less than \({ 1 }\)?

    \begin{cloze}{1}
        The series converges.
    \end{cloze}
\end{note}

\begin{note}{dd2315fb062b4bdf93ebe5072fc0d308}
    What do we conclude from the Root Test if the limit is greater than \({ 1 }\)?

    \begin{cloze}{1}
        The series diverges.
    \end{cloze}
\end{note}

\begin{note}{7701686caac7412aa1b3375ff77e5a9e}
    What do we conclude from the Root Test if the limit converges to \({ 1 }\)?

    \begin{cloze}{1}
        Nothing.
    \end{cloze}
\end{note}

\begin{note}{6200b936d6144cafb8b74ff7d9271a9d}
    Give an example when the root test is inconclusive and the series diverges.

    \begin{cloze}{1}
        The harmonic series.
    \end{cloze}
\end{note}

\begin{note}{6cd4fabac91944db96449403d2288e0a}
    Give an example when the root test is inconclusive and the series converges.

    \begin{cloze}{1}
        \({ \sum_{n=1}^{\infty} \frac{1}{n^2} }\).
    \end{cloze}
\end{note}

\begin{note}{644281f3c2614e2499993a48daca8aac}
    What is the nominal name for the Root Test?

    \begin{cloze}{1}
        Cauchy's Radical Test.
    \end{cloze}
\end{note}

\begin{note}{7021924723f142d489dc64e27e06c40b}
    What is the first step in proving the Root Test?

    \begin{cloze}{1}
        Split \({ r < 1 }\), \({ r > 1 }\).
    \end{cloze}
\end{note}

\begin{note}{ae27724cb07240fbb243221a41bb7f82}
    What is the first key idea in the proof of the Root Test (\({ r < 1 }\))?

    \begin{cloze}{1}
        For \({ r < r' < 1 }\) the roots are eventually less than \({ r' }\).
    \end{cloze}
\end{note}

\begin{note}{64f3efecadd94ca8ad1277cba95ded2e}
    What is the second key idea in the proof of the Root Test (\({ r < 1 }\))?

    \begin{cloze}{1}
        Find an upper bound using a geometric series.
    \end{cloze}
\end{note}

\begin{note}{e4b13d2a78bc4010ad92b3574943d982}
    What is the key idea in the proof of the Root Test (\({ r > 1 }\))?

    \begin{cloze}{1}
        The elements are eventually greater than \({ 1 }\).
    \end{cloze}
\end{note}
% }}}

\begin{note}{391b719f11404d53959a2e258908f1d0}
    What sequences are considered in the Summation-by-Parts formula?

    \begin{cloze}{1}
        Arbitrary.
    \end{cloze}
\end{note}

\begin{note}{f1a472048eb0400cafd7a7d7b0e049cc}
    What is the initial expression in the Summation-by-Parts formula?

    \begin{cloze}{1}
        \[
            \sum_{j=n}^{m} x_j y_j.
        \]
    \end{cloze}
\end{note}

\begin{note}{1424da07cc0f4c7e9e792ba2daad165c}
    Which terms are there in the transformed expression in the Summation-by-Parts formula?

    \begin{cloze}{1}
        Two ``free'' terms and a sum.
    \end{cloze}
\end{note}

\begin{note}{eb188d69b3c74c42814da0030ab179ca}
    What is the first free term of the transformed expression in the Summation-by-Parts formula?

    \begin{cloze}{1}
        The final partial sum times the next element.
    \end{cloze}
\end{note}

\begin{note}{1af3ccefd5714f279390596beb66afdb}
    What is the second free term of the transformed expression in the Summation-by-Parts formula?

    \begin{cloze}{1}
        Subtracting the partial sum preceding the range multiplied by the starting element.
    \end{cloze}
\end{note}

\begin{note}{ed63990568ac41ff9e0d1b7535e91d62}
    What is the sum term of the transformed expression in the Summation-by-Parts formula?

    \begin{cloze}{1}
        The sum of partial sums multiplied by the successive differences.
    \end{cloze}
\end{note}

\begin{note}{e46ff0f795c84a08a97ab92916d689f7}
    What is the order of successive differences in the sum term of the transformed expression in the Summation-by-Parts formula?

    \begin{cloze}{1}
        The current minus the next.
    \end{cloze}
\end{note}

\begin{note}{67d6011a7aa7477da37cbb1ab2899cea}
    What is the range of summation in the sum term of the transformed expression in the Summation-by-Parts formula?

    \begin{cloze}{1}
        Same as the original.
    \end{cloze}
\end{note}

\begin{note}{84d1c74bb7fd496a90c6f85103bb2793}
    What is the value of the zeroth partial sum in the Summation-by-Parts formula?

    \begin{cloze}{1}
        Zero.
    \end{cloze}
\end{note}

\begin{note}{6153ba6cc000482694e8ffdcea302fd4}
    What is the nominal name of the Summation-by-Parts formula?

    \begin{cloze}{1}
        The Abel Transformation.
    \end{cloze}
\end{note}

\begin{note}{a637cd28783d4349916b7db04a7b8eef}
    What is the key idea in the proof of the Summation-by-Parts formula?

    \begin{cloze}{1}
        Rewrite the sequence's values as the differences of successive partial sums.
    \end{cloze}
\end{note}

\begin{note}{0f341db289494976a66d37a683abea82}
    What series is considered in the Abel's Test?

    \begin{cloze}{1}
        A series formed by two sequence's products.
    \end{cloze}
\end{note}

\begin{note}{7a5c1013788240b382ef972b1f7fd607}
    What sequences are considered in the Abel's Test?

    \begin{cloze}{1}
        One, whose series converges, and one monotone and bounded.
    \end{cloze}
\end{note}

\begin{note}{ddbb9e296a424788bf71b1e3b0a066a8}
    What do we conclude from the Abel's Test?

    \begin{cloze}{1}
        The series of products converges.
    \end{cloze}
\end{note}

\begin{note}{5e107977cb19443f9b7b7c162281129a}
    When can we conclude something from the Abel's Test?

    \begin{cloze}{1}
        Whenever the hypothesis is satisfied.
    \end{cloze}
\end{note}

\begin{note}{874fa5f12343413792c9ef518001baa2}
    What is the first step in proving the Abel's Test?

    \begin{cloze}{1}
        With no loss of generality, the sequence is decreasing.
    \end{cloze}
\end{note}

\begin{note}{338b5b7693534b4ea659c8f8f55b1583}
    What is the key idea in the proof of the Abel's Test?

    \begin{cloze}{1}
        Summation-by-Parts + the definition of convergence.
    \end{cloze}
\end{note}

\begin{note}{a09ddc5d281c4426acd43d366e76dc2c}
    To which sums is Summation-by-Parts applied in the proof of the Abel's Test?

    \begin{cloze}{1}
        The products' series' partial sums.
    \end{cloze}
\end{note}

\begin{note}{300bbcd9c31945c3bf02eeb30031651b}
    In the proof of the Abel's Test, how do you show that the partial sums converge?

    \begin{cloze}{1}
        Both addends converge (after applying Summation-by-Parts).
    \end{cloze}
\end{note}

\begin{note}{c53b15140c664228832eab2b80cc06c4}
    In the proof of the Abel's Test after applying Summation-by-Parts, how do you show that the ``free'' terms converge?

    \begin{cloze}{1}
        It follows from the hypothesis.
    \end{cloze}
\end{note}

\begin{note}{7438567b1b904753b18ef4713bb2def2}
    In the proof of the Abel's Test after applying Summation-by-Parts, how do you show that the sums converge?

    \begin{cloze}{1}
        The Comparison Test for absolute convergence.
    \end{cloze}
\end{note}

\begin{note}{418af51ce9214731ae729971dc8feff4}
    To which series is the Comparison Test applied in the proof of the Abel's Test?

    \begin{cloze}{1}
        The one generated after applying Summation-by-Parts.
    \end{cloze}
\end{note}

\begin{note}{35b449cb77ee4b26a4ea6e221660aece}
    In the proof of the Abel's Test, where from do you get an upper bound when applying the Comparison Test?

    \begin{cloze}{1}
        The partial sums converge and, thus, are bounded.
    \end{cloze}
\end{note}

\begin{note}{576a36d9c5f047c59d57212e4781326f}
    What series is considered in the Dirichlet's Test?

    \begin{cloze}{1}
        A series formed by two sequence's product.
    \end{cloze}
\end{note}

\begin{note}{15db5e34f9784e7d900ceb16e32cc428}
    What sequences are considered in the Dirichlet's Test?

    \begin{cloze}{1}
        One with bounded partials sums and one decreasing to zero.
    \end{cloze}
\end{note}

\begin{note}{d4cc82820fb6425fa4c4839f216cb490}
    What do we conclude from the Dirichlet's Test?

    \begin{cloze}{1}
        The product's series converges.
    \end{cloze}
\end{note}

\begin{note}{f8624d3bb31347acba618aeb453083d1}
    When can we conclude something from the Dirichlet's Test?

    \begin{cloze}{1}
        Whenever the hypothesis is satisfied.
    \end{cloze}
\end{note}

\begin{note}{d4efe889ccf943438eb6d487589e7554}
    What is the key idea in the proof of the Dirichlet's Test?

    \begin{cloze}{1}
        Summation-by-Parts + the definition of convergence.
    \end{cloze}
\end{note}

\begin{note}{7b3d965ec2bb4498818d1b01c686ca76}
    To which sums is Summation-by-Parts applied in the proof of the Dirichlet's Test?

    \begin{cloze}{1}
        The products' series' partial sums.
    \end{cloze}
\end{note}

\begin{note}{b7c76052781e4eea809d4e1c5d892fec}
    The Alternating Series Test can be derived as a special case of \begin{icloze}{1}the Dirichlet's Test.\end{icloze}
\end{note}

\end{document}
