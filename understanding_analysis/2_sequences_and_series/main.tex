%! TeX root = ./main.tex
\documentclass[11pt, a5paper]{article}
\usepackage[width=10cm, top=0.5cm, bottom=2cm]{geometry}

\usepackage[T1,T2A]{fontenc}
\usepackage[utf8]{inputenc}
\usepackage[english,russian]{babel}
\usepackage{libertine}

\usepackage{amsmath}
\usepackage{amssymb}
\usepackage{amsthm}
\usepackage{mathrsfs}
\usepackage{framed}
\usepackage{xcolor}

\setlength{\parindent}{0pt}

% Force \pagebreak for every section
\let\oldsection\section
\renewcommand\section{\pagebreak\oldsection}

\renewcommand{\thesection}{}
\renewcommand{\thesubsection}{Note \arabic{subsection}}
\renewcommand{\thesubsubsection}{}
\renewcommand{\theparagraph}{}

\newenvironment{note}[1]{\goodbreak\par\subsection{\hfill \color{lightgray}\tiny #1}}{}
\newenvironment{cloze}[2][\ldots]{\begin{leftbar}}{\end{leftbar}}
\newenvironment{icloze}[2][\ldots]{%
  \text{\tiny \color{lightgray}\{\{c#2::}\hspace{0pt}\ignorespaces%
}{%
  \unskip\hspace{0pt}\text{\tiny\color{lightgray}\}\}}%
}


\begin{document}
\section{The Monotone Convergence Theorem and a First Look at Infinite Series} % 2.4
\begin{note}{7f744b7eecb54041a6e188d2283abcff}
    A sequence \({ (a_n) }\) is \begin{icloze}{2}increasing\end{icloze} if \begin{icloze}{1}\({ a_{n + 1} \geq a_n }\) for all \({ n \in \mathbf{N} }\).\end{icloze}
\end{note}

\begin{note}{cb73357863a14f808fcb79e9f2888e9d}
    A sequence \({ (a_n) }\) is \begin{icloze}{2}decreasing\end{icloze} if \begin{icloze}{1}\({ a_{n + 1} \leq a_n }\) for all \({ n \in \mathbf{N} }\).\end{icloze}
\end{note}

\begin{note}{428c29af1f87467cba4605f856da5dc0}
    A sequence \({ (a_n) }\) is \begin{icloze}{2}monotone\end{icloze} if \begin{icloze}{1}it is either increasing or decreasing.\end{icloze}
\end{note}

\begin{note}{f0effd26705b4fe2850675b4a8b69fa7}
    If a sequence is \begin{icloze}{3}monotone\end{icloze} and \begin{icloze}{2}bounded,\end{icloze} then \begin{icloze}{1}it converges.\end{icloze}
\end{note}

\begin{note}{f04966660a1d453499de164d33c3efd9}
    If a sequence is monotone and bounded, then it converges.

    \begin{center}
        \tiny
        <<\begin{icloze}{1}Monotone Convergence Theorem\end{icloze}>>
    \end{center}
\end{note}

\begin{note}{fe52926982cd479399d0e77cf6fbb8ae}
    What is the key idea in the proof of the Monotone Convergence Theorem?

    \begin{cloze}{1}
        The limit equals to \({ \sup \left\{ a_n \mid n \in \mathbb N \right\} }\)
    \end{cloze}
\end{note}

\begin{note}{b7b0d33916a74554bee0bb1e829b7a20}
    Let \begin{icloze}{3}\({ (a_n) }\) be a sequence.\end{icloze}
    \begin{icloze}{2}An infinite series\end{icloze} is
    \begin{icloze}{1}
        a formal expression of the form
        \[
            \sum_{n=1}^{\infty} a_n = a_1 + a_2 + a_3 + \cdots.
        \]
    \end{icloze}
\end{note}

\begin{note}{024782c9319a441f91dfd2c8e8aac542}
    Let \({ \sum_{n=1}^{\infty} a_n }\) be a series.
    We define the corresponding \begin{icloze}{2}sequence of partial sums\end{icloze} by
    \begin{icloze}{1}
        \[
            m \mapsto a_1 + a_2 + \cdots + a_m.
        \]
    \end{icloze}
\end{note}

\begin{note}{56563c7563df42c0a111a49ad4ae24ae}
    Let \({ \sum_{n=1}^{\infty} a_n }\) be a series.
    \begin{icloze}{2}The sequence of partial sums\end{icloze} is usually denoted \begin{icloze}{1}\({ (s_m) }\).\end{icloze}
\end{note}

\begin{note}{dc59f9b31fff4dcb9113d42da885c946}
    Let \({ \sum_{n=1}^{\infty} a_n }\) be a series.
    We say that \begin{icloze}{2}\({ \sum_{n=1}^{\infty} a_n }\) converges to \({ A }\)\end{icloze} if \begin{icloze}{1}the sequence of partial sums converges to \({ A }\).\end{icloze}
\end{note}

\begin{note}{356961ddcb85482c8155d43bd6d8061c}
    Let \({ \sum_{n=1}^{\infty} a_n }\) be a series.
    If \begin{icloze}{2}\({ \sum_{n=1}^{\infty} a_n }\) converges to \({ A }\),\end{icloze} we write
    \begin{icloze}{1}
        \[
            \sum_{n=1}^{\infty} a_n = A.
        \]
    \end{icloze}
\end{note}

\begin{note}{4819e0996d5d4eeb8ab8df01f58c8efe}
    Does \({ \sum_{n=1}^{\infty} \frac{1}{n^2} }\) converge?

    \begin{cloze}{1}
        Yes.
    \end{cloze}
\end{note}

\begin{note}{64c293a1a2f74541ba8e3ffa23fb54b2}
    \({ \sum_{n=1}^{\infty} \frac{1}{n^2} }\) converges.
    What is the key idea in the proof?

    \begin{cloze}{1}
        \({ \frac{1}{n^2} \leq \frac{1}{n(n - 1)} }\).
    \end{cloze}
\end{note}

\begin{note}{cd5ca73daf014641b49c5445adcd69b5}
    Does \({ \sum_{n=1}^{\infty} \frac{1}{n} }\) converge?

    \begin{cloze}{1}
        No.
    \end{cloze}
\end{note}

\begin{note}{184fe5e5e62b4c3f8a49c4ea6d26c240}
    \({ \sum_{n=1}^{\infty} \frac{1}{n} }\) diverges.
    What is the key idea in the proof?

    \begin{cloze}{1}
        Find a lower bound using powers of two.
    \end{cloze}
\end{note}

\begin{note}{4608dd8499934012aadc1209fb34ec1e}
    \begin{icloze}{2}\({ \sum_{n=1}^{\infty} \frac{1}{n} }\)\end{icloze} is called \begin{icloze}{1}the harmonic series.\end{icloze}
\end{note}

\begin{note}{ccea4c33507e4d5f9387c996a8bb13ad}
    Let \({ (a_n) }\) be \begin{icloze}{5}a decreasing sequence\end{icloze} and \begin{icloze}{4}\({ a_n \geq 0 }\).\end{icloze}
    Then
    \[
        \begin{icloze}{2}\sum_{n=1}^{\infty} a_n \text{ converges}\end{icloze}
        \begin{icloze}{3}\iff\end{icloze}
        \begin{icloze}{1}\sum_{n=1}^{\infty} 2^{n} a_{2^{n}} \text{ converges.}\end{icloze}
    \]

    \begin{center}
        \tiny
        <<\begin{icloze}{6}Cauchy Condensation Test\end{icloze}>>
    \end{center}
\end{note}

\begin{note}{88287ba71bd545459ba16b4e2ca5cb69}
    Let \({ (a_n) }\) be a decreasing sequence and \({ a_n \leq 0 }\).
    Then
    \[
        \sum_{n=1}^{\infty} a_n \text{ converges} \iff \sum_{n=1}^{\infty} 2^{n} a_{2^{n}} \text{ converges.}
    \]
    What is the key idea in the proof?

    \begin{cloze}{1}
        Group the element of a partial sum in chunks of size \({ 2^{m} }\).
    \end{cloze}
\end{note}

\begin{note}{7dfc9afff8a045caa6549458d3264c8d}
    The series \({ \sum_{n=1}^{\infty} \frac{1}{n^{p}} }\) \begin{icloze}{2}converges\end{icloze} \begin{icloze}{3}if and only if\end{icloze} \begin{icloze}{1}\({ p > 1 }\).\end{icloze}
\end{note}

\begin{note}{66666197109243728959180963a362d4}
    The series \({ \sum_{n=1}^{\infty} \frac{1}{n^{p}} }\) converges if and only if \({ p > 1 }\).
    What is the key idea in the proof?

    \begin{cloze}{1}
        The Cauchy Condensation Test and the convergence of geometric series.
    \end{cloze}
\end{note}

\section{Properties of Infinite Series} % 2.7
\begin{note}{51836e3c068e46888891ad60f449bd6}
    Let \({ \sum_{k=1}^{\infty} a_k = A }\) and \({ c \in \mathbf{R} }\).
    Under which condition does
    \[
        \sum_{k=1}^{\infty} ca_k
    \]
    converge?

    \begin{cloze}{1}
        Always.
    \end{cloze}
\end{note}

\begin{note}{548101004aba462b8e81b2c4f7cbd1b9}
    If \({ \sum_{k=1}^{\infty} a_k = A }\) and \({ c \in \mathbf{R} }\), then \({ \sum_{k=1}^{\infty} ca_k = \begin{icloze}{1}cA\end{icloze} }\).
\end{note}

\begin{note}{30607fca749d4ea9814ec7460a102865}
    Let \({ \sum_{k=1}^{\infty} a_k = A }\) and \({ \sum_{k=1}^{\infty} b_k = B }\).
    Under which condition does
    \[
        \sum_{k=1}^{\infty} a_k + b_k
    \]
    converge?

    \begin{cloze}{1}
        Always.
    \end{cloze}
\end{note}

\begin{note}{4f1064d2b18d4e889fa4e80010f532b1}
    If \({ \sum_{k=1}^{\infty} a_k = A }\) and \({ \sum_{k=1}^{\infty} b_k = B }\), then
    \[
        \sum_{k=1}^{\infty} a_k + b_k = \begin{icloze}{1}A + B.\end{icloze}
    \]
\end{note}

\begin{note}{6795efea2a204bfb90bf19f3ac01f60a}
    The series \({ \sum_{k=1}^{\infty} a_k }\) \begin{icloze}{5}converges\end{icloze} \begin{icloze}{4}if and only if,\end{icloze} given \begin{icloze}{3}\({ \epsilon > 0 }\),\end{icloze} there exists \begin{icloze}{2}an \({ N \in \mathbf{N} }\)\end{icloze} such that whenever \begin{icloze}{2}\({ n > m \geq N }\)\end{icloze} it follows that
    \begin{icloze}{1}
        \[
            \left\lvert a_{m + 1} + \cdots + a_n \right\rvert < \epsilon.
        \]
    \end{icloze}
\end{note}

\begin{note}{f83e35fa266b4b71ae674a5ae53196aa}
    The series \({ \sum_{k=1}^{\infty} a_k }\) converges if and only if, given \({ \epsilon > 0 }\), there exists an \({ N \in \mathbf{N} }\) such that whenever \({ n > m \geq N }\) it follows that
    \[
        \left\lvert a_{m + 1} + \cdots + a_n \right\rvert < \epsilon.
    \]

    \begin{center}
        \tiny
        <<\begin{icloze}{1}Cauchy Criterion\end{icloze}>>
    \end{center}
\end{note}

\begin{note}{255fd1a8d1ca40ddbe4706f396dcaad5}
    What is the key idea in the proof of the Cauchy Criterion for Series?

    \begin{cloze}{1}
        Cauchy Criterion for the sequence of partial sums.
    \end{cloze}
\end{note}

\begin{note}{2ccccd666d0d4025a48baaa6ac297e88}
    If the series \({ \sum_{k=1}^{\infty} a_k }\) \begin{icloze}{2}converges,\end{icloze} then \begin{icloze}{1}\({ (a_k) \to 0 }\).\end{icloze}
\end{note}

\begin{note}{e553a27c1b0240b4a08a2d2e1291a1c5}
    If the series \({ \sum_{k=1}^{\infty} a_k }\) converges, then \({ (a_k) \to 0 }\).
    What is the key idea in the proof?

    \begin{cloze}{1}
        Apply the Cauchy Criterion with \({ n = m + 1 }\).
    \end{cloze}
\end{note}

\begin{note}{0314d6d2761e4bd1b24b1b858e9c5086}
    Assume \({ (a_k) }\) and \({ (b_k) }\) are sequences satisfying \begin{icloze}{3}\({ 0 \leq a_k \leq b_k }\) for all \({ k \in \mathbf{N} }\).\end{icloze}
    If \({ \sum_{k=1}^{\infty} \begin{icloze}{1}b_k\end{icloze} }\) \begin{icloze}{2}converges,\end{icloze} then \({ \sum_{k=1}^{\infty} \begin{icloze}{1}a_k\end{icloze} }\) \begin{icloze}{2}converges.\end{icloze}
\end{note}

\begin{note}{03fddbcdb39340e0a421d24fe7298f2e}
    Assume \({ (a_k) }\) and \({ (b_k) }\) are sequences satisfying \({ 0 \leq a_k \leq b_k }\) for all \({ k \in \mathbf{N} }\).
    If \({ \sum_{k=1}^{\infty} \begin{icloze}{1}a_k\end{icloze} }\) \begin{icloze}{2}diverges,\end{icloze} then \({ \sum_{k=1}^{\infty} \begin{icloze}{1}b_k\end{icloze} }\) \begin{icloze}{2}diverges.\end{icloze}
\end{note}

\begin{note}{d6553b70220c4348a7c7692a58f91271}
    Assume \({ (a_k) }\) and \({ (b_k) }\) are sequences satisfying \({ 0 \leq a_k \leq b_k }\) for all \({ k \in \mathbf{N} }\).
    If \({ \sum_{k=1}^{\infty} b_k }\) converges, then \({ \sum_{k=1}^{\infty} a_k }\) converges.

    \begin{center}
        \tiny
        <<\begin{icloze}{1}Comparison Test\end{icloze}>>
    \end{center}
\end{note}

\begin{note}{7f40a1b03ff44e75af1465ca5e329e3e}
    What is the key idea in the proof of the Comparison Test for Series?

    \begin{cloze}{1}
        Use the Cauchy Criterion explicitly.
    \end{cloze}
\end{note}

% Lecture 12.09.22 {{{
\begin{note}{e02413e7068f47d28eab58d2542d2858}
    What series are considered in the Limit Comparison Test?

    \begin{cloze}{1}
        Positive and one containing no zeros.
    \end{cloze}
\end{note}

\begin{note}{cedc7eb1b2ac4f578caebcbaf4398f01}
    Which value is considered in the Limit Comparison Test?

    \begin{cloze}{1}
        The limit of the ratio of corresponding terms.
    \end{cloze}
\end{note}

\begin{note}{9ce4a06cfa6e42c7bae44e61649416d4}
    Which cases exist on the Limit Comparison Test?

    \begin{cloze}{1}
        The limit is finite or is nonzero.
    \end{cloze}
\end{note}

\begin{note}{bb6597f3ea41409da5895548c598ddae}
    What can we say from the Limit Comparison Test if the limit is finite?

    \begin{cloze}{1}
        The denominator's series convergence implies that of the numerator.
    \end{cloze}
\end{note}

\begin{note}{08f8caf25dd54ec8bd0b0a03a66d00f2}
    What can we say from the Limit Comparison Test if the limit is nonzero?

    \begin{cloze}{1}
        The numerator's series convergence implies that of the denominator.
    \end{cloze}
\end{note}

\begin{note}{8474f88f7b4140dabe637c96e7a5005d}
    What can we say from the Limit Comparison Test if the limit is finite and nonzero?

    \begin{cloze}{1}
        The two series's convergences are equivalent.
    \end{cloze}
\end{note}

\begin{note}{ca9aa1db61144f7e99c9c0ead13fed2f}
    What can we say from the Limit Comparison Test if the limit does not exist?

    \begin{cloze}{1}
        Nothing.
    \end{cloze}
\end{note}

\begin{note}{34848474b28a469dbb7bc1859e1ab612}
    What is the key idea in the proof of the Limit Comparison Test (finite limit)?

    \begin{cloze}{1}
        The set of ratios is bounded above + the Comparison Test.
    \end{cloze}
\end{note}

\begin{note}{6f66af55f5d042cb85559bf7718f0641}
    What is the key idea in the proof of the Limit Comparison Test (nonzero limit)?

    \begin{cloze}{1}
        Swap the numerator and the denominator.
    \end{cloze}
\end{note}
% }}}

\begin{note}{1f9364c8930f4fedbfb3501d9a92ee2e}
    Statements about \begin{icloze}{2}convergence\end{icloze} of sequences and series are immune to \begin{icloze}{1}changes in some finite number of initial terms.\end{icloze}
\end{note}

\begin{note}{89c3e03f687b4c4aa41185f6c668d327}
    A series is called \begin{icloze}{2}geometric\end{icloze} if it is of the form
    \begin{icloze}{1}
        \[
            \sum_{k=0}^{\infty} ar^{k}.
        \]
    \end{icloze}
\end{note}

\begin{note}{4d18a586f7754236bac47a23a54ede43}
    The series \({ \sum_{k=0}^{\infty} ar^{k} }\) \begin{icloze}{2}converges\end{icloze} \begin{icloze}{3}if and only if\end{icloze} \begin{icloze}{1}\({ \left\lvert r \right\rvert < 1 }\).\end{icloze}
\end{note}

\begin{note}{f7ab1e58f37b4580a558de06c51dc6f7}
    Given \({ \left\lvert r \right\rvert < 1 }\),
    \[
        \sum_{k=0}^{\infty} ar^{k} = \begin{icloze}{1}\frac{a}{1 - r}.\end{icloze}
    \]
\end{note}

\begin{note}{c409ec230f6741b796ea4ef3e8813d9c}
    Given \({ \left\lvert r \right\rvert < 1 }\), \({ \sum_{k=0}^{\infty} ar^{k} = \frac{a}{1 - r} }\).
    What is the key idea in the proof?

    \begin{cloze}{1}
        Rewrite partial sums.
    \end{cloze}
\end{note}

\begin{note}{28dc84fd3d384adea7a15102e07c644a}
    If \begin{icloze}{2}the series \({ \sum_{k=1}^{\infty} \left\lvert a_k \right\rvert }\) converges,\end{icloze} then \begin{icloze}{1}\({ \sum_{k=1}^{\infty} a_k }\) converges.\end{icloze}

    \begin{center}
        \tiny
        <<\begin{icloze}{3}Absolute Convergence Test\end{icloze}>>
    \end{center}
\end{note}

\begin{note}{fb10bc5e919347ffa66da221bf832aa3}
    What is the key idea in the proof of the Absolute Convergence Test?

    \begin{cloze}{1}
        The Cauchy Criterion and the Triangle Inequality.
    \end{cloze}
\end{note}

\begin{note}{998d23f7cbbb49ed885b7ef2f62bb629}
    Let \({ (a_k) }\) be \begin{icloze}{4}a decreasing sequence\end{icloze} and \begin{icloze}{3}\({ (a_k) \to 0 }\).\end{icloze}
    Then
    \begin{icloze}{2}
        \[
            \sum_{k=0}^{\infty} (-1)^{k}a_k
        \]
    \end{icloze}
    \begin{icloze}{1}converges.\end{icloze}
\end{note}

\begin{note}{df767d19abbf4031899b4a87577b2625}
    Let \({ (a_k) }\) be a decreasing sequence and \({ (a_k) \to 0 }\).
    Then
    \[
        \sum_{k=0}^{\infty} (-1)^{k}a_k
    \]
    converges.

    \begin{center}
        \tiny
        <<\begin{icloze}{1}Alternating Series Test\end{icloze}>>
    \end{center}
\end{note}

\begin{note}{5023b0a2f0ca4300bfa09b61e0ec0a9c}
    \begin{icloze}{1}An alternating series\end{icloze} is a series of the form
    \begin{icloze}{2}
        \[
            \sum_{k=0}^{\infty} (-1)^{k}a_k,
        \]
    \end{icloze}
    where \begin{icloze}{3}all \({ a_k > 0 }\).\end{icloze}
\end{note}

\begin{note}{6fb766a68cd14aa395c223e4a0e95999}
    What is the key idea in the proof of the Alternating Series Test?

    \begin{cloze}{1}
        The Cauchy criterion for the sequence of partial sums.
    \end{cloze}
\end{note}

\begin{note}{9bfa24b4310b474db9705bceed02cc45}
    Which intervals are considered in the proof of the Alternating Series Test?

    \begin{cloze}{1}
        Those formed by successive partial sums.
    \end{cloze}
\end{note}

\begin{note}{a581365ace824e89ae7a397fe6d02f1d}
    In the proof of the Alternating Series Test, how to you choose \({ \Delta_{s_m, s_{m+1}} }\), given \({ \epsilon > 0 }\)?

    \begin{cloze}{1}
        So that its length is less then \({ \epsilon }\).
    \end{cloze}
\end{note}

\begin{note}{a77a5abf0f2a46e8af759deffbaeed9e}
    In the proof of the Alternating Series Test, what do you need to show about an interval \({ \Delta_{s_m, s_{m+1}} }\)?

    \begin{cloze}{1}
        It contains all of the following partial sums.
    \end{cloze}
\end{note}

\begin{note}{cb8249219a644a12b50a90701e47e548}
    We say \({ \sum_{k=1}^{\infty} a_k }\) \begin{icloze}{2}converges absolutely,\end{icloze} if \begin{icloze}{1}\({ \sum_{k=1}^{\infty} \left\lvert a_k \right\rvert }\) converges.\end{icloze}
\end{note}

\begin{note}{c07bf73c30a04766803b1c0fae6b38d9}
    We say \({ \sum_{k=1}^{\infty} a_k }\) \begin{icloze}{2}converges conditionally,\end{icloze} if \begin{icloze}{1}it converges and does not converge absolutely.\end{icloze}
\end{note}

% Lecture 05.09.22 {{{
\begin{note}{f54a6f91b89f42c7b548ace2e106608d}
    A series \({ \sum_{k=1}^{\infty} a_k }\) is said to be \begin{icloze}{2}positive\end{icloze} if \begin{icloze}{1}\({ a_k \geq 0 }\) for all \({ k \in \mathbf{N} }\).\end{icloze}
\end{note}

\begin{note}{c5acade4dde342f8b7ac4acec2278ac6}
    Any \begin{icloze}{2}positive\end{icloze} convergent series must \begin{icloze}{1}converge absolutely.\end{icloze}
\end{note}

\begin{note}{e85b9eb09cfa4056b868f983703a571c}
    May a positive series diverge?

    \begin{cloze}{1}
        Only to \({ +\infty }\).
    \end{cloze}
\end{note}

\begin{note}{b65eba46e51c438e933833ad313a4cf8}
    A \begin{icloze}{2}positive\end{icloze} series converges \begin{icloze}{3}if and only if\end{icloze} \begin{icloze}{1}the sequence of partial sums \({ (s_n) }\) is bounded.\end{icloze}
\end{note}
% }}}

\begin{note}{4ef68f3ca3544ea98fd3c54340c65ce5}
    Let \({ \sum_{k=1}^{\infty} a_k }\) be a series and \begin{icloze}{3}\({ f : \mathbf{N} \to \mathbf{N} }\) be 1--1 and onto.\end{icloze}
    \begin{icloze}{2}The series \({ \sum_{k=1}^{\infty} a_{f(k)} }\)\end{icloze} is called \begin{icloze}{1}a rearrangement of \({ \sum_{k=1}^{\infty} a_k }\).\end{icloze}
\end{note}

\begin{note}{4071d910f5e6410cb2b01dfc73ae48da}
    If a series \begin{icloze}{2}converges absolutely,\end{icloze} then \begin{icloze}{3}any rearrangement of this series\end{icloze} \begin{icloze}{1}converges to the same limit.\end{icloze}
\end{note}

\begin{note}{057430cb21934da7ac9bc037ba169eb5}
    If a series converges absolutely, then any rearrangement of this series converges to the same limit.
    What is the key idea in the proof?

    \begin{cloze}{1}
        Substitute the original series' initial terms for the re\-ar\-range\-ment's partial sum.
    \end{cloze}
\end{note}

\begin{note}{d572332d7e36407ab1531e824f794b4b}
    If a series converges absolutely, then any rearrangement of this series converges to the same limit.
    In the proof, how many of the original series' initial terms are substituted from the rearrangement's partial sum?

    \begin{cloze}{1}
        So as to use the definition of convergence and the Cauchy Criterion for absolute convergence.
    \end{cloze}
\end{note}

\begin{note}{574ee484bcf94971932baee731b90c95}
    If a series converges absolutely, then any rearrangement of this series converges to the same limit.
    In the proof, how many of the rearrangement's terms are taken for the partial sum?

    \begin{cloze}{1}
        So as to contain the initial terms of the original sequence.
    \end{cloze}
\end{note}

\begin{note}{c50d4f3043cb4ca38411c1b1dc20ae26}
    If a series converges absolutely, then any rearrangement of this series converges to the same limit.
    In the proof we denote \begin{icloze}{2}\({ s_n }\)\end{icloze} to be \begin{icloze}{1}the original series' partial sum.\end{icloze}
\end{note}

\begin{note}{2f9195ab94ee4143800fc5300d10d80f}
    If a series converges absolutely, then any rearrangement of this series converges to the same limit.
    In the proof we denote \begin{icloze}{2}\({ t_n }\)\end{icloze} to be \begin{icloze}{1}the rearrangement' partial sum.\end{icloze}
\end{note}

\begin{note}{1bacf92272b04fc98d69ac25f5fcdfe2}
    If a series converges absolutely, then any rearrangement of this series converges to the same limit.
    In the proof, what do we show about \({ t_m - s_N }\)?

    \begin{cloze}{1}
        \({ \left\lvert t_m - s_N \right\rvert < \varepsilon }\)
    \end{cloze}
\end{note}

\begin{note}{6e8705bf5bd84118a85ac3eb8a1d5e28}
    If a series converges absolutely, then any rearrangement of this series converges to the same limit.
    In the proof, why is it that \({ \left\lvert t_m - s_N \right\rvert < \varepsilon }\)?

    \begin{cloze}{1}
        Due to the Cauchy Criterion.
    \end{cloze}
\end{note}

\begin{note}{8ffac6aca55141b29861f55f5d1dd8fb}
    If a series converges absolutely, then any rearrangement of this series converges to the same limit.
    In the proof, how do you show \({ \left\lvert t_m - A \right\rvert < \varepsilon }\)?

    \begin{cloze}{1}
        \({ \left\lvert t_m - s_N + s_N - A \right\rvert }\) and the triangle inequality.
    \end{cloze}
\end{note}

\begin{note}{b4e0eacc15f64559b6c255552fe3aadf}
    What series are considered in the Ratio Test?

    \begin{cloze}{1}
        Strictly positive.
    \end{cloze}
\end{note}

\begin{note}{dcfddd94a3304571a442fff1f7009cb8}
    Which value is considered in the Ratio Test?

    \begin{cloze}{1}
        The limit of successive ratios.
    \end{cloze}
\end{note}

\begin{note}{d00eda65eafa4efabe918bfacc3ff819}
    Which term is placed to the numerator in the Ratio Test?

    \begin{cloze}{1}
        The next one.
    \end{cloze}
\end{note}

\begin{note}{605c64a7226c48eebe5ee34d51cd470b}
    When does the Ratio Test let us conclude something?

    \begin{cloze}{1}
        When the ratios approach a value other than \({ 1 }\).
    \end{cloze}
\end{note}

\begin{note}{a70e3ac68ab947fc8e389e85e5f54588}
    Which cases exists on the Ratio Test?

    \begin{cloze}{1}
        Ratios converge to less than, or greater than, \({ 1 }\).
    \end{cloze}
\end{note}

\begin{note}{de649e2ae5cc4b3b93aac925d3b37d4b}
    What do we conclude from the Ratio Test when the ratios converge to something less than 1?

    \begin{cloze}{1}
        The series converges.
    \end{cloze}
\end{note}

\begin{note}{3bcf7fb3ba4f4ace92b222a3c8af9174}
    What do we conclude from the Ratio Test when the ratios converge to something greater than 1?

    \begin{cloze}{1}
        The series diverges.
    \end{cloze}
\end{note}

\begin{note}{90519e5b985b4f97a25636a1473b500d}
    What do we conclude from the Ratio Test when the ratios converge to 1?

    \begin{cloze}{1}
        Nothing.
    \end{cloze}
\end{note}

\begin{note}{4bab403524b240cda38745c2324966c0}
    What do we conclude from the Ratio Test when the ratios do not converge?

    \begin{cloze}{1}
        Nothing.
    \end{cloze}
\end{note}

\begin{note}{1a0caf850c00432b93871e8c66f3397b}
    Give an example when the Ratio Test is inconclusive and the series diverges.

    \begin{cloze}{1}
        The harmonic series.
    \end{cloze}
\end{note}

\begin{note}{0c417f771ac54fa3ad89fb5d65d5f10d}
    Give an example when the Ratio Test is inconclusive and the series converges.

    \begin{cloze}{1}
        \({ \sum_{n=1}^{\infty} \frac{1}{n^2} }\).
    \end{cloze}
\end{note}

\begin{note}{0a54c42a8bd74ba883e310f36f865ca6}
    What is the nominal name of the Ratio Test?

    \begin{cloze}{1}
        The d'Alambert's Ratio Test.
    \end{cloze}
\end{note}

\begin{note}{f1e24cc124f84cf3a6d14e77ee23368b}
    What is the first step in proving the Ratio Test?

    \begin{cloze}{1}
        Split \({ r < 1 }\), \({ r > 1 }\).
    \end{cloze}
\end{note}

\begin{note}{127428f8805043978b16164456c8acf5}
    What is the key idea in the proof of the Ratio Test (\({ r > 1 }\))?

    \begin{cloze}{1}
        The terms are eventually increasing.
    \end{cloze}
\end{note}

\begin{note}{535154065a884eb7bf3e87e8d4b400e5}
    What is the first key idea in the proof of the Ratio Test (\({ r < 1 }\))?

    \begin{cloze}{1}
        For \({ r < r' < 1 }\) the ratios are eventually less than \({ r' }\).
    \end{cloze}
\end{note}

\begin{note}{5ac59226423b4b8fb84c087795e5ed6f}
    What is the second key idea in the proof of the Ratio Test (\({ r < 1 }\))?

    \begin{cloze}{1}
        Find an upper bound using a geometric series.
    \end{cloze}
\end{note}

% Lecture 12.09.22 {{{
\begin{note}{ce4c6aa5f15044a2a804f11a91d677b7}
    What series are considered in the Root Test?

    \begin{cloze}{1}
        Positive.
    \end{cloze}
\end{note}

\begin{note}{02964fce0fcd409cab46d91942e3f1c2}
    What value is considered in the Root Test?

    \begin{cloze}{1}
        The limit of \({ \sqrt[n]{a_n} }\).
    \end{cloze}
\end{note}

\begin{note}{06c9e889bae041afb32a8f2da431bbf9}
    Which cases exist on the Root Test?

    \begin{cloze}{1}
        \({ n }\)-th roots approach something less than, or greater than, \({ 1 }\).
    \end{cloze}
\end{note}

\begin{note}{562b1b6b74e24c73ad75d944ff17d581}
    When does the Root Test let us conclude something?

    \begin{cloze}{1}
        When \({ n }\)-th roots approach something other than \({ 1 }\).
    \end{cloze}
\end{note}

\begin{note}{687fe6a03e28430189cd57632f9bae0b}
    What do we conclude from the Root Test if the limit is less than \({ 1 }\)?

    \begin{cloze}{1}
        The series converges.
    \end{cloze}
\end{note}

\begin{note}{dd2315fb062b4bdf93ebe5072fc0d308}
    What do we conclude from the Root Test if the limit is greater than \({ 1 }\)?

    \begin{cloze}{1}
        The series diverges.
    \end{cloze}
\end{note}

\begin{note}{7701686caac7412aa1b3375ff77e5a9e}
    What do we conclude from the Root Test if the limit converges to \({ 1 }\)?

    \begin{cloze}{1}
        Nothing.
    \end{cloze}
\end{note}

\begin{note}{6200b936d6144cafb8b74ff7d9271a9d}
    Give an example when the root test is inconclusive and the series diverges.

    \begin{cloze}{1}
        The harmonic series.
    \end{cloze}
\end{note}

\begin{note}{6cd4fabac91944db96449403d2288e0a}
    Give an example when the root test is inconclusive and the series converges.

    \begin{cloze}{1}
        \({ \sum_{n=1}^{\infty} \frac{1}{n^2} }\).
    \end{cloze}
\end{note}

\begin{note}{644281f3c2614e2499993a48daca8aac}
    What is the nominal name for the Root Test?

    \begin{cloze}{1}
        Cauchy's Radical Test.
    \end{cloze}
\end{note}

\begin{note}{7021924723f142d489dc64e27e06c40b}
    What is the first step in proving the Root Test?

    \begin{cloze}{1}
        Split \({ r < 1 }\), \({ r > 1 }\).
    \end{cloze}
\end{note}

\begin{note}{ae27724cb07240fbb243221a41bb7f82}
    What is the first key idea in the proof of the Root Test (\({ r < 1 }\))?

    \begin{cloze}{1}
        For \({ r < r' < 1 }\) the roots are eventually less than \({ r' }\).
    \end{cloze}
\end{note}

\begin{note}{64f3efecadd94ca8ad1277cba95ded2e}
    What is the second key idea in the proof of the Root Test (\({ r < 1 }\))?

    \begin{cloze}{1}
        Find an upper bound using a geometric series.
    \end{cloze}
\end{note}

\begin{note}{e4b13d2a78bc4010ad92b3574943d982}
    What is the key idea in the proof of the Root Test (\({ r > 1 }\))?

    \begin{cloze}{1}
        Elements are eventually greater than \({ 1 }\).
    \end{cloze}
\end{note}
% }}}

\begin{note}{391b719f11404d53959a2e258908f1d0}
    What sequences are considered in te Summation-by-Parts formula?

    \begin{cloze}{1}
        Arbitrary.
    \end{cloze}
\end{note}

\begin{note}{f1a472048eb0400cafd7a7d7b0e049cc}
    What is the initial expression in the Summation-by-Parts formula?

    \begin{cloze}{1}
        \[
            \sum_{j=n}^{m} x_j y_j.
        \]
    \end{cloze}
\end{note}

\begin{note}{1424da07cc0f4c7e9e792ba2daad165c}
    Which terms are there in the transformed expression in the Summation-by-Parts formula?

    \begin{cloze}{1}
        Two free terms and one sum.
    \end{cloze}
\end{note}

\begin{note}{eb188d69b3c74c42814da0030ab179ca}
    What is the first free term of the transformed expression in the Summation-by-Parts formula?

    \begin{cloze}{1}
        The final partial sum times the next element.
    \end{cloze}
\end{note}

\begin{note}{1af3ccefd5714f279390596beb66afdb}
    What is the second free term of the transformed expression in the Summation-by-Parts formula?

    \begin{cloze}{1}
        Subtracting the partial sum preceding the range multiplied by the starting element.
    \end{cloze}
\end{note}

\begin{note}{ed63990568ac41ff9e0d1b7535e91d62}
    What is the sum term of the transformed expression in the Summation-by-Parts formula?

    \begin{cloze}{1}
        The sum of partial sums multiplied by the successive differences.
    \end{cloze}
\end{note}

\begin{note}{e46ff0f795c84a08a97ab92916d689f7}
    What is the order of successive differences in the sum term of the transformed expression in the Summation-by-Parts formula?

    \begin{cloze}{1}
        The current minus the next.
    \end{cloze}
\end{note}

\begin{note}{67d6011a7aa7477da37cbb1ab2899cea}
    What is the range of summation in the sum term of the transformed expression in the Summation-by-Parts formula?

    \begin{cloze}{1}
        Same as the original.
    \end{cloze}
\end{note}

\begin{note}{84d1c74bb7fd496a90c6f85103bb2793}
    What is the value of the zeroth partial sum in the Summation-by-Parts formula?

    \begin{cloze}{1}
        Zero.
    \end{cloze}
\end{note}

\begin{note}{6153ba6cc000482694e8ffdcea302fd4}
    What is the nominal name for the Summation-by-Parts formula?

    \begin{cloze}{1}
        The Abel Transformation.
    \end{cloze}
\end{note}

\begin{note}{a637cd28783d4349916b7db04a7b8eef}
    What is the key idea in the proof of the Summation-by-Parts formula?

    \begin{cloze}{1}
        Rewrite the sequence's values as the differences of successive partial sums.
    \end{cloze}
\end{note}

\end{document}
