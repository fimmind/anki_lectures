\documentclass[11pt, a5paper]{article}
\usepackage[width=10cm, top=0.5cm, bottom=2cm]{geometry}

\usepackage[T1,T2A]{fontenc}
\usepackage[utf8]{inputenc}
\usepackage[english,russian]{babel}
\usepackage{libertine}

\usepackage{amsmath}
\usepackage{amssymb}
\usepackage{amsthm}
\usepackage{mathrsfs}
\usepackage{framed}
\usepackage{xcolor}

\setlength{\parindent}{0pt}

% Force \pagebreak for every section
\let\oldsection\section
\renewcommand\section{\pagebreak\oldsection}

\renewcommand{\thesection}{}
\renewcommand{\thesubsection}{Note \arabic{subsection}}
\renewcommand{\thesubsubsection}{}
\renewcommand{\theparagraph}{}

\newenvironment{note}[1]{\goodbreak\par\subsection{\hfill \color{lightgray}\tiny #1}}{}
\newenvironment{cloze}[2][\ldots]{\begin{leftbar}}{\end{leftbar}}
\newenvironment{icloze}[2][\ldots]{%
  \ignorespaces\text{\tiny \color{lightgray}\{\{c#2::}\hspace{0pt}%
}{%
  \hspace{0pt}\text{\tiny\color{lightgray}\}\}}\unskip%
}


\begin{document}
\section{Sets}
\begin{note}{097312afe75d4a3d9eaa0c1f4c63748a}
    Intuitively speaking, \begin{icloze}{2}a set\end{icloze} is \begin{icloze}{1}a collection of objects.\end{icloze}
\end{note}

\begin{note}{85e21cf985524b80a8c00eb4608f34be}
    Intuitively speaking, a set is a collection of objects.
    \begin{icloze}{2}Those objects\end{icloze} are referred to as \begin{icloze}{1}the elements of the set.\end{icloze}
\end{note}

\begin{note}{12b96daebbc04070b74e2a6f74e5b268}
    Given a set \({ A }\), we write \begin{icloze}{2}\({ x \in A }\)\end{icloze} if \begin{icloze}{1}\({ x }\) is an element of \({ A }\).\end{icloze}
\end{note}

\begin{note}{b25d749749a64c5b90880253d9839da8}
    Given  a set \({ A }\), we write \begin{icloze}{2}\({ x \not\in A }\)\end{icloze} if \begin{icloze}{1}\({ x }\) is not an element of \({ A }\).\end{icloze}
\end{note}

\begin{note}{39565306ec4e40e18136e7eb88fc817a}
    Given two sets \({ A }\) and \({ B }\), \begin{icloze}{1}the union\end{icloze} is written \begin{icloze}{2}\({ A \cup B }\).\end{icloze}
\end{note}

\begin{note}{73bf0eb1d16c4c5da368e326b4739d5b}
    Given two sets \({ A }\), and \({ B }\), \begin{icloze}{2}the union\end{icloze} is \begin{icloze}{3}defined\end{icloze} by the rule
    \begin{center}
        \({ x \in \begin{icloze}{2}A \cup B\end{icloze} }\) provided that \begin{icloze}{1}\({ x \in A }\) or \({ x \in B }\).\end{icloze}
    \end{center}
\end{note}

\begin{note}{8ce7db157931494bbfb6eee706e15efc}
    Given two sets \({ A }\) and \({ B }\), \begin{icloze}{1}the intersection\end{icloze} is written \begin{icloze}{2}\({ A \cap B }\).\end{icloze}
\end{note}

\begin{note}{6a277df52de2409a98e48429d69b6d05}
    Given two sets \({ A }\) and \({ B }\), \begin{icloze}{2}the intersection\end{icloze} is \begin{icloze}{3}defined\end{icloze} by the rule
    \begin{center}
        \({ x \in \begin{icloze}{2}A \cap B\end{icloze} }\) provided that \begin{icloze}{1}\({ x \in A }\) and \({ x \in B }\).\end{icloze}
    \end{center}
\end{note}

\begin{note}{684951afc378458aa7bd27e67cdc499b}
    \begin{icloze}{2}The set of natural numbers\end{icloze} is denoted \begin{icloze}{1}\({ \mathbf{N} }\).\end{icloze}
\end{note}

\begin{note}{49d36a026d4b4678ab86fb6103571cce}
    \[
        \begin{icloze}{2}\mathbf{N}\end{icloze} \overset{\text{def}}= \left\{ \begin{icloze}{1}1, 2, 3, \ldots\end{icloze} \right\}.
    \]
\end{note}

\begin{note}{797c81e5adb543e1a5d4cc67e64c5e09}
    \begin{icloze}{2}The set of integers\end{icloze} is denoted \begin{icloze}{1}\({ \mathbf{Z} }\).\end{icloze}
\end{note}

\begin{note}{d3c61bf891744c58b73cef543c6e100d}
    \[
        \begin{icloze}{2}\mathbf{Z}\end{icloze} \overset{\text{def}}= \left\{ \begin{icloze}{1}\ldots, -2, -1, 0, 1, 2, \ldots\end{icloze} \right\}.
    \]
\end{note}

\begin{note}{57f085776972449f8bc14daf5cff6603}
    \begin{icloze}{2}The set of rational numbers\end{icloze} is denoted \begin{icloze}{1}\({ \mathbf{Q} }\).\end{icloze}
\end{note}

\begin{note}{f7e3370650134607853b41b2b1ecf54b}
    \[
        \begin{icloze}{3}\mathbf{Q}\end{icloze} \overset{\text{def}}= \left\{ \text{all \begin{icloze}{2}fractions \({ \frac{p}{q} }\)\end{icloze} where \begin{icloze}{1}\({ p, q \in \mathbf{Z} }\) and \({ q \neq 0 }\)\end{icloze}} \right\}.
    \]
\end{note}

\begin{note}{faeac83cb5b740b6964551c85ad3e35b}
    \begin{icloze}{2}The set of real numbers\end{icloze} is denoted \begin{icloze}{1}\({ \mathbf{R} }\).\end{icloze}
\end{note}

\begin{note}{6e5da98964d645d09ad6989e85679c74}
    \begin{icloze}{2}The empty\end{icloze} set is \begin{icloze}{1}the set that contains no elements.\end{icloze}
\end{note}

\begin{note}{206db0a0f3d042e49a9ca532e222201f}
    \begin{icloze}{2}The empty set\end{icloze} is denoted \begin{icloze}{1}\({ \emptyset }\).\end{icloze}
\end{note}

\begin{note}{2f0448d226db4b71b150acaed349a73b}
    Two sets \({ A }\) and \({ B }\) are said to be \begin{icloze}{2}disjoint\end{icloze} if \begin{icloze}{1}\({ A \cap B = \emptyset }\).\end{icloze}
\end{note}

\begin{note}{e5d9d365e86640319ca5460ef8c4f05c}
    Given two sets \({ A }\) and \({ B }\), we say \begin{icloze}{2}\({ A }\) is a subset of \({ B }\),\end{icloze} or \begin{icloze}{2}\({ B }\) contains \({ A }\)\end{icloze} if \begin{icloze}{1}every element of \({ A }\) is also an element of \({ B }\).\end{icloze}
\end{note}

\begin{note}{c2bd27f1fc0d40e296dceef9c9789556}
    Given two sets \({ A }\) and \({ B }\), the \begin{icloze}{3}inclusion\end{icloze} relationship \begin{icloze}{2}\({ A \subseteq B }\) or \({ B \supseteq A }\)\end{icloze} is used to indicate that \begin{icloze}{1}\({ A }\) is a subset of \({ B }\).\end{icloze}
\end{note}

\begin{note}{333e7c6716af48b7b9962ad803f0732f}
    Given two sets \({ A }\) and \({ B }\), \begin{icloze}{2}\({ A = B }\)\end{icloze} means that \begin{icloze}{1}\({ A \subseteq B }\) and \({ B \subseteq A }\).\end{icloze}
\end{note}

\begin{note}{74e93b42d46746dc9ec2b54f8366c435}
    Let \({ A_1, A_2, A_3, \ldots }\) be an infinite collection of sets.
    Notationally,
    \[
        \bigcup_{n = 1}^{\infty} A_n, \quad \bigcup_{n \in \mathbf{N}}^{} A_n, \quad \text{or} \quad A_1 \cup A_2 \cup A_3 \cup \cdots
    \]
    are all equivalent ways to indicate \begin{icloze}{1}the set whose elements consist of any element that appears in at least on particular \({ A_n }\).\end{icloze}
\end{note}

\begin{note}{69e4627a3e7149ef8be05479a2587b41}
    Let \({ A_1, A_2, A_3, \ldots }\) be an infinite collection of sets.
    Notationally,
    \[
        \bigcap_{n = 1}^{\infty} A_n, \quad \bigcap_{n \in \mathbf{N}}^{} A_n, \quad \text{or} \quad A_1 \cap A_2 \cap A_3 \cap \cdots
    \]
    are all equivalent ways to indicate \begin{icloze}{1}the set whose elements consist of any element that appears in every \({ A_n }\).\end{icloze}
\end{note}

\begin{note}{11a987e10fce4ceea69672f366597729}
    Given \({ A \subseteq \mathbf{R} }\), \begin{icloze}{2}the complement of \({ A }\)\end{icloze} refers to \begin{icloze}{1}the set of all elements of \({ \mathbf{R} }\) not in \({ A }\).\end{icloze}
\end{note}

\begin{note}{8b379552450b4672af82c17476c0ff13}
    Given \({ A \subseteq \mathbf{R} }\), \begin{icloze}{2}the complement of \({ A }\)\end{icloze} is written \begin{icloze}{1}\({ A^{c} }\).\end{icloze}
\end{note}

\begin{note}{a3459afa53264a7c82d9abd760a0c93e}
    Given \({ A, B \subseteq \mathbf{R} }\),
    \[
        \begin{icloze}{2}(A \cap B)^{c}\end{icloze} = \begin{icloze}{1}A^{c} \cup B^{c}.\end{icloze}
    \]

    \begin{center}
        \tiny
        <<\begin{icloze}{3}De Morgan's Law\end{icloze}>>
    \end{center}
\end{note}

\begin{note}{c983927aa0304e51949e2f90a2ec2614}
    Given \({ A, B \subseteq \mathbf{R} }\),
    \[
        \begin{icloze}{2}(A \cup B)^{c}\end{icloze} = \begin{icloze}{1}A^{c} \cap B^{c}.\end{icloze}
    \]

    \begin{center}
        \tiny
        <<\begin{icloze}{3}De Morgan's Law\end{icloze}>>
    \end{center}
\end{note}

\begin{note}{09322548137b46529467f2946a4952d4}
    What is the key idea in the proof of De Morgan's Laws?

    \begin{cloze}{1}
        Demonstrate inclusion both ways.
    \end{cloze}
\end{note}

\section{Functions}
\begin{note}{18930cfe4e4445779bcec8a2fb53f23c}
    Given \begin{icloze}{3}two sets \({ A }\) and \({ B }\),\end{icloze} \begin{icloze}{2}a function from \({ A }\) to \({ B }\)\end{icloze} is \begin{icloze}{1}a rule or mapping that takes each element \({ x \in A }\) and associates with it a single element of \({ B }\).\end{icloze}
\end{note}

\begin{note}{dfa898ef047e418fa8dfe9ee9582fd71}
    \begin{icloze}{1}If \({ f }\) is a function from \({ A }\) to \({ B }\),\end{icloze} we write \begin{icloze}{2}\({ f : A \to B }\).\end{icloze}
\end{note}

\begin{note}{c2730dafa0fe4bf4bede66b7199b48b9}
    Let \({ f : A \to B }\).
    Given \begin{icloze}{3}\({ x \in A }\),\end{icloze} the expression \begin{icloze}{2}\({ f(x) }\)\end{icloze} is used to represent \begin{icloze}{1}the element of \({ B }\) associated with \({ x }\) by \({ f }\).\end{icloze}
\end{note}

\begin{note}{65568f366ca949888310668475dbe570}
    Let \({ f : A \to B }\).
    \begin{icloze}{2}The set \({ A }\)\end{icloze} is called \begin{icloze}{1}the domain of \({ f }\).\end{icloze}
\end{note}

\begin{note}{7870a310786142fa938bcc843ca8e1ae}
    Let \({ f : A \to B }\).
    \begin{icloze}{2}The set \({ \left\{ f(x) \mid x \in A \right\} }\)\end{icloze} is called \begin{icloze}{1}the range of \({ f }\).\end{icloze}
\end{note}

\begin{note}{716c208c9ae849b89ec722aa17f20882}
    Given a function \({ f }\) and \begin{icloze}{3}a subset \({ A }\) of its domain,\end{icloze}
    \begin{icloze}{2}
        the set
        \[
            \left\{ f(x) : x \in A \right\}
        \]
    \end{icloze}
    is called \begin{icloze}{1}the range of \({ f }\) over the set \({ A }\).\end{icloze}
\end{note}

\begin{note}{24aae21652754fcda1267ac61036a3ea}
    Given a function \({ f }\) and a subset \({ A }\) of its domain, \begin{icloze}{2}the range of \({ f }\) over \({ A }\)\end{icloze} is written \begin{icloze}{1}\({ f(A) }\).\end{icloze}
\end{note}

\begin{note}{6ed2fb1006634dcf81707a3c4d514857}
    Let \({ f : D \to \mathbf{R} }\),\: \({ A, B \subseteq D }\).
    Is it unconditionally true that
    \[
        f(A \cup B) = f(A) \cup f(B)?
    \]

    \begin{cloze}{1}
        Yes.
    \end{cloze}
\end{note}

\begin{note}{ee665e77ac9a45cf9a15d42549e6f382}
    Let \({ f : D \to \mathbf{R} }\),\: \({ A, B \subseteq D }\).
    Is it unconditionally true that
    \[
        f(A \cap B) = f(A) \cap f(B)?
    \]

    \begin{cloze}{1}
        No.
    \end{cloze}
\end{note}

\begin{note}{5d2e9d4e1e094e06b37bd87e2c9edff8}
    Given \begin{icloze}{4}\({ a, b \in \mathbf{R} }\)\end{icloze} and \begin{icloze}{3}\({ a \leq b }\)\end{icloze},
    \begin{icloze}{2}
        the set
        \[
            \left\{ x \in \mathbf{R} : a \leq x \leq b \right\}
        \]
    \end{icloze}
    is called \begin{icloze}{1}a closed interval.\end{icloze}
\end{note}

\begin{note}{9f383a22fc724f8fa43af5cb65e0cd5a}
    Given \({ a, b \in \mathbf{R} }\) and \begin{icloze}{3}\({ a < b }\)\end{icloze},
    \begin{icloze}{2}
        the set
        \[
            \left\{ x \in \mathbf{R} : a < x < b \right\}
        \]
    \end{icloze}
    is called \begin{icloze}{1}an open interval.\end{icloze}
\end{note}

\begin{note}{3143096eb895471bac4b2d5840d18758}
    Given \({ a, b \in \mathbf{R} }\) and \({ a \leq b }\),
    \begin{icloze}{1}
        the closed interval
        \[
            \left\{ x \in \mathbf{R} : a \leq x \leq b \right\}
        \]
    \end{icloze}
    is written \begin{icloze}{2}\({ [a, b] }\).\end{icloze}
\end{note}

\begin{note}{604897f024bd4de78723fe8247290371}
    Given \({ a, b \in \mathbf{R} }\) and \({ a \leq b }\),
    \begin{icloze}{1}
        the open interval
        \[
            \left\{ x \in \mathbf{R} : a < x < b \right\}
        \]
    \end{icloze}
    is written \begin{icloze}{2}\({ (a, b) }\).\end{icloze}
\end{note}

\begin{note}{a77dc72d26be45c185900ba7ff132b05}
    Let \({ f(x) = x^2 }\). Find two sets \({ A }\) and \({ B }\) for which
    \[
        f(A \cap B) \neq f(A) \cap f(B).
    \]

    \begin{cloze}{1}
        Singletons \({ \left\{ -1 \right\} }\) and \({ \left\{ 1 \right\} }\).
    \end{cloze}
\end{note}

\begin{note}{6ed2fb1006634dcf81707a3c4d514857}
    Let \({ f : D \to \mathbf{R} }\),\: \({ A, B \subseteq D }\).
    Then
    \[
        \begin{icloze}{3}f(A \cup B)\end{icloze} \begin{icloze}{1}=\end{icloze} \begin{icloze}{2}f(A) \cup f(B).\end{icloze}
    \]
\end{note}

\begin{note}{e088ae5ae1f24425a81dac09317978fd}
    Let \({ f : D \to \mathbf{R} }\),\: \({ A, B \subseteq D }\).
    Then
    \[
        \begin{icloze}{3}f(A \cap B)\end{icloze} \begin{icloze}{1}\subseteq\end{icloze} \begin{icloze}{2}f(A) \cap f(B).\end{icloze}
    \]
\end{note}

\begin{note}{f951f5a5136248dcb413f59b3271d389}
    Given \({ x \in \mathbf{R} }\), \begin{icloze}{2}the absolute value of \({ x }\)\end{icloze} is denoted \begin{icloze}{1}\({ \left\lvert x \right\rvert }\).\end{icloze}
\end{note}

\begin{note}{624dda908fd64a1cadae2b61c1277c59}
    Given \({ x \in \mathbf{R} }\),
    \[
        \left\lvert x \right\rvert \overset{\text{def}}= \begin{cases}
            \begin{icloze}{1}x,\end{icloze} & \text{if \begin{icloze}{2}\({ x \geq 0 }\)\end{icloze}}, \\
            \begin{icloze}{1}-x,\end{icloze} & \text{if \begin{icloze}{2}\({ x < 0 }\)\end{icloze}}.
        \end{cases}
    \]
\end{note}

\begin{note}{0ab23d0afe1448e397cad330aea55883}
    Given \({ a, b \in \mathbf{R} }\), \quad \({ \left\lvert ab \right\rvert = \begin{icloze}{1}\left\lvert a \right\rvert \cdot \left\lvert b \right\rvert\end{icloze} }\).
\end{note}

\begin{note}{2b51f36fba524365b72001d318791436}
    Given \({ a, b \in \mathbf{R} }\), \quad \({ \begin{icloze}{2}\left\lvert a + b \right\rvert\end{icloze} \begin{icloze}{3}\leq\end{icloze} \begin{icloze}{1}\left\lvert a \right\rvert + \left\lvert b \right\rvert\end{icloze} }\).

    \begin{center}
        \tiny
        <<\begin{icloze}{4}Triangle inequality\end{icloze}>>
    \end{center}
\end{note}

\begin{note}{4d6e77677e884f9c8ee877b9a32d48b5}
    Let \({ f : A \to B }\). The function \({ f }\) is \begin{icloze}{2}one-to-one\end{icloze} if
    \begin{icloze}{1}
        \begin{center}
            \({ a_1 \neq a_2 }\) in \({ A }\) implies that \({ f(a_1) \neq f(a_2) }\) in \({ B }\).
        \end{center}
    \end{icloze}
\end{note}

\begin{note}{56b2bf81daaf419ab1207c6693c981e6}
    Let \({ f : A \to B }\). The function \({ f }\) is \begin{icloze}{2}onto\end{icloze} if
    \begin{icloze}{1}
        \begin{center}
            the range of \({ f }\) equals \({ B }\).
        \end{center}
    \end{icloze}
\end{note}

\begin{note}{ccc8a358284a4b1f99f8e4336a2efdb9}
    Let \begin{icloze}{4}\({ f : D \to \mathbf{R} }\)\end{icloze} and \begin{icloze}{3}\({ B \subseteq \mathbf{R} }\).\end{icloze}
    \begin{icloze}{2}
        The set
        \[
            \left\{ x \in D : f(x) = B \right\}
        \]
    \end{icloze}
    is called \begin{icloze}{1}the preimage of \({ B }\) under the function \({ f }\).\end{icloze}
\end{note}

\begin{note}{b72f131ae6734bf694fd8f987bb2323d}
    Let \({ f : D \to \mathbf{R} }\) and \({ A, B \subseteq \mathbf{R} }\).
    Is it unconditionally true that
    \[
        f^{-1}(A \cup B) = f^{-1}(A) \cup f^{-1}(B)?
    \]

    \begin{cloze}{1}
        Yes.
    \end{cloze}
\end{note}

\begin{note}{5b3116f568a34fe2be32f403d7d081d9}
    Let \({ f : D \to \mathbf{R} }\) and \({ A, B \subseteq \mathbf{R} }\).
    Is it unconditionally true that
    \[
        f^{-1}(A \cap B) = f^{-1}(A) \cap f^{-1}(B)?
    \]

    \begin{cloze}{1}
        Yes.
    \end{cloze}
\end{note}

\section{Logic and Proofs}
\begin{note}{a4d52b740f5b494696a5bdc956906cf2}
    % source: https://en.wikipedia.org/wiki/Theorem
    Many mathematical theorems are conditional statements, whose proofs deduce conclusions from conditions. Given such a theorem, \begin{icloze}{1}those conditions\end{icloze} are known \begin{icloze}{2}as the theorem's hypotheses.\end{icloze}
\end{note}

\begin{note}{93f759e32dbf497cb30754e24c5b09f1}
    When in \begin{icloze}{3}a proof by contradiction\end{icloze} \begin{icloze}{2}the contradiction is with the theorem's hypothesis,\end{icloze} the proof is said to be \begin{icloze}{1}contrapositive.\end{icloze}
\end{note}

\begin{note}{1f45350926704df98b0abdf205f4319c}
    Two real number \({ a }\) and \({ b }\) are \begin{icloze}{4}equal\end{icloze} \begin{icloze}{3}if and only if\end{icloze} \begin{icloze}{2}for every real number \({ \epsilon > 0 }\) it follows that\end{icloze} \begin{icloze}{1}\({ \left\lvert a - b \right\rvert < \epsilon }\).\end{icloze}
\end{note}

\begin{note}{3ef90c9123e64df39ae9cd34271a7dcd}
    Two real number \({ a }\) and \({ b }\) are equal \({ \impliedby }\) for every real number \({ \epsilon > 0 }\) it follows that \({ \left\lvert a - b \right\rvert < \epsilon }\).
    What is the key idea in the proof?

    \begin{cloze}{1}
        By contradiction.
    \end{cloze}
\end{note}

\begin{note}{aab4bb967d814e87bd85608277093755}
    Let \begin{icloze}{3}\({ S \subseteq \mathbf{N} }\).\end{icloze}
    If \begin{icloze}{2}\({ S }\) contains \({ 1 }\)\end{icloze} and \begin{icloze}{2}whenever \({ S }\) contains \({ n }\), it also contains \({ n + 1 }\),\end{icloze} then \begin{icloze}{1}\({ S = \mathbf{N} }\).\end{icloze}
\end{note}

\begin{note}{3dd92625856f408b9dc93fd36d82588d}
    Let \({ S \subseteq \mathbf{N} }\).
    If \({ S }\) contains \({ 1 }\) and whenever \({ S }\) contains \({ n }\), it also contains \({ n + 1 }\), then \({ S = \mathbf{N} }\).
    This proposition is the fundamental principle behind \begin{icloze}{1}induction.\end{icloze}
\end{note}

\begin{note}{40977a19a0d043c985df5676daa9f776}
    Does an induction argument imply the validity of the infinite case?

    \begin{cloze}{1}
        No, it doesn't.
    \end{cloze}
\end{note}

\begin{note}{91b673c484b442ec92dd47ad0ef95f6c}
    Do De Morgan's rules hold for an infinite collection of sets?

    \begin{cloze}{1}
        Yes, they do.
    \end{cloze}
\end{note}

\begin{note}{df9aa3b9e0c74da78d7e2a0a65276fcd}
    How De Morgan's rules for an infinite collection of sets defer from that for a finite collection?

    \begin{cloze}{1}
        They are essentially the same.
    \end{cloze}
\end{note}

\section{The Axiom of Completeness}
\begin{note}{d7df92f228f64fb28a9e353f0fcb3160}
    First, \({ \mathbf{R} }\) is \begin{icloze}{1}an ordered field, which contains \({ \mathbf{Q} }\) as a subfield.\end{icloze}
\end{note}

\begin{note}{6ac3816effb14ba682f20f91ae42bfdf}
    What is the key distinction between \({ \mathbf{R} }\) and \({ \mathbf{Q} }\)?

    \begin{cloze}{1}
        The Axiom of Completeness.
    \end{cloze}
\end{note}

\begin{note}{7c2ddbcb52224d5cbad5c650d77e8a4f}
    \begin{icloze}{1}Every nonempty set of real numbers\end{icloze} that is \begin{icloze}{2}bounded above\end{icloze} has \begin{icloze}{3}a least upper bound.\end{icloze}

    \begin{center}
        \tiny
        <<\begin{icloze}{4}Axiom of completeness\end{icloze}>>
    \end{center}
\end{note}

\begin{note}{fddbb10e685c4ad49d1af25d241c03c0}
    Given a set \({ A \subseteq \mathbf{R} }\), \begin{icloze}{3}a number \({ b \in \mathbf{R} }\)\end{icloze} such that \begin{icloze}{2}\({ a \leq b }\) for all \({ a \in A }\)\end{icloze} is called \begin{icloze}{1}an upper bound for \({ A }\).\end{icloze}
\end{note}

\begin{note}{1edcfd8354464c81ab51da0d4f2f2ca4}
    A set \({ A \subseteq \mathbf{R} }\) is \begin{icloze}{2}bounded above\end{icloze} if \begin{icloze}{1}there exists an upper bound for \({ A }\).\end{icloze}
\end{note}

\begin{note}{c757fa0c676941b0a4abbccb3a67fb2a}
    Given a set \({ A \subseteq \mathbf{R} }\), \begin{icloze}{3}a number \({ b \in \mathbf{R} }\)\end{icloze} such that \begin{icloze}{2}\({ a \geq b }\) for all \({ a \in A }\)\end{icloze} is called \begin{icloze}{1}a lower bound for \({ A }\).\end{icloze}
\end{note}

\begin{note}{3c9ba92f774e439dbcfb6c364a88f0ae}
    A set \({ A \subseteq \mathbf{R} }\) is \begin{icloze}{2}bounded below\end{icloze} if \begin{icloze}{1}there exists a lower bound for \({ A }\).\end{icloze}
\end{note}

\begin{note}{40f7ae4897174d37952c83f51894ab53}
    A set \({ A \subseteq \mathbf{R} }\) is \begin{icloze}{2}bounded\end{icloze} if \begin{icloze}{1}it is bounded above and below.\end{icloze}
\end{note}

\begin{note}{9d2391299602497abd4fdfac14c71daa}
    Let \({ A \subseteq \mathbf{R} }\). \begin{icloze}{4}A real number \({ s }\)\end{icloze} is \begin{icloze}{3}the least upper bound for \({ A }\)\end{icloze} if
    \begin{itemize}
        \item{} \begin{icloze}{2}\({ s }\) is an upper bound for \({ A }\);\end{icloze}
        \item{} \begin{icloze}{1}if \({ b }\) is any upper bound for \({ A }\), then \({ s \leq b }\).\end{icloze}
    \end{itemize}
\end{note}

\begin{note}{5369939ee0f94abcaf65896355258f0d}
    \begin{icloze}{2}The least upper bound\end{icloze} of a set \({ A \subseteq \mathbf{R} }\) is also frequently called \begin{icloze}{1}the supremum of \({ A }\).\end{icloze}
\end{note}

\begin{note}{04884b60726641c6b8d7c2c3479f8b05}
    \begin{icloze}{2}The least upper bound\end{icloze} of a set \({ A \subseteq \mathbf{R} }\) is denoted \begin{icloze}{1}\({ \sup A }\).\end{icloze}
\end{note}

\begin{note}{afca84537fdd409e97254e6d36d736c3}
    Let \({ A \subseteq \mathbf{R} }\). A real number \({ s }\) is \begin{icloze}{3}the greatest lower bound for \({ A }\)\end{icloze} if
    \begin{itemize}
        \item{} \begin{icloze}{2}\({ s }\) is a lower bound for \({ A }\);\end{icloze}
        \item{} \begin{icloze}{1}if \({ b }\) is any lower bound for \({ A }\), then \({ s \geq b }\).\end{icloze}
    \end{itemize}
\end{note}

\begin{note}{41c9913ebc524f85be951737dc3e33e8}
    \begin{icloze}{2}The greatest lower bound\end{icloze} of a set \({ A \subseteq \mathbf{R} }\) is also frequently called \begin{icloze}{1}the infimum of \({ A }\).\end{icloze}
\end{note}

\begin{note}{7230c3d5f7ef4b62bc1fd6c5b94841f0}
    \begin{icloze}{2}The greatest lower bound\end{icloze} of a set \({ A \subseteq \mathbf{R} }\) is denoted \begin{icloze}{1}\({ \inf A }\).\end{icloze}
\end{note}

\begin{note}{51abcbb89d7d486c9177cfc51b6e8721}
    Is it possible for a set \({ A \subseteq \mathbf{R} }\) for have multiple upper bounds?

    \begin{cloze}{1}
        Yes.
    \end{cloze}
\end{note}

\begin{note}{1c9d5ad3f35a47b0b12f27639fe4a409}
    Is it possible for a set \({ A \subseteq \mathbf{R} }\) for have multiple least upper bounds?

    \begin{cloze}{1}
        No.
    \end{cloze}
\end{note}

\begin{note}{8068979c7a6949fc9af88258008a9801}
    If \({ s_1 }\) and \({ s_2 }\) are both least upper bounds for a set \({ A \subseteq \mathbf{R} }\), then
    \begin{icloze}{1}
        \[
            s_1 = s_2.
        \]
    \end{icloze}
\end{note}

\begin{note}{466b264de27a44d3bd21221e39347d2e}
    What is the key idea in the proof of uniqueness of the least upper bound?

    \begin{cloze}{1}
        \({ s_1 \leq s_2 }\) and \({ s_2 \leq s_1 }\).
    \end{cloze}
\end{note}

\begin{note}{7100e899d7d44ffb89dbc0bac76ffb3f}
    Let \({ A \subseteq \mathbf{R} }\). \begin{icloze}{4}A real number \({ b }\)\end{icloze} is \begin{icloze}{3}a maximum of \({ A }\)\end{icloze} if \({ b }\) is \begin{icloze}{2}an element of \({ A }\)\end{icloze} and \begin{icloze}{1}an upper bound for \({ A }\).\end{icloze}
\end{note}

\begin{note}{5795e83831c14208a2d2b3dac0e2b139}
    Let \({ A \subseteq \mathbf{R} }\). A real number \({ b }\) is \begin{icloze}{3}a minimum of \({ A }\)\end{icloze} if \({ b }\) is \begin{icloze}{2}an element of \({ A }\)\end{icloze} and \begin{icloze}{1}a lower bound for \({ A }\).\end{icloze}
\end{note}

\begin{note}{2ea41e2869754b64bdb6c221308f0c58}
    Let \({ A \subseteq \mathbf{R} }\) and \begin{icloze}{3}\({ c \in \mathbf{R} }\).\end{icloze}
    Then \({ \begin{icloze}{2}c + A\end{icloze} \overset{\text{def}}= \begin{icloze}{1}\left\{ c + a : a \in A \right\}\end{icloze} }\).
\end{note}

\begin{note}{f7518efeac7b457d86040b99720ad110}
    Let \begin{icloze}{2}\({ A \subseteq \mathbf{R} }\) be nonempty and bounded above,\end{icloze} and let \begin{icloze}{4}\({ c \in \mathbf{R} }\).\end{icloze}
    Then
    \[
        \begin{icloze}{3}\sup(c + A)\end{icloze} = \begin{icloze}{1}c + \sup A.\end{icloze}
    \]
\end{note}

\begin{note}{726f73a8cead495fa65f331e49a892ea}
    Let \({ s \in \mathbf{R} }\) be \begin{icloze}{5}an upper bound\end{icloze} for a set \({ A \subseteq \mathbf{R} }\).
    Then \begin{icloze}{3}\({ s = \sup A }\)\end{icloze} \begin{icloze}{4}if and only if,\end{icloze} \begin{icloze}{2}for every \({ \epsilon > 0 }\),\end{icloze} \begin{icloze}{1}there exists an element \({ a }\) in \({ A }\) satisfying \({ s - \epsilon < a }\).\end{icloze}
\end{note}

\begin{note}{4161e1c933ba4349978c94d951259701}
    Let \({ s \in \mathbf{R} }\) be \begin{icloze}{5}a lower bound\end{icloze} for a set \({ A \subseteq \mathbf{R} }\).
    Then \begin{icloze}{3}\({ s = \inf A }\)\end{icloze} \begin{icloze}{4}if and only if,\end{icloze} \begin{icloze}{2}for every \({ \epsilon > 0 }\),\end{icloze} \begin{icloze}{1}there exists an element \({ a }\) in \({ A }\) satisfying \({ s + \epsilon > a }\).\end{icloze}
\end{note}

\begin{note}{0f8f37e55fbe4046a19926f2955f843f}
    Let \({ A \subseteq \mathbf{R} }\) be nonempty and bounded. How do \({ \inf A }\) and \({ \sup A }\) relate?

    \begin{cloze}{1}
        \[
            \inf A \leq \sup A.
        \]
    \end{cloze}
\end{note}

\begin{note}{882685715e2143a0b51a1e43390e1dbc}
    \begin{icloze}{1}Every nonempty set of real numbers\end{icloze} that is \begin{icloze}{2}bounded below\end{icloze} has \begin{icloze}{3}a greatest lower bound.\end{icloze}
\end{note}

\begin{note}{87f1451906164b06b7ffe3cd51a2ec7f}
    Every nonempty set of real numbers that is bounded below has a greatest lower bound.
    What is the key idea in the proof?

    \begin{cloze}{1}
        Infimum is the supremum for the set of lower bounds.
    \end{cloze}
\end{note}

\begin{note}{74b4cfb8b91d47b7afc1ae11a4b94ccb}
    Let \({ A_1, \ldots, A_n \subseteq \mathbf{R} }\) be nonempty and bounded above.
    Then
    \[
        \begin{icloze}{2}\sup\left( \bigcup_{k=1}^{n} A_k \right)\end{icloze} = \begin{icloze}{1}\max_{k} \sup A_k.\end{icloze}
    \]
\end{note}

\begin{note}{c4f28c7f86554b8d83da1931799f4181}
    Let \({ \:A_1, A_2, \ldots\: }\) be a collection of nonempty sets, each of which is bounded above.
    If \begin{icloze}{3}\({ \bigcup_{k=1}^{\infty} A_k }\) is bounded above,\end{icloze} then
    \[
        \begin{icloze}{2}\sup\left( \bigcup_{k=1}^{\infty} A_k \right)\end{icloze} = \begin{icloze}{1}\sup_{k} \sup A_k.\end{icloze}
    \]
\end{note}

\begin{note}{4c14ddc5fe394879915897bbb199442d}
    Let \({ A \subseteq \mathbf{R} }\) and \({ c \in \mathbf{R} }\).
    Then \({ \begin{icloze}{2}cA\end{icloze} \overset{\text{def}}= \begin{icloze}{1}\left\{ c \cdot a : a \in A \right\}\end{icloze} }\).
\end{note}

\begin{note}{8bdedbcb920f442787c9d475958a65dd}
    Let \({ A \subseteq \mathbf{R} }\) be nonempty and bounded above, and let \({ c \in \mathbf{R} }\).
    If \begin{icloze}{2}\({ c \geq 0 }\),\end{icloze} it follows that
    \[
        \sup (cA) = \begin{icloze}{1}c \cdot \sup A.\end{icloze}
    \]
\end{note}

\begin{note}{c96971d0b0eb40c39d1773c4f89a5588}
    Let \({ A \subseteq \mathbf{R} }\) be nonempty and bounded above, and let \({ c \in \mathbf{R} }\).
    If \begin{icloze}{2}\({ c < 0 }\),\end{icloze} it follows that
    \[
        \sup (cA) = \begin{icloze}{1}c \cdot \inf A.\end{icloze}
    \]
\end{note}

\begin{note}{fded05f0fad74578a073f5a838a3a081}
    Let \({ A, B \subseteq \mathbf{R} }\).
    Then \({ \begin{icloze}{2}A + B\end{icloze} \overset{\text{def}}= \begin{icloze}{1}\left\{ a + b : a \in A \: \text{and} \: b \in B \right\}\end{icloze} }\).
\end{note}

\begin{note}{12d0a51ec08c4d2094ce3e4c6c8b506a}
    Let \({ A, B \subseteq \mathbf{R} }\) be nonempty and bounded above.
    Then
    \[
        \begin{icloze}{2}\sup (A + B)\end{icloze} = \begin{icloze}{1}\sup A + \sup B.\end{icloze}
    \]
\end{note}

\begin{note}{75698bb156aa40799fc85b1e2419efa2}
    Let \({ A, B \subseteq \mathbf{R} }\) be nonempty and bounded above.
    Then
    \[
        \sup (A + B) = \underbrace{\sup A}_{s} + \underbrace{\sup B}_{t}.
    \]
    What is the key idea in the proof?

    \begin{cloze}{1}
        For \({ \epsilon > 0 }\), choose \({ a > s - \frac{\epsilon}{2} }\) and \({ b > t - \frac{\epsilon}{2} }\).
    \end{cloze}
\end{note}

\begin{note}{a6281cfff0a84b578d8cacdc6ea4779d}
    If \begin{icloze}{3}\({ a }\) is an upper bound for A\end{icloze} and \begin{icloze}{2}\({ a \in A }\),\end{icloze} then
    \begin{icloze}{1}
        \[
            a = \sup A.
        \]
    \end{icloze}
\end{note}

\begin{note}{eb0969a772e442dd8c3f57ed4f8ee1be}
    Let \({ A, B \subseteq \mathbf{R} }\) and \begin{icloze}{3}\({ \sup A < \sup B }\).\end{icloze} Then there exists \begin{icloze}{2}\({ b \in B }\)\end{icloze} that is \begin{icloze}{1}an upper bound for \({ A }\).\end{icloze}
\end{note}

\begin{note}{6b667686c9644d8b9849c735110dac20}
    If \({ A }\) and \({ B }\) are \begin{icloze}{3}nonempty, disjoint sets with \({ A \cup B = \mathbf{R} }\)\end{icloze} and \begin{icloze}{2}\({ a < b  }\) for all \({ a \in A }\) and \({ b \in B }\),\end{icloze} then \begin{icloze}{1}there exists \({ c \in \mathbf{R} }\) that is an upper bound for \({ A }\) and a lower bound for \({ B }\).\end{icloze}

    \begin{center}
        \tiny
        <<\begin{icloze}{4}Cut Property\end{icloze}>>
    \end{center}
\end{note}

\begin{note}{545cb11592164c31badc3f21a1e29981}
    What is the key idea in the proof of the Cut Property?

    \begin{cloze}{1}
        Use the Axiom of Completeness.
    \end{cloze}
\end{note}

\begin{note}{39aa54de461b426fbe225601c0663097}
    The Cut Property implies \begin{icloze}{1}the Axiom of Completeness.\end{icloze}
\end{note}

\begin{note}{3a64720500f14d66a66401dd3f133a10}
    The Cut Property implies the Axiom of Completeness.
    What is the key idea in the proof?

    \begin{cloze}{1}
        Consider the set of the upper bounds and its complement.
    \end{cloze}
\end{note}

\begin{note}{70244652872f4c1fb020d95cfaf88365}
    Let \({ A, B \subseteq \mathbf{R} }\) be nonempty, bounded above, and satisfy \({ A \subseteq B }\).
    How do \({ \sup A }\) and \({ \sup B }\) relate?

    \begin{cloze}{1}
        \[
            \sup A \leqslant \sup B.
        \]
    \end{cloze}
\end{note}

\begin{note}{12dc792f7f78436ea2156c1cc15355de}
    Let \({ A, B \subseteq \mathbf{R} }\) be nonempty and bounded, and let \begin{icloze}{2}\({ \sup A < \inf B }\).\end{icloze}
    Then \begin{icloze}{3}there exists a \({ c \in \mathbf{R} }\)\end{icloze} satisfying
    \begin{icloze}{1}
        \[
            a < c < b
        \]
        for all \({ a \in A }\) and \({ b \in B }\).
    \end{icloze}
\end{note}

\begin{note}{b21502823c8b4f59b97c454a58895487}
    Let \({ A, B \subseteq \mathbf{R} }\) be nonempty and bounded, and let \({ \sup A < \inf B }\).
    Then there exists a \({ c \in \mathbf{R} }\) satisfying \({ a < c < b }\) for all \({ a \in A }\) and \({ b \in B }\).
    What is the key idea in the proof?

    \begin{cloze}{1}
        Let \({ c = \frac{1}{2}(\sup A + \inf B) }\).
    \end{cloze}
\end{note}

\begin{note}{12e16f85a81743be8fd6073089decbea}
    Let \({ A, B \subseteq \mathbf{R} }\) be nonempty and bounded.
    If \begin{icloze}{3}there exists a \({ c \in \mathbf{R} }\)\end{icloze} satisfying \begin{icloze}{2}\({ a \leq c \leq b }\) for all \({ a \in A }\) and \({ b \in B }\),\end{icloze} then
    \begin{icloze}{1}
        \[
            \sup A \leq \inf B.
        \]
    \end{icloze}
\end{note}

\begin{note}{7bf4ff436d8047b382ad38a9fdbc1c88}
    Let \({ A, B \subseteq \mathbf{R} }\) be nonempty and bounded.
    If there exists a \({ c \in \mathbf{R} }\) satisfying \({ a \leq c \leq b }\) for all \({ a \in A }\) and \({ b \in B }\), then
    \[
        \sup A \leq \inf B.
    \]
    What is the key idea in the proof?

    \begin{cloze}{1}
        \({ c }\) is an upper bound for \({ A }\) and a lower bound for \({ B }\).
    \end{cloze}
\end{note}

\end{document}
