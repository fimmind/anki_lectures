\documentclass[11pt, a5paper]{article}
\usepackage[width=10cm, top=0.5cm, bottom=2cm]{geometry}

\usepackage[T1,T2A]{fontenc}
\usepackage[utf8]{inputenc}
\usepackage[english,russian]{babel}
\usepackage{libertine}

\usepackage{amsmath}
\usepackage{amssymb}
\usepackage{amsthm}
\usepackage{mathrsfs}
\usepackage{framed}
\usepackage{xcolor}

\setlength{\parindent}{0pt}

% Force \pagebreak for every section
\let\oldsection\section
\renewcommand\section{\pagebreak\oldsection}

\renewcommand{\thesection}{}
\renewcommand{\thesubsection}{Note \arabic{subsection}}
\renewcommand{\thesubsubsection}{}
\renewcommand{\theparagraph}{}

\newenvironment{note}[1]{\goodbreak\par\subsection{\hfill \color{lightgray}\tiny #1}}{}
\newenvironment{cloze}[2][\ldots]{\begin{leftbar}}{\end{leftbar}}
\newenvironment{icloze}[2][\ldots]{%
  \ignorespaces\text{\tiny \color{lightgray}\{\{c#2::}\hspace{0pt}%
}{%
  \hspace{0pt}\text{\tiny\color{lightgray}\}\}}\unskip%
}


\begin{document}
\section{Sets}
\begin{note}{097312afe75d4a3d9eaa0c1f4c63748a}
    Intuitively speaking, \begin{icloze}{2}a set\end{icloze} is \begin{icloze}{1}a collection of objects.\end{icloze}
\end{note}

\begin{note}{85e21cf985524b80a8c00eb4608f34be}
    Intuitively speaking, a set is a collection of objects.
    \begin{icloze}{2}Those objects\end{icloze} are referred to as \begin{icloze}{1}the elements of the set.\end{icloze}
\end{note}

\begin{note}{12b96daebbc04070b74e2a6f74e5b268}
    Given a set \({ A }\), we write \begin{icloze}{2}\({ x \in A }\)\end{icloze} if \begin{icloze}{1}\({ x }\) is an element of \({ A }\).\end{icloze}
\end{note}

\begin{note}{b25d749749a64c5b90880253d9839da8}
    Given  a set \({ A }\), we write \begin{icloze}{2}\({ x \not\in A }\)\end{icloze} if \begin{icloze}{1}\({ x }\) is not an element of \({ A }\).\end{icloze}
\end{note}

\begin{note}{39565306ec4e40e18136e7eb88fc817a}
    Given two sets \({ A }\) and \({ B }\), \begin{icloze}{1}the union\end{icloze} is written \begin{icloze}{2}\({ A \cup B }\).\end{icloze}
\end{note}

\begin{note}{73bf0eb1d16c4c5da368e326b4739d5b}
    Given two sets \({ A }\), and \({ B }\), \begin{icloze}{2}the union\end{icloze} is \begin{icloze}{3}defined\end{icloze} by the rule
    \begin{center}
        \({ x \in \begin{icloze}{2}A \cup B\end{icloze} }\) provided that \begin{icloze}{1}\({ x \in A }\) or \({ x \in B }\).\end{icloze}
    \end{center}
\end{note}

\begin{note}{8ce7db157931494bbfb6eee706e15efc}
    Given two sets \({ A }\) and \({ B }\), \begin{icloze}{1}the intersection\end{icloze} is written \begin{icloze}{2}\({ A \cap B }\).\end{icloze}
\end{note}

\begin{note}{6a277df52de2409a98e48429d69b6d05}
    Given two sets \({ A }\) and \({ B }\), \begin{icloze}{2}the intersection\end{icloze} is \begin{icloze}{3}defined\end{icloze} by the rule
    \begin{center}
        \({ x \in \begin{icloze}{2}A \cap B\end{icloze} }\) provided that \begin{icloze}{1}\({ x \in A }\) and \({ x \in B }\).\end{icloze}
    \end{center}
\end{note}

\begin{note}{684951afc378458aa7bd27e67cdc499b}
    \begin{icloze}{2}The set of natural numbers\end{icloze} is denoted \begin{icloze}{1}\({ \mathbf{N} }\).\end{icloze}
\end{note}

\begin{note}{49d36a026d4b4678ab86fb6103571cce}
    \[
        \begin{icloze}{2}\mathbf{N}\end{icloze} \overset{\text{def}}= \left\{ \begin{icloze}{1}1, 2, 3, \ldots\end{icloze} \right\}.
    \]
\end{note}

\begin{note}{797c81e5adb543e1a5d4cc67e64c5e09}
    \begin{icloze}{2}The set of integers\end{icloze} is denoted \begin{icloze}{1}\({ \mathbf{Z} }\).\end{icloze}
\end{note}

\begin{note}{d3c61bf891744c58b73cef543c6e100d}
    \[
        \begin{icloze}{2}\mathbf{Z}\end{icloze} \overset{\text{def}}= \left\{ \begin{icloze}{1}\ldots, -2, -1, 0, 1, 2, \ldots\end{icloze} \right\}.
    \]
\end{note}

\begin{note}{57f085776972449f8bc14daf5cff6603}
    \begin{icloze}{2}The set of rational numbers\end{icloze} is denoted \begin{icloze}{1}\({ \mathbf{Q} }\).\end{icloze}
\end{note}

\begin{note}{f7e3370650134607853b41b2b1ecf54b}
    \[
        \begin{icloze}{3}\mathbf{Q}\end{icloze} \overset{\text{def}}= \left\{ \text{all \begin{icloze}{2}fractions \({ \frac{p}{q} }\)\end{icloze} where \begin{icloze}{1}\({ p, q \in \mathbf{Z} }\) and \({ q \neq 0 }\)\end{icloze}} \right\}.
    \]
\end{note}

\begin{note}{faeac83cb5b740b6964551c85ad3e35b}
    \begin{icloze}{2}The set of real numbers\end{icloze} is denoted \begin{icloze}{1}\({ \mathbf{R} }\).\end{icloze}
\end{note}

\begin{note}{6e5da98964d645d09ad6989e85679c74}
    \begin{icloze}{2}The empty\end{icloze} set is \begin{icloze}{1}the set that contains no elements.\end{icloze}
\end{note}

\begin{note}{206db0a0f3d042e49a9ca532e222201f}
    \begin{icloze}{2}The empty set\end{icloze} is denoted \begin{icloze}{1}\({ \emptyset }\).\end{icloze}
\end{note}

\begin{note}{2f0448d226db4b71b150acaed349a73b}
    Two sets \({ A }\) and \({ B }\) are said to be \begin{icloze}{2}disjoint\end{icloze} if \begin{icloze}{1}\({ A \cap B = \emptyset }\).\end{icloze}
\end{note}

\begin{note}{e5d9d365e86640319ca5460ef8c4f05c}
    Given two sets \({ A }\) and \({ B }\), we say \begin{icloze}{2}\({ A }\) is a subset of \({ B }\),\end{icloze} or \begin{icloze}{2}\({ B }\) contains \({ A }\)\end{icloze} if \begin{icloze}{1}every element of \({ A }\) is also an element of \({ B }\).\end{icloze}
\end{note}

\begin{note}{c2bd27f1fc0d40e296dceef9c9789556}
    Given two sets \({ A }\) and \({ B }\), the \begin{icloze}{3}inclusion\end{icloze} relationship \begin{icloze}{2}\({ A \subseteq B }\) or \({ B \supseteq A }\)\end{icloze} is used to indicate that \begin{icloze}{1}\({ A }\) is a subset of \({ B }\).\end{icloze}
\end{note}

\begin{note}{333e7c6716af48b7b9962ad803f0732f}
    Given two sets \({ A }\) and \({ B }\), \begin{icloze}{2}\({ A = B }\)\end{icloze} means that \begin{icloze}{1}\({ A \subseteq B }\) and \({ B \subseteq A }\).\end{icloze}
\end{note}

\begin{note}{74e93b42d46746dc9ec2b54f8366c435}
    Let \({ A_1, A_2, A_3, \ldots }\) be an infinite collection of sets.
    Notationally,
    \[
        \bigcup_{n = 1}^{\infty} A_n, \quad \bigcup_{n \in \mathbf{N}}^{} A_n, \quad \text{or} \quad A_1 \cup A_2 \cup A_3 \cup \cdots
    \]
    are all equivalent ways to indicate \begin{icloze}{1}the set whose elements consist of any element that appears in at least on particular \({ A_n }\).\end{icloze}
\end{note}

\begin{note}{69e4627a3e7149ef8be05479a2587b41}
    Let \({ A_1, A_2, A_3, \ldots }\) be an infinite collection of sets.
    Notationally,
    \[
        \bigcap_{n = 1}^{\infty} A_n, \quad \bigcap_{n \in \mathbf{N}}^{} A_n, \quad \text{or} \quad A_1 \cap A_2 \cap A_3 \cap \cdots
    \]
    are all equivalent ways to indicate \begin{icloze}{1}the set whose elements consist of any element that appears in every \({ A_n }\).\end{icloze}
\end{note}

\begin{note}{11a987e10fce4ceea69672f366597729}
    Given \({ A \subseteq \mathbf{R} }\), \begin{icloze}{2}the complement of \({ A }\)\end{icloze} refers to \begin{icloze}{1}the set of all elements of \({ \mathbf{R} }\) not in \({ A }\).\end{icloze}
\end{note}

\begin{note}{8b379552450b4672af82c17476c0ff13}
    Given \({ A \subseteq \mathbf{R} }\), \begin{icloze}{2}the complement of \({ A }\)\end{icloze} is written \begin{icloze}{1}\({ A^{c} }\).\end{icloze}
\end{note}

\begin{note}{a3459afa53264a7c82d9abd760a0c93e}
    Given \({ A, B \subseteq \mathbf{R} }\),
    \[
        \begin{icloze}{2}(A \cap B)^{c}\end{icloze} = \begin{icloze}{1}A^{c} \cup B^{c}.\end{icloze}
    \]

    \begin{center}
        \tiny
        <<\begin{icloze}{3}De Morgan's Law\end{icloze}>>
    \end{center}
\end{note}

\begin{note}{c983927aa0304e51949e2f90a2ec2614}
    Given \({ A, B \subseteq \mathbf{R} }\),
    \[
        \begin{icloze}{2}(A \cup B)^{c}\end{icloze} = \begin{icloze}{1}A^{c} \cap B^{c}.\end{icloze}
    \]

    \begin{center}
        \tiny
        <<\begin{icloze}{3}De Morgan's Law\end{icloze}>>
    \end{center}
\end{note}

\begin{note}{09322548137b46529467f2946a4952d4}
    What is the key idea in the proof of De Morgan's Laws?

    \begin{cloze}{1}
        Demonstrate inclusion both ways.
    \end{cloze}
\end{note}

\section{Functions}
\begin{note}{18930cfe4e4445779bcec8a2fb53f23c}
    Given \begin{icloze}{3}two sets \({ A }\) and \({ B }\),\end{icloze} \begin{icloze}{2}a function from \({ A }\) to \({ B }\)\end{icloze} is \begin{icloze}{1}a rule or mapping that takes each element \({ x \in A }\) and associates with it a single element of \({ B }\).\end{icloze}
\end{note}

\begin{note}{dfa898ef047e418fa8dfe9ee9582fd71}
    \begin{icloze}{1}If \({ f }\) is a function from \({ A }\) to \({ B }\),\end{icloze} we write \begin{icloze}{2}\({ f : A \to B }\).\end{icloze}
\end{note}

\begin{note}{c2730dafa0fe4bf4bede66b7199b48b9}
    Let \({ f : A \to B }\).
    Given \begin{icloze}{3}\({ x \in A }\),\end{icloze} the expression \begin{icloze}{2}\({ f(x) }\)\end{icloze} is used to represent \begin{icloze}{1}the element of \({ B }\) associated with \({ x }\) by \({ f }\).\end{icloze}
\end{note}

\begin{note}{65568f366ca949888310668475dbe570}
    Let \({ f : A \to B }\).
    \begin{icloze}{2}The set \({ A }\)\end{icloze} is called \begin{icloze}{1}the domain of \({ f }\).\end{icloze}
\end{note}

\begin{note}{7870a310786142fa938bcc843ca8e1ae}
    Let \({ f : A \to B }\).
    \begin{icloze}{2}The set \({ \left\{ f(x) \mid x \in A \right\} }\)\end{icloze} is called \begin{icloze}{1}the range of \({ f }\).\end{icloze}
\end{note}

\begin{note}{716c208c9ae849b89ec722aa17f20882}
    Given a function \({ f }\) and \begin{icloze}{3}a subset \({ A }\) of its domain,\end{icloze}
    \begin{icloze}{2}
        the set
        \[
            \left\{ f(x) : x \in A \right\}
        \]
    \end{icloze}
    is called \begin{icloze}{1}the range of \({ f }\) over the set \({ A }\).\end{icloze}
\end{note}

\begin{note}{24aae21652754fcda1267ac61036a3ea}
    Given a function \({ f }\) and a subset \({ A }\) of its domain, \begin{icloze}{2}the range of \({ f }\) over \({ A }\)\end{icloze} is written \begin{icloze}{2}\({ f(A) }\).\end{icloze}
\end{note}

\begin{note}{6ed2fb1006634dcf81707a3c4d514857}
    Let \({ f : D \to \mathbf{R} }\),\: \({ A, B \subseteq D }\).
    Is it unconditionally true that
    \[
        f(A \cup B) = f(A) \cup f(B)?
    \]

    \begin{cloze}{1}
        Yes.
    \end{cloze}
\end{note}

\begin{note}{ee665e77ac9a45cf9a15d42549e6f382}
    Let \({ f : D \to \mathbf{R} }\),\: \({ A, B \subseteq D }\).
    Is it unconditionally true that
    \[
        f(A \cap B) = f(A) \cap f(B)?
    \]

    \begin{cloze}{1}
        No.
    \end{cloze}
\end{note}

\begin{note}{5d2e9d4e1e094e06b37bd87e2c9edff8}
    Given \begin{icloze}{4}\({ a, b \in \mathbf{R} }\)\end{icloze} and \begin{icloze}{3}\({ a \leq b }\)\end{icloze},
    \begin{icloze}{2}
        the set
        \[
            \left\{ x \in \mathbf{R} : a \leq x \leq b \right\}
        \]
    \end{icloze}
    is called \begin{icloze}{1}a closed interval.\end{icloze}
\end{note}

\begin{note}{9f383a22fc724f8fa43af5cb65e0cd5a}
    Given \({ a, b \in \mathbf{R} }\) and \begin{icloze}{3}\({ a < b }\)\end{icloze},
    \begin{icloze}{2}
        the set
        \[
            \left\{ x \in \mathbf{R} : a < x < b \right\}
        \]
    \end{icloze}
    is called \begin{icloze}{1}an open interval.\end{icloze}
\end{note}

\begin{note}{3143096eb895471bac4b2d5840d18758}
    Given \({ a, b \in \mathbf{R} }\) and \({ a \leq b }\),
    \begin{icloze}{1}
        the closed interval
        \[
            \left\{ x \in \mathbf{R} : a \leq x \leq b \right\}
        \]
    \end{icloze}
    is written \begin{icloze}{2}\({ [a, b] }\).\end{icloze}
\end{note}

\begin{note}{604897f024bd4de78723fe8247290371}
    Given \({ a, b \in \mathbf{R} }\) and \({ a \leq b }\),
    \begin{icloze}{1}
        the open interval
        \[
            \left\{ x \in \mathbf{R} : a < x < b \right\}
        \]
    \end{icloze}
    is written \begin{icloze}{2}\({ (a, b) }\).\end{icloze}
\end{note}

\begin{note}{a77dc72d26be45c185900ba7ff132b05}
    Let \({ f(x) = x^2 }\). Find two sets \({ A }\) and \({ B }\) for which
    \[
        f(A \cap B) \neq f(A) \cap f(B).
    \]

    \begin{cloze}{1}
        \({ [-1, 0] }\) and \({ [0, 1] }\).
    \end{cloze}
\end{note}

\begin{note}{6ed2fb1006634dcf81707a3c4d514857}
    Let \({ f : D \to \mathbf{R} }\),\: \({ A, B \subseteq D }\).
    Then
    \[
        \begin{icloze}{3}f(A \cup B)\end{icloze} \begin{icloze}{1}=\end{icloze} \begin{icloze}{2}f(A) \cup f(B).\end{icloze}
    \]
\end{note}

\begin{note}{e088ae5ae1f24425a81dac09317978fd}
    Let \({ f : D \to \mathbf{R} }\),\: \({ A, B \subseteq D }\).
    Then
    \[
        \begin{icloze}{3}f(A \cap B)\end{icloze} \begin{icloze}{1}\subseteq\end{icloze} \begin{icloze}{2}f(A) \cap f(B).\end{icloze}
    \]
\end{note}

\begin{note}{f951f5a5136248dcb413f59b3271d389}
    Given \({ x \in \mathbf{R} }\), \begin{icloze}{2}the absolute value of \({ x }\)\end{icloze} is denoted \begin{icloze}{1}\({ \left\lvert x \right\rvert }\).\end{icloze}
\end{note}

\begin{note}{624dda908fd64a1cadae2b61c1277c59}
    Given \({ x \in \mathbf{R} }\),
    \[
        \left\lvert x \right\rvert \overset{\text{def}}= \begin{cases}
            \begin{icloze}{1}x,\end{icloze} & \text{if \begin{icloze}{2}\({ x \geq 0 }\)\end{icloze}}, \\
            \begin{icloze}{1}-x,\end{icloze} & \text{if \begin{icloze}{2}\({ x < 0 }\)\end{icloze}}.
        \end{cases}
    \]
\end{note}

\begin{note}{0ab23d0afe1448e397cad330aea55883}
    Given \({ a, b \in \mathbf{R} }\), \quad \({ \left\lvert ab \right\rvert = \begin{icloze}{1}\left\lvert a \right\rvert \cdot \left\lvert b \right\rvert\end{icloze} }\).
\end{note}

\begin{note}{2b51f36fba524365b72001d318791436}
    Given \({ a, b \in \mathbf{R} }\), \quad \({ \begin{icloze}{2}\left\lvert a + b \right\rvert\end{icloze} \begin{icloze}{3}\leq\end{icloze} \begin{icloze}{1}\left\lvert a \right\rvert + \left\lvert b \right\rvert\end{icloze} }\).

    \begin{center}
        \tiny
        <<\begin{icloze}{4}Triangle inequality\end{icloze}>>
    \end{center}
\end{note}

\begin{note}{4d6e77677e884f9c8ee877b9a32d48b5}
    Let \({ f : A \to B }\). The function \({ f }\) is \begin{icloze}{2}one-to-one\end{icloze} if
    \begin{icloze}{1}
        \begin{center}
            \({ a_1 \neq a_2 }\) in \({ A }\) implies that \({ f(a_1) \neq f(a_2) }\) in \({ B }\).
        \end{center}
    \end{icloze}
\end{note}

\begin{note}{56b2bf81daaf419ab1207c6693c981e6}
    Let \({ f : A \to B }\). The function \({ f }\) is \begin{icloze}{2}onto\end{icloze} if
    \begin{icloze}{1}
        \begin{center}
            the range of \({ f }\) equals \({ B }\).
        \end{center}
    \end{icloze}
\end{note}

\begin{note}{ccc8a358284a4b1f99f8e4336a2efdb9}
    Let \begin{icloze}{4}\({ f : D \to \mathbf{R} }\)\end{icloze} and \begin{icloze}{3}\({ B \subseteq \mathbf{R} }\).\end{icloze}
    \begin{icloze}{2}
        The set
        \[
            \left\{ x \in D : f(x) = B \right\}
        \]
    \end{icloze}
    is called \begin{icloze}{1}the preimage of \({ B }\) under the function \({ f }\).\end{icloze}
\end{note}

\begin{note}{b72f131ae6734bf694fd8f987bb2323d}
    Let \({ f : D \to \mathbf{R} }\) and \({ A, B \subseteq \mathbf{R} }\).
    Is it unconditionally true that
    \[
        f^{-1}(A \cup B) = f^{-1}(A) \cup f^{-1}(B)?
    \]

    \begin{cloze}{1}
        Yes.
    \end{cloze}
\end{note}

\begin{note}{5b3116f568a34fe2be32f403d7d081d9}
    Let \({ f : D \to \mathbf{R} }\) and \({ A, B \subseteq \mathbf{R} }\).
    Is it unconditionally true that
    \[
        f^{-1}(A \cap B) = f^{-1}(A) \cap f^{-1}(B)?
    \]

    \begin{cloze}{1}
        Yes.
    \end{cloze}
\end{note}

\section{Logic and Proofs}
\begin{note}{a4d52b740f5b494696a5bdc956906cf2}
    % source: https://en.wikipedia.org/wiki/Theorem
    Many mathematical theorems are conditional statements, whose proofs deduce conclusions from conditions. Given such a theorem, \begin{icloze}{1}those conditions\end{icloze} are known \begin{icloze}{2}as the theorem's hypotheses.\end{icloze}
\end{note}

\begin{note}{93f759e32dbf497cb30754e24c5b09f1}
    When in \begin{icloze}{3}a proof by contradiction\end{icloze} \begin{icloze}{2}the contradiction is with the theorem's hypothesis,\end{icloze} the proof is said to be \begin{icloze}{1}contrapositive.\end{icloze}
\end{note}

\begin{note}{1f45350926704df98b0abdf205f4319c}
    Two real number \({ a }\) and \({ b }\) are \begin{icloze}{4}equal\end{icloze} \begin{icloze}{3}if and only if\end{icloze} \begin{icloze}{2}for every real number \({ \epsilon > 0 }\) it follows that\end{icloze} \begin{icloze}{1}\({ \left\lvert a - b \right\rvert < \epsilon }\).\end{icloze}
\end{note}

\begin{note}{3ef90c9123e64df39ae9cd34271a7dcd}
    Two real number \({ a }\) and \({ b }\) are equal \({ \impliedby }\) for every real number \({ \epsilon > 0 }\) it follows that \({ \left\lvert a - b \right\rvert < \epsilon }\).
    What is the key idea in the proof?

    \begin{cloze}{1}
        By contradiction.
    \end{cloze}
\end{note}

\begin{note}{aab4bb967d814e87bd85608277093755}
    Let \begin{icloze}{3}\({ S \subseteq \mathbf{N} }\).\end{icloze}
    If \begin{icloze}{2}\({ S }\) contains \({ 1 }\)\end{icloze} and \begin{icloze}{2}whenever \({ S }\) contains \({ n }\), it also contains \({ n + 1 }\),\end{icloze} then \begin{icloze}{1}\({ S = \mathbf{N} }\).\end{icloze}
\end{note}

\begin{note}{3dd92625856f408b9dc93fd36d82588d}
    Let \({ S \subseteq \mathbf{N} }\).
    If \({ S }\) contains \({ 1 }\) and whenever \({ S }\) contains \({ n }\), it also contains \({ n + 1 }\), then \({ S = \mathbf{N} }\).
    This proposition is the fundamental principle behind \begin{icloze}{1}induction.\end{icloze}
\end{note}

\begin{note}{40977a19a0d043c985df5676daa9f776}
    Does an induction argument imply the validity of the infinite case?

    \begin{cloze}{1}
        No, it doesn't.
    \end{cloze}
\end{note}

\begin{note}{91b673c484b442ec92dd47ad0ef95f6c}
    Do De Morgan's rules hold for an infinite collection of sets?

    \begin{cloze}{1}
        Yes, they do.
    \end{cloze}
\end{note}

\begin{note}{df9aa3b9e0c74da78d7e2a0a65276fcd}
    How De Morgan's rules for an infinite collection of sets defer from that for a finite collection?

    \begin{cloze}{1}
        They are essentially the same.
    \end{cloze}
\end{note}

\end{document}
