%! TeX root = ./main.tex
\documentclass[11pt, a5paper]{article}
\usepackage[width=10cm, top=0.5cm, bottom=2cm]{geometry}

\usepackage[T1,T2A]{fontenc}
\usepackage[utf8]{inputenc}
\usepackage[english,russian]{babel}
\usepackage{libertine}

\usepackage{amsmath}
\usepackage{amssymb}
\usepackage{amsthm}
\usepackage{mathrsfs}
\usepackage{framed}
\usepackage{xcolor}

\setlength{\parindent}{0pt}

% Force \pagebreak for every section
\let\oldsection\section
\renewcommand\section{\pagebreak\oldsection}

\renewcommand{\thesection}{}
\renewcommand{\thesubsection}{Note \arabic{subsection}}
\renewcommand{\thesubsubsection}{}
\renewcommand{\theparagraph}{}

\newenvironment{note}[1]{\goodbreak\par\subsection{\hfill \color{lightgray}\tiny #1}}{}
\newenvironment{cloze}[2][\ldots]{\begin{leftbar}}{\end{leftbar}}
\newenvironment{icloze}[2][\ldots]{%
  \ignorespaces\text{\tiny \color{lightgray}\{\{c#2::}\hspace{0pt}%
}{%
  \hspace{0pt}\text{\tiny\color{lightgray}\}\}}\unskip%
}


\begin{document}
\section{Uniform Convergence of a Sequence of Functions} % 6.2
\begin{note}{1bf1a79b9eba47cf852e1a9c7468c5f7}
    Let \({ (f_n) }\) be \begin{icloze}{1}a sequence of function on a set \({ A }\).\end{icloze}
    We say \begin{icloze}{3}\({ \left( f_n \right) }\) converges pointwise on \({ A }\) to a function \({ f }\)\end{icloze} if
    \begin{icloze}{2}
        for all \({ x \in A }\)
        \[
            \Big(f_n(x)\Big) \underset{n \to \infty}{\longrightarrow} f(x).
        \]
    \end{icloze}
\end{note}

\begin{note}{f11dc20a5619424cafc97ab1b4d64b5f}
    Let \({ (f_n) }\) be a sequence of function on a set \({ A }\).
    If \begin{icloze}{2}\({ (f_n) }\) converges pointwise on \({ A }\) to \({ f }\),\end{icloze} we write
    \[
        \begin{icloze}{1}f_n \to f\end{icloze} \quad \text{or} \quad \begin{icloze}{1}\lim_{n \to \infty} f_n = f.\end{icloze}
    \]
\end{note}

\begin{note}{6f3f051b9e0741dcbd85037d47c4fd19}
    Let \({ f_n(x) = \frac{x^2 + nx}{n} }\).
    \[
        \lim_{n \to \infty} f_n(x) = \begin{icloze}{1}x.\end{icloze}
    \]
\end{note}

\begin{note}{3c7731c6b70a4c28972a5ea2e88a1e5f}
    Let \({ f_n(x) = x^{n} }\),\: \({ f_n : [0, 1] \to \mathbb R }\).
    \[
        \lim_{n \to \infty} f_n(x) =
        \begin{icloze}{1}
            \begin{cases}
                0 & \text{for } 0 \leq x < 1, \\
                1 & \text{for } x = 1.
            \end{cases}
        \end{icloze}
    \]
\end{note}

\begin{note}{7218c9c8b0f04d4887dc2345da75c6c6}
    Let \({ (f_n) }\) be a sequence of function on a set \({ A }\).
    We say \begin{icloze}{2}\({ (f_n) }\) converges uniformly on \({ A }\) to a function \({ f }\)\end{icloze} if
    \begin{icloze}{1}
        \[
            \begin{gathered}
                \forall \epsilon > 0  \quad \exists N \in \mathbf{N} \quad \forall n \geq N \\
                \left\lvert f_n - f \right\rvert < \epsilon.
            \end{gathered}
        \]
    \end{icloze}
\end{note}

% Lecture 26.10.22 {{{
\begin{note}{c80f9e6c9feb486fb69c66c740b4fa7b}
    Let \({ (f_n) }\) be a sequence of function on a set \({ A }\).
    If \begin{icloze}{2}\({ (f_n) }\) converges uniformly on \({ A }\) to \({ f }\),\end{icloze} we write
    \begin{icloze}{1}
        \[
            f_n \rightrightarrows f.
        \]
    \end{icloze}
\end{note}
% }}}

\begin{note}{77ef924775b2453cb303b726f3081917}
    What is the key distinction between the definitions of pointwise and uniform convergences of a sequence of functions?

    \begin{cloze}{1}
        The dependence of \({ N }\) on \({ x }\).
    \end{cloze}
\end{note}

\begin{note}{42d2e1017eac4382878c195aa5a4c54d}
    What is the visual behind the uniform convergence of a sequence of functions?

    \begin{cloze}{1}
        Eventually every \({ f_n }\) is completely contained in the \({ \epsilon }\)-strip.
    \end{cloze}
\end{note}

\begin{note}{0c853e2f4ed04acf9dae0b00c1a751f3}
    Which is stronger, uniform or pointwise convergence?

    \begin{cloze}{1}
        Uniform convergence is stronger.
    \end{cloze}
\end{note}

\begin{note}{ed7804cf8d4d48d5b0efb426d130fb52}
    Uniform convergence implies \begin{icloze}{1}pointwise convergence.\end{icloze}
\end{note}

\begin{note}{c9b4c187b4d54a78a9500289aa5899d0}
    Let \({ (f_n) }\) be a sequence of function on a set \({ A }\).
    \[
        \begin{icloze}{2}f_n \rightrightarrows f\end{icloze}
        \quad \begin{icloze}{3}\iff\end{icloze} \quad
        \begin{icloze}{1}\sup |f_n - f| \underset{n \to \infty}{\longrightarrow} 0.\end{icloze}
    \]

    \begin{center}
        \tiny
        (in terms of \({ \sup }\))
    \end{center}
\end{note}

\begin{note}{1b59f18d7ccb47829cf7b7ea7576318c}
    Let \({ (f_n) }\) be a sequence of function on a set \({ A }\).
    \begin{icloze}{3}Then \({ f_n \rightrightarrows f }\)\end{icloze} \begin{icloze}{4}if and only if\end{icloze}
    \[
        \begin{gathered}
            \begin{icloze}{1}\forall \varepsilon > 0 \quad \exists N \in \mathbf{N} \quad \forall m, n \geq N\end{icloze}
            \\
            \begin{icloze}{2}\left\lvert f_n - f_m \right\rvert < \varepsilon.\end{icloze}
        \end{gathered}
    \]
\end{note}

\begin{note}{2b9e4671775a43e9aa4a6b4d581b1658}
    Let \({ (f_n) }\) be a sequence of function on a set \({ A }\).
    Then \({ f_n \rightrightarrows f }\) if and only if
    \[
        \begin{gathered}
            \forall \varepsilon > 0 \quad \exists N \in \mathbf{N} \quad \forall m, n \geq N \\
            \left\lvert f_n - f_m \right\rvert < \varepsilon.
        \end{gathered}
    \]

    \begin{center}
        \tiny
        <<\begin{icloze}{1}Cauchy Criterion\end{icloze}>>
    \end{center}
\end{note}

\begin{note}{3fa98b94397f4cc2b2d766dd41934f67}
    What is the key idea in the proof of necessity of the Cauchy Criterion for uniform convergence?

    \begin{cloze}{1}
        Follows immediately from the definition.
    \end{cloze}
\end{note}

\begin{note}{f2d15c9af98f4b82956e48ed7df71fc9}
    What is the key idea in the proof of sufficiency of the Cauchy Criterion for uniform convergence?

    \begin{cloze}{1}
        Define a candidate for the limit and prove by definition.
    \end{cloze}
\end{note}

\begin{note}{1525b27207e74da186a95d7656e895da}
    In the proof of sufficiency of the Cauchy Criterion for uniform convergence, how do you define a candidate for the limit?

    \begin{cloze}{1}
        Use the pointwise limit.
    \end{cloze}
\end{note}

\begin{note}{0177dd65112f46c799884e104b39ef76}
    In the proof of sufficiency of the Cauchy Criterion for uniform convergence, how do you do we know the pointwise limit exists?

    \begin{cloze}{1}
        Due to the Cauchy's Criterion for sequences.
    \end{cloze}
\end{note}

\begin{note}{a1ccae80d31b4f38a4fc876e1ffe4ae7}
    In the proof of sufficiency of the Cauchy Criterion for uniform convergence, we have \({ f_n \to f }\).
    How do you show that \({ f_n \rightrightarrows f }\)?

    \begin{cloze}{1}
        Take the limit of the inequality from the Cauchy Criterion.
    \end{cloze}
\end{note}

\begin{note}{baab958475694fc08316e2031a57fa58}
    Let \({ f_n \to f }\) on a set \({ A }\) and \({ c \in A }\).
    If \begin{icloze}{3}the convergence is uniform\end{icloze} and \begin{icloze}{2}all \({ f_n }\) are continuous at \({ c }\),\end{icloze} then \begin{icloze}{1}\({ f }\) is continuous at \({ c }\).\end{icloze}
\end{note}

\begin{note}{a026cf3ddb2f4d5b9a94b36b2bc20ef9}
    Let \({ f_n \to f }\) on a set \({ A }\) and \({ c \in A }\).
    If the convergence is uniform and all \({ f_n }\) are continuous at \({ c }\), then \({ f }\) is continuous at \({ c }\).

    \begin{center}
        \tiny
        <<\begin{icloze}{1}Continuous Limit Theorem\end{icloze}>>
    \end{center}
\end{note}

\begin{note}{5fd08fca82504ff0af82d320da351ff7}
    What is the key idea in the proof of the Continuous Limit Theorem for a series of functions?

    \begin{cloze}{1}
        Triple triangle inequality after adding and subtracting \({ f_N }\).
    \end{cloze}
\end{note}

% Lecture 26.10.22 {{{
\begin{note}{06425162bee447479d3a4f5c71c9cf2a}
    Let \({ f_n \to f }\) on a set \({ A }\) and \({ c \in A }\).
    If \begin{icloze}{1}the convergence is uniform and all \({ f_n }\) are continuous at \({ c }\),\end{icloze} then
    \[
        \lim_{x \to c} \lim_{n \to \infty} f_n(x) = \begin{icloze}{2}\lim_{n \to \infty} \lim_{x \to c} f_n(x).\end{icloze}
    \]
\end{note}
% }}}

\begin{note}{05371b5a401f4756bc04fc154476e2c4}
    Let \({ f_n \to f }\) on a set \({ A }\).
    If each \({ f_n }\) is continuous, but \({ f }\) is discontinuous, then \begin{icloze}{1}the convergence is not uniform.\end{icloze}
\end{note}

\end{document}
