%! TeX root = ./main.tex
\documentclass[11pt, a5paper]{article}
\usepackage[width=10cm, top=0.5cm, bottom=2cm]{geometry}

\usepackage[T1,T2A]{fontenc}
\usepackage[utf8]{inputenc}
\usepackage[english,russian]{babel}
\usepackage{libertine}

\usepackage{amsmath}
\usepackage{amssymb}
\usepackage{amsthm}
\usepackage{mathrsfs}
\usepackage{framed}
\usepackage{xcolor}

\setlength{\parindent}{0pt}

% Force \pagebreak for every section
\let\oldsection\section
\renewcommand\section{\pagebreak\oldsection}

\renewcommand{\thesection}{}
\renewcommand{\thesubsection}{Note \arabic{subsection}}
\renewcommand{\thesubsubsection}{}
\renewcommand{\theparagraph}{}

\newenvironment{note}[1]{\goodbreak\par\subsection{\hfill \color{lightgray}\tiny #1}}{}
\newenvironment{cloze}[2][\ldots]{\begin{leftbar}}{\end{leftbar}}
\newenvironment{icloze}[2][\ldots]{%
  \ignorespaces\text{\tiny \color{lightgray}\{\{c#2::}\hspace{0pt}%
}{%
  \hspace{0pt}\text{\tiny\color{lightgray}\}\}}\unskip%
}


\begin{document}
\section{Uniform Convergence of a Sequence of Functions} % 6.2
\begin{note}{1bf1a79b9eba47cf852e1a9c7468c5f7}
    Let \({ (f_n) }\) be \begin{icloze}{1}a sequence of function on a set \({ A }\).\end{icloze}
    We say \begin{icloze}{3}\({ \left( f_n \right) }\) converges pointwise on \({ A }\) to a function \({ f }\)\end{icloze} if
    \begin{icloze}{2}
        for all \({ x \in A }\)
        \[
            \Big(f_n(x)\Big) \underset{n \to \infty}{\longrightarrow} f(x).
        \]
    \end{icloze}
\end{note}

\begin{note}{f11dc20a5619424cafc97ab1b4d64b5f}
    Let \({ (f_n) }\) be a sequence of function on a set \({ A }\).
    If \begin{icloze}{2}\({ (f_n) }\) converges pointwise on \({ A }\) to \({ f }\),\end{icloze} we write
    \[
        \begin{icloze}{1}f_n \to f\end{icloze} \quad \text{or} \quad \begin{icloze}{1}\lim_{n \to \infty} f_n = f.\end{icloze}
    \]
\end{note}

\begin{note}{6f3f051b9e0741dcbd85037d47c4fd19}
    Let \({ f_n(x) = \frac{x^2 + nx}{n} }\).
    \[
        \lim_{n \to \infty} f_n(x) = \begin{icloze}{1}x.\end{icloze}
    \]
\end{note}

\begin{note}{3c7731c6b70a4c28972a5ea2e88a1e5f}
    Let \({ f_n(x) = x^{n} }\),\: \({ f_n : [0, 1] \to \mathbb R }\).
    \[
        \lim_{n \to \infty} f_n(x) =
        \begin{icloze}{1}
            \begin{cases}
                0 & \text{for } 0 \leq x < 1, \\
                1 & \text{for } x = 1.
            \end{cases}
        \end{icloze}
    \]
\end{note}

\begin{note}{7218c9c8b0f04d4887dc2345da75c6c6}
    Let \({ (f_n) }\) be a sequence of function on a set \({ A }\).
    We say \begin{icloze}{2}\({ (f_n) }\) converges uniformly on \({ A }\) to a function \({ f }\)\end{icloze} if
    \begin{icloze}{1}
        \[
            \begin{gathered}
                \forall \epsilon > 0  \quad \exists N \in \mathbf{N} \quad \forall n \geq N \\
                \left\lvert f_n - f \right\rvert < \epsilon.
            \end{gathered}
        \]
    \end{icloze}
\end{note}

% Lecture 26.10.22 {{{
\begin{note}{c80f9e6c9feb486fb69c66c740b4fa7b}
    Let \({ (f_n) }\) be a sequence of function on a set \({ A }\).
    If \begin{icloze}{2}\({ (f_n) }\) converges uniformly on \({ A }\) to \({ f }\),\end{icloze} we write
    \begin{icloze}{1}
        \[
            f_n \rightrightarrows f.
        \]
    \end{icloze}
\end{note}
% }}}

\begin{note}{77ef924775b2453cb303b726f3081917}
    What is the key distinction between the definitions of pointwise and uniform convergences of a sequence of functions?

    \begin{cloze}{1}
        The dependence of \({ N }\) on \({ x }\).
    \end{cloze}
\end{note}

\begin{note}{42d2e1017eac4382878c195aa5a4c54d}
    What is the visual behind the uniform convergence of a sequence of functions?

    \begin{cloze}{1}
        Eventually every \({ f_n }\) is completely contained in the \({ \epsilon }\)-strip.
    \end{cloze}
\end{note}

\begin{note}{0c853e2f4ed04acf9dae0b00c1a751f3}
    Which is stronger, uniform or pointwise convergence?

    \begin{cloze}{1}
        Uniform convergence is stronger.
    \end{cloze}
\end{note}

\begin{note}{ed7804cf8d4d48d5b0efb426d130fb52}
    Uniform convergence implies \begin{icloze}{1}pointwise convergence.\end{icloze}
\end{note}

\begin{note}{c9b4c187b4d54a78a9500289aa5899d0}
    Let \({ (f_n) }\) be a sequence of function on a set \({ A }\).
    \[
        \begin{icloze}{2}f_n \rightrightarrows f\end{icloze}
        \quad \begin{icloze}{3}\iff\end{icloze} \quad
        \begin{icloze}{1}\sup |f_n - f| \underset{n \to \infty}{\longrightarrow} 0.\end{icloze}
    \]

    \begin{center}
        \tiny
        (in terms of \({ \sup }\))
    \end{center}
\end{note}

\begin{note}{1b59f18d7ccb47829cf7b7ea7576318c}
    Let \({ (f_n) }\) be a sequence of function on a set \({ A }\).
    \begin{icloze}{3}Then \({ (f_n) }\) converges uniformly on \({ A }\)\end{icloze} \begin{icloze}{4}if and only if\end{icloze}
    \[
        \begin{gathered}
            \begin{icloze}{1}\forall \varepsilon > 0 \quad \exists N \in \mathbf{N} \quad \forall m, n \geq N\end{icloze}
            \\
            \begin{icloze}{2}\left\lvert f_n - f_m \right\rvert < \varepsilon.\end{icloze}
        \end{gathered}
    \]
\end{note}

\begin{note}{2b9e4671775a43e9aa4a6b4d581b1658}
    Let \({ (f_n) }\) be a sequence of function on a set \({ A }\).
    Then \({ f_n \rightrightarrows f }\) if and only if
    \[
        \begin{gathered}
            \forall \varepsilon > 0 \quad \exists N \in \mathbf{N} \quad \forall m, n \geq N \\
            \left\lvert f_n - f_m \right\rvert < \varepsilon.
        \end{gathered}
    \]

    \begin{center}
        \tiny
        <<\begin{icloze}{1}Cauchy Criterion\end{icloze}>>
    \end{center}
\end{note}

\begin{note}{3fa98b94397f4cc2b2d766dd41934f67}
    What is the key idea in the proof of necessity of the Cauchy Criterion for uniform convergence?

    \begin{cloze}{1}
        Follows immediately from the definition.
    \end{cloze}
\end{note}

\begin{note}{f2d15c9af98f4b82956e48ed7df71fc9}
    What is the key idea in the proof of sufficiency of the Cauchy Criterion for uniform convergence?

    \begin{cloze}{1}
        Define a candidate for the limit and prove by definition.
    \end{cloze}
\end{note}

\begin{note}{1525b27207e74da186a95d7656e895da}
    In the proof of sufficiency of the Cauchy Criterion for uniform convergence, how do you define a candidate for the limit?

    \begin{cloze}{1}
        Use the pointwise limit.
    \end{cloze}
\end{note}

\begin{note}{0177dd65112f46c799884e104b39ef76}
    In the proof of sufficiency of the Cauchy Criterion for uniform convergence, how do we know the pointwise limit exists?

    \begin{cloze}{1}
        Due to the Cauchy Criterion for sequences.
    \end{cloze}
\end{note}

\begin{note}{a1ccae80d31b4f38a4fc876e1ffe4ae7}
    In the proof of sufficiency of the Cauchy Criterion for uniform convergence, we have \({ f_n \to f }\).
    How do you show that \({ f_n \rightrightarrows f }\)?

    \begin{cloze}{1}
        Take the limit of the inequality from the Cauchy Criterion.
    \end{cloze}
\end{note}

\begin{note}{baab958475694fc08316e2031a57fa58}
    Let \({ f_n \to f }\) on a set \({ A }\) and \({ c \in A }\).
    If \begin{icloze}{3}the convergence is uniform\end{icloze} and \begin{icloze}{2}all \({ f_n }\) are continuous at \({ c }\),\end{icloze} then \begin{icloze}{1}\({ f }\) is continuous at \({ c }\).\end{icloze}
\end{note}

\begin{note}{a026cf3ddb2f4d5b9a94b36b2bc20ef9}
    Let \({ f_n \to f }\) on a set \({ A }\) and \({ c \in A }\).
    If the convergence is uniform and all \({ f_n }\) are continuous at \({ c }\), then \({ f }\) is continuous at \({ c }\).

    \begin{center}
        \tiny
        <<\begin{icloze}{1}Continuous Limit Theorem\end{icloze}>>
    \end{center}
\end{note}

\begin{note}{5fd08fca82504ff0af82d320da351ff7}
    What is the key idea in the proof of the Continuous Limit Theorem for a sequence of functions?

    \begin{cloze}{1}
        Triple triangle inequality after adding and subtracting \({ f_N }\).
    \end{cloze}
\end{note}

% Lecture 26.10.22 {{{
\begin{note}{06425162bee447479d3a4f5c71c9cf2a}
    Let \({ f_n \to f }\) on a set \({ A }\) and \({ c \in A }\).
    If \begin{icloze}{1}the convergence is uniform and all \({ f_n }\) are continuous at \({ c }\),\end{icloze} then
    \[
        \lim_{x \to c} \lim_{n \to \infty} f_n(x) = \begin{icloze}{2}\lim_{n \to \infty} \lim_{x \to c} f_n(x).\end{icloze}
    \]
\end{note}
% }}}

\begin{note}{05371b5a401f4756bc04fc154476e2c4}
    Let \({ f_n \to f }\) on a set \({ A }\).
    If each \({ f_n }\) is continuous, but \({ f }\) is discontinuous, then \begin{icloze}{1}the convergence is not uniform.\end{icloze}
\end{note}

\begin{note}{a5ee2f3836bd4545afde8c2d7ecda40e}
    Give an example of a sequence of functions \({ f_n \to f }\) such that
    \begin{itemize}
        \item each \({ f_n }\) is continuous almost everywhere; and
        \item \({ f }\) is nowhere continuous.
    \end{itemize}

    \begin{cloze}{1}
        Step-by-step construction of the Dirichlet's function.
    \end{cloze}
\end{note}

\begin{note}{81c5e1a2081241d1973bb2cacde92627}
    Assume \({ f_n \to f }\) on a set \({ A }\) and each \({ f_n }\) is uniformly continuous.
    If \begin{icloze}{2}\({ f_n \rightrightarrows f }\),\end{icloze} then \begin{icloze}{1}\({ f }\) is uniformly continuous.\end{icloze}
\end{note}

\begin{note}{f819f1c60074468ba1e718298059ade4}
    Assume \({ f_n \to f }\) on a set \({ A }\) and each \({ f_n }\) is bounded.
    If \begin{icloze}{2}\({ f_n \rightrightarrows f }\),\end{icloze} then \begin{icloze}{1}\({ f }\) is bounded.\end{icloze}
\end{note}

\begin{note}{b1fded6e729d40ba99a9d087781866dd}
    Assume \({ f_n \to f }\) on a set \({ A }\) and each \({ f_n }\) has a finite number of discontinuities.
    If \({ f_n \rightrightarrows f }\), then \begin{icloze}{1}\({ f }\) has at most a countable number of discontinuities.\end{icloze}
\end{note}

\begin{note}{a010908ba95d473ea734442288757314}
    Assume \({ f_n \rightrightarrows f }\) on a set \({ A }\) and \({ c \in A }\).
    If \begin{icloze}{2}\({ f }\) is discontinuous at \({ c }\),\end{icloze} then \begin{icloze}{1}all \({ f_n }\) are eventually discontinuous at \({ c }\).\end{icloze}
\end{note}

\begin{note}{5ca0ebc56cc947d1bc6a5ed00cd1617b}
    Assume \({ f_n \rightrightarrows f }\) on a set \({ A }\) and \({ c \in A }\).
    If \({ f }\) is discontinuous at \({ c }\), then all \({ f_n }\) are eventually discontinuous at \({ c }\).
    What is the key idea in the proof?

    \begin{cloze}{1}
        By contradiction + choose a subsequence continuous at \({ c }\).
    \end{cloze}
\end{note}

\begin{note}{4c8d50b955be4fa0a3ba792c5699174f}
    Let \({ f }\) be \begin{icloze}{2}continuous\end{icloze} on all of \({ \mathbf{R} }\).
    Then \({ f(x + \frac{1}{n}) }\) \begin{icloze}{1}converges to \({ f }\).\end{icloze}
\end{note}

\begin{note}{59f59d25a40a4e72afdd62a2dd24bd13}
    Let \({ f }\) be \begin{icloze}{2}uniformly continuous\end{icloze} on all of \({ \mathbf{R} }\).
    Then \({ f(x + \frac{1}{n}) }\) \begin{icloze}{1}converges uniformly to \({ f }\).\end{icloze}
\end{note}

\section{Uniform Convergence and Differentiation} % 6.3
\begin{note}{37f46dbb09f54423a835e842d402ee19}
    What sequence is considered in the Differentiable Limit Theorem?

    \begin{cloze}{1}
        A sequence of differentiable functions that converges point\-wise on a closed interval.
    \end{cloze}
\end{note}

\begin{note}{19574e41800e43678628e78581f801cc}
    When applying the Differentiable Limit Theorem, is it necessary for the limit to be differentiable?

    \begin{cloze}{1}
        No, this is one of the implications.
    \end{cloze}
\end{note}

\begin{note}{5ef400e26d2541e589faa672492059bf}
    When do we conclude something form the Differentiable Limit Theorem?

    \begin{cloze}{1}
        When the derivatives converge uniformly.
    \end{cloze}
\end{note}

\begin{note}{f7da48c586d2457baad72d900c07defd}
    What do we conclude from The Differentiable Limit Theorem?

    \begin{cloze}{1}
        The limit \({ f }\) is differentiable and \({ f' = \lim f'_n }\).
    \end{cloze}
\end{note}

\begin{note}{61acf9aeed834980a9dbaa77746b89e0}
    Let \({ f_n \to f }\) on \({ [a, b] }\) and each \({ f_n }\) is differentiable.
    What do we know about \({ f }\) if \({ f'_n \to g }\)?

    \begin{cloze}{1}
        Nothing special.
    \end{cloze}
\end{note}

\begin{note}{63a1ccb4818a4cd281f9b4d9513500a0}
    Let \({ f_n \to f }\) on \({ [a, b] }\) and each \({ f_n }\) is differentiable.
    What do we know about \({ f }\) if \({ f'_n \rightrightarrows g }\)?

    \begin{cloze}{1}
        \({ f }\) is differentiable and \({ f' = g }\).
    \end{cloze}
\end{note}

\begin{note}{a720c08c553f46a0b0423c46f4c19a2e}
    What is the key idea in the proof of the Differentiable Limit Theorem?

    \begin{cloze}{1}
        Rewrite the limit's derivative by definition.
    \end{cloze}
\end{note}

\begin{note}{31222913007d4ceda945e1a21642c876}
    In the proof of the Differentiable Limit Theorem, how do you find an upper bound for
    \[
        \left\lvert \frac{f(x + h) - f(x)}{h} - g(x) \right\rvert\,?
    \]

    \begin{cloze}{1}
        Expand it using the triple triangle inequality involving \({ f_N }\).
    \end{cloze}
\end{note}

\begin{note}{d239aa3eedd346a69139a5a8b1d94ce7}
    In the proof of the Differentiable Limit Theorem, how do you choose \({ N }\)?

    \begin{cloze}{1}
        By the Cauchy Criterion for \({ f'_n \rightrightarrows g }\).
    \end{cloze}
\end{note}

\begin{note}{70bbcff5bceb49c7b0abb25a8ab9be35}
    In the proof of the Differentiable Limit Theorem, how do you find an upper bound for
    \[
        \left\lvert f'_N(x) - g(x) \right\rvert\,?
    \]

    \begin{cloze}{1}
        Take the limit of the inequality from the Cauchy Criterion.
    \end{cloze}
\end{note}

\begin{note}{ee6b7ee23d0a48a8a32afe978be50a7d}
    In the proof of the Differentiable Limit Theorem, how do you find an upper bound for
    \[
        \left\lvert \frac{f_N(x + h) - f_N(x)}{h} - f'_N(x) \right\rvert\,?
    \]

    \begin{cloze}{1}
        Pick \({ \delta }\) by the definition of differentiability of \({ f_N }\).
    \end{cloze}
\end{note}

\begin{note}{cdb10b03a9254c5abfe796106c1d3e9b}
    In the proof of the Differentiable Limit Theorem, how do you find an upper bound for
    \[
        \left\lvert \frac{f(x + h) - f(x)}{h} - \frac{f_{N}(x + h) - f_N(x)}{h} \right\rvert\,?
    \]

    \begin{cloze}{1}
        The Mean Value Theorem for \({ f_N - f_m }\) and make \({ m \to \infty }\).
    \end{cloze}
\end{note}

\begin{note}{b4b2753226ff4d839269bbf795c02301}
    Let \({ (f_n) }\) be \begin{icloze}{4}a sequence of differentiable functions on \({ [a, b] }\)\end{icloze} and \begin{icloze}{3}\({ (f'_n) }\) converge uniformly.\end{icloze}
    If \begin{icloze}{2}\({ \lim f_n(x_0) }\) exists for some \({ x_0 }\),\end{icloze} then \begin{icloze}{1}\({ (f_n) }\) converges uniformly.\end{icloze}
\end{note}

\begin{note}{8c542d7e30524e129805ce26973b0925}
    How can we weaken the hypothesis of the Differentiable Limit Theorem?

    \begin{cloze}{1}
        \({ (f_n) }\) converges at a single point.
    \end{cloze}
\end{note}

\section{Series of Functions} % 6.4
\begin{note}{b2a393ededc84241bb594417273dea7e}
    Let \({ (f_n) }\) be \begin{icloze}{3}a sequence of functions on a set \({ A }\).\end{icloze}
    \begin{icloze}{2}A functional series\end{icloze} is
    \begin{icloze}{1}
        a formal expression of the form
        \[
            \sum_{n=1}^{\infty} f_n(x).
        \]
    \end{icloze}
\end{note}

\begin{note}{6291bcd4e0274102bfe4090eebac24ef}
    Let \({ (f_n) }\) be a sequence of functions on a set \({ A }\).
    We say \({ \sum_n f_n(x) }\) \begin{icloze}{2}converges pointwise on \({ A }\) to a function \({ f(x) }\)\end{icloze} if \begin{icloze}{1}the sequence of partial sums converges pointwise on \({ A }\) to \({ f }\).\end{icloze}
\end{note}

\begin{note}{084d4603478b4dc48c0d1837ff30dfd8}
    Let \({ (f_n) }\) be a sequence of functions on a set \({ A }\).
    If \begin{icloze}{2}\({ \sum_n f_n(x) }\) converges pointwise to \({ f(x) }\),\end{icloze} we write
    \begin{icloze}{1}
        \[
            f(x) = \sum_n f_n(x).
        \]
    \end{icloze}
\end{note}

\begin{note}{2922cd6ac8ff42fabe5bc630fa320169}
    Let \({ (f_n) }\) be a sequence of functions on a set \({ A }\).
    We say \({ \sum f_n(x) }\) \begin{icloze}{2}converges uniformly on \({ A }\) to a function \({ f(x) }\)\end{icloze} if \begin{icloze}{1}the sequence of partial sums converges uniformly on \({ A }\) to \({ f }\).\end{icloze}
\end{note}

% Lecture 09.11.22 {{{
\begin{note}{2b28ab51bc7f45ca934cc405e7de388f}
    Let \({ \sum_n f_n(x) }\) be a functional series.
    \begin{icloze}{1}
        A series
        \[
            \sum_{n=k+1}^{\infty} f_n(x) \quad \text{for } k \in \mathbf{N},
        \]
    \end{icloze}
    is called \begin{icloze}{2}a tail of \({ \sum_n f_n(x) }\).\end{icloze}
\end{note}

\begin{note}{d633b0c9c968402aba5285afb115d682}
    A series \({ \sum_n f_n(x) }\) \begin{icloze}{2}converges pointwise\end{icloze} \begin{icloze}{3}only if\end{icloze} \begin{icloze}{1}its tail converges pointwise to \({ 0 }\).\end{icloze}

    \begin{center}
        \tiny
        (in terms of the tail)
    \end{center}
\end{note}

\begin{note}{16325daa37b14ddebc3939e1d2ea063b}
    A series \({ \sum_n f_n(x) }\) \begin{icloze}{2}converges uniformly\end{icloze} \begin{icloze}{3}only if\end{icloze} \begin{icloze}{1}its tail converges uniformly to \({ 0 }\).\end{icloze}

    \begin{center}
        \tiny
        (in terms of the tail)
    \end{center}
\end{note}

\begin{note}{891381b2ecd44c2cb160d114479f0b20}
    A series \({ \sum_n f_n(x) }\) \begin{icloze}{2}converges pointwise\end{icloze} \begin{icloze}{3}only if\end{icloze} \begin{icloze}{1}\({ f_n \to 0 }\).\end{icloze}
\end{note}

\begin{note}{767a398cce7c40b781b0c39db5f9b9ac}
    A series \({ \sum_n f_n(x) }\) \begin{icloze}{2}converges uniformly\end{icloze} \begin{icloze}{3}only if\end{icloze} \begin{icloze}{1}\({ f_n \rightrightarrows 0 }\).\end{icloze}
\end{note}
% }}}

\begin{note}{c0a25e35d11c4560a26e2e463a31f725}
    What series is considered in the Term-by-term Continuity Theorem?

    \begin{cloze}{1}
        A series of continuous functions.
    \end{cloze}
\end{note}

\begin{note}{55e76f7381cf476bb7c32155d099bf7c}
    When do we conclude something from the Term-by-term Continuity Theorem?

    \begin{cloze}{1}
        When the functional series converges uniformly.
    \end{cloze}
\end{note}

\begin{note}{84af86f380cf48048b8e6b2c91e25d6c}
    What do we conclude from the Term-by-term Continuity Theorem when the series only converges pointwise?

    \begin{cloze}{1}
        Nothing.
    \end{cloze}
\end{note}

\begin{note}{a2c89255016b4abebcc0733f8178fdef}
    What do we conclude from the Term-by-term Continuity Theorem?

    \begin{cloze}{1}
        The series' sum is continuous.
    \end{cloze}
\end{note}

\begin{note}{9a06615f719646bb8e4bde3a605344f5}
    What series is considered in the Term-by-term Differentiability Theorem?

    \begin{cloze}{1}
        A series of differentiable functions that converges pointwise on a closed interval.
    \end{cloze}
\end{note}

\begin{note}{fa6705a7ca6141eeb7056368500bbdb0}
    When do we conclude something from the Term-by-term Differentiability Theorem?

    \begin{cloze}{1}
        The derivatives' series converge uniformly.
    \end{cloze}
\end{note}

\begin{note}{50a4a0c1c82c4129a14c9af763976811}
    What do we conclude form the Term-by-term Differentiability Theorem?

    \begin{cloze}{1}
        \({ \sum f_n }\) is differentiable and \({ \left( \sum f_n \right)' = \sum f'_n }\).
    \end{cloze}
\end{note}

\begin{note}{296676411bf5475eacdde73dc1c2b008}
    What series is considered in the Weierstrass M-Test?

    \begin{cloze}{1}
        A series of bounded functions.
    \end{cloze}
\end{note}

\begin{note}{5c393b177b724cf69790bafcf0ff7b23}
    When do we conclude something from the Weierstrass M-Test?

    \begin{cloze}{1}
        When the series of ``absolute'' bounds converges.
    \end{cloze}
\end{note}

\begin{note}{f964f11937374d53be121d3893daeef6}
    Which bounds are considered in the Weierstrass M-Test?

    \begin{cloze}{1}
        The sequence of the functions' ``absolute'' upper bounds.
    \end{cloze}
\end{note}

\begin{note}{c48d5e20f5d24ca58a0c3bd71ab7b256}
    What do we conclude from the Weierstrass M-Test?

    \begin{cloze}{1}
        The functional series converges uniformly.
    \end{cloze}
\end{note}

\begin{note}{2f9827fda17c4670b0d2bd4728303ae4}
    What is the key idea in the proof of the Weierstrass M-Test?

    \begin{cloze}{1}
        It follows from the Cauchy Criterion.
    \end{cloze}
\end{note}

\begin{note}{f0d47c16fb4f4ab888dbaa2d8d17ef7a}
    What is the second implication of the Weierstrass M-Test?

    \begin{cloze}{1}
        The series converges absolutely.
    \end{cloze}
\end{note}

\begin{note}{5803d1bfa65b4180921e6b1443015177}
    Why does the Weierstrass M-Test implies absolute convergence?

    \begin{cloze}{1}
        Absolute values have the same upper bounds.
    \end{cloze}
\end{note}

\section{Power Series} % 6.5
\begin{note}{575572e782e64317ba8228d5791138da}
    What is a power series (intuitively)?

    \begin{cloze}{1}
        An infinite polynomial.
    \end{cloze}
\end{note}

\begin{note}{3cd19400150446d68e6df4a87977e765}
    \begin{icloze}{2}A power series\end{icloze} is
    \begin{icloze}{1}
        a series of the form
        \[
            \sum_{n=1}^{\infty} a_n x^{n}.
        \]
    \end{icloze}
\end{note}

\begin{note}{59c245eadd1f4c7c84641a4a81a6cf9c}
    A power series is \begin{icloze}{2}a generalisation\end{icloze} of \begin{icloze}{1}a polynomial.\end{icloze}
\end{note}

\begin{note}{034c6da627e9416d94fe7048441924c4}
    If \({ \sum a_n x^{n} }\) \begin{icloze}{2}converges at some point \({ x_0 \in \mathbf{R} }\)\end{icloze} then \begin{icloze}{1}it converges absolutely\end{icloze} for any \begin{icloze}{3}\({ x }\) satisfying \({ \left\lvert x \right\rvert < \left\lvert x_0 \right\rvert }\).\end{icloze}
\end{note}

\begin{note}{cf119f74fc394dc3a2d9d0c72dd70be5}
    What do we know about \({ \sum a_n x^{n} }\) if it converges at some \({ x_0 }\)?

    \begin{cloze}{1}
        It converges absolutely withing the open interval.
    \end{cloze}
\end{note}

\begin{note}{fed41f842cd54bb1b712f694b52659f9}
    If \({ \sum a_n x^{n} }\) converges at some point \({ x_0 \in \mathbf{R} }\) then it converges absolutely for any \({ x }\) satisfying \({ \left\lvert x \right\rvert < \left\lvert x_0 \right\rvert }\).
    What is the key idea in the proof?

    \begin{cloze}{1}
        Make a geometric series by factoring out \({ \left\lvert \frac{x}{x_0} \right\rvert^{n} }\).
    \end{cloze}
\end{note}

\begin{note}{59b428fc86c24ff8aff670ff3a284435}
    If \({ \sum a_n x^{n} }\) converges at some point \({ x_0 \in \mathbf{R} }\) then it converges absolutely for any \({ x }\) satisfying \({ \left\lvert x \right\rvert < \left\lvert x_0 \right\rvert }\).
    In the proof, how do you turn \({ \sum \left\lvert a_n x_0^{n} \right\rvert \left\lvert \frac{x}{x_0} \right\rvert^{n} }\) into a geometric series?

    \begin{cloze}{1}
        \({ (a_n x^{n}) }\) is bounded + the Comparison Test.
    \end{cloze}
\end{note}

\begin{note}{573b21be0d10467d913040dfe4d493bb}
    Which form may be taken by the set of points for which \({ \sum a_n x^{n} }\) converges?

    \begin{cloze}{1}
        An interval centered around \({ 0 }\).
    \end{cloze}
\end{note}

\begin{note}{ecdb3ab6a5bd4e23bcd67794066ab7c9}
    The set of points for which \({ \sum a_n x^{n} }\) converges is always an interval centered around \({ 0 }\).
    What is the key idea in the proof?

    \begin{cloze}{1}
        Use the ``Interior Convergence'' theorem.
    \end{cloze}
\end{note}

\begin{note}{21ae4818657c4e16b4ef4b2585bc3c18}
    How is the set of points for which \({ \sum a_n x^{n} }\) converges called?

    \begin{cloze}{1}
        The interval of convergence.
    \end{cloze}
\end{note}

\begin{note}{cc247e245b4d47ce8e408ff25ad39c6d}
    Every power series \begin{icloze}{1}converges absolutely\end{icloze} withing \begin{icloze}{2}the interior of its interval of convergence.\end{icloze}
\end{note}

\begin{note}{0fce527887bb4236b7813a76f877c418}
    Every power series converges absolutely withing the interior of its interval of convergence.
    What is the key idea in the proof?

    \begin{cloze}{1}
        Follows from the ``Interior Convergence'' theorem.
    \end{cloze}
\end{note}

\begin{note}{a11ccad617764a25a049ee310707b122}
    \begin{icloze}{2}The radius of convergence\end{icloze} of \({ \sum a_n x^{n} }\) is \begin{icloze}{1}the half length of its interval of convergence.\end{icloze}
\end{note}

\begin{note}{e3918912ac244859ab2293dcbac39594}
    How does \({ \sum a_n x^{n} }\) behave at the endpoints of its interval of convergence?

    \begin{cloze}{1}
        Who knows\ldots
    \end{cloze}
\end{note}

% Lecture 28.11.22 {{{
\begin{note}{18c470fb2da44b60a1d569d93b89f643}
    What are the simplest methods for calculating the radius of convergence of a power series?

    \begin{cloze}{1}
        Using either the Root Test or the Ratio Test.
    \end{cloze}
\end{note}

\begin{note}{b5c35bb7db58465a910f8283bf5f6196}
    How can you use the Root Test to calculate the radius of convergence of a power series?

    \begin{cloze}{1}
        Take the inverse of the coefficients' roots' limit.
    \end{cloze}
\end{note}

\begin{note}{1badd0dc0e5c4500aa468131632c62b9}
    How can you use the Ratio Test to calculate the radius of convergence of a power series?

    \begin{cloze}{1}
        Take the inverse of the coefficients' ratios' limit.
    \end{cloze}
\end{note}

\begin{note}{819016ab8f2c4bfc971839823a9fd8e0}
    Let \({ R }\) be the radius of convergence of \({ \sum a_n x^{n} }\).
    Then
    \[
        R = \begin{icloze}{1}\left( \limsup \sqrt[n]{\left\lvert a_n \right\rvert} \right)^{-1}.\end{icloze}
    \]

    \begin{center}
        \tiny
        <<\begin{icloze}{2}Cauchy–Hadamard Theorem\end{icloze}>>
    \end{center}
\end{note}

\begin{note}{197b7547ea3349ce827335925cf42930}
    In the Cauchy-Hadamard Theorem, what happens when
    \[
        \limsup \sqrt[n]{\left\lvert a_n \right\rvert} = 0\,?
    \]

    \begin{cloze}{1}
        The radius is infinite.
    \end{cloze}
\end{note}

\begin{note}{e998f9fea2924dc8b9884bfb954bfedd}
    In the Cauchy-Hadamard Theorem, what happens when
    \[
        \limsup \sqrt[n]{\left\lvert a_n \right\rvert} = \infty\,?
    \]

    \begin{cloze}{1}
        The radius equals to \({ 0 }\).
    \end{cloze}
\end{note}

\begin{note}{b32f1ebd412842729b113cf6836014e4}
    What is the key idea in the proof of the Cauchy-Hadamard Theorem?

    \begin{cloze}{1}
        The Root Test.
    \end{cloze}
\end{note}

\begin{note}{9223bf629d1a492a892f77c69e4d1cad}
    What does it mean for a power series to be centered at \({ a \neq 0 }\)?

    \begin{cloze}{1}
        It is expressed in terms of \({ (x - a) }\).
    \end{cloze}
\end{note}

\begin{note}{b41b3e3920ae4372a12438b11d262544}
    Let \({ \sum a_n (x - a)^{n} }\) be a power series.
    Then \begin{icloze}{2}the value \({ a }\)\end{icloze} is called \begin{icloze}{1}the center of the series.\end{icloze}
\end{note}

\begin{note}{36acf2e7094146dd8a30193845ea7928}
    Any power series centered at \({ a \neq 0 }\) may be turned into \begin{icloze}{2}a series centered at \({ 0 }\)\end{icloze} by
    \begin{icloze}{1}
        substituting
        \[
            \bar x = x - a.
        \]
    \end{icloze}
\end{note}
% }}}

\begin{note}{872483e057f74ba7addfdfbd543a41a8}
    If \({ \sum a_n x^{n} }\) \begin{icloze}{2}converges absolutely at a point \({ x_0 }\),\end{icloze} then \begin{icloze}{1}it converges uniformly\end{icloze} on \begin{icloze}{3}\({ [-c, c] }\), where \({ c = \left\lvert x_0 \right\rvert }\).\end{icloze}
\end{note}

\begin{note}{8bfee12c2af34918aa416ea9071592ca}
    What do we know about \({ \sum a_n x^{n} }\) if it converges absolutely at some \({ x_0 }\)?

    \begin{cloze}{1}
        It converges uniformly on the closed interval.
    \end{cloze}
\end{note}

\begin{note}{091856fb98e84114b9b21203616d0e36}
    If \({ \sum a_n x^{n} }\) converges absolutely at a point \({ x_0 }\), then it converges uniformly on \({ [-c, c] }\), where \({ c = \left\lvert x_0 \right\rvert }\).
    What is the key idea in the proof?

    \begin{cloze}{1}
        The Weierstrass M-Test.
    \end{cloze}
\end{note}

\begin{note}{ffe00dc944e34558ba9caec75dbcf5cc}
    If \({ \sum a_n x^{n} }\) converges absolutely at a point \({ x_0 }\), then it converges uniformly on \({ [-c, c] }\), where \({ c = \left\lvert x_0 \right\rvert }\).
    What is used as the sequence of upper bounds in the proof?

    \begin{cloze}{1}
        The values at \({ x_0 }\).
    \end{cloze}
\end{note}

\begin{note}{23b23a23d1eb4887a5bb4f0edc237a7d}
    Let \({ R }\) be the radius of convergence of \({ \sum a_n x^{n} }\).
    If \begin{icloze}{2}\({ \sum a_n x^{n} }\) converges absolutely at \({ x = R }\),\end{icloze} then \begin{icloze}{1}it converges uniformly on \({ [-R, R] }\).\end{icloze}
\end{note}

\begin{note}{95017b4b22bf4a64a943cc3064449625}
    Let \({ R }\) be the radius of convergence of \({ \sum a_n x^{n} }\).
    Then for any \({ r \in \begin{icloze}{3}[0, R)\end{icloze} }\), the series \({ \sum a_n x^{n} }\) \begin{icloze}{2}converges uniformly\end{icloze} on \begin{icloze}{1}\({ [-r, r] }\).\end{icloze}
\end{note}

\begin{note}{57e2d62cb3ce40e48ee6290824fbafeb}
    Let \({ R }\) be the radius of convergence of \({ \sum a_n x^{n} }\).
    Then for any \({ r \in [0, R) }\), the series \({ \sum a_n x^{n} }\) converges uniformly on \({ [-r, r] }\).
    What is the key idea in the proof?

    \begin{cloze}{1}
        The series converges absolutely at \({ x = r }\).
    \end{cloze}
\end{note}

\end{document}
