%! TeX root = ./main.tex
\documentclass[11pt, a5paper]{article}
\usepackage[width=10cm, top=0.5cm, bottom=2cm]{geometry}

\usepackage[T1,T2A]{fontenc}
\usepackage[utf8]{inputenc}
\usepackage[english,russian]{babel}
\usepackage{libertine}

\usepackage{amsmath}
\usepackage{amssymb}
\usepackage{amsthm}
\usepackage{mathrsfs}
\usepackage{framed}
\usepackage{xcolor}

\setlength{\parindent}{0pt}

% Force \pagebreak for every section
\let\oldsection\section
\renewcommand\section{\pagebreak\oldsection}

\renewcommand{\thesection}{}
\renewcommand{\thesubsection}{Note \arabic{subsection}}
\renewcommand{\thesubsubsection}{}
\renewcommand{\theparagraph}{}

\newenvironment{note}[1]{\goodbreak\par\subsection{\hfill \color{lightgray}\tiny #1}}{}
\newenvironment{cloze}[2][\ldots]{\begin{leftbar}}{\end{leftbar}}
\newenvironment{icloze}[2][\ldots]{%
  \ignorespaces\text{\tiny \color{lightgray}\{\{c#2::}\hspace{0pt}%
}{%
  \hspace{0pt}\text{\tiny\color{lightgray}\}\}}\unskip%
}


\begin{document}
\section{Лекция 08.02.23}
\begin{note}{75827b1592ce43c89cb6b0ced3b4d31f}
    Что называют точкой единственности нормальной системы ОДУ?

    \begin{cloze}{1}
        Точку, в которой любые два решения совпадают в какой-то окрестности.
    \end{cloze}
\end{note}

\begin{note}{2ea8ea4d05724302a63c54bc0da567d3}
    Что называют областью единственности нормальной системы ОДУ?

    \begin{cloze}{1}
        Множество, каждая точка которого является точкой единственности.
    \end{cloze}
\end{note}

\begin{note}{d01750d4171240d29b0a0cd7c3d3c529}
    Как называется множество, каждая точка которого является точкой единственности нормальной системы ОДУ?

    \begin{cloze}{1}
        Область единственности.
    \end{cloze}
\end{note}

\begin{note}{4d1523d66ab844aa9b0704cc711c3417}
    Какой объект рассматривается в лемме Гронуолла?

    \begin{cloze}{1}
        Вещественная функция, непрерывная на промежутке.
    \end{cloze}
\end{note}

\begin{note}{543b9220dd454bd9924332527fd9daf3}
    При каком условии мы можем что-либо заключить из леммы Гронуолла?

    \begin{cloze}{1}
        Функция неотрицательна и удовлетворяет верхней оценке специального вида.
    \end{cloze}
\end{note}

\begin{note}{a3ecb63abd4a4f7cb6f0b3816af87e42}
    Какая верхняя оценка рассматривается в условии леммы Гронуолла для функции \({ u(x) }\)?

    \begin{cloze}{1}
        \[
            u(x) \leqslant \lambda + \mu \left\lvert \int_{x_0}^{x} u \right\rvert\,.
        \]
    \end{cloze}
\end{note}

\begin{note}{d9166e03442c40348858cea60c552e7b}
    Что мы заключаем из леммы Гронуолла при
    \[
        0 \leqslant u(x) \leqslant \lambda + \mu \left\lvert \int_{x_0}^{x} u \right\rvert\,?
    \]

    \begin{cloze}{1}
        Функция мажорируется \({ \lambda e^{\mu\left\lvert x - x_0 \right\rvert} }\)
    \end{cloze}
\end{note}

\begin{note}{d0ac7971eb0848c5945bd44448c69ed2}
    Как называется утверждение, дающее верхнюю оценку значению функции, удовлетворяющей неравенству
    \[
        0 \leqslant u(x) \leqslant \lambda + \mu \left\lvert \int_{x_0}^{x} u \right\rvert\,?
    \]

    \begin{cloze}{1}
        Лемма Гронуолла.
    \end{cloze}
\end{note}

\begin{note}{027057804d3b47cda81edd42af57dd3a}
    Каков первый шаг в доказательстве леммы Гронуолла?

    \begin{cloze}{1}
        Не умаляя общности, \({ x > x_0 }\).
    \end{cloze}
\end{note}

\begin{note}{a1c68ab3f1ae4bdbbee93390da3a3de8}
    В чём основная идея доказательства леммы Гронуолла?

    \begin{cloze}{1}
        Продифференцировать правую часть и получить для неё рекуррентное неравенство.
    \end{cloze}
\end{note}

\begin{note}{e95c91c2543a47bf89a1249f77ff6307}
    В доказательстве леммы Гронуолла, что делать с
    \[
        F'(x) \leqslant \mu F(x)\,,
    \]
    где \({ F(x) }\) --- верхняя оценка из условия леммы?

    \begin{cloze}{1}
        Перенести всё налево, умножить на \({ e^{\mu(x - x_0)} }\) и ``признать врага в лицо.''
    \end{cloze}
\end{note}

\end{document}
