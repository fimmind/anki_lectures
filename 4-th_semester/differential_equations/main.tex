%! TeX root = ./main.tex
\documentclass[11pt, a5paper]{article}
\usepackage[width=10cm, top=0.5cm, bottom=2cm]{geometry}

\usepackage[T1,T2A]{fontenc}
\usepackage[utf8]{inputenc}
\usepackage[english,russian]{babel}
\usepackage{libertine}

\usepackage{amsmath}
\usepackage{amssymb}
\usepackage{amsthm}
\usepackage{mathrsfs}
\usepackage{framed}
\usepackage{xcolor}

\setlength{\parindent}{0pt}

% Force \pagebreak for every section
\let\oldsection\section
\renewcommand\section{\pagebreak\oldsection}

\renewcommand{\thesection}{}
\renewcommand{\thesubsection}{Note \arabic{subsection}}
\renewcommand{\thesubsubsection}{}
\renewcommand{\theparagraph}{}

\newenvironment{note}[1]{\goodbreak\par\subsection{\hfill \color{lightgray}\tiny #1}}{}
\newenvironment{cloze}[2][\ldots]{\begin{leftbar}}{\end{leftbar}}
\newenvironment{icloze}[2][\ldots]{%
  \text{\tiny \color{lightgray}\{\{c#2::}\hspace{0pt}\ignorespaces%
}{%
  \unskip\hspace{0pt}\text{\tiny\color{lightgray}\}\}}%
}


\begin{document}
\section{Лекция 08.02.23}
\begin{note}{75827b1592ce43c89cb6b0ced3b4d31f}
    Что называют точкой единственности нормальной системы ОДУ?

    \begin{cloze}{1}
        Точку, в которой любые два решения совпадают в какой-то окрестности.
    \end{cloze}
\end{note}

\begin{note}{2ea8ea4d05724302a63c54bc0da567d3}
    Что называют областью единственности нормальной системы ОДУ?

    \begin{cloze}{1}
        Множество, каждая точка которого является точкой единственности.
    \end{cloze}
\end{note}

\begin{note}{d01750d4171240d29b0a0cd7c3d3c529}
    Как называется множество, каждая точка которого является точкой единственности нормальной системы ОДУ?

    \begin{cloze}{1}
        Область единственности.
    \end{cloze}
\end{note}

\begin{note}{4d1523d66ab844aa9b0704cc711c3417}
    Какой объект рассматривается в лемме Гронуолла?

    \begin{cloze}{1}
        Вещественная функция, непрерывная на промежутке.
    \end{cloze}
\end{note}

\begin{note}{543b9220dd454bd9924332527fd9daf3}
    При каком условии мы можем что-либо заключить из леммы Гронуолла?

    \begin{cloze}{1}
        Функция неотрицательна и удовлетворяет верхней оценке специального вида.
    \end{cloze}
\end{note}

\begin{note}{a3ecb63abd4a4f7cb6f0b3816af87e42}
    Какая верхняя оценка рассматривается в условии леммы Гронуолла для функции \({ u(x) }\)?

    \begin{cloze}{1}
        \[
            u(x) \leqslant \lambda + \mu \left\lvert \int_{x_0}^{x} u \right\rvert\,.
        \]
    \end{cloze}
\end{note}

\begin{note}{d9166e03442c40348858cea60c552e7b}
    Что мы заключаем из леммы Гронуолла при
    \[
        0 \leqslant u(x) \leqslant \lambda + \mu \left\lvert \int_{x_0}^{x} u \right\rvert\,?
    \]

    \begin{cloze}{1}
        Функция мажорируется \({ \lambda e^{\mu\left\lvert x - x_0 \right\rvert} }\)
    \end{cloze}
\end{note}

\begin{note}{d0ac7971eb0848c5945bd44448c69ed2}
    Как называется утверждение, дающее верхнюю оценку значению функции, удовлетворяющей неравенству
    \[
        0 \leqslant u(x) \leqslant \lambda + \mu \left\lvert \int_{x_0}^{x} u \right\rvert\,?
    \]

    \begin{cloze}{1}
        Лемма Гронуолла.
    \end{cloze}
\end{note}

\begin{note}{027057804d3b47cda81edd42af57dd3a}
    Каков первый шаг в доказательстве леммы Гронуолла?

    \begin{cloze}{1}
        Не умаляя общности, \({ x > x_0 }\).
    \end{cloze}
\end{note}

\begin{note}{a1c68ab3f1ae4bdbbee93390da3a3de8}
    В чём основная идея доказательства леммы Гронуолла?

    \begin{cloze}{1}
        Продифференцировать правую часть и получить для неё рекуррентное неравенство.
    \end{cloze}
\end{note}

\begin{note}{e95c91c2543a47bf89a1249f77ff6307}
    В доказательстве леммы Гронуолла, что делать с
    \[
        F'(x) \leqslant \mu F(x)\,,
    \]
    где \({ F(x) }\) --- верхняя оценка из условия леммы?

    \begin{cloze}{1}
        Перенести всё налево, умножить на \({ e^{\mu(x - x_0)} }\) и ``признать врага в лицо.''
    \end{cloze}
\end{note}

\begin{note}{2112162dc12d4371a5d95a0a5c81dde1}
    Что мы заключаем из леммы Гронуолла при
    \[
        0 \leqslant u(x) \leqslant \mu \left\lvert \int_{x_0}^{x} u \right\rvert\,?
    \]

    \begin{cloze}{1}
        \({ u \equiv 0 }\).
    \end{cloze}
\end{note}

\begin{note}{3a4dc8a6e6c843cab187bd437bd37ed7}
    Как называется теорема, дающая достаточное условие для единственности решения нормальной системы ОДУ?

    \begin{cloze}{1}
        Теорема единственности. (Без именного названия.)
    \end{cloze}
\end{note}

\begin{note}{f10d5792225a462e818a47338d607d44}
    При каком условии мы можем что-либо заключить и теоремы единственности для нормальной системы
    \[
        \frac{dy}{dx} = f(x, y)\,?
    \]

    \begin{cloze}{1}
        \({ f }\) непрерывна и локально липшицева по \({ y }\) на области.
    \end{cloze}
\end{note}

\begin{note}{3999cae9517e45d38664d4057e294d61}
    Что мы заключаем из теоремы единственности для нормальной системы ОДУ?

    \begin{cloze}{1}
        Рассматриваемая область является областью единственности.
    \end{cloze}
\end{note}

\begin{note}{ee6667cf95fa44ea94ed335d1995449c}
    Что мы знаем про нормальную систему ОДУ \({ \frac{dy}{dx} = f(x, y) }\), если \({ f }\) непрерывна?

    \begin{cloze}{1}
        Система имеет решение в любой точке области.
    \end{cloze}
\end{note}

\begin{note}{ea18f67e9b5744b8bedca43879e2f1ce}
    Что мы знаем про нормальную систему ОДУ \({ \frac{dy}{dx} = f(x, y) }\), если \({ f }\) непрерывна и \({ f \in \operatorname{Lip}_{y,loc} }\)?

    \begin{cloze}{1}
        Система имеет единственное решение в любой точке области.
    \end{cloze}
\end{note}

\begin{note}{83c8abcc85f4421fbf2631d2d41a355e}
    В чём основная идея доказательства теоремы единственности для нормальной системы ОДУ?

    \begin{cloze}{1}
        Эквивалентное интегральное уравнение и лемма Гронуолла для модуля разности двух решений.
    \end{cloze}
\end{note}

\begin{note}{11813e9dad05416ea07fec54079ef308}
    Для каких отображений определяют понятия продолжения влево/вправо?

    \begin{cloze}{1}
        Для отображений на вещественном интервале.
    \end{cloze}
\end{note}

\begin{note}{edd36d4bee94445d87ac7321ef743c2b}
    Пусть \({ f : (a, b) \to \mathbb R^{n} }\).
    Что называется продолжением \({ f }\) вправо за точку \({ b }\)?

    \begin{cloze}{1}
        Продолжение \({ f }\) на \({ (a, b + h) }\) для \({ h > 0 }\).
    \end{cloze}
\end{note}

\begin{note}{97987fbe68164bc8a81f5a04da103eea}
    Пусть \({ f : (a, b) \to \mathbb R^{n} }\).
    Как называется продолжение \({ f }\) на \({ (a, b + h) }\) для \({ h > 0 }\)?

    \begin{cloze}{1}
        Продолжение \({ f }\) вправо за точку \({ b }\).
    \end{cloze}
\end{note}

\begin{note}{07a52d059bbb42b69130084ef38de2bf}
    Пусть \({ f : (a, b) \to \mathbb R^{n} }\).
    Что называется продолжением \({ f }\) влево за точку \({ a }\)?

    \begin{cloze}{1}
        Продолжение \({ f }\) на \({ (a - h, b) }\) для \({ h > 0 }\).
    \end{cloze}
\end{note}

\begin{note}{e4b59d0b8312421eb65334626219bd70}
    Какие решения нормальной системы ОДУ называются продолжимыми вправо?

    \begin{cloze}{1}
        Для которых существует продолжение вправо, являющееся решением на увеличенном интервале.
    \end{cloze}
\end{note}

\begin{note}{29fa852c7ae349a7bbc629ba1ccbc91d}
    Как называется решение нормальной системы ОДУ, для которого существует продолжение вправо, являющееся решением на увеличенном интервале?

    \begin{cloze}{1}
        Оно называется продолжимым вправо за правую границу интервала.
    \end{cloze}
\end{note}

\begin{note}{c491a538b7db4ce3bf0c6ca23237934b}
    Какая нормальная система ОДУ рассматривается в критерии продолжимости решения?

    \begin{cloze}{1}
        Удовлетворяющая теоремам о существовании и единственности.
    \end{cloze}
\end{note}

\begin{note}{8254ec43487a4e93a77cea2097415f38}
    Сколько условий рассматривается в критерии продолжимости решения нормальной системы ОДУ?

    \begin{cloze}{1}
        Два.
    \end{cloze}
\end{note}

\begin{note}{0a9dcea1e6074255be00f65fdea01c03}
    Каково первое условие в критерии продолжимости решения нормальной системы ОДУ?

    \begin{cloze}{1}
        Функция-решение стремится к конечному значению при стремлении аргумента к границе интервала.
    \end{cloze}
\end{note}

\begin{note}{a81ae7bc19a64929ba24fac472d948f7}
    Каково второе условие в критерии продолжимости решения нормальной системы ОДУ?

    \begin{cloze}{1}
        Предельная точка графика решения лежит в области определения системы.
    \end{cloze}
\end{note}

\begin{note}{c1a593f883dc4ca286c605eb39e9e0cb}
    В чём основная идея доказательства критерия продолжимости решения нормальной системы ОДУ (необходимость)?

    \begin{cloze}{1}
        Использовать непрерывность продолжения.
    \end{cloze}
\end{note}

\begin{note}{6bacc1bc712547c98eba8cb8df13c0d7}
    В чём основная идея доказательства критерия продолжимости решения нормальной системы ОДУ (достаточность)?

    \begin{cloze}{1}
        Построить решение в предельно точке по теореме о существовании и единственности.
    \end{cloze}
\end{note}

\section{Лекция 15.02.23}
\begin{note}{2e70121fb1cd476b88648f942378d5af}
    Какое решение нормальной системы ОДУ называется полным?

    \begin{cloze}{1}
        Не продолжимое ни вправо, ни влево.
    \end{cloze}
\end{note}

\begin{note}{292c61dbfb87414badb68d642ac18ce9}
    Как называется решение нормальной системы ОДУ, не продолжимое ни вправо, ни влево?

    \begin{cloze}{1}
        Полное решение.
    \end{cloze}
\end{note}

\section{Семинар 13.02.23}
\begin{note}{b2a41835e1e34c13a8613528a5da5984}
    Какой вопрос решает формула Остроградского-Лиувилля?

    \begin{cloze}{1}
        Поиск общего решения линейного ОДУ порядка \({ 2 }\).
    \end{cloze}
\end{note}

\begin{note}{10ed454bd6f14993b6688b0fd10d83e7}
    К каким линейным ОДУ порядка \({ 2 }\) применима формула Остроградского-Лиувилля?

    \begin{cloze}{1}
        Со старшим коэффициентом равным единице.
    \end{cloze}
\end{note}

\begin{note}{0cc85c87f3cf47e3a72a6ca696e41ed3}
    Что нужно для поиска общего решения линейного ОДУ порядка \({ 2 }\) по формуле Остроградского-Лиувилля?

    \begin{cloze}{1}
        Известное частное решение.
    \end{cloze}
\end{note}

\begin{note}{a752e29e2f11490a868adb3e833f6587}
    Формула Остроградского-Лиувилля для ОДУ
    \[
        y'' + py' + qy = 0
    \]
    с частным решением \({ y_1 }\)\ldots

    \begin{cloze}{1}
        \[
            \begin{vmatrix}
                y_1 & y \\
                y_1' & y'
            \end{vmatrix} = Ce^{-\int p\: dx}\,.
        \]
    \end{cloze}
\end{note}

\begin{note}{f2208397751c43d0be5bdc7ca1d2a728}
    В каком виде обычно ищут частное решение линейного ОДУ порядка \({ 2 }\) для применения формулы Остроградского-Лиувилля?

    \begin{cloze}{1}
        Многочлен или \({ e^{\alpha x} }\).
    \end{cloze}
\end{note}

\begin{note}{e8f4044a18b241959ff403bca242d8d1}
    Как найти степень многочлена при поиске частного решения линейного ОДУ порядка \({ 2 }\)?

    \begin{cloze}{1}
        Подставить \({ x^{n} }\) и приравнять к нулю коэффициент перед старшей степенью.
    \end{cloze}
\end{note}

\begin{note}{e0a16dee90764dffb9f4c18a7d226fd7}
    Как ищется многочлен, являющийся частным решением линейного ОДУ порядка \({ 2 }\), если уже известна его степень?

    \begin{cloze}{1}
        Методом неопределённых коэффициентов.
    \end{cloze}
\end{note}

\end{document}
