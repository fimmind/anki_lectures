%! TeX root = ./main.tex
\documentclass[11pt, a5paper]{article}
\usepackage[width=10cm, top=0.5cm, bottom=2cm]{geometry}

\usepackage[T1,T2A]{fontenc}
\usepackage[utf8]{inputenc}
\usepackage[english,russian]{babel}
\usepackage{libertine}

\usepackage{amsmath}
\usepackage{amssymb}
\usepackage{amsthm}
\usepackage{mathrsfs}
\usepackage{framed}
\usepackage{xcolor}

\setlength{\parindent}{0pt}

% Force \pagebreak for every section
\let\oldsection\section
\renewcommand\section{\pagebreak\oldsection}

\renewcommand{\thesection}{}
\renewcommand{\thesubsection}{Note \arabic{subsection}}
\renewcommand{\thesubsubsection}{}
\renewcommand{\theparagraph}{}

\newenvironment{note}[1]{\goodbreak\par\subsection{\hfill \color{lightgray}\tiny #1}}{}
\newenvironment{cloze}[2][\ldots]{\begin{leftbar}}{\end{leftbar}}
\newenvironment{icloze}[2][\ldots]{%
  \ignorespaces\text{\tiny \color{lightgray}\{\{c#2::}\hspace{0pt}%
}{%
  \hspace{0pt}\text{\tiny\color{lightgray}\}\}}\unskip%
}


\begin{document}
\section{Лекция 08.02.23 (1)}
\begin{note}{c479b17923b04cc899ccd36d430abf6e}
    Для каких функций определяется интеграл Фурье?

    \begin{cloze}{1}
        Кусочно-гладких на любом отрезке и абсолютно интегрируемых на \({ \mathbb R }\).
    \end{cloze}
\end{note}

\begin{note}{2e470cdc0266434182a07d52209c1bca}
    Откуда, в общих чертах, возникает понятие интеграла Фурье?

    \begin{cloze}{1}
        Из предельного перехода от рядов Фурье.
    \end{cloze}
\end{note}

\begin{note}{a801b28fdfec4b9fa7c62fd2092d035a}
    Как выглядит ряд Фурье \({ 2l }\)-периодической функции?

    \begin{cloze}{1}
        \[
            \frac{a_0}{2} + \sum_{n=1}^{\infty}(a_n \cos \frac{\pi nx}{l} + b_n \sin \frac{\pi nx}{l})\,.
        \]
    \end{cloze}
\end{note}

\begin{note}{a59aff7532704ca498a695bca1ca01d1}
    Как определяются коэффициенты \({ a_n }\) ряда Фурье \({ 2l }\)-пери\-о\-ди\-чес\-кой функции?

    \begin{cloze}{1}
        \[
            \frac{1}{l} \int_{-l}^{l} f(x) \cos \frac{\pi nx}{l}\: dx\,.
        \]
    \end{cloze}
\end{note}

\begin{note}{9ecca0b0846a4282ba65eb6706848d71}
    Как определяются коэффициенты \({ b_n }\) ряда Фурье \({ 2l }\)-пери\-о\-ди\-чес\-кой функции?

    \begin{cloze}{1}
        \[
            \frac{1}{l} \int_{-l}^{l} f(x) \sin \frac{\pi nx}{l}\: dx\,.
        \]
    \end{cloze}
\end{note}

\begin{note}{64752d21c9814733ac5b4fc4a8ebb571}
    Для каких функций выполняется интегральная формула Фурье?

    \begin{cloze}{1}
        Кусочно-гладких на любом отрезке и абсолютно интегрируемых на \({ \mathbb R }\).
    \end{cloze}
\end{note}

\begin{note}{b4fb252816f14260aae170ff59eb8c0d}
    Как выводится интегральная формула Фурье?

    \begin{cloze}{1}
        Как предел разложения в ряд Фурье на \({ [-r, r] }\) при \({ r \to \infty }\).
    \end{cloze}
\end{note}

\begin{note}{a459215af9634f8a9ef3a69def0eb908}
    Что в выводе интегральной формулы Фурье происходит со свободным членом разложения в ряд Фурье?

    \begin{cloze}{1}
        Он стремится к нулю.
    \end{cloze}
\end{note}

\begin{note}{58b7f21884cd410fb9bd777ec693ebc4}
    Как в выводе интегральной формулы Фурье перейти от суммы к интегралу?

    \begin{cloze}{1}
        Использовать неформальное сходство с интегральной суммой соответствующего интеграла.
    \end{cloze}
\end{note}

\begin{note}{c9abe7ec159c45c2a383de945f6c72e5}
    Как в выводе интегральной формулы Фурье показать неформальное сходство выражения
    \[
        \sum_{n=1}^{\infty} \frac{1}{l} \left( \int_{-l}^{l} f(x) \cos \frac{\pi n (x - u)}{l}\: du \right)
    \]
    с соответствующей интегральной суммой?

    \begin{cloze}{1}
        Ввести \({ \lambda_k = \frac{\pi n}{l} }\), откуда \({ \Delta \lambda_k = \frac{\pi}{l} }\).
    \end{cloze}
\end{note}

\begin{note}{3b4198d58dcc4333b79c02b0ff82b891}
    Интеграл Фурье\ldots

    \begin{cloze}{1}
        \[
            \int_{0}^{\infty} (a(\lambda) \cos \lambda x + b(\lambda) \sin \lambda x)\: d\lambda\,.
        \]
    \end{cloze}
\end{note}

\begin{note}{c889f142f0cf44e0b06b987221c635da}
    Как определяется коэффициент \({ a(\lambda) }\) в интеграле Фурье функции \({ f }\)?

    \begin{cloze}{1}
        \[
            a(\lambda) = \frac{1}{\pi} \int_{\mathbb R} f(x) \cos \lambda x\: dx\,.
        \]
    \end{cloze}
\end{note}

\begin{note}{d1a6cd839e7c4a2786c48a1a3ad6b0fa}
    Как определяется коэффициент \({ b(\lambda) }\) в интеграле Фурье функции \({ f }\)?

    \begin{cloze}{1}
        \[
            b(\lambda) = \frac{1}{\pi} \int_{\mathbb R} f(x) \sin \lambda x\: dx\,.
        \]
    \end{cloze}
\end{note}

\begin{note}{3ea5ca54f4a44a11a08a441ee3f997ea}
    Коэффициенты \({ a(\lambda) }\) и \({ b(\lambda) }\) в интеграле Фурье фактически задают \begin{icloze}{1}закон распределения амплитуд и начальных фаз в зависимости от частоты.\end{icloze}
\end{note}

\begin{note}{a77d334a54924e4ba13350819b4aa62e}
    Как называется интеграл
    \[
        \frac{1}{\pi} \int_{0}^{\infty} d\lambda \int_{\mathbb R} f(u) \cos (\lambda(t-x))\: du\,?
    \]

    \begin{cloze}{1}
        Интеграл Фурье.
    \end{cloze}
\end{note}

\begin{note}{2106cbd921f24b16abbcd357a0ca9e55}
    Что утверждает интегральная формула Фурье?

    \begin{cloze}{1}
        Равенство среднего значения односторонних пределов значению интегралу Фурье.
    \end{cloze}
\end{note}

\begin{note}{0cb0976cbc5442fd977022e6b98b5088}
    Как интеграл Фурье упрощается для нечётных функций?

    \begin{cloze}{1}
        Остаются только синусы.
    \end{cloze}
\end{note}

\begin{note}{ee2657adb0724053ab951a4022088d05}
    Как интеграл Фурье упрощается для чётных функций?

    \begin{cloze}{1}
        Остаются только косинусы.
    \end{cloze}
\end{note}

\begin{note}{405a32411509412b8a3b7a6a0384c14b}
    Как интеграл Фурье строится для функций, определённых на \({ (0, +\infty) }\)?

    \begin{cloze}{1}
        Путём (не)чётного продолжения функции на \({ \mathbb R }\).
    \end{cloze}
\end{note}

\section{Лекция 08.02.23 (2)}
\begin{note}{2997689aa1c24e939f84c1392dfc1061}
    Чем в первую очередь является интеграл, зависящий от параметра?

    \begin{cloze}{1}
        Функция, аргумент которой играет роль параметра.
    \end{cloze}
\end{note}

\begin{note}{0e0c024c4c584815bc1daf2b87d001fe}
    Какие части интеграла, зависящего от параметра, собственно могут зависеть от параметра?

    \begin{cloze}{1}
        Границы интегрирования и подынтегральная функция.
    \end{cloze}
\end{note}

\begin{note}{b96f8eb96a4d42cdb1f17860eca8ffe9}
    Какой интеграл рассматривается в теореме о непрерывности интеграла, зависящего от параметра?

    \begin{cloze}{1}
        Интеграл по отрезку, не зависящему от параметра.
    \end{cloze}
\end{note}

\begin{note}{d87dbab278f348399f0e5d5da2d74c2e}
    Какому множеству принадлежат значения параметра в теореме о непрерывности интеграла, зависящего от параметра?

    \begin{cloze}{1}
        Фиксированный отрезок.
    \end{cloze}
\end{note}

\begin{note}{118ac67b133e45dc9bd834fbf8a12aef}
    При каком условии мы можем что-либо заключить из теоремы о непрерывности интеграла, зависящего от параметра?

    \begin{cloze}{1}
        Подынтегральная функция непрерывна (на области определения в \({ \mathbb R^2 }\)).
    \end{cloze}
\end{note}

\begin{note}{7e1447d4172a450a926a1e0ccb1fcaf0}
    Что мы заключаем из теоремы о непрерывности интеграла, зависящего от параметра?

    \begin{cloze}{1}
        Интеграл по параметру непрерывен.
    \end{cloze}
\end{note}

\end{document}
