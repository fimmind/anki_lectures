\documentclass[11pt, a5paper]{article}
\usepackage[width=10cm, top=0.5cm, bottom=2cm]{geometry}

\usepackage[T1,T2A]{fontenc}
\usepackage[utf8]{inputenc}
\usepackage[english,russian]{babel}
\usepackage{libertine}

\usepackage{amsmath}
\usepackage{amssymb}
\usepackage{amsthm}
\usepackage{mathrsfs}
\usepackage{framed}
\usepackage{xcolor}

\setlength{\parindent}{0pt}

% Force \pagebreak for every section
\let\oldsection\section
\renewcommand\section{\pagebreak\oldsection}

\renewcommand{\thesection}{}
\renewcommand{\thesubsection}{Note \arabic{subsection}}
\renewcommand{\thesubsubsection}{}
\renewcommand{\theparagraph}{}

\newenvironment{note}[1]{\goodbreak\par\subsection{\hfill \color{lightgray}\tiny #1}}{}
\newenvironment{cloze}[2][\ldots]{\begin{leftbar}}{\end{leftbar}}
\newenvironment{icloze}[2][\ldots]{%
  \ignorespaces\text{\tiny \color{lightgray}\{\{c#2::}\hspace{0pt}%
}{%
  \hspace{0pt}\text{\tiny\color{lightgray}\}\}}\unskip%
}


\begin{document}
\section{Семинар 16.07.22}
\begin{note}{2523419404034ae8b9f8cccc3ba532e2}
    \begin{icloze}{2}Алгебра над полем\end{icloze} --- это \begin{icloze}{1}векторное пространство, снабжённое билинейным произведением.\end{icloze}
\end{note}

\begin{note}{2dbf3b8106f34daead1805a80b0fb28a}
    Пусть \({ X }\) --- топологическое пространство.
    \begin{icloze}{2}Алгебра всех комплексных функций, непрерывных на \({ X }\)\end{icloze} обозначается \begin{icloze}{2}\({ C(X) }\).\end{icloze}
\end{note}

\begin{note}{13d6e19dd257402ab78d8b26cf54203d}
    Пусть \({ G }\) --- группа, \({ X }\) --- топологическое пространство.
    \begin{icloze}{3}Действием группы \({ G }\) на пространство \({ X }\)\end{icloze} называется
    \begin{icloze}{1}
        отображение
        \[
            G \times X \to X, \quad (g, x) \mapsto g(x),
        \]
    \end{icloze}
    такое, что
    \begin{icloze}{2}
        \({ \forall x \in X }\)
        \[
            \begin{gathered}
                gh(x) = g(h(x)) \quad \forall g, h \in G, \\
                e(x) = x.
            \end{gathered}
        \]
    \end{icloze}
\end{note}

\begin{note}{fa2504287a174d5eb5063ef21d0e2191}
    Пусть \({ G }\) --- группа, \({ X }\) --- топологическое пространство.
    Если \begin{icloze}{2}задано действие \({ G }\) на \({ X }\),\end{icloze} то \({ G }\) называется \begin{icloze}{1}группой, действующей на \({ X }\).\end{icloze}
\end{note}

\begin{note}{f1bfa6fbf90b4a28a445080a7d8d954c}
    Пусть \({ X }\) --- топологическое пространство, \({ G }\) --- группа, действующая на \({ X }\).
    \begin{icloze}{3}Скрещенным произведением алгебры \({ C(X) }\) и группы \({ G }\)\end{icloze} называется алгебра сумм
    \begin{icloze}{1}
        \[
            a = \sum_{g \in G}^{} a_{g}(x) T_{g},
        \]
    \end{icloze}
    где
    \begin{icloze}{2}
        для всех \({ g }\)
        \begin{center}
            \({ a_{g} \in C(X) }\), \quad \({ T_{g} }\) --- формальный символ.
        \end{center}
    \end{icloze}
\end{note}

\begin{note}{19d452193f4649e48c0c687527934e5e}
    Пусть \({ X }\) --- топологическое пространство, \({ G }\) --- группа, действующая на \({ X }\).
    \begin{icloze}{2}Скрещенное произведение \({ C(X) }\) и \({ G }\)\end{icloze} обозначается
    \begin{icloze}{1}
        \[
            C(X) \rtimes G.
        \]
    \end{icloze}
\end{note}

\begin{note}{c4f3819a786b4827a2d83e150447f344}
    Пусть \({ X }\) --- топологическое пространство, \({ G }\) --- группа, действующая на \({ X }\).
    Тогда для \({ a, b \in C(X) \rtimes G }\)
    \[
        \begin{icloze}{3}ab\end{icloze} \overset{\text{def}}= \begin{icloze}{2}\sum_{g \in G}^{} c_g(x) T_g,\end{icloze}
    \]
    где
    \[
        c_g(x) = \begin{icloze}{1}\sum_{g_1g_2=g}^{} a_{g_1}(x) \cdot b_{g_2}(g_1^{-1}(x)).\end{icloze}
    \]
\end{note}
\begin{note}{d74629e9c365489a886707b75712917f}
    Пусть \({ X }\) --- топологическое пространство.
    Для удобства, любой элемент \({ f \in C(X) }\) \begin{icloze}{2}отождествляется\end{icloze} с оператором
    \begin{icloze}{1}
        \[
            \begin{gathered}
                u(x) \mapsto f(x) u(x), \\
                C(X) \to C(X).
            \end{gathered}
        \]
    \end{icloze}
\end{note}

\begin{note}{b18bc7b3cd1d490d9465207948a2a7ab}
    В общем случае, \begin{icloze}{2}вложение некоторого объекта \({ X }\) в \({ Y }\)\end{icloze} задаётся \begin{icloze}{1}инъективным отображением \({ X \to Y }\), сохраняющим некоторую структуру.\end{icloze}
\end{note}

\begin{note}{548919c39e5e4beb835a76d8e6d32d98}
    \({ X \overset{f}{\to} Y }\)
    Если \({ f : X \to Y }\) есть \begin{icloze}{2}вложение,\end{icloze} то пишут
    \begin{icloze}{1}
        \[
            f : X \hookrightarrow Y.
        \]
    \end{icloze}
\end{note}

\begin{note}{0060adf86b04424ca2977e10cf57149b}
    Пусть \({ X }\) --- топологическое пространство, \({ G }\) --- группа, действующая на \({ X }\),\: \begin{icloze}{3}\({ g \in G }\).\end{icloze}
    \begin{icloze}{2}
        Отображение вида
        \[
            u(x) \mapsto u(g^{-1}(x))
        \]
    \end{icloze}
    называется \begin{icloze}{1}оператором сдвига по элементу \({ g }\).\end{icloze}
\end{note}

\begin{note}{c1649977ba8340c9adc2dbb8096586c6}
    Пусть \({ X }\) --- топологическое пространство, \({ G }\) --- группа, действующая на \({ X }\),\: \begin{icloze}{3}\({ g \in G }\).\end{icloze}
    \begin{icloze}{2}Оператор сдвига по элементу \({ g }\)\end{icloze} обозначается \begin{icloze}{1}\({ T_{g} }\).\end{icloze}
\end{note}

\begin{note}{f297c8bf8e714c45b1849042d8856179}
    Пусть \({ X }\) --- топологическое пространство, \({ G }\) --- группа, действующая на \({ X }\),\: \({ g, h \in G }\).
    Тогда
    \[
        T_{g}T_{h} = \begin{icloze}{1}T_{gh}.\end{icloze}
    \]
\end{note}

\begin{note}{300bb47328c6444bb9b5075d0838b28c}
    Пусть \({ X }\) --- топологическое пространство, \({ G }\) --- группа, действующая на \({ X }\).
    Тогда \begin{icloze}{3}\({ C(X) \rtimes G }\)\end{icloze} --- это подалгебра в \begin{icloze}{2}алгебре \({ \mathcal L(C(X), C(X)) }\),\end{icloze} порождённая \begin{icloze}{1}\({ C(X) }\) и всеми \({ T_{g} }\).\end{icloze}
\end{note}

\begin{note}{91b2aedf157843449d78550095cf2a5a}
    Пусть \({ X }\) --- топологическое пространство, \({ G }\) --- группа, действующая на \({ X }\).
    Тогда если \({ a, b \in C(X) }\) и \({ g, h \in G }\), то
    \[
        a(x)T_{g} \cdot b(x) T_{h} = \begin{icloze}{1}a(x) \cdot b(g^{-1}(x)) \cdot T_{gh}\end{icloze}
    \]
\end{note}

\begin{note}{530881c476554c1298eec9c9922e8976}
    Пусть \({ X }\) --- топологическое пространство, \({ G }\) --- группа, действующая на \({ X }\).
    Тогда если \({ a, b \in C(X) }\) и \({ g, h \in G }\), то
    \[
        a(x)T_{g} \cdot b(x) T_{h} = a(x) \cdot b(g^{-1}(x)) \cdot T_{gh}
    \]
    В чём ключевая идея доказательства?

    \begin{cloze}{1}
        \[
            T_{g} b(x) = T_{g} b(x) T_{g}^{-1} \cdot T_{g}.
        \]
    \end{cloze}
\end{note}

\begin{note}{8cc6c193acf84003aae8cb34420dd671}
    Пусть \({ X }\) --- топологическое пространство, \({ G }\) --- группа, действующая на \({ X }\).
    Тогда если \({ a \in C(X) }\) и \({ g \in G }\), то
    \[
        T_{g} a(x) T_{g}^{-1} = \begin{icloze}{1}a(g^{-1}(x))\end{icloze}
    \]
\end{note}

\begin{note}{283c514e338044a1b7080d65431ea8a9}
    Пусть \({ X }\) --- топологическое пространство, \({ G }\) --- группа, действующая на \({ X }\).
    Тогда если \({ g \in G }\), то
    \[
        T_{g}^{-1} : u(x) \mapsto \begin{icloze}{1}u(g(x)).\end{icloze}
    \]
\end{note}

\begin{note}{95f5285bfc0d4800a355f5f549bcbbe0}
    Пусть \({ X }\) --- топологическое пространство, \({ G }\) --- группа, действующая на \({ X }\).
    Тогда если \({ G = \begin{icloze}{2}\left\{ e \right\}\end{icloze} }\), то
    \[
        C(X) \rtimes G \simeq \begin{icloze}{1}C(X).\end{icloze}
    \]
\end{note}

\begin{note}{031cb4243696472893716f4da12f30ce}
    Пусть \({ X }\) --- топологическое пространство.
    Тогда если \({ X = \begin{icloze}{2}\left\{ p \right\}\end{icloze} }\), то
    \[
        C(X) \simeq \begin{icloze}{1}\mathbb C.\end{icloze}
    \]
\end{note}

\begin{note}{f3ee7aac98c146a6b535a01cb58d9cdd}
    Пусть \({ G }\) --- группа.
    \begin{icloze}{1}Комплексное линейное пространство с базисом из всех элементов \({ G }\) и умножением, индуцированным от группы,\end{icloze} называется \begin{icloze}{2}групповой алгеброй \({ G }\).\end{icloze}
\end{note}

\begin{note}{5cb452b8847e48d3a0ee11a002b6b93e}
    Пусть \({ G }\) --- группа.
    \begin{icloze}{2}Групповая алгебра \({ G }\)\end{icloze} обозначается \begin{icloze}{1}\({ \mathbb C[G] }\).\end{icloze}
\end{note}

\begin{note}{4d516ad8a53f4d1ebb0491f89c44b511}
    Пусть \({ X }\) --- топологическое пространство, \({ G }\) --- группа, действующая на \({ X }\).
    Тогда если \({ X = \begin{icloze}{2}\left\{ p \right\}\end{icloze} }\), то
    \[
        C(X) \rtimes G \simeq \begin{icloze}{1}\mathbb C[G].\end{icloze}
    \]
\end{note}

\end{document}
