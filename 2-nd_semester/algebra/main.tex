\documentclass[11pt, a5paper]{article}
\usepackage[width=10cm, top=0.5cm, bottom=2cm]{geometry}

\usepackage[T1,T2A]{fontenc}
\usepackage[utf8]{inputenc}
\usepackage[english,russian]{babel}
\usepackage{libertine}

\usepackage{amsmath}
\usepackage{amssymb}
\usepackage{amsthm}
\usepackage{mathrsfs}
\usepackage{framed}
\usepackage{xcolor}

\setlength{\parindent}{0pt}

% Force \pagebreak for every section
\let\oldsection\section
\renewcommand\section{\pagebreak\oldsection}

\renewcommand{\thesection}{}
\renewcommand{\thesubsection}{Note \arabic{subsection}}
\renewcommand{\thesubsubsection}{}
\renewcommand{\theparagraph}{}

\newenvironment{note}[1]{\goodbreak\par\subsection{\hfill \color{lightgray}\tiny #1}}{}
\newenvironment{cloze}[2][\ldots]{\begin{leftbar}}{\end{leftbar}}
\newenvironment{icloze}[2][\ldots]{%
  \ignorespaces\text{\tiny \color{lightgray}\{\{c#2::}\hspace{0pt}%
}{%
  \hspace{0pt}\text{\tiny\color{lightgray}\}\}}\unskip%
}


\begin{document}
\section{Лекция 07.02.22}
\begin{note}{b84aca6df42d4d74ad1fea51970c01d9}
    Пусть \begin{icloze}{3}\( W \) --- линейное пространство, \( V \subset W. \) \end{icloze} Тогда \( V \) называется \begin{icloze}{2}линейным подпространством\end{icloze}, если
    \begin{icloze}{1}
        \begin{enumerate}
            \item \( \forall v \in V, k \in \mathbb R \implies kv \in V,  \)
            \item \( \forall v_1, v_2 \in V \implies v_1 + v_2 \in V. \)
        \end{enumerate}
    \end{icloze}
\end{note}

\begin{note}{a2e780e4b5ff4b4199b594e34bf762c6}
    Выражение <<\( V \) есть линейное подпространство в \( W \)>> обозначают
    \begin{icloze}{1}
        \[
            V \triangleleft W
        \]
    \end{icloze}
\end{note}

\begin{note}{baa489a3d13c4978866a82630be13e73}
    Пусть \( W \) --- линейное пространство, \( V \triangleleft W \). Тогда \( V \) --- \begin{icloze}{1}тоже линейное пространство\end{icloze}.
\end{note}

\begin{note}{3c2988d9ae174eb4aa377f43ebd61f74}
    Является ли прямая проходящая через начало координат подпространством в \( \mathbb R ^{n}  \)?

    \begin{cloze}{1}
        Да, поскольку любая линейная комбинация векторов на прямой тоже лежит на этой прямой.
    \end{cloze}
\end{note}

\begin{note}{18b402a364da457aaaf95095b9113dcd}
    Пусть \( W = \mathbb R ^{n}, A \sim m \times n. \)
    Является ли множество
    \[
        V = \{ x \in W \mid Ax = 0  \}
    \]
    линейным подпространством?

    \begin{cloze}{1}
        Да, поскольку \( \forall u, v \in V, \quad \alpha, \beta \in \mathbb R \quad A(\alpha u + \beta v) = 0. \)
    \end{cloze}
\end{note}

\begin{note}{a5081684e6014eeb8d4cd352f7dfd46b}
    Пусть \( V \triangleleft \mathbb R ^{n}. \) Тогда всегда существует \( A \in \mathbb R ^{\begin{icloze}{2}m \times n\end{icloze}}  \) такая, что
    \begin{icloze}{1}\[
        V = \ker A.
    \]\end{icloze}
\end{note}

\begin{note}{eecf9dfacd2b41218565f8582275c53b}
    Пусть \({ V = \mathscr L (a_1, \ldots, a_m) \triangleleft \mathbb R^{n} }\). Как найти матрицу \({ А }\) такую, что \({ \ker A = V }\)?

    \begin{cloze}{1}
        Строки матрицы \({ A }\) --- (транспонированная) ФСР соответствующей СЛАУ.
    \end{cloze}
\end{note}

\begin{note}{dcb727a8588c412db845188bf547fd9e}
    Пусть \( W = \mathbb R ^{n}, \quad a_1, a_2, \ldots a_n \in W. \) Является ли
    \[
        \mathscr L (a_1, a_2, \ldots a_n)
    \]
    подпространством в \( W \)?

    \begin{cloze}{1}
        Да, является, поскольку любая линейная комбинация линейных комбинаций \( a_1, a_2, \ldots a_n  \)  тоже является их линейной комбинацией.
    \end{cloze}
\end{note}

\begin{note}{d633780bbade46968c2bcb66d05be478}
    Пусть \( W \) --- линейное пространство, \( V_1, V_2 \triangleleft W \).
    Всегда ли
    \[
        V_1 \cap V_2 \triangleleft W?
    \]


    \begin{cloze}{1}
        Да, всегда.
    \end{cloze}
\end{note}

\begin{note}{9c714ab9fa4b457f993438ef25421061}
    Пусть \( W \) --- линейное пространство, \( V_1, V_2 \triangleleft W \).
    Всегда ли
    \[
        V_1 \cup V_2 \triangleleft W?
    \]

    \begin{cloze}{1}
        Нет, не всегда.
    \end{cloze}
\end{note}

\begin{note}{2b9216d113914ad98cbc81b055dc174b}
    Пусть \( W \) --- линейное пространство, \( V_1, V_2 \triangleleft W \).
    Тогда
    \[
        \begin{icloze}{2}V_1 + V_2\end{icloze} \overset{\text{def}}= \begin{icloze}{1}
            \{ v_1 + v_2 \mid v_1 \in V_1, \quad v_2 \in V_2 \} .
        \end{icloze}
    \]
\end{note}

\begin{note}{cd25e86c13c141be80e3673edfece8d2}
    Пусть \( W \) --- линейное пространство, \( V_1, V_2 \triangleleft W \).
    Тогда
    \[
        \dim (V_1 + V_2) = \begin{icloze}{1}\dim V_1 + \dim V_2 - \dim (V_1 \cap V_2).\end{icloze}
    \]
\end{note}

\begin{note}{ecf370041c6b4016a92ca63a4b3675eb}
    Пусть \( W \) --- линейное пространство, \( V_1, V_2 \triangleleft W \).
    Всегда ли
    \[
        V_1 + V_2 \triangleleft W?
    \]

    \begin{cloze}{1}
        Да, всегда.
    \end{cloze}
\end{note}

\begin{note}{fe58542dc0ee4e48ab330cd68be1fd77}
    Пусть \( W \) --- линейное пространство, \( V \triangleleft W \) и \( e_1, e_2, \ldots, e_k  \)  --- \begin{icloze}{2}базис в \( V. \)\end{icloze}
    Тогда в \( W \) существует базис вида \begin{icloze}{1}
        \[
            e_1, e_2, \ldots, e_k, e_{k + 1}, \ldots, e_n.
        \]
    \end{icloze}
\end{note}

\begin{note}{7e41e14368b94d50be88c6e5b025c706}
    В чем основная идея доказательства теоремы о размерности суммы подпространств?

    \begin{cloze}{1}
        Дополнить базис в \( V_1 \cap V_2 \) до базисов в \( V_1 \) и \( V_2 \) соответственно и построить на их основе базис в \( V_1 + V_2 \).
    \end{cloze}
\end{note}

\begin{note}{01ac0beb84404bed8a9f676002a2804c}
    Пусть
        \({ \left\{ e_i \right\} }\) --- базис в \({ V_1 \cap V_2 }\),\:
        \({ \left\{ e_i, f_j \right\} }\) --- базис в \({ V_1 }\),\:
        \({ \left\{ e_i, g_k \right\} }\) --- базис в \({ V_2 }\).
    Как можно построить базис в \( V_1 + V_2 \)?

    \begin{cloze}{1}
        Объединить их в одну систему \({ \left\{ e_i, f_j, g_k \right\} }\).
    \end{cloze}
\end{note}

\begin{note}{d6aa3baccb104c5d857dad61f06b75e7}
    Пусть
        \({ \left\{ e_i \right\} }\) --- базис в \({ V_1 \cap V_2 }\),\:
        \({ \left\{ e_i, f_j \right\} }\) --- базис в \({ V_1 }\),\:
        \({ \left\{ e_i, g_k \right\} }\) --- базис в \({ V_2 }\).
    Как показать, что \({ \left\{ e_i, f_j, g_k \right\} }\) --- базис в \({ V_1 + V_2 }\)?

    \begin{cloze}{1}
        Показать, что \( \mathscr L (\left\{ g_k \right\}) \cap V_1 = \left\{  0 \right\} \).
    \end{cloze}
\end{note}

\begin{note}{28934bf74ae1452191c8e81b8cef0cf5}
    Пусть
        \({ \left\{ e_i \right\} }\) --- базис в \({ V_1 \cap V_2 }\),\:
        \({ \left\{ e_i, f_j \right\} }\) --- базис в \({ V_1 }\),\:
        \({ \left\{ e_i, g_k \right\} }\) --- базис в \({ V_2 }\).
    В чём ключевая идея доказательства того, что
    \[
        \mathscr L (\left\{ g_k \right\}) \cap V_1 = \left\{  0 \right\}?
    \]

    \begin{cloze}{1}
        Если \({ \sum_{k}^{} \lambda_k g_k \in V_1 }\), то она принадлежит и \({ V_1 \cap V_2 }\).
    \end{cloze}
\end{note}

\section{Семинар 09.02.22}
\begin{note}{3fd21160928849f8acbc526a60229e49}
    Пусть \( e_1, e_2, \ldots, e_n \) и \( e'_1, e'_2, \ldots, e'_n \) --- два базиса в линейном пространстве \( V \).
    Тогда \begin{icloze}{2}матрицей перехода от базиса \( e \) к базису \( e' \)\end{icloze} называют
    \begin{icloze}{1}матрицу \( C \) такую, что для любого \( v \in V \), если
    \[
        \begin{gathered}
            v = \lambda_1 e_1 + \lambda_2 e_2 + \cdots + \lambda_n e_n, \\
            v = \mu_1 e'_1 + \mu_2 e'_2 + \cdots + \mu_n e'_n,
        \end{gathered}
    \]
    то
    \[
        C \begin{bmatrix}
            \mu_1 \\ \mu_2 \\ \vdots \\ \mu_n
        \end{bmatrix}
        =
        \begin{bmatrix}
            \lambda_1 \\ \lambda_2 \\ \vdots \\ \lambda_n
        \end{bmatrix}.
    \]
\end{icloze}
\end{note}

\begin{note}{88fab27df46a451190278cbc1d38698f}
    \begin{icloze}{2}Матрицу перехода от базиса \( e \) к базису \( e' \)\end{icloze} обычно обозначают \begin{icloze}{1}\( C_{e \to e'}  \).\end{icloze}
\end{note}

\begin{note}{c9e84965d5ea4157b50f6576e2cbddad}
    Пусть \( e_1, e_2, \ldots, e_n \) и \( e'_1, e'_2, \ldots, e'_n \) --- два базиса в линейном пространстве.
    Как в явном виде задать матрицу \( C_{e \to e'} \)?

    \begin{cloze}{1}
        Столбцы \( C_{e \to e'} \) --- это координаты векторов \( e'_1, e'_2, \ldots, e'_n \) в базисе \( e_1, e_2, \ldots, e_n \).
    \end{cloze}
\end{note}

\section{Лекция 14.02.22}
\begin{note}{825be05cbe9f4850806682f4db48f5e1}
    Пусть \( W \) --- линейное пространство, \( V_1, V_2 \triangleleft W \).
    \begin{icloze}{2}Сумму \( V_1 + V_2 \)\end{icloze} называют \begin{icloze}{1}прямой сум\-мой,\end{icloze} если \begin{icloze}{2}\( V_1 \cap V_2 = \{ 0 \} \).\end{icloze}
\end{note}

\begin{note}{90c98477312541878454fb9689685fc8}
    \begin{icloze}{2}Прямая сумма подпространств \( V_1 \) и \( V_2 \)\end{icloze} обозначается \begin{icloze}{1}
        \[
            V_1 \oplus V_2.
        \]
    \end{icloze}
\end{note}

\begin{note}{951dc5cc9d7d4722ac40423e92273c7a}
    Пусть \( V_1 \) и \( V_2 \) --- два линейных подпространства. Тогда эквивалентны следующие утверждения:
    \begin{enumerate}
        \item {}\begin{icloze}{1}\( V_1 + V_2 \) --- прямая сумма;\end{icloze}
        \item {}\begin{icloze}{2}\( \dim (V_1 + V_2) = \dim V_1 + \dim V_2 \);\end{icloze}
        \item {}\begin{icloze}{3}Для любого \( a \in V_1 + V_2 \) разложение разложение \( a \) в сумму \( v_1 + v_2 \), где \( v_1 \in V_1, v_2 \in V_2 \), единственно.\end{icloze}
    \end{enumerate}
\end{note}

\begin{note}{fc93fb548c854d70af3f9cf3017866cb}
    В чем основная идея доказательства того, что если для любого \( a \in V_1 + V_2 \) разложение разложение \( a \) в сумму \( v_1 + v_2 \), где \( v_1 \in V_1, v_2 \in V_2 \), единственно, то \( V_1 + V_2 \) --- прямая сумма?

    \begin{cloze}{1}
        Показать, что если \( a = \underset{\in V_1}{v_1} + \underset{\in V_2}{v_2} \in V_1 \cap V_2 \), то \( v_1 = v_2 = 0 \).
    \end{cloze}
\end{note}

\begin{note}{78239c298e504fa9841235fdd06ac419}
    \subsubsection{<<\begin{icloze}{3}Монотонность размерности подпространств\end{icloze}>>}

    Пусть \( W \) --- линейное пространство, \( V \triangleleft W \). Тогда
    \begin{enumerate}
        \item {}\begin{icloze}{1}\( \dim V \leqslant \dim W \),\end{icloze}
        \item {}\begin{icloze}{2}\( \dim V = \dim W \iff V = W \).\end{icloze}
    \end{enumerate}
\end{note}

\begin{note}{a6b854ec7f5b4473a76276e0bff1e272}
    \begin{icloze}{3}Отображение \( f : V \to W \)\end{icloze} называется \begin{icloze}{2}линейным отображением,\end{icloze} если \begin{icloze}{1}
        \begin{enumerate}
            \item \( f(x + y) = f(x) + f(y), \quad \forall x, y \in V \),
            \item \( f(\lambda x) = \lambda f(x), \quad \forall \lambda \in \mathbb R, x \in V \).
        \end{enumerate}
    \end{icloze}
\end{note}

\begin{note}{4008d3f9d2224ec38cb2e9b8a78aab64}
    Линейное отображение так же ещё называют \begin{icloze}{1}линейным оператором.\end{icloze}
\end{note}

\begin{note}{df5862f6f1d4456cb943a7f07c8d8b68}
    Линейный оператор \( f : V \to W \) называется \begin{icloze}{1}и\-зо\-мор\-физм\-ом линейных пространств\end{icloze} тогда и только тогда, когда \begin{icloze}{2}\( f \) --- биекция.\end{icloze}
\end{note}

\begin{note}{d8bd78dfda034119ae049b476da96449}
    Линейные пространства \( V \) и \( W \) называются \begin{icloze}{1}и\-зо\-мор\-фны\-ми\end{icloze} тогда и только тогда, когда \begin{icloze}{2}
        существует изоморфизм
        \[
            f : V \to W.
        \]
    \end{icloze}
\end{note}

\begin{note}{2d4f456313e24261b688216f4b7f199e}
    Отношение \begin{icloze}{2}изоморфности\end{icloze} обозначается символом \begin{icloze}{1}
        \[
            \simeq
        \]
    \end{icloze}
\end{note}

\begin{note}{7112c4ddaf614005b6a37c3f4fbd3edc}
    Если \( f : V \to W \) --- изоморфизм, то \( f^{-1} : W \to V \) \begin{icloze}{1}--- тоже изоморфизм.\end{icloze}
\end{note}

\begin{note}{b439505227ea4814b084a811815b59d3}
    Отношение изоморфности удовлетворяет аксиомам отношения \begin{icloze}{1}эквивалентности.\end{icloze}
\end{note}

\begin{note}{9fa02b16e5e74fcea192355d84b99109}
    Пусть \( V, W \) --- конечномерные линейные пространства. Тогда
    \[
        \begin{icloze}{2}V \simeq W\end{icloze} \begin{icloze}{3}\iff\end{icloze} \begin{icloze}{1}\dim V = \dim W.\end{icloze}
    \]
\end{note}

\begin{note}{13b90eb2ff704cc69e067a3f047966cc}
    Пусть \( f : V \to W \) --- линейный оператор. Тогда \begin{icloze}{2}матрицей линейного оператора \( f \) в паре базисов в \( V \) и \( W \) со\-от\-вет\-ствен\-но\end{icloze} называют \begin{icloze}{1}матрицу \( A \), переводящую координаты любого вектора \( v \in V \) в координаты вектора \( f(v) \in W \) в соответствующих базисах.\end{icloze}
\end{note}

\begin{note}{74ef91d29ce940f8b894341a5836c812}
    Пусть \( f : V \to W \) --- линейный оператор.
    \begin{icloze}{2}Матрица оператора \({ f }\) в паре базисов \({ e, \tilde e }\) в пространствах \({ V }\) и \({ W }\) соответственно\end{icloze} обозначается
    \begin{icloze}{1}
        \[
            M_{e,\tilde e} (f).
        \]
    \end{icloze}
\end{note}

\begin{note}{d8ecf4d0e7a546668528944588ba6060}
    \subsubsection{<<\begin{icloze}{2}Теорема о матрице линейного оператора\end{icloze}>>}
    Пусть \( f : V \to W \) --- линейный оператор,
    \begin{icloze}{3}\( \left\{ e_i \right\}_{i = 1}^{n} \)\end{icloze} --- базис в \( V \),
    \begin{icloze}{3}\( \left\{ \tilde e_j \right\}_{j = 1}^{m} \)\end{icloze} --- базис в \( W \).
    Как в явном виде задать матрицу оператора \( f \) в этих базисах?

    \begin{cloze}{1}
        \( i \)-ый столбец --- это координаты \( f(e_i) \) в базисе \( \left\{ \tilde e_j \right\} \).
    \end{cloze}
\end{note}

\begin{note}{1235d9dc6038426387ee1c7475309a4f}
    Как можно компактно перефразировать утверждение теоремы о матрице линейного оператора?

    \begin{cloze}{1}
        \[
            f(e) = \tilde e A.
        \]
    \end{cloze}
\end{note}

\begin{note}{8e1ba2b68d414caeb7d229ba34833e8d}
    В чем ключевая идея доказательства теоремы о матрице линейного оператора?

    \begin{cloze}{1}
        \[
            f(e\lambda) = f(e)\lambda = \tilde e A \lambda,
        \]
        где \( \lambda \) --- координаты вектора из \( V \) в базисе \( e \).
    \end{cloze}
\end{note}

\begin{note}{b595ad9b198f46299eb5af10d49e413d}
    Композиция линейных операторов --- тоже \begin{icloze}{1}линейный оператор.\end{icloze}
\end{note}

\begin{note}{c13a12af79d9432ab1df0d1bab6f905c}
    Матрица композиции линейных операторов есть \begin{icloze}{1}произведение матриц этих операторов.\end{icloze}
\end{note}

\section{Лекция 21.02.22}
\begin{note}{13db7f12a2e14ffca2f5e09197cd3e07}
    Пусть \( f : V \to W \) --- линейный оператор,  \( A \) --- матрица оператора \( f \) в базисах \( e \) и \( \tilde e \) соответственно. Как преобразуется матрица \( A \) при замене базисов \( e \to e', \tilde e \to \tilde e'? \)

    \begin{cloze}{1}
        \[
            A' = C^{-1}_{\tilde e \to \tilde e'} \; A \; C_{e \to e'}.
        \]
    \end{cloze}
\end{note}

\begin{note}{015e02c15f134a53b50a24729fb6ac3d}
    Пусть \( f : V \to V \) --- линейный оператор,  \( A \) --- матрица оператора \( f \) в базисе \( e \). Как преобразуется матрица \( A \) при замене базиса \( e \to e' \)?

    \begin{cloze}{1}
        \[
            A' = C^{-1}_{e \to e'} \; A \; C_{e \to e'}.
        \]
    \end{cloze}
\end{note}

\begin{note}{e3c3292adefb4657a177843c8840476d}
    Пусть \( f : V \to V \) --- линейный оператор, \( A \) и \( A' \) --- матрицы оператора \( f \) в двух базисах \( e \) и \( e' \) соответственно.
    Тогда \( \det A' = \begin{icloze}{1}\det A\end{icloze} \).
\end{note}

\begin{note}{79b8fed369c447dfb53f352258ed6940}
    \begin{icloze}{2}Определителем оператора \( f : V \to V \)\end{icloze} называется \begin{icloze}{1}о\-пре\-де\-ли\-тель матрицы оператора \( f \) в произвольном базисе.\end{icloze}
\end{note}

\begin{note}{79b8fed369c447dfb53f352258ed6940}
    \begin{icloze}{2}Рангом оператора \( f : V \to V \)\end{icloze} называется \begin{icloze}{1}ранг матрицы оператора \( f \) в произвольном базисе.\end{icloze}
\end{note}

\begin{note}{d36be29fb7a342599a7f73709043bb1f}
    \begin{icloze}{2}След матрицы \( A \)\end{icloze} обозначается \begin{icloze}{1}\( \operatorname{tr}  A \).\end{icloze}
\end{note}

\begin{note}{3c423489fc4f422aaa906fbcc2041ec3}
    Пусть \( A \in \begin{icloze}{3}\mathbb R ^{n \times n}\end{icloze} \). Тогда \( \displaystyle \begin{icloze}{2}\operatorname{tr} A\end{icloze} \overset{\text{def}}= \begin{icloze}{1}\sum_{i=1}^{n} a_{ii}\end{icloze} \).
\end{note}

\begin{note}{e0b3b870a8444704a8569d15e3f761ed}
    Пусть \({ A, B \in \mathbb R^{n \times n} }\). Тогда
    \[
        \operatorname{tr} (BA) = \begin{icloze}{1}\operatorname{tr} (AB).\end{icloze}
    \]
\end{note}

\begin{note}{55e76656e4fc4920969acdfb57634355}
    \begin{icloze}{2}Следом оператора \( f : V \to V \)\end{icloze} называется \begin{icloze}{1}след матрицы оператора \( f \) в произвольном базисе.\end{icloze}
\end{note}

\begin{note}{1da0c4fffac341f89821707b4a1b38a6}
    Пусть \( f : V \to W \) --- линейный оператор. Тогда
    \[
        \begin{icloze}{2}\ker f\end{icloze} \overset{\text{def}}= \begin{icloze}{1}f^{-1}(\left\{ 0 \right\}).\end{icloze}
    \]
\end{note}

\begin{note}{f8fe0ceb74f84386932c4100743fb775}
    Пусть \( f : V \to W \) --- линейный оператор. Тогда
    \[
        \begin{icloze}{2}\operatorname{im} f\end{icloze} \overset{\text{def}}= \begin{icloze}{1}f(V).\end{icloze}
    \]
\end{note}

\begin{note}{56a80e8376154f29b490e470ceac8bc3}
    Пусть \( f : V \to W \) --- линейный оператор. Можно ли утверждать, что всегда \( \ker f \triangleleft V \)?

    \begin{cloze}{1}
        Да, поскольку линейная комбинация нулей \( f \) --- тоже нуль \( f \).
    \end{cloze}
\end{note}

\begin{note}{28f55b0f2daa4b35b1859196e2d41ede}
    Пусть \( f : V \to W \) --- линейный оператор. Можно ли утверждать, что всегда \( \ker f \triangleleft W \)?

    \begin{cloze}{1}
        Нет, \( \ker f \triangleleft V \).
    \end{cloze}
\end{note}

\begin{note}{a4bde4e9272d4bef89c915f6390ca148}
    Пусть \( f : V \to W \) --- линейный оператор. Можно ли утверждать, что всегда \( \operatorname{im} f \triangleleft W \)?

    \begin{cloze}{1}
        Да, поскольку \( \forall f(u), f(v) \in \operatorname{im} f \)
        \[
            \alpha f(u) + \beta f(v) = f(\alpha u + \beta v) \in \operatorname{im} f.
        \]
    \end{cloze}
\end{note}

\begin{note}{7b17eb03a5e640f8bddefa0aaa6656c3}
    Пусть \( f : V \to W \) --- линейный оператор. Можно ли утверждать, что всегда \( \operatorname{im} f \triangleleft V \)?

    \begin{cloze}{1}
        Нет, \( \operatorname{im} f \triangleleft W \).
    \end{cloze}
\end{note}

\begin{note}{5c7bf3d386eb4fa181cdb696fc0f9ab5}
    Пусть \( f : V \to W \) --- линейный оператор. Как связаны размерности \( V \), \( \ker f \) и \( \operatorname{im} f \)?

    \begin{cloze}{1}
        \[
            \dim \ker f + \dim \operatorname{im} f = \dim V.
        \]
    \end{cloze}
\end{note}

\begin{note}{b6ef54a20af44801aceb30b556b95011}
    Пусть \( f : V \to W \) --- линейный оператор. В чем основная идея доказательства следующей формулы?
    \[
        \dim \ker f + \dim \operatorname{im} f = \dim V
    \]

    \begin{cloze}{1}
        Дополнить базис в \( \ker f \) до базиса в \( V \) и построить из них базис в \( \operatorname{im} f \).
    \end{cloze}
\end{note}

\begin{note}{26a0af100d5b4c459a74ba6384b7c554}
    Пусть \( f : V \to W \) --- линейный оператор,
    \begin{itemize}
        \item \( e_1,e_2, \ldots, e_k  \) --- базис в \( \ker f \);
        \item \( e_1,e_2, \ldots, e_k, e_{k+1}, \ldots, e_n  \) --- базис в \( V \).
    \end{itemize}
    Как выглядит базис в \( \operatorname{im} f \)?

    \begin{cloze}{1}
        \[
            f(e_{k+1}), \ldots, f(e_n).
        \]
    \end{cloze}
\end{note}

\begin{note}{8a962591377f49c1a6b297a1efe008e9}
    Пусть \( f : W \to W \) --- линейный оператор. Тогда
    \[
        \begin{icloze}{2}\operatorname{rk} f\end{icloze} = \begin{icloze}{1}\dim \operatorname{im} f.\end{icloze}
    \]

    \begin{center}
        \tiny (в терминах размерностей)
    \end{center}
\end{note}

\begin{note}{2acbea4466f54360bc19e2065a44fc95}
    Пусть \( f : W \to W \) --- линейный оператор. Как показать, что
    \[
        \operatorname{rk} f = \dim \operatorname{im} f.
    \]

    \begin{cloze}{1}
        Показать, что в координатном выражении \( \operatorname{im} f \) есть линейная оболочка столбцов матрицы оператора \( f \).
    \end{cloze}
\end{note}

\begin{note}{a85a7d7b1e3d47939cc717cb8da889ac}
    Пусть \( f : W \to W \) --- линейный оператор. \begin{icloze}{1}Пространство \( V \triangleleft W \)\end{icloze} называется \begin{icloze}{2}инвариантным относительно оператора \( f \),\end{icloze} если
    \begin{icloze}{1}\[
        f(V) \subset V.
    \]\end{icloze}
\end{note}

\begin{note}{e3d31c73908d4103b6c9caf2377e4432}
    Примеры инвариантных подпространств в контексте произвольного оператора \( f : W \to W \).

    \begin{cloze}{1}
        \[
            \ker f, \operatorname{im} f.
        \]
    \end{cloze}
\end{note}

\begin{note}{e64a247c0efb47f8be38d4ab4ef17b05}
    Пусть \( f : W \to W \) --- линейный оператор,  \( e_1, \ldots, e_n  \) --- \begin{icloze}{4}такой базис в \({ W }\), что \({ e_1, \ldots, e_k }\) (где \({ k \leqslant n }\)) --- базис в инвариантном подпространстве \({ V \triangleleft W }\).\end{icloze} Тогда \begin{icloze}{3}матрица оператора \( f \) в базисе \( e_1, \ldots, e_n  \)\end{icloze} примет вид
    \[
        A = \begin{icloze}{1}
            \begin{bmatrix}
                T_{11} & T_{12} \\
                0 & T_{22}
            \end{bmatrix},
        \end{icloze}
    \]
    где \( T_{11}  \) --- это \begin{icloze}{2}матрица \( f |_{V}  \) в базисе \( e_1, \ldots, e_k \).\end{icloze}
\end{note}

\section{Лекция 28.02.22}
\begin{note}{9932dc2853764661928eedc8d44ddd74}
    Линейный оператор \( f : W \to W \) называется \begin{icloze}{2}невырожденным,\end{icloze} если \begin{icloze}{1}\( \det f \neq 0 \).\end{icloze}
\end{note}

\begin{note}{2e565e676da342fb8cdacf4d62de05e8}
    Пусть \( f : V \to V \) --- линейный оператор. Следующие 5 условий эквивалентны:
    \begin{enumerate}
        \item \( f \) невырождено;
        \begin{icloze}{1}
            \item \( \ker f = \left\{ 0 \right\} \);
            \item \( \operatorname{im} f = V \);
            \item \( \operatorname{rk} f = \dim V \);
            \item \( f \) --- биекция.
        \end{icloze}
    \end{enumerate}
\end{note}

\begin{note}{8f9f5108ac8847299f21fd40619c6612}
    Пусть \( f : W \to W \) --- линейный оператор. Как доказать, что если \( f \) --- невырожденный оператор, то \( f \) --- биекция?

    \begin{cloze}{1}
        Показать, что если \( f \) задаётся матрицей \( A \), то \( f^{-1} \) задаётся матрицей \( A^{-1} \).
    \end{cloze}
\end{note}

\begin{note}{0c8915aebdc24427ab211efa79c6e07a}
    Пусть \( f : W \to W \) --- линейный оператор. Как доказать, что если \( f \) --- биекция, то \( f \) --- невырожденный оператор.

    \begin{cloze}{1}
        \[
            \det (f \circ f^{-1}) = |E| \implies \det f \neq 0.
        \]
    \end{cloze}
\end{note}

\begin{note}{198b26e615c745edbd313c2f62029546}
    Пусть \begin{icloze}{3}\( f : V \to V \) --- линейный оператор.\end{icloze} Тогда \begin{icloze}{1}число \( \lambda \in \mathbb C  \)\end{icloze} называется \begin{icloze}{2}собственным значением оператора \( f \),\end{icloze} если
    \begin{icloze}{1}
        \[
            \exists v \in V \setminus \left\{ 0 \right\} \quad f(v) = \lambda v.
        \]
    \end{icloze}
\end{note}

\begin{note}{f0b8dcb8a69748a0a51393ae495884b4}
    Пусть \begin{icloze}{3}\( f : V \to V \) --- линейный оператор.\end{icloze} Тогда \begin{icloze}{1}вектор \( v \in V \setminus \left\{ 0 \right\} \)\end{icloze} называется \begin{icloze}{2}собственным вектором оператора \( f \),\end{icloze} если
    \begin{icloze}{1}
        \[
            \exists \lambda \in \mathbb C \quad f(v) = \lambda v.
        \]
    \end{icloze}
\end{note}

\begin{note}{22a614bf26ea4db3ae297b5c647e6517}
    \begin{icloze}{2}Спектром оператора\end{icloze} называется \begin{icloze}{1}множество собственных значений этого оператора.\end{icloze}
\end{note}

\begin{note}{1f331a6bd4c84dc4996f323fd40b5a22}
    \begin{icloze}{2}Спектр\end{icloze} оператора \( f \) обозначается \begin{icloze}{1}\( \operatorname{spec} f \).\end{icloze}
\end{note}

\begin{note}{ff82c9b056384c19b0a176b637c3941c}
    Пусть \begin{icloze}{3}\( f : V \to V \) --- линейный оператор,  \( \lambda \in \mathbb C  \).\end{icloze} Тогда \( \lambda \) является собственным значением \( f \) \begin{icloze}{2}тогда и только тогда, когда\end{icloze}
    \begin{icloze}{1}
        \[
            \det (f - \lambda E) = 0.
        \]
    \end{icloze}
\end{note}

\begin{note}{a96c7b61477946699a72e8a792c8bf75}
    Пусть \begin{icloze}{3}\( f : V \to V \) --- линейный оператор.\end{icloze} Тогда \begin{icloze}{2}уравнение
    \[
        \det (f - \lambda E) = 0
    \]
    \end{icloze} называется \begin{icloze}{1}характеристическим уравнением оператора \( f \).\end{icloze}
\end{note}

\begin{note}{a7a86475fc014d3c8fe1d63fa3a766ea}
    Пусть \begin{icloze}{3}\( f : V \to V \) --- линейный оператор.\end{icloze} Тогда \begin{icloze}{2}выражение
    \[
        \det (f - \lambda E)
    \]
    \end{icloze} называется \begin{icloze}{1}характеристическим многочленом оператора \( f \).\end{icloze}
\end{note}

\begin{note}{976ac89d4ea7486080b6c2c8473946d9}
    Пусть \( f : V \to V \) --- линейный оператор. Почему
    \[
        \det (f - \lambda E)
    \]
    является многочленом переменной \( \lambda \)?

    \begin{cloze}{1}
        Если \( A \) --- матрица оператора \( f \), то \( \left| A - \lambda E \right|  \) --- многочлен переменной \( \lambda \).
    \end{cloze}
\end{note}

\begin{note}{5376672e8b21438896bc774aa4ac2275}
    Пусть
    \[
        A = \begin{bmatrix}
            a_{11} & a_{12} \\
            a_{21} & a_{22}
        \end{bmatrix}.
    \]
    Тогда
    \[
        \begin{icloze}{2}\left| A - \lambda E \right|\end{icloze}
        = \begin{icloze}{1}|A| - \lambda \operatorname{tr} A + \lambda^2.\end{icloze}
\]
\end{note}

\section{Лекция 07.03.22}
\begin{note}{0d6c679eb377462e90e8ac9bba29dd61}
    Пусть \( f : W \to W \) --- линейный оператор.
    \begin{icloze}{2}Характеристический многочлен оператора \( f \)\end{icloze} обозначается \begin{icloze}{1}
    \[
        \chi_f.
    \]
\end{icloze}
\end{note}

\begin{note}{78106143b649485eb1c075b2388eb22e}
    Пусть \begin{icloze}{3}\( f : W \to W \) --- линейный оператор и \( V \triangleleft W \) инвариантно относительно \( f \).\end{icloze}
    Тогда
    \begin{center}
        \begin{icloze}{2}\( \chi_{f|_V} \)\end{icloze} --- \begin{icloze}{1}делитель \( \chi_f \).\end{icloze}
    \end{center}
\end{note}

\begin{note}{6deeef304fd8465bbff331e4241bde67}
    Пусть \( f : W \to W \) --- линейный оператор и \( V \triangleleft W \) инвариантно относительно \( f \).
    Тогда
    \begin{center}
        \( \chi_{f|_V} \) --- делитель \( \chi_f \).
    \end{center}

    В чем основная идея доказательства?

    \begin{cloze}{1}
        Показать, что \( \chi_f \) --- определитель соответствующей квазитреугольной матрицы оператора \( f \).
    \end{cloze}
\end{note}

\begin{note}{785c107694984499a5fd89afd052841c}
    Пусть \( f : W \to W \) --- линейный оператор, \( \lambda \in \operatorname{spec} f \).
    Тогда \begin{icloze}{2}множество всех собственных векторов \( f \), отвечающих собственному значению \( \lambda \), объединённое с нулём,\end{icloze} называется \begin{icloze}{1}собственным подпространством оператора \( f \), отвечающим собственному значению \({ \lambda }\).\end{icloze}
\end{note}

\begin{note}{cdb0a7bde4e044e48a5a798a8052f163}
    Пусть \( f : W \to W \) --- линейный оператор, \( \lambda \in \operatorname{spec} f \).
    \begin{icloze}{1}Собственное подпространство \({ f }\), отвечающее собственному значению \({ \lambda }\),\end{icloze} обозначается \begin{icloze}{2}\( V_{f} (\lambda) \).\end{icloze}
\end{note}

\begin{note}{545e4fc3988d45fdafc099f74fe38f36}
    Пусть \( f : W \to W \) --- линейный оператор, \( \lambda \) --- собственное значение \( f \).
    В кратком выражении
    \[
        \begin{icloze}{2}V_f(\lambda)\end{icloze} \overset{\text{def}}= \begin{icloze}{1}\ker (f - \lambda E).\end{icloze}
    \]
\end{note}

\begin{note}{edf7cad1b7df422181105ad8bf31a210}
    Пусть \( f : W \to W \) --- линейный оператор, \( \lambda \) --- собственное значение \( f \).
    Всегда ли
    \[
        V_{f} (\lambda) \triangleleft W?
    \]

    \begin{cloze}{1}
        Да, всегда, потому что \( V_{f} (\lambda) = \ker (f - \lambda E) \).
    \end{cloze}
\end{note}

\begin{note}{de964305c22b4993819a8d5095504e53}
    Пусть \( f : V \to V \) --- линейный оператор, \( \lambda \) --- собственное значение \( f \).
    \begin{icloze}{1}Размерность \( V_f (\lambda) \)\end{icloze} называют \begin{icloze}{2}геометрической кратностью собственного значения \( \lambda \).\end{icloze}
\end{note}

\begin{note}{f6b8139d2f0e46d38a2dd075ff83b2f4}
    Пусть \( f : V \to V \) --- линейный оператор, \( \lambda \) --- собственное значение \( f \).
    \begin{icloze}{2}Геометрическая кратность собственного значения \( \lambda \)\end{icloze} обозначается \begin{icloze}{1}\( S_f (\lambda) \).\end{icloze}
\end{note}

\begin{note}{eff6d05e42b34f078450044f6153939b}
    Пусть \( f : V \to V \) --- линейный оператор, \( \lambda \) --- собственное значение \( f \).
    \begin{icloze}{1}Кратность \( \lambda \) как корня \( \chi_f \)\end{icloze} называют \begin{icloze}{2}алгебраической кратностью собственным значением \( \lambda \).\end{icloze}
\end{note}

\begin{note}{856a933db82641cd87b0ee5f34647b1a}
    Пусть \( f : V \to V \) --- линейный оператор, \( \lambda \) --- собственное значение \( f \).
    \begin{icloze}{2}Алгебраическая кратность собственного значения \( \lambda \)\end{icloze} обозначается \begin{icloze}{1}\( m_f (\lambda) \).\end{icloze}
\end{note}

\begin{note}{b7431a88515043deacf49cf7fdb735c6}
    Пусть \( f : V \to V \) --- линейный оператор, \( \lambda \) --- собственное значение \( f \).
    Тогда \begin{icloze}{1}\( S_f (\lambda) \leqslant m_f (\lambda) \).\end{icloze}
\end{note}

\begin{note}{6b913f908a194114bee71fb9a7526282}
    Пусть \( f : V \to V \) --- линейный оператор, \( \lambda \) --- собственное значение \( f \).
    Тогда \( S_f (\lambda) \leqslant m_f (\lambda) \).

    В чем основная идея доказательства?

    \begin{cloze}{1}
        \( V_f (\lambda) \) инвариантно относительно \( f \) \\
        \phantom{} \hfill \( \implies \) \( \chi_f \) делится на \( \chi_{f|_{V_f(\lambda)}} \).
    \end{cloze}
\end{note}

\begin{note}{58579b404ae34478b736df96c853c6e6}
    Пусть \( f : V \to V \) --- линейный оператор, \( \lambda \) --- собственное значение \( f \), \: \begin{icloze}{2}\( \tilde f = f|_{V_f(\lambda)} \).\end{icloze}
    Тогда
    \[
        \begin{icloze}{3}\chi_{\tilde f} (t)\end{icloze} = \begin{icloze}{1}(\lambda - t)^{S_f(\lambda)}\end{icloze}
    \]
\end{note}

\begin{note}{8d63ff53045545709809018e1492b231}
    Пусть \( f : V \to V \) --- линейный оператор, \( \lambda \) --- собственное значение \( f \), \: \( \tilde f = f|_{V_f(\lambda)} \).
    Откуда следует, что
    \[
        \chi_{\tilde f} (t) = (\lambda - t)^{S_f(\lambda)} \quad?
    \]

    \begin{cloze}{1}
        \( \tilde f \) представляется  матрицей \( \lambda E \) порядка \( \dim V_f(\lambda) \).
    \end{cloze}
\end{note}

\begin{note}{a3b9ba1c4e884a7bb1e3c4764f063d1f}
    \begin{icloze}{2}Оператор \( f : x \mapsto \lambda x \), где \( \lambda \in \mathbb R \),\end{icloze} называется \begin{icloze}{1}скалярным оператором.\end{icloze}
\end{note}

\begin{note}{51a455604c9c4d7eadc3fe5ab0af6397}
    Пусть \begin{icloze}{3}\( f : V \to V \) --- линейный оператор.\end{icloze}
    \( f \) называется \begin{icloze}{2}диагонализуемым оператором,\end{icloze} если \begin{icloze}{1}существует базис в \( V \), в котором матрица оператора \( f \) является диагональной.\end{icloze}
\end{note}

\begin{note}{b01b69fc3ebc4c0c839a0c153f85d041}
    \begin{icloze}{1}Диагональная матрица с элементами \( a_1, a_2, \ldots, a_n \) на диагонали\end{icloze} обозначается
    \begin{icloze}{2}
        \[
            \operatorname{diag}(a_1, a_2, \ldots, a_n).
        \]
    \end{icloze}
\end{note}

\begin{note}{8066b576097a49fb9d5aa3c4580a27c5}
    Пусть \( f : V \to V \)--- линейный оператор.
    Если в базисе \( e_1, e_2, \ldots, e_n \) матрица оператора \( f \) равна \( \operatorname{diag} (a_1, a_2, \ldots, a_n) \), то \begin{icloze}{2}\( e_1, e_2, \ldots, e_n \)\end{icloze} --- \begin{icloze}{1}собственные векторы \( f \).\end{icloze}
\end{note}

\begin{note}{19e6a7fb9c8e4f04a3711d479f2c628e}
    Пусть \( f : V \to V \)--- линейный оператор.
    Если в базисе \( e_1, e_2, \ldots, e_n \) матрица оператора \( f \) равна \( \operatorname{diag} (a_1, a_2, \ldots, a_n) \), то \begin{icloze}{2}\( a_1, a_2, \ldots, a_n \)\end{icloze} --- \begin{icloze}{1}собственные значения \( f \).\end{icloze}
\end{note}

\begin{note}{1176411a2bf147348b94dd69b9bbad73}
    Пусть \begin{icloze}{4}\( f : V \to V \) --- линейный оператор.\end{icloze} Тогда оператор \( f \) \begin{icloze}{2}диагонализуем\end{icloze} \begin{icloze}{3}тогда и только тогда, когда\end{icloze} \begin{icloze}{1}для любого собственного значения \( \lambda \)
    \[
        S_f(\lambda) = m_f(\lambda).
    \]\end{icloze}
\end{note}

\begin{note}{ca827a11abb047fda276763e1e593ef1}
    В чем основная идея доказательства критерия диагонализуемости оператора (необходимость)?

    \begin{cloze}{1}
        Покзать, что если \( f \) представляется матрицей \( \operatorname{diag} (a_1, a_2, \ldots, a_n) \), то по определению
        \[
            \chi_f (\lambda) = \prod_{i = 1}^{n} (a_i - \lambda).
        \]
    \end{cloze}
\end{note}

\begin{note}{0fc84832f2b548cfa9a4ef9a51326b77}
    Пусть \( f : V \to V \) --- линейный оператор, \begin{icloze}{3}\( \lambda_1, \ldots, \lambda_n \) --- различные собственные значения оператора \( f \),\end{icloze} \begin{icloze}{2}
        \[
            \forall j \quad v_j \in V_f (\lambda_j).
        \]
    \end{icloze}
    Тогда \begin{icloze}{1}система векторов \( v_1, \ldots, v_n \) линейно независима.\end{icloze}
\end{note}

\begin{note}{2a1e5294e5c34d889ca747ab0b44fa0a}
    Пусть \( f : V \to V \) --- линейный оператор, \( \lambda_1, \ldots, \lambda_n \) --- различные собственные значения оператора \( f \),
    \[
        \forall j \quad v_j \in V_f (\lambda_j).
    \]
    Тогда система векторов \( v_1, \ldots, v_n \) линейно независима.

    В чем основная идея доказательства?

    \begin{cloze}{1}
         Применяем \( f \) к произвольной равной нулю линейной комбинации, пока не получится СЛАУ с основной матрицей --- определителем Вандермонда.
    \end{cloze}
\end{note}

\begin{note}{cfe344113f4e40b2b27ecfee11beb647}
    В чем основная идея доказательства критерия диагонализуемости оператора (достаточность)?

    \begin{cloze}{1}
        Составить систему векторов из базисов в \( V_f (\lambda_j) \) и показать, что она является базисом \( V \).
    \end{cloze}
\end{note}

\begin{note}{fbb72d710ce84fe6b5237ee1f15112a8}
    Почему система векторов, составленная в доказательстве критерия диагонализуемости оператора (достаточность), является порождающей?

    \begin{cloze}{1}
        Из условия \( \dim V_f (\lambda_j) = m_f (\lambda_j) \), а значит система содержит \( \deg \chi_f = \dim V \) элементов.
    \end{cloze}
\end{note}

\begin{note}{5fd54902e7f34d00bad3222902a6bdf6}
    Почему система векторов, составленная в доказательстве критерия диагонализуемости оператора (достаточность), является линейно независимой?

    \begin{cloze}{1}
        Любая её линейная комбинация есть линейная комбинация системы векторов \( v_1, \ldots, v_n \), где \( v_j \in V_f (\lambda_j) \).
    \end{cloze}
\end{note}

\begin{note}{435490ce764048d9a55b762d6175cf59}
    Если оператор \( f : V  \to V \) имеет \( \dim V \) различных собственных значений, то \begin{icloze}{1}\( f \) диагонализуем.\end{icloze}
\end{note}

\begin{note}{8757ff57337847268575f5903d640f08}
    Как доказать, что если оператор \( f : V \to V \) имеет \( \dim V \) различных собственных значений, то \( f \) диагонализуем.

    \begin{cloze}{1}
        \begin{multline*}
            \forall \lambda \in \operatorname{spec} f \quad 1 \leqslant S_f(\lambda) \leqslant m_f(\lambda) = 1 \\
            \implies  S_f (\lambda) = m_f (\lambda).
        \end{multline*}
    \end{cloze}
\end{note}

\begin{note}{b7cd455d24424dd0879b90d7cad89a6b}
    Пусть \begin{icloze}{3}пространство \( V = V_1 \oplus V_2 \).\end{icloze}
    \begin{icloze}{1}Оператор
    \[
        P : \underset{\in V_1}{v_1} + \underset{\in V_2}{v_2} \mapsto v_1, \quad V \to V
    \]
    \end{icloze} называется \begin{icloze}{2}оператором проектирования на \( V_1 \) параллельно \( V_2 \).\end{icloze}
\end{note}

\begin{note}{522c1911d5d04c898b070c53537026b2}
    Пусть \( V = V_1 \oplus V_2 \) и \( P : V \to V \) --- оператор проектирования на \( V_1 \) параллельно \( V_2 \).
    Тогда
    \[
        \operatorname{im} P = \begin{icloze}{1}V_1.\end{icloze}
    \]
\end{note}

\begin{note}{0e8f1308502e41f9bfbddc3a9a153514}
    Пусть \( V = V_1 \oplus V_2 \) и \( P : V \to V \) --- оператор проектирования на \( V_1 \) параллельно \( V_2 \).
    Тогда
    \[
        \ker P = \begin{icloze}{1}V_2.\end{icloze}
    \]
\end{note}

\begin{note}{27181bd7474e4091aee4fa9dba20ae0f}
    Пусть \( V = V_1 \oplus V_2 \) и \( P : V \to V \) --- оператор проектирования на \( V_1 \) параллельно \( V_2 \).
    Тогда
    \[
        \operatorname{spec} P = \begin{icloze}{1}\left\{ 0, 1 \right\}.\end{icloze}
    \]
\end{note}

\begin{note}{448f428dbef544a9a7ad66228e473bea}
    Пусть \( V = V_1 \oplus V_2 \) и \( P : V \to V \) --- оператор проектирования на \( V_1 \) параллельно \( V_2 \).
    Тогда
    \[
        m_P (0) = \begin{icloze}{1}\dim V_2.\end{icloze}
    \]
\end{note}

\begin{note}{d4a2a9780d1a4e1db35238e91f3875b9}
    Пусть \( V = V_1 \oplus V_2 \) и \( P : V \to V \) --- оператор проектирования на \( V_1 \) параллельно \( V_2 \).
    Тогда
    \[
        S_P (0) = \begin{icloze}{1}\dim V_2.\end{icloze}
    \]
\end{note}

\begin{note}{322376ccf5e4418bb64b5e8b886d8aac}
    Пусть \( V = V_1 \oplus V_2 \) и \( P : V \to V \) --- оператор проектирования на \( V_1 \) параллельно \( V_2 \).
    Тогда
    \[
        m_P (1) = \begin{icloze}{1}\dim V_1.\end{icloze}
    \]
\end{note}

\begin{note}{c81e19cdfaa649f18565d2f7625646ce}
    Пусть \( V = V_1 \oplus V_2 \) и \( P : V \to V \) --- оператор проектирования на \( V_1 \) параллельно \( V_2 \).
    Тогда
    \[
        S_P (1) = \begin{icloze}{1}\dim V_1.\end{icloze}
    \]
\end{note}

\section{Лекция 14.03.22}
\begin{note}{d32917879c284285842d17bbfc251d30}
    Пусть \begin{icloze}{3}\( f : V \to V \) --- линейный оператор, \( v \in V \), \( k \in \mathbb N \).\end{icloze} Вектор \( v \) называется \begin{icloze}{2}корневым вектором высоты \( k \) оператора \( f \),\end{icloze} если \begin{icloze}{1}существует такое \( \lambda \in \mathbb C \), что
    \[
        \begin{gathered}
            (f - \lambda E)^{k} v = 0, \\
            (f - \lambda E)^{k - 1} v \neq 0.
        \end{gathered}
    \]\end{icloze}
\end{note}

\begin{note}{83d2e0cc0a894b54ac4d3604babf2d57}
    Корневой вектор высоты \begin{icloze}{2}\( 1 \)\end{icloze} оператора \( f \) --- это \begin{icloze}{1}собственный вектор этого оператора.\end{icloze}
\end{note}

\begin{note}{9e3747b6754c4bad9076277f39c4e920}
    \( \lambda \) из определения корневого вектора оператора \( f \) --- это всегда \begin{icloze}{1}собственное значение \( f \).\end{icloze}
\end{note}

\begin{note}{a4093e0c9f55478ebd2eb2defda323df}
    Как показать, что \( \lambda \) из определения корневого вектора всегда является собственным значением?

    \begin{cloze}{1}
        Из определения \( (f - \lambda E)^{k} v = 0 \implies \det (f - \lambda E) = 0 \).
    \end{cloze}
\end{note}

\begin{note}{999c7f68724546db81750f9e997d0a1b}
    Пусть \begin{icloze}{3}\( v \) --- корневой вектор высоты \( k \geqslant 2 \) оператора \( f \).\end{icloze}
    Тогда \begin{icloze}{2}\( (f - \lambda E) v \)\end{icloze} --- \begin{icloze}{1}корневой вектор высоты \( k - 1 \).\end{icloze}
\end{note}

\begin{note}{264901faf0bb401e91105512f04f06dc}
    Пусть \( v \) --- корневой вектор высоты \( k \geqslant 2 \) оператора \( f \).
    Тогда \( (f - \lambda E) v \) --- корневой вектор высоты \( k - 1 \).
    В чем основная идея доказательства?

    \begin{cloze}{1}
        Из определения корневого вектора
        \[
            (f - \lambda E)^{k - 1} \cdot (f - \lambda E)v = 0
        \]
        и аналогично с неравенством нулю для степени \( k - 2 \).
    \end{cloze}
\end{note}

\begin{note}{50c2388c1fa843dfa616f85d4cecfa2f}
    Система \begin{icloze}{3}корневых векторов разных высот,\end{icloze} отвечающих \begin{icloze}{2}одному и тому же собственному значению оператора,\end{icloze} \begin{icloze}{1}линейно независима.\end{icloze}
\end{note}

\begin{note}{de47eb56e219455a8497a97ad90b861d}
    Как доказать, что система корневых векторов разных высот, отвечающих одному и тому же собственному значению оператора, линейно независима.

    \begin{cloze}{1}
        Приравнять линейную комбинацию к нулю и домножать её на \( (f - \lambda E)^{k_j - 1} \) в порядке убывания высот \( k_j \) корневых векторов системы.
    \end{cloze}
\end{note}

\begin{note}{187218f20c2b46ab9309b3385f2012f4}
    Пусть \({ f : V \to V }\) --- линейный оператор,\: \begin{icloze}{3}\( v \) --- корневой вектор высоты \( k \) оператора \( f \).\end{icloze}
    Тогда система
    \begin{icloze}{2}
        \[
            v,\: (f - \lambda E) v,\: (f - \lambda E)^{2} v,\: \ldots,\: (f - \lambda E)^{k - 1} v
        \]
    \end{icloze}
    \begin{icloze}{1}линейно независима.\end{icloze}
\end{note}

\begin{note}{f77f36f44a0a4dbfb7fe6d8a6b58db75}
    Пусть \( v \) --- корневой вектор высоты \( k \) оператора \( f \).
    Тогда система
    \[
        v,\: (f - \lambda E) v,\: (f - \lambda E)^{2} v,\: \ldots,\: (f - \lambda E)^{k - 1} v
    \]
    линейно независима.
    В чем основная идея доказательства?

    \begin{cloze}{1}
        Показать, что это система корневых векторов разных высот, отвечающих одному и тому же собственному значению \( \lambda \).
    \end{cloze}
\end{note}

\begin{note}{3ab579b8e03a47ec865a43fc21bd39b7}
    Система \begin{icloze}{3}корневых векторов,\end{icloze} отвечающих \begin{icloze}{2}разным собственным значениям оператора,\end{icloze} \begin{icloze}{1}линейно независима.\end{icloze}
\end{note}

\begin{note}{04c77a5799504d088141691461b44095}
    Пусть \( v \) --- корневой вектор высоты \( k \) оператора \( f \). Тогда \begin{icloze}{2}\( (f - \lambda E)^{k - 1}v \)\end{icloze} --- \begin{icloze}{1}это собственный вектор оператора \( f \).\end{icloze}
\end{note}

\begin{note}{59e9653333744cccaf670372a881ab06}
    Как доказать, что система корневых векторов, отвечающих разным собственным значениям оператора, линейно независима.

    \begin{cloze}{1}
        Домножить произвольную линейную комбинацию на
        \[
            (f - \lambda_1 E)^{k_1 - 1} \;\; (f - \lambda_2 E)^{k_2} \cdots (f - \lambda_l E)^{k_l}
        \]
        и получить равенство нулю первого коэффициента. Далее аналогично для остальных коэффициентов.
    \end{cloze}
\end{note}

\begin{note}{5b16ae3e6ef643508aa2e1f086ffde51}
    Пусть \( f : V \to V \) --- линейный оператор, \( \lambda \in \operatorname{spec} f \).
    \begin{icloze}{1}Множество всех корневых векторов, отвечающих собственному значению \( \lambda \), объединённое с нулём,\end{icloze} называется \begin{icloze}{2}корневым подпространством, отвечающим собственному значению \( \lambda \).\end{icloze}
\end{note}

\begin{note}{2779025573314db7aa326077599c90b3}
    Пусть \( f : V \to V \) --- линейный оператор.
    \begin{icloze}{2}Корневое подпространство, отвечающее собственному значению \( \lambda \),\end{icloze} обозначается \begin{icloze}{1}\( K_f (\lambda) \).\end{icloze}
\end{note}

\begin{note}{d70f06c975144b6e835736be38336c4a}
    Пусть \( f : V \to V \) --- линейный оператор, \( \lambda \in \operatorname{spec} f \).
    Всегда ли \( K_f(\lambda) \triangleleft V \)?

    \begin{cloze}{1}
        Да, всегда (тривиально следует из определения).
    \end{cloze}
\end{note}

\begin{note}{e3330d597cd547a385f694495c2dc291}
    Пусть \begin{icloze}{3}\( f : V \to V \) --- линейный оператор, \( k \in \mathbb N \).\end{icloze}
    \[
        \begin{icloze}{2}N_{f,k}(\lambda)\end{icloze} \overset{\text{def}}= \begin{icloze}{1}\ker (f - \lambda E)^{k}.\end{icloze}
    \]
\end{note}

\begin{note}{42d32fc206824eafb2be52cb821ffafd}
    Пусть \( f : V \to V \) --- линейный оператор, \( k \in \mathbb N \).
    Всегда ли \( N_{f,k}(\lambda) \triangleleft V \)?

    \begin{cloze}{1}
        Да, всегда (тривиально следует из определения).
    \end{cloze}
\end{note}

\begin{note}{ba89f8d6240947edac91e39df44d92bc}
    Пусть \( f : V \to V \) --- линейный оператор, \( \lambda \in \operatorname{spec} f \).
    Как \( K_f(\lambda) \) выражается через \( N_{f,k} (\lambda) \)?

    \begin{cloze}{1}
        \[
            K_f(\lambda) = \bigcup_{k}^{} N_{f,k}(\lambda)
        \]
    \end{cloze}
\end{note}

\begin{note}{c11610dbf64143fbaeeb57dfc3d66af0}
    Пусть \( f : V \to V \) --- линейный оператор, \( \lambda \in \operatorname{spec} f \).
    Тогда \( \dim K_f(\lambda) = \begin{icloze}{1}m_f(\lambda)\end{icloze} \).
\end{note}

\begin{note}{efee3536114a40d28eb925c540f796bf}
    Пусть \( f : V \to V \) --- линейный оператор, \( \lambda \in \operatorname{spec} f \).
    Тогда \( \dim K_f(\lambda) = m_f(\lambda) \).
    В чем основная идея доказательства?

    TODO (?)
\end{note}

\begin{note}{d928d6cded3a434a9c3f615c48190ff0}
    Пусть \( f : V \to V \) --- линейный оператор, \begin{icloze}{3}\( \lambda_1, \ldots, \lambda_l \) --- все различные собственные значения \( f \).\end{icloze}
    Тогда
    \[
        \begin{icloze}{2}V\end{icloze} = \begin{icloze}{1}K_f (\lambda_1) \oplus \cdots \oplus K_f (\lambda_l).\end{icloze}
    \]
\end{note}

\begin{note}{16e24ba8ea07492581171c4ee92a6c95}
    Пусть \( f : V \to V \) --- линейный оператор, \( \lambda_1, \ldots, \lambda_l \) --- все различные собственные значения \( f \).
    Тогда
    \[
        V = K_f (\lambda_1) \oplus \cdots \oplus K_f (\lambda_l).
    \]
    Какова общая структура доказательства?

    \begin{cloze}{1}
        Показать, что сумма \( K_f(\lambda_j) \)
        \begin{enumerate}
            \item является прямой,
            \item порождает все пространство \( V \).
        \end{enumerate}
    \end{cloze}
\end{note}

\begin{note}{12b8b0705d6b4daf886d155d26b8d4f4}
    Пусть \( f : V \to V \) --- линейный оператор, \( \lambda_1, \ldots, \lambda_l \) --- все различные собственные значения \( f \).
    Тогда
    \[
        V = K_f (\lambda_1) \oplus \cdots \oplus K_f (\lambda_l).
    \]
    Почему сумма \( K_f(\lambda_j) \) прямая?

    \begin{cloze}{1}
        Линейная комбинация векторов \( v_j \) из \( K_f (\lambda_j) \) --- это линяния комбинация корневых векторов, отвечающих разным собственным значениям.
    \end{cloze}
\end{note}

\begin{note}{ae4ac17697d146c194afc0f17091b028}
    Пусть \( f : V \to V \) --- линейный оператор, \( \lambda_1, \ldots, \lambda_l \) --- все различные собственные значения \( f \).
    Тогда
    \[
        V = K_f (\lambda_1) \oplus \cdots \oplus K_f (\lambda_l).
    \]
    Почему сумма \( K_f(\lambda_j) \) порождает все \( V \)?

    \begin{cloze}{1}
        \[
            \sum_{j=1}^{l} \dim K_f(\lambda_j) = \sum_{j=1}^{l} m_f(\lambda_j)
        \]
    \end{cloze}
\end{note}

\begin{note}{e23c324999e1436d8c6d50a246244d60}
    \begin{icloze}{2}Жорданова клетка\end{icloze} --- это \begin{icloze}{1}квадратная матрица вида
    \[
        \begin{bmatrix}
            \lambda & 1 & 0 & \cdots & 0 \\
            0 & \lambda & 1 & \cdots & 0 \\
            0 & 0 & \lambda & \cdots & 0 \\
            \vdots & \vdots & \vdots & \ddots & \vdots \\
            0 & 0 & 0 & \cdots & \lambda
        \end{bmatrix}.
    \]\end{icloze}
\end{note}

\begin{note}{d354e3255a1a46e99261a422c4e41207}
    Жорданова клетка высоты \( q \), соответствующая некоторому числу \( \lambda \), обозначается
    \begin{icloze}{1}
        \[
            J_q (\lambda).
        \]
    \end{icloze}
\end{note}

\begin{note}{49446743b36c41b2825ed009c2fe6cd6}
    \begin{icloze}{2}Жорданова матрица\end{icloze} --- это \begin{icloze}{1}блочно-диагональная матрица, составленная из жордановых клеток.\end{icloze}
\end{note}

\begin{note}{c2e8392343e8487288fc8b5d700aeafa}
    Пусть \( f : V \to V \) --- линейный оператор.
    Тогда, если \begin{icloze}{1}в некотором базисе в \( V \) матрица \( A \) оператора \( f \) имеет жорданов вид,\end{icloze} то \( A \) называют \begin{icloze}{2}жордановой нормальной формой оператора \( f \).\end{icloze}
\end{note}

\begin{note}{4e0cce0726054534a2f4f1fa1beaffbb}
    Пусть \( f : V \to V \) --- линейный оператор.
    Тогда, если \begin{icloze}{1}в некотором базисе в \( V \) матрица оператора \( f \) имеет жорданов вид,\end{icloze} то этот базис называют \begin{icloze}{2}жордановым базисом оператора \( f \).\end{icloze}
\end{note}

\begin{note}{4617ac459f3846a1b581c79a9c044b7e}
    \subsubsection{<<\begin{icloze}{2}Теорема о жордановой нормальной форме\end{icloze}>>}

    \begin{icloze}{1}Любой оператор в векторном пространстве над полем \({ \mathbb C }\) имеем жорданову нормальную форму.\end{icloze}
\end{note}

\begin{note}{d8f181b2d5004a47bd308a35849cddec}
    Пусть \( f : V \to V \) --- линейный оператор, \( \lambda \in \operatorname{spec} f \).
    Как для \( k > 0 \) соотносятся \( N_{f,k} (\lambda) \) и \( N_{f, k+1} (\lambda) \)?

    \begin{cloze}{1}
        Для всех \( k \) меньше некоторого \( q \)
        \[
            N_{f, k} (\lambda) \subsetneqq N_{f, k+1} (\lambda),
        \]
        а для всех \( k \geqslant q \):
        \[
            N_{f, k} (\lambda) = N_{f, k+1} (\lambda)
        \]
    \end{cloze}
\end{note}

\begin{note}{414400f8f69b41b58c7d5b2930735317}
    Каков первый шаг в построении жордановой нормальной формы оператора \( f : V \to V \)?

    \begin{cloze}{1}
        Найти все собственные значения оператора \( f \).
    \end{cloze}
\end{note}

\begin{note}{a79be36515f64439b4db0f075099cbc3}
    Каков второй шаг в построении жордановой нормальной формы оператора \( f : V \to V \)?

    \begin{cloze}{1}
        Для каждого собственного значения \( \lambda \) найти все подпространства \( N_{f, k} (\lambda) \).
    \end{cloze}
\end{note}

\begin{note}{adf2c488db4640a1aba232fba8286d63}
    Каков третий шаг в построении жордановой нормальной формы оператора \( f : V \to V \)?

    \begin{cloze}{1}
        Построить жорданову лестницу в каждом из корневых подпространств \( f \).
    \end{cloze}
\end{note}

\begin{note}{2fe8afa7a09b49a1a7219ce868aaf67e}
    Каков заключительный шаг в построении жордановой нормальной формы оператора \( f : V \to V \)?

    \begin{cloze}{1}
        Объединить все построенные базисы в одну систему и построить матрицу \( f \) в полученном базисе.
    \end{cloze}
\end{note}

\section{Лекция 21.03.22}
\begin{note}{61582b48320a46c3ad047eec84da3eb3}
    Пусть \( A, A' \in \mathbb C^{\begin{icloze}{3}n \times n\end{icloze}} \). Тогда матрицы \( A \) и \( A' \) называются \begin{icloze}{2}подобными,\end{icloze} если \begin{icloze}{1}существует невырожденная матрица \( T \) такая, что
    \[
        A = T \: A' \: T^{-1}.
    \]\end{icloze}
\end{note}

\begin{note}{6366e6bbaa1149eb8bba346a3cc38654}
    Отношение подобия матриц обозначается символом
    \begin{icloze}{1}
        \[
            \sim
        \]
    \end{icloze}
\end{note}

\begin{note}{1ae63106d8d0480b82ef6f9e9b3d62bb}
    Подобие матриц является отношением \begin{icloze}{1}эквивалентности.\end{icloze}
\end{note}

\begin{note}{de743729325e43f79f35a7b8c22d5bb2}
    Любая \begin{icloze}{2}квадратная матрица\end{icloze} подобна \begin{icloze}{1}своей жордановой нормальной форме.\end{icloze}

    \begin{center}
        \tiny (следствие из \begin{icloze}{3}теоремы о жордановой форме\end{icloze})
    \end{center}
\end{note}

\begin{note}{82aa01fcbfb7476d84662ca5802dae5b}
    \begin{icloze}{2}Две квадратные матрицы подобны\end{icloze} \begin{icloze}{3}тогда и только тогда, когда\end{icloze} \begin{icloze}{1}их жордановы формы совпадают с точностью до перестановки клеток.\end{icloze}

    \begin{center}
        \tiny (следствие из \begin{icloze}{4}теоремы о жордановой форме\end{icloze})
    \end{center}
\end{note}

\begin{note}{198e1f3eef67411c89f83a35ade066d2}
    Пусть \( A, \Lambda, T \in \mathbb C^{n \times n} \),\: \( A = T^{-1} \Lambda T \),\: \( k \in \mathbb N \). Тогда
    \[
        A^{k} = \begin{icloze}{1}T^{-1} \Lambda^{k} T.\end{icloze}
    \]
\end{note}

\begin{note}{c1cf4048c475426683c811de00771765}
    Пусть \({ A \in \mathbb C^{n \times n} }\),\: \({ p \in \mathbb C[x] }\),\: \( \displaystyle p(x) = \sum_{k=0}^{n} a_k x^{k}. \)
    Тогда
    \[
        p(A) \overset{\text{def}}= \begin{icloze}{1}\sum_{k=0}^{n} a_k A^{k}, \quad \text{где \({ A^{0} \overset{\text{def}}= E }\)}.\end{icloze}
    \]
\end{note}

\begin{note}{59cb3566c41d4eca89ef63e626740c4e}
    Пусть \({ A, T \in \mathbb C^{n \times n} }\),\: \({ \det T \neq 0 }\),\: \({ p \in \mathbb C[x] }\). Тогда
    \[
        p(T A T^{-1}) = \begin{icloze}{1}T \: p(A) \: T^{-1}.\end{icloze}
    \]
\end{note}

\begin{note}{ad579382cf8a42caabf0b8b6a5a4d76f}
    Пусть \({ f : D \subset \mathbb C \to \mathbb C }\), \({ \lambda \in D }\).
    \[
        f(\lambda E) \overset{\text{def}}= \begin{icloze}{1}f(\lambda) E.\end{icloze}
    \]
\end{note}

\begin{note}{be2002dbe01149aa91e229d1c991143e}
    Пусть \({ f : D \subset \mathbb C \to \mathbb C }\),
    \[
        A = \begin{bmatrix}
            A_{11} & 0 \\
            0 & A_{22}
        \end{bmatrix}
        \in \mathbb C^{n \times n}.
    \]
    Тогда
    \[
        f(A) \overset{\text{def}}= \begin{icloze}{1}\begin{bmatrix}
            f(A_{11}) & 0 \\
            0 & f(A_{22})
        \end{bmatrix}.\end{icloze}
    \]
\end{note}

\begin{note}{455a3d16cf6744b39c1d1e21cab4e7f5}
    Пусть \({ f : D \subset \mathbb C \to \mathbb C }\), \({ \lambda \in D }\).
    Как определяют значение
    \[
        f(J_k(\lambda))?
    \]

    \begin{cloze}{1}
        Представляют \({ f(J_k(\lambda)) }\) как \({ f(\lambda E + \varepsilon) }\) и далее используют разложение \({ f }\) в ряд Тейлора в точке \({ \lambda E }\).
    \end{cloze}
\end{note}

\begin{note}{c435657fd33d4705ae2de65b4bf5c682}
    Пусть \({ f : D \subset \mathbb C \to \mathbb C }\), \({ \lambda \in D }\).
    Для каких \({ k }\) и \({ \lambda }\) определено значение \({ f(J_k (\lambda)) }\)?

    \begin{cloze}{1}
        Должен существовать многочлен \({ T_{\lambda, k} f }\).
   \end{cloze}
\end{note}

\begin{note}{3450a4591ff748cb856f4578b3cda3c2}
    Пусть \({ p \in \mathbb C[x] }\),\: \({ A \in \mathbb C^{n \times n} }\). \begin{icloze}{1}Многочлен \( p \)\end{icloze} называется \begin{icloze}{2}аннулирующим многочленом для матрицы \( A \),\end{icloze} если
    \begin{icloze}{1}
        \[
            p(A) = 0.
        \]
    \end{icloze}
\end{note}

\begin{note}{34b1edb015384033870e10717e8bbdb2}
    \subsubsection{<<\begin{icloze}{2}Теорема Гамильтона-Кэли\end{icloze}>>}

    \begin{icloze}{1}Характеристический многочлен квадратной матрицы является для неё аннулирующим.\end{icloze}
\end{note}

\begin{note}{07bbead6e007486e93d2daa598a265b6}
    В чем ключевая идея доказательства теоремы Гамильтона-Кэли?

    \begin{cloze}{1}
        Для любого корневого вектора \({ x }\) имеем \({ \chi_{\text{\tiny $A$}}(A) \: x = 0 }\).
    \end{cloze}
\end{note}

\section{Лекция 28.03.22}
\begin{note}{c4787ae5340942d2a27db89ea5f9d4df}
    Пусть \({ V }\) --- линейное пространство над \({ \mathbb R }\).
    Билинейная форма \({ f }\) в \({ V }\) называется \begin{icloze}{2}положительно определённой,\end{icloze} если \begin{icloze}{1}для любого \({ v \in V }\)
    \[
        \begin{gathered}
            f(v, v) \geqslant 0; \quad f(v, v) = 0 \iff v = 0.
        \end{gathered}
    \]\end{icloze}
\end{note}

\begin{note}{18f442014f0e4614a642e429958b8931}
    Пусть \({ V }\) --- линейное пространство над \({ \mathbb R }\).
    \begin{icloze}{2}Скалярным произведением в \({ V }\)\end{icloze} называется \begin{icloze}{1}симметричная положительно определённая билинейная форма в \({ V }\).\end{icloze}
\end{note}

\begin{note}{cea78871e8124a29945d3540057c0c68}
    \begin{icloze}{2}Евклидовым пространством\end{icloze} называется \begin{icloze}{1}вещественное линейное пространство с заданным на нём скалярным произведением\end{icloze}.
\end{note}

\begin{note}{79a607edba4945a4a562d9b1fd8f2ce9}
    Пусть \({ V }\) --- евклидово пространство над \({ \mathbb R }\).
    Скалярное произведение векторов \({ v, w \in V }\) обозначается
    \begin{icloze}{1}
        \[
            (v, w).
        \]
    \end{icloze}
\end{note}

\begin{note}{717ab493f110448bb867a49b37d29d83}
    Пусть \({ V }\) --- евклидово пространство над \({ \mathbb R }\),\: \({ v \in V }\).
    \begin{icloze}{2}Длиной вектора \({ v }\)\end{icloze} называется \begin{icloze}{1}величина \( \sqrt{(v, v)} \).\end{icloze}
\end{note}

\begin{note}{7bc89a880fb244a78c3e204575ac9005}
    Пусть \({ V }\) --- евклидово пространство над \({ \mathbb R }\),\: \({ v \in V }\).
    \begin{icloze}{2}Длина вектора \({ v }\)\end{icloze} обозначается \begin{icloze}{1}\( \left\lvert v \right\rvert \) или \({ \left\lVert v \right\rVert }\).\end{icloze}
\end{note}

\begin{note}{de4db3a6688f4b198b8238b0e07dfce7}
    Длину вектора в еклидовом пространства так же ещё называют \begin{icloze}{1}нормой этого вектора.\end{icloze} В таком случае чаще используется обозначение \begin{icloze}{2}\( \left\lVert v \right\rVert \).\end{icloze}
\end{note}

\begin{note}{c0b109c4be9e4749ad794e9e38fffb2d}
    Пусть \({ V }\) --- евклидово пространство над \({ \mathbb R }\),\: \({ v_0 \in V }\),\: \({ \begin{icloze}{3}r \in \mathbb R_+\end{icloze} }\).
    \begin{icloze}{2}Сферой радиуса \({ r }\) с центром в точке \({ v_0 }\)\end{icloze} называют \begin{icloze}{1}множество
    \[
        \left\{ v \in V \mid \left\lVert v - v_0 \right\rVert = r \right\}.
    \]
\end{icloze}\end{note}

\begin{note}{09b61a41cf5f45109c79e7cc61f63740}
    Пусть \({ V }\) --- евклидово пространство над \({ \mathbb R }\),\: \({ v_0 \in V }\),\: \({ r \in \mathbb R_+ }\).
    \begin{icloze}{2}Сфера радиуса \({ r }\) с центром в точке \({ v_0 }\)\end{icloze} обозначается
    \begin{icloze}{1}
        \[
            S_r (v_0).
        \]
    \end{icloze}
\end{note}

\begin{note}{e63df21bb26d42269a7a5d45c6b828b8}
    Пусть \({ V }\) --- евклидово пространство над \({ \mathbb R }\),\: \({ v_0 \in V }\),\: \({ \begin{icloze}{3}r \in \mathbb R_+\end{icloze} }\).
    \begin{icloze}{2}Шаром радиуса \({ r }\) с центром в точке  \({ v_0 }\)\end{icloze} называют \begin{icloze}{1}множество
        \[
            \left\{ v \in V | \left\lVert v - v_0 \right\rVert \leqslant r \right\}.
        \]
    \end{icloze}
\end{note}

\begin{note}{d0d10cbbdb664b428b1f3284ff5321f9}
    Пусть \({ V }\) --- евклидово пространство над \({ \mathbb R }\),\: \({ v_0 \in V }\),\: \({ r \in \mathbb R_+ }\).
    \begin{icloze}{2}Шар радиуса \({ r }\) с центром в точке \({ v_0 }\)\end{icloze} обозначается
    \begin{icloze}{1}
        \[
            B_r (v_0).
        \]
    \end{icloze}
\end{note}

\begin{note}{a7021008185a411e99300286ac245d14}
    Пусть \({ V }\) --- евклидово пространство над \({ \mathbb R }\),\: \begin{icloze}{3}\({ v, w \in V \setminus \left\{ 0 \right\} }\).\end{icloze}
    Векторы \({ v }\) и \({ w }\) называются \begin{icloze}{2}сонаправленными,\end{icloze} если
    \begin{icloze}{1}
        \[
            \exists \lambda > 0 \quad v = \lambda w.
        \]
    \end{icloze}
\end{note}

\begin{note}{0cfd3b2d9f17418eb0b8fd2dd36ef1d4}
    Пусть \({ V }\) --- евклидово пространство над \({ \mathbb R }\),\: \begin{icloze}{3}\({ v, w \in V \setminus \left\{ 0 \right\} }\).\end{icloze}
    \begin{icloze}{2}Углом между векторами \({ v, w }\)\end{icloze} называется \begin{icloze}{1}угол \({ \varphi \in \left[ 0, \pi \right] }\) такой, что
    \[
        \cos \varphi = \frac{(v, w)}{\left\lVert v \right\rVert \cdot \left\lVert w \right\rVert}.
    \]\end{icloze}
\end{note}

\begin{note}{097fc51b1eab4a699e7110a38f0bd670}
    \subsubsection{<<\begin{icloze}{2}Неравенство Коши-Буняковского\end{icloze}>>}

    Пусть \({ V }\) --- евклидово пространство над \({ \mathbb R }\),\: \begin{icloze}{3}\({ v, w \in V }\).\end{icloze}
    Тогда всегда \begin{icloze}{1}\({ \left\lvert (v, w) \right\rvert \leqslant \left\lVert v \right\rVert \cdot \left\lVert w \right\rVert }\).\end{icloze}
\end{note}

\begin{note}{570b086e7e1b48e3b3012778f4841d1e}
    В чем основная идея доказательства неравенства Коши-Буняковского?

    \begin{cloze}{1}
        Оценить дискриминант квадратного уравнения \({ \left\lVert v - \lambda w \right\rVert^2 = 0 }\) относительно неизвестной \({ \lambda }\).
    \end{cloze}
\end{note}

\begin{note}{96bb9d37dba3499d8890f7b3eb1f04d4}
    Пусть \({ V }\) --- евклидово пространство над \({ \mathbb R }\),\: \({ v, w \in V }\).
    Тогда
    \begin{center}
        \({ \left\lvert (v, w) \right\rvert = \left\lVert v \right\rVert \cdot \left\lVert w \right\rVert }\)
        \begin{icloze}{2}\({ \iff }\)\end{icloze}
        \begin{icloze}{1}\({ v }\) и \({ w }\) пропорциональны.\end{icloze}
    \end{center}
\end{note}

\begin{note}{d1941ac59ee44c82b045d6d1e954e0d8}
    \subsubsection{<<\begin{icloze}{2}Неравенство треугольника\end{icloze}>>}

    Пусть \({ V }\) --- евклидово пространство над \({ \mathbb R }\),\: \begin{icloze}{3}\({ v, w \in V }\).\end{icloze}
    Тогда
    \begin{icloze}{1}
        \[
            \left\lVert v + w \right\rVert \leqslant \left\lVert v \right\rVert + \left\lVert w \right\rVert.
        \]
    \end{icloze}
\end{note}

\begin{note}{4759501bf4b84cf0acf58f945229396c}
    В чем основная идея доказательства неравенства треугольника?

    \begin{cloze}{1}
    Рассмотреть скалярное произведение
    \[
        (v + w, v + w) = \left\lVert v + w \right\rVert^2.
    \]
    \end{cloze}
\end{note}

\begin{note}{5378eb0c9d81404c9cd8ca40925b9ce8}
    Пусть \({ V }\) --- евклидово пространство над \({ \mathbb R }\),\: \({ v, w \in V }\).
    Тогда
    \begin{center}
        \({ \left\lVert v + w \right\rVert = \left\lVert v \right\rVert + \left\lVert w \right\rVert }\)
        \begin{icloze}{2}\({ \iff }\) \end{icloze}
        \begin{icloze}{1}\({ v \operatorname{\uparrow\uparrow} w }\)\end{icloze}
    \end{center}
\end{note}

\begin{note}{8238aebbcc724e708990b61d8a0e3603}
    Пусть \({ V }\) --- евклидово пространство над \({ \mathbb R }\),\: \({ v, w \in V }\).
    Векторы \({ v }\) и \({ w }\) называются \begin{icloze}{2}ортогональными,\end{icloze} если \begin{icloze}{1}\({ (v, w) = 0 }\).\end{icloze}
\end{note}

\begin{note}{ce138d9eefe6445bbe72ecb3cafe43e8}
    Пусть \({ V }\) --- евклидово пространство над \({ \mathbb R }\).
    Система векторов в \({ V }\) называется \begin{icloze}{2}ортогональной,\end{icloze} если \begin{icloze}{1}её векторы попарно ортогональны.\end{icloze}
\end{note}

\begin{note}{2dbaa8c8157c42e08de67ebd6cc42e47}
    Пусть \({ V }\) --- евклидово пространство над \({ \mathbb R }\),\: \begin{icloze}{4}\({ \left\{ e_j \right\}_{j = 1}^{n} }\) --- ортогональная система векторов в \({ V }\).\end{icloze}
    Тогда \begin{icloze}{3}система \({ \left\{ e_j \right\} }\) линейно независима\end{icloze} \begin{icloze}{2}\({ \iff }\)\end{icloze} \begin{icloze}{1}\({ e_j \neq 0 }\) для всех \({ j }\).\end{icloze}
\end{note}

\begin{note}{d20a32cfc1c3440a9e22f5d28c36b9d5}
    Пусть \({ V }\) --- евклидово пространство над \({ \mathbb R }\),\: \({ \left\{ e_j \right\}_{j = 1}^{n} }\) --- ортогональная система ненулевых векторов в \({ V }\).
    Как показать, что система \({ \left\{ e_j \right\} }\) линейно независима?

    \begin{cloze}{1}
        Умножить линейную комбинацию векторов \({ \left\{ e_j \right\} }\), равную нулю, на \({ e_i }\) для произвольного \({ i }\) и показать равентсво нулю \({ i }\)-ого коэффициента.
    \end{cloze}
\end{note}

\begin{note}{b9cf4cdf374445c4bc8412c8ca72847c}
    Пусть \({ V }\) --- евклидово пространство над \({ \mathbb R }\),\: \begin{icloze}{3}\({ v \in V }\),\: \({ \left\{ e_j \right\}_{j = 1}^{n} }\) --- ортогональный базис в \({ V }\).\end{icloze}
    Тогда \begin{icloze}{2}координаты вектора \({ v }\) в базисе \({ \left\{ e_j \right\} }\)\end{icloze} имеют вид
    \[
        v_j = \begin{icloze}{1}\frac{(v, e_j)}{\left\lVert e_j \right\rVert^2}.\end{icloze}
    \]
\end{note}

\begin{note}{5a4e71f923b84eb5b5f3e2b66ea26470}
    Пусть \({ V }\) --- евклидово пространство над \({ \mathbb R }\),\: \({ v \in V }\),\: \({ \left\{ e_j \right\}_{j = 1}^{n} }\) --- ортогональный базис в \({ V }\).
    Как показать, что координаты вектора \({ v }\) в базисе \({ \left\{ e_j \right\} }\) имеют вид
    \[
        v_j = \frac{(v, e_j)}{\left\lVert e_j \right\rVert^2}?
    \]

    \begin{cloze}{1}
        Вычислить \({ (v, e_j) }\), разложив \({ v }\) по базису \({ \left\{ e_j \right\} }\).
    \end{cloze}
\end{note}

\begin{note}{7ede17a5d2d049c690090d4850f4ef60}
    Пусть \({ V }\) --- евклидово пространство над \({ \mathbb R }\),\: \begin{icloze}{3}\({ v \in V }\),\: \({ \left\{ e_j \right\}_{j = 1}^{n} }\) --- ортогональная линейно независима система в \({ V }\).\end{icloze}
    Тогда \begin{icloze}{1}числа
    \[
        \frac{(v, e_j)}{\left\lVert e_j \right\rVert^2}
    \]\end{icloze}
    называют \begin{icloze}{2}коэффициентами Фурье вектора \({ v }\) в системе \({ \left\{ e_j \right\} }\).\end{icloze}
\end{note}

\begin{note}{04027e25ae114e76a8cf6f9500e1ae28}
    Пусть \({ V }\) --- евклидово пространство над \({ \mathbb R }\).
    Система векторов \({ \left\{ e_j \right\}_{j = 1}^{n} }\) в \({ V }\) называется \begin{icloze}{2}ортонормированной,\end{icloze} если \begin{icloze}{1}её векторы попарно ортогональны и \({ \left\lVert e_j \right\rVert = 1 }\) для всех \({ j }\).\end{icloze}
\end{note}

\section{Лекция 04.04.22}
\begin{note}{48fccb0908e94a3bbfc768f249c233a4}
    Пример ортогональной системы в пространстве \({ C[0, 2\pi] }\) со скалярным произведением
    \[
        (f, g) \overset{\text{def}}= \int_{0}^{2\pi} f g.
    \]

    \begin{cloze}{1}
        \[
            1, \cos x, \sin x, \cos 2x, \sin 2x, \ldots, \cos nx, \sin nx.
        \]
    \end{cloze}
\end{note}

\begin{note}{25230410b91d47619feafd9dd1e3909e}
    \subsubsection{<<\begin{icloze}{3}Ортогонализация Грама-Шмидта\end{icloze}>>}

    Пусть \begin{icloze}{2}\({ V }\) --- евклидово пространство, \({ e_1, \ldots, e_n }\) --- базис в пространстве \({ V }\).\end{icloze}
    Тогда \begin{icloze}{1}всегда существует ортогональный базис \({ a_1, \ldots, a_n }\) в \({ V }\) такой, что
    \[
        a_j \in \mathscr L (e_1, \ldots, e_j) \quad \forall j.
    \]\end{icloze}
\end{note}

\begin{note}{89394003d65441209a81ec6be5c7f2df}
    В чем основная идея доказательства истинности теоремы об ортогонализации Грама-Шмидта?

    \begin{cloze}{1}
        Положить
        \begin{align*}
            a_1 &= e_1, \\
            a_2 &= e_2 + \alpha_1 a_1, \\
            a_3 &= e_3 + \beta_1 a_1 + \beta_2 a_2 \\
                &\ldots
        \end{align*}
    \end{cloze}
\end{note}

\begin{note}{067af76850ea49929f538a99ef2fb445}
    Пусть \begin{icloze}{3}\({ W }\) --- евклидово пространство, \({ V \triangleleft W }\).\end{icloze}
    \begin{icloze}{1}
        Множество
        \[
            \Big\{ w \in W \mid (v, w) = 0 \quad \forall v \in V \Big\}
        \]
    \end{icloze}
    называется \begin{icloze}{2}ортогональным дополнением к \({ V }\).\end{icloze}
\end{note}

\begin{note}{dc34194cc9a642aeb10ad2ba1cbab7ad}
    Пусть \({ W }\) --- евклидово пространство, \({ V \triangleleft W }\).
    \begin{icloze}{1}Ортогональное дополнение к пространству \({ V }\)\end{icloze} обозначается \begin{icloze}{2}\({ V^{\perp} }\).\end{icloze}
\end{note}

\begin{note}{800460fc49ee4f3b915a92addaba5141}
    Пусть \({ W }\) --- евклидово пространство, \({ V \triangleleft W }\).
    Всегда ли \({ V^{\perp} \triangleleft W }\)?

    \begin{cloze}{1}
        Да, всегда.
    \end{cloze}
\end{note}

\begin{note}{ab8d62b25a294edebe7a3735b84dab19}
    Пусть \({ W }\) --- евклидово пространство, \({ V \triangleleft W }\).
    Тогда
    \[
        \dim V^{\perp} = \begin{icloze}{1}\dim W - \dim V.\end{icloze}
    \]
\end{note}

\begin{note}{70166548d05745278d7a8f9de584d211}
    Пусть \({ W }\) --- евклидово пространство, \({ V \triangleleft W }\).
    Тогда
    \[
        V + V^{\perp} = \begin{icloze}{1}V \oplus V^{\perp} = W.\end{icloze}
    \]
\end{note}

\begin{note}{eee9a5f3a40047629e2192983ab08770}
    Пусть \({ W }\) --- евклидово пространство, \({ V \triangleleft W }\).
    Как показать, что \({ W = V \oplus V^{\perp} }\)?

    \begin{cloze}{1}
        Выбрать ортогональный базис в \({ V }\), дополнить его до ортогонального базиса в \({ W }\) и показать, что дополнение --- базис в \({ V^{\perp} }\).
    \end{cloze}
\end{note}

\begin{note}{53d600a53a4f48a7b4d1e3a3822918fe}
    Пусть \({ W }\) --- евклидово пространство, \({ V \triangleleft W }\),
    \({ e_1, \ldots, e_k }\) --- ортогональный базис в \({ V }\), \({ e_1, \ldots, e_n }\) --- ортогональный базис в \({ W }\).
    Как показать, что \({ e_{k + 1}, \ldots, e_n }\) --- базис в \({ V^{\perp} }\)?

    \begin{cloze}{1}
        Показать, что \({ \mathscr L (e_{k + 1}, \ldots, e_n) }\) и \({ V^{\perp} }\) равны как множества.
    \end{cloze}
\end{note}

\begin{note}{a9fc50cec2cc442d87f7f6a551043a18}
    Пусть \({ W }\) --- евклидово пространство, \begin{icloze}{3}\({ V \triangleleft W }\), \({ w \in W }\).\end{icloze}
    Тогда \begin{icloze}{1}проекция \({ w }\) на \({ V }\) параллельно \({ V^{\perp} }\)\end{icloze} называется \begin{icloze}{2}проекцией вектора \({ w }\) на \({ V }\).\end{icloze}
\end{note}

\begin{note}{bcb91b5b6d4048febe0fd4e8da7302e9}
    Пусть \({ W }\) --- евклидово пространство, \begin{icloze}{3}\({ V \triangleleft W }\), \({ w \in W }\).\end{icloze}
    Тогда \begin{icloze}{1}проекция \({ w }\) на \({ V^{\perp} }\) параллельно \({ V }\)\end{icloze} называется \begin{icloze}{2}перпендикуляром, опущенным из \({ w }\) на \({ V }\).\end{icloze}
\end{note}

\begin{note}{4e448e8833f94547ad7848fd34666613}
    Пусть \({ e_1, \ldots, e_k }\) --- система векторов в евклидовом пространстве.
    \begin{icloze}{2}Матрицей Грама системы \({ e_1, \ldots, e_k }\)\end{icloze} называют \begin{icloze}{1}матрицу
    \[
        \Big[ \left( e_i, e_j \right) \Big] \sim k \times k.
    \]\end{icloze}
\end{note}

\begin{note}{3bff6be501ed49109d5041f018ecab96}
    Пусть \({ e_1, \ldots, e_k }\) --- система векторов в евклидовом пространстве.
    \begin{icloze}{2}Матрица Грама системы \({ e_1, \ldots, e_k }\)\end{icloze} обозначается
    \begin{icloze}{1}
        \[
            G(e_1, \ldots, e_k).
        \]
    \end{icloze}
\end{note}

\begin{note}{f45df626ca1d4db1866e3f7aae0c6f2a}
    Пусть \({ W }\) --- евклидово пространство, \({ w \in W }\), \({ e_1, \ldots, e_k }\) --- базис в \({ V \triangleleft W }\).
    Как найти проекцию \({ w_0 }\) вектора \({ w }\) на \({ V }\)?

    \begin{cloze}{1}
        \[
            \begin{gathered}
                G(e_1, \ldots, e_n) \cdot \begin{bmatrix}
                    \alpha_1 \\ \vdots \\ \alpha_k
                \end{bmatrix}
                =
                \begin{bmatrix}
                    (w, e_1) \\ \vdots \\ (w, e_k)
                \end{bmatrix}, \\
                w_0 = e\alpha.
            \end{gathered}
        \]
    \end{cloze}
\end{note}

\section{Лекция 18.04.22}
\begin{note}{b04c0920040847d0a5d99e72e1d5f32f}
    Пусть \({ W }\) --- евклидово пространство, \({ V \triangleleft W }\), \({ f \in W }\).
    \begin{icloze}{2}Расстоянием от точки \({ f }\) до пространства \({ V }\)\end{icloze} называется \begin{icloze}{1}величина
    \[
        \min \left\{ \left\lVert f - g \right\rVert \mid g \in V \right\}.
    \]\end{icloze}
\end{note}

\begin{note}{3c3b40b9167c47199457c3614706c26e}
    Пусть \({ W }\) --- евклидово пространство, \({ V \triangleleft W }\), \({ f \in W }\).
    \begin{icloze}{1}Расстояние от точки \({ f }\) до подпространства \({ V }\)\end{icloze} обозначается
    \begin{icloze}{2}
        \[
            d(f, V).
        \]
    \end{icloze}
\end{note}

\begin{note}{3e4d6f8b3aae4e73806dfc7764e669e3}
    Пусть \({ W }\) --- евклидово пространство, \({ V \triangleleft W }\), \({ f \in W }\),
    \begin{icloze}{3}
        \[
            f = \underset{\in V}{f_0} + \underset{\in V^{\perp}}{f^{\perp}}.
        \]
    \end{icloze}
    Тогда
    \[
        \begin{icloze}{2}d(f, V)\end{icloze} = \begin{icloze}{1}\left\lVert f^{\perp} \right\rVert.\end{icloze}
    \]
\end{note}

\begin{note}{d34143b5a6f347ebbd35b66500be29d0}
    Пусть \({ W }\) --- евклидово пространство, \({ V \triangleleft W }\), \({ f \in W }\).
    В чем  основная идея доказательства равенства \({ d(f, V) = \left\lVert f^{\perp} \right\rVert }\)?

    \begin{cloze}{1}
        Представить \({ f }\) как \({ f^{\perp} + f_0 }\) и явно вычислить \({ \left\lVert f - g \right\rVert }\) для \({ g \in V }\).
    \end{cloze}
\end{note}

\begin{note}{c3802fe76b8740e28ef0bb2ce4d4aca0}
    Пусть \({ W }\) --- евклидово пространство, \({ f, g \in W }\).
    \begin{icloze}{2}Угол между векторами \({ f }\) и \({ g }\)\end{icloze} обозначается
    \begin{icloze}{1}
        \[
            (\widehat{f, g}).
        \]
    \end{icloze}
\end{note}

\begin{note}{d84678ac809d4c229902770a60fb6c11}
    Пусть \({ W }\) --- евклидово пространство, \({ V \triangleleft W }\), \({ f \in W }\).
    \begin{icloze}{2}Углом между вектором \({ f }\) и подпространством \({ V }\)\end{icloze} называется \begin{icloze}{1}величина
    \[
        \min_{g \in V} (\widehat{f, g}).
    \]\end{icloze}
\end{note}

\begin{note}{75d1e7b4a26f4f578bdaf51afa099e06}
    Пусть \({ W }\) --- евклидово пространство, \({ V \triangleleft W }\), \({ f \in W }\).
    \begin{icloze}{2}Угол между вектором \({ f }\) и подпространством \({ V }\)\end{icloze} обозначается
    \begin{icloze}{1}
        \[
            (\widehat{f, V}).
        \]
    \end{icloze}
\end{note}

\begin{note}{19d3715d268b4c07a6ec6013b8a60e50}
    Пусть \({ W }\) --- евклидово пространство, \({ V \triangleleft W }\), \({ f \in W }\),
    \begin{icloze}{3}
        \[
            f = \underset{\in V}{f_0} + \underset{\in V^{\perp}}{f^{\perp}}.
        \]
    \end{icloze}
    Тогда
    \[
        \begin{icloze}{2}(\widehat{f, V})\end{icloze} = \begin{icloze}{1}(\widehat{f, f_0}).\end{icloze}
    \]
\end{note}

\begin{note}{24059676faa84badb5f0199c24f6f28b}
    Пусть \({ W }\) --- евклидово пространство, \({ V \triangleleft W }\), \({ f \in W }\).
    В чем основная идея доказательства равенства \({ (\widehat{f, V}) = (\widehat{f, f_0}) }\)?

    \begin{cloze}{1}
        Сравнить для \({ g \in V }\) величины \({ \cos (\widehat{f, g}) }\)  и \({ \cos (\widehat{f, f_0}) }\).
    \end{cloze}
\end{note}

\begin{note}{e0b0c8a4f09c400ea5decb5c86c75027}
    Пусть \({ W }\) --- евклидово пространство, \({ V_1, V_2 \triangleleft W }\).
    \begin{icloze}{1}Угол между подпространствами \({ V_1, V_2 }\)\end{icloze} обозначается
    \begin{icloze}{2}
        \[
            (\widehat{V_1, V_2}).
        \]
    \end{icloze}
\end{note}

\begin{note}{fe6711839e2e4292aa55fdd4f80c6c80}
    Пусть \({ W }\) --- евклидово пространство, \({ V_1, V_2 \triangleleft W }\).
    \begin{gather*}
        \begin{icloze}{3}
            (\widehat{V_1, V_2})
        \end{icloze}
        \overset{\text{def}}=
        \begin{icloze}{1}
            \min \left\{ (\widehat{v_1, v_2}) \mid v_1 \in L_1, v_2 \in L_2 \right\},
        \end{icloze}
        \\
        L_{1,2} := \begin{icloze}{2}V_{1,2} \cap (V_1 \cap V_2)^{\perp}.\end{icloze}
    \end{gather*}
\end{note}

\begin{note}{1b70f983a1ad42c5919ee83b15a479c3}
    Пусть \({ V }\) --- евклидово пространство, \({ \left\{ a_j \right\}_{j = 1}^{n} \subset V }\).
    \begin{icloze}{2}Параллелепипедом, натянутым на систему векторов \({ \left\{ a_j \right\} }\)\end{icloze} называется \begin{icloze}{1}множество
    \[
        \left\{ \sum_{i=1}^{n} k_j a_j \mid k_j \in [0, 1] \quad \forall j \in [1 : n] \right\}.
    \]\end{icloze}
\end{note}

\begin{note}{3fa74674fa86420ab3f78529ea808264}
    Пусть \({ V }\) --- евклидово пространство, \({ \left\{ a_j \right\}_{j = 1}^{n} \subset V }\).
    \begin{icloze}{2}Параллелепипед, натянутый на систему векторов \({ \left\{ a_j \right\} }\)\end{icloze} обозначается
    \begin{icloze}{1}
        \[
            \Pi (a_1, \ldots, a_n).
        \]
    \end{icloze}
\end{note}

\begin{note}{b967d3e120fc46c7b7b7bd6a4ffe07a8}
    Пусть \({ V }\) --- евклидово пространство, \({ \left\{ a_j \right\}_{j = 1}^{n} \subset V }\).
    \begin{icloze}{2}\({ n }\)-мерный объём параллелепипеда \({ \Pi (a_1, \ldots, a_n) }\)\end{icloze} обозначается
    \begin{icloze}{1}
        \[
            \operatorname{vol}_{n} \Pi(a_1, \ldots, a_n).
        \]
    \end{icloze}
\end{note}

\begin{note}{a7e0867fc1bc4ede8b8c8baf077decb8}
    Пусть \({ V }\) --- евклидово пространство, \begin{icloze}{3}\({ a_1 \in V }\).\end{icloze}
    \[
        \begin{icloze}{2}\operatorname{vol}_{1} \Pi(a_1)\end{icloze} \overset{\text{def}}= \begin{icloze}{1}\left\lVert a_1 \right\rVert.\end{icloze}
    \]
\end{note}

\begin{note}{27fc106d10454b36acb1d4fdb4d5ebc6}
    Пусть \({ V }\) --- евклидово пространство, \begin{icloze}{3}\({ \left\{ a_j \right\}_{j = 1}^{n} \subset V }\), \({ n \geqslant 2 }\).\end{icloze}
    \begin{multline*}
        \begin{icloze}{2}\operatorname{vol}_{n} \Pi(a_1, \ldots, a_n) \end{icloze}
        \overset{\text{def}}= \\
        \begin{icloze}{1}\operatorname{vol}_{n - 1} \Pi(a_1, \ldots, a_{n - 1}) \cdot d(a_n, \mathscr L (a_1, \ldots, a_{n - 1})).\end{icloze}
    \end{multline*}
\end{note}

\begin{note}{62dd58a35b6e41618c5f4c14ba96d7df}
    Пусть \({ V }\) --- евклидово пространство, \({ \left\{ a_j \right\}_{j = 1}^{n} \subset V }\).
    Тогда
    \[
        \begin{icloze}{2}\left( \operatorname{vol}_{n} \Pi(a_1, \ldots, a_n) \right)^2\end{icloze} = \begin{icloze}{1}\det G(a_1, \ldots, a_n).\end{icloze}
    \]
\end{note}

\begin{note}{8803537671b2486285e407e1661e183c}
    Пусть \({ V }\) --- евклидово пространство, \({ \left\{ a_j \right\}_{j = 1}^{n} \subset V }\).
    Тогда
    \[
        \left( \operatorname{vol}_{n} \Pi(a_1, \ldots, a_n) \right)^2 =  \det G(a_1, \ldots, a_n).
    \]
    На каком методе основано  доказательство?

    \begin{cloze}{1}
        Индукция по \({ n }\).
    \end{cloze}
\end{note}

\begin{note}{4644b404d3cf48b3aa477e6cf6346d8a}
    Пусть \({ V }\) --- евклидово пространство, \({ \left\{ a_j \right\}_{j = 1}^{n} \subset V }\).
    Тогда
    \[
        \left( \operatorname{vol}_{n} \Pi(a_1, \ldots, a_n) \right)^2 =  \det G(a_1, \ldots, a_n).
    \]
    В чём основная идея доказательства (индукционный переход)?

    \begin{cloze}{1}
        Представить \({ a_n }\) как
        \[
            \sum_{k=1}^{n - 1} \lambda_k a_k + a^{\perp},
        \]
        где \({ a^{\perp} }\) --- перпендикуляр из \({ a_n }\) на \({ \mathscr L (a_1, \ldots, a_{n - 1}) }\).
    \end{cloze}
\end{note}

\begin{note}{5825ffe0a68e4a83a7e469b67b266431}
    Откуда следует корректность определения величины
    \[
        \operatorname{vol}_{n} \Pi(a_1, \ldots, a_n)?
    \]

    \begin{cloze}{1}
        Из теоремы о связи \({ \operatorname{vol}_{n} \Pi(a_1, \ldots, a_n) }\) с матрицей Грама.
    \end{cloze}
\end{note}

\begin{note}{8167d79d15c5496986c4ed42e064fa03}
    Пусть \({ V }\) --- векторное пространство над \({ \mathbb C }\).
    Чем определение скалярного произведения для векторных пространств над \({ \mathbb C }\) отличается от определения для пространств над \({ \mathbb R }\)?

    \begin{cloze}{1}
        Линейность только по первому аргументу и
        \[
            (v, w) = \overline{(w, v)}.
        \]
    \end{cloze}
\end{note}

\begin{note}{b2db7b3282094eea8733437c52bba06d}
    \begin{icloze}{1}Унитарным/эрмитовым пространством\end{icloze} называется \begin{icloze}{2}комплексное линейное пространство с заданным на нём скалярным произведением.\end{icloze}
\end{note}

\begin{note}{dee379a767e44b7fbc29a556ce456b02}
    Пусть \({ V }\) --- унитарное пространство, \({ v \in V }\).
    Откуда следует, что \({ (v, v) \in \mathbb R }\)?

    \begin{cloze}{1}
        Из аксиом линейного пространства \({ (v, v) = \overline{(v, v)} }\).
    \end{cloze}
\end{note}

\begin{note}{5d93df02453b469989ec31cb02334953}
    Пусть \({ V }\) --- унитарное пространство, \({ u, v \in V }\),\: \({ \lambda \in \mathbb C }\).
    Тогда
    \[
        \begin{icloze}{2}(u, \lambda v)\end{icloze} = \begin{icloze}{1}\overline{\lambda} (u, v).\end{icloze}
    \]
\end{note}

\begin{note}{3c66004b6eb3476f85bb3505bdce5da3}
    Пример определения скалярного произведения для \({ \mathbb C^{n} }\).

    \begin{cloze}{1}
        \[
            (z, w) = \sum_{j} z_j \overline{w_j}.
        \]
    \end{cloze}
\end{note}

\section{Лекция 25.04.22}
\begin{note}{dd4a52e4947c482987ee915067979415}
    Пусть \begin{icloze}{3}\({ V }\) --- линейное пространство над \({ \mathbb C }\).\end{icloze}
    \begin{icloze}{1}Рассмотрение \({ V }\), как векторного пространства над \({ \mathbb R }\),\end{icloze} называется \begin{icloze}{2}овеществлением \({ V }\).\end{icloze}
\end{note}

\begin{note}{fce4b5036a48493086a124057e1f048d}
    Пусть \begin{icloze}{3}\({ V }\) --- линейное пространство над \({ \mathbb C }\).\end{icloze}
    \begin{icloze}{1}Овеществление  \({ V }\)\end{icloze} обозначается \begin{icloze}{2}\({ V_{\mathbb R} }\).\end{icloze}
\end{note}

\begin{note}{1502c0b949d740cc9b70927038d34e79}
    Пусть \({ V }\) --- линейное пространство над \({ \mathbb C }\),\: \({ f \in \operatorname{End} V }\).
    \begin{icloze}{1}Рассмотрение \({ f }\) как оператора \({ V_{\mathbb R} \to V_{\mathbb R} }\)\end{icloze} называется \begin{icloze}{2}овеществлением \({ f }\).\end{icloze}
\end{note}

\begin{note}{8fd578e0ac514b2c826528aa165fb19a}
    Пусть \({ V }\) --- линейное пространство над \({ \mathbb C }\),\: \({ f \in \operatorname{End} V }\).
    \begin{icloze}{2}Овеществление \({ f }\)\end{icloze} обозначается \begin{icloze}{1}\({ f_{\mathbb R} }\).\end{icloze}
\end{note}

\begin{note}{76444b467049412d855fd6a8bb955fef}
    Пусть \({ V }\) --- линейное пространство над \({ \mathbb C }\),\: \begin{icloze}{3}\({ \left\{ e_j \right\}_{j = 1}^{n} }\) --- базис в \({ V }\).\end{icloze}
    Тогда \begin{icloze}{1}\({ \left\{ e_j \right\} \cup \left\{ ie_j \right\} }\)\end{icloze} --- \begin{icloze}{2}базис в \({ V_{\mathbb R} }\).\end{icloze}
\end{note}

\begin{note}{8b1e99362cbc41f9ad58dce068e5a358}
    Пусть \({ V }\) --- линейное пространство над \({ \mathbb C }\). Тогда
    \[
        \dim V_{\mathbb R} = \begin{icloze}{1}2 \cdot \dim V.\end{icloze}
    \]
\end{note}

\begin{note}{cc3fd18a829c481eb6a18fd4944621b2}
    Пусть \({ f : V  \to V }\) --- линейный оператор в комплексном пространстве,\: \begin{icloze}{4}\({ \left\{ e_j \right\}_{j = 1}^{n} }\) --- базис в \({ V }\).\end{icloze}
    Тогда для базиса
    \begin{icloze}{3}
        \[
            \left\{ \tilde e_j \right\}_{j = 1}^{2n} = \left\{ e_1, \ldots, e_n, ie_1, \ldots, ie_n \right\}
        \]
    \end{icloze}
    пространства \begin{icloze}{5}\({ V_{\mathbb R} }\)\end{icloze} имеем
    \[
        \begin{icloze}{2}M_{\tilde e} (f_{\mathbb R})\end{icloze} = \begin{icloze}{1}\begin{bmatrix}
            B & -C \\
            C & B
        \end{bmatrix}, \quad
        \text{где \({ B + iC = M_{e}(f) }\)}.\end{icloze}
    \]
\end{note}

\begin{note}{cc938e458f944909a9c35a4d1dc9cea0}
    Пусть \begin{icloze}{3}\({ V }\) --- линейное пространство над \({ \mathbb R }\).\end{icloze}
    \[
        \begin{icloze}{2}V_{\mathbb C}\end{icloze} \overset{\text{def}}= \begin{icloze}{1}\left\{ (u, v) \mid u, v \in V \right\}.\end{icloze}
    \]
\end{note}

\begin{note}{117bb526f78d4eceace2fdea3f410d63}
    Пусть \({ V }\) --- линейное пространство над \({ \mathbb R }\),\: \({ (u, v) \in V_{\mathbb C} }\).
    Тогда
    \[
        \begin{icloze}{2}i(u, v)\end{icloze} \overset{\text{def}}= \begin{icloze}{1}(-v, u).\end{icloze}
    \]
\end{note}

\begin{note}{f3dbbf009e9b4e94972dfdbf0af435f6}
    Как запомнить правило умножения в пространстве \({ V_{\mathbb C} }\)?

    \begin{cloze}{1}
        ``Представить'' элемент \({ (u, v) \in V_{\mathbb C} }\) как \({ u + iv }\).
    \end{cloze}
\end{note}

\begin{note}{389f838bfe634a94bbac20ddbd28d838}
    \begin{icloze}{2}Пространство \({ V_{\mathbb C} }\)\end{icloze} называется \begin{icloze}{1}комплексификацией пространства \({ V }\).\end{icloze}
\end{note}

\begin{note}{d7b4cacadaa641759cab38db5ae47b15}
    Пусть \({ f: V \to W }\) линейный оператор в евклидовых пространствах.
    Оператор \({ g : \begin{icloze}{3}W \to V\end{icloze} }\) называется \begin{icloze}{2}сопряжённым оператором к оператору \({ f }\),\end{icloze} если
    \begin{icloze}{1}
        \[
            (f(v), w) = (v, g(w)) \quad \forall v \in V, w \in W.
        \]
    \end{icloze}
\end{note}

\begin{note}{e40069d511a1495fb9dd3107a6d11084}
    Пусть \({ f: V \to W }\) линейный оператор в евклидовых пространствах.
    \begin{icloze}{2}Сопряжённый оператор к оператору \({ f }\)\end{icloze} обозначается \begin{icloze}{1}\({ f^{*} }\).\end{icloze}
\end{note}

\begin{note}{5b47c38358684d31a1d171bdc613fa5f}
    Пусть \({ f: V \to W }\) линейный оператор в евклидовых пространствах.
    Как показать, что \({ f^* }\) линеен?

    \begin{cloze}{1}
        Показать, что \({ f^*(\lambda w) - \lambda f^* (w) }\) ортогонален всем векторам в \({ V }\). Аналогично для суммы.
    \end{cloze}
\end{note}

\begin{note}{d9bfeeafc3314b3582a8263231d7301b}
    Пусть \({ f: V \to W }\) линейный оператор в евклидовых пространствах.
    Как показать существование \({ f^* }\)?

    \begin{cloze}{1}
        Явным образом найти его матрицу.
    \end{cloze}
\end{note}

\begin{note}{d532583798ec4eafb0bcec5c1a718f50}
    Пусть \({ f: V \to W }\) линейный оператор в евклидовых пространствах.
    Однозначно ли определён оператор \({ f^* }\)?

    \begin{cloze}{1}
        Да, однозначно.
    \end{cloze}
\end{note}

\begin{note}{49ba268bb22b4db9858de680fb15c62b}
    Пусть \({ f: V \to W }\) линейный оператор в евклидовых пространствах, \begin{icloze}{3}\({ \left\{ e_i \right\} }\) и \({ \left\{ \tilde e_j \right\} }\) --- ортонормированные базисы в \({ V }\) и \({ W }\), соответственно.\end{icloze}
    Тогда
    \[
        \begin{icloze}{2}M_{\tilde e, e} (f^*)\end{icloze} = \begin{icloze}{1}(M_{e, \tilde e} (f))^{T}.\end{icloze}
    \]
\end{note}

\begin{note}{bb4d8ef0dcd644128b6a2e296af6f5d2}
    Пусть \({ f: V \to W }\) линейный оператор в евклидовых пространствах, \({ \left\{ e_i \right\} }\) и \({ \left\{ \tilde e_j \right\} }\) --- ортонормированные базисы в \({ V }\) и \({ W }\), соответственно.
    Как показать, что
    \[
        M_{\tilde e, e} (f) = (M_{e, \tilde e} (f))^{T}?
    \]

    \begin{cloze}{1}
        Вычислить коэффициенты Фурье \({ (e_i, f^*(\tilde e_j)) }\).
    \end{cloze}
\end{note}

\begin{note}{fa653accdda24b31ab20506ac8538332}
    Пусть \({ f: V \to W }\) линейный оператор в эрмитовых пространствах.
    Тогда
    \[
        (f^*)^* = \begin{icloze}{1}f.\end{icloze}
    \]
\end{note}

\begin{note}{cbcfd9ad889c446d8de6ca2776680205}
    Пусть \({ f_1, f_2 : V \to W }\) линейные операторы в эрмитовых пространствах,\: \({ \lambda, \mu  \in \mathbb R }\).
    Тогда
    \[
        (\lambda f_1 + \mu f_2)^* = \begin{icloze}{1}\overline{\lambda} f_1^* + \overline{\mu} f_2^*.\end{icloze}
    \]
\end{note}

\begin{note}{6e9f045a1c4e4808bfcbd69523e55ac3}
    Пусть \({ f_1, f_2 : V \to W }\) линейные операторы в эрмитовых пространствах.
    Тогда
    \[
        (fg)^* = \begin{icloze}{1}g^* f^*.\end{icloze}
    \]
\end{note}

\begin{note}{1520bcac46ee4b76aefe766faef8769e}
    Пусть \({ f : V \to V }\) линейный оператор в эрмитовом пространстве, \({ v }\) --- собственный вектор операторов \begin{icloze}{2}\({ f }\) и \({ f^* }\),\end{icloze} отвечающий \begin{icloze}{3}собственным значениям \({ \lambda }\) и \({ \mu }\)\end{icloze} соответственно.
    Тогда
    \begin{icloze}{1}
        \[
            \mu = \overline{\lambda}.
        \]
    \end{icloze}
\end{note}

\section{Лекция 16.05.22}
\begin{note}{3954a1f946d54d49843bb75beba5c6a2}
    Пусть \({ f : V \to V }\) --- линейный оператор в эрмитовом пространстве.
    Тогда
    \[
        \begin{icloze}{2}\ker f^*\end{icloze} = \begin{icloze}{1}(\operatorname{im}f)^{\perp}\end{icloze}.
    \]
\end{note}

\begin{note}{2b5bf9cde3b944f3829f6df51e0a5d46}
    Пусть \({ f : V \to V }\) --- линейный оператор в эрмитовом пространстве.
    Тогда
    \[
        \begin{icloze}{2}\operatorname{im}f^*\end{icloze} = \begin{icloze}{1}(\ker f)^{\perp}\end{icloze}.
    \]
\end{note}

\begin{note}{82658444368d432c84797667377faa14}
    Пусть \({ f : V \to V }\) --- линейный оператор в эрмитовом пространстве.
    Тогда \({ \ker f^* = (\operatorname{im}f)^{\perp} }\).
    В чём основная идея доказательства?

    \begin{cloze}{1}
        \({ v \in \ker f^* \iff (v, f(w)) = 0 \quad \forall w }\).
    \end{cloze}
\end{note}

\begin{note}{ac28c64b9e0843ee85ea8d67e40ff6d1}
    Пусть \({ f : V \to V }\) --- линейный оператор в эрмитовом пространстве.
    Тогда \(\operatorname{im}f^* = (\ker f)^{\perp}\).
    В чём основная идея доказательства?

    \begin{cloze}{1}
        Следует из равенства \({ \ker (f^*)^* = (\operatorname{im} f^*)^{\perp} }\).
    \end{cloze}
\end{note}

\begin{note}{0b903e3801544d2a9284c6c06caa3e11}
    Пусть \({ f : V \to V }\) --- линейный оператор в эрмитовом пространстве, \({ V \triangleleft W }\).
    Тогда, если \begin{icloze}{4}\({ V }\) инвариантно относительно \({ f }\),\end{icloze} то \begin{icloze}{1}\({ V^{\perp} }\)\end{icloze} \begin{icloze}{3}инвариантно\end{icloze} относительно \begin{icloze}{2}\({ f^* }\).\end{icloze}
\end{note}

\begin{note}{6a68e47fbd554019b4f16972953073da}
    Пусть \({ f : V \to V }\) --- линейный оператор в эрмитовом пространстве, \({ V \triangleleft W }\).
    Тогда, если \({ V }\) инвариантно относительно \({ f }\), то \({ V^{\perp} }\) инвариантно относительно \({ f^* }\).
    В чём основная идея доказательства?

    \begin{cloze}{1}
        \({ (v, f^*(w)) = (f(v), w) = 0 \quad \forall w \in V^{\perp} }\).
    \end{cloze}
\end{note}

\begin{note}{cb0fd56398174603b26c14a08091c430}
    Пусть \({ a, \lambda, \mu \in \mathbb C }\),\: \({ \lambda \neq \mu }\). Как из равенства \({ \lambda a = \mu a }\) следует, что \({ a = 0 }\)?

    \begin{cloze}{1}
        \[
            \underbrace{(\lambda - \mu)}_{\neq 0} a = 0.
        \]
    \end{cloze}
\end{note}

\begin{note}{398d95972a0746bb8ca3fe90ee7fafe6}
    Пусть \({ f : V \to V }\) --- линейный оператор в эрмитовом пространстве.
    Тогда
    \[
        f^* = f \implies \operatorname{spec} f \begin{icloze}{1}\subset \mathbb R.\end{icloze}
    \]
\end{note}

\begin{note}{6eec0c089bb9471397a72db6e6ea7c4b}
    Пусть \({ f : V \to V }\) --- линейный оператор в эрмитовом пространстве.
    Тогда \({ f^* = f \implies \operatorname{spec} f \subset \mathbb R }\).
    В чём основная идея доказательства?

    \begin{cloze}{1}
        \({ \forall \lambda \in \operatorname{spec} f \quad \lambda = \overline{\lambda} }\).
    \end{cloze}
\end{note}

\begin{note}{01f37aa032dc4ae184b41741f1009cba}
    Пусть \({ f : V \to V }\) --- линейный оператор в эрмитовом пространстве, \begin{icloze}{3}\({ f = f^* }\),\end{icloze}\: \({ x, y \in V }\).
    Тогда, если \({ x }\) и \({ y }\) --- \begin{icloze}{1}собственные векторы оператора \({ f }\), отвечающие разным собственным значениям,\end{icloze} то \begin{icloze}{2}\({ x \perp y }\).\end{icloze}
\end{note}

\begin{note}{9aab3b15d5bd4c1db803f7d44757e9f2}
    Пусть \({ f : V \to V }\) --- линейный оператор в эрмитовом пространстве, \({ f = f^* }\),\: \({ x, y \in V }\).
    Тогда если \({ x }\) и \({ y }\) --- собственные векторы оператора \({ f }\), отвечающие разным собственным значениям, то \({ x \perp y }\).
    В чём основная идея доказательства?

    \begin{cloze}{1}
        Рассмотреть скалярное произведение
        \[
            (f(x), y) = (x, f(y)).
        \]
    \end{cloze}
\end{note}

\begin{note}{9cd8e8889d15407b9c8bc0d715fc7b96}
    \subsubsection{<<\begin{icloze}{3}Спектральная теорема для \\\phantom{<<}самосопряжённых операторов\end{icloze}>>}

    Пусть \({ f : V \to V }\) --- линейный оператор в эрмитовом пространстве.
    Тогда если \begin{icloze}{2}\({ f^* = f }\),\end{icloze} то в пространстве \({ V }\) существует \begin{icloze}{1}ортонормированный базис из собственных векторов оператора \({ f }\).\end{icloze}
\end{note}

\begin{note}{7e1c25eb54d844309b458da40780c8f1}
    В чём основная идея доказательства спектральной теоремы для самосопряжённых операторов?

    \begin{cloze}{1}
        Для \({ \lambda \in \operatorname{spec} f }\) имеем \({ V = V_f(\lambda) \oplus V_f(\lambda)^{\perp} }\), но оба этих пространства инвариантны относительно \({ f }\).
    \end{cloze}
\end{note}

\begin{note}{4884039a91ca44c2a72e903879e0cb15}
    Почему в доказательстве спектральной теоремы для самосопряжённых операторов нам важно, что оба пространства в прямой сумме \({ V_f(\lambda) \oplus V_f(\lambda)^{\perp} = V }\) инвариантны относительно \({ f }\)?

    \begin{cloze}{1}
        Из этого следует, что \({ f }\) представляется соответствующей квазидиагональной матрицей.
    \end{cloze}
\end{note}

\begin{note}{b4ac85781d214b28a62641aa58b35dac}
    Пусть \({ f : V \to V }\) --- линейный оператор в эрмитовом пространстве, \({ f = f^* }\),\: \({ \lambda \in \operatorname{spec} f }\).
    Почему пространство \({ V_f(\lambda)^{\perp} }\) инвариантно относительно \({ f }\)?

    \begin{cloze}{1}
        \({ V_f(\lambda) }\) инвариантно относительно \({ f }\) \({ \implies }\) \({ V_f(\lambda)^{\perp} }\) инвариантно относительно \({ f^* = f }\).
    \end{cloze}
\end{note}

\begin{note}{6a3db890388d426ba9b6f47900bb8d01}
    Пусть \({ f : V \to V }\) --- линейный оператор в эрмитовом пространстве, \({ f = f^* }\).
    Почему \({ f }\) не может не иметь действительных собственных значений?

    \begin{cloze}{1}
        Любой оператор имеет комплексные собственные значения, но из самосопряжённости следует, что эти значения действительны.
    \end{cloze}
\end{note}

\begin{note}{ca9ea18afa5c40f8873a302849e8b0b3}
    Пусть \({ f : V \to V }\) --- линейный оператор в \begin{icloze}{3}эрмитовом пространстве.\end{icloze}
    Оператор \({ f }\) называется \begin{icloze}{2}унитарным,\end{icloze} если
    \begin{icloze}{1}
        \[
            (f(v), f(w)) = (v, w)  \quad \forall v, w \in V.
        \]
    \end{icloze}
\end{note}

\begin{note}{7ba81b35301b4520996b24d87bc4fd09}
    Пусть \({ f : V \to V }\) --- линейный оператор в \begin{icloze}{3}евклидовом пространстве.\end{icloze}
    Оператор \({ f }\) называется \begin{icloze}{2}ортогональным,\end{icloze} если
    \begin{icloze}{1}
        \[
            (f(v), f(w)) = (v, w)  \quad \forall v, w \in V.
        \]
    \end{icloze}
\end{note}

\begin{note}{c7ddf42a012945c888c4d5178b28f2f0}
    Пусть \({ f : V \to V }\) --- унитарный оператор.
    Тогда помимо скалярного произведения \({ f }\) сохраняет \begin{icloze}{1}длины и углы.\end{icloze}
\end{note}

\begin{note}{7671b924a9f647578fd2fe26e4253413}
    Пусть \({ f : V \to V }\) --- унитарный оператор.
    Тогда помимо скалярного произведения \({ f }\) сохраняет длины и углы.
    В чём основная идея доказательства?

    \begin{cloze}{1}
        Длины и углы выражаются через скалярное произведение.
    \end{cloze}
\end{note}

\begin{note}{89e9f8ff95e74665b47de711b04ebf2e}
    Пусть \({ f : V \to V }\) --- линейный оператор в \begin{icloze}{3}эрмитовом пространстве.\end{icloze}
    Тогда \begin{icloze}{2}\({ f }\) унитарнен\end{icloze} тогда и только тогда, когда
    \begin{icloze}{1}
        \[
            f^* = f^{-1}.
        \]
    \end{icloze}

    \begin{center}
        \tiny
        (в терминах \({ f^* }\))
    \end{center}
\end{note}

\begin{note}{41a4909e68244e2e9fd1312185a42e07}
    Пусть \({ f : V \to V }\) --- линейный оператор в \begin{icloze}{3}эрмитовом пространстве.\end{icloze}
    Тогда \begin{icloze}{2}\({ f }\) унитарнен\end{icloze} тогда и только тогда, когда
    \begin{icloze}{1}
        \[
            \left\lVert v \right\rVert = \left\lVert f(v) \right\rVert \quad \forall v \in V.
        \]
    \end{icloze}

    \begin{center}
        \tiny
        (в терминах норм)
    \end{center}
\end{note}

\begin{note}{865b571c22cf4ed9bf1c3a80e8e88c8c}
    Пусть \({ f : V \to V }\) --- линейный оператор в эрмитовом пространстве.
    Тогда \({ f }\) унитарнен \({ \impliedby }\) \({ \left\lVert v \right\rVert = \left\lVert f(v) \right\rVert \quad \forall v \in V }\).
    В чём ключевая идея доказательства?

    \begin{cloze}{1}
        Рассмотреть \({ \left\lVert a + b \right\rVert^2 }\) и \({ \left\lVert a + ib \right\rVert^2 }\), получив сохранение отдельно вещественной и отдельно мнимой частей скалярного произведения.
    \end{cloze}
\end{note}

\begin{note}{e1de7c2dfa3d4ea9b1cd75ce8c764c17}
    Пусть \({ z \in \mathbb C }\).
    Тогда \({ z + \overline{z} = \begin{icloze}{1}\mathfrak{R} (z).\end{icloze} }\)
\end{note}

\begin{note}{5e9e6d646883410db5ecf8c401d3026d}
    Пусть \({ z \in \mathbb C }\).
    Тогда \({ z - \overline{z} = \begin{icloze}{1}2i \cdot \mathfrak{I}(z).\end{icloze} }\)
\end{note}

\begin{note}{2ecb4d7e7be047b887fa8953465857cb}
    Пусть \({ f : V \to V }\) --- линейный оператор в \begin{icloze}{3}эрмитовом пространстве.\end{icloze}
    Тогда \begin{icloze}{2}\({ f }\) унитарнен\end{icloze} тогда и только тогда, когда
    \begin{icloze}{1}\({ f }\) переводит любой ортонормированный базис в ортонормированный базис.\end{icloze}

    \begin{center}
        \tiny
        (в терминах базисов)
    \end{center}
\end{note}

\begin{note}{b4fdcc1c898f454cb1ca4e9fe9d18ef0}
    Пусть \({ f : V \to V }\) --- линейный оператор в эрмитовом пространстве.
    Тогда \({ f }\) унитарнен \({ \impliedby }\)
    \({ f }\) переводит любой ортонормированный базис в ортонормированный базис.
    В чём ключевая идея доказательства?

    \begin{cloze}{1}
        Показать, что \({ f }\) сохраняет длины.
    \end{cloze}
\end{note}

\begin{note}{55a699f79e064eb7bcbad7d49660f64b}
    Пусть \({ A \in \mathbb C^{\begin{icloze}{3}n \times n\end{icloze}} }\).
    Матрица \({ A }\) называется \begin{icloze}{2}унитарной\end{icloze} если
    \begin{icloze}{1}
        \[
            \overline{A}^{\perp} = A^{-1}.
        \]
    \end{icloze}
\end{note}

\begin{note}{3ddc6bb6b47d44c6900bc4593cea65ba}
    Пусть \({ f : V \to V }\) --- линейный оператор в \begin{icloze}{3}эрмитовом пространстве.\end{icloze}
    Тогда \begin{icloze}{2}\({ f }\) унитарнен\end{icloze} тогда и только тогда, когда
    \begin{icloze}{1}матрица \({ f }\) в ортонормированном базисе унитарна.\end{icloze}

    \begin{center}
        \tiny
        (в терминах матриц оператора)
    \end{center}
\end{note}

\begin{note}{4754847a2e4c47108a24063de1aeb2e7}
    Пусть \({ f : V \to V }\) --- линейный оператор в эрмитовом пространстве.
    Тогда \({ f }\) унитарнен тогда и только тогда, когда
    матрица \({ f }\) в ортонормированном базисе унитарна.
    В чём ключевая идея доказательства?

    \begin{cloze}{1}
        \({ f^* = f^{-1} }\) \({ \iff }\) равны и их матрицы.
    \end{cloze}
\end{note}

\begin{note}{a749d29f6a92475aa3bac79533cb3dfe}
    Пусть \({ f : V \to V }\) --- унитарный оператор.
    Тогда \({ \forall \lambda \in \operatorname{spec} f }\) имеем \begin{icloze}{1}\({ \lvert \lambda \rvert = 1 }\).\end{icloze}
\end{note}

\begin{note}{1977a9b870034e868dc7565fb5174779}
    Пусть \({ f : V \to V }\) --- унитарный оператор.
    Тогда \({ \forall \lambda \in \operatorname{spec} f }\) имеем \({ \lvert \lambda \rvert = 1 }\).
    В чём основная идея доказательства?

    \begin{cloze}{1}
        Для \({ v \in V_f(\lambda) \setminus \left\{ 0 \right\} }\) рассмотреть \({ (f(v), f(v)) }\).
    \end{cloze}
\end{note}

\begin{note}{f3a34c6f96f24f3c91b5b5cb5b85386d}
    Пусть \({ f : W \to W }\) --- унитарный оператор, \({ V \triangleleft W }\).
    Тогда если \begin{icloze}{4}\({ V }\) инвариантно относительно \({ f }\),\end{icloze} то \begin{icloze}{1}\({ V^{\perp} }\)\end{icloze} \begin{icloze}{3}инвариантно\end{icloze} относительно \begin{icloze}{2}\({ f }\).\end{icloze}
\end{note}

\begin{note}{fd57ae2625734c759ffb57bc00f69d8f}
    Пусть \({ f : W \to W }\) --- унитарный оператор, \({ V \triangleleft W }\).
    Тогда если \({ V }\) инвариантно относительно \({ f }\),  то и \({ V^{\perp} }\) инвариантно относительно \({ f }\).
    В чём основная идея доказательства?

    \begin{cloze}{1}
        \({ V^{\perp} }\) инвариантно относительно \({ f^* = f^{-1} }\).
    \end{cloze}
\end{note}

\begin{note}{05351f096d4a47cc90c2f500fbe4fe16}
    \subsubsection{<<\begin{icloze}{3}Спектральная теорема для \\\phantom{<<}унитарных операторов\end{icloze}>>}

    Пусть \({ f : V \to V }\) --- \begin{icloze}{2}унитарный оператор.\end{icloze}
    Тогда в пространстве \({ V }\) существует \begin{icloze}{1}ортонормированный базис из собственных векторов оператора \({ f }\).\end{icloze}
\end{note}

\begin{note}{5469fe81bf8641b2bca4525ca9e6599f}
    В чём основная идея доказательства спектральной теоремы для унитарных операторов?

    \begin{cloze}{1}
        Для \({ \lambda \in \operatorname{spec} f }\) имеем \({ V = V_f(\lambda) \oplus V_f(\lambda)^{\perp} }\), но оба этих пространства инвариантны относительно \({ f }\).
    \end{cloze}
\end{note}

\begin{note}{1e6b5642085f4ef7a716fb7a5422b363}
    В \({ \mathbb R^2 }\) любое ортогональное преобразование --- есть либо \begin{icloze}{1}поворот,\end{icloze} либо \begin{icloze}{2}отражение относительно прямой.\end{icloze}
\end{note}

\section{Семинар 20.04.22}
\begin{note}{4af2a0956e564d7a8dcff91122d2862c}
    В процессе ортогонализации Грама-Шмидта определитель Грама \begin{icloze}{1}не меняется.\end{icloze}
\end{note}

\begin{note}{dd5abffedea24431af7beee39693bcb2}
    Пусть \({ V }\) --- эрмитово пространство, \begin{icloze}{2}\({ \left\{ e_j \right\}_{j = 1}^{n} \subset V }\).\end{icloze}
    Тогда
    \[
        \operatorname{sgn} \left\lvert G(e_1, \ldots, e_n) \right\rvert \in \begin{icloze}{1}\left\{ 0, 1 \right\}.\end{icloze}
    \]
\end{note}

\begin{note}{550eea6aa0654fe1bcff80057656cabf}
    Пусть \({ V }\) --- эрмитово пространство, \begin{icloze}{4}\({ \left\{ e_j \right\}_{j = 1}^{n} \subset V }\).\end{icloze}
    Тогда \begin{icloze}{2}
        \[
            \left\lvert G(e_1, \ldots, e_n) \right\rvert = 0
        \]
    \end{icloze} \begin{icloze}{3}тогда и только тогда, когда\end{icloze} \begin{icloze}{1}система \({ \left\{ e_j \right\} }\) линейно зависима.\end{icloze}
\end{note}

\begin{note}{c01adb60f10f465ea82f0eb8c910472c}
    Пусть \({ V }\) --- эрмитово пространство, \begin{icloze}{3}\({ \left\{ e_j \right\}_{j = 1}^{n} \subset V }\).\end{icloze}
    Тогда
    \[
        \left\lvert G(e_1, \ldots, e_n) \right\rvert \begin{icloze}{2}\leqslant\end{icloze} \begin{icloze}{1}\prod_{j = 1}^{n} \left\lVert e_j \right\rVert^2.\end{icloze}
    \]
\end{note}

\begin{note}{e3a8ea052f1b41b48d35050e054c8f59}
    Пусть \({ V }\) --- эрмитово пространство, \begin{icloze}{5}\({ \left\{ e_j \right\}_{j = 1}^{n} \subset V }\).\end{icloze}
    Тогда
    \[
        \left\lvert G(e_1, \ldots, e_n) \right\rvert \begin{icloze}{4}=\end{icloze} \begin{icloze}{3}\prod_{j = 1}^{n} \left\lVert e_j \right\rVert^2\end{icloze} \begin{icloze}{2}\iff\end{icloze}
        \begin{icloze}{1}
            \left[ {
                \begin{array}{@{}l@{\quad}l@{}}
                    \text{\({ \left\{ e_j \right\} }\) ортогональна}, \\
                    \exists j \quad e_j = 0.
                \end{array}
            } \right.
        \end{icloze}
    \]
\end{note}

\section{Семинар 27.04.22}
\begin{note}{15065bb7284d464eb733caa7ce69f5c2}
    Пусть \({ L_1, L_2 }\) --- векторные подпространства,\: \begin{icloze}{5}\({ L_1 \cap L_2 = \left\{ 0 \right\} }\).\end{icloze}
    Тогда \({ \begin{icloze}{4}(\widehat{L_1, L_2})\end{icloze} = \begin{icloze}{3}(\widehat{g, g_1})\end{icloze} }\), где
    \begin{center}
        \begin{tabular}{rl}
            \({ g }\) &--- \begin{icloze}{1}проекция ненулевого вектора \({ x \in L_1 }\) на \({ L_2 }\),\end{icloze} \\
            \({ g_1 }\) &--- \begin{icloze}{2}проекция \({ g }\) на \({ L_1 }\).\end{icloze}
        \end{tabular}
    \end{center}
\end{note}

\begin{note}{6bc4328f156248a99ab740a54db23882}
    Пусть \({ f : V \to V }\) --- линейный оператор в эрмитовом пространстве.
    Тогда оператор \({ ff^* }\) \begin{icloze}{1}самосопряжён.\end{icloze}
\end{note}

\begin{note}{ee802fb93fc34532abc98ffcbb813a0a}
    Пусть \({ f : V \to V }\) --- линейный оператор в эрмитовом пространстве.
    Тогда оператор \({ f^*f }\) \begin{icloze}{1}самосопряжён.\end{icloze}
\end{note}

\begin{note}{2298652c987c4036be1f1244e0d2d16a}
    Пусть \({ f : V \to V }\) --- линейный оператор в эрмитовом пространстве,\: \begin{icloze}{3}\({ \det f \neq 0 }\)\end{icloze}.
    Тогда \({ \begin{icloze}{2}(f^{-1})^*\end{icloze} = \begin{icloze}{1}(f^*)^{-1}\end{icloze} }\).
\end{note}

\begin{note}{f9e1ce5c80d84b1fa0d487d5057a623f}
    Пусть \({ A = [a_{ij}] \in \mathbb C^{n \times m} }\).
    Тогда
    \[
        \overline{A} \overset{\text{def}}= \begin{icloze}{1}\big[ \overline{a_{ij}} \big].\end{icloze}
    \]
\end{note}

\begin{note}{3ed5769b04134886b2dae82e3c375951}
    Пусть \({ f : V \to V }\) --- линейный оператор в эрмитовом пространстве,\: \({ \left\{ e_j \right\}_{j = 1}^{n} }\) --- \begin{icloze}{4}базис в \({ V }\).\end{icloze}
    Тогда
    \[
        \begin{icloze}{5}M_e(f^*)\end{icloze} = \begin{icloze}{1}\overline{G^{-1}A^{T}G},\end{icloze}
    \]
    где \({ A = \begin{icloze}{2}M_e(f)\end{icloze} }\),\: \({ G = \begin{icloze}{3}G(e_1, \ldots, e_n)\end{icloze} }\).
\end{note}

\begin{note}{2e0ddee23d1142f492e32a3030b2abbe}
    Пусть \({ f : V \to V }\) --- линейный оператор в эрмитовом пространстве,\: \({ \left\{ e_j \right\}_{j = 1}^{n} }\) --- базис в \({ V }\).
    Тогда \({ M_e(f^*) = \overline{G^{-1}A^{T}G} }\), где \({ A = M_e(f) }\),\: \({ G = G(e_1, \ldots, e_n) }\).
    В чём основная идея доказательства?

    \begin{cloze}{1}
        Использовать \({ G }\) как матрицу полуторалинейной формы.
    \end{cloze}
\end{note}

\begin{note}{fa1bb01afd4c43bbb1e902fcdf3891a2}
    Пусть \({ A \in \mathbb C^{n \times m} }\),\: \({ B \in \mathbb C^{m \times l} }\).
    Тогда
    \[
        \overline{AB} = \begin{icloze}{1}\overline{A} \: \overline{B}.\end{icloze}
    \]
\end{note}

\begin{note}{4a0309bd276e44448efb76c62e0fbfcf}
    Пусть \({ f, g : V \to V }\) --- самосопряжённые операторы в эрмитовом пространстве.
    Тогда \begin{icloze}{2}оператор \({ fg }\) самосопряжён\end{icloze} \begin{icloze}{3}\({ \iff }\)\end{icloze} \begin{icloze}{1}\({ fg = gf }\).\end{icloze}
\end{note}

\begin{note}{3b50b418e4594bfeb03b83bb0f393857}
    Пусть \({ f, g : V \to V }\) --- самосопряжённые операторы в эрмитовом пространстве.
    Тогда оператор \begin{icloze}{2}\({ fg + gf }\)\end{icloze} \begin{icloze}{1}самосопряжён.\end{icloze}
\end{note}

\begin{note}{480f285619264421ad7eebf15db6c3c5}
    Пусть \({ f, g : V \to V }\) --- самосопряжённые операторы в эрмитовом пространстве,\: \({ \lambda \in \mathbb C }\).
    Тогда если \begin{icloze}{2}\({ \overline{\lambda} = -\lambda }\),\end{icloze} то оператор \begin{icloze}{3}\({ \lambda(fg - gf) }\)\end{icloze} \begin{icloze}{1}самосопряжён.\end{icloze}
\end{note}

\section{Лекция 23.05.22}
\begin{note}{5a8ab0eed63d4e1cb7f6f69e11c2aabd}
    В \({ \mathbb R^2 }\) любое ортогональное преобразование \({ f }\) --- есть либо поворот, либо отражение относительно прямой.
    Какие  два случая рассматриваются в доказательстве?

    \begin{cloze}{1}
        1. \({ \operatorname{spec} f \subset  \mathbb R }\),\:  2. \({ \operatorname{spec} f \cap \mathbb R = \emptyset }\).
    \end{cloze}
\end{note}

\begin{note}{20fe0d9d75484cf1858bc7ea943f9e34}
    В \({ \mathbb R^2 }\) любое ортогональное преобразование \({ f }\) --- есть либо поворот, либо отражение относительно прямой.
    В чём основная идея доказательства (случай \({ \operatorname{spec} f \subset \mathbb R }\))?

    \begin{cloze}{1}
        \({ \operatorname{spec} f = \left\{ \pm 1, \pm 1 \right\} }\), и во всех случаях получаем нужное преобразование.
    \end{cloze}
\end{note}

\begin{note}{231ac7e677014b5c906070674ee0e438}
    В \({ \mathbb R^2 }\) любое ортогональное преобразование \({ f }\) --- есть либо поворот, либо отражение относительно прямой.
    В чём основная идея доказательства (случай \({ \operatorname{spec} f \cap \mathbb R = \emptyset }\))?

    \begin{cloze}{1}
        Для \({ \lambda = \cos \varphi - i \sin \varphi \in \operatorname{spec} f }\) и \({ e = a + bi \in V_f(\lambda) }\) расписать
        \[
            f(e) = \lambda e.
        \]
    \end{cloze}
\end{note}

\begin{note}{31d69048b8654670a45bed290ffb4052}
    Пусть \({ f : \begin{icloze}{4}\mathbb R^{n} \to \mathbb R^{n}\end{icloze} }\) --- линейный оператор,\: \({ \lambda \in \begin{icloze}{5}\operatorname{spec} f \setminus \mathbb R\end{icloze} }\).
    Тогда если \begin{icloze}{2}\({ a + ib \in V_{f_{\mathbb C}}(\lambda) \setminus \left\{ 0 \right\} }\)\end{icloze} (где \begin{icloze}{3}\({ a, b \in \mathbb R^{n} }\)\end{icloze}), то \begin{icloze}{1}\({ a }\) и \({ b }\) линейно независимы.\end{icloze}
\end{note}

\begin{note}{a8555ff624e746ceae88a1597c9ebcf4}
    Пусть \({ f : \mathbb R^{n} \to \mathbb R^{n} }\) --- линейный оператор,\: \({ \lambda \in \operatorname{spec} f \setminus \mathbb R }\).
    Тогда если \({ a + ib \in V_{f_{\mathbb C}}(\lambda) \setminus \left\{ 0 \right\} }\) (где \({ a, b \in \mathbb R^{n} }\)), то \({ a }\) и \({ b }\) линейно независимы.
    В чём ключевая идея доказательства?

    \begin{cloze}{1}
        От обратного и тогда \({ f(a) = \lambda a }\), что невозможно, поскольку \({ a, f(a) \in \mathbb R^{n} }\).
    \end{cloze}
\end{note}

\begin{note}{678857a6b6514b7e87cade6a771f4c53}
    \begin{icloze}{2}Матрица поворота на угол \({ \varphi }\)\end{icloze} имеет вид:
    \begin{icloze}{1}
        \[
            \begin{bmatrix}
                \cos \varphi & -\sin \varphi \\
                \sin \varphi & \cos \varphi
            \end{bmatrix}.
        \]
    \end{icloze}
\end{note}

\begin{note}{398eb33a3cd34a718118f2be8f7095a2}
    Пусть \({ \varphi \in \mathbb R }\) --- произвольный угол.
    \begin{icloze}{1}Матрица поворота на угол \({ \varphi }\)\end{icloze} обозначается \begin{icloze}{2}\({ R_{\varphi} }\).\end{icloze}
\end{note}

\begin{note}{87e8bdb8fccb4fd9ac6ce183b1724513}
    \begin{icloze}{1}
        Матрица вида
        \[
            \operatorname{diag}(R_{\varphi_1}, \ldots, R_{\varphi_k}, \pm 1, \ldots, \pm 1).
        \]
    (с точностью до порядка клеток)
    \end{icloze}
    называется \begin{icloze}{2}каноническим видом матрицы ортогонального оператора.\end{icloze}
\end{note}

\begin{note}{951f954d7fc94632bfd90e091d13396e}
    В \({ \mathbb R^{n} }\) для любого ортогонального оператора существует \begin{icloze}{1}ортонормированный базис, в котором матрица оператора имеет канонический вид.\end{icloze}
\end{note}

\begin{note}{e8a3805f9a3f4bae8dd14745c6603d37}
    В \({ \mathbb R^{n} }\) для любого ортогонального оператора существует ортонормированный базис, в котором матрица оператора имеет канонический вид.
    В чём ключевая идея доказательства?

    \begin{cloze}{1}
        Выбрать собственное значение, построить отвечающий ему блок и далее ``по индукции'' для сужения на ортогональное дополнение.
    \end{cloze}
\end{note}

\begin{note}{3dbcabf7585c41e590b7515d2294cb74}
    Пусть \({ f }\) --- ортогональный оператор,\: \({ a + bi }\) --- его собственный вектор.
    Тогда \({ a }\) и \({ b }\) \begin{icloze}{1}ортогональны и имеют равную длину.\end{icloze}
\end{note}

\begin{note}{5b4055b476de4754a0eab034334b564d}
    Пусть \({ f }\) --- ортогональный оператор,\: \({ a + bi }\) --- его собственный вектор.
    Тогда \({ a }\) и \({ b }\) ортогональны и имеют равную длину.
    В чём ключевая идея доказательства?

    \begin{cloze}{1}
        Выразить \({ (a, b) }\) и \({ (a, a) }\) через значения \({ f(a) }\) и \({ f(b) }\) и составить СЛАУ.
    \end{cloze}
\end{note}

\begin{note}{ebbc584e56104a56ab20c37cb3f76036}
    Пусть \({ f }\) --- ортогональный оператор,\: \({ a + bi }\) --- его собственный вектор.
    Тогда \({ a }\) и \({ b }\) ортогональны и имеют равную длину.
    Относительно каких переменных составляется СЛАУ в доказательстве?

    \begin{cloze}{1}
         Относительно \({ (a, b) }\) и \({ (a, a) - (b, b) }\)
    \end{cloze}
\end{note}

\begin{note}{91344cb034d9420b9a63807b59bc1cbf}
    Отображение \({ q : \begin{icloze}{1}\mathbb R^{n} \to \mathbb R\end{icloze} }\) вида
    \[
        q(x) = \begin{icloze}{2}\sum_{i,j}^{} a_{ij} \cdot x_i x_j, \quad a_{ij} \in \mathbb R.\end{icloze}
    \]
    называется \begin{icloze}{3}квадратичной формой в \({ \mathbb R^{n} }\).\end{icloze}
\end{note}

\begin{note}{9f5070390cb64df59f3c3a3edb21964d}
    Пусть \({ q : x \mapsto  \sum_{}^{} a_{ij} \cdot x_i x_j }\) --- квадратичная форма в \({ \mathbb R^{n} }\).
    Для удобства полагают, что \begin{icloze}{1}\({ a_{ij} = a_{ji} }\).\end{icloze}
\end{note}

\begin{note}{f3a4dfc29aec43e4be378badd310835b}
    Матрицей квадратичной формы \({ x \mapsto  \sum_{}^{} a_{ij} \cdot x_i x_j }\) в \({ \mathbb R^{n} }\) называется
    \begin{icloze}{1}
        матрица
        \[
            \begin{bmatrix}
                a_{ij}
            \end{bmatrix}
            \sim n \times n.
        \]
    \end{icloze}
\end{note}

\begin{note}{c20dec1b0cdf4cfb824cfb96603e87c6}
    Пусть \({ q }\) --- квадратичная форма в \({ \mathbb R^{n} }\), \({ A }\) --- матрица \({ q }\).
    Тогда
    \[
        A^{T} = \begin{icloze}{1}A.\end{icloze}
    \]
\end{note}

\begin{note}{f47b5351d6cc4c4792454ba3ae253b45}
    Пусть \({ q }\) --- квадратичная форма в \({ \mathbb R^{n} }\), \({ A }\) --- матрица \({ q }\).
    Как \({ q(x) }\) выражается через произведение матриц?

    \begin{cloze}{1}
        \[
            q(x) = x^{T}Ax.
        \]
    \end{cloze}
\end{note}

\begin{note}{80787c73077441aa88a40c590206c15e}
    Пусть \({ q }\) --- квадратичная форма в \({ \mathbb R^{n} }\), \({ A }\) --- матрица \({ q }\).
    Как \({ q(x) }\) выражается через евклидово скалярное произведение в \({ \mathbb R^{n} }\)?

    \begin{cloze}{1}
        \[
            q(x) = (Ax, x)
        \]
    \end{cloze}
\end{note}

\begin{note}{78fd5f73a8bd4fe2bd9715e0dfc63f3e}
    Пусть \({ q : x \mapsto x^{T}Ax }\) --- квадратичная форма в \({ \mathbb R^{n} }\).
    Форма \({ q }\) называется \begin{icloze}{2}положительно определённой,\end{icloze} если
    \begin{icloze}{1}
        \({ \forall x }\)
        \begin{center}
            \({ q(x) \geqslant 0 }\) \quad и \quad \({ q(x) = 0 \iff x = 0 }\).
        \end{center}
    \end{icloze}
\end{note}

\begin{note}{e2a3814c661f4081a60e07f3ae077705}
    Пусть \({ q : x \mapsto x^{T}Ax }\) --- квадратичная форма в \({ \mathbb R^{n} }\).
    Тогда всегда \begin{icloze}{3}существует\end{icloze} такая \begin{icloze}{2}замена переменных \({ x = By }\),\end{icloze} что
    \begin{icloze}{1}
        \[
            q(By) = \sum_{i=1}^{\operatorname{rk} A} \mu_i y_i^2.
        \]
    \end{icloze}
\end{note}

\begin{note}{5efae3fe5ee64edabaa2a50d91d8714f}
    Пусть \({ q : x \mapsto x^{T}Ax }\) --- квадратичная форма в \({ \mathbb R^{n} }\).
    Тогда всегда существует такая замена переменных \({ x = By }\), что
    \[
        q(By) = \sum_{i=1}^{\operatorname{rk} A} \mu_i y_i^2.
    \]
    В чём ключевая идея доказательства (без использования спектральной теоремы)?

    \begin{cloze}{1}
        Элементарными преобразованиями строк и столбцов привести матрицу \({ q }\) к диагональному виду.
    \end{cloze}
\end{note}

\begin{note}{2b6bd1f239a0448cb62e2d9e6ba8e5a7}
    Пусть \({ q : x \mapsto x^{T}Ax }\) --- квадратичная форма в \({ \mathbb R^{n} }\).
    \begin{icloze}{1}
        Представление \({ q(x) }\) в виде
        \[
            q(By) = \sum_{i=1}^{\operatorname{rk} A} \mu_i y_i^2.
        \]
    \end{icloze}
    называется \begin{icloze}{2}каноническим видом квадратичной формы \({ q }\).\end{icloze}
\end{note}

\begin{note}{d28a23b8889943d4b3ffc14e52cb4e21}
    Пусть \({ q : x \mapsto x^{T}Ax }\) --- квадратичная форма в \({ \mathbb R^{n} }\).
    \begin{icloze}{1}Число положительных коэффициентов в каноническом виде \({ q }\)\end{icloze} называется \begin{icloze}{2}положительным индексом инерции \({ q }\).\end{icloze}
\end{note}

\begin{note}{e3c4dedc4bea411e996ba571573fe4ab}
    Пусть \({ q : x \mapsto x^{T}Ax }\) --- квадратичная форма в \({ \mathbb R^{n} }\).
    \begin{icloze}{2}Положительный индекс инерции \({ q }\)\end{icloze} обычно обозначается \begin{icloze}{1}\({ \pi }\).\end{icloze}
\end{note}

\begin{note}{45d2e04c6dc0495c8c53da1a9a52fa03}
    Пусть \({ q : x \mapsto x^{T}Ax }\) --- квадратичная форма в \({ \mathbb R^{n} }\).
    \begin{icloze}{2}Отрицательный индекс инерции \({ q }\)\end{icloze} обычно обозначается \begin{icloze}{1}\({ \nu }\).\end{icloze}
\end{note}

\begin{note}{ccfa3006b8dd4d76a4d29b6bd06efa0e}
    Пусть \({ q : x \mapsto x^{T}Ax }\) --- квадратичная форма в \({ \mathbb R^{n} }\).
    \begin{icloze}{1}Число отрицательных коэффициентов в каноническом виде \({ q }\)\end{icloze} называется \begin{icloze}{2}отрицательным индексом инерции \({ q }\).\end{icloze}
\end{note}

\begin{note}{e0192c4d7f9b4c8da63ea34e5b91d7b5}
    Пусть \({ q : x \mapsto x^{T}Ax }\) --- квадратичная форма в \({ \mathbb R^{n} }\).
    \begin{icloze}{3}Положительные и отрицательные индексы инерции \({ q }\)\end{icloze} \begin{icloze}{2}не зависят\end{icloze} от \begin{icloze}{1}замены переменных, приводящей \({ q }\) к каноническому виду.\end{icloze}
\end{note}

\begin{note}{7c4f4a41668546169076f4a02bb73f41}
    Пусть \({ q : x \mapsto x^{T}Ax }\) --- квадратичная форма в \({ \mathbb R^{n} }\).
    Тогда \begin{icloze}{2}\({ \pi }\)\end{icloze} --- это \begin{icloze}{1}максимальная размерность подпространства, на котором форма \({ q }\) положительно определена.\end{icloze}

    \begin{center}
        \tiny
        (в терминах положительной определённости)
    \end{center}
\end{note}

\begin{note}{24dea29600d54b75b2530902cfa4a5af}
    Пусть \({ q : x \mapsto x^{T}Ax }\) --- квадратичная форма в \({ \mathbb R^{n} }\).
    Тогда \begin{icloze}{2}\({ \nu }\)\end{icloze} --- это \begin{icloze}{1}максимальная размерность подпространства, на котором форма \({ q }\) отрицательно определена.\end{icloze}

    \begin{center}
        \tiny
        (в терминах положительной определённости)
    \end{center}
\end{note}

\begin{note}{2efa1a15074a49d19d8568cb77fd70d7}
    Пусть \({ q }\) --- квадратичная форма в \({ \mathbb R^{n} }\).
    Тогда \({ \pi }\) --- это максимальная размерность подпространства, на котором форма \({ q }\) положительно определена.
    В чём ключевая идея доказательства?

    \begin{cloze}{1}
        Выбрать базис \({ e }\), в котором
        \[
            q(e\lambda) = \lambda^{T} \begin{bmatrix}
                E_{\pi}   \\
                & -E_{\nu} \\
                && 0      \\
            \end{bmatrix}
            \lambda.
        \]
    \end{cloze}
\end{note}

\begin{note}{7a2d2be95e6543ff9f5642410cda1fc7}
    Пусть \({ q }\) --- квадратичная форма в \({ \mathbb R^{n} }\).
    Почему мы знаем, что существует базис \({ e }\), в котором
    \[
        q(e\lambda) = \lambda^{T} \begin{bmatrix}
            E_{\pi}   \\
            & -E_{\nu} \\
            && 0      \\
        \end{bmatrix}
        \lambda.
    \]

    \begin{cloze}{1}
        Диагональный вид существует из спектральной теоремы. Остаётся нормировать и переставить базисные векторы.
    \end{cloze}
\end{note}

\begin{note}{2002e141fa3345999d474a7c75a27b69}
    Пусть \({ q }\) --- квадратичная форма в \({ \mathbb R^{n} }\),
    \({ e }\) --- базис такой, что
    \[
        q(e\lambda) = \lambda^{T} \begin{bmatrix}
            E_{\pi}   \\
            & -E_{\nu} \\
            && 0      \\
        \end{bmatrix}
        \lambda.
    \]
    Что можно сказать про \({ L = \mathscr L (e_1, \ldots, e_{\pi}) }\)?

    \begin{cloze}{1}
        \({ q }\) положительно определена на \({ L }\).
    \end{cloze}
\end{note}

\begin{note}{a5149aa8efff4d5ba3531d0dc96ec6ba}
    Пусть \({ q }\) --- квадратичная форма в \({ \mathbb R^{n} }\),
    \({ e }\) --- базис такой, что
    \[
        q(e\lambda) = \lambda^{T} \begin{bmatrix}
            E_{\pi}   \\
            & -E_{\nu} \\
            && 0      \\
        \end{bmatrix}
        \lambda.
    \]
    Что можно сказать про \({ L = \mathscr L (e_{\pi + 1}, \ldots, e_{\pi + \nu}) }\)?

    \begin{cloze}{1}
        \({ q }\) отрицательно определена на \({ L }\).
    \end{cloze}
\end{note}

\begin{note}{456d6973756f4ab9bfbfd28f53e7a060}
    Пусть \({ q }\) --- квадратичная форма в \({ \mathbb R^{n} }\),
    \({ e }\) --- базис такой, что
    \[
        q(e\lambda) = \lambda^{T} \begin{bmatrix}
            E_{\pi}   \\
            & -E_{\nu} \\
            && 0      \\
        \end{bmatrix}
        \lambda.
    \]
    Тогда если \begin{icloze}{2}\({ q }\) положительно определена на \({ G }\),\end{icloze} то \begin{icloze}{1}\({ \dim G \leqslant \pi }\).\end{icloze}
\end{note}

\begin{note}{93b97c5f8fa2484e9cd48faeac9b577e}
    Пусть \({ q }\) --- квадратичная форма в \({ \mathbb R^{n} }\),
    \({ e }\) --- базис такой, что
    \[
        q(e\lambda) = \lambda^{T} \begin{bmatrix}
            E_{\pi}   \\
            & -E_{\nu} \\
            && 0      \\
        \end{bmatrix}
        \lambda.
    \]
    Тогда если \({ q }\) положительно определена на \({ G }\), то \({ \dim G \leqslant \pi }\).
    В чём ключевая идея доказательства?

    \begin{cloze}{1}
        В \({ G }\) не может лежать векторов из \({ \mathscr L (e_{\pi + 1}, \ldots, e_n) }\).
    \end{cloze}
\end{note}

\begin{note}{5fc4db6c932e4c5bac06afaafb2e8a68}
    Пусть \({ q }\) --- квадратичная форма в \({ \mathbb R^{n} }\).
    Почему \({ \pi }\) и \({ \nu }\) корректно определены?

    \begin{cloze}{1}
        Следует из связи значений \({ \pi }\) и \({ \nu }\) с размерностями подпространств, на которых \({ q }\) положительно/отрицательно определена.
    \end{cloze}
\end{note}

\begin{note}{97b6d8cfbb8e499e8552699c8f4d0bf8}
    Пусть \({ q : x \mapsto x^{T}Ax }\) --- квадратичная форма в \({ \mathbb R^{n} }\),\: \({ i \in [1 : n] }\).
    \[
        \begin{icloze}{2}\Delta_i\end{icloze} \overset{\text{def}}= \begin{icloze}{1}M^{1 \ldots i}_{1 \ldots i}(A).\end{icloze}
    \]
\end{note}

\begin{note}{70b0c89b87844a56bfbafcab882e879b}
    Пусть \({ q : x \mapsto x^{T}Ax }\) --- квадратичная форма в \({ \mathbb R^{n} }\).
    Тогда \begin{icloze}{2}\({ q }\) положительно определена\end{icloze} \begin{icloze}{3}\({ \iff }\)\end{icloze}
    \begin{icloze}{1}
        \[
            \forall j \quad \Delta_j > 0.
        \]
    \end{icloze}

    \begin{center}
        \tiny <<\begin{icloze}{3}Критерий\end{icloze} \begin{icloze}{4}Сильвестра\end{icloze}>>
    \end{center}
\end{note}

\begin{note}{6fc3ef1864d246d2a2ee658a1f0902fc}
    В чём основная идея доказательства критерия Сильвестра для квадратичных форм?

    \begin{cloze}{1}
        Элементарными преобразованиями, не меняющими значений угловых миноров, привести матрицу формы к диагональному виду.
    \end{cloze}
\end{note}

\begin{note}{b67b00fb8e7741c2bde46724543c741d}
    К чему применяются элементарные преобразования в доказательстве критерия Сильвестра для квадратичных форм: к строкам или к столбцам?

    \begin{cloze}{1}
        И к строкам, и к столбцам одновременно.
    \end{cloze}
\end{note}

\begin{note}{5673368bd606421b82198acfa11aa4d1}
    Какие элементарные преобразования применяются к матрице квадратичной формы в доказательстве критерия Сильвестра?

    \begin{cloze}{1}
        Прибавление к одной строке другой строки, умноженной на число. Для столбцов то же, но зеркально.
    \end{cloze}
\end{note}

\begin{note}{ee57ba62b118492b8faabc390f478fce}
    Почему в доказательства критерия Сильвестра для квадратичных форм (необходимость) можно считать, что все \({ \lambda_i }\) в полученном диагональном виде отличны от нуля?

    \begin{cloze}{1}
        Первое \({ \lambda_i = 0 \iff \Delta_i = 0 }\).
    \end{cloze}
\end{note}

\begin{note}{8629aff0d2a945f2bb5ae5fbb13440e6}
    Почему в доказательства критерия Сильвестра для квадратичных форм (достаточность) можно считать, что все \({ \lambda_i }\) в полученном диагональном виде отличны от нуля?

    \begin{cloze}{1}
        От обратного и тогда \({ \exists x \neq 0 : q(x) = 0 }\).
    \end{cloze}
\end{note}

\begin{note}{75e17707e4e446ed9ca2db4710251d36}
    Пусть \({ q : x \mapsto x^{T}Ax }\) --- квадратичная форма в \({ \mathbb R^{n} }\).
    Тогда всегда \begin{icloze}{4}существует\end{icloze} такая \begin{icloze}{3}ортогональная замена \({ x = By }\),\end{icloze} что
    \begin{icloze}{1}
        \[
            q(By) = \sum_{i=1}^{\operatorname{rk} A} \lambda_i y_i^2,
        \]
    \end{icloze}
    где \({ \left\{ \lambda_j \right\} \begin{icloze}{2}= \operatorname{spec} A\end{icloze} }\).
\end{note}

\begin{note}{2cbc861e5ee44f50a82f862ca8317fbf}
    Пусть \({ q : x \mapsto x^{T}Ax }\) --- квадратичная форма в \({ \mathbb R^{n} }\).
    Тогда всегда существует такая ортогональная замена \({ x = By }\), что
    \[
        q(By) = \sum_{i=1}^{\operatorname{rk} A} \lambda_i y_i^2,
    \]
    где \({ \left\{ \lambda_j \right\} = \operatorname{spec} A }\).
    В чём ключевая идея доказательства?

    \begin{cloze}{1}
        Спектральная теорема для самосопряжённых операторов.
    \end{cloze}
\end{note}

\begin{note}{27069c1e2308471e94c9d33659f5c2c0}
    Пусть \({ A \in \mathbb R^{n \times n} }\) \begin{icloze}{2}симметрична.\end{icloze}
    Тогда по \begin{icloze}{3}спектральной теореме для самосопряжённых операторов\end{icloze} \({ A }\) \begin{icloze}{1}диагонализуема.\end{icloze}
\end{note}

\begin{note}{092c2617bcad4d609ecc2818ad32ac07}
    Пусть \({ A \in \mathbb R^{n \times n} }\) симметрична.
    Почему \({ A }\) самосопряжена?

    \begin{cloze}{1}
        \[
            A^* = \overline{A^{T}}.
        \]
    \end{cloze}
\end{note}

\begin{note}{31b3b048ada040338acc82b240da632e}
    Пусть \({ A \in \mathbb R^{n \times n} }\) симметрична.
    Тогда по спектральной теореме для самосопряжённых операторов
    \[
        A = C \Lambda C^{-1}.
    \]
    Что можно сказать про матрицу \({ C }\)?

    \begin{cloze}{1}
        Она ортогональна.
    \end{cloze}
\end{note}

\begin{note}{e9126cafe6c945e795c28f3684b96f54}
    Пусть \({ A \in \mathbb R^{n \times n} }\) симметрична.
    Тогда по спектральной теореме для самосопряжённых операторов
    \[
        A = C \Lambda C^{-1}.
    \]
    Почему матрица \({ C }\) ортогональна?

    \begin{cloze}{1}
        \({ C }\) --- матрица перехода от ортогонального базиса к ортогональному.
    \end{cloze}
\end{note}

\begin{note}{c71cb32466464fd1b45a573370778a8a}
    Пусть \({ A \in \begin{icloze}{4}\mathbb R^{n \times n}\end{icloze} }\) \begin{icloze}{3}ортогональна\end{icloze}.
    Тогда \({ \begin{icloze}{2}A^{-1}\end{icloze} = \begin{icloze}{1}A^{T}\end{icloze} }\).
\end{note}

\begin{note}{b68e5c8fd9d34bb59b547f8004aabf07}
    Пусть \({ A \in \mathbb R^{n \times n} }\) симметрична.
    Тогда по спектральной теореме для самосопряжённых операторов
    \[
        A = C \Lambda C^{-1}.
    \]
    Почему \({ \Lambda }\) не может иметь комплексных значений на диагонали?

    \begin{cloze}{1}
        \({ A }\) самосопряжена \({ \implies }\) \({ \operatorname{spec} A \subset \mathbb R }\).
    \end{cloze}
\end{note}

\section{Семинар 18.05.22}
\begin{note}{b6887df7065f4a8ab54f2db4f3a0d40b}
    Пусть \({ q : x \mapsto x^{T}Ax }\) --- квадратичная форма в \({ \mathbb R^{n} }\).
    Тогда если
    \begin{icloze}{2}
        \[
            \forall j \leqslant \operatorname{rk} A \quad  \Delta_j \neq 0,
        \]
    \end{icloze}
    то \({ q }\) приводится к каноническому виду
    \[
        q(By) = \sum_{i=1}^{\operatorname{rk} A} \lambda_i y_i^2, \quad
        \text{где \begin{icloze}{1}\({ \lambda_k = \frac{\Delta_k}{\Delta_{k - 1}} }\).\end{icloze}}
    \]

    \begin{center}
        \tiny <<\begin{icloze}{3}формула Якоби\end{icloze}>>
    \end{center}
\end{note}

\begin{note}{4f977e40188141d2a3865d203cba08d5}
    В чём основная идея доказательства формулы Якоби для квадратичных форм?

    \begin{cloze}{1}
        Элементарными преобразованиями, не меняющими значений угловых миноров, привести матрицу формы к диагональному виду.
    \end{cloze}
\end{note}

\begin{note}{52494560052c4c2ca0b504c059601cdd}
    Пусть \({ q : x \mapsto x^{T}Ax }\) --- квадратичная форма в \({ \mathbb R^{n} }\).
    \begin{icloze}{1}Величина \({ \pi - \nu }\)\end{icloze} называется \begin{icloze}{2}сигнатурой \({ q }\).\end{icloze}
\end{note}

\begin{note}{0fbf18cec5d54ec6823be4c8821294e1}
    Пусть \({ q : x \mapsto x^{T}Ax }\) --- квадратичная форма в \({ \mathbb R^{n} }\).
    \begin{icloze}{2}Сигнатура \({ q }\)\end{icloze} обычно обозначается \begin{icloze}{1}\({ \sigma }\).\end{icloze}
\end{note}

\section{Семинар 25.05.22}
\begin{note}{3388e39da8554cabbe1722fa7348f91b}
    Пусть \begin{icloze}{4}\({ M }\) --- конечное множество,\end{icloze}\: \begin{icloze}{5}\({ f : M \to M }\).\end{icloze}
    Тогда
    \begin{center}
        \begin{icloze}{3}\({ f }\) инъективно\end{icloze} \begin{icloze}{1}\({ \iff }\)\end{icloze} \begin{icloze}{2}\({ f }\) сюрьективно.\end{icloze}
    \end{center}
\end{note}

\begin{note}{377d2404fa034f7c938eb2f12061ab6f}
    \begin{icloze}{2}Группа обратимых элементов\end{icloze} кольца \({ K }\) обозначается \begin{icloze}{1}\({ K^* }\).\end{icloze}
\end{note}

\end{document}

% vim: spelllang=ru_yo,en
