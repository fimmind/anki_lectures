\documentclass[11pt, a5paper]{article}
\usepackage[width=10cm, top=0.5cm, bottom=2cm]{geometry}

\usepackage[T1,T2A]{fontenc}
\usepackage[utf8]{inputenc}
\usepackage[english,russian]{babel}
\usepackage{libertine}

\usepackage{amsmath}
\usepackage{amssymb}
\usepackage{amsthm}
\usepackage{mathrsfs}
\usepackage{framed}
\usepackage{xcolor}

\setlength{\parindent}{0pt}

\renewcommand{\thesection}{}
\renewcommand{\thesubsection}{Note \arabic{subsection}}
\renewcommand{\thesubsubsection}{}
\renewcommand{\theparagraph}{}

\newenvironment{note}[1]{\goodbreak\par\subsection{\hfill \color{lightgray}\tiny #1}}{}
\newenvironment{cloze}[2][\ldots]{\begin{leftbar}}{\end{leftbar}}
\newenvironment{icloze}[2][\ldots]{%
  \ignorespaces\text{\tiny \color{lightgray} \{\{c#2:: }%
}{%
  \text{\tiny \color{lightgray}\}\}}\unskip%
}

\begin{document}
\section{Лекция 07.02.22}
\begin{note}{b84aca6df42d4d74ad1fea51970c01d9}
    Пусть \begin{icloze}{3}\( W \) --- линейное пространство, \( V \subset W. \) \end{icloze} Тогда \( V \) называется \begin{icloze}{2}линейным подпространством\end{icloze}, если
    \begin{icloze}{1}
        \begin{enumerate}
            \item \( \forall v \in V, k \in \mathbb R \implies kv \in V,  \)
            \item \( \forall v_1, v_2 \in V \implies v_1 + v_2 \in V. \)
        \end{enumerate}
    \end{icloze}
\end{note}

\begin{note}{baa489a3d13c4978866a82630be13e73}
    Пусть \begin{icloze}{2}\( W \) --- линейное пространство, \( V \subset W. \)\end{icloze} Тогда \( V \) --- \begin{icloze}{1}тоже линейное пространство\end{icloze}.
\end{note}

\begin{note}{3c2988d9ae174eb4aa377f43ebd61f74}
    Является ли прямая проходящая через начало координат подпространством в \( \mathbb R ^{n}  \)?

    \begin{cloze}{1}
        Да, поскольку любая линейная комбинация векторов на прямой тоже лежит на этой прямой.
    \end{cloze}
\end{note}

\begin{note}{18b402a364da457aaaf95095b9113dcd}
    Пусть \( W = \mathbb R ^{n}, A \sim m \times n. \)
    Является ли множество
    \[
        V = \{ x \in W \mid Ax = 0  \}
    \]
    линейным подпространством?

    \begin{cloze}{1}
        Да, поскольку \( \forall u, v \in V, \quad \alpha, \beta \in \mathbb R \quad A(\alpha u + \beta v) = 0. \)
    \end{cloze}
\end{note}

\begin{note}{a5081684e6014eeb8d4cd352f7dfd46b}
    Пусть \( V \) --- подпространство \( R^{n}. \) Тогда всегда существует \( A \in R^{m \times n}  \) такая, что
    \begin{icloze}{1}\[
        V = \{ x \in W \mid Ax = 0  \}
    \]\end{icloze}
\end{note}

\begin{note}{dcb727a8588c412db845188bf547fd9e}
    Пусть \( W = \mathbb R ^{n}, \quad a_1, a_2, \ldots a_n \in W. \) Является ли
    \[
        \mathscr L (a_1, a_2, \ldots a_n)
    \]
    подпространством в \( W \)?

    \begin{cloze}{1}
        Да, является, поскольку любая линейная комбинация линейных комбинаций \( a_1, a_2, \ldots a_n  \)  тоже является их линейной комбинацией.
    \end{cloze}
\end{note}

\begin{note}{d633780bbade46968c2bcb66d05be478}
    Пусть \( W = \mathbb R ^{n}, \quad V_1, V_2 \subset W \) --- два линейных подпространства в \( W. \)
    Всегда ли \( V_1 \cap V_2 \) --- тоже линейное подпространство в \( W \)?

    \begin{cloze}{1}
        Да, всегда.
    \end{cloze}
\end{note}

\begin{note}{9c714ab9fa4b457f993438ef25421061}
    Пусть \( W = \mathbb R ^{n}, \quad V_1, V_2 \subset W \) --- два линейных подпространства в \( W. \)
    Всегда ли \( V_1 \cup V_2 \) --- тоже линейное подпространство в \( W \)?

    \begin{cloze}{1}
        Нет, не всегда.
    \end{cloze}
\end{note}

\begin{note}{2b9216d113914ad98cbc81b055dc174b}
    Пусть \( W = \mathbb R ^{n}, \quad V_1, V_2 \subset W \) --- два линейных подпространства в \( W. \)
    Тогда
    \[
        \begin{icloze}{2}V_1 + V_2\end{icloze} \overset{\text{def}}= \begin{icloze}{1}
            \{ v_1 + v_2 \mid v_1 \in V_1, \quad v_2 \in V_2 \} .
        \end{icloze}
    \]
\end{note}

\begin{note}{cd25e86c13c141be80e3673edfece8d2}
    Пусть \( W = \mathbb R ^{n}, \quad V_1, V_2 \subset W \) --- два линейных подпространства в \( W. \)
    Тогда
    \[
        \dim (V_1 + V_2) = \begin{icloze}{1}\dim V_1 + \dim V_2 - \dim (V_1 \cap V_2).\end{icloze}
    \]
\end{note}

\begin{note}{fe58542dc0ee4e48ab330cd68be1fd77}
    Пусть \( W = \mathbb R ^{n}, \quad V \subset W  \) --- линейное подпространство в \( W, \quad \) \( e_1, e_2, \ldots e_k  \)  --- \begin{icloze}{2}базис в \( V. \)\end{icloze}
    Тогда в \( W \) существует базис вида \begin{icloze}{1}\( e_1, e_2, \ldots e_k, e_{k + 1}, \ldots e_n.  \)\end{icloze}
\end{note}
\end{document}
