\documentclass[11pt, a5paper]{article}
\usepackage[width=10cm, top=0.5cm, bottom=2cm]{geometry}

\usepackage[T1,T2A]{fontenc}
\usepackage[utf8]{inputenc}
\usepackage[english,russian]{babel}
\usepackage{libertine}

\usepackage{amsmath}
\usepackage{amssymb}
\usepackage{amsthm}
\usepackage{mathrsfs}
\usepackage{framed}
\usepackage{xcolor}

\setlength{\parindent}{0pt}

% Force \pagebreak for every section
\let\oldsection\section
\renewcommand\section{\pagebreak\oldsection}

\renewcommand{\thesection}{}
\renewcommand{\thesubsection}{Note \arabic{subsection}}
\renewcommand{\thesubsubsection}{}
\renewcommand{\theparagraph}{}

\newenvironment{note}[1]{\goodbreak\par\subsection{\hfill \color{lightgray}\tiny #1}}{}
\newenvironment{cloze}[2][\ldots]{\begin{leftbar}}{\end{leftbar}}
\newenvironment{icloze}[2][\ldots]{%
  \ignorespaces\text{\tiny \color{lightgray}\{\{c#2::}\hspace{0pt}%
}{%
  \hspace{0pt}\text{\tiny\color{lightgray}\}\}}\unskip%
}


\begin{document}
\section{Лекция 07.02.22}
\begin{note}{b84aca6df42d4d74ad1fea51970c01d9}
    Пусть \begin{icloze}{3}\( W \) --- линейное пространство, \( V \subset W. \) \end{icloze} Тогда \( V \) называется \begin{icloze}{2}линейным подпространством\end{icloze}, если
    \begin{icloze}{1}
        \begin{enumerate}
            \item \( \forall v \in V, k \in \mathbb R \implies kv \in V,  \)
            \item \( \forall v_1, v_2 \in V \implies v_1 + v_2 \in V. \)
        \end{enumerate}
    \end{icloze}
\end{note}

\begin{note}{a2e780e4b5ff4b4199b594e34bf762c6}
    Выражение <<\( V \) есть линейное подпространство в \( W \)>> обозначают
    \begin{icloze}{1}
        \[
            V \triangleleft W
        \]
    \end{icloze}
\end{note}

\begin{note}{baa489a3d13c4978866a82630be13e73}
    Пусть \( W \) --- линейное пространство, \( V \triangleleft W \). Тогда \( V \) --- \begin{icloze}{1}тоже линейное пространство\end{icloze}.
\end{note}

\begin{note}{3c2988d9ae174eb4aa377f43ebd61f74}
    Является ли прямая проходящая через начало координат подпространством в \( \mathbb R ^{n}  \)?

    \begin{cloze}{1}
        Да, поскольку любая линейная комбинация векторов на прямой тоже лежит на этой прямой.
    \end{cloze}
\end{note}

\begin{note}{18b402a364da457aaaf95095b9113dcd}
    Пусть \( W = \mathbb R ^{n}, A \sim m \times n. \)
    Является ли множество
    \[
        V = \{ x \in W \mid Ax = 0  \}
    \]
    линейным подпространством?

    \begin{cloze}{1}
        Да, поскольку \( \forall u, v \in V, \quad \alpha, \beta \in \mathbb R \quad A(\alpha u + \beta v) = 0. \)
    \end{cloze}
\end{note}

\begin{note}{a5081684e6014eeb8d4cd352f7dfd46b}
    Пусть \( V \triangleleft \mathbb R ^{n}. \) Тогда всегда существует \( A \in \mathbb R ^{\begin{icloze}{2}m \times n\end{icloze}}  \) такая, что
    \begin{icloze}{1}\[
        V = \ker A.
    \]\end{icloze}
\end{note}

\begin{note}{dcb727a8588c412db845188bf547fd9e}
    Пусть \( W = \mathbb R ^{n}, \quad a_1, a_2, \ldots a_n \in W. \) Является ли
    \[
        \mathscr L (a_1, a_2, \ldots a_n)
    \]
    подпространством в \( W \)?

    \begin{cloze}{1}
        Да, является, поскольку любая линейная комбинация линейных комбинаций \( a_1, a_2, \ldots a_n  \)  тоже является их линейной комбинацией.
    \end{cloze}
\end{note}

\begin{note}{d633780bbade46968c2bcb66d05be478}
    Пусть \( W \) --- линейное пространство, \( V_1, V_2 \triangleleft W \).
    Всегда ли
    \[
        V_1 \cap V_2 \triangleleft W?
    \]


    \begin{cloze}{1}
        Да, всегда.
    \end{cloze}
\end{note}

\begin{note}{9c714ab9fa4b457f993438ef25421061}
    Пусть \( W \) --- линейное пространство, \( V_1, V_2 \triangleleft W \).
    Всегда ли
    \[
        V_1 \cup V_2 \triangleleft W?
    \]

    \begin{cloze}{1}
        Нет, не всегда.
    \end{cloze}
\end{note}

\begin{note}{2b9216d113914ad98cbc81b055dc174b}
    Пусть \( W \) --- линейное пространство, \( V_1, V_2 \triangleleft W \).
    Тогда
    \[
        \begin{icloze}{2}V_1 + V_2\end{icloze} \overset{\text{def}}= \begin{icloze}{1}
            \{ v_1 + v_2 \mid v_1 \in V_1, \quad v_2 \in V_2 \} .
        \end{icloze}
    \]
\end{note}

\begin{note}{cd25e86c13c141be80e3673edfece8d2}
    Пусть \( W \) --- линейное пространство, \( V_1, V_2 \triangleleft W \).
    Тогда
    \[
        \dim (V_1 + V_2) = \begin{icloze}{1}\dim V_1 + \dim V_2 - \dim (V_1 \cap V_2).\end{icloze}
    \]
\end{note}

\begin{note}{ecf370041c6b4016a92ca63a4b3675eb}
    Пусть \( W \) --- линейное пространство, \( V_1, V_2 \triangleleft W \).
    Всегда ли
    \[
        V_1 + V_2 \triangleleft W?
    \]

    \begin{cloze}{1}
        Да, всегда.
    \end{cloze}
\end{note}

\begin{note}{fe58542dc0ee4e48ab330cd68be1fd77}
    Пусть \( W \) --- линейное пространство, \( V \triangleleft W \) и \( e_1, e_2, \ldots, e_k  \)  --- \begin{icloze}{2}базис в \( V. \)\end{icloze}
    Тогда в \( W \) существует базис вида \begin{icloze}{1}
        \[
            e_1, e_2, \ldots, e_k, e_{k + 1}, \ldots, e_n.
        \]
    \end{icloze}
\end{note}

\begin{note}{7e41e14368b94d50be88c6e5b025c706}
    В чем основная идея доказательства теоремы о размерности суммы подпространств?

    \begin{cloze}{1}
        Дополнить базис в \( V_1 \cap V_2 \) до базисов в \( V_1 \) и \( V_2 \) соответственно и построить на их основе базис в \( V_1 + V_2 \).
    \end{cloze}
\end{note}

\begin{note}{01ac0beb84404bed8a9f676002a2804c}
    Пусть
    \begin{itemize}
        \item \( e_1,e_2,\ldots e_k \) --- базис в \( V_1 \cap V_2 \),
        \item \( e_1,e_2,\ldots e_k, f_1,\ldots f_p \) --- базис в \( V_1 \),
        \item \( e_1,e_2,\ldots, e_k, g_1,\ldots g_q \) --- базис в \( V_2 \).
    \end{itemize}
    Как можно построить базис в \( V_1 + V_2 \)?

    \begin{cloze}{1}
        \( e_1,\ldots e_k, f_1,\ldots f_p, g_1,\ldots, g_q \) --- базис в \( V_1 + V_2 \).
    \end{cloze}
\end{note}

\begin{note}{d6aa3baccb104c5d857dad61f06b75e7}
    Пусть
    \begin{itemize}
        \item \( e_1,e_2,\ldots e_k \) --- базис в \( V_1 \cap V_2 \),
        \item \( e_1,e_2,\ldots e_k, f_1,\ldots f_p \) --- базис в \( V_1 \),
        \item \( e_1,e_2,\ldots, e_k, g_1,\ldots g_q \) --- базис в \( V_2 \).
    \end{itemize}
    Как доказать, что
    \[
        e_1,\ldots e_k, f_1,\ldots f_p, g_1,\ldots, g_q
    \]
    --- базис в \( V_1 + V_2 \)?

    \begin{cloze}{1}
        Показать, что \( \forall i \quad g_i \not\in V_1 \), а значит
        \[
            V_1 + V_2 = V_1 \oplus \mathscr L (g_1, \ldots, g_q).
        \]
    \end{cloze}
\end{note}

\section{Семинар 09.02.22}
\begin{note}{3fd21160928849f8acbc526a60229e49}
    Пусть \( e_1, e_2, \ldots, e_n \) и \( e'_1, e'_2, \ldots, e'_n \) --- два базиса в линейном пространстве \( V \).
    Тогда \begin{icloze}{2}матрицей перехода от базиса \( e \) к базису \( e' \)\end{icloze} называют
    \begin{icloze}{1}матрицу \( C \) такую, что для любого \( v \in V \), если
    \[
        \begin{gathered}
            v = \lambda_1 e_1 + \lambda_2 e_2 + \cdots + \lambda_n e_n, \\
            v = \mu_1 e'_1 + \mu_2 e'_2 + \cdots + \mu_n e'_n,
        \end{gathered}
    \]
    то
    \[
        C \begin{bmatrix}
            \mu_1 \\ \mu_2 \\ \vdots \\ \mu_n
        \end{bmatrix}
        =
        \begin{bmatrix}
            \lambda_1 \\ \lambda_2 \\ \vdots \\ \lambda_n
        \end{bmatrix}.
    \]
\end{icloze}
\end{note}

\begin{note}{88fab27df46a451190278cbc1d38698f}
    \begin{icloze}{2}Матрицу перехода от базиса \( e \) к базису \( e' \)\end{icloze} обычно обозначают \begin{icloze}{1}\( C_{e \to e'}  \).\end{icloze}
\end{note}

\begin{note}{c9e84965d5ea4157b50f6576e2cbddad}
    Пусть \( e_1, e_2, \ldots, e_n \) и \( e'_1, e'_2, \ldots, e'_n \) --- два базиса в линейном пространстве.
    Как в явном виде задать матрицу \( C_{e \to e'} \)?

    \begin{cloze}{1}
        Столбцы \( C_{e \to e'} \) --- это координаты векторов \( e'_1, e'_2, \ldots, e'_n \) в базисе \( e_1, e_2, \ldots, e_n \).
    \end{cloze}
\end{note}

\section{Лекция 14.02.22}
\begin{note}{825be05cbe9f4850806682f4db48f5e1}
    Пусть \( W \) --- линейное пространство, \( V_1, V_2 \triangleleft W \).
    \begin{icloze}{2}Сумму \( V_1 + V_2 \)\end{icloze} называют \begin{icloze}{1}прямой сум\-мой,\end{icloze} если \begin{icloze}{2}\( V_1 \cap V_2 = \{ 0 \} \).\end{icloze}
\end{note}

\begin{note}{90c98477312541878454fb9689685fc8}
    \begin{icloze}{2}Прямая сумма подпространств \( V_1 \) и \( V_2 \)\end{icloze} обозначается \begin{icloze}{1}
        \[
            V_1 \oplus V_2.
        \]
    \end{icloze}
\end{note}

\begin{note}{951dc5cc9d7d4722ac40423e92273c7a}
    Пусть \( V_1 \) и \( V_2 \) --- два линейных подпространства. Тогда эквивалентны следующие утверждения:
    \begin{enumerate}
        \item {}\begin{icloze}{1}\( V_1 + V_2 \) --- прямая сумма;\end{icloze}
        \item {}\begin{icloze}{2}\( \dim (V_1 + V_2) = \dim V_1 + \dim V_2 \);\end{icloze}
        \item {}\begin{icloze}{3}Для любого \( a \in V_1 + V_2 \) разложение разложение \( a \) в сумму \( v_1 + v_2 \), где \( v_1 \in V_1, v_2 \in V_2 \), единственно.\end{icloze}
    \end{enumerate}
\end{note}

\begin{note}{fc93fb548c854d70af3f9cf3017866cb}
    В чем основная идея доказательства того, что если для любого \( a \in V_1 + V_2 \) разложение разложение \( a \) в сумму \( v_1 + v_2 \), где \( v_1 \in V_1, v_2 \in V_2 \), единственно, то \( V_1 + V_2 \) --- прямая сумма?

    \begin{cloze}{1}
        Показать, что если \( a = v_1 + v_2 \in V_1 \cap V_2 \), то \( v_1 = v_2 = 0 \).
    \end{cloze}
\end{note}

\begin{note}{78239c298e504fa9841235fdd06ac419}
    \subsubsection{<<\begin{icloze}{3}Монотонность размерности подпространств\end{icloze}>>}

    Пусть \( W \) --- линейное пространство, \( V \triangleleft W \). Тогда
    \begin{enumerate}
        \item {}\begin{icloze}{1}\( \dim V \leqslant \dim W \),\end{icloze}
        \item {}\begin{icloze}{2}\( \dim V = \dim W \iff V = W \).\end{icloze}
    \end{enumerate}
\end{note}

\begin{note}{a6b854ec7f5b4473a76276e0bff1e272}
    \begin{icloze}{3}Отображение \( f : V \to W \)\end{icloze} называется \begin{icloze}{2}линейным отображением,\end{icloze} если \begin{icloze}{1}
        \begin{enumerate}
            \item \( f(x + y) = f(x) + f(y), \quad \forall x, y \in V \),
            \item \( f(\lambda x) = \lambda f(x), \quad \forall \lambda \in \mathbb R, x \in V \).
        \end{enumerate}
    \end{icloze}
\end{note}

\begin{note}{4008d3f9d2224ec38cb2e9b8a78aab64}
    Линейное отображение так же ещё называют \begin{icloze}{1}линейным оператором.\end{icloze}
\end{note}

\begin{note}{df5862f6f1d4456cb943a7f07c8d8b68}
    Линейный оператор \( f : V \to W \) называется \begin{icloze}{1}и\-зо\-мор\-физм\-ом линейных пространств\end{icloze} тогда и только тогда, когда \begin{icloze}{2}\( f \) --- биекция.\end{icloze}
\end{note}

\begin{note}{d8bd78dfda034119ae049b476da96449}
    Линейные пространства \( V \) и \( W \) называются \begin{icloze}{1}и\-зо\-мор\-фны\-ми\end{icloze} тогда и только тогда, когда \begin{icloze}{2}
        существует изоморфизм
        \[
            f : V \to W.
        \]
    \end{icloze}
\end{note}

\begin{note}{2d4f456313e24261b688216f4b7f199e}
    Отношение \begin{icloze}{2}изоморфности\end{icloze} обозначается символом \begin{icloze}{1}
        \[
            \simeq
        \]
    \end{icloze}
\end{note}

\begin{note}{7112c4ddaf614005b6a37c3f4fbd3edc}
    Если \( f : V \to W \) --- изоморфизм, то \( f^{-1} : W \to V \) \begin{icloze}{1}--- тоже изоморфизм.\end{icloze}
\end{note}

\begin{note}{b439505227ea4814b084a811815b59d3}
    Отношение изоморфности удовлетворяет аксиомам отношения \begin{icloze}{1}эквивалентности.\end{icloze}
\end{note}

\begin{note}{9fa02b16e5e74fcea192355d84b99109}
    Пусть \( V, W \) --- конечномерные линейные пространства. Тогда
    \[
        \begin{icloze}{2}V \simeq W\end{icloze} \begin{icloze}{3}\iff\end{icloze} \begin{icloze}{1}\dim V = \dim W.\end{icloze}
    \]
\end{note}

\begin{note}{13b90eb2ff704cc69e067a3f047966cc}
    Пусть \( f : V \to W \) --- линейный оператор. Тогда \begin{icloze}{2}матрицей линейного оператора \( f \) в паре базисов в \( V \) и \( W \) со\-от\-вет\-ствен\-но\end{icloze} называют \begin{icloze}{1}матрицу \( A \), переводящую координаты любого вектора \( v \in V \) в координаты вектора \( f(v) \in W \) в соответствующих базисах.\end{icloze}
\end{note}

\begin{note}{d8ecf4d0e7a546668528944588ba6060}
    \subsubsection{<<\begin{icloze}{2}Теорема о матрице линейного оператора\end{icloze}>>}
    Пусть \( f : V \to W \) --- линейный оператор,
    \begin{itemize}
        \item {}\begin{icloze}{3}\( e_1, e_2, \ldots, e_n  \)\end{icloze} --- базис в \( V \),
        \item {}\begin{icloze}{3}\( \tilde e_1, \tilde e_2, \ldots, \tilde e_m  \)\end{icloze} --- базис в \( W \).
    \end{itemize}
    Как в явном виде задать матрицу оператора \( f \) в этих базисах?

    \begin{cloze}{1}
        \( j \)-ый столбец --- это координаты вектора \( f(e_j) \) в базисе \( \tilde e_1, \tilde e_2, \ldots, \tilde e_m \).
    \end{cloze}
\end{note}

\begin{note}{1235d9dc6038426387ee1c7475309a4f}
    Как можно компактно перефразировать утверждение теоремы о матрице линейного оператора?

    \begin{cloze}{1}
        \[
            f(e) = \tilde e A.
        \]
    \end{cloze}
\end{note}

\begin{note}{8e1ba2b68d414caeb7d229ba34833e8d}
    В чем ключевая идея доказательства теоремы о матрице линейного оператора?

    \begin{cloze}{1}
        \[
            f(e\lambda) = f(e)\lambda = \tilde e A \lambda,
        \]
        где \( \lambda \) --- координаты вектора из \( V \) в базисе \( e \).
    \end{cloze}
\end{note}

\begin{note}{b595ad9b198f46299eb5af10d49e413d}
    Композиция линейных операторов --- тоже \begin{icloze}{1}линейный оператор.\end{icloze}
\end{note}

\begin{note}{c13a12af79d9432ab1df0d1bab6f905c}
    Матрица композиции линейных операторов есть \begin{icloze}{1}произведение матриц этих операторов.\end{icloze}
\end{note}

\section{Лекция 21.02.22}
\begin{note}{13db7f12a2e14ffca2f5e09197cd3e07}
    Пусть \( f : V \to W \) --- линейный оператор,  \( A \) --- матрица оператора \( f \) в базисах \( e \) и \( \tilde e \) соответственно. Как преобразуется матрица \( A \) при замене базисов \( e \to e', \tilde e \to \tilde e'? \)

    \begin{cloze}{1}
        \[
            A' = C^{-1}_{\tilde e \to \tilde e'} \; A \; C_{e \to e'}.
        \]
    \end{cloze}
\end{note}

\begin{note}{015e02c15f134a53b50a24729fb6ac3d}
    Пусть \( f : V \to V \) --- линейный оператор,  \( A \) --- матрица оператора \( f \) в базисе \( e \). Как преобразуется матрица \( A \) при замене базиса \( e \to e' \)?

    \begin{cloze}{1}
        \[
            A' = C^{-1}_{e \to e'} \; A \; C_{e \to e'}.
        \]
    \end{cloze}
\end{note}

\begin{note}{e3c3292adefb4657a177843c8840476d}
    Пусть \( f : V \to V \) --- линейный оператор, \( A \) и \( A' \) --- матрицы оператора \( f \) в двух базисах \( e \) и \( e' \) соответственно.
    Тогда \( \det A' = \begin{icloze}{1}\det A\end{icloze} \).
\end{note}

\begin{note}{79b8fed369c447dfb53f352258ed6940}
    \begin{icloze}{2}Определителем оператора \( f : V \to V \)\end{icloze} называется \begin{icloze}{1}о\-пре\-де\-ли\-тель матрицы оператора \( f \) в произвольном базисе.\end{icloze}
\end{note}

\begin{note}{79b8fed369c447dfb53f352258ed6940}
    \begin{icloze}{2}Рангом оператора \( f : V \to V \)\end{icloze} называется \begin{icloze}{1}ранг матрицы оператора \( f \) в произвольном базисе.\end{icloze}
\end{note}

\begin{note}{d36be29fb7a342599a7f73709043bb1f}
    \begin{icloze}{2}След матрицы \( A \)\end{icloze} обозначается \begin{icloze}{1}\( \operatorname{tr}  A \).\end{icloze}
\end{note}

\begin{note}{3c423489fc4f422aaa906fbcc2041ec3}
    Пусть \( A \in \begin{icloze}{3}\mathbb R ^{n \times n}\end{icloze} \). Тогда \( \displaystyle \begin{icloze}{2}\operatorname{tr} A\end{icloze} \overset{\text{def}}= \begin{icloze}{1}\sum_{i=1}^{n} a_{ii}\end{icloze} \).
\end{note}

\begin{note}{e0b3b870a8444704a8569d15e3f761ed}
    Пусть \({ A, B \in \mathbb R^{n \times n} }\). Тогда
    \[
        \operatorname{tr} (BA) = \begin{icloze}{1}\operatorname{tr} (AB).\end{icloze}
    \]
\end{note}

\begin{note}{55e76656e4fc4920969acdfb57634355}
    \begin{icloze}{2}Следом оператора \( f : V \to V \)\end{icloze} называется \begin{icloze}{1}след матрицы оператора \( f \) в произвольном базисе.\end{icloze}
\end{note}

\begin{note}{1da0c4fffac341f89821707b4a1b38a6}
    Пусть \( f : V \to W \) --- линейный оператор. Тогда
    \[
        \begin{icloze}{2}\ker f\end{icloze} \overset{\text{def}}= \begin{icloze}{1}f^{-1}(\left\{ 0 \right\}).\end{icloze}
    \]
\end{note}

\begin{note}{f8fe0ceb74f84386932c4100743fb775}
    Пусть \( f : V \to W \) --- линейный оператор. Тогда
    \[
        \begin{icloze}{2}\operatorname{im} f\end{icloze} \overset{\text{def}}= \begin{icloze}{1}f(V).\end{icloze}
    \]
\end{note}

\begin{note}{56a80e8376154f29b490e470ceac8bc3}
    Пусть \( f : V \to W \) --- линейный оператор. Можно ли утверждать, что всегда \( \ker f \triangleleft V \)?

    \begin{cloze}{1}
        Да, поскольку линейная комбинация нулей \( f \) --- тоже нуль \( f \).
    \end{cloze}
\end{note}

\begin{note}{28f55b0f2daa4b35b1859196e2d41ede}
    Пусть \( f : V \to W \) --- линейный оператор. Можно ли утверждать, что всегда \( \ker f \triangleleft W \)?

    \begin{cloze}{1}
        Нет, \( \ker f \triangleleft V \).
    \end{cloze}
\end{note}

\begin{note}{a4bde4e9272d4bef89c915f6390ca148}
    Пусть \( f : V \to W \) --- линейный оператор. Можно ли утверждать, что всегда \( \operatorname{im} f \triangleleft W \)?

    \begin{cloze}{1}
        Да, поскольку \( \forall f(u), f(v) \in \operatorname{im} f \)
        \[
            \alpha f(u) + \beta f(v) = f(\alpha u + \beta v) \in \operatorname{im} f.
        \]
    \end{cloze}
\end{note}

\begin{note}{7b17eb03a5e640f8bddefa0aaa6656c3}
    Пусть \( f : V \to W \) --- линейный оператор. Можно ли утверждать, что всегда \( \operatorname{im} f \triangleleft V \)?

    \begin{cloze}{1}
        Нет, \( \operatorname{im} f \triangleleft W \).
    \end{cloze}
\end{note}

\begin{note}{5c7bf3d386eb4fa181cdb696fc0f9ab5}
    Пусть \( f : V \to W \) --- линейный оператор. Как связаны размерности \( V \), \( \ker f \) и \( \operatorname{im} f \)?

    \begin{cloze}{1}
        \[
            \dim \ker f + \dim \operatorname{im} f = \dim V.
        \]
    \end{cloze}
\end{note}

\begin{note}{b6ef54a20af44801aceb30b556b95011}
    Пусть \( f : V \to W \) --- линейный оператор. В чем основная идея доказательства следующей формулы?
    \[
        \dim \ker f + \dim \operatorname{im} f = \dim V
    \]

    \begin{cloze}{1}
        Дополнить базис в \( \ker f \) до базиса в \( V \) и построить из них базис в \( \operatorname{im} f \).
    \end{cloze}
\end{note}

\begin{note}{26a0af100d5b4c459a74ba6384b7c554}
    Пусть \( f : V \to W \) --- линейный оператор,
    \begin{itemize}
        \item \( e_1,e_2, \ldots, e_k  \) --- базис в \( \ker f \);
        \item \( e_1,e_2, \ldots, e_k, e_{k+1}, \ldots, e_n  \) --- базис в \( V \).
    \end{itemize}
    Как выглядит базис в \( \operatorname{im} f \)?

    \begin{cloze}{1}
        \[
            f(e_{k+1}), \ldots, f(e_n).
        \]
    \end{cloze}
\end{note}

\begin{note}{8a962591377f49c1a6b297a1efe008e9}
    Пусть \( f : W \to W \) --- линейный оператор. Тогда
    \[
        \dim \operatorname{im} f = \begin{icloze}{1}\operatorname{rk} f.\end{icloze}
    \]
\end{note}

\begin{note}{2acbea4466f54360bc19e2065a44fc95}
    Пусть \( f : W \to W \) --- линейный оператор. Как показать, что
    \[
        \dim \operatorname{im} f = \operatorname{rk} f?
    \]

    \begin{cloze}{1}
        Показать, что в координатном выражении \( \operatorname{im} f \) есть линейная оболочка столбцов матрицы оператора \( f \).
    \end{cloze}
\end{note}

\begin{note}{a85a7d7b1e3d47939cc717cb8da889ac}
    Пусть \( f : W \to W \) --- линейный оператор. \begin{icloze}{1}Пространство \( V \triangleleft W \)\end{icloze} называется \begin{icloze}{2}инвариантным относительно оператора \( f \),\end{icloze} если
    \begin{icloze}{1}\[
        f(V) \subset V.
    \]\end{icloze}
\end{note}

\begin{note}{e3d31c73908d4103b6c9caf2377e4432}
    Примеры инвариантных подпространств в контексте произвольного оператора \( f : W \to W \).

    \begin{cloze}{1}
        \[
            \ker f, \operatorname{im} f.
        \]
    \end{cloze}
\end{note}

\begin{note}{e64a247c0efb47f8be38d4ab4ef17b05}
    Пусть \( f : W \to W \) --- линейный оператор,  \( e_1,e_2, \ldots, e_n  \) --- \begin{icloze}{4}дополнение до базиса в \( W \) базиса \( e_1,e_2, \ldots, e_k  \) в инвариантном подпространстве \( V \triangleleft W \).\end{icloze} Тогда \begin{icloze}{3}матрица оператора \( f \) в базисе \( e_1,e_2, \ldots, e_n  \)\end{icloze} примет вид
    \[
        A = \begin{icloze}{1}
            \begin{bmatrix}
                T_{11} & T_{12} \\
                0 & T_{22}
            \end{bmatrix},
        \end{icloze}
    \]
    где \( T_{11}  \) --- это \begin{icloze}{2}матрица \( f |_{V}  \) в базисе \( e_1,e_2, \ldots, e_k  \).\end{icloze}
\end{note}

\section{Лекция 28.02.22}
\begin{note}{9932dc2853764661928eedc8d44ddd74}
    Линейный оператор \( f : W \to W \) называется \begin{icloze}{2}невырожденным,\end{icloze} если \begin{icloze}{1}\( \det f \neq 0 \).\end{icloze}
\end{note}

\begin{note}{2e565e676da342fb8cdacf4d62de05e8}
    Пусть \( f : V \to V \) --- линейный оператор. Следующие 5 условий эквивалентны:
    \begin{enumerate}
        \item \( f \) невырождено;
        \begin{icloze}{1}
            \item \( \ker f = \left\{ 0 \right\} \);
            \item \( \operatorname{im} f = V \);
            \item \( \operatorname{rk} f = \dim V \);
            \item \( f \) --- биекция.
        \end{icloze}
    \end{enumerate}
\end{note}

\begin{note}{8f9f5108ac8847299f21fd40619c6612}
    Пусть \( f : W \to W \) --- линейный оператор. Как доказать, что если \( f \) --- невырожденный оператор, то \( f \) --- биекция?

    \begin{cloze}{1}
        Показать, что если \( f \) задаётся матрицей \( A \), то \( f^{-1} \) задаётся матрицей \( A^{-1} \).
    \end{cloze}
\end{note}

\begin{note}{0c8915aebdc24427ab211efa79c6e07a}
    Пусть \( f : W \to W \) --- линейный оператор. Как доказать, что если \( f \) --- биекция, то \( f \) --- невырожденный оператор.

    \begin{cloze}{1}
        \[
            \det (f \circ f^{-1}) = |E| \implies \det f \neq 0.
        \]
    \end{cloze}
\end{note}

\begin{note}{198b26e615c745edbd313c2f62029546}
    Пусть \begin{icloze}{3}\( f : V \to V \) --- линейный оператор.\end{icloze} Тогда \begin{icloze}{1}число \( \lambda \in \mathbb C  \)\end{icloze} называется \begin{icloze}{2}собственным значением оператора \( f \),\end{icloze} если
    \begin{icloze}{1}
        \[
            \exists v \in V \setminus \left\{ 0 \right\} \quad f(v) = \lambda v.
        \]
    \end{icloze}
\end{note}

\begin{note}{f0b8dcb8a69748a0a51393ae495884b4}
    Пусть \begin{icloze}{3}\( f : V \to V \) --- линейный оператор.\end{icloze} Тогда \begin{icloze}{1}вектор \( v \in V \setminus \left\{ 0 \right\} \)\end{icloze} называется \begin{icloze}{2}собственным вектором оператора \( f \),\end{icloze} если
    \begin{icloze}{1}
        \[
            \exists \lambda \in \mathbb C \quad f(v) = \lambda v.
        \]
    \end{icloze}
\end{note}

\begin{note}{22a614bf26ea4db3ae297b5c647e6517}
    \begin{icloze}{2}Спектром оператора\end{icloze} называется \begin{icloze}{1}множество собственных значений этого оператора.\end{icloze}
\end{note}

\begin{note}{1f331a6bd4c84dc4996f323fd40b5a22}
    \begin{icloze}{2}Спектр\end{icloze} оператора \( f \) обозначается \begin{icloze}{1}\( \operatorname{spec} f \).\end{icloze}
\end{note}

\begin{note}{ff82c9b056384c19b0a176b637c3941c}
    Пусть \begin{icloze}{3}\( f : V \to V \) --- линейный оператор,  \( \lambda \in \mathbb C  \).\end{icloze} Тогда \( \lambda \) является собственным значением \( f \) \begin{icloze}{2}тогда и только тогда, когда\end{icloze}
    \begin{icloze}{1}
        \[
            \det (f - \lambda E) = 0.
        \]
    \end{icloze}
\end{note}

\begin{note}{a96c7b61477946699a72e8a792c8bf75}
    Пусть \begin{icloze}{3}\( f : V \to V \) --- линейный оператор.\end{icloze} Тогда \begin{icloze}{2}уравнение
    \[
        \det (f - \lambda E) = 0
    \]
    \end{icloze} называется \begin{icloze}{1}характеристическим уравнением оператора \( f \).\end{icloze}
\end{note}

\begin{note}{a7a86475fc014d3c8fe1d63fa3a766ea}
    Пусть \begin{icloze}{3}\( f : V \to V \) --- линейный оператор.\end{icloze} Тогда \begin{icloze}{2}выражение
    \[
        \det (f - \lambda E)
    \]
    \end{icloze} называется \begin{icloze}{1}характеристическим многочленом оператора \( f \).\end{icloze}
\end{note}

\begin{note}{976ac89d4ea7486080b6c2c8473946d9}
    Пусть \( f : V \to V \) --- линейный оператор. Почему
    \[
        \det (f - \lambda E)
    \]
    является многочленом переменной \( \lambda \)?

    \begin{cloze}{1}
        Если \( A \) --- матрица оператора \( f \), то \( \left| A - \lambda E \right|  \) --- многочлен переменной \( \lambda \).
    \end{cloze}
\end{note}

\begin{note}{5376672e8b21438896bc774aa4ac2275}
    Пусть
    \[
        A = \begin{bmatrix}
            a_{11} & a_{12} \\
            a_{21} & a_{22}
        \end{bmatrix}.
    \]
    Тогда
    \[
        \begin{icloze}{2}\left| A - \lambda E \right|\end{icloze}
        = \begin{icloze}{1}|A| - \lambda \operatorname{tr} A + \lambda^2.\end{icloze}
\]
\end{note}

\section{Лекция 07.03.22}
\begin{note}{0d6c679eb377462e90e8ac9bba29dd61}
    Пусть \( f : W \to W \) --- линейный оператор.
    \begin{icloze}{2}Характеристический многочлен оператора \( f \)\end{icloze} обозначается \begin{icloze}{1}
    \[
        \chi_f.
    \]
\end{icloze}
\end{note}

\begin{note}{78106143b649485eb1c075b2388eb22e}
    Пусть \begin{icloze}{3}\( f : W \to W \) --- линейный оператор и \( V \triangleleft W \) инвариантно относительно \( f \).\end{icloze}
    Тогда
    \begin{center}
        \begin{icloze}{2}\( \chi_{f|_V} \)\end{icloze} --- \begin{icloze}{1}делитель \( \chi_f \).\end{icloze}
    \end{center}
\end{note}

\begin{note}{6deeef304fd8465bbff331e4241bde67}
    Пусть \( f : W \to W \) --- линейный оператор и \( V \triangleleft W \) инвариантно относительно \( f \).
    Тогда
    \begin{center}
        \( \chi_{f|_V} \) --- делитель \( \chi_f \).
    \end{center}

    В чем основная идея доказательства?

    \begin{cloze}{1}
        Показать, что \( \chi_f \) --- определитель соответствующей квазитреугольной матрицы оператора \( f \).
    \end{cloze}
\end{note}

\begin{note}{cdb0a7bde4e044e48a5a798a8052f163}
    Пусть \begin{icloze}{3}\( f : W \to W \) --- линейный оператор, \( \lambda \in \operatorname{spec} f \).\end{icloze}
    \begin{icloze}{1}Множество всех собственных векторов \( f \) с собственным значением \( \lambda \), объединённое с нулём,\end{icloze} обозначается \begin{icloze}{2}\( V_{f} (\lambda) \).\end{icloze}
\end{note}

\begin{note}{785c107694984499a5fd89afd052841c}
    Пусть \( f : W \to W \) --- линейный оператор, \( \lambda \in \operatorname{spec} f \).
    Тогда \begin{icloze}{2}\( V_f(\lambda) \)\end{icloze} называется \begin{icloze}{1}собственным подпространством оператора \( f \).\end{icloze}
\end{note}

\begin{note}{545e4fc3988d45fdafc099f74fe38f36}
    Пусть \( f : W \to W \) --- линейный оператор, \( \lambda \) --- собственное значение \( f \).
    В кратком выражении
    \[
        \begin{icloze}{2}V_f(\lambda)\end{icloze} \overset{\text{def}}= \begin{icloze}{1}\ker (f - \lambda E).\end{icloze}
    \]
\end{note}

\begin{note}{edf7cad1b7df422181105ad8bf31a210}
    Пусть \( f : W \to W \) --- линейный оператор, \( \lambda \) --- собственное значение \( f \).
    Всегда ли
    \[
        V_{f} (\lambda) \triangleleft W?
    \]

    \begin{cloze}{1}
        Да, всегда, потому что \( V_{f} (\lambda) = \ker (f - \lambda E) \).
    \end{cloze}
\end{note}

\begin{note}{de964305c22b4993819a8d5095504e53}
    Пусть \( f : V \to V \) --- линейный оператор, \( \lambda \) --- собственное значение \( f \).
    \begin{icloze}{1}Размерность \( V_f (\lambda) \)\end{icloze} называют \begin{icloze}{2}геометрической кратностью собственного значения \( \lambda \).\end{icloze}
\end{note}

\begin{note}{f6b8139d2f0e46d38a2dd075ff83b2f4}
    Пусть \( f : V \to V \) --- линейный оператор, \( \lambda \) --- собственное значение \( f \).
    \begin{icloze}{2}Геометрическая кратность собственного значения \( \lambda \)\end{icloze} обозначается \begin{icloze}{1}\( S_f (\lambda) \).\end{icloze}
\end{note}

\begin{note}{eff6d05e42b34f078450044f6153939b}
    Пусть \( f : V \to V \) --- линейный оператор, \( \lambda \) --- собственное значение \( f \).
    \begin{icloze}{1}Кратность \( \lambda \) как корня \( \chi_f \)\end{icloze} называют \begin{icloze}{2}алгебраической кратностью собственным значением \( \lambda \).\end{icloze}
\end{note}

\begin{note}{856a933db82641cd87b0ee5f34647b1a}
    Пусть \( f : V \to V \) --- линейный оператор, \( \lambda \) --- собственное значение \( f \).
    \begin{icloze}{2}Алгебраическая кратность собственного значения \( \lambda \)\end{icloze} обозначается \begin{icloze}{1}\( m_f (\lambda) \).\end{icloze}
\end{note}

\begin{note}{b7431a88515043deacf49cf7fdb735c6}
    Пусть \( f : V \to V \) --- линейный оператор, \( \lambda \) --- собственное значение \( f \).
    Тогда \begin{icloze}{1}\( S_f (\lambda) \leqslant m_f (\lambda) \).\end{icloze}
\end{note}

\begin{note}{6b913f908a194114bee71fb9a7526282}
    Пусть \( f : V \to V \) --- линейный оператор, \( \lambda \) --- собственное значение \( f \).
    Тогда \( S_f (\lambda) \leqslant m_f (\lambda) \).

    В чем основная идея доказательства?

    \begin{cloze}{1}
        Показать, что \( V_f (\lambda) \) инвариантно относительно \( f \) \\
        \phantom{} \hfill \( \implies \) \( \chi_f \) делится на \( \chi_{\tilde f} \), где \( \tilde f = f|_{V_f(\lambda)} \).
    \end{cloze}
\end{note}

\begin{note}{58579b404ae34478b736df96c853c6e6}
    Пусть \( f : V \to V \) --- линейный оператор, \( \lambda \) --- собственное значение \( f \), \: \begin{icloze}{2}\( \tilde f = f|_{V_f(\lambda)} \).\end{icloze}
    Тогда
    \[
        \begin{icloze}{3}\chi_{\tilde f} (t)\end{icloze} = \begin{icloze}{1}(\lambda - t)^{S_f(\lambda)}\end{icloze}
    \]
\end{note}

\begin{note}{8d63ff53045545709809018e1492b231}
    Пусть \( f : V \to V \) --- линейный оператор, \( \lambda \) --- собственное значение \( f \), \: \( \tilde f = f|_{V_f(\lambda)} \).
    Откуда следует, что
    \[
        \chi_{\tilde f} (t) = (\lambda - t)^{S_f(\lambda)} \quad?
    \]

    \begin{cloze}{1}
        \( \tilde f \) представляется  матрицей \( \lambda E \) порядка \( \dim V_f(\lambda) \).
    \end{cloze}
\end{note}

\begin{note}{a3b9ba1c4e884a7bb1e3c4764f063d1f}
    \begin{icloze}{2}Оператор \( f : x \mapsto \lambda x \), где \( \lambda \in \mathbb R \),\end{icloze} называется \begin{icloze}{1}скалярным оператором.\end{icloze}
\end{note}

\begin{note}{51a455604c9c4d7eadc3fe5ab0af6397}
    Пусть \begin{icloze}{3}\( f : V \to V \) --- линейный оператор.\end{icloze}
    \( f \) называется \begin{icloze}{2}диагонализуемым оператором,\end{icloze} если \begin{icloze}{1}существует базис в \( V \), в котором матрица оператора \( f \) является диагональной.\end{icloze}
\end{note}

\begin{note}{b01b69fc3ebc4c0c839a0c153f85d041}
    \begin{icloze}{1}Диагональная матрица с элементами \( a_1, a_2, \ldots, a_n \) на диагонали\end{icloze} обозначается
    \begin{icloze}{2}
        \[
            \operatorname{diag}(a_1, a_2, \ldots, a_n).
        \]
    \end{icloze}
\end{note}

\begin{note}{8066b576097a49fb9d5aa3c4580a27c5}
    Пусть \( f : V \to V \)--- линейный оператор.
    Если в базисе \( e_1, e_2, \ldots, e_n \) матрица оператора \( f \) равна \( \operatorname{diag} (a_1, a_2, \ldots, a_n) \), то \begin{icloze}{2}\( e_1, e_2, \ldots, e_n \)\end{icloze} --- \begin{icloze}{1}собственные векторы \( f \).\end{icloze}
\end{note}

\begin{note}{19e6a7fb9c8e4f04a3711d479f2c628e}
    Пусть \( f : V \to V \)--- линейный оператор.
    Если в базисе \( e_1, e_2, \ldots, e_n \) матрица оператора \( f \) равна \( \operatorname{diag} (a_1, a_2, \ldots, a_n) \), то \begin{icloze}{2}\( a_1, a_2, \ldots, a_n \)\end{icloze} --- \begin{icloze}{1}собственные значения \( f \).\end{icloze}
\end{note}

\begin{note}{1176411a2bf147348b94dd69b9bbad73}
    Пусть \begin{icloze}{4}\( f : V \to V \) --- линейный оператор.\end{icloze} Тогда оператор \( f \) \begin{icloze}{2}диагонализуем\end{icloze} \begin{icloze}{3}тогда и только тогда, когда\end{icloze} \begin{icloze}{1}для любого собственного значения \( \lambda \)
    \[
        S_f(\lambda) = m_f(\lambda).
    \]\end{icloze}
\end{note}

\begin{note}{ca827a11abb047fda276763e1e593ef1}
    В чем основная идея доказательства критерия диагонализуемости оператора (необходимость)?

    \begin{cloze}{1}
        Покзать, что если \( f \) представляется матрицей \( \operatorname{diag} (a_1, a_2, \ldots, a_n) \), то по определению
        \[
            \chi_f (\lambda) = \prod_{i = 1}^{n} (a_i - \lambda).
        \]
    \end{cloze}
\end{note}

\begin{note}{0fc84832f2b548cfa9a4ef9a51326b77}
    Пусть \( f : V \to V \) --- линейный оператор, \begin{icloze}{3}\( \lambda_1, \ldots, \lambda_n \) --- различные собственные значения оператора \( f \),\end{icloze} \begin{icloze}{2}
        \[
            \forall j \quad v_j \in V_f (\lambda_j).
        \]
    \end{icloze}
    Тогда \begin{icloze}{1}система векторов \( v_1, \ldots, v_n \) линейно независима.\end{icloze}
\end{note}

\begin{note}{2a1e5294e5c34d889ca747ab0b44fa0a}
    Пусть \( f : V \to V \) --- линейный оператор, \( \lambda_1, \ldots, \lambda_n \) --- различные собственные значения оператора \( f \),
    \[
        \forall j \quad v_j \in V_f (\lambda_j).
    \]
    Тогда система векторов \( v_1, \ldots, v_n \) линейно независима.

    В чем основная идея доказательства?

    \begin{cloze}{1}
         Применяем \( f \) к произвольной равной нулю линейной комбинации, пока не получится СЛАУ с основной матрицей --- определителем Вандермонда.
    \end{cloze}
\end{note}

\begin{note}{cfe344113f4e40b2b27ecfee11beb647}
    В чем основная идея доказательства критерия диагонализуемости оператора (достаточность)?

    \begin{cloze}{1}
        Составить систему векторов из базисов в \( V_f (\lambda_j) \) и показать, что она является базисом \( V \).
    \end{cloze}
\end{note}

\begin{note}{fbb72d710ce84fe6b5237ee1f15112a8}
    Почему система векторов, составленная в доказательстве критерия диагонализуемости оператора (достаточность), является порождающей?

    \begin{cloze}{1}
        Из условия \( \dim V_f (\lambda_j) = m_f (\lambda_j) \), а значит система содержит \( \deg \chi_f = \dim V \) элементов.
    \end{cloze}
\end{note}

\begin{note}{5fd54902e7f34d00bad3222902a6bdf6}
    Почему система векторов, составленная в доказательстве критерия диагонализуемости оператора (достаточность), является линейно независимой?

    \begin{cloze}{1}
        Любая её линейная комбинация есть линейная комбинация системы векторов \( v_1, \ldots, v_n \), где \( v_j \in V_f (\lambda_j) \).
    \end{cloze}
\end{note}

\begin{note}{435490ce764048d9a55b762d6175cf59}
    Если оператор \( f : V  \to V \) имеет \( \dim V \) различных собственных значений, то \begin{icloze}{1}\( f \) диагонализуем.\end{icloze}
\end{note}

\begin{note}{8757ff57337847268575f5903d640f08}
    Как доказать, что если оператор \( f : V \to V \) имеет \( \dim V \) различных собственных значений, то \( f \) диагонализуем.

    \begin{cloze}{1}
        \begin{multline*}
            \forall \lambda \in \operatorname{spec} f \quad 1 \leqslant S_f(\lambda) \leqslant m_f(\lambda) = 1 \\
            \implies  S_f (\lambda) = m_f (\lambda).
        \end{multline*}
    \end{cloze}
\end{note}

\begin{note}{b7cd455d24424dd0879b90d7cad89a6b}
    Пусть \begin{icloze}{3}пространство \( V = V_1 \oplus V_2 \).\end{icloze}
    \begin{icloze}{1}Оператор \( P : V \to V \), переводящий сумму \( v_1 + v_2 \) векторов из \( V_1 \) и \( V_2 \) соответственно в вектор \( v_1 \),\end{icloze} называется \begin{icloze}{2}оператором проектирования на \( V_1 \) параллельно \( V_2 \).\end{icloze}
\end{note}

\begin{note}{522c1911d5d04c898b070c53537026b2}
    Пусть \( V = V_1 \oplus V_2 \) и \( P : V \to V \) --- оператор проектирования на \( V_1 \) параллельно \( V_2 \).
    Тогда
    \[
        \operatorname{im} P = \begin{icloze}{1}V_1.\end{icloze}
    \]
\end{note}

\begin{note}{0e8f1308502e41f9bfbddc3a9a153514}
    Пусть \( V = V_1 \oplus V_2 \) и \( P : V \to V \) --- оператор проектирования на \( V_1 \) параллельно \( V_2 \).
    Тогда
    \[
        \ker P = \begin{icloze}{1}V_2.\end{icloze}
    \]
\end{note}

\begin{note}{27181bd7474e4091aee4fa9dba20ae0f}
    Пусть \( V = V_1 \oplus V_2 \) и \( P : V \to V \) --- оператор проектирования на \( V_1 \) параллельно \( V_2 \).
    Тогда
    \[
        \operatorname{spec} P = \begin{icloze}{1}\left\{ 0, 1 \right\}.\end{icloze}
    \]
\end{note}

\begin{note}{448f428dbef544a9a7ad66228e473bea}
    Пусть \( V = V_1 \oplus V_2 \) и \( P : V \to V \) --- оператор проектирования на \( V_1 \) параллельно \( V_2 \).
    Тогда
    \[
        m_P (0) = \begin{icloze}{1}\dim V_2.\end{icloze}
    \]
\end{note}

\begin{note}{d4a2a9780d1a4e1db35238e91f3875b9}
    Пусть \( V = V_1 \oplus V_2 \) и \( P : V \to V \) --- оператор проектирования на \( V_1 \) параллельно \( V_2 \).
    Тогда
    \[
        S_P (0) = \begin{icloze}{1}\dim V_2.\end{icloze}
    \]
\end{note}

\begin{note}{322376ccf5e4418bb64b5e8b886d8aac}
    Пусть \( V = V_1 \oplus V_2 \) и \( P : V \to V \) --- оператор проектирования на \( V_1 \) параллельно \( V_2 \).
    Тогда
    \[
        m_P (1) = \begin{icloze}{1}\dim V_1.\end{icloze}
    \]
\end{note}

\begin{note}{c81e19cdfaa649f18565d2f7625646ce}
    Пусть \( V = V_1 \oplus V_2 \) и \( P : V \to V \) --- оператор проектирования на \( V_1 \) параллельно \( V_2 \).
    Тогда
    \[
        S_P (1) = \begin{icloze}{1}\dim V_1.\end{icloze}
    \]
\end{note}

\section{Лекция 14.03.22}
\begin{note}{d32917879c284285842d17bbfc251d30}
    Пусть \begin{icloze}{3}\( f : V \to V \) --- линейный оператор, \( v \in V \), \( k \in \mathbb N \).\end{icloze} Вектор \( v \) называется \begin{icloze}{2}корневым вектором высоты \( k \) оператора \( f \),\end{icloze} если \begin{icloze}{1}существует такое \( \lambda \in \mathbb C \), что
    \[
        \begin{gathered}
            (f - \lambda E)^{k} v = 0, \\
            (f - \lambda E)^{k - 1} v \neq 0.
        \end{gathered}
    \]\end{icloze}
\end{note}

\begin{note}{83d2e0cc0a894b54ac4d3604babf2d57}
    Корневой вектор высоты \begin{icloze}{2}\( 1 \)\end{icloze} оператора \( f \) --- это \begin{icloze}{1}собственный вектор этого оператора.\end{icloze}
\end{note}

\begin{note}{9e3747b6754c4bad9076277f39c4e920}
    \( \lambda \) из определения корневого вектора оператора \( f \) --- это всегда \begin{icloze}{1}собственное значение \( f \).\end{icloze}
\end{note}

\begin{note}{a4093e0c9f55478ebd2eb2defda323df}
    Как показать, что \( \lambda \) из определения корневого вектора всегда является собственным вектором?

    \begin{cloze}{1}
        Из определения \( (f - \lambda E)^{k} v = 0 \implies \det (f - \lambda E) = 0 \).
    \end{cloze}
\end{note}

\begin{note}{999c7f68724546db81750f9e997d0a1b}
    Пусть \begin{icloze}{3}\( v \) --- корневой вектор высоты \( k \geqslant 2 \) оператора \( f \).\end{icloze}
    Тогда \begin{icloze}{2}\( (f - \lambda E) v \)\end{icloze} --- \begin{icloze}{1}корневой вектор высоты \( k - 1 \).\end{icloze}
\end{note}

\begin{note}{264901faf0bb401e91105512f04f06dc}
    Пусть \( v \) --- корневой вектор высоты \( k \geqslant 2 \) оператора \( f \).
    Тогда \( (f - \lambda E) v \) --- корневой вектор высоты \( k - 1 \).
    В чем основная идея доказательства?

    \begin{cloze}{1}
        Из определения корневого вектора
        \[
            (f - \lambda E)^{k - 1} \cdot (f - \lambda E)v = 0
        \]
        и аналогично с неравенством нулю для степени \( k - 2 \).
    \end{cloze}
\end{note}

\begin{note}{50c2388c1fa843dfa616f85d4cecfa2f}
    Система \begin{icloze}{3}корневых векторов разных высот,\end{icloze} отвечающих \begin{icloze}{2}одному и тому же собственному значению оператора,\end{icloze} \begin{icloze}{1}линейно независима.\end{icloze}
\end{note}

\begin{note}{de47eb56e219455a8497a97ad90b861d}
    Как доказать, что система корневых векторов разных высот, отвечающих одному и тому же собственному значению оператора, линейно независима.

    \begin{cloze}{1}
        Приравнять линейную комбинацию к нулю и домножать её на \( (f - \lambda E)^{k_j - 1} \) в порядке убывания высот \( k_j \) корневых векторов системы.
    \end{cloze}
\end{note}

\begin{note}{187218f20c2b46ab9309b3385f2012f4}
    Пусть \begin{icloze}{3}\( v \) --- корневой вектор высоты \( k \geqslant 2 \) оператора \( f \).\end{icloze}
    Тогда система
    \begin{icloze}{2}
        \[
            v,\: (f - \lambda E) v,\: (f - \lambda E)^{2} v,\: \ldots,\: (f - \lambda E)^{k - 1} v
        \]
    \end{icloze}
    \begin{icloze}{1}линейно независима.\end{icloze}
\end{note}

\begin{note}{f77f36f44a0a4dbfb7fe6d8a6b58db75}
    Пусть \( v \) --- корневой вектор высоты \( k \geqslant 2 \) оператора \( f \).
    Тогда система
    \[
        v,\: (f - \lambda E) v,\: (f - \lambda E)^{2} v,\: \ldots,\: (f - \lambda E)^{k - 1} v
    \]
    линейно независима.
    В чем основная идея доказательства?

    \begin{cloze}{1}
        Показать, что это система корневых векторов разных высот, отвечающих одному и тому же собственному значению \( \lambda \).
    \end{cloze}
\end{note}

\begin{note}{3ab579b8e03a47ec865a43fc21bd39b7}
    Система \begin{icloze}{3}корневых векторов,\end{icloze} отвечающих \begin{icloze}{2}разным собственным значениям оператора,\end{icloze} \begin{icloze}{1}линейно независима.\end{icloze}
\end{note}

\begin{note}{04c77a5799504d088141691461b44095}
    Пусть \( v \) --- корневой вектор высоты \( k \) оператора \( f \). Тогда \begin{icloze}{2}\( (f - \lambda E)^{k - 1}v \)\end{icloze} --- \begin{icloze}{1}это собственный вектор оператора \( f \).\end{icloze}
\end{note}

\begin{note}{59e9653333744cccaf670372a881ab06}
    Как доказать, что система корневых векторов, отвечающих разным собственным значениям оператора, линейно независима.

    \begin{cloze}{1}
        Домножить произвольную линейную комбинацию на
        \[
            (f - \lambda_1 E)^{k_1 - 1} \;\; (f - \lambda_2 E)^{k_2} \cdots (f - \lambda_l E)^{k_l}
        \]
        и получить равенство нулю первого коэффициента. Далее аналогично для остальных коэффициентов.
    \end{cloze}
\end{note}

\begin{note}{5b16ae3e6ef643508aa2e1f086ffde51}
    Пусть \( f : V \to V \) --- линейный оператор, \( \lambda \in \operatorname{spec} f \).
    \begin{icloze}{1}Множество всех корневых векторов, отвечающих собственному значению \( \lambda \), объединённое с нулём,\end{icloze} называется \begin{icloze}{2}корневым подпространством, отвечающим собственному значению \( \lambda \).\end{icloze}
\end{note}

\begin{note}{2779025573314db7aa326077599c90b3}
    Пусть \( f : V \to V \) --- линейный оператор.
    \begin{icloze}{2}Корневое подпространство, отвечающее собственному значению \( \lambda \),\end{icloze} обозначается \begin{icloze}{1}\( K_f (\lambda) \).\end{icloze}
\end{note}

\begin{note}{d70f06c975144b6e835736be38336c4a}
    Пусть \( f : V \to V \) --- линейный оператор, \( \lambda \in \operatorname{spec} f \).
    Всегда ли \( K_f(\lambda) \triangleleft V \)?

    \begin{cloze}{1}
        Да, всегда (тривиально следует из определения).
    \end{cloze}
\end{note}

\begin{note}{e3330d597cd547a385f694495c2dc291}
    Пусть \begin{icloze}{3}\( f : V \to V \) --- линейный оператор, \( k \in \mathbb N \).\end{icloze}
    \[
        \begin{icloze}{2}N_{f,k}(\lambda)\end{icloze} \overset{\text{def}}= \begin{icloze}{1}\ker (f - \lambda E)^{k}.\end{icloze}
    \]
\end{note}

\begin{note}{42d32fc206824eafb2be52cb821ffafd}
    Пусть \( f : V \to V \) --- линейный оператор, \( k \in \mathbb N \).
    Всегда ли \( N_{f,k}(\lambda) \triangleleft V \)?

    \begin{cloze}{1}
        Да, всегда (тривиально следует из определения).
    \end{cloze}
\end{note}

\begin{note}{ba89f8d6240947edac91e39df44d92bc}
    Пусть \( f : V \to V \) --- линейный оператор, \( \lambda \in \operatorname{spec} f \).
    Как \( K_f(\lambda) \) выражается через \( N_{f,k} (\lambda) \)?

    \begin{cloze}{1}
        \[
            K_f(\lambda) = \bigcup_{k \geqslant 1}^{} N_{f,k}(\lambda)
        \]
    \end{cloze}
\end{note}

\begin{note}{c11610dbf64143fbaeeb57dfc3d66af0}
    Пусть \( f : V \to V \) --- линейный оператор, \( \lambda \in \operatorname{spec} f \).
    Тогда \( \dim K_f(\lambda) = \begin{icloze}{1}m_f(\lambda)\end{icloze} \).
\end{note}

\begin{note}{efee3536114a40d28eb925c540f796bf}
    Пусть \( f : V \to V \) --- линейный оператор, \( \lambda \in \operatorname{spec} f \).
    Тогда \( \dim K_f(\lambda) = m_f(\lambda) \).
    В чем основная идея доказательства?

    TODO (?)
\end{note}

\begin{note}{d928d6cded3a434a9c3f615c48190ff0}
    Пусть \begin{icloze}{3}\( f : V \to V \) --- линейный оператор, \( \lambda_1, \ldots, \lambda_l \) --- все различные собственные значения \( f \).\end{icloze}
    Тогда
    \[
        \begin{icloze}{2}V\end{icloze} = \begin{icloze}{1}K_f (\lambda_1) \oplus \cdots \oplus K_f (\lambda_l).\end{icloze}
    \]
\end{note}

\begin{note}{16e24ba8ea07492581171c4ee92a6c95}
    Пусть \( f : V \to V \) --- линейный оператор, \( \lambda_1, \ldots, \lambda_l \) --- все различные собственные значения \( f \).
    Тогда
    \[
        V = K_f (\lambda_1) \oplus \cdots \oplus K_f (\lambda_l).
    \]
    В чем основная идея доказательства?

    \begin{cloze}{1}
        Показать, что сумма \( K_f(\lambda_j) \)
        \begin{enumerate}
            \item является прямой,
            \item порождает все пространство \( V \).
        \end{enumerate}
    \end{cloze}
\end{note}

\begin{note}{12b8b0705d6b4daf886d155d26b8d4f4}
    Пусть \( f : V \to V \) --- линейный оператор, \( \lambda_1, \ldots, \lambda_l \) --- все различные собственные значения \( f \).
    Тогда
    \[
        V = K_f (\lambda_1) \oplus \cdots \oplus K_f (\lambda_l).
    \]
    Почему сумма \( K_f(\lambda_j) \) прямая?

    \begin{cloze}{1}
        Линейная комбинация векторов \( v_j \) из \( K_f (\lambda_j) \) --- это линяния комбинация корневых векторов, отвечающих разным собственным значениям.
    \end{cloze}
\end{note}

\begin{note}{ae4ac17697d146c194afc0f17091b028}
    Пусть \( f : V \to V \) --- линейный оператор, \( \lambda_1, \ldots, \lambda_l \) --- все различные собственные значения \( f \).
    Тогда
    \[
        V = K_f (\lambda_1) \oplus \cdots \oplus K_f (\lambda_l).
    \]
    Почему сумма \( K_f(\lambda_j) \) порождает все \( V \)?

    \begin{cloze}{1}
        \[
            \sum_{j=1}^{l} \dim K_f(\lambda_j) = \sum_{j=1}^{l} m_f(\lambda_j)
        \]
    \end{cloze}
\end{note}

\begin{note}{e23c324999e1436d8c6d50a246244d60}
    \begin{icloze}{2}Жорданова клетка\end{icloze} --- это \begin{icloze}{1}квадратная матрица вида
    \[
        \begin{bmatrix}
            \lambda & 1 & 0 & \cdots & 0 \\
            0 & \lambda & 1 & \cdots & 0 \\
            0 & 0 & \lambda & \cdots & 0 \\
            \vdots & \vdots & \vdots & \ddots & \vdots \\
            0 & 0 & 0 & \cdots & \lambda
        \end{bmatrix}.
    \]\end{icloze}
\end{note}

\begin{note}{d354e3255a1a46e99261a422c4e41207}
    Жорданова клетка высоты \( q \), соответствующая некоторому числу \( \lambda \), обозначается
    \begin{icloze}{1}
        \[
            J_q (\lambda).
        \]
    \end{icloze}
\end{note}

\begin{note}{49446743b36c41b2825ed009c2fe6cd6}
    \begin{icloze}{2}Жорданова матрица\end{icloze} --- это \begin{icloze}{1}блочно-диагональная матрица, составленная из жордановых клеток.\end{icloze}
\end{note}

\begin{note}{c2e8392343e8487288fc8b5d700aeafa}
    Пусть \( f : V \to V \) --- линейный оператор.
    Тогда, если \begin{icloze}{1}в некотором базисе в \( V \) матрица \( A \) оператора \( f \) имеет жорданов вид,\end{icloze} то \( A \) называют \begin{icloze}{2}жордановой нормальной формой оператора \( f \).\end{icloze}
\end{note}

\begin{note}{4e0cce0726054534a2f4f1fa1beaffbb}
    Пусть \( f : V \to V \) --- линейный оператор.
    Тогда, если \begin{icloze}{1}в некотором базисе в \( V \) матрица оператора \( f \) имеет жорданов вид,\end{icloze} то этот базис называют \begin{icloze}{2}жордановым базисом оператора \( f \).\end{icloze}
\end{note}

\begin{note}{d8f181b2d5004a47bd308a35849cddec}
    Пусть \( f : V \to V \) --- линейный оператор, \( \lambda \in \operatorname{spec} f \).
    Как для \( k > 0 \) соотносятся \( N_{f,k} (\lambda) \) и \( N_{f, k+1} (\lambda) \)?

    \begin{cloze}{1}
        Для всех \( k \) меньше некоторого \( q \)
        \[
            N_{f, k} (\lambda) \subsetneqq N_{f, k+1} (\lambda),
        \]
        а для всех \( k \geqslant q \):
        \[
            N_{f, k} (\lambda) = N_{f, k+1} (\lambda)
        \]
    \end{cloze}
\end{note}

\begin{note}{414400f8f69b41b58c7d5b2930735317}
    Каков первый шаг в построении жордановой нормальной формы оператора \( f : V \to V \)?

    \begin{cloze}{1}
        Найти все собственные значения оператора \( f \).
    \end{cloze}
\end{note}

\begin{note}{a79be36515f64439b4db0f075099cbc3}
    Каков второй шаг в построении жордановой нормальной формы оператора \( f : V \to V \)?

    \begin{cloze}{1}
        Для каждого собственного значения \( \lambda \) найти все подпространства \( N_{f, k} (\lambda) \).
    \end{cloze}
\end{note}

\begin{note}{adf2c488db4640a1aba232fba8286d63}
    Каков третий шаг в построении жордановой нормальной формы оператора \( f : V \to V \)?

    \begin{cloze}{1}
        Построить жорданову лестницу в каждом из корневых подпространств \( f \).
    \end{cloze}
\end{note}

\begin{note}{2fe8afa7a09b49a1a7219ce868aaf67e}
    Каков заключительный шаг в построении жордановой нормальной формы оператора \( f : V \to V \)?

    \begin{cloze}{1}
        Объединить все построенные базисы в одну систему и построить матрицу \( f \) в полученном базисе.
    \end{cloze}
\end{note}

\section{Лекция 21.03.22}
\begin{note}{61582b48320a46c3ad047eec84da3eb3}
    Пусть \( A, A' \in \mathbb C^{\begin{icloze}{3}n \times n\end{icloze}} \). Тогда матрицы \( A \) и \( A' \) называются \begin{icloze}{2}подобными,\end{icloze} если \begin{icloze}{1}существует невырожденная матрица \( T \) такая, что
    \[
        A = T \: A' \: T^{-1}.
    \]\end{icloze}
\end{note}

\begin{note}{6366e6bbaa1149eb8bba346a3cc38654}
    Отношение подобия матриц обозначается символом
    \begin{icloze}{1}
        \[
            \sim
        \]
    \end{icloze}
\end{note}

\begin{note}{1ae63106d8d0480b82ef6f9e9b3d62bb}
    Подобие матриц является отношением \begin{icloze}{1}эквивалентности.\end{icloze}
\end{note}

\begin{note}{de743729325e43f79f35a7b8c22d5bb2}
    Любая \begin{icloze}{2}квадратная матрица\end{icloze} подобна \begin{icloze}{1}своей жордановой нормальной форме.\end{icloze}

    \begin{center}
        \tiny (следствие из \begin{icloze}{3}теоремы о жордановой форме\end{icloze})
    \end{center}
\end{note}

\begin{note}{82aa01fcbfb7476d84662ca5802dae5b}
    \begin{icloze}{2}Две квадратные матрицы подобны\end{icloze} \begin{icloze}{3}тогда и только тогда, когда\end{icloze} \begin{icloze}{1}их жордановы формы совпадают с точностью до перестановки клеток.\end{icloze}

    \begin{center}
        \tiny (следствие из \begin{icloze}{4}теоремы о жордановой форме\end{icloze})
    \end{center}
\end{note}

\begin{note}{198e1f3eef67411c89f83a35ade066d2}
    Пусть \( A, \Lambda, T \in \mathbb C^{n \times n} \),\: \( A = T^{-1} \Lambda T \),\: \( k \in \mathbb N \). Тогда
    \[
        A^{k} = \begin{icloze}{1}T^{-1} \Lambda^{k} T.\end{icloze}
    \]
\end{note}

\begin{note}{c1cf4048c475426683c811de00771765}
    Пусть \({ A \in \mathbb C^{n \times n} }\),\: \({ p \in \mathbb C[x] }\),\: \( \displaystyle p(x) = \sum_{k=0}^{n} a_k x^{k}. \)
    Тогда
    \[
        p(A) \overset{\text{def}}= \begin{icloze}{1}\sum_{k=0}^{n} a_k A^{k}, \quad \text{где \({ A^{0} \overset{\text{def}}= E }\)}.\end{icloze}
    \]
\end{note}

\begin{note}{59cb3566c41d4eca89ef63e626740c4e}
    Пусть \({ A, T \in \mathbb C^{n \times n} }\),\: \({ \det T \neq 0 }\),\: \({ p \in \mathbb C[x] }\). Тогда
    \[
        p(T A T^{-1}) = \begin{icloze}{1}T \: p(a) \: T^{-1}.\end{icloze}
    \]
\end{note}

\begin{note}{ad579382cf8a42caabf0b8b6a5a4d76f}
    Пусть \({ f : D \subset \mathbb C \to \mathbb C }\), \({ \lambda \in D }\).
    \[
        f(\lambda E) \overset{\text{def}}= \begin{icloze}{1}f(\lambda) E.\end{icloze}
    \]
\end{note}

\begin{note}{be2002dbe01149aa91e229d1c991143e}
    Пусть \({ f : D \subset \mathbb C \to \mathbb C }\),
    \[
        A = \begin{bmatrix}
            A_{11} & 0 \\
            0 & A_{22}
        \end{bmatrix}
        \in \mathbb C^{n \times n}.
    \]
    Тогда
    \[
        f(A) \overset{\text{def}}= \begin{icloze}{1}\begin{bmatrix}
            f(A_{11}) & 0 \\
            0 & f(A_{22})
        \end{bmatrix}.\end{icloze}
    \]
\end{note}

\begin{note}{455a3d16cf6744b39c1d1e21cab4e7f5}
    Пусть \({ f : D \subset \mathbb C \to \mathbb C }\), \({ \lambda \in D }\).
    Как определяют значение
    \[
        f(J_k(\lambda))?
    \]

    \begin{cloze}{1}
        Представляют \({ f(J_k(\lambda)) }\) как \({ f(\lambda E + \varepsilon) }\) и далее используют разложение \({ f }\) в ряд Тейлора в точке \({ \lambda E }\).
    \end{cloze}
\end{note}

\begin{note}{c435657fd33d4705ae2de65b4bf5c682}
    Пусть \({ f : D \subset \mathbb C \to \mathbb C }\), \({ \lambda \in D }\).
    Для каких \({ k }\) и \({ \lambda }\) определено значение \({ J_k (\lambda) }\)?

    \begin{cloze}{1}
        Должен существовать многочлен \({ T_{\lambda, k} f }\).
   \end{cloze}
\end{note}

\begin{note}{3450a4591ff748cb856f4578b3cda3c2}
    Пусть \({ p \in \mathbb C[x] }\),\: \({ A \in \mathbb C^{n \times n} }\). \begin{icloze}{1}Многочлен \( p \)\end{icloze} называется \begin{icloze}{2}аннулирующим многочленом для матрицы \( A \),\end{icloze} если
    \begin{icloze}{1}
        \[
            p(A) = 0.
        \]
    \end{icloze}
\end{note}

\begin{note}{34b1edb015384033870e10717e8bbdb2}
    \subsubsection{<<\begin{icloze}{2}Теорема Гамильтона-Кэли\end{icloze}>>}

    \begin{icloze}{1}Характеристический многочлен квадратной матрицы является для неё аннулирующим.\end{icloze}
\end{note}

\begin{note}{07bbead6e007486e93d2daa598a265b6}
    В чем ключевая идея доказательства теоремы Гамильтона-Кэли?

    \begin{cloze}{1}
        Для любого корневого вектора \({ x }\) имеем \({ \chi_{\text{\tiny $A$}}(A) \: x = 0 }\).
    \end{cloze}
\end{note}

\section{Лекция 28.03.22}
\begin{note}{c4787ae5340942d2a27db89ea5f9d4df}
    Пусть \({ V }\) --- линейное пространство над \({ \mathbb R }\).
    Билинейная форма \({ f }\) в \({ V }\) называется \begin{icloze}{2}положительно определённой,\end{icloze} если \begin{icloze}{1}для любого \({ v \in V }\)
    \[
        \begin{gathered}
            f(v, v) \geqslant 0; \quad f(v, v) = 0 \iff v = 0.
        \end{gathered}
    \]\end{icloze}
\end{note}

\begin{note}{18f442014f0e4614a642e429958b8931}
    Пусть \({ V }\) --- линейное пространство над \({ \mathbb R }\).
    \begin{icloze}{2}Скалярным произведением в \({ V }\)\end{icloze} называется \begin{icloze}{1}симметричная положительно определённая билинейная форма в \({ V }\).\end{icloze}
\end{note}

\begin{note}{cea78871e8124a29945d3540057c0c68}
    \begin{icloze}{1}Линейное пространство с заданным на нём скалярным произведением\end{icloze} называется \begin{icloze}{2}евклидовым пространством.\end{icloze}
\end{note}

\begin{note}{79a607edba4945a4a562d9b1fd8f2ce9}
    Пусть \({ V }\) --- евклидово пространство над \({ \mathbb R }\).
    Скалярное произведение векторов \({ v, w \in V }\) обозначается
    \begin{icloze}{1}
        \[
            (v, w).
        \]
    \end{icloze}
\end{note}

\begin{note}{717ab493f110448bb867a49b37d29d83}
    Пусть \({ V }\) --- евклидово пространство над \({ \mathbb R }\),\: \({ v \in V }\).
    \begin{icloze}{2}Длиной вектора \({ v }\)\end{icloze} называется \begin{icloze}{1}величина \( \sqrt{(v, v)} \).\end{icloze}
\end{note}

\begin{note}{7bc89a880fb244a78c3e204575ac9005}
    Пусть \({ V }\) --- евклидово пространство над \({ \mathbb R }\),\: \({ v \in V }\).
    \begin{icloze}{2}Длина вектора \({ v }\)\end{icloze} обозначается \begin{icloze}{1}\( \left\lvert v \right\rvert \) или \({ \left\lVert v \right\rVert }\).\end{icloze}
\end{note}

\begin{note}{de4db3a6688f4b198b8238b0e07dfce7}
    Длину вектора в еклидовом пространства так же ещё называют \begin{icloze}{1}нормой этого вектора.\end{icloze} В таком случае чаще используется обозначение \begin{icloze}{2}\( \left\lVert v \right\rVert \).\end{icloze}
\end{note}

\begin{note}{c0b109c4be9e4749ad794e9e38fffb2d}
    Пусть \({ V }\) --- евклидово пространство над \({ \mathbb R }\),\: \({ v_0 \in V }\),\: \({ \begin{icloze}{3}r \in \mathbb R\end{icloze} }\).
    \begin{icloze}{2}Сферой радиуса \({ r }\) с центром в точке \({ v_0 }\)\end{icloze} называют \begin{icloze}{1}множество
    \[
        \left\{ v \in V \mid \left\lVert v - v_0 \right\rVert = r \right\}.
    \]
\end{icloze}\end{note}

\begin{note}{09b61a41cf5f45109c79e7cc61f63740}
    Пусть \({ V }\) --- евклидово пространство над \({ \mathbb R }\),\: \({ v_0 \in V }\),\: \({ r \in \mathbb R }\).
    \begin{icloze}{2}Сфера радиуса \({ r }\) с центром в точке \({ v_0 }\)\end{icloze} обозначается
    \begin{icloze}{1}
        \[
            S_r (v_0).
        \]
    \end{icloze}
\end{note}

\begin{note}{e63df21bb26d42269a7a5d45c6b828b8}
    Пусть \({ V }\) --- евклидово пространство над \({ \mathbb R }\),\: \({ v_0 \in V }\),\: \({ \begin{icloze}{3}r \in \mathbb R\end{icloze} }\).
    \begin{icloze}{2}Шаром радиуса \({ r }\) с ценстром в точке  \({ v_0 }\)\end{icloze} называют \begin{icloze}{1}множество
        \[
            \left\{ v \in V | \left\lVert v - v_0 \right\rVert \leqslant r \right\}.
        \]
    \end{icloze}
\end{note}

\begin{note}{d0d10cbbdb664b428b1f3284ff5321f9}
    Пусть \({ V }\) --- евклидово пространство над \({ \mathbb R }\),\: \({ v_0 \in V }\),\: \({ r \in \mathbb R }\).
    \begin{icloze}{2}Шар радиуса \({ r }\) с центром в точке \({ v_0 }\)\end{icloze} обозначается
    \begin{icloze}{1}
        \[
            B_r (v_0).
        \]
    \end{icloze}
\end{note}

\begin{note}{a7021008185a411e99300286ac245d14}
    Пусть \({ V }\) --- евклидово пространство над \({ \mathbb R }\),\: \begin{icloze}{3}\({ v, w \in V \setminus \left\{ 0 \right\} }\).\end{icloze}
    Векторы \({ v }\) и \({ w }\) называются \begin{icloze}{2}сонаправленными,\end{icloze} если
    \begin{icloze}{1}
        \[
            \exists \lambda > 0 \quad v = \lambda w.
        \]
    \end{icloze}
\end{note}

\begin{note}{0cfd3b2d9f17418eb0b8fd2dd36ef1d4}
    Пусть \({ V }\) --- евклидово пространство над \({ \mathbb R }\),\: \begin{icloze}{3}\({ v, w \in V \setminus \left\{ 0 \right\} }\).\end{icloze}
    \begin{icloze}{2}Углом между векторами \({ v, w }\)\end{icloze} называется \begin{icloze}{1}угол \({ \varphi \in \left[ 0, \pi \right] }\) такой, что
    \[
        \cos \varphi = \frac{(v, w)}{\left\lVert v \right\rVert \cdot \left\lVert w \right\rVert}.
    \]\end{icloze}
\end{note}

\begin{note}{097fc51b1eab4a699e7110a38f0bd670}
    \subsubsection{<<\begin{icloze}{2}Неравенство Коши-Буниковского\end{icloze}>>}

    Пусть \({ V }\) --- евклидово пространство над \({ \mathbb R }\),\: \begin{icloze}{3}\({ v, w \in V }\).\end{icloze}
    Тогда всегда \begin{icloze}{1}\({ \left\lvert (v, w) \right\rvert \leqslant \left\lVert v \right\rVert \cdot \left\lVert w \right\rVert }\).\end{icloze}
\end{note}

\begin{note}{570b086e7e1b48e3b3012778f4841d1e}
    В чем основная идея доказательства неравенства Коши-Буниковского?

    \begin{cloze}{1}
        Рассмотреть скалярное произведение
        \[
            (v - \lambda w, v - \lambda w) \geqslant 0.
        \]
        И показать, что дискриминант соответствующего квадратного уравнения \({ \leqslant 0 }\).
    \end{cloze}
\end{note}

\begin{note}{96bb9d37dba3499d8890f7b3eb1f04d4}
    Пусть \({ V }\) --- евклидово пространство над \({ \mathbb R }\),\: \({ v, w \in V }\).
    Тогда
    \begin{center}
        \({ \left\lvert (v, w) \right\rvert = \left\lVert v \right\rVert \cdot \left\lVert w \right\rVert }\)
        \begin{icloze}{2}\({ \iff }\)\end{icloze}
        \begin{icloze}{1}\({ v }\) и \({ w }\) пропорциональны.\end{icloze}
    \end{center}
\end{note}

\begin{note}{d1941ac59ee44c82b045d6d1e954e0d8}
    \subsubsection{<<\begin{icloze}{2}Неравенство треугольника\end{icloze}>>}

    Пусть \({ V }\) --- евклидово пространство над \({ \mathbb R }\),\: \begin{icloze}{3}\({ v, w \in V }\).\end{icloze}
    Тогда
    \begin{icloze}{1}
        \[
            \left\lVert v + w \right\rVert \leqslant \left\lVert v \right\rVert + \left\lVert w \right\rVert.
        \]
    \end{icloze}
\end{note}

\begin{note}{4759501bf4b84cf0acf58f945229396c}
    В чем основная идея доказательства неравенства треугольника?

    \begin{cloze}{1}
    Рассмотреть скалярное произведение
    \[
        (v + w, v + w) = \left\lVert v + w \right\rVert^2.
    \]
    \end{cloze}
\end{note}

\begin{note}{5378eb0c9d81404c9cd8ca40925b9ce8}
    Пусть \({ V }\) --- евклидово пространство над \({ \mathbb R }\),\: \({ v, w \in V }\).
    Тогда
    \begin{center}
        \({ \left\lVert v + w \right\rVert = \left\lVert v \right\rVert + \left\lVert w \right\rVert }\)
        \begin{icloze}{2}\({ \iff }\) \end{icloze}
        \begin{icloze}{1}\({ v \operatorname{\uparrow\uparrow} w }\)\end{icloze}
    \end{center}
\end{note}

\begin{note}{8238aebbcc724e708990b61d8a0e3603}
    Пусть \({ V }\) --- евклидово пространство над \({ \mathbb R }\),\: \({ v, w \in V }\).
    Векторы \({ v }\) и \({ w }\) называются \begin{icloze}{2}ортогональными,\end{icloze} если \begin{icloze}{1}\({ (v, w) = 0 }\).\end{icloze}
\end{note}

\begin{note}{ce138d9eefe6445bbe72ecb3cafe43e8}
    Пусть \({ V }\) --- евклидово пространство над \({ \mathbb R }\).
    Система векторов в \({ V }\) называется \begin{icloze}{2}ортогональной,\end{icloze} если её \begin{icloze}{1}векторы попарно ортогональны.\end{icloze}
\end{note}

\begin{note}{2dbaa8c8157c42e08de67ebd6cc42e47}
    Пусть \({ V }\) --- евклидово пространство над \({ \mathbb R }\),\: \begin{icloze}{4}\({ \left\{ e_j \right\}_{j = 1}^{n} }\) --- ортогональная система векторов в \({ V }\).\end{icloze}
    Тогда \begin{icloze}{3}система \({ \left\{ e_j \right\} }\) линейно независима\end{icloze} \begin{icloze}{2}\({ \iff }\)\end{icloze} \begin{icloze}{1}\({ e_j \neq 0 }\) для всех \({ j }\).\end{icloze}
\end{note}

\begin{note}{d20a32cfc1c3440a9e22f5d28c36b9d5}
    Пусть \({ V }\) --- евклидово пространство над \({ \mathbb R }\),\: \({ \left\{ e_j \right\}_{j = 1}^{n} }\) --- ортогональная система ненулевых векторов в \({ V }\).
    Как показать, что система \({ \left\{ e_j \right\} }\) линейно независима?

    \begin{cloze}{1}
        Умножить линейную комбинацию векторов \({ \left\{ e_j \right\} }\), равную нулю, на \({ e_i }\) для произвольного \({ i }\) и показать равентсво нулю \({ i }\)-ого коэффициента.
    \end{cloze}
\end{note}

\begin{note}{b9cf4cdf374445c4bc8412c8ca72847c}
    Пусть \({ V }\) --- евклидово пространство над \({ \mathbb R }\),\: \begin{icloze}{3}\({ v \in V }\),\: \({ \left\{ e_j \right\}_{j = 1}^{n} }\) --- ортогональный базис в \({ V }\).\end{icloze}
    Тогда \begin{icloze}{2}координаты вектора \({ v }\) в базисе \({ \left\{ e_j \right\} }\)\end{icloze} имеют вид
    \[
        v_j = \begin{icloze}{1}\frac{(v, e_j)}{\left\lVert e_j \right\rVert^2}.\end{icloze}
    \]
\end{note}

\begin{note}{5a4e71f923b84eb5b5f3e2b66ea26470}
    Пусть \({ V }\) --- евклидово пространство над \({ \mathbb R }\),\: \({ v \in V }\),\: \({ \left\{ e_j \right\}_{j = 1}^{n} }\) --- ортогональный базис в \({ V }\).
    Как показать, что координаты вектора \({ v }\) в базисе \({ \left\{ e_j \right\} }\) имеют вид
    \[
        v_j = \frac{(v, e_j)}{\left\lVert e_j \right\rVert^2}?
    \]

    \begin{cloze}{1}
        Вычислить \({ (v, e_j) }\), разложив \({ v }\) по базису \({ \left\{ e_j \right\} }\).
    \end{cloze}
\end{note}

\begin{note}{7ede17a5d2d049c690090d4850f4ef60}
    Пусть \({ V }\) --- евклидово пространство над \({ \mathbb R }\),\: \begin{icloze}{3}\({ v \in V }\),\: \({ \left\{ e_j \right\}_{j = 1}^{n} }\) --- ортогональная линейно независима система в \({ V }\).\end{icloze}
    Тогда \begin{icloze}{1}числа
    \[
        \frac{(v, e_j)}{\left\lVert e_j \right\rVert^2}
    \]\end{icloze}
    называют \begin{icloze}{2}коэффициентами Фурье вектора \({ v }\) в системе \({ \left\{ e_j \right\} }\).\end{icloze}
\end{note}

\begin{note}{04027e25ae114e76a8cf6f9500e1ae28}
    Пусть \({ V }\) --- евклидово пространство над \({ \mathbb R }\).
    Система векторов \({ \left\{ e_j \right\}_{j = 1}^{n} }\) в \({ V }\) называется \begin{icloze}{2}ортонормированной,\end{icloze} если \begin{icloze}{1}её векторы попарно ортогональны и \({ \left\lVert e_j \right\rVert = 1 }\) для всех \({ j }\).\end{icloze}
\end{note}

\section{Лекция 04.04.22}
\begin{note}{25230410b91d47619feafd9dd1e3909e}
    \subsubsection{<<\begin{icloze}{3}Ортогонализация Грама-Шмидта\end{icloze}>>}

    Пусть \begin{icloze}{2}\({ V }\) --- евклидово пространство, \({ e_1, \ldots, e_n }\) --- базис в пространстве \({ V }\).\end{icloze}
    Тогда \begin{icloze}{1}всегда существует ортогональный базис \({ a_1, \ldots, a_n }\) в \({ V }\) такой, что
    \[
        a_j \in \mathscr L (e_1, \ldots, e_j) \quad \forall j.
    \]\end{icloze}
\end{note}

\begin{note}{89394003d65441209a81ec6be5c7f2df}
    В чем основная идея доказательства истинности теоремы об ортогонализации Грама-Шмидта?

    \begin{cloze}{1}
        Положить
        \begin{align*}
            a_1 &= e_1, \\
            a_2 &= e_2 + \alpha_1 a_1, \\
            a_3 &= e_3 + \beta_1 a_1 + \beta_2 a_2 \\
                &\ldots
        \end{align*}
    \end{cloze}
\end{note}

\begin{note}{067af76850ea49929f538a99ef2fb445}
    Пусть \begin{icloze}{3}\({ W }\) --- евклидово пространство, \({ V \triangleleft W }\).\end{icloze}
    \begin{icloze}{1}
        Множество
        \[
            \Big\{ w \in W \mid (v, w) = 0 \quad \forall v \in V \Big\}
        \]
    \end{icloze}
    называется \begin{icloze}{2}ортогональным дополнением к \({ V }\).\end{icloze}
\end{note}

\begin{note}{dc34194cc9a642aeb10ad2ba1cbab7ad}
    Пусть \({ W }\) --- евклидово пространство, \({ V \triangleleft W }\).
    \begin{icloze}{1}Ортогональное дополнение к пространству \({ V }\)\end{icloze} обозначается \begin{icloze}{2}\({ V^{\perp} }\).\end{icloze}
\end{note}

\begin{note}{800460fc49ee4f3b915a92addaba5141}
    Пусть \({ W }\) --- евклидово пространство, \({ V \triangleleft W }\).
    Всегда ли \({ V^{\perp} \triangleleft W }\)?

    \begin{cloze}{1}
        Да, всегда.
    \end{cloze}
\end{note}

\begin{note}{ab8d62b25a294edebe7a3735b84dab19}
    Пусть \({ W }\) --- евклидово пространство, \({ V \triangleleft W }\).
    Тогда
    \[
        \dim V^{\perp} = \begin{icloze}{1}\dim W - \dim V.\end{icloze}
    \]
\end{note}

\begin{note}{70166548d05745278d7a8f9de584d211}
    Пусть \({ W }\) --- евклидово пространство, \({ V \triangleleft W }\).
    Тогда
    \[
        V + V^{\perp} = \begin{icloze}{1}V \oplus V^{\perp} = W.\end{icloze}
    \]
\end{note}

\begin{note}{eee9a5f3a40047629e2192983ab08770}
    Пусть \({ W }\) --- евклидово пространство, \({ V \triangleleft W }\).
    Как показать, что \({ W = V \oplus V^{\perp} }\)?

    \begin{cloze}{1}
        Выбрать ортогональный базис в \({ V }\), дополнить его до ортогонального базиса в \({ W }\) и показать, что дополнение --- базис в \({ V^{\perp} }\).
    \end{cloze}
\end{note}

\begin{note}{53d600a53a4f48a7b4d1e3a3822918fe}
    Пусть \({ W }\) --- евклидово пространство, \({ V \triangleleft W }\),
    \({ e_1, \ldots, e_k }\) --- ортогональный базис в \({ V }\), \({ e_1, \ldots, e_n }\) --- ортогональный базис в \({ W }\).
    Как показать, что \({ e_{k + 1}, \ldots, e_n }\) --- базис в \({ V^{\perp} }\)?

    \begin{cloze}{1}
        Показать, что \({ \mathscr L (e_{k + 1}, \ldots, e_n) = V^{\perp} }\).
    \end{cloze}
\end{note}

\begin{note}{a9fc50cec2cc442d87f7f6a551043a18}
    Пусть \({ W }\) --- евклидово пространство, \begin{icloze}{3}\({ V \triangleleft W }\), \({ w \in W }\).\end{icloze}
    Тогда \begin{icloze}{1}проекция \({ w }\) на \({ V }\) параллельно \({ V^{\perp} }\)\end{icloze} называется \begin{icloze}{2}проекцией вектора \({ w }\) на \({ V }\).\end{icloze}
\end{note}

\begin{note}{bcb91b5b6d4048febe0fd4e8da7302e9}
    Пусть \({ W }\) --- евклидово пространство, \begin{icloze}{3}\({ V \triangleleft W }\), \({ w \in W }\).\end{icloze}
    Тогда \begin{icloze}{1}проекция \({ w }\) на \({ V^{\perp} }\) параллельно \({ V }\)\end{icloze} называется \begin{icloze}{2}перпендикуляром, опущенным из \({ w }\) на \({ V }\).\end{icloze}
\end{note}

\begin{note}{4e448e8833f94547ad7848fd34666613}
    Пусть \({ e_1, \ldots, e_k }\) --- система векторов в евклидовом пространстве.
    \begin{icloze}{2}Матрицей Грема системы \({ e_1, \ldots, e_k }\)\end{icloze} называют \begin{icloze}{1}матрицу
    \[
        \Big[ \left( e_i, e_j \right) \Big] \sim k \times k.
    \]\end{icloze}
\end{note}

\begin{note}{3bff6be501ed49109d5041f018ecab96}
    Пусть \({ e_1, \ldots, e_k }\) --- система векторов в евклидовом пространстве.
    \begin{icloze}{2}Матрица Грема системы \({ e_1, \ldots, e_k }\)\end{icloze} обозначается
    \begin{icloze}{1}
        \[
            G(e_1, \ldots, e_k).
        \]
    \end{icloze}
\end{note}

\begin{note}{f45df626ca1d4db1866e3f7aae0c6f2a}
    Пусть \({ W }\) --- евклидово пространство, \({ w \in W }\), \({ e_1, \ldots, e_k }\) --- базис в \({ V \triangleleft W }\).
    Как найти проекцию \({ w_0 }\) вектора \({ w }\) на \({ V }\)?

    \begin{cloze}{1}
        \[
            \begin{gathered}
                G(e_1, \ldots, e_n) \cdot \begin{bmatrix}
                    \alpha_1 \\ \vdots \\ \alpha_k
                \end{bmatrix}
                =
                \begin{bmatrix}
                    (w, e_1) \\ \vdots \\ (w, e_k)
                \end{bmatrix}, \\
                w_0 = e\alpha.
            \end{gathered}
        \]
    \end{cloze}
\end{note}

\end{document}

% vim: spelllang=ru_yo,en
