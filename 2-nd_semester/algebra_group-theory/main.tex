\documentclass[11pt, a5paper]{article}
\usepackage[width=10cm, top=0.5cm, bottom=2cm]{geometry}

\usepackage[T1,T2A]{fontenc}
\usepackage[utf8]{inputenc}
\usepackage[english,russian]{babel}
\usepackage{libertine}

\usepackage{amsmath}
\usepackage{amssymb}
\usepackage{amsthm}
\usepackage{mathrsfs}
\usepackage{framed}
\usepackage{xcolor}

\setlength{\parindent}{0pt}

% Force \pagebreak for every section
\let\oldsection\section
\renewcommand\section{\pagebreak\oldsection}

\renewcommand{\thesection}{}
\renewcommand{\thesubsection}{Note \arabic{subsection}}
\renewcommand{\thesubsubsection}{}
\renewcommand{\theparagraph}{}

\newenvironment{note}[1]{\goodbreak\par\subsection{\hfill \color{lightgray}\tiny #1}}{}
\newenvironment{cloze}[2][\ldots]{\begin{leftbar}}{\end{leftbar}}
\newenvironment{icloze}[2][\ldots]{%
  \ignorespaces\text{\tiny \color{lightgray}\{\{c#2::}\hspace{0pt}%
}{%
  \hspace{0pt}\text{\tiny\color{lightgray}\}\}}\unskip%
}


\begin{document}
\section{30.05.22}
\begin{note}{6fdd3ac4b4f644cea3704bcc79918836}
    \begin{icloze}{2}Группой \({ (G, \circ) }\)\end{icloze} называется \begin{icloze}{1}непустое множество \({ G }\) с заданной на нём бинарной операцией \({ \circ : G \times G \to G }\), удовлетворяющей аксиомам группы.\end{icloze}
\end{note}

\begin{note}{827b57c3950c42b28e381d37a49ddf39}
    Сколько утверждений представлено в наборе аксиом из определения группы \({ (G, \circ) }\)?

    \begin{cloze}{1}
        Три.
    \end{cloze}
\end{note}

\begin{note}{f526d0257921478ca77a37b97abb9d06}
    Какова первая аксиома в наборе аксиом из определения группы \({ (G, \circ) }\)?

    \begin{cloze}{1}
        Операция \({ \circ }\) ассоциативна.
    \end{cloze}
\end{note}

\begin{note}{ce2298302937453e87e0cf850f17af90}
    Какова вторая аксиома в наборе аксиом из определения группы \({ (G, \circ) }\)?

    \begin{cloze}{1}
        Для операции \({ \circ }\) существует нейтральный элемент.
    \end{cloze}
\end{note}

\begin{note}{9f917456f2bf4fe6bf4e35f8042c9499}
    \begin{icloze}{2}Нейтральный элемент\end{icloze} из определения группы \({ (G, \circ) }\) обычно обозначают \begin{icloze}{1}\({ e }\).\end{icloze}
\end{note}

\begin{note}{3a8f693c011348fd9e88038d036a5b42}
    Пусть \({ (G, \circ) }\) --- группа,\: \begin{icloze}{4}\({ a \in G }\).\end{icloze} \begin{icloze}{2}Элемент \({ \tilde a \in G }\)\end{icloze} называется \begin{icloze}{3}обратным к \({ a }\),\end{icloze} если
    \begin{icloze}{1}
        \[
            a \circ \tilde a = \tilde a \circ a = e.
        \]
    \end{icloze}
\end{note}

\begin{note}{13c9853893a445d9a33db6823c3a5146}
    Какова третья аксиома в наборе аксиом из определения группы \({ (G, \circ) }\)?

    \begin{cloze}{1}
        \({ \forall a \in G }\) существует обратный к \({ a }\) элемент.
    \end{cloze}
\end{note}

\begin{note}{5ba5e27ac8a9481eac4302c3159a6596}
    Пусть \({ (G, \circ) }\) --- группа, \({ a \in G }\). \begin{icloze}{2}Обратный элемент к \({ a }\)\end{icloze} обычно обозначают \begin{icloze}{1}\({ a^{-1} }\).\end{icloze}
\end{note}

\begin{note}{9f4da30e71b1403a998b7c3fdf192252}
    \begin{icloze}{2}Множество всех невырожденных \({ n \times n }\) матриц над полем \({ F }\)\end{icloze} вместе с \begin{icloze}{3}операцией умножения\end{icloze} \begin{icloze}{1}называется общей линейной группой.\end{icloze}
\end{note}

\begin{note}{27a09e6a00d14e859d7ad1d78a4f74a3}
    \begin{icloze}{2}Общая линейная группа из \({ n \times n }\) матриц над полем \({ F }\)\end{icloze} обозначается \begin{icloze}{1}\({ \operatorname{GL}(n, F) }\).\end{icloze}
\end{note}

\begin{note}{809c8a8f790e4a2a998a4a8038c03971}
    Группа \({ (G, \circ) }\) называется \begin{icloze}{2}абелевой,\end{icloze} если \begin{icloze}{1}операция \({ \circ }\) коммутативна.\end{icloze}
\end{note}

\begin{note}{e59ac970ec54461083354dae9eeb4047}
    Может ли группа иметь несколько нейтральных элементов?

    \begin{cloze}{1}
        Нет, нейтральный элемент единственен.
    \end{cloze}
\end{note}

\begin{note}{13fee55238844118889a790b6e0c7e37}
    Пусть \({ (G, \circ) }\) --- группа.
    Тогда если \({ e }\) и \({ e' }\) --- нейтральные элементы для \({ \circ }\), то \({ e = e' }\). В чём основная идея доказательства?

    \begin{cloze}{1}
        Рассмотреть \({ e \circ e' }\).
    \end{cloze}
\end{note}

\begin{note}{afa616033db44cee8d39131bb90173bd}
    Пусть \({ (G, \circ) }\) --- группа, \({ a \in G }\).
    Может ли в \({ G }\) существовать несколько элементов, обратных к \({ a }\)?

    \begin{cloze}{1}
        Нет, обратный элемент единственен.
    \end{cloze}
\end{note}

\begin{note}{9f4dcde939af46639169bda602d721c5}
    Пусть \({ (G, \circ)}\) --- группа, \({ a \in G }\).
    Тогда если \({ a^{-1} }\) и \({ \tilde a }\) --- обратные элементы к \({ a }\), то \({ \tilde a = a^{-1} }\). В чём основная идея доказательства?

    \begin{cloze}{1}
        Представить \({ \tilde a }\) как \({ \tilde a \circ \left( a \circ a^{-1} \right) }\).
    \end{cloze}
\end{note}

\begin{note}{3db3d03590c84407bfb64b2a80b0e1c5}
    Пусть \({ (G, \circ) }\) --- группа, \begin{icloze}{2}\({ a, b \in G }\).\end{icloze} Тогда
    \[
        (a \circ b)^{-1} =  \begin{icloze}{1}b^{-1} \circ a^{-1}.\end{icloze}
    \]
\end{note}

\begin{note}{10144a83e52a4f5cbf0f96c818e229a5}
    Пусть \({ (G, \circ) }\) --- группа, \begin{icloze}{3}\({ H \subset G }\).\end{icloze}
    Тогда \begin{icloze}{4}\({ (H, \circ) }\)\end{icloze} называется \begin{icloze}{2}подгруппой группы \({ (G, \circ) }\),\end{icloze} если \begin{icloze}{1}\({ (H, \circ) }\) является группой.\end{icloze}
\end{note}

\begin{note}{9de4580c8d2545bcad2c525fe42930ec}
    Пусть \({ (G, \circ) }\) --- группа, \({ H \subset G }\).
    Выражение ``\({ (H, \circ) }\) является подгруппой \({ (G, \circ) }\)'' обозначается
    \[
        (H, \circ) \leqslant (G, \circ).
    \]
\end{note}

\begin{note}{bd4835b2c522436fac41030bf6b13a66}
    Пусть \({ (G, \circ)}\) --- группа, \begin{icloze}{4}\({ a \in G }\),\end{icloze}\: \begin{icloze}{3}\({ n \in \mathbb N }\).\end{icloze}
    \[
        \begin{icloze}{2}a^{n}\end{icloze} \overset{\text{def}}= \begin{icloze}{1}\underbrace{a \circ \cdots \circ a}_{\text{\({ n }\) раз}}.\end{icloze}
    \]
\end{note}

\begin{note}{2e41bce96a5249ca9d372d04f772b9b4}
    Пусть \({ (G, \circ)}\) --- группа, \begin{icloze}{2}\({ a \in G }\).\end{icloze}
    \[
        a^{0} \overset{\text{def}}= \begin{icloze}{1}e.\end{icloze}
    \]
\end{note}

\begin{note}{2cfa92bf39b847d4aa21d381a0d2c428}
    Пусть \({ (G, \circ)}\) --- группа, \({ a \in G }\),\: \({ n \in \mathbb N }\).
    \[
        \begin{icloze}{2}a^{-n}\end{icloze} \overset{\text{def}}= \begin{icloze}{1}\left( a^{-1} \right)^{n}.\end{icloze}
    \]
\end{note}

\begin{note}{3994ad9b38154ec081e7042011939b50}
    Пусть \({ (G, \circ)}\) --- группа, \begin{icloze}{3}\({ a \in G }\).\end{icloze}
    \begin{icloze}{2}Порядком элемента \({ a }\)\end{icloze} называется \begin{icloze}{1}либо \({ \min \left\{ n \in \mathbb N \mid a^{n} = e \right\} }\), либо \({ \infty }\), если таких \({ n }\) не существует.\end{icloze}
\end{note}

\begin{note}{78e264e39e824819ace538828da51d7c}
    Пусть \({ (G, \circ)}\) --- группа, \({ a \in G }\).
    \begin{icloze}{2}Порядок элемента \({ a }\)\end{icloze} обозначается \begin{icloze}{1}\({ \operatorname{ord} a }\).\end{icloze}
\end{note}

\begin{note}{2e3b057efc1e40b1843700b41b2052b9}
    Пусть \({ (G, \circ)}\) --- группа, \begin{icloze}{3}\({ a \in G }\).\end{icloze}
    \begin{icloze}{1}Множество \({ \{ a^{k} \mid k \in \mathbb Z \} }\) с операций \({ \circ }\)\end{icloze} называется \begin{icloze}{2}подгруппой \({ (G, \circ) }\), порождённой элементом \({ a }\).\end{icloze}
\end{note}

\begin{note}{fd96a89fdb1b45559782a7213101e400}
    Пусть \({ (G, \circ)}\) --- группа, \({ a \in G }\).
    \begin{icloze}{2}Подгруппа \({ (G, \circ) }\), порождённая элементом \({ a }\),\end{icloze} обозначается \begin{icloze}{1}\({ \langle a \rangle }\).\end{icloze}
\end{note}

\begin{note}{54a6a6775d1940b09be51518008fabdc}
    Пусть \({ (G, \circ)}\) --- группа, \({ a \in G }\).
    Тогда если \begin{icloze}{2}\({ \operatorname{ord} a < \infty }\),\end{icloze} то
    \[
        \begin{icloze}{3}\langle a \rangle\end{icloze} \simeq \begin{icloze}{1}\mathbb Z_{\operatorname{ord} a}.\end{icloze}
    \]
\end{note}

\begin{note}{d83fe9abbfca4fc99b99e08866cc83a9}
    Пусть \({ (G, \circ)}\) --- группа, \({ a \in G }\).
    Тогда если \begin{icloze}{2}\({ \operatorname{ord} a = \infty }\),\end{icloze} то
    \[
        \begin{icloze}{3}\langle a \rangle\end{icloze} \simeq \begin{icloze}{1}\mathbb Z.\end{icloze}
    \]
\end{note}

\end{document}

% vim: spelllang=ru_yo,en
