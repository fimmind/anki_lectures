\documentclass[11pt, a5paper]{article}
\usepackage[width=10cm, top=0.5cm, bottom=2cm]{geometry}

\usepackage[T1,T2A]{fontenc}
\usepackage[utf8]{inputenc}
\usepackage[english,russian]{babel}
\usepackage{libertine}

\usepackage{amsmath}
\usepackage{amssymb}
\usepackage{amsthm}
\usepackage{mathrsfs}
\usepackage{framed}
\usepackage{xcolor}

\setlength{\parindent}{0pt}

% Force \pagebreak for every section
\let\oldsection\section
\renewcommand\section{\pagebreak\oldsection}

\renewcommand{\thesection}{}
\renewcommand{\thesubsection}{Note \arabic{subsection}}
\renewcommand{\thesubsubsection}{}
\renewcommand{\theparagraph}{}

\newenvironment{note}[1]{\goodbreak\par\subsection{\hfill \color{lightgray}\tiny #1}}{}
\newenvironment{cloze}[2][\ldots]{\begin{leftbar}}{\end{leftbar}}
\newenvironment{icloze}[2][\ldots]{%
  \ignorespaces\text{\tiny \color{lightgray}\{\{c#2::}\hspace{0pt}%
}{%
  \hspace{0pt}\text{\tiny\color{lightgray}\}\}}\unskip%
}


\begin{document}
\section{30.05.22}
\begin{note}{9a902d381d8f4e4fb5ff8c1e77b38c57}
    Пусть \({ G }\) --- непустое множество.
    \begin{icloze}{1}
        Отображение вида
        \[
            G \times G \to G
        \]
    \end{icloze}
    называется \begin{icloze}{2}бинарной операцией на множестве \({ G }\).\end{icloze}
\end{note}

\begin{note}{6fdd3ac4b4f644cea3704bcc79918836}
    Пусть \begin{icloze}{1}\({ G }\) --- непустое множество,\end{icloze} \begin{icloze}{2}\({ \circ }\) --- бинарная операция на \({ G }\).\end{icloze}
    \begin{icloze}{4}Пара \({ (G, \circ) }\)\end{icloze} называется \begin{icloze}{3}группой,\end{icloze} если \begin{icloze}{4}она удовлетворяет аксиомам группы.\end{icloze}
\end{note}

\begin{note}{827b57c3950c42b28e381d37a49ddf39}
    Сколько утверждений представлено в наборе аксиом из определения группы \({ (G, \circ) }\)?

    \begin{cloze}{1}
        Три.
    \end{cloze}
\end{note}

\begin{note}{f526d0257921478ca77a37b97abb9d06}
    Какова первая аксиома в наборе аксиом из определения группы \({ (G, \circ) }\)?

    \begin{cloze}{1}
        Операция \({ \circ }\) ассоциативна.
    \end{cloze}
\end{note}

\begin{note}{ce2298302937453e87e0cf850f17af90}
    Какова вторая аксиома в наборе аксиом из определения группы \({ (G, \circ) }\)?

    \begin{cloze}{1}
        Для операции \({ \circ }\) существует нейтральный элемент.
    \end{cloze}
\end{note}

\begin{note}{9f917456f2bf4fe6bf4e35f8042c9499}
    \begin{icloze}{2}Нейтральный элемент\end{icloze} из определения группы \({ (G, \circ) }\) обычно обозначают \begin{icloze}{1}\({ e }\).\end{icloze}
\end{note}

\begin{note}{3a8f693c011348fd9e88038d036a5b42}
    Пусть \({ (G, \circ) }\) --- группа,\: \begin{icloze}{4}\({ a \in G }\).\end{icloze} \begin{icloze}{2}Элемент \({ \tilde a \in G }\)\end{icloze} называется \begin{icloze}{3}обратным к \({ a }\),\end{icloze} если
    \begin{icloze}{1}
        \[
            a \circ \tilde a = \tilde a \circ a = e.
        \]
    \end{icloze}
\end{note}

\begin{note}{13c9853893a445d9a33db6823c3a5146}
    Какова третья аксиома в наборе аксиом из определения группы \({ (G, \circ) }\)?

    \begin{cloze}{1}
        \({ \forall a \in G }\) существует обратный к \({ a }\) элемент.
    \end{cloze}
\end{note}

\begin{note}{5ba5e27ac8a9481eac4302c3159a6596}
    Пусть \({ (G, \circ) }\) --- группа, \({ a \in G }\). \begin{icloze}{2}Обратный элемент к \({ a }\)\end{icloze} обычно обозначают \begin{icloze}{1}\({ a^{-1} }\).\end{icloze}
\end{note}

\begin{note}{9f4da30e71b1403a998b7c3fdf192252}
    \begin{icloze}{2}Множество всех невырожденных \({ n \times n }\) матриц над полем \({ F }\)\end{icloze} вместе с \begin{icloze}{3}операцией умножения\end{icloze} называется \begin{icloze}{1}общей линейной группой.\end{icloze}
\end{note}

\begin{note}{27a09e6a00d14e859d7ad1d78a4f74a3}
    \begin{icloze}{2}Общая линейная группа из \({ n \times n }\) матриц над полем \({ F }\)\end{icloze} обозначается \begin{icloze}{1}\({ \operatorname{GL}(n, F) }\).\end{icloze}
\end{note}

\begin{note}{809c8a8f790e4a2a998a4a8038c03971}
    Группа \({ (G, \circ) }\) называется \begin{icloze}{2}абелевой,\end{icloze} если \begin{icloze}{1}операция \({ \circ }\) коммутативна.\end{icloze}
\end{note}

\begin{note}{e59ac970ec54461083354dae9eeb4047}
    Может ли группа иметь несколько нейтральных элементов?

    \begin{cloze}{1}
        Нет, нейтральный элемент единственен.
    \end{cloze}
\end{note}

\begin{note}{13fee55238844118889a790b6e0c7e37}
    Пусть \({ (G, \circ) }\) --- группа.
    Тогда если \({ e }\) и \({ e' }\) --- нейтральные элементы для \({ \circ }\), то \({ e = e' }\). В чём основная идея доказательства?

    \begin{cloze}{1}
        Рассмотреть \({ e \circ e' }\).
    \end{cloze}
\end{note}

\begin{note}{afa616033db44cee8d39131bb90173bd}
    Пусть \({ (G, \circ) }\) --- группа, \({ a \in G }\).
    Может ли в \({ G }\) существовать несколько элементов, обратных к \({ a }\)?

    \begin{cloze}{1}
        Нет, обратный элемент единственен.
    \end{cloze}
\end{note}

\begin{note}{9f4dcde939af46639169bda602d721c5}
    Пусть \({ (G, \circ)}\) --- группа, \({ a \in G }\).
    Тогда если \({ a^{-1} }\) и \({ \tilde a }\) --- обратные элементы к \({ a }\), то \({ \tilde a = a^{-1} }\). В чём основная идея доказательства?

    \begin{cloze}{1}
        Представить \({ \tilde a }\) как \({ \tilde a \circ \left( a \circ a^{-1} \right) }\).
    \end{cloze}
\end{note}

\begin{note}{3db3d03590c84407bfb64b2a80b0e1c5}
    Пусть \({ (G, \circ) }\) --- группа, \begin{icloze}{2}\({ a, b \in G }\).\end{icloze} Тогда
    \[
        (a \circ b)^{-1} =  \begin{icloze}{1}b^{-1} \circ a^{-1}.\end{icloze}
    \]
\end{note}

\begin{note}{10144a83e52a4f5cbf0f96c818e229a5}
    Пусть \({ (G, \circ) }\) --- группа, \begin{icloze}{3}\({ H \subset G }\).\end{icloze}
    Тогда \begin{icloze}{4}\({ (H, \circ) }\)\end{icloze} называется \begin{icloze}{2}подгруппой группы \({ (G, \circ) }\),\end{icloze} если \begin{icloze}{1}\({ (H, \circ) }\) является группой.\end{icloze}
\end{note}

\begin{note}{9de4580c8d2545bcad2c525fe42930ec}
    Пусть \({ (G, \circ) }\) --- группа, \({ H \subset G }\).
    Выражение ``\({ (H, \circ) }\) является подгруппой \({ (G, \circ) }\)'' обозначается
    \[
        (H, \circ) \leqslant (G, \circ).
    \]
\end{note}

\begin{note}{bd4835b2c522436fac41030bf6b13a66}
    Пусть \({ (G, \circ)}\) --- группа, \begin{icloze}{4}\({ a \in G }\),\end{icloze}\: \begin{icloze}{3}\({ n \in \mathbb N }\).\end{icloze}
    \[
        \begin{icloze}{2}a^{n}\end{icloze} \overset{\text{def}}= \begin{icloze}{1}\underbrace{a \circ \cdots \circ a}_{\text{\({ n }\) раз}}.\end{icloze}
    \]
\end{note}

\begin{note}{2e41bce96a5249ca9d372d04f772b9b4}
    Пусть \({ (G, \circ)}\) --- группа, \begin{icloze}{2}\({ a \in G }\).\end{icloze}
    \[
        a^{0} \overset{\text{def}}= \begin{icloze}{1}e.\end{icloze}
    \]
\end{note}

\begin{note}{2cfa92bf39b847d4aa21d381a0d2c428}
    Пусть \({ (G, \circ)}\) --- группа, \({ a \in G }\),\: \({ n \in \mathbb N }\).
    \[
        \begin{icloze}{2}a^{-n}\end{icloze} \overset{\text{def}}= \begin{icloze}{1}\left( a^{-1} \right)^{n}.\end{icloze}
    \]
\end{note}

\begin{note}{3994ad9b38154ec081e7042011939b50}
    Пусть \({ (G, \circ)}\) --- группа, \begin{icloze}{3}\({ a \in G }\).\end{icloze}
    \begin{icloze}{2}Порядком элемента \({ a }\)\end{icloze} называется \begin{icloze}{1}либо
    \[
        \min \left\{ n \in \mathbb N \mid a^{n} = e \right\}.
    \]
    либо \({ \infty }\), если таких \({ n }\) не существует.\end{icloze}
\end{note}

\begin{note}{78e264e39e824819ace538828da51d7c}
    Пусть \({ (G, \circ)}\) --- группа, \({ a \in G }\).
    \begin{icloze}{2}Порядок элемента \({ a }\)\end{icloze} обозначается \begin{icloze}{1}\({ \operatorname{ord} a }\).\end{icloze}
\end{note}

\begin{note}{2e3b057efc1e40b1843700b41b2052b9}
    Пусть \({ (G, \circ)}\) --- группа, \begin{icloze}{3}\({ a \in G }\).\end{icloze}
    \begin{icloze}{1}Множество \({ \{ a^{k} \mid k \in \mathbb Z \} }\) с операций \({ \circ }\)\end{icloze} называется \begin{icloze}{2}подгруппой \({ (G, \circ) }\), порождённой элементом \({ a }\).\end{icloze}
\end{note}

\begin{note}{fd96a89fdb1b45559782a7213101e400}
    Пусть \({ (G, \circ)}\) --- группа, \({ a \in G }\).
    \begin{icloze}{2}Подгруппа \({ (G, \circ) }\), порождённая элементом \({ a }\),\end{icloze} обозначается \begin{icloze}{1}\({ \langle a \rangle }\).\end{icloze}
\end{note}

\begin{note}{54a6a6775d1940b09be51518008fabdc}
    Пусть \({ (G, \circ)}\) --- группа, \({ a \in G }\).
    Тогда если \begin{icloze}{2}\({ \operatorname{ord} a < \infty }\),\end{icloze} то
    \[
        \begin{icloze}{3}(\langle a \rangle, \circ)\end{icloze} \simeq \begin{icloze}{1}(\mathbb Z_{\operatorname{ord} a}, +).\end{icloze}
    \]
\end{note}

\begin{note}{d83fe9abbfca4fc99b99e08866cc83a9}
    Пусть \({ (G, \circ)}\) --- группа, \({ a \in G }\).
    Тогда если \begin{icloze}{2}\({ \operatorname{ord} a = \infty }\),\end{icloze} то
    \[
        \begin{icloze}{3}(\langle a \rangle, \circ)\end{icloze} \simeq \begin{icloze}{1}(\mathbb Z, +).\end{icloze}
    \]
\end{note}

\section{06.06.22}
\begin{note}{053e51258ecd4ca588d279e34a89a3d3}
    Пусть \({ (G, \circ), (H, *) }\) --- группы,\: \begin{icloze}{3}\({ f : G \to H }\).\end{icloze}
    Отображение \({ f }\) называется \begin{icloze}{2}гомоморфизмом групп,\end{icloze} если
    \begin{icloze}{1}
        \[
            \forall a, b \in G \quad f(a \circ b) = f(a) * f(b).
        \]
    \end{icloze}
\end{note}

\begin{note}{5266d124dc1d4300b1204c6286b3e25e}
    Пусть \({ (G, \circ), (H, *) }\) --- группы,\: \({ f : G \to H }\) --- гомоморфизм.
    Тогда
    \[
        f(e) = \begin{icloze}{1}e.\end{icloze}
    \]
\end{note}

\begin{note}{6fa9d3c343dc4c9dbd8cee9c37bbac42}
    Пусть \({ (G, \circ), (H, *) }\) --- группы,\: \({ f : G \to H }\) --- гомоморфизм.
    Тогда
    \[
        f(a^{-1}) = \begin{icloze}{1}f(a)^{-1}\end{icloze} \quad \forall a \in G.
    \]
\end{note}

\begin{note}{181a648ef262451fb18b4237c6c7f429}
    Пусть \({ (G, \circ), (H, *) }\) --- группы,\: \({ f : G \to H }\).
    Отображение \({ f }\) называется \begin{icloze}{2}изоморфизмом групп,\end{icloze} если \begin{icloze}{1}оно является гомоморфизмом и биективно.\end{icloze}
\end{note}

\begin{note}{743a7ef3a0c045548f43006f58969493}
    \[
        \begin{icloze}{2}\mathbb R_+\end{icloze} \overset{\text{def}}= \begin{icloze}{1}\left\{ x \in \mathbb R \mid x > 0 \right\}.\end{icloze}
    \]

    \begin{center}
        \tiny
        (не как в матане!)
    \end{center}
\end{note}

\begin{note}{7618af52019f4c6bb8a64f426a797e41}
    \[
        \begin{icloze}{2}\overline{\mathbb R}_+\end{icloze} \overset{\text{def}}= \begin{icloze}{1}\left\{ x \in \mathbb R \mid x \geqslant 0 \right\}.\end{icloze}
    \]

    \begin{center}
        \tiny
        (не как в матане!)
    \end{center}
\end{note}

\begin{note}{7dd1206881114b28bc9ef9c14a7fd882}
    Пример изоморфизма групп \({ (\mathbb R_+, \cdot) }\) и \({ (\mathbb R, +) }\).

    \begin{cloze}{1}
        \[
            f : x \mapsto \ln x.
        \]
    \end{cloze}
\end{note}

\begin{note}{2ec8dcb4e81d40eebde4db2b2702daa4}
    Пусть \({ n \in \mathbb N }\).
    \[
        \mathbb Z_{n} \overset{\text{def}}= \begin{icloze}{1}[0 : n - 1].\end{icloze}
    \]
\end{note}

\begin{note}{ae71026122c54154a213e03843c8abcb}
    Пусть \({ a, b \in Z_n }\).
    \[
        a + b \overset{\text{def}}= \begin{icloze}{1}(a + b) \bmod n.\end{icloze}
    \]
\end{note}

\begin{note}{8e7c4384053947bc8f40faae3d3bc34f}
    Пусть \({ (G, \circ)}\) --- группа, \({ a \in G }\), \({ \operatorname{ord} a < \infty }\).
    Тогда
    \[
        (\langle a \rangle, \circ) \simeq (\mathbb Z_{\operatorname{ord} a}, +).
    \]
    В чём основная идея доказательства?

    \begin{cloze}{1}
        Построить изоморфизм \({ \mathbb Z_{\operatorname{ord} a} \to \langle a \rangle, \quad k \mapsto a^{k} }\).
    \end{cloze}
\end{note}

\begin{note}{129a1bab504e409cb12b31bb2da9c1ff}
    Пусть \({ (G, \circ)}\) --- группа, \({ a \in G }\), \({ \operatorname{ord} a < \infty }\).
    Как показать, что \({ f : k \mapsto a^{k},\: \mathbb Z_{\operatorname{ord} a} \to G }\) --- гомоморфизм?

    \begin{cloze}{1}
        Представить \({ f(k_1 + k_2) }\) как \({ g^{k_1 + k_2 - l \cdot n},\: l \in \left\{ 0, 1 \right\} }\).
    \end{cloze}
\end{note}

\begin{note}{69b8b587049647ca85d7cdc871bebb05}
    Пусть \({ (G, \circ)}\) --- группа, \({ a \in G }\), \({ \operatorname{ord} a < \infty }\).
    Как показать, что \({ f : k \mapsto a^{k},\: \mathbb Z_{\operatorname{ord} a} \to G }\) --- сюръекция?

    \begin{cloze}{1}
        Представить \({ a^{p} \in \langle a \rangle }\) как \({ a^{l \cdot n + k_0} }\).
    \end{cloze}
\end{note}

\begin{note}{86c7386b47444f4cab166aecea358d5b}
    Пусть \({ (G, \circ)}\) --- группа, \({ a \in G }\), \({ \operatorname{ord} a < \infty }\).
    Как показать, что \({ f : k \mapsto a^{k},\: \mathbb Z_{\operatorname{ord} a} \to G }\) --- инъекция?

    \begin{cloze}{1}
        \[
            k \neq l \implies a^{k - l} \neq e.
        \]
    \end{cloze}
\end{note}

\begin{note}{326a83d344554cb38aab476534b6f5e8}
    Пусть \({ (G, \circ)}\) --- группа, \({ a \in G }\), \({ \operatorname{ord} a = \infty }\).
    Тогда
    \[
        (\langle a \rangle, \circ) \simeq (\mathbb Z, +).
    \]
    В чём основная идея доказательства?

    \begin{cloze}{1}
        Построить изоморфизм \({ \mathbb Z \to \langle a \rangle, \quad k \mapsto a^{k} }\).
    \end{cloze}
\end{note}

\begin{note}{31fd624715c244b2ba453e6ffe19dd74}
    Пусть \({ (G, \circ) }\) --- группа, \begin{icloze}{3}\({ (H, \circ) }\) --- подгруппа,\: \({ g \in G }\).\end{icloze}
    \[
        \begin{icloze}{2}g \circ H\end{icloze} \overset{\text{def}}= \begin{icloze}{1}\left\{ g \circ h \mid h \in H \right\}.\end{icloze}
    \]
\end{note}

\begin{note}{ac542c349e5b43e886540f1f0e62bacc}
    Пусть \({ (G, \circ) }\) --- группа, \begin{icloze}{3}\({ (H, \circ) }\) --- подгруппа,\: \({ g \in G }\).\end{icloze}
    \[
        \begin{icloze}{2}H \circ g\end{icloze} \overset{\text{def}}= \begin{icloze}{1}\left\{ h \circ g \mid h \in H \right\}.\end{icloze}
    \]
\end{note}

\begin{note}{20affff668b04e9e80ea15dc66eab2c2}
    Пусть \({ (G, \circ) }\) --- группа, \({ (H, \circ) }\) --- подгруппа,\: \({ g \in G }\).
    \begin{icloze}{2}Множество \({ g \circ H }\)\end{icloze} называется \begin{icloze}{1}левым классом смежности элемента \({ g }\) по подгруппе \({ H }\).\end{icloze}
\end{note}

\begin{note}{ca40d5b36a764e62924dfd73ea9ebc66}
    Пусть \({ (G, \circ) }\) --- группа, \({ (H, \circ) }\) --- подгруппа,\: \({ g \in G }\).
    \begin{icloze}{2}Множество \({ H \circ g }\)\end{icloze} называется \begin{icloze}{1}правым классом смежности элемента \({ g }\) по подгруппе \({ H }\).\end{icloze}
\end{note}

\begin{note}{810cc5be7cb2498280729b27d347be4f}
    Пусть \({ (G, \circ) }\) --- группа, \begin{icloze}{3}\({ (H, \circ) }\) --- подгруппа,\end{icloze}\: \begin{icloze}{4}\({ a, b \in G }\).\end{icloze}
    \[
        \begin{icloze}{2}a \equiv b \pmod{H}\end{icloze} \overset{\hspace{-1pt}\text{\tiny def}}\iff \begin{icloze}{1}a \circ b^{-1} \in H.\end{icloze}
    \]
\end{note}

\begin{note}{ff25dee3ae6f4b1ab34700578cceaed5}
    Пусть \({ (G, \circ) }\) --- группа, \({ (H, \circ) }\) --- подгруппа,\: \({ a, b \in G }\).
    Тогда
    \[
        \begin{icloze}{2}a \equiv b \pmod H\end{icloze} \iff \begin{icloze}{1}a \circ H = b \circ H.\end{icloze}
    \]

    \begin{center}
        \tiny
        (в терминах классов смежности)
    \end{center}
\end{note}

\begin{note}{489a77d7bd2a4523886a65a220d953f4}
    Пусть \({ (G, \circ) }\) --- группа, \({ (H, \circ) }\) --- подгруппа.
    Отношение
    \[
        \cdot \equiv \cdot \pmod{H}
    \]
    является отношением \begin{icloze}{1}эквивалентности.\end{icloze}
\end{note}

\begin{note}{a07284200e0b4649bb1357b2aeaf3cc0}
    Пусть \({ (G, \circ) }\) --- группа, \({ (H, \circ) }\) --- подгруппа.
    Как показать, что отношение \({ \cdot \equiv \cdot \pmod{H} }\) является симметричным?

    \begin{cloze}{1}
        \[
            a \circ b^{-1} \in H \implies (a \circ b^{-1})^{-1} \in H.
        \]
    \end{cloze}
\end{note}

\begin{note}{745cc90590ef4d0784af24f93c539a9f}
    Пусть \({ (G, \circ) }\) --- группа, \({ (H, \circ) }\) --- подгруппа,\: \begin{icloze}{2}\({ g_1, g_2 \in G }\).\end{icloze}
    Тогда всегда \({ g_1 \circ H }\) и \({ g_2 \circ H }\) либо \begin{icloze}{1}не пересекаются,\end{icloze} либо \begin{icloze}{1}совпадают.\end{icloze}
\end{note}

\begin{note}{2bafb8136a75400481ba0f463cb2dc9c}
    Пусть \({ (G, \circ) }\) --- группа, \({ (H, \circ) }\) --- подгруппа,\: \begin{icloze}{3}\({ g \in G }\).\end{icloze}
    \begin{icloze}{2}Тогда количество элементов в \({ g \circ H }\)\end{icloze} равно \begin{icloze}{1}количеству элементов в \({ H }\).\end{icloze}
\end{note}

\begin{note}{45bd2c9c51fd4a398ac4adaf9172dfc6}
    Пусть \({ (G, \circ) }\) --- группа.
    \begin{icloze}{1}Количество элементов в \({ G }\)\end{icloze} называется \begin{icloze}{2}порядком группы \({ (G, \circ) }\).\end{icloze}
\end{note}

\begin{note}{0590e16f8b204e27a704de1a4d810d76}
    Пусть \({ (G, \circ) }\) --- \begin{icloze}{3}конечная группа,\end{icloze} \begin{icloze}{2}\({ (H, \circ) }\) --- подгруппа.\end{icloze}
    Тогда \begin{icloze}{1}порядок группы \({ G }\) делится на порядок группы \({ H }\).\end{icloze}

    \begin{center}
        \tiny
        <<\begin{icloze}{4}Теорема Лагранжа\end{icloze}>>
    \end{center}
\end{note}

\begin{note}{6bbf33cf39f34f34afa5cf2be59fd219}
    В чём основная идея доказательства теоремы Лагранжа для конечных групп?

    \begin{cloze}{1}
        Представить \({ G }\) как конечное объединение непересекающихся классов смежности \({ g_i \circ H }\).
    \end{cloze}
\end{note}

\begin{note}{7fa9d6859025408f868211197328bf30}
    Пусть \({ (G, \circ) }\) ---  группа, \({ (H, \circ) }\) --- подгруппа,\: \({ a, b \in G }\).
    Тогда
    \[
        \begin{icloze}{2}(a \circ H) \cdot (b \circ H)\end{icloze} \overset{\text{def}}= \begin{icloze}{1}(a \circ b) \circ H\end{icloze}
    \]
\end{note}

\begin{note}{daf66fd18e1b4e50b007b6a820bfc2b7}
    Пусть \({ (G, \circ) }\) --- группа, \({ (H, \circ) }\) --- подгруппа.
    Подгруппа \({ (H, \circ) }\) называется \begin{icloze}{2}нормальной,\end{icloze} если
    \begin{icloze}{1}
        \[
            \forall g \in G  \quad g \circ H = H \circ g.
        \]
    \end{icloze}
\end{note}

\begin{note}{94d010d0ee6748df9997775ed206113d}
    Пусть \({ (G, \circ) }\) --- группа, \({ (H, \circ) }\) --- подгруппа.
    Тогда если \begin{icloze}{3}\({ (H, \circ) }\) --- нормальная подгруппа,\end{icloze} то
    \begin{center}
        \begin{icloze}{2}\({ \big(\left\{ g \circ H \mid g \in G \right\}, \:\cdot\:\big) }\)\end{icloze} --- \begin{icloze}{1}группа.\end{icloze}
    \end{center}
\end{note}

\begin{note}{f8a3768ad050440ab84f132b57ff2665}
    Пусть \({ (G, \circ) }\) --- группа, \({ (H, \circ) }\) --- подгруппа.
    Тогда если \({ (H, \circ) }\) --- нормальная подгруппа, то
    \begin{center}
        \({ \big(\left\{ g \circ H \mid g \in G \right\}, \:\cdot\:\big) }\) --- группа.
    \end{center}
    Почему важно, что \({ (H, \circ) }\) --- нормальная подгруппа?

    \begin{cloze}{1}
        В противном случае операция умножения может не быть корректно определённой.
    \end{cloze}
\end{note}

\begin{note}{6df4f13013d04e2d81bc271465e769b9}
    Пусть \({ (G, \circ) }\) --- группа, \({ (H, \circ) }\) --- нормальная подгруппа.
    \begin{icloze}{2}Группа классов смежности по подгруппе \({ H }\)\end{icloze} называется \begin{icloze}{1}фактор группой группы \({ (G, \circ) }\) по подгруппе \({ H }\).\end{icloze}
\end{note}

\begin{note}{30b68180adac4dab81ea034157975d43}
    Пусть \({ (G, \circ) }\) --- группа, \({ (H, \circ) }\) --- нормальная подгруппа.
    \begin{icloze}{2}Фактор группа \({ (G, \circ) }\) по подгруппе \({ H }\)\end{icloze} обозначается
    \begin{icloze}{1}
        \[
            G / H.
        \]
    \end{icloze}
\end{note}

\begin{note}{ce44f6c96679478284d511f7a3be6f0e}
    \[
        \begin{icloze}{2}\mathbb Z_n\end{icloze} \simeq \begin{icloze}{1}\mathbb Z / n \mathbb Z.\end{icloze}
    \]
\end{note}

\begin{note}{9f1bb49a26844d51a59a5c4aac626fa9}
    Как показать, что \({ f : k \mapsto k + n \mathbb Z, \: \mathbb Z_n \to \mathbb Z / n \mathbb Z }\) --- биекция?

    \begin{cloze}{1}
        Из теоремы о делении с остатком определить \({ f^{-1} }\).
    \end{cloze}
\end{note}

\end{document}

% vim: spelllang=ru_yo,en
