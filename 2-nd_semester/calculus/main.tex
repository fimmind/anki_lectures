\documentclass[11pt, a5paper]{article}
\usepackage[width=10cm, top=0.5cm, bottom=2cm]{geometry}

\usepackage[T1,T2A]{fontenc}
\usepackage[utf8]{inputenc}
\usepackage[english,russian]{babel}
\usepackage{libertine}

\usepackage{amsmath}
\usepackage{amssymb}
\usepackage{amsthm}
\usepackage{mathrsfs}
\usepackage{framed}
\usepackage{xcolor}

\setlength{\parindent}{0pt}

\renewcommand{\thesection}{}
\renewcommand{\thesubsection}{Note \arabic{subsection}}
\renewcommand{\thesubsubsection}{}
\renewcommand{\theparagraph}{}

\newenvironment{note}[1]{\goodbreak\par\subsection{\hfill \color{lightgray}\tiny #1}}{}
\newenvironment{cloze}[2][\ldots]{\begin{leftbar}}{\end{leftbar}}
\newenvironment{icloze}[2][\ldots]{%
  \ignorespaces\text{\tiny \color{lightgray} \{\{c#2::}\ %
}{%
  \ \text{\tiny\color{lightgray}\}\}}\unskip%
}

\begin{document}
\section{Лекция 07.02.22}
\begin{note}{662fbe59ca984f5b820ad1041f1eb840}
    Пусть \( f(x) : D \subset \mathbb R  \to \mathbb R, a \in D. \)
    \begin{icloze}{2}Многочлен \( p(x) \) степени \( n \)  такой, что \[
        \begin{gathered}
            f(x) = p(x) + o((x - a)^{n}), \\
            f(a) = p(a),
        \end{gathered}
    \]\end{icloze}
    называется \begin{icloze}{1}многочленом Тейлора функции \( f \) порядка \( n \) в точке \( a. \)\end{icloze}
\end{note}

\begin{note}{738279ec323b45e29a170a4e41b4bce0}
    Если многочлен Тейлора функции \( f \) порядка \( n \) в точке \( a \) существует, то \begin{icloze}{1}он единственен.\end{icloze}
\end{note}

\begin{note}{8f605243b193465799ba06e1576d171e}
    В чём ключевая идея доказательства единственности многочлена Тейлора?

    \begin{cloze}{1}
        Пусть коэффициент \( r_m  \) при \( (x - a)^{m} \) --- первый ненулевой коэффициент в многочлене \( p - q \).
        Тогда \[
            \frac{p - q}{(x - a)^{m} } \underset{x \to a}\longrightarrow r_m,
        \] но при этом \[
            \frac{p - q}{(x - a)^{m} } = o((x - a)^{n - m}) \underset{x \to a}\longrightarrow 0 \implies r_m = 0.
        \]
    \end{cloze}
\end{note}

\begin{note}{f4110a9b63c640be96d810d835d0d1fd}
    \begin{icloze}{2}Многочлен Тейлора функции \( f \) порядка \( n \) в точке \( a \)\end{icloze} обозначается \begin{icloze}{1}\( T_{a, n} f.  \)\end{icloze}
\end{note}

\begin{note}{1b7244a616994615a1d41bbc85768a3f}
    \subsubsection{<<\begin{icloze}{3}Формула Тейлора для многочленов\end{icloze}>>}

    Пусть \( p \) --- \begin{icloze}{2}многочлен степени не более \( n \).\end{icloze} Тогда \begin{icloze}{1}
        \[
            p(x) = \sum_{k=0}^{n} \frac{p^{(k)} (a)}{k! } (x - a)^{k}.
        \]
    \end{icloze}
\end{note}

\begin{note}{97c12315facb454e987cb94fae99be75}
\[
    \left. f(x) \right|_{x = a} \overset{\text{def}}= \begin{icloze}{1}f(a).\end{icloze}
\]
\end{note}

\begin{note}{cf7e5ab30b564c139557fd0a940f8204}
    \[
        \left. \left((x - a)^{k} \right)^{(n)}  \right|_{x = a} =
        \begin{icloze}{1}
            \begin{cases}
                0, & n \neq k, \\
                n!, & n = k.
            \end{cases}
        \end{icloze}
    \]
\end{note}

\begin{note}{9b6c61f4867142bea860ca4d00c07174}
    В чем основная идея доказательства истинности формулы Тейлора для многочленов?

    \begin{cloze}{1}
        Записать \( p(x) \) с неопределенными коэффициентами и вычислить \( p^{(k)} (a) \) для \( k = 0, 1, 2, \ldots, n \).
    \end{cloze}
\end{note}

\begin{note}{7597b782ce5f4e92998cc6445ce6f40e}
    \subsubsection{<<\begin{icloze}{3}Свойство n раз дифференцируемой функции\end{icloze}>>}

    Пусть \begin{icloze}{2}\( f : D \subset R \to \mathbb R, a \in D, n \in \mathbb N \) и \[
        f(a) = f'(a) = \cdots = f^{(n)} (a) = 0.
    \]\end{icloze}
    Тогда \begin{icloze}{1}\( f(x) = o((x - a)^{n} ),  x \to a \).\end{icloze}
\end{note}

\begin{note}{22aa07051d4c4e0ebb08ce0114be5429}
    \subsubsection{<<Определение \( o \)-малого в терминах \( \varepsilon, \delta. \)>>}

    Пусть \( f, g : D \subset \mathbb R \to \mathbb R \), \( a \) --- предельная точка \( D \). Тогда
    \begin{multline*}
        f(x) = o(g(x)), \quad x \to a \overset{\hspace{-1pt}\text{\tiny def}}\iff \\
        \begin{icloze}{1}
            \forall \varepsilon > 0 \quad \exists \delta > 0 \quad \forall x \in D \cap \dot V_{\delta}(a) \quad |f(x)| \leqslant \varepsilon |g(x)|.
        \end{icloze}
    \end{multline*}
\end{note}

\begin{note}{b7ddf1bbcdf84c769dd7b409e5be494d}
    Какой метод используется в доказательстве свойства \( n \)-раз дифференцируемой функции?

    \begin{cloze}{1}
        Индукция по \( n. \)
    \end{cloze}
\end{note}

\begin{note}{f04179797fd64614827341d425616341}
    Какова основная идея в доказательстве свойства \( n \)-раз дифференцируемой функции (базовый случай)?

    \begin{cloze}{1}
        Подставить \( f(a) = f'(a) = 0 \) в определение дифференцируемости.
    \end{cloze}
\end{note}

\begin{note}{7a10e93958724ee6b93bc1637a13773f}
    Каков первый шаг в доказательстве свойства \( n \)-раз дифференцируемой функции (индукционный переход)?

    \begin{cloze}{1}
        Заметить, что из индукционного предположения
        \[
            f'(x) = o((x - a)^{n} )
        \]
        и расписать это равенство в терминах \( \varepsilon, \delta. \)
    \end{cloze}
\end{note}

\begin{note}{b863b13c8a8b45c09c6444b48e5c0b75}
    Какие ограничения накладываются на \( \delta \) в доказательстве свойства \( n \)-раз дифференцируемой функции (индукционный переход)?

    \begin{cloze}{1}
        \( V_{\delta} (a) \cap D \) есть невырожденный промежуток.
    \end{cloze}
\end{note}

\begin{note}{2506d5781f234e13a94358880699831a}
    Почему в доказательстве свойства \( n \)-раз дифференцируемой функции (индукционный переход) мы можем сказать, что
    \( \exists \delta > 0 \) такой,  что \( V_{\delta} (a) \cap D \) есть невырожденный отрезок?

    \begin{cloze}{1}
        По определению дифференцируемости функции.
    \end{cloze}
\end{note}

\begin{note}{73ed2cdbb8b444ce991d587d9ed279ed}
    В чем ключевая идея доказательства свойства \( n \)-раз дифференцируемой функции (индукционный переход)?

    \begin{cloze}{1}
        Выразить \( f(x) = f'(c) \cdot (x - a) \) по симметричной формуле конечных приращений и показать, что \( |f'(c)| < \varepsilon |x - a|^{n} \).
    \end{cloze}
\end{note}

\begin{note}{a08796d96ad841bd91a8e7daaab1857d}
    Откуда следует, что \( |f'(c)| < \varepsilon|x - a|^{n}  \) в доказательстве свойства \( n \)-раз дифференцируемой функции (индукционный переход)?

    \begin{cloze}{1}
        \[
            |c - a| < \delta \implies |f'(c)| < \varepsilon|c - a|^{n} < \varepsilon|x - a|^{n}
        \]
    \end{cloze}
\end{note}

\begin{note}{957fd9747bd84545bd6b1cca723d72ba}
    Пусть \( f : D \subset \mathbb R \to \mathbb R, a \in D, n \in \mathbb N \), \begin{icloze}{2}\(f(a) = 0, \)
    \[
        f'(x) = o((x - a)^{n} ), \quad x \to a.
    \]\end{icloze}

    Тогда \( f(x) = \begin{icloze}{1}o((x - a)^{n + 1}), \quad x \to a.\end{icloze} \)
\end{note}

\begin{note}{99a8f041e1a34dba923a682c6500c46b}
    \subsubsection{<<\begin{icloze}{3}Формула Тейлора-Пеано\end{icloze}>>}

    Пусть \begin{icloze}{2}\( f : D \subset R \to \mathbb R, a \in D, n \in \mathbb N \) и \( f \) \( n \) раз дифференцируема в точке \( a. \)  \end{icloze}
    Тогда \begin{icloze}{1}\[
        f(x) = \sum_{k=0}^{n} \frac{f^{(k)} (a)}{k!} (x - a)^{k} + o((x - a)^{n} ).
    \]\end{icloze}
\end{note}

\section{Лекция 11.02.22}
\begin{note}{3bf65c72c3374838aecaa626de8a3a4d}
    Каков первый шаг в доказательстве истинности формулы Тейлора-Пеано?

    \begin{cloze}{1}
        Обозначить через \( p(x) \) многочлен в формуле:
        \[
            f(x) = \underbrace{\sum_{k=0}^{n} \frac{f^{(k)} (a)}{k!} (x - a)^{k}}_{p(x)}  + o((x - a)^{n} ).
        \]
    \end{cloze}
\end{note}

\begin{note}{6f41684761ec41308bf9f95619ec1849}
    Чему для \( k \leqslant n \) равна \( p^{(k)} (a) \) в доказательстве истинности формулы Тейлора-Пеано?

    \begin{cloze}{1}
        \[
            p^{(k)} (a) = f^{(k)} (a).
        \]
    \end{cloze}
\end{note}

\begin{note}{72455c0671414c80aca4c9ef2ba63d44}
    В чем основная идея доказательства истинности формулы Тейлора-Пеано?

    \begin{cloze}{1}
        По свойству \( n \) раз дифференцируемой функции \( f(x) - p(x) = o((x - a)^{n} )\).
    \end{cloze}
\end{note}

\begin{note}{db6e4a55afed4c5d95a38869cf9d2e00}
    Что позволяет применить свойство \( n \) раз дифференцируемой функции в доказательстве формулы Тейлора-Пеано?

    \begin{cloze}{1}
        \[
            \forall k \left. \leqslant n \quad \left(f(x) - p(x)\right)^{(k)} \right|_{x = a} = 0
        \]
    \end{cloze}
\end{note}

\begin{note}{8c823210f5c94ab99024c3e8c3d6778a}
    \[
        \begin{icloze}{2}\Delta _{a, b}\end{icloze}
        \overset{\text{def}}=
        \begin{icloze}{1}\begin{cases}
            [a, b], & a \leqslant b, \\
            [b, a], & a \geqslant b.
        \end{cases}\end{icloze}
    \]
\end{note}

\begin{note}{9755fb6343494fa9b0034b4542e518d3}
    \[
        \begin{icloze}{2}\widetilde \Delta _{a, b}\end{icloze}
        \overset{\text{def}}=
        \begin{icloze}{1}\begin{cases}
            (a, b), & a \leqslant b, \\
            (b, a), & a \geqslant b.
        \end{cases}\end{icloze}
    \]
\end{note}

\begin{note}{dbb25fcd6e834aa2ae54ec6ddc0c6787}
    \[
        \begin{icloze}{2}R_{a, n} f \end{icloze}
        \overset{\text{def}}=
        \begin{icloze}{1}f - T_{a, n} f\end{icloze}
    \]
\end{note}

\begin{note}{0d92b12a18f34554a0251578aa811b7f}
    \subsubsection{<<\begin{icloze}{3}Формула Тейлора-Лагранжа\end{icloze}>>}

    Пусть \( f : D \subset \mathbb R \to \mathbb R \), \quad \( a, x \in \mathbb R, a \neq x \), \quad \begin{icloze}{2}\( f \in C^{n} (\Delta _{a, x} ) \), \( f^{(n)} \) дифференцируема на \( \widetilde \Delta _{a, x}  \).\end{icloze} Тогда
    \begin{icloze}{1}
        найдется \( c \in \widetilde \Delta _{a, x}  \), для которой
        \[
            f(x) = T_{a, n} f (x) + \frac{f^{(n + 1)} (c)}{(n + 1)! }  (x - a)^{n + 1}.
        \]
    \end{icloze}
\end{note}

\begin{note}{f9314b4b0e184f52826c8f740c873e21}
    При \( n = 0 \) формула Тейлора-Лагранжа эквивалентна \begin{icloze}{1}теореме Лагранжа\end{icloze}.
\end{note}

\begin{note}{5fe508cfd3c445c4b15093e8d2c8c504}
    В чем основная идея доказательства истинности формулы Тейлора-Лагранжа?

    \begin{cloze}{1}
        Вычислить производную функции \( F(t) = R_{t, n} f(x) \) и найти точку \( c \) по теореме Коши.
    \end{cloze}
\end{note}

\begin{note}{e1a329fbc3ef4c5981773d8baad7d3b1}
    Для каких \( t \) определяется функция \( F(t) \) в доказательстве истинности формулы Тейлора-Лагранжа?

    \begin{cloze}{1}
        \[
            t \in \Delta _{a, x}.
        \]
    \end{cloze}
\end{note}

\begin{note}{a4f7e43161cc4c9fb58ac7a250610c50}
    Для каких \( t \) вычисляется \( F'(t) \) в доказательстве истинности формулы Тейлора-Лагранжа?

    \begin{cloze}{1}
        \[
            t \in \widetilde \Delta _{a, x}.
        \]
    \end{cloze}
\end{note}

\begin{note}{73e4df5e1b074010a95ee5dbe0458338}
    К каким функциям применяется теорема Коши в доказательстве истинности формулы Тейлора-Лагранжа?

    \begin{cloze}{1}
        К \( F(t) \) и \( \varphi(t) = (x - t)^{n + 1}  \).
    \end{cloze}
\end{note}

\begin{note}{b1d63dae062e4a438ceb891f94a33e96}
    К каким точкам применяется теорема Коши в доказательстве истинности формулы Тейлора-Лагранжа?

    \begin{cloze}{1}
        К границам отрезка \( \Delta _{a, x}  \).
    \end{cloze}
\end{note}

\begin{note}{b8f3f99b66794d59b6fa546eb06d7fb3}
    Какое неявное условие позволяет применить теорему коши к функциям \( F(t) \) и \( \varphi(t) \) с точках \( a \) и \( x \)?

    \begin{cloze}{1}
        \[
            F(x) = 0, \quad \varphi(x) = 0.
        \]
    \end{cloze}
\end{note}

\begin{note}{e425a1ef13124799b6b391e3884f86f1}
    По формуле Тейлора-Пиано при \( x \to 0 \)
    \[
        \begin{icloze}{2}e^{x}\end{icloze} = \begin{icloze}{1}\sum_{k=0}^{n} \frac{x^{k} }{k! } + o(x^{n} ).\end{icloze}
    \]
\end{note}

\begin{note}{70a13102af174271b95762b24e6b1169}
    По формуле Тейлора-Пиано при \( x \to 0 \)
    \[
        \begin{icloze}{2}\sin x\end{icloze} = \begin{icloze}{1}\sum_{k=0}^{n} (-1)^{k} \frac{x^{2k + 1} }{(2k + 1)! } + o\!\left(x^{2n + 2}\right). \end{icloze}
    \]
\end{note}

\begin{note}{9c528f645b0741ef90f268989f7701eb}
    По формуле Тейлора-Пиано при \( x \to 0 \)
    \[
        \begin{icloze}{2}\cos x\end{icloze} = \begin{icloze}{1}\sum_{k=0}^{n} (-1)^{k} \frac{x^{2k} }{(2k)! } + o\!\left(x^{2n + 1}\right). \end{icloze}
    \]
\end{note}

\begin{note}{90ff22c33f67493fae3fa800e93905f4}
    По формуле Тейлора-Пиано при \( x \to 0 \)
    \[
        \begin{icloze}{2}\ln (1 + x)\end{icloze} = \begin{icloze}{1}\sum_{k=1}^{n} (-1)^{k - 1} \frac{x^{k} }{k } + o\!\left(x^{n}\right). \end{icloze}
    \]
\end{note}

\begin{note}{aaf8ef38d3bb409baf7c7fcc1df14f48}
    \begin{icloze}{3}Обобщённый биномиальный коэффициент\end{icloze} задаётся формулой
    \[
        C_\alpha^k = \begin{icloze}{1}\frac{\alpha(\alpha - 1) \cdots (\alpha - k + 1)}{k! }\end{icloze}, \quad \alpha \in \begin{icloze}{2}\mathbb R\end{icloze}.
    \]
\end{note}

\begin{note}{5ed01e7f4e8e4b22adf1929f60e4d4f5}
    По формуле Тейлора-Пиано при \( x \to 0 \)
    \[
        \begin{icloze}{2}(1 + x)^{\alpha} \end{icloze} = \begin{icloze}{1}\sum_{k=0}^{n} C_\alpha^k x^{k} + o\!\left(x^{n}\right). \end{icloze}
    \]
\end{note}

\begin{note}{eb36b5f5a2b04e44b4d5b13d2278ff40}
    Формулу Тейлора-Пеано для \( (1 + x)^{\alpha} \) называют \begin{icloze}{1}биномиальным разложением\end{icloze}.
\end{note}

\begin{note}{7d3d35d9fcb344458f0d82ed7b2d940f}
    Пусть \begin{icloze}{3}функция \( f \) удовлетворяет условиям для разложения по формуле Тейлора-Лагранжа.\end{icloze}
    Тогда если
    \begin{icloze}{2}
        \[
            \forall t \in \widetilde \Delta _{a, x} \quad |f^{(n + 1)} (t)| \leqslant M,
        \]
    \end{icloze}
    то
    \begin{icloze}{1}
        \[
            |R_{a, n} f(x)| \leqslant \frac{M|x - a|^{n + 1} }{(n + 1)! }.
        \]
    \end{icloze}
\end{note}
\end{document}
