\documentclass[11pt, a5paper]{article}
\usepackage[width=10cm, top=0.5cm, bottom=2cm]{geometry}

\usepackage[T1,T2A]{fontenc}
\usepackage[utf8]{inputenc}
\usepackage[english,russian]{babel}
\usepackage{libertine}

\usepackage{amsmath}
\usepackage{amssymb}
\usepackage{amsthm}
\usepackage{mathrsfs}
\usepackage{framed}
\usepackage{xcolor}

\setlength{\parindent}{0pt}

% Force \pagebreak for every section
\let\oldsection\section
\renewcommand\section{\pagebreak\oldsection}

\renewcommand{\thesection}{}
\renewcommand{\thesubsection}{Note \arabic{subsection}}
\renewcommand{\thesubsubsection}{}
\renewcommand{\theparagraph}{}

\newenvironment{note}[1]{\goodbreak\par\subsection{\hfill \color{lightgray}\tiny #1}}{}
\newenvironment{cloze}[2][\ldots]{\begin{leftbar}}{\end{leftbar}}
\newenvironment{icloze}[2][\ldots]{%
  \ignorespaces\text{\tiny \color{lightgray}\{\{c#2::}\hspace{0pt}%
}{%
  \hspace{0pt}\text{\tiny\color{lightgray}\}\}}\unskip%
}


\begin{document}
\section{Лекция 07.02.22}
\begin{note}{662fbe59ca984f5b820ad1041f1eb840}
    Пусть \( f(x) : D \subset \mathbb R  \to \mathbb R, a \in D. \)
    \begin{icloze}{2}Многочлен \( p(x) \) степени \( n \)  такой, что \[
        \begin{gathered}
            f(x) = p(x) + o((x - a)^{n}), \\
            f(a) = p(a),
        \end{gathered}
    \]\end{icloze}
    называется \begin{icloze}{1}многочленом Тейлора функции \( f \) порядка \( n \) в точке \( a. \)\end{icloze}
\end{note}

\begin{note}{738279ec323b45e29a170a4e41b4bce0}
    Если многочлен Тейлора функции \( f \) порядка \( n \) в точке \( a \) существует, то \begin{icloze}{1}он единственен.\end{icloze}
\end{note}

\begin{note}{8f605243b193465799ba06e1576d171e}
    В чём ключевая идея доказательства единственности многочлена Тейлора?

    \begin{cloze}{1}
        Младший ненулевой коэффициент при \({ (x - a)^{m} }\) в \({ p - q }\) равен нулю \({ \implies \bot }\). (Доказательство от противного.)
    \end{cloze}
\end{note}

\begin{note}{91af14a03bea4b9fadee06859cbab64d}
    Пусть \({ p }\) и \({ q }\) --- два многочлена Тейлора функции \({ f }\),\: коэффициент \({ r_m }\) перед \({ (x - a)^{m} }\) --- младший ненулевой коэффициент в \({ p - q }\). Как показать, что \({ r_m = 0 }\)?

    \begin{cloze}{1}
        Рассмотреть многочлен
        \[
            \frac{p(x) - q(x)}{(x - a)^{m}}.
        \]
    \end{cloze}
\end{note}

\begin{note}{f4110a9b63c640be96d810d835d0d1fd}
    \begin{icloze}{2}Многочлен Тейлора функции \( f \) порядка \( n \) в точке \( a \)\end{icloze} обозначается \begin{icloze}{1}\( T_{a, n} f.  \)\end{icloze}
\end{note}

\begin{note}{1b7244a616994615a1d41bbc85768a3f}
    \subsubsection{<<\begin{icloze}{3}Формула Тейлора для многочленов\end{icloze}>>}

    Пусть \( p \) --- \begin{icloze}{2}многочлен степени не более \( n \).\end{icloze} Тогда \begin{icloze}{1}
        \[
            p(x) = \sum_{k=0}^{n} \frac{p^{(k)} (a)}{k! } (x - a)^{k}.
        \]
    \end{icloze}
\end{note}

\begin{note}{97c12315facb454e987cb94fae99be75}
\[
    \left. f(x) \right|_{x = a} \overset{\text{def}}= \begin{icloze}{1}f(a).\end{icloze}
\]
\end{note}

\begin{note}{cf7e5ab30b564c139557fd0a940f8204}
    \[
        \left. \left((x - a)^{k} \right)^{(n)}  \right|_{x = a} =
        \begin{icloze}{1}
            \begin{cases}
                0, & n \neq k, \\
                n!, & n = k.
            \end{cases}
        \end{icloze}
    \]
\end{note}

\begin{note}{9b6c61f4867142bea860ca4d00c07174}
    В чем основная идея доказательства истинности формулы Тейлора для многочленов?

    \begin{cloze}{1}
        Записать \( p(x) \) с неопределенными коэффициентами и вычислить \( p^{(k)} (a) \) для \( k = 0, 1, 2, \ldots, n \).
    \end{cloze}
\end{note}

\begin{note}{7597b782ce5f4e92998cc6445ce6f40e}
    \subsubsection{<<\begin{icloze}{3}Свойство n раз дифференцируемой функции\end{icloze}>>}

    Пусть \( f : D \subset \mathbb R \to \mathbb R, a \in D \) и
    \begin{icloze}{2}
        \[
            f(a) = f'(a) = \cdots = f^{(n)} (a) = 0.
        \]
    \end{icloze}
    Тогда \begin{icloze}{1}\( f(x) = o((x - a)^{n} ),  x \to a \).\end{icloze}
\end{note}

\begin{note}{22aa07051d4c4e0ebb08ce0114be5429}
    \subsubsection{<<Определение \( o \)-малого в терминах \( \varepsilon, \delta. \)>>}

    Пусть \( f, g : D \subset \mathbb R \to \mathbb R \), \( a \) --- предельная точка \( D \). Тогда
    \begin{multline*}
        f(x) = o(g(x)), \quad x \to a \overset{\hspace{-1pt}\text{\tiny def}}\iff \\
        \begin{icloze}{1}
            \forall \varepsilon > 0 \quad \exists \delta > 0 \quad \forall x \in D \cap \dot V_{\delta}(a) \quad |f(x)| \leqslant \varepsilon |g(x)|.
        \end{icloze}
    \end{multline*}
\end{note}

\begin{note}{b7ddf1bbcdf84c769dd7b409e5be494d}
    Какой метод используется в доказательстве свойства \( n \)-раз дифференцируемой функции?

    \begin{cloze}{1}
        Индукция по \( n. \)
    \end{cloze}
\end{note}

\begin{note}{f04179797fd64614827341d425616341}
    Какова основная идея в доказательстве свойства \( n \)-раз дифференцируемой функции (базовый случай)?

    \begin{cloze}{1}
        Подставить \( f(a) = f'(a) = 0 \) в определение дифференцируемости.
    \end{cloze}
\end{note}

\begin{note}{7a10e93958724ee6b93bc1637a13773f}
    Каков первый шаг в доказательстве свойства \( n \)-раз дифференцируемой функции (индукционный переход)?

    \begin{cloze}{1}
        Заметить, что из индукционного предположения
        \[
            f'(x) = o((x - a)^{n} )
        \]
        и расписать это равенство в терминах \( \varepsilon, \delta. \)
    \end{cloze}
\end{note}

\begin{note}{b863b13c8a8b45c09c6444b48e5c0b75}
    Какие ограничения накладываются на \( \delta \) в доказательстве свойства \( n \)-раз дифференцируемой функции (индукционный переход)?

    \begin{cloze}{1}
        \( V_{\delta} (a) \cap D \) есть невырожденный промежуток.
    \end{cloze}
\end{note}

\begin{note}{2506d5781f234e13a94358880699831a}
    Почему в доказательстве свойства \( n \)-раз дифференцируемой функции (индукционный переход) мы можем сказать, что
    \( \exists \delta > 0 \) такой,  что \( V_{\delta} (a) \cap D \) есть невырожденный отрезок?

    \begin{cloze}{1}
        По определению дифференцируемости функции.
    \end{cloze}
\end{note}

\begin{note}{73ed2cdbb8b444ce991d587d9ed279ed}
    В чем ключевая идея доказательства свойства \( n \)-раз дифференцируемой функции (индукционный переход)?

    \begin{cloze}{1}
        Выразить \( f(x) = f'(c) \cdot (x - a) \) по симметричной формуле конечных приращений и показать, что \( |f'(c)| < \varepsilon |x - a|^{n} \).
    \end{cloze}
\end{note}

\begin{note}{a08796d96ad841bd91a8e7daaab1857d}
    Откуда следует, что \( |f'(c)| < \varepsilon|x - a|^{n}  \) в доказательстве свойства \( n \)-раз дифференцируемой функции (индукционный переход)?

    \begin{cloze}{1}
        \[
            |c - a| < \delta \implies |f'(c)| < \varepsilon|c - a|^{n} < \varepsilon|x - a|^{n}
        \]
    \end{cloze}
\end{note}

\begin{note}{957fd9747bd84545bd6b1cca723d72ba}
    Пусть \( f : D \subset \mathbb R \to \mathbb R, a \in D, n \in \mathbb N \), \begin{icloze}{3}\( f(a) = 0 \),\end{icloze}
    \begin{icloze}{2}
        \[
            f'(x) = o((x - a)^{n} ), \quad x \to a.
        \]
    \end{icloze}

    Тогда \( f(x) = \begin{icloze}{1}o((x - a)^{n + 1}), \quad x \to a.\end{icloze} \)
\end{note}

\begin{note}{99a8f041e1a34dba923a682c6500c46b}
    \subsubsection{<<\begin{icloze}{3}Формула Тейлора-Пеано\end{icloze}>>}

    Пусть \begin{icloze}{2}\( f : D \subset R \to \mathbb R \) и \( f \) \( n \) раз дифференцируема в точке \( a. \)  \end{icloze}
    Тогда \begin{icloze}{1}\[
        f(x) = \sum_{k=0}^{n} \frac{f^{(k)} (a)}{k!} (x - a)^{k} + o((x - a)^{n} ).
    \]\end{icloze}
\end{note}

\section{Лекция 11.02.22}
\begin{note}{3bf65c72c3374838aecaa626de8a3a4d}
    Каков первый шаг в доказательстве истинности формулы Тейлора-Пеано?

    \begin{cloze}{1}
        Обозначить через \( p(x) \) многочлен в формуле:
        \[
            f(x) = \underbrace{\sum_{k=0}^{n} \frac{f^{(k)} (a)}{k!} (x - a)^{k}}_{p(x)}  + o((x - a)^{n} ).
        \]
    \end{cloze}
\end{note}

\begin{note}{6f41684761ec41308bf9f95619ec1849}
    Чему для \( k \leqslant n \) равна \( p^{(k)} (a) \) в доказательстве истинности формулы Тейлора-Пеано?

    \begin{cloze}{1}
        \[
            p^{(k)} (a) = f^{(k)} (a).
        \]
    \end{cloze}
\end{note}

\begin{note}{72455c0671414c80aca4c9ef2ba63d44}
    В чем основная идея доказательства истинности формулы Тейлора-Пеано?

    \begin{cloze}{1}
        По свойству \( n \) раз дифференцируемой функции \( f(x) - p(x) = o((x - a)^{n} )\).
    \end{cloze}
\end{note}

\begin{note}{db6e4a55afed4c5d95a38869cf9d2e00}
    Что позволяет применить свойство \( n \) раз дифференцируемой функции в доказательстве формулы Тейлора-Пеано?

    \begin{cloze}{1}
        \[
            \forall k \left. \leqslant n \quad \left(f(x) - p(x)\right)^{(k)} \right|_{x = a} = 0
        \]
    \end{cloze}
\end{note}

\begin{note}{8c823210f5c94ab99024c3e8c3d6778a}
    \[
        \begin{icloze}{2}\Delta _{a, b}\end{icloze}
        \overset{\text{def}}=
        \begin{icloze}{1}\begin{cases}
            [a, b], & a \leqslant b, \\
            [b, a], & a \geqslant b.
        \end{cases}\end{icloze}
    \]
\end{note}

\begin{note}{9755fb6343494fa9b0034b4542e518d3}
    \[
        \begin{icloze}{2}\widetilde \Delta _{a, b}\end{icloze}
        \overset{\text{def}}=
        \begin{icloze}{1}\begin{cases}
            (a, b), & a < b, \\
            (b, a), & a > b.
        \end{cases}\end{icloze}
    \]
\end{note}

\begin{note}{dbb25fcd6e834aa2ae54ec6ddc0c6787}
    \[
        \begin{icloze}{2}R_{a, n} f \end{icloze}
        \overset{\text{def}}=
        \begin{icloze}{1}f - T_{a, n} f\end{icloze}
    \]
\end{note}

\begin{note}{0d92b12a18f34554a0251578aa811b7f}
    \subsubsection{<<\begin{icloze}{3}Формула Тейлора-Лагранжа\end{icloze}>>}

    Пусть \( f : D \subset \mathbb R \to \mathbb R \), \quad \( a, x \in \mathbb R, a \neq x \), \quad \begin{icloze}{2}\( f \in C^{n} (\Delta _{a, x} ) \), \( f^{(n)} \) дифференцируема на \( \widetilde \Delta _{a, x}  \).\end{icloze} Тогда
    \begin{icloze}{1}
        найдется \( c \in \widetilde \Delta _{a, x}  \), для которой
        \[
            f(x) = T_{a, n} f (x) + \frac{f^{(n + 1)} (c)}{(n + 1)! }  (x - a)^{n + 1}.
        \]
    \end{icloze}
\end{note}

\begin{note}{f9314b4b0e184f52826c8f740c873e21}
    При \( n = 0 \) формула Тейлора-Лагранжа эквивалентна \begin{icloze}{1}теореме Лагранжа\end{icloze}.
\end{note}

\begin{note}{5fe508cfd3c445c4b15093e8d2c8c504}
    В чем основная идея доказательства истинности формулы Тейлора-Лагранжа?

    \begin{cloze}{1}
        Вычислить производную функции \( F(t) = R_{t, n} f(x) \) и далее по теореме Коши.
    \end{cloze}
\end{note}

\begin{note}{e1a329fbc3ef4c5981773d8baad7d3b1}
    Для каких \( t \) определяется функция \( F(t) \) в доказательстве истинности формулы Тейлора-Лагранжа?

    \begin{cloze}{1}
        \[
            t \in \Delta _{a, x}.
        \]
    \end{cloze}
\end{note}

\begin{note}{a4f7e43161cc4c9fb58ac7a250610c50}
    Для каких \( t \) вычисляется \( F'(t) \) в доказательстве истинности формулы Тейлора-Лагранжа?

    \begin{cloze}{1}
        \[
            t \in \widetilde \Delta _{a, x}.
        \]
    \end{cloze}
\end{note}

\begin{note}{73e4df5e1b074010a95ee5dbe0458338}
    К каким функциям применяется теорема Коши в доказательстве истинности формулы Тейлора-Лагранжа?

    \begin{cloze}{1}
        К \( F(t) \) и \( \varphi(t) = (x - t)^{n + 1}  \).
    \end{cloze}
\end{note}

\begin{note}{b1d63dae062e4a438ceb891f94a33e96}
    К каким точкам применяется теорема Коши в доказательстве истинности формулы Тейлора-Лагранжа?

    \begin{cloze}{1}
        К границам отрезка \( \Delta _{a, x}  \).
    \end{cloze}
\end{note}

\begin{note}{b8f3f99b66794d59b6fa546eb06d7fb3}
    Какое неявное условие позволяет применить теорему коши к функциям \( F(t) \) и \( \varphi(t) \) с точках \( a \) и \( x \) в доказательстве истинности формулы Тейлора-Лагранжа?

    \begin{cloze}{1}
        \[
            F(x) = 0, \quad \varphi(x) = 0.
        \]
    \end{cloze}
\end{note}

\begin{note}{e425a1ef13124799b6b391e3884f86f1}
    По формуле Тейлора-Пиано при \( x \to 0 \)
    \[
        \begin{icloze}{2}e^{x}\end{icloze} = \begin{icloze}{1}\sum_{k=0}^{n} \frac{x^{k} }{k! } + o(x^{n} ).\end{icloze}
    \]
\end{note}

\begin{note}{70a13102af174271b95762b24e6b1169}
    По формуле Тейлора-Пиано при \( x \to 0 \)
    \[
        \begin{icloze}{2}\sin x\end{icloze} = \begin{icloze}{1}\sum_{k=0}^{n} (-1)^{k} \frac{x^{2k + 1} }{(2k + 1)! } + o\!\left(x^{2n + 2}\right). \end{icloze}
    \]
\end{note}

\begin{note}{9c528f645b0741ef90f268989f7701eb}
    По формуле Тейлора-Пиано при \( x \to 0 \)
    \[
        \begin{icloze}{2}\cos x\end{icloze} = \begin{icloze}{1}\sum_{k=0}^{n} (-1)^{k} \frac{x^{2k} }{(2k)! } + o\!\left(x^{2n + 1}\right). \end{icloze}
    \]
\end{note}

\begin{note}{90ff22c33f67493fae3fa800e93905f4}
    По формуле Тейлора-Пиано при \( x \to 0 \)
    \[
        \begin{icloze}{2}\ln (1 + x)\end{icloze} = \begin{icloze}{1}\sum_{k=1}^{n} (-1)^{k - 1} \frac{x^{k} }{k } + o\!\left(x^{n}\right). \end{icloze}
    \]
\end{note}

\begin{note}{aaf8ef38d3bb409baf7c7fcc1df14f48}
    \begin{icloze}{3}Обобщённый биномиальный коэффициент\end{icloze} задаётся формулой
    \[
        C_\alpha^k = \begin{icloze}{1}\frac{\alpha(\alpha - 1) \cdots (\alpha - k + 1)}{k! }\end{icloze}, \quad \alpha \in \begin{icloze}{2}\mathbb R\end{icloze}.
    \]
\end{note}

\begin{note}{5ed01e7f4e8e4b22adf1929f60e4d4f5}
    По формуле Тейлора-Пиано при \( x \to 0 \)
    \[
        \begin{icloze}{2}(1 + x)^{\alpha} \end{icloze} = \begin{icloze}{1}\sum_{k=0}^{n} C_\alpha^k x^{k} + o\!\left(x^{n}\right). \end{icloze}
    \]
\end{note}

\begin{note}{eb36b5f5a2b04e44b4d5b13d2278ff40}
    Формулу Тейлора-Пеано для \( (1 + x)^{\alpha} \) называют \begin{icloze}{1}биномиальным разложением\end{icloze}.
\end{note}

\begin{note}{c766c427b7e44be8a2e40e872ec7dd2b}
    \[
        C_{-1}^{k} = \begin{icloze}{1}(-1)^{k}.\end{icloze}
    \]
\end{note}

\begin{note}{82717b22134b4f66b014c17df3ba337c}
    По формуле Тейлора-Пиано при \( x \to 0 \)
    \[
        \begin{icloze}{2}(1 + x)^{-1} \end{icloze} = \begin{icloze}{1}\sum_{k=0}^{n} (-1)^{k} x^{k} + o\!\left(x^{n}\right). \end{icloze}
    \]
\end{note}

\begin{note}{7d3d35d9fcb344458f0d82ed7b2d940f}
    Пусть \begin{icloze}{3}функция \( f \) удовлетворяет условиям для разложения по формуле Тейлора-Лагранжа.\end{icloze}
    Тогда если
    \begin{icloze}{2}
        \[
            \forall t \in \widetilde \Delta _{a, x} \quad |f^{(n + 1)} (t)| \leqslant M,
        \]
    \end{icloze}
    то
    \begin{icloze}{1}
        \[
            |R_{a, n} f(x)| \leqslant \frac{M|x - a|^{n + 1} }{(n + 1)! }.
        \]
    \end{icloze}
\end{note}

\section{Семинар 17.02.22}
\begin{note}{05fb49aabf444b3daf73947c33bf8f10}
    \[
        \int x^{n} \: dx = \begin{icloze}{1}\frac{x^{n+1} }{n + 1} + C\end{icloze}, \quad (\begin{icloze}{2}n \neq -1\end{icloze}).
    \]
\end{note}

\begin{note}{3eae90c7fe9944e6a9d07784205f0d1d}
    \[
        \int \begin{icloze}{2}\frac{1}{x}\end{icloze} \: dx = \begin{icloze}{1}\ln |x| + C\end{icloze}.
    \]
\end{note}

\begin{note}{af533d11b4c2421baaad26c4fca61b2a}
    \[
        \int \begin{icloze}{2}\frac{1}{1 - x^2 }\end{icloze} \: dx = \begin{icloze}{1}\frac{1}{2} \ln\left|\frac{1 + x}{1 - x}\right| + C\end{icloze}.
    \]
\end{note}

\begin{note}{8939b90686dc43ae81c37c01fa728294}
    \[
        \begin{gathered}
            \int \begin{icloze}{2}\frac{1}{\sqrt{x^2 \pm a^2} }\end{icloze} \: dx = \begin{icloze}{1}\ln |x + \sqrt{x^2 \pm a^2}| + C\end{icloze}.
        \end{gathered}
    \]
\end{note}

\begin{note}{edb57ab590834e5db5946311b9910393}
    \[
        \int \frac{1}{\sqrt{\begin{icloze}{1}x^2 \pm a^2\end{icloze}} } \: dx = \ln |x + \sqrt{\begin{icloze}{1}x^2 \pm a^2\end{icloze}}| + C.
    \]
\end{note}

\begin{note}{709b5fa5f404426ea7b67b17dc16f830}
    \[
        \int a^{x} \: dx = \begin{icloze}{1}\frac{a^{x} }{\ln a} + C\end{icloze}.
    \]
\end{note}

\section{Лекция 18.02.22}
\begin{note}{b55d92bf361d4e31b5e60975656b3fb4}
    Пусть \begin{icloze}{4}\( f \in C\langle A, B \rangle  \) и дифференцируема на \( (A, B) \).\end{icloze} Тогда
    \begin{itemize}
        \item {}\begin{icloze}{2}\( f \!\nearrow \) на \( \langle A, B \rangle  \)\end{icloze}
            \begin{icloze}{3}\( \iff  \)\end{icloze}
            \begin{icloze}{1}\( f'(x) \geqslant 0 \quad \forall x \in (A, B) \).\end{icloze}
    \end{itemize}
\end{note}

\begin{note}{eb69e8bd92104c0ab3b235de95941521}
    Каков основной шаг в доказательстве критерия возрастания функции на промежутке (необходимость)?

    \begin{cloze}{1}
        Показать, что произвольное разностное отношение неотрицательно.
    \end{cloze}
\end{note}

\begin{note}{7d9850f850c2465aa217f34c4dbd1a66}
    Каков основной шаг в доказательстве критерия возрастания функции на промежутке (достаточность)?

    \begin{cloze}{1}
        Выразить для \( a < b \) разность \( f(b) - f(a) \) через формулу конечных приращений.
    \end{cloze}
\end{note}

\begin{note}{63e919dff3ba4ea282cb06d25b445300}
    Пусть \begin{icloze}{4}\( f \in C\langle A, B \rangle  \) и дифференцируема на \( (A, B) \).\end{icloze} Тогда
    \begin{itemize}
        \item {}\begin{icloze}{2}\( f \!\nearrow\!\!\nearrow \) на \( \langle A, B \rangle  \)\end{icloze}
            \begin{icloze}{3}\( \impliedby  \)\end{icloze}
            \begin{icloze}{1}\( f'(x) > 0 \quad \forall x \in (A, B) \).\end{icloze}
    \end{itemize}
\end{note}

\begin{note}{0e1b8bb37eca4c29af2ca084fcedc196}
    Каков основной шаг в доказательстве достаточного условия строгого возрастания функции на промежутке?

    \begin{cloze}{1}
        Выразить для \( a < b \) разность \( f(b) - f(a) \) через формулу конечных приращений.
    \end{cloze}
\end{note}

\begin{note}{2e3edf0757ba4f72bbdbb5b66dca690d}
    Пусть \begin{icloze}{4}\( f \in C\langle A, B \rangle  \) и дифференцируема на \( (A, B) \).\end{icloze} Тогда
    \begin{itemize}
        \item {}\begin{icloze}{2}\( f \) постоянна на \( \langle A, B \rangle  \)\end{icloze}
            \begin{icloze}{3}\( \iff  \)\end{icloze}
            \begin{icloze}{1}\( f'(x) = 0 \quad \forall x \in (A, B) \).\end{icloze}
    \end{itemize}
\end{note}

\begin{note}{b036d705ddbe49b6814f53a6ad2b93f9}
    Каков основной шаг в доказательстве критерия постоянства функции на промежутке (достаточность)?

    \begin{cloze}{1}
        Выразить для произвольных \( a \) и \( b \) разность \( f(b) - f(a) \) через формулу конечных приращений.
    \end{cloze}
\end{note}

\begin{note}{2dfd421d331745a0a8b2da63493d1b4f}
    Пусть \begin{icloze}{3}\( f, g \in C[A, B\rangle \) и дифференцируемы на \( (A,  B) \).\end{icloze} Тогда
    Если \begin{icloze}{2}\( f(A) = g(A) \) и
        \[
            f'(x) > g'(x) \quad \forall x \in (A, B),
        \]
    \end{icloze}    то
    \begin{icloze}{1}\[
        f(x) > g(x) \quad \forall x \in (A, B\rangle.
    \]\end{icloze}
\end{note}

\begin{note}{e2c4b9fb4f4147a3bf25e2ab97a3e24f}
    Пусть \begin{icloze}{3}\( f, g \in C \langle A, B] \) и дифференцируемы на \( (A,  B) \).\end{icloze}
    Тогда если \begin{icloze}{2}\( f(B) = g(B) \) и
        \[
            f'(x) < g'(x) \quad \forall x \in (A, B),
        \]
    \end{icloze}    то
    \begin{icloze}{1}\[
        f(x) > g(x) \quad \forall x \in \langle A, B).
    \]\end{icloze}
\end{note}

\begin{note}{0f2a5e13f0a2495388e631ac0b4776aa}
    Пусть \( f : D \subset \mathbb R \to \mathbb R, a \in D \). Тогда точка \( a \) называется \begin{icloze}{2}точкой максимума функции \( f \),\end{icloze} если
    \begin{icloze}{1}\[
                         \exists \delta > 0 \quad \forall x \in \dot V_{\delta} (a) \cap D \quad f(x) \leqslant f(a).
                     \]\end{icloze}
\end{note}

\begin{note}{a89063cdc4a34df7aa891ad50a98d0a8}
    Пусть \( f : D \subset \mathbb R \to \mathbb R, a \in D \). Тогда точка \( a \) называется \begin{icloze}{2}точкой строгого максимума функции \( f \),\end{icloze} если
    \begin{icloze}{1}\[
        \exists \delta > 0 \quad \forall x \in \dot V_{\delta} (a) \cap D \quad f(x) < f(a).
    \]\end{icloze}
\end{note}

\begin{note}{0c2db077ea274453a5c14d982fe1c571}
    Пусть \( f : D \subset \mathbb R \to \mathbb R, a \in D \). Тогда точка \( a \) называется \begin{icloze}{2}точкой минимума функции \( f \),\end{icloze} если
    \begin{icloze}{1}\[
        \exists \delta > 0 \quad \forall x \in \dot V_{\delta} (a) \cap D \quad f(x) \geqslant f(a).
    \]\end{icloze}
\end{note}

\begin{note}{3bc6223309d34118a582302414c9632e}
    Пусть \( f : D \subset \mathbb R \to \mathbb R, a \in D \). Тогда точка \( a \) называется \begin{icloze}{2}точкой строгого минимума функции \( f \),\end{icloze} если
    \begin{icloze}{1}\[
        \exists \delta > 0 \quad \forall x \in \dot V_{\delta} (a) \cap D \quad f(x) > f(a).
    \]\end{icloze}
\end{note}

\begin{note}{a1e964e24fc6456ca0a297c008405c34}
    Если \begin{icloze}{2}точка \( a \) является точкой минимума или максимума функции \( f \),\end{icloze} то \( a \) называется \begin{icloze}{1}точкой экстремума \( f \).\end{icloze}
\end{note}

\begin{note}{98f3cebf02ca464ab3cf9e94355caaa2}
    \subsubsection{<<\begin{icloze}{3}Необходимое условие экстремума\end{icloze}>>}

    Пусть \begin{icloze}{2}\( f : \langle A, B \rangle \to \mathbb R, a \in (A, B) \), \( f \) дифференцируема в точке \( a \).\end{icloze}
    Тогда \begin{icloze}{1}если \( a \) является точкой экстремума \( f \), то \( f'(a) = 0 \). \end{icloze}
\end{note}

\begin{note}{acfe3357868e41809070b12ea6034081}
    Каков основной шаг в доказательстве необходимого условия экстремума?

    \begin{cloze}{1}
        Применить теорему Ферма к сужению \({ f }\) на отрезок, включённый в окрестность \({ V_\delta(a) }\) из определения экстремума.
    \end{cloze}
\end{note}

\begin{note}{96502706cad4449ab9ac44074765a384}
    Точка \( a \) называется \begin{icloze}{1}стационарной точкой функции \( f \),\end{icloze} если
    \begin{icloze}{2}\[
        f'(a) = 0.
    \]\end{icloze}
\end{note}

\begin{note}{99ca6c71ff484416941c4e10086ca6ea}
    Пусть \( f : \langle A, B \rangle \to \mathbb R \). Тогда
    \begin{icloze}{1}точка \( a \in (A, B) \)\end{icloze} называется \begin{icloze}{2}критической
    точкой,\end{icloze} если \begin{icloze}{1}либо \( a \) стационарна для \( f \), либо \(
    f \) не дифференцируема в точке \( a \).\end{icloze}
\end{note}

\begin{note}{40f1ebf761e14f5ba885b2276d64dae7}
    Пусть \( f : \langle A, B \rangle \to \mathbb R  \). Тогда все
    \begin{icloze}{2}точки экстремума \( f \), принадлежащие \( (A, B) \),\end{icloze}
    лежат в \begin{icloze}{1}множестве её критических точек.\end{icloze}
\end{note}

\begin{note}{e8adcc7d8b474840907e72b38014fcdc}
    Пусть \( f \in C[a, b] \). Тогда
    \[
        \begin{icloze}{3}\max f([a, b])\end{icloze} = \begin{icloze}{1}\max \left\{ f(a), f(b), \max f(C) \right\},\end{icloze}
    \]
    где \( C \) --- \begin{icloze}{2}множество критических точек \( f \).\end{icloze}
\end{note}

\begin{note}{909932c22cec4a5fb5d8cfb506e7dbfb}
    Пусть \( f : \langle A, B \rangle \to \mathbb R \), \begin{icloze}{3}\( a \in (A, B)  \), \( f \) непрерывна в точке \( a \) и дифференцируема на \( \dot V_{\delta} (a) \), \( \delta > 0 \).\end{icloze}
    Если
    \begin{icloze}{1}\[
        \operatorname{sgn} f'(x) = \operatorname{sgn} (a - x) \quad \forall x \in \dot V_{\delta} (a),
    \]\end{icloze}
    то \begin{icloze}{2}\( a \) --- точка строго максимума \( f \).\end{icloze}
\end{note}

\begin{note}{1b1674e5941040ee87e83073a1a0d57b}
    Пусть \( f : \langle A, B \rangle \to \mathbb R \), \( a \in (A, B)  \), \( f \) непрерывна в точке \( a \) и дифференцируема на \( \dot V_{\delta} (a) \), \( \delta > 0 \).
    Если
    \begin{icloze}{1}\[
        \operatorname{sgn} f'(x) = \operatorname{sgn} (x - a) \quad \forall x \in \dot V_{\delta} (a),
    \]\end{icloze}
    то \begin{icloze}{2}\( a \) --- точка строго минимума \( f \).\end{icloze}
\end{note}

\section{Лекция 21.02.22}
\begin{note}{4d119e495cf043019ed8ee01f9a7957a}
    Пусть \begin{icloze}{3}\( f : \langle A, B \rangle \to \mathbb R, a \in (A, B) \), \( f'' \) определена в точке \( a \), \( f'(a) = 0 \).\end{icloze}
    Тогда если \begin{icloze}{1}\( f''(a) > 0 \),\end{icloze} то \begin{icloze}{2}\( a \) --- точка строгого минимума \( f \).\end{icloze}
\end{note}

\begin{note}{f8b71055f7eb427f8226b47df9ed1e05}
    Пусть \( f : \langle A, B \rangle \to \mathbb R, a \in (A, B) \), \( f'' \) определена в точке \( a \), \( f'(a) = 0 \).
    Тогда если \begin{icloze}{1}\( f''(a) < 0 \),\end{icloze} то \begin{icloze}{2}\( a \) --- точка строгого максимума \( f \).\end{icloze}
\end{note}

\begin{note}{5e0ea19ce2b043c693e2cbc7752fcaf1}
    Каков первый шаг в доказательстве достаточного условия экстремума в терминах \( f'' \)?

    \begin{cloze}{1}
        Выразить \( f(x) - f(a) \)  по формуле Тейлора-Пиано с
        \[
            o((x - a)^2 ).
        \]
    \end{cloze}
\end{note}

\begin{note}{3124302c512c44bfac961f48e231e1cc}
    В чем основная идея доказательства достаточного условия экстремума в терминах \( f'' \)?

    \begin{cloze}{1}
        Вынести в формуле Тейлора-Пиано \( \frac{f''(a)}{2} (x - a)^2  \) за скобки, далее по теореме о стабилизации функции.
    \end{cloze}
\end{note}

\begin{note}{bb068aa42bfe43deb084eaa739cd08c6}
    Пусть \begin{icloze}{4}\( f : \langle A, B \rangle \to \mathbb R, a \in (A, B) \),\end{icloze}
    \begin{icloze}{3}\[
                         \begin{gathered}
                             f'(a) = f''(a) = \cdots = f^{(n - 1)}(a) = 0, \\
                             f^{(n)}(a) \neq 0.
                         \end{gathered}
                     \]\end{icloze}
    Тогда если \begin{icloze}{2}\( n \) нечётно,\end{icloze} то \begin{icloze}{1}\( f \) не имеет экстремума в точке \( a \).\end{icloze}
\end{note}

\begin{note}{b8ec49e21174443588a98b2e5c8cc032}
    Пусть \( f : \langle A, B \rangle \to \mathbb R, a \in (A, B) \),
    \[
        \begin{gathered}
            f'(a) = f''(a) = \cdots = f^{(n - 1)}(a) = 0, \\
            f^{(n)}(a) \neq 0.
        \end{gathered}
    \]
    Тогда если \begin{icloze}{2}\( n \) чётно,\end{icloze} \begin{icloze}{1}то достаточное условие аналогично достаточному условию в терминах \( f'' \).\end{icloze}
\end{note}

\begin{note}{d2426d6723fd4c20966bd4397dce3eb3}
    \subsubsection{<<\begin{icloze}{3}Теорема Дарбу\end{icloze}>>}

    Пусть \begin{icloze}{2}\( f \) дифференцируема на \( \langle A, B \rangle  \), \( a, b \in \langle A, B \rangle \),
    \[
        f'(a) < 0, \quad f'(b) > 0.
    \]\end{icloze}
    Тогда \begin{icloze}{1}\( \exists c \in (a, b) \quad f'(c) = 0 \).\end{icloze}
\end{note}

\begin{note}{43152412fd6f41e984fc4a4e96521633}
    В чем основная идея доказательства теоремы Дарбу?

    \begin{cloze}{1}
        По теореме Вейерштрасса существует точка минимума \( c \), далее по теореме Ферма.
    \end{cloze}
\end{note}

\begin{note}{b0b7d5c649bf4839bde1e90102df6405}
    Что позволяет применить теорему Ферма в доказательстве теоремы Дарбу?

    \begin{cloze}{1}
        \( c \) --- внутренняя точка отрезка \( [a, b] \).
    \end{cloze}
\end{note}

\begin{note}{d480b573cf054a67a6bf5596881b0afb}
    Как в доказательстве теорему Дарбу показать, что \( c \) не лежит на границе \( [a, b] \)?

    \begin{cloze}{1}
        Расписать \( f'(a) \) через правосторонний предел и показать, что \( a \) --- не локальный минимум. Аналогично для \( b \).
    \end{cloze}
\end{note}

\begin{note}{bc1402d472ba422ea18b051e2a0615c4}
    Пусть \begin{icloze}{3}\( f \) дифференцируема на \( \langle A, B \rangle  \).\end{icloze} Если
    \begin{icloze}{2}\[
        f'(x) \neq 0 \quad \forall x \in \langle A, B \rangle,
    \]\end{icloze}
    то \begin{icloze}{1}\( f \) строго монотонна на \( \langle A, B \rangle  \).\end{icloze}
\end{note}

\begin{note}{e29cdd0f22c346cab64fe288db3fbdb8}
    В чем основная идея доказательства следствия о монотонности функции с ненулевой производной?

    \begin{cloze}{1}
        Доказать от противного, что \( f' \) не меняет знак на \( \langle A, B \rangle  \).
        Далее по достаточному условию строгой монотонности.
    \end{cloze}
\end{note}

\begin{note}{9fc77ac828a342f885c48ee472c09734}
    \subsubsection{<<\begin{icloze}{2}Следствие из теоремы Дарбу \\
    \phantom{<<} \quad о сохранении промежутка.\end{icloze}>>}

    \begin{icloze}{1}Пусть \( f \) дифференцируема на \( \langle A, B \rangle  \). Тогда \( f'(\langle A, B \rangle ) \) --- промежуток.\end{icloze}
\end{note}

\begin{note}{56d20a83493a46d1ac834fec9f4ebdef}
    В чем основная идея доказательства следствия из теоремы Дарбу о сохранении промежутка?

    \begin{cloze}{1}
        Показать, что для любых \( a, b \in \langle A, B \rangle  \)
        \[
            [f'(a), f'(b)] \subset f'(\langle A, B \rangle ).
        \]
    \end{cloze}
\end{note}

\begin{note}{0cd99b9f1fae4d1aadfac35788f440c6}
    Какое упрощение принимается (для определённости) для точек \( a, b \in \langle A, B \rangle  \) в доказательстве следствия из теоремы Дарбу о сохранении промежутка?

    \begin{cloze}{1}
        \[
            f'(a) \leqslant f'(b).
        \]
    \end{cloze}
\end{note}

\begin{note}{9ee92cbcb63b46e78fe63b31bbf7f924}
    Как в доказательстве следствия из теоремы Дарбу о сохранении промежутка показать, что
    \[
        \forall y \in (f'(a), f'(b)) \quad y \in f(\langle A, B \rangle )?
    \]

    \begin{cloze}{1}
        Применить теорему Дарбу к функции
        \[
            F(x) = f(x) - y \cdot x
        \]
        в точках \( a \) и \( b \).
    \end{cloze}
\end{note}

\begin{note}{3c1144d31e264164b099479d41f9abe3}
    \subsubsection{<<\begin{icloze}{2}Следствие из теоремы Дарбу \\
    \phantom{<<} \quad о скачках производной.\end{icloze}>>}

    \begin{icloze}{1}Пусть \( f \) дифференцируема на \( \langle A, B \rangle  \). Тогда функция \( f' \) не имеет скачков на \( \langle A, B \rangle  \).\end{icloze}
\end{note}

\begin{note}{f94b4bdf90b14fa0a4256a492cf742a5}
    В чем основная идея доказательства следствия из теоремы Дарбу о скачках производной?

    \begin{cloze}{1}
        Допустить противное и показать, что \({ \operatorname{im} f'|_{[a, a + \delta)} }\) --- не промежуток.
    \end{cloze}
\end{note}

\begin{note}{933fb7290ce844da8f84c48835915d5c}
    Какие допущения принимаются (для определённости) в доказательстве следствия из теоремы Дарбу о скачках производной?

    \begin{cloze}{1}
        \({ f }\) имеет скачёк справа в точке \({ a \in \langle A, B \rangle }\) и
        \[
            L := \lim_{x \to a^{+}} f'(x) < f'(a).
        \]
    \end{cloze}
\end{note}

\begin{note}{4fc5c84bb2b14241a99633260e7f76fc}
    Как выбирается \({ \delta }\) в доказательстве следствия из теоремы Дарбу о скачках производной?

    \begin{cloze}{1}
        Так, что для некоторого \({ y \in (L, f'(a)) }\)
        \[
            f'(x) < y \quad \forall x \in (a, a + \delta).
        \]
    \end{cloze}
\end{note}

\begin{note}{027449ca442a449786b58ca872e4aff2}
    \begin{icloze}{3}Функция \( f : \langle A, B \rangle \to \mathbb R  \)\end{icloze} называется \begin{icloze}{2}выпуклой на \( \langle A, B \rangle  \),\end{icloze} если
    \begin{icloze}{1}\[
        \begin{gathered}
            \forall a, b \in \langle A, B \rangle, \lambda \in (0,1) \\
            f(\lambda a + (1 - \lambda)b) \leqslant \lambda f(a) + (1 - \lambda)f(b).
        \end{gathered}
    \]\end{icloze}
\end{note}

\begin{note}{0073407c9c4f473cb4759784548208bd}
    \begin{icloze}{3}Функция \( f : \langle A, B \rangle \to \mathbb R  \)\end{icloze} называется \begin{icloze}{2}строго выпуклой на \( \langle A, B \rangle  \),\end{icloze} если
    \begin{icloze}{1}\[
        \begin{gathered}
            \forall a, b \in \langle A, B \rangle, \lambda \in (0,1) \\
            f(\lambda a + (1 - \lambda)b) < \lambda f(a) + (1 - \lambda)f(b).
        \end{gathered}
    \]\end{icloze}
\end{note}

\begin{note}{a0e64a51b1ac405c9e5806d135c272da}
    \subsubsection{<<\begin{icloze}{3}Критерий\end{icloze} \begin{icloze}{1}строгой выпуклости \( f \) на \( \langle A, B \rangle  \)\end{icloze}>>}

    Пусть \( f : \langle A, B \rangle \to \mathbb R  \). Тогда \begin{icloze}{3}равносильны\end{icloze} следующие утверждения.
    \begin{itemize}
        \item {}\begin{icloze}{1}\( f \) строго выпукла на \( \langle A, B \rangle  \).\end{icloze}
        \item {}\begin{icloze}{2}\( \forall a, b, c \in \langle A, B \rangle, a < c < b \) справедливо неравенство
                \[
                    \frac{f(c) - f(a)}{c - a} < \frac{f(b) - f(c)}{b - c}.
                \]
            \end{icloze}
    \end{itemize}

    \begin{center}
        \tiny (из леммы о трёх хордах)
    \end{center}
\end{note}

\begin{note}{8969555424f24b2c8347358586f381e8}
    \subsubsection{<<\begin{icloze}{4}Лемма о трёх хордах\end{icloze}>>}

    Пусть \( f : \langle A, B \rangle \to \mathbb R  \). Тогда \begin{icloze}{3}равносильны\end{icloze} следующие утверждения.
    \begin{itemize}
        \item {}\begin{icloze}{1}\( f \) строго выпукла на \( \langle A, B \rangle  \).\end{icloze}
        \item {}\begin{icloze}{2}\( \forall a, b, c \in \langle A, B \rangle, a < c < b \) справедливы неравенства
                \[
                    \frac{f(c) - f(a)}{c - a} < \frac{f(b) - f(a)}{b - a}  < \frac{f(b) - f(c)}{b - c}.
                \]
            \end{icloze}
    \end{itemize}
\end{note}

\begin{note}{c8e0bc0478654a5bbabdf38890b5f942}
    Каким образом доказываются критерий строгой выпуклости и лемма о трёх хордах?

    \begin{cloze}{1}
        Строится цепочка импликаций
        \[
            (1) \implies (2) \implies (3) \implies (1).
        \]
        \begin{itemize}
            \item \({ (1) }\) --- строгая выпуклость \({ f }\),
            \item \({ (2) }\) --- неравенство из леммы о трёх хордах.
            \item \({ (3) }\) --- неравенство из критерия выпуклости,
        \end{itemize}
    \end{cloze}
\end{note}

\begin{note}{11c8d7341f374a2e8d7d5c8df97eaba8}
    В чём основная идея доказательства критерия строгой выпуклости из леммы о трёх хордах (достаточность)?

    \begin{cloze}{1}
        Положить \({ c = \lambda a + (1 - \lambda) b }\) и отсюда выразить \({ c - a }\) и \({ b - c }\).
    \end{cloze}
\end{note}

\begin{note}{1338324da4ea4cb7a93a5a15718852b5}
    В чем основная идея доказательства леммы о трёх хордах (необходимость)?

    \begin{cloze}{1}
        Положить в определении выпуклости
        \[
            \lambda = \frac{b - c}{b - a}.
        \]
    \end{cloze}
\end{note}

\section{Лекция 25.02.22}
\begin{note}{0abcc31a29c74496883c555de61b5af7}
    Пусть \( f : \langle A, B \rangle \to \mathbb R \), \( a \in \begin{icloze}{3}\langle A, B \rangle\end{icloze} \),
    \[
        F(x) \coloneqq \begin{icloze}{4}\frac{f(x) - f(a)}{x - a}.\end{icloze}
    \]
    Тогда если \begin{icloze}{2}\( f \) выпукла на \( \langle A, B \rangle  \),\end{icloze} то
    \begin{icloze}{1}
        \begin{center}
            \( F \!\!\nearrow \) на \( \langle A, B \rangle \setminus \left\{ a \right\} \).
        \end{center}
    \end{icloze}
\end{note}

\begin{note}{6658c8d28bde461584886f85aacf4977}
    Пусть \( f : \langle A, B \rangle \to \mathbb R \), \( a \in \begin{icloze}{3}\langle A, B \rangle\end{icloze} \),
    \[
        F(x) \coloneqq \begin{icloze}{4}\frac{f(x) - f(a)}{x - a}.\end{icloze}
    \]
    Тогда если \begin{icloze}{2}\( f \) строго выпукла на \( \langle A, B \rangle  \),\end{icloze} то
    \begin{icloze}{1}
        \begin{center}
            \( F \!\!\nearrow\!\nearrow  \) на \( \langle A, B \rangle \setminus \left\{ a \right\} \).
        \end{center}
    \end{icloze}
\end{note}

\begin{note}{d547aa237c104089813102cd73487563}
    Пусть  \({ f : \langle A, B \rangle \to \mathbb R }\), \({ a \in \langle A, B \rangle }\), \({ f }\) выпукла на \({ \langle A, B \rangle }\).
    Откуда следует возрастание функции \({ F(x) = \frac{f(x) - f(a)}{x - a} }\)?

    \begin{cloze}{1}
        Из леммы о трёх хордах.
    \end{cloze}
\end{note}

\begin{note}{0bb5876454d448878db0853372d90fe7}
    Пусть \begin{icloze}{3}\( f \) выпукла на \( \langle A, B \rangle  \),\end{icloze} \begin{icloze}{2}\( a \in \langle A, B )  \).\end{icloze} Тогда
    \begin{icloze}{1}
        \[
            \exists f'_+(a) \in [-\infty, +\infty ).
        \]
    \end{icloze}
\end{note}

\begin{note}{960c7add5b8c4ab4b798301f26f12648}
    Пусть \begin{icloze}{3}\( f \) выпукла на \( \langle A, B \rangle  \),\end{icloze} \begin{icloze}{2}\( a \in (A, B \rangle  \).\end{icloze} Тогда
    \begin{icloze}{1}
        \[
            \exists f'_-(a) \in (-\infty, +\infty ].
        \]
    \end{icloze}
\end{note}

\begin{note}{1ea150313caa4817b9f27c00d5c8e6d8}
    Откуда следует существование односторонних производных у выпуклой фукнции?

    \begin{cloze}{1}
        Из теоремы о пределе монотонной функции для
        \[
            F(x) = \frac{f(x) - f(a)}{x - a}.
        \]
    \end{cloze}
\end{note}

\begin{note}{2e664465fdc5410ca8b72059cfe627bc}
    Пусть \begin{icloze}{3}\( f \) выпукла на \( \langle A, B \rangle  \),\end{icloze} \begin{icloze}{2}\( a \in (A, B) \).\end{icloze} Тогда \begin{icloze}{1}\( f'_+ (a) \) и \( f'_- (a) \) конечны и \( f'_- (a) \leqslant f'_+ (a) \).\end{icloze}
\end{note}


\begin{note}{82fe965871ac446facad207a4f246b18}
    Пусть \( f \) выпукла на \( \langle A, B \rangle  \), \( a \in (A, B) \).
    Откуда следует, что \( f'_- (a) \leqslant f'_+ (a) \)?

    \begin{cloze}{1}
        Из теоремы о пределе монотонной функции.
    \end{cloze}
\end{note}

\begin{note}{eb64f07db3d3434197d40b0980a78e66}
    Если функция \( f \) выпукла на \( \langle A, B \rangle  \), то она \begin{icloze}{1}непрерывна на \( (A, B) \).\end{icloze}
\end{note}

\begin{note}{9390116052df401f8413ffb225259a9d}
    Пусть \( f \) выпукла на \( \langle A, B \rangle  \). Откуда следует, что она непрерывна на \({ (A, B) }\)?

    \begin{cloze}{1}
        Из существования конечных односторонних производных \({ f }\) в любой точке \({ (A, B) }\).
    \end{cloze}
\end{note}

\begin{note}{9f16939e7619449e9fe1d75a7aae2e87}
    Пусть \begin{icloze}{3}\( f : \langle A, B \rangle \to \mathbb R  \), \( a \in \langle A, B \rangle  \).\end{icloze}
    \begin{icloze}{2}Прямая \( y = g(x) \)\end{icloze} называется \begin{icloze}{1}опорной для функции в точке \( a \),\end{icloze} если
    \begin{icloze}{2}
        она проходит через точку \( (a, f(a)) \) и
        \[
            \begin{gathered}
                f(x) \geqslant g(x) \quad \forall x \in \langle A, B \rangle.
            \end{gathered}
        \]
    \end{icloze}
\end{note}

\begin{note}{7b835ae738654ba5a0921df5133181e7}
    Пусть \( f : \langle A, B \rangle \to \mathbb R  \), \( a \in \langle A, B \rangle  \).
    \begin{icloze}{2}Прямая \( y = g(x) \)\end{icloze} называется \begin{icloze}{1}строго опорной для функции в точке \( a \),\end{icloze} если
    \begin{icloze}{2}
        она проходит через точку \( (a, f(a)) \) и
        \[
            \begin{gathered}
                f(x) > g(x) \quad \forall x \in \langle A, B \rangle \setminus \left\{ a \right\}.
            \end{gathered}
        \]
    \end{icloze}
\end{note}

\begin{note}{fedf029d618e48ddabe81280b131b72b}
    Пусть \begin{icloze}{5}\( f : \langle A, B \rangle \to \mathbb R  \), \( f \) выпукла на \( \langle A, B \rangle  \), \( a \in (A, B) \),\end{icloze} прямая \( \ell \) задаётся
    \begin{icloze}{4}
        уравнением
        \[
            y = f(a) + k(x - a).
        \]
    \end{icloze}
    Тогда прямая \( \ell \) является \begin{icloze}{1}опорной для функции \( f \)
    в точке \( a \)\end{icloze} \begin{icloze}{3}тогда и только тогда, когда\end{icloze}
    \begin{icloze}{2}
        \( k \in [f'_-(a), f'_+(a)] \).
    \end{icloze}
\end{note}

\begin{note}{8ceccffa4cbe4c8d8330451f4f53876c}
    Пусть \begin{icloze}{4}\( f : \langle A, B \rangle \to \mathbb R  \), \( f \) строго выпукла на \( \langle A, B \rangle  \), \( a \in (A, B) \),\end{icloze} прямая \( \ell \) задаётся уравнением
    \[
        y = f(a) + k(x - a).
    \]
    Тогда прямая \( \ell  \) является \begin{icloze}{1}строго опорной для функции \( f \) в точке \( a \)\end{icloze} \begin{icloze}{3}тогда и только тогда, когда\end{icloze} \begin{icloze}{2}\( k \in [f'_-(a), f'_+(a)]\).\end{icloze}
\end{note}

\begin{note}{1da04fcb23dc406eba98567735e9e6dc}
    Пусть \({ f : \langle A, B \rangle \to \mathbb R }\), \({ f }\) выпукла на \({ \langle A, B \rangle }\), \({ a \in (A, B) }\).
    Как показать, что если прямая \( y = f(a) + k(x - a) \) является опорной для \({ f }\), то \({ k \in [f'_-(a), f'_+(a)] }\)?

    \begin{cloze}{1}
        В определении опорной прямой выразить \({ f(x) }\) через односторонние производные по определению дифференцируемости.
    \end{cloze}
\end{note}

\begin{note}{9ab922ea4b1f422c855c9dc14925580a}
    Пусть \({ f : \langle A, B \rangle \to \mathbb R }\), \({ f }\) выпукла на \({ \langle A, B \rangle }\), \({ a \in (A, B) }\).
    Как показать, что прямая \( y = f(a) + k(x - a) \) является опорной для \({ f }\), если \({ k \in [f'_-(a), f'_+(a)] }\)?

    \begin{cloze}{1}
        По теореме о пределе монотонной функции сравнить
        \[
            x \mapsto \frac{f(x) - f(a)}{x - a}
        \]
        с односторонними производными в точке \({ a }\).
    \end{cloze}
\end{note}

\begin{note}{f8f5608de51344b89b749bf6fb673e89}
    Пусть \({ f : \langle A, B \rangle \to \mathbb R }\).
    Если \begin{icloze}{2}в каждой точке \({ (A, B) }\) функция \({ f }\) имеет опорную прямую,\end{icloze} то \begin{icloze}{1}она выпукла на \({ \langle A, B \rangle }\).\end{icloze}

    \begin{center}
        \tiny
        (в терминах опорных прямых)
    \end{center}
\end{note}

\begin{note}{0a5cbb4429524954af423e27fe0c32bc}
    Пусть \({ f : \langle A, B \rangle \to \mathbb R }\).
    Если \begin{icloze}{2}в каждой точке \({ (A, B) }\) функция \({ f }\) имеет строго опорную прямую,\end{icloze} то \begin{icloze}{1}она строго выпукла на \({ \langle A, B \rangle }\).\end{icloze}

    \begin{center}
        \tiny
        (в терминах опорных прямых)
    \end{center}
\end{note}

\begin{note}{cfe92784a4e244ecb916f9627404f421}
    Пусть \({ f : \langle A, B \rangle \to \mathbb R }\).
    Если в каждой точке \({ a \in (A, B) }\) функция \({ f }\) имеет опорную прямую, то она выпукла на \({ \langle A, B \rangle }\).
    В чем основная идея доказательства?

    \begin{cloze}{1}
        Выбрать \({ x < a < y }\) и показать для них выполнение неравенства критерия выпуклости из леммы о трёх хордах.
    \end{cloze}
\end{note}

\section{Лекция 04.03.22}
\begin{note}{acc9492d0b4f4c4a8e6b1688ee26ed5e}
    В чем геометрический смысл \( T_{a, 1} f(x) \)?

    \begin{cloze}{1}
        График \( T_{a, 1} f(x) \) --- это касательная к функции \( f \) в точке \( a \).
    \end{cloze}
\end{note}

\begin{note}{570272578ee74dd988ea80f9e95cbc6f}
    \subsubsection{<<Связь \begin{icloze}{2}выпуклости функции\end{icloze} с её касательными>>}

    Пусть \begin{icloze}{4}\( f : \langle A, B \rangle \to \mathbb R \), \( f \) дифференцируема на \( (A, B) \).\end{icloze}
    Тогда \begin{icloze}{2}функция \( f \) выпукла на \( \langle A, B \rangle \)\end{icloze} \begin{icloze}{3}тогда и только тогда, когда\end{icloze}
    \begin{icloze}{1}
        \[
            \begin{gathered}
                \forall a \in (A, B), \quad x \in \langle A, B \rangle \\
                f(x) \geqslant T_{a, 1} f(x).
            \end{gathered}
        \]
    \end{icloze}
\end{note}

\begin{note}{32700c2a93204435b3f66db20ea03bf7}
    \subsubsection{<<Связь \begin{icloze}{2}выпуклости функции\end{icloze} с её касательными>>}

    Пусть \begin{icloze}{4}\( f : \langle A, B \rangle \to \mathbb R \), \( f \) дифференцируема на \( (A, B) \).\end{icloze}
    Тогда \begin{icloze}{2}функция \( f \) строго выпукла на \( \langle A, B \rangle \)\end{icloze} \begin{icloze}{3}тогда и только тогда, когда\end{icloze}
    \begin{icloze}{1}
        \[
            \begin{gathered}
                \forall a \in (A, B), x \in \langle A, B \rangle \setminus \left\{ a \right\} \\
                f(x) > T_{a, 1} f(x).
            \end{gathered}
        \]
    \end{icloze}
\end{note}

\begin{note}{76ff105d143e49dea8fe8db2b74ee9ff}
    В чем основная идея доказательства теоремы о связи выпуклости функции с её касательными (необходимость)?

    \begin{cloze}{1}
        \( f \) дифференцируема в любой точке \( (A, B) \) \( \implies \) касательная совпадает с опорной прямой.
    \end{cloze}
\end{note}

\begin{note}{a71b501cf8434876a0ff83fdc763c8d3}
    В чем основная идея доказательства теоремы о связи выпуклости функции с её касательными (достаточность)?

    \begin{cloze}{1}
        Из условия \({ f }\) имеет опорную прямую в каждой точке \({ (A, B) }\).
    \end{cloze}
\end{note}

\begin{note}{3b6d6467bd5144febe2b52fd934c971a}
    Пусть \begin{icloze}{3}\( f : (A, +\infty) \to \mathbb R \) имеет при \( x \to +\infty \) асимптоту \( y = kx + b \).\end{icloze}
    Тогда если \begin{icloze}{2}\( f \) выпукла на \( (A, +\infty) \),\end{icloze} то
    \begin{icloze}{1}
        \[
            f(x) \geqslant kx + b \quad \forall x \in (A, +\infty).
        \]
    \end{icloze}
\end{note}

\begin{note}{e766cccf8cdf4765b58203bef6244390}
    Пусть \begin{icloze}{3}\( f : (A, +\infty) \to \mathbb R \) имеет при \( x \to +\infty \) асимптоту \( y = kx + b \).\end{icloze}
    Тогда если \begin{icloze}{2}\( f \) строго выпукла на \( (A, +\infty) \),\end{icloze} то
    \begin{icloze}{1}
        \[
            f(x) > kx + b \quad \forall x \in (A, +\infty).
        \]
    \end{icloze}
\end{note}

\begin{note}{7046fd62e87e44c7a6dc18f4e94f7bd8}
    Пусть \( f : (A, +\infty) \to \mathbb R \) имеет при \( x \to +\infty \) асимптоту \( y = kx + b \).
    Тогда если \( f \) строго выпукла на \( (A, +\infty) \), то
    \[
        f(x) > kx + b \quad \forall x \in (A, +\infty).
    \]
    В чем основная идея доказательства?

    \begin{cloze}{1}
        Показать, что \({ f(x) - kx \ \!\!\searrow\!\searrow }\). Далее по теореме о пределе монотонной функции.
    \end{cloze}
\end{note}

\begin{note}{f0e5b2b8f6a74445a42cf0b35e854f39}
    Пусть \( f : (A, +\infty) \to \mathbb R \) имеет при \( x \to +\infty \) асимптоту \( y = kx + b \) и \( f \) строго выпукла на \( (A, +\infty) \).
    Как показать, что \({ f(x) - kx \ \!\!\searrow\!\searrow }\)?

    \begin{cloze}{1}
        По теореме о пределе монотонной функции для
        \[
            t \mapsto \frac{f(t) - f(x)}{t - x}.
        \]
    \end{cloze}
\end{note}

\begin{note}{94e7cdb6145142c3bb7cc8115035e5ad}
    \subsubsection{<<Связь \begin{icloze}{2}выпуклости функции\end{icloze} с \( f' \)>>}

    Пусть \begin{icloze}{4}\( f \in C\langle A, B \rangle \), \( f \) дифференцируема на \( (A, B) \).\end{icloze}
    Тогда \begin{icloze}{2}\( f \) выпукла на \( \langle A, B \rangle \)\end{icloze} \begin{icloze}{3}тогда и только тогда, когда\end{icloze}
    \begin{icloze}{1}
        \begin{center}
            \( f' \!\!\nearrow \) на \( (A, B) \).
        \end{center}
    \end{icloze}
\end{note}

\begin{note}{cfdb1a58f41247169b530e3bc3f5b061}
    \subsubsection{<<Связь \begin{icloze}{2}выпуклости функции\end{icloze} с \( f' \)>>}
    Пусть \begin{icloze}{4}\( f \in C\langle A, B \rangle \), \( f \) дифференцируема на \( (A, B) \).\end{icloze}
    Тогда \begin{icloze}{2}\( f \) строго выпукла на \( \langle A, B \rangle \)\end{icloze} \begin{icloze}{3}тогда и только тогда, когда\end{icloze}
    \begin{icloze}{1}
        \begin{center}
            \( f' \!\!\nearrow\!\nearrow \) на \( (A, B) \).
        \end{center}
    \end{icloze}
\end{note}

\begin{note}{8b55ad03aaca4dfcb1ec7ce171dee0ce}
    В чем основная идея доказательства теоремы о связи выпуклости функции с \({ f' }\) (необходимость)?

    \begin{cloze}{1}
        Для \({ x < y }\) сравнить значения \({ f'(x), f'(y) }\) с \({ \frac{f(y) - f(x)}{y - x} }\).
    \end{cloze}
\end{note}

\begin{note}{b1782e215a3d4a948b9fbddfbaed55d3}
    В чем основная идея доказательства теоремы о связи выпуклости функции с \({ f' }\) (достаточность)?

    \begin{cloze}{1}
        По теореме Лагранжа выполняется неравенство критерия выпуклости из леммы о трёх хордах.
    \end{cloze}
\end{note}

\begin{note}{1db6c044058c49e68328ad272c648da8}
    \subsubsection{<<Связь \begin{icloze}{2}выпуклости функции\end{icloze} с \( f'' \)>>}
    Пусть \begin{icloze}{4}\( f \in C\langle A, B \rangle \), \( f \) дважды дифференцируема на \( (A, B) \).\end{icloze}
    Тогда \begin{icloze}{2}\( f \) выпукла на \( \langle A, B \rangle \)\end{icloze} \begin{icloze}{3}тогда и только тогда, когда\end{icloze}
    \begin{icloze}{1}
        \[
            f''(x) \geqslant 0 \quad \forall x \in (A, B).
        \]
    \end{icloze}
\end{note}

\begin{note}{d78c1dfaebde4a2e89fdccfb43309163}
    \subsubsection{<<Связь \begin{icloze}{2}выпуклости функции\end{icloze} с \( f'' \)>>}
    Пусть \begin{icloze}{4}\( f \in C\langle A, B \rangle \), \( f \) дважды дифференцируема на \( (A, B) \).\end{icloze}
    Тогда \begin{icloze}{2}\( f \) строго выпукла на \( \langle A, B \rangle \),\end{icloze} \begin{icloze}{3}если\end{icloze}
    \begin{icloze}{1}
        \[
            f''(x) > 0 \quad \forall x \in (A, B).
        \]
    \end{icloze}
\end{note}

\begin{note}{d912e4ab9b6a4459b2f104fabfc198f8}
    В чем основная идея доказательства теоремы о связи выпуклости функции с \({ f'' }\)?

    \begin{cloze}{1}
        Применить критерий возрастания функции к \({ f' }\).
    \end{cloze}
\end{note}

\begin{note}{399c82ffb7094f2e8e4a74da8023fc60}
    Пусть \begin{icloze}{3}\( f :  \langle A, B \rangle \to \mathbb R, a \in (A, B) \).\end{icloze}
    Точка \( a \) называется \begin{icloze}{2}точкой перегиба функции \( f \),\end{icloze} если
    \begin{icloze}{1}
        \begin{itemize}
            \item \( \exists \delta > 0 \) такое, что \( V_\delta (a) \subset (A, B) \) и \( f \) имеет разный характер выпуклости на \( (a - \delta, a] \) и \( [a, a + \delta) \);
            \item \( f \) непрерывна в точке \( a \);
            \item \( \exists f'(a) \in \overline{\mathbb R} \).
        \end{itemize}
    \end{icloze}
\end{note}

\begin{note}{9aa5847a39ac46e8ad8dbee41e14a904}
    Пусть \( f : \langle A, B \rangle \to \mathbb R, a \in (A, B) \), \( f \) дважды дифференцируема на \( a \).
    Если \begin{icloze}{2}\( a \) является точкой перегиба \( f \),\end{icloze} то \begin{icloze}{1}\( f''(a) = 0 \).\end{icloze}
\end{note}

\begin{note}{aca76c8bcbef4e38ad13dd619d48d19d}
    Является ли нулевая вторая производная достаточным условием перегиба?

    \begin{cloze}{1}
        Нет, это только необходимое условие.
    \end{cloze}
\end{note}

\begin{note}{c3615f4ec8d84748bde8c518c9e98375}
    Пусть \begin{icloze}{3}\( f : \langle A, B \rangle \to \mathbb R, a \in (A, B) \), \( f \) непрерывна в точке \( a \) и имеет в ней производную из \( \overline{\mathbb R} \).\end{icloze}
    Тогда если \begin{icloze}{1}\( \exists \delta > 0 \) такое, что \( f \) дважды дифференцируема на \( \dot V_{\delta}(a) \) и
    \begin{itemize}
        \item либо \quad \( \operatorname{sgn} f''(x) = \operatorname{sgn} (a - x) \quad \forall x \in \dot V_{\delta}(a), \)
        \item либо \quad \( \operatorname{sgn} f''(x) = \operatorname{sgn} (x - a) \quad \forall x \in \dot V_{\delta}(a), \)
    \end{itemize}\end{icloze}
    то \begin{icloze}{2}\( a \) --- точка перегиба \( f \).\end{icloze}
\end{note}

\section{Семинар 03.03.22}
\begin{note}{655ebf6da8c1489f84fdaeea82dcc793}
    \[
        \int \begin{icloze}{2}\ln x\end{icloze}\: dx = \begin{icloze}{1}x \ln x - x\end{icloze} + C
    \]
\end{note}

\begin{note}{310668af95114f9fbe87673be333fec8}
    \[
        \int \begin{icloze}{2}\frac{1}{\sin x}\end{icloze}\: dx = \begin{icloze}{1}\ln\left| \tan \frac{x}{2} \right|\end{icloze} + C
    \]
\end{note}

\begin{note}{898276fe3ef943c49921748d594000c8}
    \[
        \int \begin{icloze}{2}\frac{1}{\cos x}\end{icloze}\: dx = \begin{icloze}{1}\ln\left| \frac{1 + \tan \frac{x}{2}}{1 - \tan \frac{x}{2}} \right|\end{icloze} + C
    \]
\end{note}

\begin{note}{ce3022e62a4f4a6ea2d13195a9f94d31}
    \[
        \int \begin{icloze}{2}\frac{1}{x^2 + a^2}\end{icloze}\: dx
        = \begin{icloze}{1}\frac{1}{a} \arctan \frac{x}{a}\end{icloze} + C
        \quad (\begin{icloze}{3}a \neq 0\end{icloze})
    \]
\end{note}

\begin{note}{8661888336db411a89fed337ad926a76}
    \[
        \int \begin{icloze}{2}\frac{A}{x + a}\end{icloze}\: dx
        = \begin{icloze}{1}A \ln\left| x + a \right|\end{icloze} + C
    \]
\end{note}

\begin{note}{2cd6c699811f4760be34715a24b0081f}
    \[
        \int \begin{icloze}{2}\frac{1}{nx + a}\end{icloze}\: dx
        = \begin{icloze}{1}\frac{1}{n} \ln\left| x + \frac{a}{n} \right|\end{icloze} + C
        \quad (\begin{icloze}{3}n \neq 0\end{icloze})
    \]
\end{note}

\begin{note}{b7b778e748574ee8b52225ae5669cbe6}
    \[
        \int \begin{icloze}{2}\frac{A}{(x + a)^{k}}\end{icloze}\: dx
        = \begin{icloze}{1}\frac{A}{(1 - k)(x + a)^{k - 1}}\end{icloze} + C
        \quad (\begin{icloze}{3}k \neq 1\end{icloze})
    \]
\end{note}

\begin{note}{72b0aaea0b254078bbfcc47745885653}
    \begin{multline*}
        \int \begin{icloze}{4}\frac{Mx + N}{x^2 + px + q}\end{icloze}\: dx = \\
        \begin{icloze}{1}\frac{M}{2} \ln\left| x^2 + px + q \right|\end{icloze}
        + \begin{icloze}{2}\frac{2N - pM}{2a} \arctan \frac{2x + p}{2a}\end{icloze} \\
        + C,
    \end{multline*}
    где \( \displaystyle a^2 := \begin{icloze}{3}\frac{4q - p^2}{4}\end{icloze} \), \quad \begin{icloze}{5}\({ p^2 - 4q < 0 }\).\end{icloze}
\end{note}

\begin{note}{c7fcc3d1ab9443d2855e310bfb0beee8}
    \begin{multline*}
        \int \begin{icloze}{3}\frac{Mx + N}{\left( x^2 + px + q \right)^{k}}\end{icloze}\: dx = \\
        \int \begin{icloze}{1}\frac{N - M \frac{p}{2}}{\left( t^2 + a^2 \right)^{k}}\end{icloze}\: dt +
        \begin{icloze}{1}\int \frac{Mt}{\left( t^2 + a^2 \right)^{k}}\end{icloze}\: dt + C,
    \end{multline*}
    где \( \displaystyle \quad \begin{icloze}{4}t\end{icloze} := \begin{icloze}{2}x + \frac{p}{2},\end{icloze} \quad \begin{icloze}{4}a^2\end{icloze} := \begin{icloze}{2}\frac{4q - p^2}{4}\end{icloze} \), \quad \begin{icloze}{5}\({ p^2 - 4q < 0 }\).\end{icloze}
\end{note}

\begin{note}{a3d0cc7201b74c4c9fab9590e7a6c0b2}
    \begin{align*}
        I_k &=: \int \begin{icloze}{4}\frac{1}{\left( t^2 + a^2 \right)^{k}}\end{icloze}\: dt \quad (\begin{icloze}{5}k > 1, a \neq 0\end{icloze}) \\
        I_k &= \begin{icloze}{3}\frac{1}{2(k - 1)a^2}\end{icloze} \cdot \left( \begin{icloze}{1}(2k - 3) I_{k - 1}\end{icloze} + \begin{icloze}{2}\frac{t}{(t^2 + a^2)^{k - 1}}\end{icloze} \right)
    \end{align*}
\end{note}

\begin{note}{972b3ecb92a94f62b12e46795945593d}
    \[
        \int \begin{icloze}{2}\frac{Mt}{\left( t^2 + a^2 \right)^{k}}\end{icloze}\: dt = \begin{icloze}{1}\frac{M}{2(1 - k)(t^2 + a^2)^{k - 1}}\end{icloze} + C
    \]
\end{note}

\section{Лекция 07.03.22}
\begin{note}{8d4e84ad6e1a4cdc91020e2f61878f24}
    Пусть \begin{icloze}{3}\( f : \langle A, B \rangle \to \mathbb R \).\end{icloze}
    \begin{icloze}{1}Функция \( F : \langle A, B \rangle \to \mathbb R \)\end{icloze} наызвается \begin{icloze}{2}первообразной функции \( f \),\end{icloze} если
    \begin{icloze}{1}F дифференцируема на \( \langle A, B \rangle \) и
    \[
        F'(x) = f(x) \quad \forall x \in \langle A, B \rangle.
    \]\end{icloze}
\end{note}

\begin{note}{5436ab9b46cf488eb5fa6c2353bd3616}
    \begin{icloze}{1}Множество всех первообразных функции \( f \) на промежутке \( \langle A, B \rangle \)\end{icloze} обозначается \begin{icloze}{2}\( \mathscr P_f(\langle A, B \rangle) \).\end{icloze}
\end{note}

\begin{note}{ec64c5e7734140f888511699374deaec}
    Пусть \begin{icloze}{4}\( f, F, G : \langle A, B \rangle \to \mathbb R \), \( F \in \mathscr P_f(\langle A, B \rangle) \).\end{icloze}
    Тогда
    \[
        \begin{icloze}{2}G \in \mathscr P_f (\langle A, B \rangle) \end{icloze}
        \begin{icloze}{3}\iff\end{icloze}
        \begin{icloze}{1}\exists c \in \mathbb R \quad G(x) = F(x) + c.\end{icloze}
    \]
\end{note}

\begin{note}{e9bbf7b29a8d40b48aad130674b03cc9}
    Пусть \( f, F, G : \langle A, B \rangle \to \mathbb R \), \( F \in \mathscr P_f(\langle A, B \rangle) \).
    Тогда
    \[
        G \in \mathscr P_f (\langle A, B \rangle) \implies
        \exists c \in \mathbb R \quad G(x) = F(x) + c.
    \]

    В чем основная идея доказательства?

    \begin{cloze}{1}
        \( (F - G)' \equiv 0 \implies F - G = const \).
    \end{cloze}
\end{note}

\begin{note}{64bcacf18cb94a4e9b96e551eff15e5b}
    Пусть \( f, F, G : \langle A, B \rangle \to \mathbb R \), \( F \in \mathscr P_f(\langle A, B \rangle) \).
    Тогда
    \[
        G \in \mathscr P_f (\langle A, B \rangle) \impliedby
        \exists c \in \mathbb R \quad G(x) = F(x) + c.
    \]

    В чем основная идея доказательства?

    \begin{cloze}{1}
        Тривиально следует из определения первообразной.
    \end{cloze}
\end{note}

\begin{note}{b196b146568446a2b31a62a77bcddd45}
    Пусть \begin{icloze}{3}\( f : \langle A, B \rangle \to \mathbb R, \quad F \in \mathscr P_f (\langle A, B \rangle) \).\end{icloze}
    \begin{icloze}{1}Множество функций
                     \[
                         \left\{ F(x) + c \mid c \in \mathbb R \right\}
                     \]\end{icloze}
    называется \begin{icloze}{2}неопределённым интегралом \( f \) на \( \langle A, B \rangle \).\end{icloze}
\end{note}

\begin{note}{98516b869bc740b9bacfcc5244a89cb0}
    Пусть \begin{icloze}{3}\( f : \langle A, B \rangle \to \mathbb R \).\end{icloze}
    \begin{icloze}{1}Неопределённый интеграл функции \( f \)  на \( \langle A, B \rangle \)\end{icloze} обозначается
    \begin{icloze}{2}
        \[
            \int f(x)\: dx.
        \]
    \end{icloze}
\end{note}

\begin{note}{7581f732c1c44de4bc99eae39e01f4ea}
    Корректна ли запись
    \[
        \int f(x)\: dx = F(x) + C \quad?
    \]

    \begin{cloze}{1}
        Строго говоря нет, поскольку формально интеграл является множеством, а не функцией, но такая запись удобна на практике.
    \end{cloze}
\end{note}

\begin{note}{ad021cd0f9bd4d9ca316d3574a3b67a4}
    Пусть \( f : \langle A, B \rangle \to \mathbb R \) и \( f \) имеет первообразную на \( \langle A, B \rangle \).
    \[
        \left( \int f(x)\: dx \right)' \overset{\text{def}}= \begin{icloze}{1}f(x).\end{icloze}
    \]
\end{note}

\begin{note}{a2f17fea47484277b1a9d9349fbea7ff}
    Пусть \( f, g : \langle A, B \rangle \to \mathbb R \), \quad \( F \in \mathscr P_f (\langle A, B \rangle), G \in \mathscr P_g (\langle A, B \rangle) \).
    \[
        \int f(x)\: dx + \int g(x)\: dx \overset{\text{def}}= \begin{icloze}{1}\Big\{ F(x) + H(x) + C \mid C \in \mathbb R \Big\}.\end{icloze}
    \]
\end{note}

\begin{note}{7d5f8b97d72747df93959cee3fb0bae9}
    Пусть \( f : \langle A, B \rangle \to \mathbb R \) и \( f \) имеет первообразную на \( \langle A, B \rangle \), \( \lambda \in \mathbb R \).
    \[
        \lambda \int f(x)\: dx \overset{\text{def}}= \begin{icloze}{1}\Big\{ \lambda F(x) + C \mid C \in \mathbb R \Big\}.\end{icloze}
    \]
\end{note}

\begin{note}{3fb6e723afb54981be16c06cf2bfb210}
    Из \begin{icloze}{3}теоремы Дарбу\end{icloze} следует, что
    если \begin{icloze}{2}\( f \) имеет первообразную на промежутке \( \langle A, B \rangle \),\end{icloze}
    то \begin{icloze}{1}\( f \) не имеет скачков на \( \langle A, B \rangle \)\end{icloze}
\end{note}

\begin{note}{3c586c7317d247a3be4f7b50373a0d46}
    Является ли непрерывность функции \( f \) на промежутке необходимым условием для существования у неё первообразной?

    \begin{cloze}{1}
        Нет, поскольку \( f \) может иметь точки разрыва второго рода.
    \end{cloze}
\end{note}

\begin{note}{ca1243ec222b4440903a1f5a22a53b16}
    \subsubsection{<<Достаточное условие существования \\\phantom{<<}первообразной>>}
    \begin{icloze}{1}Если \( f \) непрерывна на \( \langle A, B \rangle \), то \( f \) имеет первообразную на \( \langle A, B \rangle \).\end{icloze}
\end{note}

\section{Лекция 11.03.22}
\begin{note}{8d01db3371424aba95e1092ffa2cd4dc}
    Пусть \begin{icloze}{3}\( f : E \subset \mathbb R \to \mathbb R \).\end{icloze} \begin{icloze}{1}Функция \( F : E \to \mathbb R \)\end{icloze} называется \begin{icloze}{2}первообразной \( f \) на множестве \( E \),\end{icloze} если \begin{icloze}{1}\( F \) дифференцируема на \( E \) и \( F'(x) = f(x) \) для любого \( x \in E \).\end{icloze}
\end{note}

\begin{note}{a36222511f224d049fc0a1fc0c465aa5}
    Интеграл \( \int f(x)\: dx \) называется \begin{icloze}{2}берущимся,\end{icloze} если \begin{icloze}{1}функция \( f \) имеет элементарную первообразную.\end{icloze}
\end{note}

\begin{note}{937d08196fed4fea9d424dfd802f1c82}
    Пусть \( f, g : \langle A, B \rangle \to \mathbb R \) имеют на  \( \langle A, B \rangle \) первообразную. Тогда для любых \( \alpha, \beta \in \mathbb R \setminus \left\{ 0 \right\} \)
    \[
        \int \left( \alpha f(x) + \beta g(x) \right)\: dx = \begin{icloze}{1}\alpha \int f(x)\: dx + \beta \int g(x)\: dx.\end{icloze}
    \]
\end{note}

\begin{note}{2f7dd89b9a244dacbf41650571c4f13c}
    Как доказать свойство линейности неопределённого интеграла?

    \begin{cloze}{1}
        По определению интеграла и первообразной.
    \end{cloze}
\end{note}

\begin{note}{26b34c9a101f488aaed5ddee4ddd43d2}
    \subsubsection{<<\begin{icloze}{3}Теорема о замене переменной \\
    \phantom{<<}в неопределённом интеграле\end{icloze}>>}

    Пусть \begin{icloze}{2}\( f : \langle A, B \rangle \to \mathbb R \), \( F \in \mathscr P_f (\langle A, B \rangle) \), \( \varphi : \langle C, D \rangle \to \langle A, B \rangle \) и \( \varphi \) дифференцируема на \( \langle C, D \rangle \).\end{icloze}
    Тогда
    \begin{icloze}{1}
        \[
            \int f(\varphi(x)) \cdot \varphi'(x)\: dx = F(\varphi(x)) + C.
        \]
    \end{icloze}
\end{note}

\begin{note}{2f7dd89b9a244dacbf41650571c4f13c}
    Как доказать теорему о замене переменной в неопределённом интеграле?

    \begin{cloze}{1}
        По определению интеграла и первообразной.
    \end{cloze}
\end{note}

\begin{note}{cf45cd81236549efb89f81fcce13349f}
    Пусть \begin{icloze}{3}\( f : \langle A, B \rangle \to \mathbb R \), \( \varphi : \langle C, D \rangle \to \langle A, B \rangle \) и \( \varphi \) дифференцируема на \( \langle A, B \rangle \) и обратима.\end{icloze} Тогда если \begin{icloze}{1}\( G \) --- первообразная функции \( \left( f \circ \varphi \right) \cdot \varphi' \),\end{icloze} то
    \[
        \begin{icloze}{2}\int f(x)\: dx\end{icloze} = \begin{icloze}{1}G(\varphi^{-1} (x)) + C.\end{icloze}
    \]
\end{note}

\begin{note}{f1d541a0c135409c8aef89920ad254e8}
    \subsubsection{<<\begin{icloze}{3}Формула интегрирования по частям\end{icloze}>>}
    Пусть \begin{icloze}{2}\( f, g \in C^{1} \langle A, B \rangle \).\end{icloze} Тогда
    \begin{icloze}{1}
        \[
            \int f(x) g'(x)\: dx = f(x) g(x) -  \int g(x) f'(x)\: dx.
        \]
    \end{icloze}
\end{note}

\begin{note}{e2df459e1699495f980cdddacc633f6f}
    В чем основная идея доказательства формулы интегрирования по частям для неопределённого интеграла?

    \begin{cloze}{1}
        \[
            (uv)'  = u' v + u v' \implies uv = \int v u'\: dx + \int u v'\: dx.
        \]
    \end{cloze}
\end{note}

\section{Лекция 18.03.22}
\begin{note}{ae4062806eca4ddd9b9f4afa5197e8e5}
    Любая рациональная функция имеет \begin{icloze}{1}элементарную первообразную.\end{icloze}
\end{note}

\begin{note}{e8574dd4be844dd3a30f41aa822525cb}
    \[
        \begin{icloze}{2}[a : b]\end{icloze} \overset{\text{def}}= \begin{icloze}{1}[a, b] \cap \mathbb Z.\end{icloze}
    \]
\end{note}

\begin{note}{c4e15a9924f5453cbaa5673cf84f62f5}
    Пусть \begin{icloze}{3}\( [a, b] \) -- невырожденный отрезок.\end{icloze}
    \begin{icloze}{1}Набор точек
    \[
        \left\{ x_k \right\}_{k = 0}^{n} : \quad a = x_0 < \cdots < x_n = b.
    \]\end{icloze}
    называется \begin{icloze}{2}разбиением отрезка \( [a, b] \).\end{icloze}
\end{note}

\begin{note}{e301682aa933430591e748e6973a1843}
    Пусть \({ [a, b] }\) --- невырожденный отрезок.
    \begin{icloze}{1}Множество всех разбиений отрезка \({ [a, b] }\)\end{icloze} обозначается \begin{icloze}{2}\({ T[a, b] }\).\end{icloze}
\end{note}

\begin{note}{6f5e8266e0b44eeebba980ac5d8c6112}
    Пусть \( \left\{ x_k \right\}_{k = 0}^{n} \) --- некоторое разбиение отрезка \( [a, b] \).
    Тогда
    \[
        \begin{icloze}{2}\Delta x_k\end{icloze} \overset{\text{def}}= \begin{icloze}{1}x_{k + 1} - x_k.\end{icloze}
    \]
\end{note}

\begin{note}{22701dee44544e9092fe48e0e077273a}
    Пусть \( \tau = \left\{ x_k \right\}_{k = 0}^{n} \) --- некоторое разбиение отрезка \( [a, b] \).
    \begin{icloze}{1}Величина
    \[
        \max \left\{ \Delta x_k \right\}
    \]\end{icloze}
    называется \begin{icloze}{2}рангом разбиения \( \tau \).\end{icloze}
\end{note}

\begin{note}{7c1e8de0a92a44b897b789c2e84da964}
    Пусть \( \tau = \left\{ x_k \right\}_{k = 0}^{n} \) --- некоторое разбиение отрезка \( [a, b] \).
    \begin{icloze}{2}Ранг разбиения \( \tau \)\end{icloze} обозначается \begin{icloze}{1}\( \lambda_\tau \).\end{icloze}
\end{note}

\begin{note}{47c24c1487804ce88e30a8dfb2519b37}
    Пусть \( \tau = \left\{ x_k \right\}_{k = 0}^{n} \) --- некоторое разбиение отрезка \( [a, b] \).
    \begin{icloze}{1}Набор точек \( \left\{ \xi_k \right\}_{k = 0}^{n - 1} \) таких, что \( \xi_k \in [x_k, x_{k + 1}] \)\end{icloze} называется \begin{icloze}{2}оснащением разбиения \( \tau \).\end{icloze}
\end{note}

\begin{note}{5e83015672844d94a0a89355f7af372e}
    Пусть \begin{icloze}{3}\( \tau \) --- некоторое разбиение отрезка \( [a, b] \), \( \xi \) --- оснащение разбиения \( \tau \).\end{icloze}
    Тогда \begin{icloze}{1}пара \( (\tau, \xi) \)\end{icloze} называется \begin{icloze}{2}оснащённым разбиением отрезка \( [a, b] \).\end{icloze}
\end{note}

\begin{note}{974eeb7d70c24d318e71abd3d9a95f3f}
    Пусть \({ [a, b] }\) --- невырожденный отрезок.
    \begin{icloze}{1}Множество всех оснащённых разбиений отрезка \({ [a,  b] }\)\end{icloze} обозначается \begin{icloze}{2}\({ T'[a, b] }\).\end{icloze}
\end{note}

\begin{note}{ef2c57fbd464435c9896c8e8f24db8b5}
    Пусть \begin{icloze}{3}\( f : [a, b] \to \mathbb R \), \: \( (\tau, \xi) = \left( \left\{ x_k \right\}, \left\{ \xi_k \right\} \right) \) --- оснащённое разбиение \( [a, b] \).\end{icloze}
    \begin{icloze}{1}Сумма
    \[
        \sum_{k=0}^{n - 1} f(\xi_k) \Delta x_k
    \]\end{icloze}
    называется \begin{icloze}{2}интегральной суммой функции \( f \), отвечающей оснащённому разбиению \( (\tau, \xi) \).\end{icloze}
\end{note}

\begin{note}{f4adab8132d7489fb5594271853a86c7}
    Интегральные суммы так же называют \begin{icloze}{1}суммами Римана.\end{icloze}
\end{note}

\begin{note}{c14685fee7ff492d9e5452c059f94fb6}
    \begin{icloze}{2}Интегральная сумма функции \( f \), отвечающая оснащённому разбиению \( (\tau, \xi) \),\end{icloze} обозначается как
    \begin{icloze}{1}
        \[
            \sigma_\tau (f, \xi).
        \]
    \end{icloze}
\end{note}

\begin{note}{f356a2fc28ae4487aae50ba5b3064cee}
    Пусть \begin{icloze}{3}\( f : [a, b] \to \mathbb R \).\end{icloze}
    \begin{icloze}{4}Число \( I \in \mathbb R \)\end{icloze} называют \begin{icloze}{2}пределом интегральных сумм функции \( f \) при ранге разбиения, стремящемся к нулю,\end{icloze} если
    \begin{icloze}{1}
        \[
            \forall \varepsilon > 0 \quad \exists \delta > 0 \quad \forall (\tau, \xi) : \lambda_\tau < \delta \quad \left| \sigma_\tau (f, \xi) - I \right| < \varepsilon,
        \]
        где \( (\tau, \xi) \) --- оснащённое разбиение отрезка \( [a, b] \).
    \end{icloze}

    \begin{center}
        \tiny (определение в терминах \( (\varepsilon, \delta) \))
    \end{center}
\end{note}

\begin{note}{ed766ec774814eba83502c9dd75a2e49}
    \begin{icloze}{2}Предел интегральных сумм функции \( f \) при ранге разбиения стремящемся к нулю\end{icloze} обозначается
    \begin{icloze}{1}
        \[
            \lim_{\lambda_\tau \to 0} \sigma_\tau (f, \xi) \quad \text{или} \quad \lim_{\lambda \to 0} \sigma.
        \]
    \end{icloze}
\end{note}

\begin{note}{46f5a6ad385a4386813c6f707bd08927}
    Пусть \( f : [a, b] \to \mathbb R \), \( I \in \mathbb R \).
    Число \( I \) называют пределом интегральных сумм функции \( f \) при ранге разбиения, стремящемся к нулю, если
    \begin{icloze}{1}
        для любой последовательности оснащённых разбиений \( \left\{ (\tau_j, \xi_j) \right\}_{j = 1}^{\infty} \) такой, что \( \lambda_{\tau_j} \underset{j \to \infty}\longrightarrow 0 \),
    \[
        \sigma_{\tau_j} (f, \xi_j) \underset{j \to \infty}\longrightarrow I.
    \]
    \end{icloze}

    \begin{center}
        \tiny (определение в терминах последовательностей)
    \end{center}
\end{note}

\section{Семинар 17.03.22}
\begin{note}{e25a48aad5c048c3b2d3b7e2d9af0b98}
    Алгоритм взятия интеграла вида
    \[
        \displaystyle \int \frac{Mx + N}{\sqrt{ax^2 + bx + c}}\: dx.
    \]

    \begin{cloze}{1}
        Выделить полный квадрат под радикалом и почленно поделить числитель на знаменатель.
    \end{cloze}
\end{note}

\begin{note}{79c04c292b2a4aeb8fb583ccc7916c2a}
    Алгоритм взятия интеграла вида
    \[
        \displaystyle \int (Mx + N) \sqrt{ax^2 + bx + c}\: dx.
    \]

    \begin{cloze}{1}
        Выделить полный квадрат под радикалом и раскрыть скобки.
    \end{cloze}
\end{note}

\begin{note}{30fa84062ed64fdabc405fa09e0c6148}
    Алгоритм взятия интеграла вида
    \[
        \int \frac{P_n(x)}{\sqrt{ax^2 + bx + c}}\: dx, \quad \text{где \( P_n \in \mathbb R[x]_n \).}
    \]

    \begin{cloze}{1}
        Представить ответ в виде
        \[
            Q_{n - 1}(x) \sqrt{ax^2 + bx + c} + \lambda \int \frac{1}{\sqrt{ax^2 + bx + c}}\: dx,
        \]
        продифференцировать левую и правую часть равенства и найти неизвестные коэффициенты в \( Q_{n - 1}(x) \) и \( \lambda \) из полученного соотношения.
    \end{cloze}
\end{note}

\begin{note}{15f51247a7d04b4fb445ea745f418ca4}
    \[
        \int \begin{icloze}{2}\frac{1}{\sqrt{x^2 + px + q}}\end{icloze}\: dx = \begin{icloze}{1}\ln\left| x + \frac{p}{2} + \sqrt{x^2 + px + q} \right|\end{icloze} + C.
    \]
\end{note}

\begin{note}{9b1156318f464dc79a658d6e94fe214d}
    \[
        \int \begin{icloze}{3}\frac{1}{\sqrt{-x^2 + px + q}}\end{icloze}\: dx = \begin{icloze}{1}\arcsin \frac{2x - p}{2a}\end{icloze} + C,
    \]
    где \( a := \begin{icloze}{2}\sqrt{\frac{4q + p^2}{4}}\end{icloze} \).
\end{note}

\section{Лекция 21.03.22}
\begin{note}{679c0a0615d44749bc685cda9a47b233}
    Пусть \begin{icloze}{3}\( f : [a, b] \to \mathbb R \).\end{icloze} \( f \) называется \begin{icloze}{2}интегрируемой по Риману на \( [a, b] \),\end{icloze} если \begin{icloze}{1}существует \( \displaystyle \lim_{\lambda_\tau \to 0} \sigma_\tau(f, \xi) \).\end{icloze}
\end{note}

\begin{note}{d6b62c8f08a842b2829447ab45a27e8c}
    Пусть \begin{icloze}{3}\({ f : [a, b] \to \mathbb R }\) интегрируема по Риману на \( [a, b] \).\end{icloze} Тогда
    \begin{icloze}{2}
        \[
            \lim_{\lambda_\tau \to 0} \sigma_\tau(f, \xi)
        \]
    \end{icloze}
    называется \begin{icloze}{1}определённым интегралом Римана от функции \( f \) по отрезку \( [a, b] \).\end{icloze}
\end{note}

\begin{note}{7dc12d32c0ce407f87be1d7c51d0b1b3}
    \begin{icloze}{2}Интеграл Римана от функции \( f \) по отрезку \( [a, b] \)\end{icloze} обозначается
    \begin{icloze}{1}
        \[
            \int_{a}^{b} f \quad \text{или} \quad \int_{a}^{b} f(x)\: dx.
        \]
    \end{icloze}
\end{note}

\begin{note}{e5e082ef6db649858cc60a662cb312b1}
    В выражении
    \[
        \int_{a}^{b} f
    \]
    \begin{icloze}{2}числа \( a, b \)\end{icloze} называют \begin{icloze}{1}пределами интегрирования.\end{icloze}
\end{note}

\begin{note}{44bd096f622b475f908006fcf8e88426}
    В выражении
    \[
        \int_{a}^{b} f
    \]
    \begin{icloze}{2}функцию \( f \)\end{icloze} называют \begin{icloze}{1}подынтегральной функцией.\end{icloze}
\end{note}

\begin{note}{4e8ab8723a9e485abea045a4aa0c79f0}
    Пусть \({ [a, b] }\) --- невырожденный отрезок.
    \begin{icloze}{1}Множество всех функций интегрируемых по Риману на \( [a, b] \)\end{icloze} обозначается \begin{icloze}{2}\( \mathcal R[a, b] \).\end{icloze}
\end{note}

\begin{note}{1294b085870d432eae003c1159bbcb60}
    Пусть \begin{icloze}{3}\({ f : [a, b] \to \mathbb R }\),\: \({ \tau = \left\{ x_k \right\}_{k = 0}^{n} }\) --- разбиение отрезка \({ [a, b] }\).\end{icloze}
    Тогда \begin{icloze}{1}сумма
    \[
        \begin{gathered}
            \displaystyle \sum_{k=0}^{n - 1} M_k \Delta x_k, \quad \text{где \({ M_k := \sup f([x_k, x_{k + 1}]) }\)},
        \end{gathered}
    \]\end{icloze}
    называется \begin{icloze}{2}верхней интегральной суммой Дарбу, отвечающей разбиению \({ \tau }\).\end{icloze}
\end{note}

\begin{note}{8807ccc652554a53aa9f97a7ee09ad99}
    Пусть \({ f : [a, b] \to \mathbb R }\), \({ \tau \in T[a, b] }\).
    \begin{icloze}{1}Верхняя интегральная сумма Дарбу функции \({ f }\), отвечающая разбиению \({ \tau }\),\end{icloze} обозначается
    \begin{icloze}{2}
        \[
            S_\tau (f).
        \]
    \end{icloze}
\end{note}

\begin{note}{220907a5d6e248b78f987af0d058e64c}
    Пусть \begin{icloze}{3}\({ f : [a, b] \to \mathbb R }\), \({ \tau = \left\{ x_k \right\}_{k = 0}^{n} }\) --- разбиение отрезка \({ [a, b] }\).\end{icloze}
    Тогда \begin{icloze}{1}сумма
    \[
        \begin{gathered}
            \displaystyle \sum_{k=0}^{n - 1} m_k \Delta x_k, \quad \text{где \({ m_k := \inf f([x_k, x_{k + 1}]) }\)},
        \end{gathered}
    \]\end{icloze}
    называется \begin{icloze}{2}нижней интегральной суммой Дарбу, отвечающей разбиению \({ \tau }\).\end{icloze}
\end{note}

\begin{note}{2bbefff21b2c4c8fba866cd8eea6b02c}
    Пусть \({ f : [a, b] \to \mathbb R }\), \({ \tau \in T[a, b] }\).
    \begin{icloze}{1}Нижняя интегральная сумма Дарбу функции \({ f }\), отвечающая разбиению \({ \tau }\),\end{icloze} обозначается
    \begin{icloze}{2}
        \[
            s_\tau (f).
        \]
    \end{icloze}
\end{note}

\begin{note}{189f37e44f0a45048e5cb16973582e14}
    Пусть \begin{icloze}{4}\({ f : [a, b] \to \mathbb R }\), \({ \tau }\) --- разбиение \({ [a, b] }\).\end{icloze}
    Тогда \begin{icloze}{2}\({ f }\) ограничена сверху\end{icloze} \begin{icloze}{3}тогда и только тогда, когда\end{icloze} \begin{icloze}{1}сумма \({ S_\tau (f) }\) конечна.\end{icloze}
\end{note}

\begin{note}{083512018d304036a80002a9df45af7e}
    Пусть \begin{icloze}{4}\({ f : [a, b] \to \mathbb R }\), \({ \tau }\) --- разбиение \({ [a, b] }\).\end{icloze}
    Тогда \begin{icloze}{2}\({ f }\) ограничена снизу\end{icloze} \begin{icloze}{3}тогда и только тогда, когда\end{icloze} \begin{icloze}{1}сумма \({ s_\tau (f) }\) конечна.\end{icloze}
\end{note}

\begin{note}{1c8af1c02f864877bddd4971a256a30e}
    Пусть \({ f : [a, b] \to \mathbb R }\), \({ \tau }\) --- разбиение \({ [a, b] }\).
    Как \({ S_\tau (f) }\) выражается через суммы Римана?

    \begin{cloze}{1}
        \[
            S_\tau (f) = \sup \left\{ \sigma_\tau (f, \xi) \mid \forall \xi \right\}
        \]
    \end{cloze}
\end{note}

\begin{note}{7958d85410954f6280755a33f7bff6fb}
    Пусть \({ f : [a, b] \to \mathbb R }\), \({ \tau }\) --- разбиение \({ [a, b] }\).
    Как \({ s_\tau (f) }\) выражается через суммы Римана?

    \begin{cloze}{1}
        \[
            s_\tau (f) = \inf \left\{ \sigma_\tau (f, \xi) \mid \forall \xi \right\}
        \]
    \end{cloze}
\end{note}

\begin{note}{53ffcba153934fda879e5241f8e85387}
    Пусть \({ f : [a, b] \to \mathbb R }\), \({ \tau }\) --- разбиение \({ [a, b] }\).
    Как, в общих чертах, доказать, что \({ S_\tau (f) = \sup \left\{ \sigma_\tau (f, \xi) \mid \forall \xi \right\} }\)?

    \begin{cloze}{1}
        Представить \({ \left\{ \sigma_\tau (f, \xi) \mid \forall \xi \right\} }\) как сумму множеств
        \[
            \Delta x_k \cdot f([x_k, x_{k + 1}]).
        \]
    \end{cloze}
\end{note}

\begin{note}{453749996f00487b9b845f66318e9f7c}
    Пусть \begin{icloze}{2}\({ f : [a, b] \to \mathbb R }\),\: \({ \tau, \tilde \tau }\) --- два разбиения \({ [a, b] }\),\: \({ \tau \subset \tilde \tau }\).\end{icloze} Тогда
    \begin{icloze}{1}
        \[
            \begin{gathered}
                S_{\tilde \tau} (f) \leqslant S_\tau (f), \\
                s_{\tilde \tau} (f) \geqslant s_\tau (f).
            \end{gathered}
        \]
    \end{icloze}
\end{note}

\section{Лекция 25.03.22}
\begin{note}{a23a2495841f4894a31b489127b41054}
    Пусть \({ f : [a, b] \to \mathbb R }\). Как связаны \({ s_{\tau_1}(f) }\) и \({ S_{\tau_2} (f) }\) для произвольных разбиений \({ \tau_1, \tau_2 }\) отрезка \({ [a, b] }\)?

    \begin{cloze}{1}
        \[
            s_{\tau_1} (f) \leqslant S_{\tau_2} (f)
        \]
    \end{cloze}
\end{note}

\begin{note}{84c295b304a64dd3a80a791f82958c91}
    Верно ли, что каждая нижняя сумма Дарбу функции \({ f }\) не превосходит каждой верхней суммы Дарбу этой же функции даже для разных разбиений отрезка?

    \begin{cloze}{1}
        Да. \({ s_{\tau_1} (f) \leqslant S_{\tau_2} (f) }\) для любых \({ \tau_1, \tau_2 }\)
    \end{cloze}
\end{note}

\begin{note}{b7fac4e6a3324160adefc29c06d73479}
    Каждая нижняя сумма Дарбу не превосходит каждой верхней суммы Дарбу.
    В чем основная идея доказательства?

    \begin{cloze}{1}
        Для \({ \tau_1 = \tau_2 }\) утверждение тривиально. В ином случае рассмотреть суммы Дарбу для разбиения \({ \tau = \tau_1 \cup \tau_2 }\).
    \end{cloze}
\end{note}

\begin{note}{be394bd9e8e2456284b7c108e7e973f8}
    Существует ли ограниченная на отрезке функция, неинтегрируемая на нём?

    \begin{cloze}{1}
        Да. Например, функция Дирихле.
    \end{cloze}
\end{note}

\begin{note}{3b28a2ca07d44ea38f8d2df0ce9f396f}
    Существует ли интегрируемая на отрезке функция, неограниченная на нём?

    \begin{cloze}{1}
        Нет. Любая интегрируемая на отрезке функция ограничена на нём.
    \end{cloze}
\end{note}

\begin{note}{c6120328fd3e40a48f6d7e69fce29c9d}
    Как показать, что любая интегрируемая на отрезке функция ограничена на нём?

    \begin{cloze}{1}
        Если допустить, что \({ f }\) не ограниченна, то \({ \forall \tau }\) имеем \({ S_\tau (f) = \sup \left\{ \sigma_\tau (f, \xi) \right\} = +\infty }\), а значит
        \[
            \nexists \lim_{\lambda_\tau \to 0} \sigma_\tau (f, \xi).
        \]
    \end{cloze}
\end{note}

\begin{note}{e5921a1f2caa4ed583198b136ce6b34c}
    Пусть \({ f : \left[ a, b \right] \to \mathbb R }\). Величина
    \begin{icloze}{1}
        \[
            \inf \left\{ S_{\tau} (f) \mid \forall \tau \right\}
        \]
    \end{icloze}
    называется \begin{icloze}{2}верхним интегралом Дарбу функции \({ f }\).\end{icloze}
\end{note}

\begin{note}{fcfb0f775cac40c9a18563576c086827}
    Пусть \({ f : \left[ a, b \right] \to \mathbb R }\). \begin{icloze}{1}Верхний интеграл Дарбу функции \({ f }\)\end{icloze} обычно обозначатся \begin{icloze}{2}\( I^{*} \).\end{icloze}
\end{note}

\begin{note}{304c0f4c87fe44cb922eeaf557997d02}
    Пусть \({ f : \left[ a, b \right] \to \mathbb R }\). Величина
    \begin{icloze}{1}
        \[
            \sup \left\{ s_{\tau} (f) \mid \forall \tau \right\}
        \]
    \end{icloze}
    называется \begin{icloze}{2}нижним интегралом Дарбу функции \({ f }\).\end{icloze}
\end{note}

\begin{note}{bd7f75d9e429454599a993144985b2dc}
    Пусть \({ f : \left[ a, b \right] \to \mathbb R }\). \begin{icloze}{1}Нижний интеграл Дарбу функции \({ f }\)\end{icloze} обычно обозначатся \begin{icloze}{2}\( I_{*} \).\end{icloze}
\end{note}

\begin{note}{83287fd934bc4878a99a742db1668220}
    \subsubsection{<<\begin{icloze}{3}Критерий\end{icloze} \begin{icloze}{2}интегрируемости функции\end{icloze}>>}
    Пусть \begin{icloze}{4}\({ f : \left[ a, b \right] \to \mathbb R }\).\end{icloze} Тогда \begin{icloze}{2}\({ f \in\mathcal R\left[ a, b \right] }\)\end{icloze} \begin{icloze}{3}тогда и только тогда, когда\end{icloze}
    \begin{icloze}{1}
        \[
            S_\tau(f) - s_\tau(f) \underset{\lambda_t \to 0}\longrightarrow 0.
        \]
    \end{icloze}
\end{note}

\begin{note}{68b782b8c09040dfa994ede932b748bc}
    \begin{multline*}
        \begin{icloze}{2}S_\tau(f) - s_\tau(f) \underset{\lambda_\tau \to 0}\longrightarrow 0\end{icloze}
        \overset{\hspace{-1pt}\text{\tiny def}}\iff \\
        \begin{icloze}{1}
            \begin{gathered}
                \forall \varepsilon > 0 \quad \exists \delta > 0 \quad \forall \tau : \lambda_\tau < \delta \\
                S_\tau(f) - s_\tau(f) < \varepsilon.
            \end{gathered}
        \end{icloze}
    \end{multline*}

    \begin{center}
        \tiny
        (в терминах \({ \varepsilon, \delta }\))
    \end{center}
\end{note}

\begin{note}{6a94745f7ac74ef7a5be1b0a6128e303}
    В чем ключевая идея доказательства критерия интегрируемости функции (необходимость)?

    \begin{cloze}{1}
        По определению \({ \sup }\) и \({ \inf }\)
        \[
            I - \varepsilon \leqslant s_\tau (f), \quad S_\tau (f) \leqslant I + \varepsilon.
        \]
    \end{cloze}
\end{note}

\begin{note}{53f49c34063149d892ad1eb1015abebd}
    В чем ключевая идея доказательства критерия интегрируемости функции (достаточность)?

    \begin{cloze}{1}
        \[
            \begin{gathered}
                s_\tau (f) \leqslant I_* \leqslant I^* \leqslant S_\tau (f), \\
                s_\tau (f) \leqslant \sigma_\tau (f, \xi) \leqslant S_\tau (f).
            \end{gathered}
        \]
        для любого оснащённого разбиения \({ (\tau, \xi) }\) отрезка \({ \left[ a, b \right] }\).
    \end{cloze}
\end{note}

\begin{note}{f200bb84909d45898f1313f053135d3a}
    Пусть \({ f \in \mathcal R[a, b] }\). Как соотносятся \({ I^* }\), \({ I_* }\) и \({ \int_{a}^{b} f }\)?

    \begin{cloze}{1}
        \[
            I_* = I^* = \int_{a}^{b} f.
        \]
    \end{cloze}
\end{note}

\begin{note}{1e7ec80a8e8f4caf87b70493394d837a}
    Пусть \({ f : [a, b] \to \mathbb R }\).
    Как для произвольного разбиения \({ \tau }\) соотносятся \({ s_\tau (f) }\), \({ S_\tau (f) }\) и \({ \int_{a}^{b} f }\)?

    \begin{cloze}{1}
        \[
            s_\tau (f) \leqslant \int_{a}^{b} f \leqslant S_\tau (f).
        \]
    \end{cloze}
\end{note}

\begin{note}{c1ca9a97a9a948e685e22801cc7e1ee5}
    Если \({ f \in \mathcal R\left[ a, b \right] }\), то
    \[
        \lim_{\lambda_\tau \to 0} S_\tau (f) = \begin{icloze}{1}\int_{a}^{b} f.\end{icloze}
    \]
\end{note}

\begin{note}{6a890ec9b5384ed2944133d968407712}
    Как показать, что
    \[
        f \in \mathcal R[a, b] \implies \lim_{\lambda_\tau \to 0} S_\tau (f) = \int_{a}^{b} f?
    \]

    \begin{cloze}{1}
        Тривиально следует из критерия интегрируемости и неравенства
        \[
            s_\tau (f) \leqslant \int_{a}^{b} f \leqslant S_\tau (f).
        \]
    \end{cloze}
\end{note}

\begin{note}{1a704d6d276749bdaa56a88c05622b02}
    Если \({ f \in \mathcal R\left[ a, b \right] }\), то
    \[
        \lim_{\lambda_\tau \to 0} s_\tau (f) = \begin{icloze}{1}\int_{a}^{b} f.\end{icloze}
    \]
\end{note}

\begin{note}{f274e2ec8d6648a383184e816782dedf}
    Пусть \begin{icloze}{3}\({ f : D \subset \mathbb R \to \mathbb R }\).\end{icloze}
    \begin{icloze}{1}
        Величина
        \[
            \sup \Big\{ f(x) - f(\hat{x}) \mid x, \hat{x} \in D \Big\}
        \]
    \end{icloze}
    называется \begin{icloze}{2}колебанием функции \({ f  }\) на множестве \({ D }\).\end{icloze}
\end{note}

\begin{note}{073304f993f94da7ab2f56d20b074752}
    Пусть \({ f : D \subset \mathbb R \to \mathbb R }\). \begin{icloze}{1}Колебание функции \({ f }\) на множестве \({ D }\)\end{icloze} обозначается \begin{icloze}{2}\( \omega(f) \).\end{icloze}
\end{note}

\begin{note}{0a7ceb7d5f804209a506b41349ce11c9}
    Пусть \({ f : D \subset \mathbb R \to \mathbb R }\). Тогда
    \[
        \begin{icloze}{2}\omega(f)\end{icloze} = \begin{icloze}{1}\sup f(D) - \inf f(D).\end{icloze}
    \]

    \begin{center}
        \tiny
        (в терминах \({ \sup f, \inf f }\))
    \end{center}
\end{note}

\begin{note}{ec97e108f2394602934c87c2f28f2a39}
    Пусть \({ f : \left[ a, b \right] \to \mathbb R }\),\: \({ \tau = \left\{ x_k \right\}_{k = 0}^{n} }\) --- разбиение \({ \left[ a, b \right] }\).
    Тогда
    \[
        \begin{icloze}{2}\omega_k (f)\end{icloze} \overset{\text{def}}= \begin{icloze}{1}\omega (f | _{[x_k, x_{k + 1}]}).\end{icloze}
    \]
\end{note}

\begin{note}{649783b3a0c14571bdbbb8caba0d07a3}
    Пусть \({ f : \left[ a, b \right] \to \mathbb R }\),\: \({ \tau = \left\{ x_k \right\}_{k = 0}^{n} }\) --- разбиение \({ \left[ a, b \right] }\).
    Тогда
    \[
        S_\tau (f) - s_\tau (f) = \begin{icloze}{1}\sum_{k=0}^{n - 1} \omega_k (f) \Delta x_k.\end{icloze}
    \]

    \begin{center}
        \tiny
        (в терминах \({ \omega_k (f) }\))
    \end{center}
\end{note}

\begin{note}{82a86f84ecc44f89aac1396d471738d1}
    Пусть \({ f : \left[ a, b \right] \to \mathbb R }\). Тогда
    \[
        \begin{icloze}{2}\lim_{\lambda_\tau \to 0} S_\tau (f)\end{icloze} = \begin{icloze}{1}I^{*}\end{icloze}.
    \]

    \begin{center}
        \tiny
        (в терминах предела при \({ \lambda_\tau \to 0 }\))
    \end{center}
\end{note}

\begin{note}{79915436f5b44d41a053bf8c0bf9e3ac}
    Пусть \({ f : \left[ a, b \right] \to \mathbb R }\). Тогда
    \[
        \begin{icloze}{2}\lim_{\lambda_\tau \to 0} s_\tau (f)\end{icloze} = \begin{icloze}{1}I_{*}\end{icloze}.
    \]

    \begin{center}
        \tiny
        (в терминах предела при \({ \lambda_\tau \to 0 }\))
    \end{center}
\end{note}

\begin{note}{180307492ee647ac8b1bb30c91dcfb0d}
    \subsubsection{\begin{icloze}{3}Критерий\end{icloze} \begin{icloze}{4}Дарбу\end{icloze} \begin{icloze}{2}интегрируемости функции\end{icloze}}

    Пусть \({ f : \left[ a, b \right]  \to \mathbb R }\). Тогда
    \begin{center}
        \begin{icloze}{2}\({ f \in \mathcal R\left[ a, b \right] }\)\end{icloze} \begin{icloze}{3}\({ \iff }\)\end{icloze} \begin{icloze}{1}\({ f }\) ограничена на \({ \left[ a, b \right] }\) и \({ I_* = I^* }\).\end{icloze}
    \end{center}
\end{note}

\begin{note}{4fec9ca16e3d4d6a8ee9ac0f388eb31c}
    \subsubsection{\begin{icloze}{3}Критерий\end{icloze} \begin{icloze}{4}Римана\end{icloze} \begin{icloze}{2}интегрируемости функции\end{icloze}}

    Пусть \({ f : \left[ a, b \right]  \to \mathbb R }\). Тогда
    \begin{center}
        \begin{icloze}{2}\({ f \in \mathcal R\left[ a, b \right] }\)\end{icloze} \begin{icloze}{3}\({ \iff }\)\end{icloze} \begin{icloze}{1}\({ \forall \varepsilon > 0 \quad \exists \tau \quad S_\tau (f) - s_\tau (f) < \varepsilon }\).\end{icloze}
    \end{center}
\end{note}

\begin{note}{8ba2a0bdc9754111b4f3eae4493a895d}
    Существует ли непрерывная на отрезке функция, неинтегрируемая на этом отрезке?

    \begin{cloze}{1}
        Нет. Любая непрерывная на отрезке функция интегрируема на этом отрезке.
    \end{cloze}
\end{note}

\begin{note}{da9a4e99c01a42a4a8c5bccf8e3d24f7}
    Как, в общих чертах, доказать, что любая непрерывная на отрезке функция интегрируема на нём?

    \begin{cloze}{1}
        Из теоремы Кантора получить равномерную непрерывность и по теореме Вейерштрасса оценить для \({ \lambda_\tau < \delta }\) величину \({ \omega_k(f) }\).
    \end{cloze}
\end{note}

\begin{note}{cb2eb65d72554cea90a47503d1d97474}
    Существует ли монотонная на отрезке функция, неинтегрируемая на этом отрезке?

    \begin{cloze}{1}
        Нет. Любая монотонная на отрезке функция интегрируема на этом отрезке.
    \end{cloze}
\end{note}

\begin{note}{7a4501cbb6414feaaf12589452716ae3}
    Как, в общих чертах, доказать, что любая монотонная на отрезке функция интегрируема на этом отрезке?

    \begin{cloze}{1}
        Для определённости \({ f \!\!\nearrow }\). Для произвольного \({ \varepsilon }\) взять
        \[
            \delta = \frac{\varepsilon}{f(b) - f(a)}.
        \]
        Далее по критерию интегрируемости в терминах колебаний.
    \end{cloze}
\end{note}

\section{Семинар 24.03.22}
\begin{note}{e2a6745c55e2452bb18cc67eb3b51eb9}
    Интегалы вида
    \[
        \int \begin{icloze}{2}\frac{Mx + N}{(x - \alpha)^{k} \sqrt{ax^2 + bx + c}}\end{icloze}\: dx
    \]
    берутся с помощью
    \begin{icloze}{1}замены
        \[
            t := \frac{1}{x - \alpha}.
        \]
    \end{icloze}
\end{note}

\begin{note}{76f25092a99044e7b985eeafa85ca9ca}
    Интегралы вида \begin{icloze}{2}
    \[
        \int R(x, \sqrt{ax^2 + bx + c})\: dx,
    \]
    где \({ R }\) --- рациональная функция,\end{icloze} берутся с помощью \begin{icloze}{1}подстановок Эйлера.\end{icloze}
\end{note}

\begin{note}{f341e2ce094542f98b0a97abf09c7254}
    Каковы условия для применения каждой из подстановок Эйлера?

    \begin{cloze}{1}
        \begin{enumerate}
            \item \({ a > 0 }\);
            \item \({ c > 0 }\);
            \item \({ ax^2 + bx + c }\) приводим над \({ \mathbb R }\).
        \end{enumerate}
    \end{cloze}
\end{note}

\begin{note}{f0d585d86f5542a088764032d1d2eef8}
    Замена переменной в подстановке Эйлера для \({ a > 0 }\).

    \begin{cloze}{1}
        \[
            \sqrt{ax^2 + bx + c} = \pm x \sqrt{a} \pm t
        \]
    \end{cloze}
\end{note}

\begin{note}{3eb88f39c10241c1a11c805b799d8ae2}
    Замена переменной в подстановке Эйлера для \({ c > 0 }\).

    \begin{cloze}{1}
        \[
            \sqrt{ax^2 + bx + c} = \pm \sqrt{c} \pm xt
        \]
    \end{cloze}
\end{note}

\begin{note}{9df1ed5f198d4793bc9ff466bc983224}
    Замена переменной в подстановке Эйлера для приводимого \({ ax^2 + bx + c }\).

    \begin{cloze}{1}
        \[
            \sqrt{ax^2 + bx + c} = \pm t (x - x_1),
        \]
        где \({ x_1 }\) --- корень \({ ax^2 + bx + c }\).
    \end{cloze}
\end{note}

\begin{note}{5defd336f3aa4b4aa26b239431f04d17}
    \[
        \begin{gathered}
            \int \begin{icloze}{2}\sqrt{x^2 \pm a^2}\end{icloze}\: dx =
            \begin{icloze}{1}\frac{x}{2} \sqrt{x^2 \pm a^2} \pm \frac{a^2}{2} \ln\left\lvert x + \sqrt{x^2 \pm a^2} \right\rvert\end{icloze}
            + C \\
            (a > 0)
        \end{gathered}
    \]
\end{note}

\begin{note}{c6f5af39f62e4d7f9a4cee9e0c45d107}
    \[
        \begin{gathered}
            \int \sqrt{\begin{icloze}{1}x^2 \pm a^2\end{icloze}}\: dx =
            \frac{x}{2} \sqrt{\begin{icloze}{1}x^2 \pm a^2\end{icloze}} \pm \frac{a^2}{2} \ln\left\lvert x + \sqrt{\begin{icloze}{1}x^2 \pm a^2\end{icloze}} \right\rvert
            + C \\
            (a > 0)
        \end{gathered}
    \]
\end{note}

\begin{note}{56abb403ab43437da9bf41b7ca9d15e8}
    \[
        \begin{gathered}
            \int \begin{icloze}{2}\sqrt{a^2 - x^2}\end{icloze}\: dx =
            \begin{icloze}{1}\frac{x}{2} \sqrt{a^2 - x^2} + \frac{a^2}{2} \arcsin \frac{x}{a}\end{icloze}
            + C \\
            (a > 0).
        \end{gathered}
    \]
\end{note}

\begin{note}{f3627073257a4cd7b1d3adf499699cb9}
    \[
        \begin{gathered}
            \int \sqrt{\begin{icloze}{1}a^2 - x^2\end{icloze}}\: dx =
            \frac{x}{2} \sqrt{\begin{icloze}{1}a^2 - x^2\end{icloze}} + \frac{a^2}{2} \arcsin \frac{x}{a}
            + C \\
            (a > 0).
        \end{gathered}
    \]
\end{note}

\section{Лекция 01.04.22}
\begin{note}{60ff32d5ed7347ae8036518373a5bc61}
    Пусть \begin{icloze}{3}\({ f, \tilde f : [a, b] \to \mathbb R }\),\: \({ f \in  \mathcal R[a, b] }\),\end{icloze}\: \begin{icloze}{4}\({ T \subset [a, b] }\),\: \({ \left\lvert T \right\rvert < \aleph_0 }\).\end{icloze} Если
    \begin{icloze}{1}
        \[
            \forall x \in [a, b] \setminus T \quad f(x) = \tilde f(x),
        \]
    \end{icloze}
    то \begin{icloze}{2}\({ \tilde f \in \mathcal R[a, b] }\) и \({ \int_{a}^{b} f = \int_{a}^{b} \tilde f }\).\end{icloze}
\end{note}

\begin{note}{9f281cfda3464766a189207f5029d7ed}
    В чем ключевая идея доказательства того, что изменение значений функции \({ f \in\mathcal R[a, b] }\) в конечном числе точек не влияет на интегрируемость?

    \begin{cloze}{1}
        \[
            \sigma(f) - \sigma(\tilde f) \underset{\lambda_\tau \to 0}\longrightarrow 0.
        \]
    \end{cloze}
\end{note}

\begin{note}{94c3dcad5bc247568a21704cb1d05f72}
    Пусть \begin{icloze}{2}\({ f \in \mathcal R[a, b] }\), \({ [\alpha, \beta] \subset [a, b] }\).\end{icloze} Тогда
    \begin{icloze}{1}
        \[
            f |_{[\alpha, \beta]} \in \mathcal R[\alpha, \beta].
        \]
    \end{icloze}
\end{note}

\begin{note}{385f602d08114dd39630009f76dbb7e0}
    Пусть \({ f \in \mathcal R[a, b] }\), \({ [\alpha,  \beta]   \subset [a, b] }\). Тогда \({ f|_{[\alpha, \beta]} \in \mathcal R[\alpha, \beta] }\).

    В чем ключевая идея доказательства?

    \begin{cloze}{1}
        Если \({ \tau_0 \in T[\alpha, \beta], \tau \in T[a, b], \tau_0 \subset \tau }\), то
        \[
            S_{\tau_0} - s_{\tau_0} \leqslant S_{\tau} - s_{\tau}.
        \]
    \end{cloze}
\end{note}

\begin{note}{ee426e669ae545b8ad6d128a5870373e}
    Пусть \begin{icloze}{3}\({ f : [a, b] \to \mathbb R }\), \({ c \in (a, b) }\).\end{icloze} Тогда если
    \begin{icloze}{1}
        \[
            f|_{[a, c]} \in \mathcal R[a, c] \quad \land \quad f|_{[c, b]} \in \mathcal R[c, b],
        \]
    \end{icloze}
    то \begin{icloze}{2}\({ f \in \mathcal R[a, b] }\).\end{icloze}
\end{note}

\begin{note}{c8c134e6d7c442c3a7214585d20c41ac}
    Пусть \({ f : [a, b] \to \mathbb R }\), \({ c \in (a, b) }\). Тогда если
    \[
        f|_{[a, c]} \in \mathcal R[a, c] \quad \land \quad f|_{[c, b]} \in \mathcal R[c, b],
    \]
    то \({ f \in \mathcal R[a, b] }\).

    В чем ключевая идея доказательства?

    \begin{cloze}{1}
        \[
            \begin{gathered}
                S_\tau - s_\tau \leqslant S_{\tau_1} - s_{\tau_1} + S_{\tau_2} - s_{\tau_2} + \omega(f) \cdot \lambda_\tau, \\
                \tau \in T[a, b], \quad \tau' = \tau \cup \left\{ c \right\}, \\
                \tau_1 = \tau' \cap [a, c], \quad \tau_2 = \tau' \cap [c, a].
            \end{gathered}
        \]
    \end{cloze}
\end{note}

\begin{note}{ccba27d6a29c468f8a60fbd0f106fec4}
    Пусть \begin{icloze}{3}\({ f : [a, b] \to \mathbb R }\).\end{icloze} Функция \({ f }\) называется \begin{icloze}{2}кусочно непрерывной,\end{icloze} если \begin{icloze}{1}множество её точек разрыва пусто или конечно, и все её разрывы суть разрывы первого рода.\end{icloze}
\end{note}

\begin{note}{86f712cafa53498fa3b282915340635c}
    Пусть \({ f : [a, b] \to \mathbb R }\). Если \({ f }\) кусочно непрерывна на \({ [a, b] }\), то \begin{icloze}{1}\({ f \in \mathcal R[a, b] }\).\end{icloze}
\end{note}

\begin{note}{78e23f5a381e4d01b4c4467b16955dcb}
    Как показать, что кусочно непрерывная функция \({ f : [a, b] \to \mathbb R }\) интегрируема на \({ [a, b] }\)?

    \begin{cloze}{1}
        Показать, что она интегрируема на каждом из непрерывных <<кусков>>.
    \end{cloze}
\end{note}

\begin{note}{682bfc21e35a489ebc7df5514e1d4690}
    Пусть \({ E \subset \mathbb R }\). Говорят, что \begin{icloze}{2}множество \({ E }\) имеет нулевую меру,\end{icloze} если \begin{icloze}{1}для любого \({ \varepsilon > 0 }\) множество \({ E }\) можно заключить  в не более чем счётное объединение интервалов, суммарная длина которых меньше \({ \varepsilon }\).\end{icloze}
\end{note}

\begin{note}{3be97c537ee44236bfc5f8bae48390e3}
    \subsubsection{<<\begin{icloze}{3}Критерий\end{icloze} \begin{icloze}{4}Лебега\end{icloze} \begin{icloze}{2}интегрируемости функции\end{icloze}>>}

    Пусть \begin{icloze}{5}\({ f : [a, b]  \to \mathbb R }\).\end{icloze}
    Тогда \begin{icloze}{2}\({ f \in \mathcal R[a, b] }\)\end{icloze} \begin{icloze}{3}тогда и только тогда, когда\end{icloze} \begin{icloze}{1}\({ f }\) ограничена на \({ [a, b] }\) и множество точек  разрыва \({ f }\) имеет нулевую меру.\end{icloze}
\end{note}

\begin{note}{1b3f6df593ba44b6b9558ee83720dd7e}
    Пусть \({ f, g \in \mathcal R[a, b], \alpha \in \mathbb R }\). Тогда
    \[
        \begin{icloze}{1}f + g\end{icloze},\: \begin{icloze}{1}fg,\end{icloze}\: \begin{icloze}{1}\alpha f\end{icloze},\: \begin{icloze}{1}\left\lvert f \right\rvert\end{icloze} \in \mathcal R[a, b].
    \]
\end{note}

\begin{note}{275d09f8b4c9445992ecc4f02cd9f6ba}
    Пусть \({ f, g \in \mathcal R[a, b] }\). Тогда
    \[
        \begin{icloze}{2}\underset{x \in [a, b]}{\inf} \left\lvert g(x) \right\rvert > 0\end{icloze}
        \implies
        \begin{icloze}{1}
                \frac{f}{g} \in \mathcal R[a, b].
        \end{icloze}
    \]
\end{note}

\begin{note}{20347e63d70945cdbe073a675dbcb23a}
    Пусть \({ f, g \in \mathcal R[a, b] }\). Тогда \({ f + g \in \mathcal R[a, b] }\).

    В чем основная идея доказательства?

    \begin{cloze}{1}
        Тривиально следует из определения предела интегральных сумм в терминах последовательностей.
    \end{cloze}
\end{note}

\begin{note}{909d22f843c9421598278c307e29edb5}
    Пусть \({ f, g \in \mathcal R[a, b] }\). Тогда \({ fg \in \mathcal R[a, b] }\).

    В чем основная идея доказательства?

    \begin{cloze}{1}
        Дать верхнюю оценку для \({ \omega_k(f \cdot g) }\) через \({ \omega_k(f), \omega_k(g) }\) и верхние границы \({ f }\) и \({ g }\).
    \end{cloze}
\end{note}

\begin{note}{aad31d4599b94b8f8ee128bfbfec8251}
    Пусть \({ f \in \mathcal R[a, b], \alpha \in \mathbb R }\). Тогда \({ \alpha f \in \mathcal R[a, b] }\).

    В чем основная идея доказательства?

    \begin{cloze}{1}
        Частный случай произведения двух функций.
    \end{cloze}
\end{note}

\begin{note}{76538508e5574126a26e4dbf06fe4160}
    Пусть \({ f \in \mathcal R[a, b] }\). Тогда \({ \left\lvert f \right\rvert \in \mathcal R[a, b] }\).

    В чем основная идея доказательства?

    \begin{cloze}{1}
        \[
            \left\lvert f \right\rvert = f \cdot \operatorname{sgn} f \in \mathcal R[a, b].
        \]
    \end{cloze}
\end{note}

\begin{note}{dea0bdf999ae4306b554167c63eeb231}
    Как показать, что \({ \operatorname{sgn} }\) интегрируем?

    \begin{cloze}{1}
        Показать, что \({ \operatorname{sng} }\) кусочно непрерывен.
    \end{cloze}
\end{note}

\begin{note}{6bb4953a2e6d41fb86cf8a801779a97f}
    Пусть \({ f, g \in \mathcal R[a, b] }\). Тогда
    \[
        \underset{x \in [a, b]}{\inf} \left\lvert g(x) \right\rvert > 0 \implies \frac{f}{g} \in \mathcal R[a, b].
    \]

    В чем основная идея доказательства?

    \begin{cloze}{1}
        Представить \({ \frac{f}{g} }\) как произведение функций \({ f \cdot \frac{1}{g} \in \mathcal R[a, b] }\).
    \end{cloze}
\end{note}

\begin{note}{8dcb53c3642547a5bec77126b490908d}
    Пусть \({ f \in \mathcal R[a, b] }\). Тогда
    \[
        \underset{x \in [a, b]}{\inf} \left\lvert f(x) \right\rvert > 0 \implies \frac{1}{f} \in \mathcal R[a, b].
    \]

    В чем основная идея доказательства?

    \begin{cloze}{1}
        Оценить \({ \omega_k( 1 / f ) }\) сверху через \({ \omega_k(f) }\) и \({ \underset{x \in [a, b]}{\inf} \left\lvert f(x) \right\rvert     }\).
    \end{cloze}
\end{note}

\section{Лекция 04.04.22}
\begin{note}{40ffb14c933540e0a82a0f491c2ea946}
    Интегрируема ли функция Дирихле \({ \chi }\) на произвольном невырожденном отрезке?

    \begin{cloze}{1}
        Нет.
    \end{cloze}
\end{note}

\begin{note}{488460418bc246fe90907796c4db58be}
    Пусть \({ [a, b]  }\) --- невырожденный отрезок, \({ \chi }\) --- функция Дирихле.
    Как показать, что \({ \chi \not\in \mathcal R[a, b] }\)?

    \begin{cloze}{1}
        \({ \omega(\chi|_{[\alpha, \beta]}) = 1 }\) для любого отрезка \({ [\alpha, \beta] \subset [a, b] }\).
    \end{cloze}
\end{note}

\begin{note}{fdad8aea720d4949805e6d411e4d9cd0}
    Интегрируема ли функция Римана \({ \psi }\) на произвольном промежутке \({ [a, b] }\)?

    \begin{cloze}{1}
        Да.
    \end{cloze}
\end{note}

\begin{note}{c7099fd03a894c53b8c80146045e9127}
    Пусть \({ [a, b]  }\) --- невырожденный отрезок, \({ \psi }\) --- функция Римана.
    \[
        \int_{a}^{b} \psi = \begin{icloze}{1}0.\end{icloze}
    \]
\end{note}

\begin{note}{6350173ae50b44d8ac481bc4e58df52a}
    Пусть \({ [a, b]  }\) --- невырожденный отрезок, \({ \psi }\) --- функция Римана.
    В чём ключевая идея доказательства того, что \({ \psi \in \mathcal R[a, b] }\)?

    \begin{cloze}{1}
        Показать, что множество
        \[
            A = \left\{ \frac{p}{q} \in \mathbb Q \cap [a, b] \mid q \leqslant N \right\}
        \]
        конечно.
    \end{cloze}
\end{note}

\begin{note}{83b18cc44cff41fc8f13f721bb95887f}
    Пусть \({ [a, b]  }\) --- невырожденный отрезок, \({ \psi }\) --- функция Римана.
    Как выбирается \({ N }\) в доказательстве того, что \({ \psi \in \mathcal R[a, b] }\)?

    \begin{cloze}{1}
        Так, что \({ \frac{1}{N} < \varepsilon }\).
    \end{cloze}
\end{note}

\begin{note}{e897ef9db02f489f8f78e18c71f5ee40}
    Пусть \({ [a, b]  }\) --- невырожденный отрезок, \({ \psi }\) --- функция Римана.
    Как выбирается \({ \delta }\) в доказательстве того, что \({ \psi \in \mathcal R[a, b] }\)?

    \begin{cloze}{1}
        \[
            \delta = \frac{\varepsilon}{\left\lvert A \right\rvert}.
        \]
    \end{cloze}
\end{note}

\begin{note}{aa4eaee1713d444292a6cdb20f5d2bed}
    Пусть \({ [a, b]  }\) --- невырожденный отрезок, \({ \psi }\) --- функция Римана.
    Какой критерий интегрируемости используется в доказательстве того, что \({ \psi \in \mathcal R[a, b] }\)?

    \begin{cloze}{1}
        Критерий в терминах \({ S_\tau - s_\tau }\).
    \end{cloze}
\end{note}

\begin{note}{ac79fc72aeb440e9a666f8dc2433dcc8}
    Пусть \begin{icloze}{3}\({ f : [a, b] \to \mathbb R }\), \({ g : [c, d] \to [a, b] }\).\end{icloze}
    Тогда
    \[
        \begin{icloze}{2}f \in C[a, b],\:  g \in \mathcal R[c, d]\end{icloze} \implies \begin{icloze}{1}f \circ g \in \mathcal R[c, d].\end{icloze}
    \]
\end{note}

\begin{note}{bc11b9f7a4304f74b4bb3118d30cea22}
    Пусть \begin{icloze}{3}\({ a > b }\), \({ f \in \mathcal R[b, a] }\).\end{icloze} Тогда
    \[
        \begin{icloze}{2}\int_{a}^{b} f\end{icloze} \overset{\text{def}}= \begin{icloze}{1}-\int_{b}^{a} f.\end{icloze}
    \]
\end{note}

\begin{note}{2be98235b82942f7bf08141dd983fe07}
    \[
        \int_{a}^{a} f \overset{\text{def}}= \begin{icloze}{1}0.\end{icloze}
    \]
\end{note}

\begin{note}{649acec0ca304bd5bc72134401215095}
    Пусть \({ f : [a, a] \to \mathbb R }\). Тогда
    \[
        f \in \mathcal R[a, a] \overset{\hspace{-1pt}\text{\tiny def}}\iff \begin{icloze}{1}\top.\end{icloze}
    \]
\end{note}

\begin{note}{cc427e206f4d435889e87acd827bc5b4}
    Пусть \({ f \in \mathcal R[a, b] }\),  \({ c \in (a, b) }\).
    Тогда
    \[
        \begin{icloze}{2}\int_{a}^{b} f\end{icloze} = \begin{icloze}{1}\int_{a}^{c} f + \int_{c}^{b} f.\end{icloze}
    \]
\end{note}

\begin{note}{f57dd9f306a5429bb89c68b128c0e01f}
    Пусть \({ f \in \mathcal R[a, b] }\), \({ \alpha \in \mathbb R }\).
    Тогда
    \[
        \begin{icloze}{2}\int_{a}^{b} \alpha f\end{icloze} = \begin{icloze}{1}\alpha \int_{a}^{b} f.\end{icloze}
    \]
\end{note}

\begin{note}{5aaf74ed1c3e414cb5b7c501e5970206}
    Пусть \({ f, g \in \mathcal R[a, b] }\).
    Тогда
    \[
        \begin{icloze}{2}\int_{a}^{b} (f \pm g)\end{icloze} = \begin{icloze}{1}\int_{a}^{b} f \pm \int_{a}^{b} g.\end{icloze}
    \]
\end{note}

\begin{note}{d43171cbe636478098887433ff64ea61}
    Откуда следует линейность интеграла Римана?

    \begin{cloze}{1}
        Из определения в терминах последовательностей.
    \end{cloze}
\end{note}

\begin{note}{8db1f43273d041d6b3fbc674270d4f5b}
    Пусть \({ f \in \mathcal R[a, b] }\).
    Тогда
    \[
        \begin{icloze}{2}f \geqslant 0\end{icloze} \implies \begin{icloze}{1}\int_{a}^{b} f \geqslant 0.\end{icloze}
    \]
\end{note}

\begin{note}{9778c1f4a58e4df9a9a25064935a647f}
    Пусть \({ f, g \in \mathcal R[a, b] }\).
    Тогда
    \[
        \begin{icloze}{2}f \leqslant g\end{icloze} \implies \begin{icloze}{1}\int_{a}^{b} f \leqslant \int_{a}^{b} g.\end{icloze}
    \]
\end{note}

\begin{note}{9ba16cbdf8fb4b65bd032cb56c483a1b}
    Пусть \({ f \in \mathcal R[a, b] }\).
    Тогда
    \[
        \begin{icloze}{2}\left\lvert \int_{a}^{b} f \right\rvert\end{icloze} \begin{icloze}{1}\leqslant \int_{a}^{b} \left\lvert f \right\rvert.\end{icloze}
    \]
\end{note}

\begin{note}{275d3bae3cfa48e9882d2d2a90ed21b8}
    Пусть \({ f \in \mathcal R[a, b] }\).
    Тогда
    \[
        \begin{icloze}{2}\left\lvert f \right\rvert \leqslant M \in \mathbb R\end{icloze} \implies \begin{icloze}{1}\left\lvert \int_{a}^{b} f \right\rvert \leqslant M(b - a).\end{icloze}
    \]
\end{note}

\begin{note}{84dfee5723b34bec8f05c42879c3e85f}
    Пусть \({ f \in C[a, b] }\).
    Тогда
    \[
        \begin{icloze}{2}f \geqslant 0\end{icloze} \land \int_{a}^{b} f = 0 \implies \begin{icloze}{1}f \equiv 0.\end{icloze}
    \]
\end{note}

\begin{note}{2f438df6b5304cd7acbd273214875635}
    Пусть \({ f \in C[a, b] }\).
    Тогда
    \[
        f \geqslant 0 \land \int_{a}^{b} f = 0 \implies f \equiv 0.
    \]

    В чем основная идея доказательства?

    \begin{cloze}{1}
        От противного; допустить, что \({ \exists x_0 : f(x_0) > 0 }\) и использовать то, что \({ \exists \delta : f|_{V_{\delta}(x_0)} > \frac{f(x_0)}{2} }\).
    \end{cloze}
\end{note}

\begin{note}{03b2fb7149444b2db71b150b49f267a9}
    Пусть \begin{icloze}{3}\({ f \in \mathcal R[a, b] }\) непрерывна в точке \({ x_0 \in [a, b] }\),\end{icloze}
    \[
        \varphi(x) = \begin{icloze}{2}\int_{a}^{x} f.\end{icloze}
    \]
    Тогда \begin{icloze}{1}\({ \varphi }\) дифференцируема в точке \({ x_0 }\) и \({ \varphi'(x_0) = f(x_0) }\).\end{icloze}

    \begin{center}
        \tiny
        <<\begin{icloze}{4}Теорема Барроу\end{icloze}>>
    \end{center}
\end{note}

\begin{note}{2255bf6318be43ee834d6071ed740c89}
    В чем основная идея доказательства теоремы Барроу?

    \begin{cloze}{1}
        Оценить разность \({ \varphi(x_0 + h) - \varphi(x_0) }\) и получить дифференцируемость \({ \varphi }\) по определению.
    \end{cloze}
\end{note}

\begin{note}{ec57d94464cc4ecf8920954c5d4cbf76}
    Как в доказательстве теоремы Барроу непосредственно используется непрерывность \({ f }\)?

    \begin{cloze}{1}
        Для представления \({ f(x) }\) как \({ f(x_0) + \Delta x }\), где \({ \Delta x \underset{x \to x_0}\longrightarrow 0 }\).
    \end{cloze}
\end{note}

\begin{note}{d26fa11d0b1e4d83bb4f4b224b9359a4}
    Почему в доказательстве теоремы Барроу
    \[
        \Delta  x \underset{x \to x_0}\longrightarrow 0?
    \]

    \begin{cloze}{1}
        \[
            \Delta x = f(x) - f(x_0) \to 0.
        \]
    \end{cloze}
\end{note}

\begin{note}{cdc6a92b44fd4d488ca3b30c5e2b4232}
    В доказательстве теоремы Барроу
    \[
        \varphi(x_0 + h) - \varphi(x_0) = \begin{icloze}{1}\int_{x_0}^{x_0 + h} f(x)\: dx.\end{icloze}
    \]
\end{note}

\begin{note}{c5dc36411f3a45598a70c9522a651022}
    В доказательстве теоремы Барроу
    \[
        \int_{x_0}^{x_0 + h} f(x_0)\: dx = \begin{icloze}{1}f(x_0) h.\end{icloze}
    \]
\end{note}

\begin{note}{1cea51f7fd9b4b438856a6ec7f4c1a48}
    В доказательстве  теоремы  Барроу
    \[
        \int_{x_0}^{x_0 + h} \Delta x\: dx = \begin{icloze}{1}o(h).\end{icloze}
    \]
\end{note}

\begin{note}{455761e28fe94191b0bebcfa38fc5b89}
    Откуда в доказательстве теоремы Барроу следует, что
    \[
        \int_{x_0}^{x_0 + h} \Delta x\: dx = o(h)?
    \]

    \begin{cloze}{1}
        \[
            \Delta x \underset{x \to x_0}\longrightarrow 0 \overset{\hspace{-1pt}\text{\tiny def}}\iff \dots \ \left\lvert \Delta x \right\rvert <  \varepsilon.
        \]
    \end{cloze}
\end{note}

\begin{note}{1109a435ae7543cf8263f72942677e95}
    Пусть \begin{icloze}{2}\({ f, g \in \mathcal R[a, b] }\), \({ g \geqslant 0 }\) (или \({ g \leqslant 0 }\)), \({ m \leqslant f \leqslant M }\).\end{icloze}
    Тогда
    \[
        \begin{icloze}{4}\exists \mu \in [m, M]\end{icloze} \quad \begin{icloze}{1}\int_{a}^{b} fg = \mu \int_{a}^{b} g\end{icloze}
    \]

    \begin{center}
        \tiny
        <<\begin{icloze}{3}Первая теорема о среднем \\\phantom{<<}интегрального исчисления\end{icloze}>>
    \end{center}
\end{note}

\begin{note}{2e6a92e370ed4445a1cd67bd8d0241f9}
    В чем основная идея доказательства первой теоремы о среднем интегрального исчисления?

    \begin{cloze}{1}
        Проинтегрировать все части неравенства
        \[
            mg \leqslant fg \leqslant Mg \quad \text{(для \({ g \geqslant 0 }\))}.
        \]
    \end{cloze}
\end{note}

\begin{note}{bede3b8a7d14462fbd5fcb32d222eb2d}
    Чему равно \({ \mu }\) из первой теоремы о среднем интегрального исчисления (\({ \int_{a}^{b} g = 0 }\))?

    \begin{cloze}{1}
        \({ \mu }\) --- произвольное значение из \({ [m, M] }\).
    \end{cloze}
\end{note}

\begin{note}{9137b877b68a4313b09fc0b2e748f6a4}
    Чему равно \({ \mu }\) из первой теоремы о среднем интегрального исчисления (\({ \int_{a}^{b} g \neq 0 }\))?

    \begin{cloze}{1}
        \[
            \mu = \frac{\int_{a}^{b} fg}{\int_{a}^{b} g}.
        \]
    \end{cloze}
\end{note}

\section{Лекция 08.04.22}
\begin{note}{156407f3795145ac967eccf82e541fe0}
    Пусть \begin{icloze}{2}\({ f \in C[a, b] }\), \({ g \in \mathcal R[a, b] }\), \({ g \geqslant 0 }\) (или \({ g \leqslant 0 }\)).\end{icloze}
    Тогда
    \begin{icloze}{1}
        \[
            \exists c \in [a, b] \quad \int_{a}^{b} fg = f(c) \int_{a}^{b} g.
        \]
    \end{icloze}

    \begin{center}
        \tiny
        (следствие из \begin{icloze}{3}первой теоремы о среднем\end{icloze})
    \end{center}
\end{note}

\begin{note}{ba06ad7c7c3c453fa47427d955b922bb}
    Пусть \begin{icloze}{2}\({ f \in R[a, b] }\), \({ m \leqslant f \leqslant M }\).\end{icloze}
    Тогда
    \begin{icloze}{1}
        \[
            \exists \mu \in [m, M] \quad \int_{a}^{b} f = \mu(b - a)
        \]
    \end{icloze}

    \begin{center}
        \tiny
        (следствие из \begin{icloze}{3}первой теоремы о среднем\end{icloze})
    \end{center}
\end{note}

\begin{note}{ba06ad7c7c3c453fa47427d955b922bb}
    Пусть \begin{icloze}{2}\({ f \in C[a, b] }\).\end{icloze}
    Тогда
    \begin{icloze}{1}
        \[
            \exists c \in [a, b] \quad \int_{a}^{b} f = f(c) \cdot (b - a)
        \]
    \end{icloze}

    \begin{center}
        \tiny
        (следствие из \begin{icloze}{3}первой теоремы о среднем\end{icloze})
    \end{center}
\end{note}

\begin{note}{df108f6ef527491996f1a2b6672c17fc}
    \subsubsection{<<\begin{icloze}{3}Формула Ньютона-Лейбница\end{icloze}>>}

    Пусть \begin{icloze}{2}\({ f \in \mathcal R[a, b] }\), \({ F \in \mathscr P_f ([a, b]) }\).\end{icloze}
    Тогда
    \begin{icloze}{1}
        \[
            \int_{a}^{b} f = F(b) - F(a).
        \]
    \end{icloze}
\end{note}

\begin{note}{bddc676c2daa42f0b218c91eac244ace}
    В чем основная идея доказательства формулы Ньютона-Лейбница?

    \begin{cloze}{1}
        Выбрать нужное оснащение используя формулу конечных приращений.
    \end{cloze}
\end{note}

\begin{note}{882e212c55614bb78548d3b6b7f2fbf2}
    Пусть \({ f : [a, b] \to \mathbb R }\). Тогда
    \begin{icloze}{1}
        разность
        \[
            f(b) - f(a)
        \]
    \end{icloze}
    называется \begin{icloze}{2}двойной подстановкой функции \({ f }\)  на \({ [a, b] }\).\end{icloze}
\end{note}

\begin{note}{b7e1b154b8de47bfbc3d2afdf73ef75c}
    Пусть \({ f : [a, b] \to \mathbb R }\).
    \begin{icloze}{1}Двойная постановка функции \({ f }\) на \({ [a, b] }\)\end{icloze} обозначается
    \begin{icloze}{2}
        \[
            f \big|_{a}^{b},\: f(x) \big|_{a}^{b},\: f(x) \big|_{x = a}^{b}
        \]
    \end{icloze}
\end{note}

\begin{note}{df6dee2f53354eb8ae50cd3a673878a0}
    Пусть \begin{icloze}{3}\({ f : [a, b] \to \mathbb R }\) дифференцируема на \({ [a, b] }\), \({ f' \in \mathcal R[a, b] }\).\end{icloze} Тогда
    \[
        \begin{icloze}{2}\int_{a}^{b} f'\end{icloze} = \begin{icloze}{1}f \big|_{a}^{b}.\end{icloze}
    \]
\end{note}

\begin{note}{1146663e8ded4d63ba85b1b1ed35cb14}
    Пусть \({ f \in \mathcal R[a, b] }\), \({ F \in C[a, b] }\), \({ F }\) --- первообразная \({ f }\) за исключением конечного числа точек.
    Тогда
    \begin{icloze}{1}
        \[
            \int_{a}^{b} f = F(b) - F(a).
        \]
    \end{icloze}

    \begin{center}
        \tiny
        (обобщение формулы Ньютона-Лейбница)
    \end{center}
\end{note}

\begin{note}{496225bfd55a484fa6f61e7174af39f4}
    В чём основная идея доказательства обобщения формулы Ньютона-Лейбница для \({ F \in \mathscr P_f ([a, b] \setminus T) }\),\: \({ |T| < \aleph_0 }\).

    \begin{cloze}{1}
        Разбить \({ [a, b] }\) на отрезки, во всех внутренних точках которых \({ F' = f }\).
    \end{cloze}
\end{note}

\begin{note}{6abc693b3b104008b356cbca06692bdd}
    Пусть \({ f : \mathcal R[a, b] }\), \({ F \in C[a, b] }\), \({ F'|_{(a, b)} = f|_{(a, b)} }\).
    Как показать, что \({ \int_{a}^{b} f = F\big|_{a}^{b} }\)?

    \begin{cloze}{1}
        \[
            \int_{a}^{b} f  = \lim_{\varepsilon \to 0^{+}} \int_{a + \varepsilon}^{b - \varepsilon} f.
        \]
    \end{cloze}
\end{note}

\begin{note}{a8b4457947d3495c9b5fca5d72fdae28}
    Пусть \({ f : \mathcal R[a, b] }\). Как показать, что
    \[
        \int_{a}^{b} f = \lim_{\varepsilon \to 0^{+}} \int_{a + \varepsilon}^{b - \varepsilon} f?
    \]

    \begin{cloze}{1}
        Показать, что их разность стремится к нулю.
    \end{cloze}
\end{note}

\begin{note}{8b853a07bfa94f17a0a6e43bd75e8226}
    Пусть \({ f : \mathcal R[a, b] }\). Как показать, что
    \[
        \int_{a}^{a + \varepsilon} f \underset{\varepsilon \to 0^{+}}\longrightarrow 0?
    \]

    \begin{cloze}{1}
        \[
            m\varepsilon \leqslant \int_{a}^{a + \varepsilon} f \leqslant M\varepsilon.
        \]
    \end{cloze}
\end{note}

\begin{note}{6e131dfc6d004e86a827f88367981953}
    Пусть \({ f : [a, b] \to \mathbb R }\).
    Какова, в общем случае, зависимость между интегрируемостью \({ f }\) и существованием у неё первообразной?

    \begin{cloze}{1}
        В общем случае прямой зависимости нет.
    \end{cloze}
\end{note}

\begin{note}{dcb24ef22e6c4921924ad580a30722c6}
    \subsubsection{<<\begin{icloze}{3}Интегрирование по частям \\\phantom{<<}для определённого интеграла\end{icloze}>>}

    Пусть \begin{icloze}{2}\({ f, g }\) дифференцируемы на \({ [a, b] }\),\: \({ f', g' \in \mathcal R[a,  b] }\).\end{icloze}
    Тогда
    \begin{icloze}{1}
        \[
            \int_{a}^{b} fg' = fg \big|_{a}^{b} - \int_{a}^{b} f'g.
        \]
    \end{icloze}
\end{note}

\begin{note}{a938ca4e00ae4fffaf1db99f384e1785}
    В чем основная идея доказательства формулы интегрирования по частям для определённого интеграла?

    \begin{cloze}{1}
        \[
            \begin{gathered}
                (fg)' = f g' + f' g \in \mathcal R[a, b], \\
                \int_{a}^{b} (fg)' = fg \big|_{a}^{b}.
            \end{gathered}
        \]
    \end{cloze}
\end{note}

\begin{note}{46828f8f846b4de1a73aed39d5b9d7a0}
    \subsubsection{<<\begin{icloze}{3}Замена переменной в определённом интеграле\end{icloze}>>}

    Пусть \begin{icloze}{2}\({ \varphi : [\alpha, \beta] \to [a, b] }\), \({ \varphi }\) дифференцируема на \({ [\alpha, \beta] }\), \({ \varphi' \in \mathcal R[\alpha, \beta] }\), \({ f \in C[a, b] }\).\end{icloze}
    Тогда
    \begin{icloze}{1}
        \[
            \int_{\alpha}^{\beta} (f \circ \varphi) \varphi' = \int_{\varphi(\alpha)}^{\varphi(\beta)} f
        \]
    \end{icloze}
\end{note}

\begin{note}{7d8af4366130490fbb87408d659837e7}
    В чем основная идея доказательства теоремы о замене переменной в определённом интеграле?

    \begin{cloze}{1}
        Формула Ньютона-Лейбница для \({ \int_{\alpha}^{\beta} (f \circ \varphi) \varphi' }\).
    \end{cloze}
\end{note}

\begin{note}{89c18701291a4123afd5111e00b0e9c8}
    Как преобразуется \({ F \circ \varphi \big|_{\alpha}^{\beta} }\) в доказательстве теоремы о замене переменной в определённом интеграле?

    \begin{cloze}{1}
        \[
            F \circ \varphi \big|_{\alpha}^{\beta} = F \big|_{\varphi(\alpha)}^{\varphi(\beta)}
        \]
    \end{cloze}
\end{note}

\begin{note}{b6aa7eee0f404285b693fb392e8ea744}
    Пусть \begin{icloze}{2}\({ f \in \mathcal R[-a, a] }\), \({ f }\) --- чётна.\end{icloze}
    Тогда
    \[
        \begin{gathered}
            \int_{-a}^{a} f = \begin{icloze}{1}2\int_{0}^{a} f.\end{icloze}
        \end{gathered}
    \]
\end{note}

\begin{note}{53ccd4a61cc742818562b58b9bdce5b4}
    Пусть \begin{icloze}{2}\({ f \in \mathcal R[-a, a] }\), \({ f }\) --- нечётна.\end{icloze}
    Тогда
    \[
        \begin{gathered}
            \int_{-a}^{a} f = \begin{icloze}{1}0.\end{icloze}
        \end{gathered}
    \]
\end{note}

\section{Семинар 31.03.22}
\begin{note}{f6c10d5634604b8785e6d50a3b4179bb}
    Пусть \begin{icloze}{3}\({ y = \frac{ax + b}{cx + d} \in \mathbb R(x) \setminus \mathbb R }\),\: \({ p_1, p_2, \ldots, p_n \in \mathbb Q }\).\end{icloze}
    Тогда интеграл вида
    \[
        \int \begin{icloze}{2}R(x, y^{p_1}, y^{p_2}, \ldots, y^{p_n})\end{icloze}\: dx
    \]
    берётся заменой \begin{icloze}{1}\({ t^{N} = y }\), где \({ N }\) --- общий знаменатель дробей \({ p_1, \ldots, p_n }\).\end{icloze}
\end{note}

\begin{note}{f28ac103f505434d9a7caa2eb9756579}
    \begin{icloze}{2}Дифференциальным биномом\end{icloze} называется \begin{icloze}{1}дифференциал вида
    \[
        \begin{gathered}
            x^{m} (a + b x^{n})^{p}\: dx, \\
            a, b \in \mathbb R, \quad m, n, p \in \mathbb Q.
        \end{gathered}
    \]\end{icloze}
\end{note}

\begin{note}{6d980687cd794cb2b67482197892cb24}
    Каковы условия для применения каждой из подстановок применяемых для взятия интеграла от дифференциального бинома?

    \begin{cloze}{1}
        \[
            p \in \mathbb Z; \quad
            \frac{m + 1}{n} \in \mathbb Z; \quad
            \frac{m + 1}{n} + p \in \mathbb Z.
        \]
    \end{cloze}
\end{note}

\begin{note}{be2650fc8a6c4b88b1f39cd872a556b1}
    Какая подстановка используется для взятия интеграла от дифференциального бинома (случай \({ p \in \mathbb Z }\))?

    \begin{cloze}{1}
        \({ t^{N} = x }\), где \({ N }\) --- общий знаменатель \({ m }\) и \({ n }\).
    \end{cloze}
\end{note}

\begin{note}{bcc008ea2e804fcc8d3df70bfe88cdc1}
    Какая подстановка используется для взятия интеграла от дифференциального бинома (случай \({ \frac{m + 1}{n} \in \mathbb Z }\))?

    \begin{cloze}{1}
        \({ t^{k} = a + bx^{n} }\), где \({ k }\) --- знаменатель \({ p }\).
    \end{cloze}
\end{note}

\begin{note}{90e931c9722e44219fe1e08170da8e53}
    Какая подстановка используется для взятия интеграла от дифференциального бинома (случай \({ \frac{m + 1}{n} + p \in \mathbb Z }\))?

    \begin{cloze}{1}
        \({ t^{k} = ax^{-n} + b }\), где \({ k }\) --- знаменатель \({ p }\).
    \end{cloze}
\end{note}

\section{Лекция 23.04.22 (1)}
\begin{note}{57c65a69d4ce4beca2a4ef3d19482330}
    Пусть \begin{icloze}{3}\({ f \in C^{n + 1}\langle A, B \rangle }\), \({ a, x \in \langle A, B \rangle }\),\end{icloze}\: \({ n \in \begin{icloze}{4}\mathbb Z_+\end{icloze} }\).
    Тогда
    \[
        \begin{icloze}{2}R_{a,n}f(x)\end{icloze} = \begin{icloze}{1}\frac{1}{n!} \int_{a}^{x} f^{(n + 1)}(t)(x - t)^{n}\: dt.\end{icloze}
    \]

    \begin{center}
        \tiny
        (в интегральной форме)
    \end{center}
\end{note}

\begin{note}{0a72287655664b3e99f6388731f26bfe}
    Какой метод доказательства используется в доказательстве формулы Тейлора с остатком в интегральной форме?

    \begin{cloze}{1}
        Индукция по \({ n }\).
    \end{cloze}
\end{note}

\begin{note}{82254691571e452294750327ef22eb89}
    В чём основная в доказательстве формулы Тейлора с остатком в интегральной форме (базовый случай)?

    \begin{cloze}{1}
        При \({ n = 0 }\) получаем формулу Ньютона-Лейбница.
    \end{cloze}
\end{note}

\begin{note}{4a33a11f8ade46f08ce296db479a9007}
    В чём основная в доказательстве формулы Тейлора с остатком в интегральной форме (индукционный переход)?

    \begin{cloze}{1}
        Применить к остатку формулу интегрирования по частям.
    \end{cloze}
\end{note}

\begin{note}{aad636315afd4ee5a6744160d4b3e42d}
    \begin{icloze}{2}Интегральную форму\end{icloze} остатка \({ R_{a,n}f(x) }\) иногда называют \begin{icloze}{1}формой Якоби.\end{icloze}
\end{note}

\begin{note}{397b0796dce04587adb64efce1dd1140}
    \[
        0!! \overset{\text{def}}= \begin{icloze}{1}1.\end{icloze}
    \]
\end{note}

\begin{note}{cac7fe0bc23d40a59bc9dd6812577d01}
    \[
        (-1)!! \overset{\text{def}}= \begin{icloze}{1}1.\end{icloze}
    \]
\end{note}

\begin{note}{7da7dd0bd19944abbe2d01d155086010}
    \subsubsection{<<\begin{icloze}{2}Формула Валлиса\end{icloze}>>}

    \begin{icloze}{1}
        \[
            \pi = \lim_{n \to \infty} \frac{1}{n} \left( \frac{(2n)!!}{(2n - 1)!!} \right)^2
        \]
    \end{icloze}
\end{note}

\begin{note}{f54929de1ece44289c241d6590700ab5}
    В чём  основная идея доказательства формулы Валлиса?

    \begin{cloze}{1}
        Проинтегрировать на \({ [0, \frac{\pi}{2}] }\) неравенство
        \[
            \sin^{2n + 1} \leqslant  \sin^{2n} \leqslant \sin^{2n - 1}.
        \]
    \end{cloze}
\end{note}

\begin{note}{b5c778232bc94244b23a8e4a617ed0e6}
    Пусть \begin{icloze}{3}\({ m \in \mathbb Z_+ }\).\end{icloze}
    \[
        \begin{icloze}{2}\int_{0}^{\frac{\pi}{2}} \sin^{m} x\: dx\end{icloze}
        =
        \begin{icloze}{1}
            \frac{(m - 1)!!}{m!!} \cdot \begin{cases}
                \frac{\pi}{2}, & \text{\({ m }\) чётно}, \\
                1, & \text{\({ m }\) нечётно}.
            \end{cases}
        \end{icloze}
    \]
\end{note}

\begin{note}{715b1cec8b8041949185220e4e40a41b}
    В чём основная идея в доказательстве явной формулы для
    \[
        \int_{0}^{\frac{\pi}{2}} \sin^{m} x\: dx?
    \]

    \begin{cloze}{1}
        Индукция по \({ m }\) и формула интегрирования по частям.
    \end{cloze}
\end{note}

\begin{note}{38d390969b2741acbfc82267d2fe35fd}
    \subsubsection{<<\begin{icloze}{5}Вторая  теорема о среднем \\ \phantom{<<}интегрального исчисления\end{icloze}>>}

    Пусть \begin{icloze}{4}\({ f \in C[a, b] }\), \({ g \in C^{1}[a, b] }\), \({ g }\) монотонна на \({ [a, b] }\).\end{icloze} Тогда
    \[
        \begin{icloze}{3} \exists c \in [a,  b]\end{icloze} \quad \begin{icloze}{2}\int_{a}^{b} fg\end{icloze} = \begin{icloze}{1}g(a) \int_{a}^{c} f + g(b) \int_{c}^{b} f.\end{icloze}
    \]
\end{note}

\begin{note}{69afab3fc5004157a7f35af02f3e3d12}
    \begin{icloze}{2}Вторую теорему о среднем интегрального исчисления\end{icloze} так же называют \begin{icloze}{1}теоремой Бонне.\end{icloze}
\end{note}

\begin{note}{f0f631536a7f4b96b0398f89ecaf9118}
    Каков первый шаг в доказательстве теоремы Бонне?

    \begin{cloze}{1}
        Представить \({ \int_{a}^{b} fg }\) как \({ \int_{a}^{b} F'g }\), где \({ F(x) \coloneqq \int_{a}^{x} f }\).
    \end{cloze}
\end{note}

\begin{note}{2166048a6ca44d23af43256399b234c9}
    Какое преобразование применяется к интегралу \({ \int_{a}^{b} F' g }\) в доказательстве теоремы Бонне?

    \begin{cloze}{1}
        Интегрирование по частям.
    \end{cloze}
\end{note}

\begin{note}{3029dc6c343e4d1c9888bf539fba3503}
    Какое преобразование применяется к интегралу \({ \int_{a}^{b} Fg'? }\) в доказательстве теоремы Бонне?

    \begin{cloze}{1}
        Первая теорема о среднем.
    \end{cloze}
\end{note}

\begin{note}{441ce64036224c3ebf472a05f4829979}
    Почему в доказательстве теоремы Бонне мы можем применить первую теорему о среднем к интегралу \({ \int_{a}^{b} Fg'? }\)

    \begin{cloze}{1}
        По следствию из теоремы Дарбу \({ g' }\) не меняет знак на \({ [a, b] }\).
    \end{cloze}
\end{note}

\begin{note}{882ef3d613b947198a7bea4ad32d5160}
    Теорема Бонне \begin{icloze}{2}так же будет выполняться,\end{icloze} если ослабить предположение до:
    \begin{icloze}{1}
        \begin{center}
            \({ f \in \mathcal R[a, b] }\),\: \({ g }\) монотонна.
        \end{center}
    \end{icloze}

    \begin{center}
        \tiny
        (без доказательства)
    \end{center}
\end{note}

\section{Лекция 23.04.22 (2)}
\begin{note}{8b2c86d7991e4a2b91921581b1ef4490}
    \subsubsection{<<\begin{icloze}{4}Неравенство Йенсена для интеграла\end{icloze}>>}

    Пусть \begin{icloze}{3}\({ f \in C\langle A, B \rangle }\), \({ \varphi, \lambda \in C[a, b] }\), \({ \varphi : [a, b] \to \langle A, B \rangle }\), \({ \lambda \geqslant 0 }\).\end{icloze}
    Тогда если \begin{icloze}{2}\({ f }\) выпукла на \({ \langle A, B \rangle }\) и \({ \int_{a}^{b} \lambda = 1 }\),\end{icloze} то
    \begin{icloze}{1}
        \[
            f\left( \int_{a}^{b} \lambda\varphi \right) \leqslant \int_{a}^{b} \lambda \cdot (f \circ \varphi).
        \]
    \end{icloze}
\end{note}

\begin{note}{b1fef24356cb40989e268892f941d2bc}
    Какие два случая рассматриваются в доказательстве неравенства Йенсена для интеграла?

    \begin{cloze}{1}
        \begin{center}
            1.\: \({ \varphi =  const }\), \quad 2.\: \({ \varphi \neq const }\).
        \end{center}
    \end{cloze}
\end{note}

\begin{note}{0135fda5379a4582960532ada8cb52a9}
    В чём основная идея в доказательстве неравенства Йенсена (случай \({ \varphi = const }\))?

    \begin{cloze}{1}
       Доказываемое неравенство тривиальным образом обращается в равенство.
    \end{cloze}
\end{note}

\begin{note}{77790770b54541099c9e486f28074f26}
    В чём основная идея в доказательстве неравенства Йенсена (случай \({ \varphi \neq const }\))?

    \begin{cloze}{1}
        \({ \int_{a}^{b} \lambda\varphi \in (A, B) }\), а значит \({ f }\) имеет в этой точке опорную прямую.
    \end{cloze}
\end{note}

\begin{note}{402dc5f48e084c56acdbac1d8691cd46}
    Как в доказательстве неравенства Йенсена (случай \({ \varphi \neq const }\)) показать, что \({ \int_{a}^{b} \lambda \varphi \in (A, B) }\)?

    \begin{cloze}{1}
        \begin{center}
            \({ \int_{a}^{b} \lambda \varphi \in (\inf \varphi, \sup \varphi) \subset [A, B] }\).
        \end{center}
    \end{cloze}
\end{note}

\begin{note}{69bcf26030d34b71839c245e2f0b80ca}
    Как в доказательстве неравенства Йенсена (случай \({ \varphi \neq const }\)) показать, что \({ \int_{a}^{b} \lambda \varphi \neq \sup \varphi }\)?

    \begin{cloze}{1}
        От противного.
    \end{cloze}
\end{note}

\begin{note}{3584b635c4dc44e7b9b1624c4019860f}
    Если в неравенстве Йенсена для интеграла \begin{icloze}{2}\({ \varphi \neq const }\), а \({ f }\) строго выпукла,\end{icloze} то \begin{icloze}{1}имеет место строгое неравенство.\end{icloze}
\end{note}

\begin{note}{83892edbede849aeb7d89f4abc718e41}
    Пусть \begin{icloze}{3}\({ p, q \in (1, +\infty) }\).\end{icloze}
    \begin{icloze}{1}Числа \({ p }\) и \({ q }\)\end{icloze} называются \begin{icloze}{2}сопряжёнными показателями,\end{icloze} если
    \begin{icloze}{1}
        \[
            \frac{1}{p} + \frac{1}{q} = 1.
        \]
    \end{icloze}
\end{note}

\begin{note}{40347c04583b4c95b7cac9a0158ee490}
    \subsubsection{<<\begin{icloze}{3}Неравенство Гёльдера для сумм\end{icloze}>>}

    Пусть \begin{icloze}{4}\({ \left\{ a_i \right\}, \left\{ b_i \right\}_{i = 1}^{n} \subset \mathbb R_+ }\),\:\end{icloze} \begin{icloze}{2}\({ p }\) и \({ q }\) --- сопряжённые показатели.\end{icloze}
    Тогда
    \begin{icloze}{1}
        \[
            \sum_{i=1}^{n} a_i b_i \leqslant \left( \sum_{i=1}^{n} a_i^{p} \right)^{\frac{1}{p}} \left( \sum_{i=1}^{n} b_i^{q} \right)^{\frac{1}{q}}.
        \]
    \end{icloze}
\end{note}

\begin{note}{97e460d86c3c4db6b2267d79218280cd}
    \subsubsection{<<\begin{icloze}{3}Неравенство Гёльдера для интегралов\end{icloze}>>}

    Пусть \begin{icloze}{2}\({ f, g \in C[a, b] }\),\end{icloze}\: \begin{icloze}{4}\({ p }\) и \({ q }\) --- сопряжённые показатели.\end{icloze} Тогда
    \begin{icloze}{1}
        \[
            \left\lvert \int_{a}^{b} fg \right\rvert \leqslant \left( \int_{a}^{b} \left\lvert f \right\rvert^{p} \right)^{\frac{1}{p}} \left( \int_{a}^{b} \left\lvert g \right\rvert^{q} \right)^{\frac{1}{q}}.
        \]
    \end{icloze}
\end{note}

\begin{note}{fc914f6a007e42fc8512d6d3d277c03d}
    В чём основная идея в доказательстве неравенства Гёльдера для интегралов?

    \begin{cloze}{1}
        Применить неравенство Гёльдера для сумм к интегральным суммам.
    \end{cloze}
\end{note}

\begin{note}{e3416ff410fb47098739b40e48782ab7}
    Как представляется  \({ \Delta x_k }\) в доказательстве неравенства Гёльдера для интегралов?

    \begin{cloze}{1}
        \[
            \Delta x_k = (\Delta x_k)^{\frac{1}{p}} \cdot (\Delta x_k)^{\frac{1}{q}}.
        \]
    \end{cloze}
\end{note}

\begin{note}{89fc34798a62448d9c6bd5173c1ac247}
    \subsubsection{<<\begin{icloze}{3}Неравенство Коши-Буняковского для интегралов\end{icloze}>>}

    Пусть \begin{icloze}{2}\({ f, g \in C[a, b] }\).\end{icloze} Тогда
    \begin{icloze}{1}
        \[
            \left\lvert \int_{a}^{b} fg \right\rvert \leqslant \sqrt{\int_{a}^{b} f^2} \cdot \sqrt{\int_{a}^{b} g^2}.
        \]
    \end{icloze}
\end{note}

\begin{note}{72d317fe9af54aa4b717962551ba6ea0}
    \subsubsection{<<\begin{icloze}{3}Неравенство Минковского для сумм\end{icloze}>>}

    Пусть \begin{icloze}{2}\({ \left\{ a_i \right\}, \left\{ b_i \right\}_{i = 1}^{n} \subset \mathbb R }\),\end{icloze}\: \begin{icloze}{4}\({ p \geqslant 1 }\).\end{icloze}
    Тогда
    \begin{icloze}{1}
        \[
            \left( \sum_{i=1}^{n} \left\lvert a_i + b_i \right\rvert^{p} \right)^{\frac{1}{p}} \leqslant \left( \sum_{i=1}^{n} \left\lvert a_i \right\rvert^{p} \right)^{\frac{1}{p}} + \left( \sum_{i=1}^{n} \left\lvert b_i \right\rvert^{p} \right)^{\frac{1}{p}}.
        \]
    \end{icloze}
\end{note}

\begin{note}{19e57ef490ac48889603d288b4d8edd5}
    \subsubsection{<<\begin{icloze}{3}Неравенство Минковского для интегралов\end{icloze}>>}

    Пусть \begin{icloze}{2}\({ f, g \in  C[a, b] }\),\end{icloze} \begin{icloze}{4}\({ p \geqslant  1 }\).\end{icloze} Тогда
    \begin{icloze}{1}
        \[
            \left( \int_{a}^{b} \left\lvert f + g \right\rvert^{p} \right)^{\frac{1}{p}} \leqslant \left( \int_{a}^{b} \left\lvert f \right\rvert^{p} \right)^{\frac{1}{p}} + \left( \int_{a}^{b} \left\lvert g \right\rvert^{p} \right)^{\frac{1}{p}}.
        \]
    \end{icloze}
\end{note}

\begin{note}{e0ef8b3744ed4d93a9138b1d967a1792}
    В чём основная идея доказательства неравенства Минковского для интегралов?

    \begin{cloze}{1}
        Применить неравенство Минковского для сумм к интегральным суммам.
    \end{cloze}
\end{note}

\begin{note}{000b6517531b4ecc9a052a60b3a741a4}
    Функция \({ f : \langle A, B \rangle \to \mathbb R }\) называется \begin{icloze}{2}локально интегрируемой на \({ \langle A, B \rangle }\),\end{icloze} если
    \begin{icloze}{1}
        \[
            \forall [a, b] \subset \langle A, B \rangle \quad f|_{[a, b]} \in \mathcal R[a, b].
        \]
    \end{icloze}
\end{note}

\begin{note}{6b661945d0ee47a7a04ba92793c704f2}
    \begin{icloze}{1}Множество функций, локально интегрируемых на \({ \langle A, B \rangle }\),\end{icloze} обозначается \begin{icloze}{2}\({ \mathcal R_{loc}(\langle A, B \rangle) }\) или \({ \mathcal R_{loc}\langle A, B \rangle }\).\end{icloze}
\end{note}

\begin{note}{435bbbcd954b4c8bb3be63c6b63acc2a}
    Пусть \begin{icloze}{3}\({ -\infty < a < b \leqslant +\infty }\), \({ f \in \mathcal R_{loc}[a, b) }\).\end{icloze}
    \[
        \begin{icloze}{2}\int_{a}^{\to b} f\end{icloze} \overset{\text{def}}= \begin{icloze}{1}\lim_{c \to b^{-}} \int_{a}^{c} f.\end{icloze}
    \]
\end{note}

\begin{note}{51919a133fa34a6b8ed5ed7ad5b76bac}
    Пусть \begin{icloze}{3}\({ -\infty \leqslant a < b < +\infty }\), \({ f \in \mathcal R_{loc}(a, b] }\).\end{icloze}
    \[
        \begin{icloze}{2}\int_{\to a}^{b} f\end{icloze} \overset{\text{def}}= \begin{icloze}{1}\lim_{c \to a^{+}} \int_{c}^{b} f.\end{icloze}
    \]
\end{note}

\begin{note}{27d9a13709de4fa5ab8baff3cd9fbaa0}
    Выражение
    \[
        { \int_{a}^{\to b} f }
    \]
    называют \begin{icloze}{1}несобственным интегралом \({ f }\) по промежутку \({ [a, b) }\).\end{icloze}
\end{note}

\begin{note}{bb25f0567eef47e7b707c500e9674f65}
    Выражение
    \[
        { \int_{\to a}^{b} f }
    \]
    называют \begin{icloze}{1}несобственным интегралом \({ f }\) по промежутку \({ (a, b] }\).\end{icloze}
\end{note}

\begin{note}{f741cd46bba044cfa110acd5e82b1af1}
    Несобственный интеграл называют \begin{icloze}{2}сходящимся,\end{icloze} если \begin{icloze}{1}он определён и конечен.\end{icloze}
\end{note}

\begin{note}{db034e9065e54430baecac9bfd0a644f}
    Несобственный интеграл называют \begin{icloze}{2}расходящимся,\end{icloze} если \begin{icloze}{1}он имеет бесконечное значение или не существует.\end{icloze}
\end{note}

\begin{note}{b0514d3cbba648c8a6e3b58d057869cd}
    Пусть \begin{icloze}{2}\({ f \in \mathcal R[a, b] }\).\end{icloze}
    Тогда
    \[
        \int_{a}^{\to b} f = \begin{icloze}{1}\int_{a}^{b} f.\end{icloze}
    \]
\end{note}

\begin{note}{c8d35c7d4dfa4a8ab001d84f09a382bd}
    Пусть \({ f \in \mathcal R[a, b] }\). Как показать, что \({ \int_{a}^{\to b} f = \int_{a}^{b} f }\)?

    \begin{cloze}{1}
        Непосредственно следует из непрерывности функции \({ x \mapsto \int_{a}^{x} f }\) (теорема Барроу).
    \end{cloze}
\end{note}

\begin{note}{ad8461ebb11542bead198798c15c6a91}
    Пусть \begin{icloze}{3}\({ f \in \mathcal R_{loc}[a, b) }\), \({ c \in (a, b) }\).\end{icloze}
    Тогда
    \[
        \begin{icloze}{2}\int_{a}^{\to b} f\end{icloze} = \begin{icloze}{1}\int_{a}^{c} f + \int_{c}^{\to b} f.\end{icloze}
    \]
\end{note}

\begin{note}{4873742f993743639632296fcb9c3fda}
    Пусть \({ f \in \mathcal R_{loc}[a, b) }\), \({ c \in (a, b) }\).
    Тогда
    \begin{center}
        \begin{icloze}{2}\({ \displaystyle \int_{a}^{\to b} f }\) сходится\end{icloze} \begin{icloze}{3}\({ \iff }\)\end{icloze} \begin{icloze}{1}\({ \displaystyle \int_{c}^{\to b} f }\) сходится.\end{icloze}
    \end{center}
\end{note}

\begin{note}{7f5c166670d24acc8348a74343d875cc}
    Пусть \begin{icloze}{3}\({ f \in \mathcal R_{loc} [a, b) }\),\: \({ A \in (a, b) }\).\end{icloze}
    \begin{icloze}{1}Несобственный интеграл \({ \int_{A}^{\to b} }\)\end{icloze} называется \begin{icloze}{2}остатком интеграла \({ \int_{a}^{\to b} f }\).\end{icloze}
\end{note}

\begin{note}{3e68bd52c99f47b59cb0f6c550c12420}
    Пусть \({ f \in \mathcal R_{loc}[a, b) }\).
    Тогда
    \[
        \begin{icloze}{2}\int_{a}^{\to b} f \in \mathbb R\end{icloze}
        \begin{icloze}{3}\implies\end{icloze}
        \begin{icloze}{1}\int_{A}^{\to b} f \underset{A \to b^{-}}\longrightarrow 0.\end{icloze}
    \]
\end{note}

\begin{note}{6f388abbe0394963abf8ebd7c8090616}
    Пусть \({ f \in \mathcal R_{loc}[a, b) }\).
    Как показать, что
    \[
        \int_{a}^{\to b} f \in \mathbb R \implies \int_{A}^{\to b} f \underset{A \to b^{-}}\longrightarrow 0?
    \]

    \begin{cloze}{1}
        Представить \({ \int_{A}^{\to b} f }\) как \({ \int_{a}^{\to b} - \int_{a}^{A} f }\).
    \end{cloze}
\end{note}

\begin{note}{eed850367bd24674b9fa0cdbd5af00f5}
    Пусть \({ f, g \in \mathcal R_{loc}[a, b) }\),\: \({ \alpha, \beta \in \mathbb R }\).
    Тогда
    \[
        \begin{icloze}{2}\int_{a}^{\to b} (\alpha f + \beta g)\end{icloze} = \begin{icloze}{1}\alpha \int_{a}^{\to b} f + \beta \int_{a}^{\to b} g.\end{icloze}
    \]
\end{note}

\begin{note}{1dcd21125ff841338d9ddad83b33302d}
    Пусть \({ f, g \in \mathcal R_{loc}[a, b) }\),\: \({ \alpha, \beta \in \mathbb R }\).
    Тогда
    \begin{center}
        \begin{icloze}{2}\({ \displaystyle \int_{a}^{\to b} (\alpha f + \beta g) }\) сходится \end{icloze}
        \begin{icloze}{3}\({ \iff }\) \end{icloze}
        \begin{icloze}{1}\({ \displaystyle \int_{a}^{\to b} f, \int_{a}^{\to b} g }\) сходятся.\end{icloze}
    \end{center}
\end{note}

\begin{note}{5ca16eceb7f145959f15bd1feb33442e}
    \subsubsection{<<\begin{icloze}{3}Замена переменной в несобственном интеграле\end{icloze}>>}

    Пусть \begin{icloze}{2}\({ \varphi \in C^{1}[\alpha, \beta) }\) монотонна,\: \({ f \in C[\varphi(\alpha), \varphi(\beta^{-})) }\).\end{icloze}
    Тогда
    \begin{icloze}{1}
        \[
            \int_{\alpha}^{\to \beta} (f \circ \varphi) \varphi' = \int_{\varphi(\alpha)}^{\to \varphi(\beta^{-})} f.
        \]
    \end{icloze}
\end{note}

\begin{note}{e96879968f0f4e3f8bcf844f10abfcc0}
    Пусть \({ f : [a, b) \to \mathbb R }\).
    \[
        \begin{icloze}{2}f \big|_{a}^{\to b}\end{icloze} \overset{\text{def}}= \begin{icloze}{1}f(b^{-}) - f(a).\end{icloze}
    \]
\end{note}

\begin{note}{a4c1c50607984cd88be2763ea1bce888}
    Пусть \({ f : (a, b] \to \mathbb R }\).
    \[
        \begin{icloze}{2}f \big|_{\to a}^{b}\end{icloze} \overset{\text{def}}= \begin{icloze}{1}f(b) - f(a^{+}).\end{icloze}
    \]
\end{note}

\begin{note}{ceae3033619d408da982f6affab84e5b}
    Пусть \begin{icloze}{3}\({ f, g \in C^{1} [a, b) }\) и \({ \exists \displaystyle \lim_{x \to b^{-}} (f \cdot g)(x) }\).\end{icloze}
    Тогда
    \[
        \begin{icloze}{2}\int_{a}^{\to b} f g'\end{icloze}
        =
        \begin{icloze}{1}fg \Big|_{a}^{\to b} - \int_{a}^{\to b} f' g.\end{icloze}
    \]
\end{note}

\section{Семинар 21.04.22}
\begin{note}{c06a4a56c952416593e6c6aa940af048}
    Основное тригонометрическое тождество для гиперболических функций:
    \begin{icloze}{1}
        \[
            \operatorname{ch}^2 x - \operatorname{sh}^2 x = 1
        \]
    \end{icloze}
\end{note}

\begin{note}{e38aaa87d3a54ae0aa44b08a84229224}
    Интегралы вида
    \[
        \int \begin{icloze}{2}R(x, \sqrt{a^2 - x^2})\end{icloze}\: dx
    \]
    берутся \begin{icloze}{1}подстановкой \({ a \sin t = x }\)\end{icloze} для \({ t \in \begin{icloze}{3}[-\frac{\pi}{2}, \frac{\pi}{2}]\end{icloze} }\).
\end{note}

\begin{note}{9c734267d24143218a9ed85452b8fb67}
    Интегралы вида
    \[
        \int \begin{icloze}{2}R(x, \sqrt{a^2  + x^2})\end{icloze}\: dx
    \]
    берутся \begin{icloze}{1}заменой \({ a \operatorname{sh} t = x }\)\end{icloze} для \({ t \in \begin{icloze}{3}\mathbb R\end{icloze} }\).
\end{note}

\begin{note}{8c5f51e8a73d4e3fbf734d43dd62b423}
    Интегралы вида
    \[
        \int \begin{icloze}{2}R(x, \sqrt{x^2 - a^2})\end{icloze}\: dx
    \]
    берутся \begin{icloze}{1}заменой \({ a \operatorname{ch} t = x }\)\end{icloze} для \({ t \in \begin{icloze}{3}\mathbb R_+\end{icloze} }\).
\end{note}

\begin{note}{56604ede765d4039b4eba27776ff61d0}
    Интегралы вида
    \[
        \int \begin{icloze}{2}R(\sin x, \cos x)\end{icloze}\: dx
    \]
    берутся заменой \begin{icloze}{1}\({ t = \tan \frac{x}{2} }\).\end{icloze}
\end{note}

\begin{note}{a709df92e9fe4310a7c2687c5de50054}
    \begin{icloze}{2}Подстановку \({ t = \tan \frac{x}{2} }\)\end{icloze} называют \begin{icloze}{1}универсальной тригонометрической подстановкой.\end{icloze}
\end{note}

\begin{note}{a6d2e564dd8246af81224496b8f276b0}
    При взятии интегралов с помощью универсальной тригонометрической подстановки для определённости полагают, что \begin{icloze}{1}\({ x \in (-\pi, \pi) }\).\end{icloze}
\end{note}

\begin{note}{4baa637b1c464ab9b9b94bf7ef71fcc8}
    Как \({ \sin x }\) выражается через \({ t = \tan \frac{x}{2} }\)?

    \begin{cloze}{1}
        \[
            \sin x =  \frac{2t}{1 + t^2}.
        \]
    \end{cloze}
\end{note}

\begin{note}{3cee332cac3a4eee9f02834f87c159ea}
    Как \({ \cos x }\) выражается через \({ t = \tan \frac{x}{2} }\)?

    \begin{cloze}{1}
        \[
            \cos x = \frac{1 - t^2}{1 + t^2}.
        \]
    \end{cloze}
\end{note}

\begin{note}{dd942a562634467e97ec85771a1d7a27}
    Каковы условия для частных случаев при взятии интеграла
    \[
        \int R(\sin x,  \cos x)\: dx?
    \]

    \begin{cloze}{1}
        \begin{enumerate}
            \item \({ R(-\sin x, \cos x) = -R(\sin x, \cos x) }\);
            \item \({ R(\sin x, -\cos x) = -R(\sin x, \cos x) }\);
            \item \({ R(-\sin x, -\cos x) = R(\sin x, \cos x) }\).
        \end{enumerate}
    \end{cloze}
\end{note}

\begin{note}{ff62f0455f5449e6a2187727fb759f0d}
    Если \begin{icloze}{2}\({ R(-\sin x, \cos x) = -R(\sin x, \cos x) }\),\end{icloze} то интеграл
    \[
        \int R(\sin x, \cos x)\: dx
    \]
    берётся \begin{icloze}{1}заменой \({ t = \cos x }\).\end{icloze}
\end{note}

\begin{note}{05898276b49140939ed972fa0dd24b19}
    Если \begin{icloze}{2}\({ R(\sin x, -\cos x) = -R(\sin x, \cos x) }\),\end{icloze} то интеграл
    \[
        \int R(\sin x, \cos x)\: dx
    \]
    берётся \begin{icloze}{1}заменой \({ t = \sin x }\).\end{icloze}
\end{note}

\begin{note}{6d9149e716954e3694079d557eb41976}
    Если \begin{icloze}{2}\({ R(-\sin x, -\cos x) = R(\sin x, \cos x) }\),\end{icloze} то интеграл
    \[
        \int R(\sin x, \cos x)\: dx
    \]
    берётся \begin{icloze}{1}заменой \({ t = \tan x }\).\end{icloze}
\end{note}

\begin{note}{ac6a5109ec254fdfbe4146361aef5293}
    Интегралы вида
    \[
        \int \begin{icloze}{2}\sin^{m} x \cdot \cos^{m} x\end{icloze}\: dx, \quad n, m \in \mathbb Q
    \]
    берутся \begin{icloze}{1}заменой \({ t = \sin x }\)\end{icloze}
\end{note}

\section{Лекция 30.04.22 (1)}
\begin{note}{c145ba4d6dca4a06a50cec14e1551b20}
    Пусть \({ f \in \mathcal R_{loc}[a, b) }\), \begin{icloze}{4}\({ f \geqslant 0 }\).\end{icloze}
    Тогда \({ \int_{a}^{\to b} f }\) \begin{icloze}{2}сходится\end{icloze} \begin{icloze}{3}\({ \iff }\)\end{icloze} функция \begin{icloze}{1}\({ x \mapsto \int_{a}^{x} f }\)\end{icloze} ограничена сверху на \({ [a, b) }\).
\end{note}

\begin{note}{0be99a47ac0e428faad6ea9879f39822}
    Пусть \({ f \in \mathcal R_{loc}[a, b) }\), \({ f \geqslant 0 }\).
    Тогда \({ \int_{a}^{\to b} f }\) сходится \({ \iff }\) функция \({ x \mapsto \int_{a}^{x} f }\) ограничена сверху на \({ [a, b) }\).
    В чём основная идея доказательства?

    \begin{cloze}{1}
        \({ \int_{a}^{x} f \!\!\nearrow }\) и теорема о пределе монотонной функции.
    \end{cloze}
\end{note}

\begin{note}{e94553827bfa462995cb7abf3f169453}
    Пусть \({ f \in \mathcal R_{loc}[a, b) }\), \begin{icloze}{2}\({ f \geqslant 0 }\).\end{icloze}
    Тогда \({ \int_{a}^{\to b} f }\) либо \begin{icloze}{1}сходится,\end{icloze} либо \begin{icloze}{1}расходится к \({ +\infty }\).\end{icloze}
\end{note}

\begin{note}{2613f68b7e7b444ea053850e9887a1e8}
    Пусть \({ f \in \mathcal R_{loc}[a, b) }\), \begin{icloze}{2}\({ f \geqslant 0 }\).\end{icloze}
    Тогда \({ \int_{a}^{\to b} f = \begin{icloze}{1}\underset{x \in [a, b)}{\sup} \int_{a}^{x} f\end{icloze} }\).
\end{note}

\begin{note}{05b94ccacfa246db99eb2dcbfc408f2a}
    Пусть \({ f, g \in \mathcal R_{loc} [a, b) }\),\: \begin{icloze}{5}\({ f, g \geqslant 0 }\),\end{icloze}
    \[
        f(x) = \begin{icloze}{4}O(g(x)), \quad x \to b^{-}.\end{icloze}
    \]
    Тогда \({ \begin{icloze}{1}\int_{a}^{\to b} f \in \mathbb R\end{icloze}
    \begin{icloze}{3}\impliedby\end{icloze}
\begin{icloze}{2}\int_{a}^{\to b} g \in \mathbb R\end{icloze} }\)
\end{note}

\begin{note}{635015d74b6c4594817b074aa5417e29}
    Пусть \({ f, g \in \mathcal R_{loc} [a, b) }\),\: \begin{icloze}{5}\({ f, g \geqslant 0 }\),\end{icloze}
    \[
        f(x) = \begin{icloze}{4}O(g(x)), \quad x \to b^{-}.\end{icloze}
    \]
    Тогда \({ \begin{icloze}{2}\int_{a}^{\to b} f \not\in \mathbb R\end{icloze} \begin{icloze}{3}\implies\end{icloze} \begin{icloze}{1}\int_{a}^{\to b} g \not\in \mathbb R\end{icloze} }\)
\end{note}

\begin{note}{95f69b2a23484af79ddfafb4fda8cf09}
    Пусть \({ f, g \in \mathcal R_{loc} [a, b) }\),\: \({ f, g \geqslant 0 }\),
    \[
        f(x) = O(g(x)), \quad x \to b^{-}.
    \]
    Тогда, если \({ \int_{a}^{\to b} g }\) сходится, то и \({ \int_{a}^{\to b} f }\) сходится.

    В чём основная идея доказательства?

    \begin{cloze}{1}
        Из определения \({ O }\)-большого сравнить остатки \({ \int_{\varepsilon}^{\to b} f }\) и \({ \int_{\varepsilon}^{\to b} g }\).
    \end{cloze}
\end{note}

\begin{note}{88104faac5ad425e8ad1e9d194e82fcf}
    Пусть \({ f, g \in \mathcal R_{loc}[a, b) }\),\: \begin{icloze}{5}\({ f \geqslant 0 }\), \({ g > 0 }\)\end{icloze} и \begin{icloze}{4}\({ \displaystyle \frac{f}{g}(b^{-}) \in [0, +\infty) }\).\end{icloze}
    Тогда
    \[
        \begin{icloze}{1}\int_{a}^{\to b} f \in \mathbb R\end{icloze}
        \begin{icloze}{2}\impliedby\end{icloze}
        \begin{icloze}{3}\int_{a}^{\to b} g \in \mathbb R.\end{icloze}
    \]
\end{note}

\begin{note}{1c33032b82c74c258d783b6f03d0ec39}
    Пусть \({ f, g \in \mathcal R_{loc}[a, b) }\),\: \({ f \geqslant 0 }\), \({ g > 0 }\) и \({ \displaystyle \frac{f}{g}(b^{-}) \in [0, +\infty) }\).
    Тогда
    \[
        \int_{a}^{\to b} g \in \mathbb R \implies \int_{a}^{\to b} f \in \mathbb R.
    \]
    В чём основная идея доказательства?

    \begin{cloze}{1}
        \({ \displaystyle \frac{f}{g} }\) ограничена в окрестности \({ b }\) \({ \implies }\) \({ f(x) = O(g(x)) }\).
    \end{cloze}
\end{note}

\begin{note}{cec6bac50bc74bf1b743dc23079418dc}
    Пусть \({ f, g \in \mathcal R_{loc}[a, b) }\),\: \begin{icloze}{5}\({ f \geqslant 0 }\), \({ g > 0 }\)\end{icloze} и \begin{icloze}{4}\({ \displaystyle \frac{f}{g}(b^{-}) \in (0, +\infty] }\).\end{icloze}
    Тогда
    \[
        \begin{icloze}{2}\int_{a}^{\to b} f \in \mathbb R\end{icloze}
        \begin{icloze}{3}\implies\end{icloze}
        \begin{icloze}{1}\int_{a}^{\to b} g \in \mathbb R.\end{icloze}
    \]
\end{note}

\begin{note}{156267ccd8034eca99dbb3d574fbf25e}
    Пусть \({ f, g \in \mathcal R_{loc}[a, b) }\),\: \({ f \geqslant 0 }\), \({ g > 0 }\) и \({ \displaystyle \frac{f}{g}(b^{-}) \in (0, +\infty] }\).
    Тогда
    \[
        \int_{a}^{\to b} f \in \mathbb R \implies \int_{a}^{\to b} g \in \mathbb R.
    \]

    В чём основная идея доказательства?

    \begin{cloze}{1}
        Поменять местами \({ f }\) и \({ g }\), рассмотрев \({ \displaystyle \frac{g}{f}(b^{-}) \in [0, +\infty) }\).
    \end{cloze}
\end{note}

\begin{note}{1139c594c8b941fc9a945b4ca914a566}
    Пусть \({ f, g \in \mathcal R_{loc}[a, b) }\),\: \begin{icloze}{5}\({ f \geqslant 0 }\), \({ g > 0 }\)\end{icloze} и \begin{icloze}{4}\({ \displaystyle \frac{f}{g}(b^{-}) \in (0, +\infty) }\).\end{icloze}
    Тогда
    \[
        \begin{icloze}{2}\int_{a}^{\to b} f \in \mathbb R\end{icloze}
        \begin{icloze}{3}\iff \end{icloze}
        \begin{icloze}{1}\int_{a}^{\to b} g \in \mathbb R.\end{icloze}
    \]
\end{note}

\begin{note}{1588fad5b7a94fb7a874e07813b56e6a}
    Пусть \({ f, g \in \mathcal R_{loc}[a, b) }\),\: \begin{icloze}{5}\({ f, g \geqslant 0 }\),\end{icloze}
    \begin{icloze}{4}
        \[
            f(x) \sim g(x), \quad x \to b^{-}.
        \]
    \end{icloze}
    Тогда \({ \begin{icloze}{2}\int_{a}^{\to b} f \in \mathbb R\end{icloze} \begin{icloze}{3}\iff\end{icloze} \begin{icloze}{1}\int_{a}^{\to b} g \in \mathbb R\end{icloze} }\).
\end{note}

\begin{note}{3804d89199d040dea2d1a4ab5975760d}
    Пусть \({ f, g \in \mathcal R_{loc}[a, b) }\),\: \({ f, g \geqslant 0 }\),
    \[
        f(x) \sim g(x), \quad x \to b^{-}.
    \]
    Тогда \({ \int_{a}^{\to b} f \in \mathbb R \iff \int_{a}^{\to b} g \in \mathbb R }\).

    В чём основная идея доказательства?

    \begin{cloze}{1}
        Показать, что \({ f }\) и \({ g }\) ограничены по сравнению друг с другом.
    \end{cloze}
\end{note}

\begin{note}{03152527c42d4e8bb83dd25771b6c907}
    Пусть \({ f \in C[a, +\infty) }\),\: \({ f \geqslant 0 }\).
    Тогда
    \[
        \int_{a}^{+\infty} f \in \mathbb R
        \begin{icloze}{1}\:\not\!\!\!\implies\end{icloze}
        f(x) \underset{x \to +\infty}\longrightarrow 0.
    \]
\end{note}

\begin{note}{3b29658e766c4cd28321fc274785321b}
    Пусть \({ f \in \mathcal R_{loc}[a, b) }\).
    Является ограниченность функции
    \[
        x \mapsto \int_{a}^{x} f
    \]
    необходимым условием для сходимости \({ \int_{a}^{\to b} f }\)?

    \begin{cloze}{1}
        Да, является.
    \end{cloze}
\end{note}

\begin{note}{d8d5ca8f054b4f198c844c2fa4947b52}
    Пусть \({ f \in \mathcal R_{loc}[a, b) }\).
    Является ограниченность функции
    \[
        x \mapsto \int_{a}^{x} f
    \]
    достаточным условием для сходимости \({ \int_{a}^{\to b} f }\)?

    \begin{cloze}{1}
        Нет, она является только необходимым условием.
    \end{cloze}
\end{note}

\begin{note}{0293c1fbae2c45939b37aee500e2e937}
    Пусть \({ f \in \mathcal R_{loc}[a, b) }\).
    Интеграл \({ \int_{a}^{\to b} f }\) называют \begin{icloze}{2}сходящимся абсолютно,\end{icloze} если \begin{icloze}{1}\({ \int_{a}^{\to b} \left\lvert f \right\rvert }\) сходится.\end{icloze}
\end{note}

\begin{note}{9f03ada404444f7b806ae914e3c10af8}
    Пусть \({ f, g \in \mathcal R_{loc}[a, b) }\), \({ \alpha, \beta \in \mathbb R }\).
    Если \({ \int_{a}^{\to b} f }\) и \({ \int_{a}^{\to b} g }\) \begin{icloze}{2}сходятся абсолютно,\end{icloze} то \begin{icloze}{3}\({ \int_{a}^{\to b} (\alpha f + \beta g) }\)\end{icloze} \begin{icloze}{1}сходится абсолютно.\end{icloze}
\end{note}

\begin{note}{877bbbaf0aad47f18e73982ecb61a25c}
    Пусть \({ f \in \mathcal R_{loc}[a, b) }\).
    Если \({ \int_{a}^{\to b} f }\) сходится абсолютно, то он \begin{icloze}{1}сходится.\end{icloze}
\end{note}

\begin{note}{1e256a23b0234f618630bb0ecd81a5c1}
    Пусть \({ f \in \mathcal R_{loc}[a, b) }\).
    Если \({ \int_{a}^{\to b} f }\) сходится абсолютно, то он сходится.
    В чём ключевая идея доказательства?

    \begin{cloze}{1}
        Критерий Больцано-Коши сходимости функции для соответствующих интегралов с переменным верхним пределом.
    \end{cloze}
\end{note}

\begin{note}{90872b464e394488bd0388fbf5e5fe3e}
    Может ли несобственный интеграл сходиться, не сходясь при этом абсолютно?

    \begin{cloze}{1}
        Да, может.
    \end{cloze}
\end{note}

\begin{note}{0268c46e88c24c1d9a2339d7f0d3ee01}
    Пусть \({ f \in \mathcal R_{loc}[a, b) }\).
    Если \({ \int_{a}^{\to b} f }\) \begin{icloze}{2}сходится, но не абсолютно,\end{icloze} то говорят, что \begin{icloze}{1}он сходится условно или неабсолютно.\end{icloze}
\end{note}

\begin{note}{8cf6a70282a945a2ba0dc4b0a63abad8}
    Пусть \({ f, g \in \mathcal R_{loc}[a, b) }\).
    Если интеграл \({ \int_{a}^{\to b} f }\) \begin{icloze}{2}сходится условно,\end{icloze} а \({ \int_{a}^{\to b} g }\) \begin{icloze}{2}сходится абсолютно,\end{icloze} то \({ \int_{a}^{\to b} (f + g) }\) \begin{icloze}{1}сходится условно.\end{icloze}
\end{note}

\begin{note}{710cc42101c14046a49743d5070986c1}
    Пусть \({ f, g \in \mathcal R_{loc}[a, b) }\).
    Если интеграл \({ \int_{a}^{\to b} f }\) сходится условно, а \({ \int_{a}^{\to b} g }\) сходится абсолютно, то \({ \int_{a}^{\to b} (f + g) }\) сходится условно.
    В чём основная идея доказательства?

    \begin{cloze}{1}
        Представить \({ f }\) как сумму \({ (f + g) - g }\).
    \end{cloze}
\end{note}

\begin{note}{4ff4773b59de42daa0a880e4a8a170e7}
    Пусть \begin{icloze}{4}\({ f \in C[a, b) }\),\end{icloze} \begin{icloze}{3}\({ g \in C^{1}[a, b) }\),\: \({ g }\) --- монотонна.\end{icloze} Если
    \begin{icloze}{1}
        \({ g }\) бесконечно мала в точке \({ b^{-} }\), a
        \({ x \mapsto \int_{a}^{x} f }\) ограничена на \({ (a, b) }\),
    \end{icloze}
    то \begin{icloze}{2}\({ \int_{a}^{\to b} (f \cdot g) }\) сходится.\end{icloze}

    \begin{center}
        \tiny
        <<Признак \begin{icloze}{5}Дирихле\end{icloze} \begin{icloze}{2}сходимости \\\phantom{<<}несобственного интеграла\end{icloze}>>
    \end{center}
\end{note}

\begin{note}{ae6efc013b634d36a35078f6d529988d}
    В чём основная идея доказательства признака Дирихле сходимости несобственного интеграла?

    \begin{cloze}{1}
        Применить формулу интегрирования по частям и показать абсолютную сходимость \({ \int_{a}^{\to b} Fg' }\).
    \end{cloze}
\end{note}

\begin{note}{2f08599a01e3436a9f48422e767ba763}
    Как в доказательстве признака Дирихле сходимости несобственного интеграла показать, что \({ \int_{a}^{\to b} Fg' }\) сходится абсолютно?

    \begin{cloze}{1}
        Оценить сверху значение \({ \int_{a}^{\to b} \left\lvert F g' \right\rvert }\).
    \end{cloze}
\end{note}

\begin{note}{599be6e31a83462881df0f2734d270d5}
    Почему в доказательстве признака Дирихле сходимости несобственного интеграла интеграл \({ \int_{a}^{\to b} \left\lvert Fg' \right\rvert }\) не может не существовать?

    \begin{cloze}{1}
        Потому что \({ \left\lvert F g' \right\rvert \geqslant 0 }\).
    \end{cloze}
\end{note}

\begin{note}{fdf163be427a40e1a02bdfb1d3392f46}

    Пусть \begin{icloze}{4}\({ f \in C[a, b) }\),\end{icloze} \begin{icloze}{3}\({ g \in C^{1}[a, b) }\),\: \({ g }\) --- монотонна.\end{icloze} Если
    \begin{icloze}{1}
        \({ g }\) --- ограничена, а \({ \int_{a}^{\to b} f }\) сходится,
    \end{icloze}
    то \begin{icloze}{2}\({ \int_{a}^{\to b} (f \cdot g) }\) сходится.\end{icloze}

    \begin{center}
        \tiny
        <<Признак \begin{icloze}{5}Абеля\end{icloze} \begin{icloze}{2}сходимости \\\phantom{<<}несобственного интеграла\end{icloze}>>
    \end{center}
\end{note}

\begin{note}{5714e2e7542d4b4aa0316cb086c35aac}
    В чём основная идея доказательства признака Абеля сходимости несобственного интеграла?

    \begin{cloze}{1}
        Признак Дирихле для функций \({ f }\) и
        \[
            x \mapsto g(x) - g(b^{-}).
        \]
    \end{cloze}
\end{note}

\begin{note}{8ceee057705b4a778730645db78a3fb2}
    Признак Дирихле сходимости несобственного интеграла \begin{icloze}{2}так же будет выполняться,\end{icloze} если ослабить допущения до:
    \begin{icloze}{1}
        \begin{center}
            \({ f \in \mathcal R_{loc}[a, b) }\), \qquad \({ g }\) монотонна на \({ [a, b) }\).
        \end{center}
    \end{icloze}

    \begin{center}
        \tiny
        (без доказательства)
    \end{center}
\end{note}

\begin{note}{3d0b9c7aa10142388b565e673f847be1}
    Признак Абеля сходимости несобственного интеграла \begin{icloze}{2}так же будет выполняться,\end{icloze} если ослабить допущения до:
    \begin{icloze}{1}
        \begin{center}
            \({ f \in \mathcal R_{loc}[a, b) }\), \qquad \({ g }\) монотонна на \({ [a, b) }\).
        \end{center}
    \end{icloze}

    \begin{center}
        \tiny
        (без доказательства)
    \end{center}
\end{note}

\section{Лекция 30.04.22 (2)}
\begin{note}{519087b524d64a7db308cfc813fb90f7}
    Отображение \({ U : \mathbb R^{n} \to \mathbb R^{n} }\) называется \begin{icloze}{2}движением пространства \({ \mathbb R^{n} }\),\end{icloze} если \begin{icloze}{1}оно сохраняет расстояние между точками.\end{icloze}
\end{note}

\begin{note}{b1453ceeeece4b72b798234b7da1813d}
    \begin{icloze}{3}Площадью\end{icloze} называется отображение \({ S : \begin{icloze}{1}\left\{ P \right\} \to \mathbb R_+\end{icloze} }\), заданное на \begin{icloze}{4}некотором классе \({ \left\{ P \right\} }\) подмножеств плоскости,\end{icloze} которое при этом
    \begin{icloze}{2}
        \begin{itemize}
            \item аддитивно;
            \item нормируемо на прямоугольниках;
            \item и инвариантно относительно движений.
        \end{itemize}
    \end{icloze}
\end{note}

\begin{note}{47470cb51b9e4e21bb69da097aba5d2b}
    \begin{icloze}{2}Фигуры из класса подмножеств плоскости, на которых задано отображение площади,\end{icloze} называются \begin{icloze}{1}квадрируемыми фигурами.\end{icloze}
\end{note}

\begin{note}{18345644cb79468f9ec377bda0dc0a8b}
    Пусть \begin{icloze}{3}\({ P_1 }\) и \({ P_2 }\) --- квадрируемые фигуры и \({ P_1 \cap P_2 = \emptyset }\).\end{icloze}
    Тогда \begin{icloze}{1}\({ P_1 \cup P_2 }\) --- квадрируемая фигура\end{icloze} и
    \begin{icloze}{2}
        \[
            S(P_1 \cup P_2) = S(P_1) + S(P_2).
        \]
    \end{icloze}

    \begin{center}
        \tiny
        (свойство \begin{icloze}{4}\textbf{аддитивности}\end{icloze} из определения площади)
    \end{center}
\end{note}

\begin{note}{76c9c3e2743049ffa9cc6d12f6dc926d}
    \begin{icloze}{1}Площадь прямоугольника со сторонами \({ a }\) и \({ b }\) равна \({ a b }\).\end{icloze}

    \begin{center}
        \tiny
        (свойство \begin{icloze}{2}\textbf{нормируемости на прямоугольниках}\end{icloze} из определения площади)
    \end{center}
\end{note}

\begin{note}{21333bd49d034ec7af62abe5e5b689e7}
    Пусть \begin{icloze}{3}\({ P }\) --- квадрируемая фигура и \({ U }\) --- движение плоскости.\end{icloze}
    Тогда \begin{icloze}{1}\({ U(P) }\) --- квадрируемая фигура\end{icloze} и
    \begin{icloze}{2}
        \[
            S(U(P)) = S(P).
        \]
    \end{icloze}

    \begin{center}
        \tiny
        (свойство \begin{icloze}{4}\textbf{инвариантности относительно движений}\end{icloze} из определения площади)
    \end{center}
\end{note}

\begin{note}{b8107a1dd5d24f29a317e003bc6c9181}
    Пусть \({ P, P_1 }\) --- квадрируемые фигуры, \begin{icloze}{2}\({ P_1 \subset P }\)\end{icloze}.
    Тогда
    \begin{icloze}{1}
        \[
            S(P_1) \leqslant S(P).
        \]
    \end{icloze}

    \begin{center}
        \tiny
        (\begin{icloze}{3}\textbf{монотонность}\end{icloze} площади)
    \end{center}
\end{note}

\begin{note}{c493291ddbb94c7682c53a108ac7fce4}
    В чём ключевая идея в доказательстве монотонности площади?

    \begin{cloze}{1}
        Представить \({ P }\) как \({ P \cup (P \setminus P_1) }\).
   \end{cloze}
\end{note}

\begin{note}{d66f7a47310e439da162bfb5d766daa8}
    Если квадрируемая фигура \({ P }\) \begin{icloze}{2}содержится в некотором отрезке,\end{icloze} то \begin{icloze}{1}\({ S(P) = 0 }\).\end{icloze}
\end{note}

\begin{note}{4f73cf95243c4ae1a203f5a5cce2a024}
    Если квадрируемая фигура \({ P }\) содержится в некотором отрезке, то \({ S(P) = 0 }\).
    В чём основная идея доказательства?

    \begin{cloze}{1}
        \({ P }\) можно заключить в прямоугольник сколь угодно малой площади.
    \end{cloze}
\end{note}

\begin{note}{eea6ca09c1c445469a7fafb3cdf7c5c6}
    Пусть \begin{icloze}{1}\({ P_1, P_2 }\) --- квадрируемые фигуры, \({ S(P_1 \cap P_2) = 0 }\).\end{icloze}
    Тогда
    \begin{icloze}{2}
        \[
            S(P_1 \cup P_2) = S(P_1) + S(P_2).
        \]
    \end{icloze}

    \begin{center}
        \tiny
        (\begin{icloze}{3}\textbf{усиленная аддитивность}\end{icloze} площади)
    \end{center}
\end{note}

\begin{note}{430f2d5b1ff841c29c7b3b184b5cd509}
    В чём основная идея доказательства усиленной аддитивности площади?

    \begin{cloze}{1}
        Представить \({ P_1 \cup P_2 }\) как \({ (P_1 \setminus P_2) \cup P_2 }\).
    \end{cloze}
\end{note}

\begin{note}{14e7b163e187481dbd050b6109d9c55a}
    Чему равна \({ S(P_1 \setminus P_2) }\) в доказательства усиленной аддитивности площади?

    \begin{cloze}{1}
        \[
            S(P_1 \setminus P_2) = S(P_1).
        \]
    \end{cloze}
\end{note}

\begin{note}{a8841fa849b54fccb1d646f27b165f0d}
    \begin{icloze}{3}Объёмом\end{icloze} называется отображение \({ V : \begin{icloze}{1}\left\{ T \right\} \to \mathbb R_+\end{icloze} }\), заданное на \begin{icloze}{4}некотором классе \({ \left\{ T \right\} }\) подмножеств \({ \mathbb R^3 }\),\end{icloze} которое при этом
    \begin{icloze}{2}
        \begin{itemize}
            \item аддитивно;
            \item нормируемо на прямоугольных параллелепипедах;
            \item и инвариантно относительно движений.
        \end{itemize}
    \end{icloze}
\end{note}

\begin{note}{7cfda5f9df304d888f1d18b59b9c2805}
    \begin{icloze}{2}Элементы из класса подмножеств \({ \mathbb R^3 }\), на которых задано отображение объёма,\end{icloze} называются \begin{icloze}{1}кубируемыми телами.\end{icloze}
\end{note}

\begin{note}{0857e7d4cb3949fa8150811b0b482d43}
    \begin{icloze}{1}Объём прямоугольного параллелепипеда с рёбрами \({ a, b }\) и \({ c }\) равен \({ a b c }\).\end{icloze}

    \begin{center}
        \tiny
        (свойство \begin{icloze}{2}\textbf{нормируемости на прямоугольных параллелепипедах}\end{icloze} из определения объёма)
    \end{center}
\end{note}

\begin{note}{5718ba5a41374083b6fcc1508f440aa6}
    Если кубируемое тело \({ T }\) \begin{icloze}{2}содержится в некотором прямоугольнике,\end{icloze} то \begin{icloze}{1}\({ V(T) = 0 }\).\end{icloze}
\end{note}

\begin{note}{429679b187e449738c280e35b1c167ba}
    Пусть \begin{icloze}{3}\({ P \subset \mathbb R^2 }\), \({ h \geqslant 0 }\).\end{icloze}
    \begin{icloze}{1}Множество \({ P \times [0, h] }\), а так же всякий его образ при движении,\end{icloze} называется \begin{icloze}{2}прямым цилиндром с основанием \({ P }\) и высотой \({ h }\).\end{icloze}
\end{note}

\begin{note}{910d0a9f9f5342c18ec7291c3bcf5a75}
    Пусть \begin{icloze}{3}\({ P }\) --- квадрируемая фигура, \({ h \geqslant 0 }\).\end{icloze}
    Тогда \begin{icloze}{1}цилиндр \({ P  \times [0, h] }\) кубируем\end{icloze} и
    \begin{icloze}{2}
        \[
            V(P \times [0, h]) = S(P)h.
        \]
    \end{icloze}
\end{note}

\begin{note}{09254f53532b46f2a53c623d390ebb80}
    Пусть \({ T \subset \mathbb R^3 }\), \begin{icloze}{3}\({ x \in \mathbb R }\).\end{icloze}
    \begin{icloze}{1}
        Множество
        \[
            \Big\{ (y, z) \in \mathbb R^2 \mid (x, y, z) \in T \Big\}
        \]
    \end{icloze}
    называется \begin{icloze}{2}сечением множества \({ Т }\) первой координатой \({ x }\).\end{icloze}
\end{note}

\begin{note}{364174b8e50c41769a37bd8c66b37c2f}
    Пусть \({ T \subset \mathbb R^3 }\), \({ x \in \mathbb R }\).
    \begin{icloze}{2}Сечение множества \({ T }\) первой  координатой \({ x }\)\end{icloze} обозначается \begin{icloze}{1}\({ T(x) }\).\end{icloze}
\end{note}

\begin{note}{897caf9d2c9e4338bb71941389908a3c}
    Пусть \({ \gamma : [a, b] \to \mathbb R^{m} }\), \({ t \in [a, b] }\).
    \begin{icloze}{2}\({ i }\)-я координата вектора \({ \gamma(t) }\)\end{icloze} обозначается \begin{icloze}{1}\({ \gamma_i(t) }\).\end{icloze}
\end{note}

\begin{note}{bfb45d16ad184797a2996ad01ffcef3e}
    Пусть \({ \gamma : [a, b] \to \mathbb R^{m} }\), \({ i \in [1 : m] }\).
    \begin{icloze}{2}Функция \({ t \mapsto \gamma_i(t) }\)\end{icloze} называется \begin{icloze}{1}\({ i }\)-й координатной функцией отображения \({ \gamma }\).\end{icloze}
\end{note}

\begin{note}{b14417cf3a944cb4a66a5e8e9d998e8e}
    Пусть \({ \gamma : [a, b] \to \mathbb R^{m} }\), \({ i \in [1 : m] }\).
    \begin{icloze}{2}\({ i }\)-я координатная функция \({ \gamma }\)\end{icloze} обозначается \begin{icloze}{1}\({ \gamma_i }\).\end{icloze}
\end{note}

\begin{note}{dd1a71a2fed04fe9afe31c7632361372}
    Пусть \({ \gamma : [a, b] \to \mathbb R^{m} }\).
    Отображение \({ \gamma }\) называется \begin{icloze}{2}непрерывным на \({ [a, b] }\),\end{icloze} если \begin{icloze}{1}каждая его координатная функция непрерывна на \({ [a, b] }\).\end{icloze}
\end{note}

\begin{note}{b547615a5824427faef3dd11a3b25d6d}
    \begin{icloze}{2}Путём в \({ \mathbb R^{m} }\)\end{icloze} называется
    \begin{icloze}{1}
        непрерывное отображение
        \[
            [a, b] \to \mathbb R^{m}.
        \]
    \end{icloze}
\end{note}

\begin{note}{22935aef61d54e6ca59da2c35faa2d07}
    Пусть \({ \gamma : [a, b] \to \mathbb R^{m} }\) --- путь в \({ \mathbb R^{m} }\).
    \begin{icloze}{2}Точка \({ \gamma(a) }\)\end{icloze} называется \begin{icloze}{1}началом пути \({ \gamma }\).\end{icloze}
\end{note}

\begin{note}{306f29e0450e438d99bc99ecf7bfc7f6}
    Пусть \({ \gamma : [a, b] \to \mathbb R^{m} }\) --- путь в \({ \mathbb R^{m} }\).
    \begin{icloze}{2}Точка \({ \gamma(b) }\)\end{icloze} называется \begin{icloze}{1}концом пути \({ \gamma }\).\end{icloze}
\end{note}

\begin{note}{8dd5f0f6a866470697eebadc0467f96f}
    Пусть \({ \gamma : [a, b] \to \mathbb R^{m} }\) --- путь в \({ \mathbb R^{m} }\).
    \begin{icloze}{1}Множество \({ \gamma([a, b]) }\)\end{icloze} называется \begin{icloze}{2}носителем пути \({ \gamma }\).\end{icloze}
\end{note}

\begin{note}{7d4bcf2b08264eb9ab15db7043690c48}
    Пусть \({ \gamma : [a, b] \to \mathbb R^{m} }\) --- путь в \({ \mathbb R^{m} }\).
    Путь \({ \gamma }\) называется \begin{icloze}{2}замкнутым,\end{icloze} если \begin{icloze}{1}\({ \gamma(a) = \gamma(b) }\).\end{icloze}
\end{note}

\begin{note}{5093bb30b14c48e3b3fadaef0c12afb8}
    Пусть \({ \gamma : [a, b] \to \mathbb R^{m} }\) --- путь в \({ \mathbb R^{m} }\).
    Путь \({ \gamma }\) называется \begin{icloze}{2}простым или несамопересекающимся,\end{icloze} если
    \begin{icloze}{1}
        \[
            \gamma(t_1) = \gamma(t_2) \implies t_1 = t_2 \text{ или } t_1, t_2 \in \left\{ a, b \right\}.
        \]
    \end{icloze}
\end{note}

\begin{note}{d8f2c782ae694dd988da84bcb859ff74}
    Пусть \({ \gamma : [a, b] \to \mathbb R^{m} }\) --- путь в \({ \mathbb R^{m} }\).
    Путь \({ \gamma }\) называется \begin{icloze}{2}\({ k }\) раз непрерывно дифференцируемым или \({ k }\)-гладким,\end{icloze} если \begin{icloze}{1}все \({ \gamma_i \in C^{k}[a, b] }\).\end{icloze}
\end{note}

\begin{note}{05da85b7f0a84b368ee3c6bef725c8da}
    Множество всех \({ k }\)-гладких путей \({ [a, b] \to \mathbb R^{m} }\) обозначается
    \begin{icloze}{1}
        \[
            C^{k}[a, b].
        \]
    \end{icloze}
\end{note}

\begin{note}{2915b1df3b9e42158ec7ffe997e98a04}
    Пусть \({ \gamma : [a, b] \to \mathbb R^{m} }\) --- путь в \({ \mathbb R^{m} }\).
    Путь \({ \gamma }\) называется \begin{icloze}{2}ку\-соч\-но-гладким,\end{icloze} если \begin{icloze}{1}существует такое разбиение отрезка \({ [a, b] }\), что
    сужение \({ \gamma }\) на любой из отрезков разбиения --- гладкий путь.\end{icloze}
\end{note}

\begin{note}{9c342e15007543ee80fa38b9df7f3251}
    Пусть \({ \gamma : [a, b] \to \mathbb R^{m} }\) --- путь в \({ \mathbb R^{m} }\).
    Путь, задаваемый формулой
    \begin{icloze}{1}
        \[
            t \mapsto \gamma(a + b - t), \qquad t \in [a, b],
        \]
    \end{icloze}
    называется \begin{icloze}{2}противоположным пути \({ \gamma }\).\end{icloze}
\end{note}

\begin{note}{f069a263f684409d8ff3e26967cba803}
    Пусть \({ \gamma : [a, b] \to \mathbb R^{m} }\) --- путь в \({ \mathbb R^{m} }\).
    \begin{icloze}{2}Путь, противоположный пути \({ \gamma }\),\end{icloze} обозначается \begin{icloze}{1}\({ \gamma^{-} }\).\end{icloze}
\end{note}

\begin{note}{591160f41fd4454bafe7bb81051e9799}
    Два пути \({ \gamma_1 : [a, b] \to \mathbb R^{m},\: \gamma_2 : [\alpha, \beta] \to \mathbb R^{m} }\) называются \begin{icloze}{2}эквивалентными,\end{icloze} если \begin{icloze}{1}существует строго возрастающая сюръекция
    \[
        u : [a, b] \to [\alpha, \beta]
    \]
    такая, что \({ \gamma_1 = \gamma_2 \circ u }\).\end{icloze}
\end{note}

\begin{note}{fe470bb82a2b4926851a2edfc7a0e8bb}
    \begin{icloze}{2}Каждый класс эквивалентности путей в \({ \mathbb R^{m} }\)\end{icloze} называется \begin{icloze}{1}кривой.\end{icloze}
\end{note}

\begin{note}{cbc0b68e357849a2ad4608976c677898}
    \begin{icloze}{2}Каждый из представителей класса эквивалентности, составляющего данную кривую,\end{icloze} называется \begin{icloze}{1}параметризацией этой кривой.\end{icloze}
\end{note}

\begin{note}{466347f360424036ba201c4d9083bb8b}
    Кривую в пространстве \({ \mathbb R^{m} }\) обозначают как \begin{icloze}{1}\({ \left\{ \gamma \right\} }\), где \({ \gamma }\) --- некоторая её параметризация.\end{icloze}
\end{note}

\begin{note}{1faf6fe6c5d7403c89e416216d3e9699}
    Пусть \({ \gamma : [a, b] \to \mathbb R^{m} }\) --- путь в \({ \mathbb R^{m} }\).
    \begin{icloze}{1}Семейство отрезков, соединяющих точки \({ \gamma(\tau_k) }\) и \({ \gamma(\tau_{k + 1}) }\) для \({ \left\{ \tau_k \right\} \in T[a, b] }\),\end{icloze} называют \begin{icloze}{2}ломаной, вписанной в путь \({ \gamma }\).\end{icloze}
\end{note}

\begin{note}{4be9559c1fbc4e42804714ea29cdf641}
    \begin{icloze}{2}Длиной\end{icloze} ломаной называют \begin{icloze}{1}сумму длин составляющих её отрезков.\end{icloze}
\end{note}

\begin{note}{2da98f4689254999b9cebf8a314d0ea1}
    Пусть \({ \gamma : [a, b] \to \mathbb R^{m} }\) --- путь в \({ \mathbb R^{m} }\).
    \begin{icloze}{2}Длиной\end{icloze} пути \({ \gamma }\) называется
    \begin{icloze}{1}
        величина
        \[
            \underset{\tau \in T[a, b]}{\sup} \ell_\tau,
        \]
        где \({ \ell_\tau }\) --- длина ломаной, отвечающей разбиению \({ \tau }\).
    \end{icloze}
\end{note}

\begin{note}{60df0982954545929b0ff63a854ca1d8}
    Пусть \({ \gamma : [a, b] \to \mathbb R^{m} }\) --- путь в \({ \mathbb R^{m} }\).
    \begin{icloze}{2}Длина пути \({ \gamma }\)\end{icloze} обозначается \begin{icloze}{1}\({ s_\gamma }\).\end{icloze}
\end{note}

\begin{note}{38b6d13987e84d16b683136f8cbc95af}
    Пусть \({ \gamma : [a, b] \to \mathbb R^{m} }\) --- путь в \({ \mathbb R^{m} }\).
    Путь \({ \gamma }\) называется \begin{icloze}{2}спрямляемым,\end{icloze} если \begin{icloze}{1}\({ s_\gamma < +\infty }\).\end{icloze}
\end{note}

\section{Лекция 07.05.22 (1)}
\begin{note}{a086c7c565474cc28e5e635aa58fbc68}
    Пусть \({ f : [a, b] \to \mathbb R }\).
    Множество
    \[
        \Big\{ \begin{icloze}{2}(x, y)\end{icloze} \mid \begin{icloze}{1}x \in [a, b],\: y \in \Delta_{0, f(x)}\end{icloze} \Big\}
    \]
    называется \begin{icloze}{3}подграфиком функции \({ f }\).\end{icloze}
\end{note}

\begin{note}{ce2924abf7304cfb861aa52962241aa0}
    Пусть \({ f : [a, b] \to \mathbb R }\).
    \begin{icloze}{1}Подграфик функции \({ f }\)\end{icloze} обозначается \begin{icloze}{2}\({ Q_{f} }\).\end{icloze}
\end{note}

\begin{note}{938420b690a6454bb2d2f1eea1281a1b}
    Пусть \({ f : [a, b] \to \mathbb R }\).
    \begin{icloze}{3}Подграфик\end{icloze} функции \({ f }\) называется \begin{icloze}{2}криволинейной трапецией,\end{icloze} если \begin{icloze}{1}\({ f \geqslant 0,\: f \in C[a, b] }\).\end{icloze}
\end{note}

\begin{note}{8454114fa9b446d495fd999f8d6522b6}
    Как показать, что подграфик криволинейной трапеции является квадрируемой фигурой?

    \begin{cloze}{1}
        Принять на веру. (В нашем курсе это не доказывается.)
    \end{cloze}
\end{note}

\begin{note}{1bbedb3d7edc4466be39badd75fe893f}
    Пусть \({ f \in \mathcal R[a, b] }\),\: \begin{icloze}{3}\({ f \geqslant 0 }\).\end{icloze}
    Тогда \({ \begin{icloze}{2}S(Q_f)\end{icloze} = \begin{icloze}{1}\int_{a}^{b} f\end{icloze} }\).
\end{note}

\begin{note}{913f7f98ebc549638163e89e85c4e29b}
    Пусть \({ f \in C[a, b] }\),\: \({ f \geqslant 0 }\).
    Тогда \({ S(Q_f) = \int_{a}^{b} f }\).

    В чём ключевая идея доказательства?

    \begin{cloze}{1}
        Ограничить значение \({ S(Q_f) }\) через интегральные суммы Дарбу для произвольного разбиения.
    \end{cloze}
\end{note}

\begin{note}{ce632e26e2cf4cb19eea3f0a4ab3c709}
    Пусть \({ f \in \mathcal R[a, b] }\).
    Тогда \({ \begin{icloze}{2}S(Q_f)\end{icloze} = \begin{icloze}{1}\int_{a}^{b} \left\lvert f \right\rvert\end{icloze} }\).
\end{note}

\begin{note}{381eae56ef164e66acf7fefebf662c4a}
    Пусть \({ f \in C[a, b] }\),\: \({ f \geqslant 0 }\).
    Как соотносятся величины \({ S(Q_f) }\) и \({ S(Q_f \setminus \Gamma_f) }\)?

    \begin{cloze}{1}
        Они равны.
    \end{cloze}
\end{note}

\begin{note}{4cde7099bf1543dabb6d8f15b462ae9e}
    Пусть \({ f \in C[a, b] }\),\: \({ f \geqslant 0 }\).
    Откуда следует, что
    \[
        S(Q_f) = S(Q_f \setminus \Gamma_f)?
    \]

    \begin{cloze}{1}
        \({ S(\Gamma_f) = 0 }\).
    \end{cloze}
\end{note}

\begin{note}{b6e7714d201749cd9fd4e7b0bd716384}
    Пусть \begin{icloze}{2}\({ f, g \in \mathcal R[a, b] }\),\: \({ f \leqslant g }\).\end{icloze}
    Фигура, заключённая между графиками \({ f }\) и \({ g }\) тоже называется \begin{icloze}{1}криволинейной трапецией.\end{icloze}
\end{note}

\begin{note}{bc1564f4497f46e687c00d29b41fdb88}
    Пусть \({ f, g \in \mathcal R[a, b] }\),\: \({ f \leqslant g }\).
    \begin{icloze}{2}Площадь криволинейной трапеции, заключённой между графиками \({ f }\) и \({ g }\)\end{icloze} равна
    \begin{icloze}{1}
        \[
            \int_{a}^{b} (g - f).
        \]
    \end{icloze}
\end{note}

\begin{note}{7b57ff51f8f74d7a9e4b6bc05043f32c}
    Пусть \begin{icloze}{5}\({ 0 < \beta - \alpha \leqslant 2\pi }\),\end{icloze}\: \begin{icloze}{4}\({ f \in C[\alpha, \beta] }\),\: \({ f \geqslant 0 }\).\end{icloze}
    Множество точек
    \[
        \Big\{ \begin{icloze}{1} r \cdot (\cos \varphi, \sin \varphi)\end{icloze} \mid \begin{icloze}{2}\varphi \in [\alpha, \beta],\: r \in [0, f(\varphi)]\end{icloze} \Big\}
    \]
    называется \begin{icloze}{3}криволинейным сектором, ограниченным функцией \({ f }\).\end{icloze}
\end{note}

\begin{note}{92d60a9a490b49c2af35b60f685c4b05}
    Пусть \({ 0 < \beta - \alpha \leqslant 2\pi,\: f \in C[\alpha, \beta] }\),\: \({ f \geqslant 0 }\).
    \begin{icloze}{2}Криволинейный сектор, ограниченный функцией \({ f }\),\end{icloze} обозначается \begin{icloze}{1}\({ \widetilde Q_f }\).\end{icloze}
\end{note}

\begin{note}{45120420e85a4503b25a5dd34b000e9f}
    Пусть \({ 0 < \beta - \alpha \leqslant 2\pi,\: f \in C[\alpha, \beta] }\),\: \({ f \geqslant 0 }\).
    Тогда
    \[
        \begin{icloze}{2}S(\widetilde Q_f)\end{icloze} = \begin{icloze}{1}\frac{1}{2} \int_{\alpha}^{\beta} f^2.\end{icloze}
    \]
\end{note}

\begin{note}{896e566b8a2a4fe7817b2f52377d7fbd}
    Пусть \({ 0 < \beta - \alpha \leqslant 2\pi,\: f \in C[\alpha, \beta] }\),\: \({ f \geqslant 0 }\).
    Тогда
    \[
        S(\widetilde Q_f) = \frac{1}{2} \int_{\alpha}^{\beta} f^2.
    \]

    В чём ключевая идея доказательства?

    \begin{cloze}{1}
        Составить суммы, аналогичные суммам Дарбу, но составленные из площадей секторов, а не прямоугольников.
    \end{cloze}
\end{note}

\begin{note}{8e9b6c0f033b44bc81a2f3d4bf467a83}
    При вычислении объёмов с помощью интеграла на фигуру \({ T \subset \mathbb R^3 }\) накладываются следующие ограничения:
    \begin{itemize}
        \item{} \begin{icloze}{1}\({ \exists a, b \in \mathbb R \quad T \subset [a, b] \times \mathbb R^2 }\)\end{icloze}
        \item{} \begin{icloze}{2}\({ \forall x \in [a, b] }\) \quad \({ T(x) }\) --- квадрируемая фигура с площадью \({ S(x) }\), причём \({ S \in C[a, b] }\);\end{icloze}
        \item{}
            \begin{icloze}{3}
                \({ \forall }\) отрезка \({ \Delta \subset [a, b] }\) \quad \({ \exists \xi_\Delta^*, \xi_\Delta^{**} \in \Delta \quad \forall x \in \Delta }\)
                \[
                    T(\xi_\Delta^*) \subset T(x) \subset T(\xi_\Delta^{**}).
                \]
            \end{icloze}
    \end{itemize}
\end{note}

\begin{note}{a7e13633747d4baaad4c2c6b3fa89048}
    Пусть \({ T \subset \mathbb R^3 }\) удовлетворяет условиям для вычисления объёма с помощью интеграла.
    Тогда
    \[
        \begin{icloze}{2}V(T)\end{icloze} = \begin{icloze}{1}\int_{a}^{b} S.\end{icloze}
    \]
\end{note}

\begin{note}{1470b2fb4bbc4e3f8bc341fee4c90bc8}
    Пусть \({ T \subset \mathbb R^3 }\) удовлетворяет условиям для вычисления объёма с помощью интеграла.
    Тогда \({ V(T) = \int_{a}^{b} S }\).
    В чём основная идея доказательства?

    \begin{cloze}{1}
        Составить суммы Дарбу из объёмов цилиндров.
    \end{cloze}
\end{note}

\begin{note}{bd53f21dcde04ea0b85cb5af9f3591a3}
    Пусть \({ f : [a, b] \to \mathbb R }\).
    \begin{icloze}{2}Тело вращения подграфика \({ f }\) вокруг оси \({ Ox }\)\end{icloze} обозначается \begin{icloze}{1}\({ T_f }\).\end{icloze}
\end{note}

\begin{note}{ba286a05719545d7989f5b131598cf49}
    Пусть \begin{icloze}{3}\({ f : C[a, b] }\).\end{icloze}
    Тогда
    \[
        \begin{icloze}{2}V(T_f)\end{icloze} = \begin{icloze}{1}\pi \int_{a}^{b} f^2.\end{icloze}
    \]
\end{note}

\begin{note}{0a669c3f221f45deb163b883985df4d8}
    Пусть \({ f : [a, b] \to \mathbb R }\).
    \begin{icloze}{2}Тело вращения подграфика \({ f }\) вокруг оси \({ Oy }\)\end{icloze} обозначается \begin{icloze}{1}\({ T'_f }\).\end{icloze}
\end{note}

\begin{note}{f584d6707a1844349499bd55a4a9ea9c}
    Пусть \({ f \in C[a, b] }\).
    Тогда
    \[
        \begin{icloze}{2}V(T'_f)\end{icloze} = \begin{icloze}{1}2\pi \int_{a}^{b} x f(x)\: dx.\end{icloze}
    \]
\end{note}

\begin{note}{28b42e81a08c4032a3b4ba6c93022f1f}
    Пусть \({ f \in C[a, b] }\).
    Тогда \({ V(T'_f) = 2\pi \int_{a}^{b} x f(x)\: dx. }\)
    В чём основная идея доказательства (на интуитивном уровне)?

    \begin{cloze}{1}
        Интегрировать по ``площадям'' сечений цилиндрами, построенными на окружностях радиуса \({ x }\).
    \end{cloze}
\end{note}

\begin{note}{37ce05b3e5534f20b8ea802974f37e7d}
    Пусть \begin{icloze}{4}\({ f : [\alpha, \beta] \to \mathbb R_+ }\),\end{icloze}\: \begin{icloze}{3}\({ \alpha, \beta \in [-\frac{\pi}{2}, \frac{\pi}{2}] }\).\end{icloze}
    \begin{icloze}{2}Тело вращения криволинейного сектора, ограниченного функцией \({ f }\), вокруг оси \({ Oy }\)\end{icloze} обозначается \begin{icloze}{1}\({ \widetilde T_f }\).\end{icloze}
\end{note}

\begin{note}{f3588c46da91488b9544a0278ca24931}
    Пусть \({ f : [\alpha, \beta] \to \mathbb R_+ }\) непрерывна,\: \({ -\frac{\pi}{2} \leqslant \alpha < \beta \leqslant \frac{\pi}{2} }\).
    Тогда
    \[
        \begin{icloze}{2}V(\widetilde T_f)\end{icloze} = \begin{icloze}{1}\frac{2\pi}{3} \int_{\alpha}^{\beta} f^3(\theta) \cos \theta\: d\theta.\end{icloze}
    \]
\end{note}

\begin{note}{752d8f77f5274d05967dca413b1c9472}
    Будут ли верны интегральные формулы площадей и объёмов для неограниченных множеств?

    \begin{cloze}{1}
        Да, но выражающие их интегралы будут несобственными.
    \end{cloze}
\end{note}

\begin{note}{081bd87639824313a4c4b6ea6161ff62}
    Пусть \({ \gamma : [a, b] \to \mathbb R^{m} }\),\: \begin{icloze}{3}все \({ \gamma_i }\) --- дифференцируемые функции.\end{icloze}
    Тогда
    \[
        \begin{icloze}{2}\gamma'\end{icloze} \overset{\text{def}}= \begin{icloze}{1}(\gamma'_1, \ldots, \gamma'_m).\end{icloze}
    \]
\end{note}

\begin{note}{78afc1de7241408b95a68a8f5ed2006a}
    Пусть \({ \gamma : [a, b] \to \mathbb R^{m} }\).
    Тогда
    \[
        \begin{icloze}{2}\left\lVert \gamma \right\rVert\end{icloze} = \begin{icloze}{1}\sqrt{\sum_{i}^{} \gamma_i^2}.\end{icloze}
    \]
\end{note}

\begin{note}{0fa2547514204ecfb3eac95208adab8e}
    Пусть \({ \gamma : [a, b] \to \mathbb R^{m} }\) --- путь в \({ \mathbb R^{m} }\).
    Тогда, если \begin{icloze}{4}\({ \gamma \in C^{1}[a, b] }\),\end{icloze} то \begin{icloze}{2}\({ \gamma }\) спрямляем\end{icloze} и
    \({ \begin{icloze}{3}s_\gamma\end{icloze} = \begin{icloze}{1}\int_{a}^{b} \left\lVert \gamma' \right\rVert\end{icloze} }\).
\end{note}

\section{Лекция 07.05.22 (2)}
\begin{note}{ddc3406fefdc4dccb10eeb102957c8ce}
    Пусть \begin{icloze}{2}\({ f \in C[a, b] }\).\end{icloze}
    При рассмотрении \({ \Gamma_f }\) как пути, полагают
    \[
        \Gamma_f (t) = \begin{icloze}{1}(t, f(t)), \quad t \in [a, b].\end{icloze}
    \]
\end{note}

\begin{note}{c2ae0466ac724aa8b74a18651e9b01b8}
    Пусть \begin{icloze}{3}\({ f \in C^{1}[a, b] }\).\end{icloze} Тогда путь \begin{icloze}{4}\({ \Gamma_f }\) спрямляем\end{icloze} и
    \[
        \begin{icloze}{2}s_{\Gamma_f}\end{icloze} = \begin{icloze}{1}\int_{a}^{b} \sqrt{1 + (f')^2}.\end{icloze}
    \]
\end{note}

\begin{note}{d7845b70c0774392b73e28626dab39f8}
    Пусть \begin{icloze}{3}\({ f \in C[\alpha, \beta] }\),\end{icloze} \begin{icloze}{4}\({ f \geqslant 0 }\).\end{icloze}
    Выражение ``\begin{icloze}{2}путь \({ \gamma }\) задаётся в полярных координатах равенством \({ r = f(\theta) }\),\: \({ \theta \in [\alpha, \beta] }\)\end{icloze}'' означает, что
    \[
        \begin{gathered}
            \gamma(\theta) = \begin{icloze}{1}f(\theta) \cdot (\cos \theta, \sin \theta),\end{icloze} \\
            \gamma : \begin{icloze}{5}[\alpha, \beta] \to \mathbb R^2.\end{icloze}
        \end{gathered}
    \]
\end{note}

\begin{note}{58dcc2f7e9c04a209117dce56a3fdacb}
    Пусть \({ f \in C^{1}[\alpha, \beta] }\), \({ f \geqslant 0 }\), путь \({ \gamma }\) задаётся \begin{icloze}{2}в полярных координатах неравенством \({ r = f(\theta) }\),\: \({ \theta \in [\alpha, \beta] }\).\end{icloze} Тогда
    \[
        \begin{icloze}{3}s_\gamma\end{icloze} = \begin{icloze}{1}\int_{a}^{b} \sqrt{f^2 + (f')^2}.\end{icloze}
    \]
\end{note}

\begin{note}{25add1ae8161404b8e2dbf2e360cb9b2}
    Пусть \({ f \in C^{1}[\alpha, \beta] }\), \({ f \geqslant 0 }\), путь \({ \gamma }\) задаётся в полярных координатах неравенством \({ r = f(\theta) }\),\: \({ \theta \in [\alpha, \beta] }\). Тогда
    \[
        s_\gamma = \int_{a}^{b} \sqrt{f^2 + (f')^2}.
    \]
    В чём основная идея доказательства?

    \begin{cloze}{1}
        Явным образом показать, что \({ \left\lVert \gamma' \right\rVert = \sqrt{f^2 + (f')^2} }\).
    \end{cloze}
\end{note}

\section{Лекция 14.05.22 (1)}
\begin{note}{db30293ae2e34ccaad4737c4f196a676}
    \begin{icloze}{2}Отображение \({ E \subset \mathbb R^{n} \to \mathbb R^{m} }\)\end{icloze} называется \begin{icloze}{1}отображением \({ n }\) вещественных переменных.\end{icloze}
\end{note}

\begin{note}{09a496c79f5646e49e9cbe8c39bc708e}
    \begin{icloze}{2}Отображение \({ E \subset \mathbb R^{n} \to \mathbb R }\)\end{icloze} называется \begin{icloze}{1}функцией \({ n }\) вещественных переменных.\end{icloze}
\end{note}

\begin{note}{50d467eacead432fb753be92cd3e3d5b}
    Пусть \({ x \in \mathbb R^{n} }\). \begin{icloze}{2}\({ k }\)-я координата точки \({ x }\)\end{icloze} обозначается \begin{icloze}{1}\({ x_k }\).\end{icloze}
\end{note}

\begin{note}{6298966bfb9a436cafb65c0ac1f3b462}
    Пусть \({ x, y \in \mathbb R^{n} }\).
    Тогда \({ \displaystyle \begin{icloze}{2}x \cdot y\end{icloze} \overset{\text{def}}= \begin{icloze}{1}\sum_{i=1}^{n} x_i y_i\end{icloze} }\).
\end{note}

\begin{note}{82049479525249098f11edd6b86662f5}
    \begin{icloze}{2}Евклидовой нормой в \({ \mathbb R^{n} }\)\end{icloze} называется \begin{icloze}{1}функция
    \[
        \mathbb R^{n} \to \mathbb R_+, \quad x \mapsto \sqrt{x \cdot x}.
    \]\end{icloze}
\end{note}

\begin{note}{0820d783bfc7483c983f2acd41905870}
    Пусть \({ x \in \mathbb R^{n} }\). \begin{icloze}{2}Евклидова норма вектора \({ x }\)\end{icloze} обозначается
    \begin{icloze}{1}
        \[
            \left\lVert x \right\rVert.
        \]
    \end{icloze}
\end{note}

\begin{note}{f7962d62d97a4e45966cfbd5258d7be7}
    Пусть \({ x \in \mathbb R^{n},\: \lambda \in \mathbb R }\).
    Тогда
    \[
        \begin{icloze}{2}\left\lVert \lambda x \right\rVert\end{icloze} = \begin{icloze}{1}\left\lvert \lambda \right\rvert \cdot \left\lVert x \right\rVert.\end{icloze}
    \]
\end{note}

\begin{note}{963d46ce6e6740c2ada2ae1586e864c8}
    Пусть \({ x, y \in \mathbb R^{n} }\). Тогда
    \begin{icloze}{1}\({
        \left\lVert x - y \right\rVert \geqslant \big\lvert \lVert x \rVert - \lVert y \rVert \big\rvert
    }\).\end{icloze}

    \begin{center}
        \tiny
        (обратное неравенство треугольника для норм)
    \end{center}
\end{note}

\begin{note}{197c20e358c34941810836f09234bb95}
    Пусть \({ x \in \mathbb R^{n},\: k \in [1 : n] }\).
    Тогда
    \[
        \begin{icloze}{2}\left\lvert x_k \right\rvert\end{icloze} \begin{icloze}{4}\leqslant\end{icloze} \begin{icloze}{1}\left\lVert x \right\rVert\end{icloze} \begin{icloze}{4}\leqslant\end{icloze} \begin{icloze}{3}\sqrt{n} \cdot \underset{i \in [1 : n]}{\max} \left\lvert x_i \right\rvert.\end{icloze}
    \]
\end{note}

\begin{note}{56e8be63cc924fadaa65ad1a43d24b40}
    \[
        \begin{icloze}{2}\overline{\mathbb R^{n}}\end{icloze} \overset{\text{def}}= \begin{icloze}{1}\mathbb R^{n} \cup \left\{ \infty \right\}.\end{icloze}
    \]
\end{note}

\begin{note}{2d0130af589f4bc8be65d79c610c9313}
    Пусть \({ \begin{icloze}{3}\delta > 0,\end{icloze}\: a \in \mathbb R^{n} }\). Тогда
    \[
        \begin{icloze}{2}V_{\delta}(a)\end{icloze} \overset{\text{def}}= \begin{icloze}{1}\left\{ x \in \mathbb R^{n} \mid \left\lVert x - a \right\rVert < \delta \right\}\end{icloze}
    \]
\end{note}

\begin{note}{3bdf65ade40048daae6226ad13c91370}
    Пусть \({ \begin{icloze}{3}\delta > 0,\end{icloze}\: a \in \mathbb R^{n} }\). Тогда
    \[
        \begin{icloze}{2}\dot V_{\delta}(a)\end{icloze} \overset{\text{def}}= \begin{icloze}{1}V_{\delta}(a) \setminus \left\{ a \right\}.\end{icloze}
    \]
\end{note}

\begin{note}{a872bc2223d54bab976847e9cf14cc91}
    Пусть \begin{icloze}{3}\({ \delta > 0 }\).\end{icloze} Тогда \begin{icloze}{1}множество \({ \left\{ x \in \mathbb R^{n} \mid \left\lVert x \right\rVert > \delta \right\} }\)\end{icloze} называется \begin{icloze}{2}\({ \delta }\)-окрестностью бесконечно удалённой точки в \({ \mathbb R^{n} }\).\end{icloze}
\end{note}

\begin{note}{2a6f33f0df8d4fd78ba6c02bccbc4f26}
    Пусть \({ \delta > 0 }\). Тогда \begin{icloze}{2}\({ \delta }\)-окрестность бесконечно удалённой точки\end{icloze} в \({ \mathbb R^{n} }\) обозначается \begin{icloze}{1}\({ V_{\delta}(\infty) }\).\end{icloze}
\end{note}

\begin{note}{4180d71589654850ba33a99a79e3a0d0}
    Пусть \({ \delta > 0 }\). Тогда в \begin{icloze}{2}\({ \mathbb R^{n} }\)\end{icloze} полагаем \({ \dot V_{\delta}(\infty) \overset{\text{def}}= \begin{icloze}{1}V_{\delta}(\infty)\end{icloze} }\).
\end{note}

\begin{note}{cf41f3caf0a74d8392d85139c9968c07}
    В общем случае окрестности точек в \({ \mathbb R^{n} }\) так же ещё называют \begin{icloze}{1}открытыми \({ n }\)-мерными шарами.\end{icloze}
\end{note}

\begin{note}{bd99061f38de4188b72bb2a2fe1f68e8}
    Пусть \({ \delta >  0 }\). Тогда множество \begin{icloze}{1}\({ V_{\delta}(\infty) \cup \left\{ \infty \right\} }\)\end{icloze} называется \begin{icloze}{2}окрестностью бесконечно  удалённой точки в \({ \overline{\mathbb R^{n}} }\).\end{icloze}
\end{note}

\begin{note}{40f1bd82241c4c7d8c8f327f0fd4f80e}
    Пусть \({ a \in \mathbb \overline{\mathbb R^{n}} }\),\: \({ \delta_1, \delta_2 > 0 }\).
    Тогда
    \[
        V_{\delta_1}(a) \cap V_{\delta_2}(a) = \begin{icloze}{1}V_{\min \left\{ \delta_1, \delta_2 \right\}}(a).\end{icloze}
    \]
\end{note}

\begin{note}{c295589ac6fd4a91b7607784a4f6b94b}
    Пусть \({ \delta_1, \delta_2 > 0 }\).
    Тогда в пространстве \({ \mathbb R^{n} }\)
    \[
        V_{\delta_1}(\infty) \cap V_{\delta_2}(\infty) = \begin{icloze}{1}V_{\max \left\{ \delta_1, \delta_2 \right\}}(\infty).\end{icloze}
    \]
\end{note}

\begin{note}{be9be70ce56541a69bfeec3663baa83f}
    Пересечение двух окрестностей одной и той же точки из \({ \overline{\mathbb R^{n}} }\) --- тоже \begin{icloze}{1}окрестность этой точки.\end{icloze}
\end{note}

\begin{note}{1eab3119b89e42c7bb0b893175654a5b}
    Пусть \({ a, b \in \overline{\mathbb R^{n}},\: \begin{icloze}{2}a \neq b\end{icloze} }\).
    Тогда существуют \begin{icloze}{1}такие окрестности \({ V_{\delta_1}(a), V_{\delta_2}(b) }\), что \({ V_{\delta_1}(a) \cap V_{\delta_2}(b) = \emptyset }\).\end{icloze}
\end{note}

\begin{note}{91077edbd0ea4f8b85dd14c5a192ff70}
    Пусть \({ a, b \in \overline{\mathbb R^{n}},\: a \neq b }\).
    Тогда существуют такие окрестности \({ V_{\delta_1}(a), V_{\delta_2}(b) }\), что \({ V_{\delta_1}(a) \cap V_{\delta_2}(b) = \emptyset }\).
    Какие два случая рассматриваются в доказательстве?

    \begin{cloze}{1}
        1. \({ a, b \neq \infty }\),\: 2. \({ a }\) или \({ b = \infty }\).
    \end{cloze}
\end{note}

\begin{note}{8409f167cccb4e7988de4cebf04f3fc9}
    Пусть \({ a, b \in \overline{\mathbb R^{n}},\: a \neq b }\).
    Тогда существуют такие окрестности \({ V_{\delta_1}(a), V_{\delta_2}(b) }\), что \({ V_{\delta_1}(a) \cap V_{\delta_2}(b) = \emptyset }\).
    В чём основная идея доказательства для \({ a, b \neq \infty }\)?

    \begin{cloze}{1}
        Взять \({ \delta_1 = \delta_2 = \frac{\left\lVert a - b \right\rVert}{2} }\).
    \end{cloze}
\end{note}

\begin{note}{7eb5b28213054f63a30990a588f653c9}
    Пусть \({ a, b \in \overline{\mathbb R^{n}},\: a \neq b }\).
    Тогда существуют такие окрестности \({ V_{\delta_1}(a), V_{\delta_2}(b) }\), что \({ V_{\delta_1}(a) \cap V_{\delta_2}(b) = \emptyset }\).
    В чём основная идея доказательства для \({ b = \infty }\)?

    \begin{cloze}{1}
        Для произвольного \({ \delta_1 > 0 }\) положить \({ \delta_2 = \left\lVert a \right\rVert + \delta_1 }\).
    \end{cloze}
\end{note}

\begin{note}{6a3a01e6983e4a8cb972e4752f7aaa36}
    Пусть \({ f, g : E \subset \mathbb R^{n} \to  \mathbb R^{m} }\).
    Тогда
    \[
        (f + g)(x) \overset{\text{def}}= \begin{icloze}{1}f(x) + g(x).\end{icloze}
    \]
\end{note}

\begin{note}{431a63ea7a8f400f997c1789317d2efc}
    Пусть \({ f : E \subset \mathbb R^{n} \to \mathbb R^{m} }\),\: \begin{icloze}{2}\({ \lambda : E \to \mathbb R }\).\end{icloze}
    Тогда
    \[
        (\lambda f)(x) \overset{\text{def}}= \begin{icloze}{1}\lambda(x) f(x).\end{icloze}
    \]
\end{note}

\begin{note}{fe02a3d518ad41bcaa8eab77a41edcd0}
    Пусть \({ f, g : E \subset \mathbb R^{n} \to  \mathbb R^{m} }\).
    Тогда
    \[
        (f \cdot g)(x) \overset{\text{def}}= \begin{icloze}{1}f(x) \cdot g(x).\end{icloze}
    \]
\end{note}

\begin{note}{06ea693e03bf49faa4cea59e4de68b73}
    Чем определения предела последовательности в \({ \mathbb R^{n} }\) отличается от такового для \({ \mathbb R }\)?

    \begin{cloze}{1}
        Вместо модулей используется евклидова норма.
    \end{cloze}
\end{note}

\begin{note}{e7496a01680c4188ab2b77d49a956769}
    Пусть \({ \left\{ x^{k} \right\}_{k = 1}^{\infty} \subset \mathbb R^{n} }\),\: \begin{icloze}{4}\({ a \in \mathbb R^{n} }\).\end{icloze}
    Тогда
    \[
        \begin{icloze}{2}\lim_{k \to \infty} x^{k} = a\end{icloze} \begin{icloze}{3}\iff\end{icloze} \begin{icloze}{1}\lim_{k \to \infty} \lVert x^{k} - a \rVert = 0.\end{icloze}
    \]

    \begin{center}
        \tiny
        (в терминах вещественных последовательностей)
    \end{center}
\end{note}

\begin{note}{346e3e4f67cb4742a0a8c03f522ac66d}
    Пусть \({ \left\{ x^{k} \right\}_{k = 1}^{\infty} \subset \mathbb R^{n} }\).
    Тогда
    \[
        \begin{icloze}{2}\lim_{k \to \infty} x^{k} = \infty\end{icloze} \begin{icloze}{3}\iff\end{icloze} \begin{icloze}{1}\lim_{k \to \infty} \lVert x^{k} \rVert = +\infty.\end{icloze}
    \]

    \begin{center}
        \tiny
        (в терминах вещественных последовательностей)
    \end{center}
\end{note}

\begin{note}{181e5459680740b4b4bb06e2b4204fba}
    Пусть \({ \left\{ x^{k} \right\}_{k = 1}^{\infty} \subset \mathbb R^{n} }\),\: \({ a, b \in \begin{icloze}{2}\overline{\mathbb R^{n}}\end{icloze} }\).
    Тогда
    \[
        x^{k} \underset{k \to \infty}\longrightarrow a \: \land \: x^{k} \underset{k \to \infty}\longrightarrow b \implies \begin{icloze}{1}a = b.\end{icloze}
    \]
\end{note}

\begin{note}{a1f21b775a3b4ee5b487baccd44d3d6e}
    Пусть \({ \left\{ x^{k} \right\}_{k = 1}^{\infty} \subset \mathbb R^{n} }\),\: \({ a, b \in \overline{\mathbb R^{n}} }\).
    Тогда
    \[
        x^{k} \underset{k \to \infty}\longrightarrow a \: \land \: x^{k} \underset{k \to \infty}\longrightarrow b \implies a = b.
    \]
    В чём основная идея доказательства?

    \begin{cloze}{1}
        От обратного и использовать существование непересекающихся окрестностей \({ V(a) }\) и \({ V(b) }\).
    \end{cloze}
\end{note}

\begin{note}{a2015de5744c4595a00a42a327399672}
    Пусть \({ \left\{ x^{k} \right\}_{k = 1}^{\infty} \subset \mathbb R^{n} }\),\: \begin{icloze}{3}\({ x^{k} \to a \in \mathbb R^{n} }\).\end{icloze}
    Тогда
    \[
        \lim_{k \to \infty} \begin{icloze}{2}\lVert x^{k} \rVert\end{icloze} = \begin{icloze}{1}\lVert a \rVert.\end{icloze}
    \]
\end{note}

\begin{note}{635954ced2b945f795fa2415739e634c}
    Пусть \({ \left\{ x^{k} \right\}_{k = 1}^{\infty} \subset \mathbb R^{n} }\),\: \({ x^{k} \to a \in \mathbb R^{n} }\).
    Тогда \({ \displaystyle \lim_{k \to \infty} \lVert x^{k} \rVert = \lVert a \rVert }\).
    В чём основная идея доказательства?

    \begin{cloze}{1}
        Обратное неравенство треугольника для \({ \big\lvert \lVert x^{k} \rVert - \lVert a \rVert \big\rvert }\).
    \end{cloze}
\end{note}

\begin{note}{398fb78214164adf83ccec280065c505}
    Пусть \({ \left\{ x^{k} \right\}_{k = 1}^{\infty} \subset \mathbb R^{n} }\),\: \({ i \in [1 : n] }\).
    \begin{icloze}{1}Последовательность \({ \left\{ x_i^{k} \right\} }\)\end{icloze} называется \begin{icloze}{2}\({ i }\)-й координатной последовательностью последовательности \({ \left\{ x^{k} \right\} }\).\end{icloze}
\end{note}

\begin{note}{435eee779e914a3b8c6aaa95f6a763e7}
    Пусть \({ \left\{ x^{k} \right\}_{k = 1}^{\infty} \subset \mathbb R^{n} }\),\: \begin{icloze}{4}\({ a \in \mathbb R^{n} }\).\end{icloze}
    Тогда \begin{icloze}{2}\({ \displaystyle \lim_{k \to \infty} x^{k} = a }\)\end{icloze} \begin{icloze}{3}тогда и только тогда, когда\end{icloze} \({ \forall i \in [1 : n] }\)
    \begin{icloze}{1}
        \[
            \displaystyle \lim_{k \to \infty} x_i^{k} = a_i.
        \]
    \end{icloze}
\end{note}

\begin{note}{3fa906784d584b6e8ffd965de8a6322d}
    В чём основная идея в доказательстве теоремы о покоординатной сходимости последовательности в \({ \mathbb R^{n} }\) (необходимость)?

    \begin{cloze}{1}
        \[
            \left\lvert (x^{k} -  a)_i \right\rvert \leqslant \left\lVert x^{k} - a \right\rVert \to 0.
        \]
    \end{cloze}
\end{note}

\begin{note}{40c5520ef19c4bb39e271e243477ccd9}
    В чём основная идея в доказательстве теоремы о покоординатной сходимости последовательности в \({ \mathbb R^{n} }\) (достаточность)?

    \begin{cloze}{1}
        Показать, что \({ \left\lVert x^{k} - a \right\rVert \to 0 }\).
    \end{cloze}
\end{note}

\begin{note}{3fd0844b01144973a767e49a2c96e78e}
    Пусть \({ \left\{ x^{k} \right\}_{k = 1}^{\infty} \subset \mathbb R^{n} }\).
    Если \({ \displaystyle \lim_{k \to \infty} x^{k} = \infty }\), то координатные последовательности \({ \left\{ x_i^{k} \right\} }\) могут \begin{icloze}{1}не иметь предела.\end{icloze}
\end{note}

\begin{note}{33857f75ca574dc4808d8d21766be6fe}
    Пусть \({ \left\{ x^{k} \right\}, \left\{ y^{k} \right\}_{k = 1}^{\infty} \subset \mathbb R^{n} }\),
    \[
        x^{k} \to a \in \begin{icloze}{2}\mathbb R^{n},\end{icloze}\: y^{k} \to b \in \begin{icloze}{2}\mathbb R^{n}.\end{icloze}
    \]
    Тогда
    \[
        (x^{k} +  y^{k}) \to \begin{icloze}{1}a + b.\end{icloze}
    \]
\end{note}

\begin{note}{0d43a16117a1444dba74a29d0ca712b4}
    Пусть \({ \left\{ x^{k} \right\}_{k = 1}^{\infty} \subset \mathbb R^{n} }\), \({ \left\{ \lambda_k \right\}_{k = 1}^{\infty} \subset \mathbb R }\),
    \[
        x^{k} \to a \in \begin{icloze}{2}\mathbb R^{n},\end{icloze}\: \lambda_k \to \lambda \in \begin{icloze}{2}\mathbb R.\end{icloze}
    \]
    Тогда
    \[
        \lambda_k x^{k} \to \begin{icloze}{1}\lambda a.\end{icloze}
    \]
\end{note}

\begin{note}{daf93a77a6d5452c882b4ec1bc279fc6}
    Пусть \({ \left\{ x^{k} \right\}, \left\{ y^{k} \right\}_{k = 1}^{\infty} \subset \mathbb R^{n} }\),
    \[
        x^{k} \to a \in \begin{icloze}{2}\mathbb R^{n},\end{icloze}\: y^{k} \to b \in \begin{icloze}{2}\mathbb R^{n}.\end{icloze}
    \]
    Тогда
    \[
        x^{k} \cdot y^{k} = \begin{icloze}{1}a \cdot b.\end{icloze}
    \]
\end{note}

\begin{note}{abf40faf2fc04bae88f96f229e5991d1}
    В чём основная идея в доказательстве теоремы об арифметических операциях над пределами последовательностей в \({ \mathbb R^{n} }\)?

    \begin{cloze}{1}
        Использовать покоординатную сходимость.
    \end{cloze}
\end{note}

\begin{note}{84e580fe871e4a9099a8b893a7f7ff11}
    Пусть \({ \left\{ x^{k} \right\}_{k = 1}^{\infty} \subset \mathbb R^{n} }\),\: \({ \left\{ x^{k_i} \right\} }\) --- подпоследовательность \({ \left\{ x^{k} \right\} }\),\: \({ a \in \begin{icloze}{2}\overline{\mathbb R^{n}}\end{icloze} }\).
    Тогда
    \[
        \lim_{k \to \infty} x^{k} = a \implies \begin{icloze}{1}\lim_{i \to \infty} x^{k_i} = a.\end{icloze}
    \]
\end{note}

\begin{note}{b3b121ad972c4c2eb01797117a795d2b}
    Пусть \({ \left\{ x^{k} \right\}_{k = 1}^{\infty} \subset \mathbb R^{n} }\),\: \({ \left\{ x^{k_i} \right\} }\) --- подпоследовательность \({ \left\{ x^{k} \right\} }\),\: \begin{icloze}{2}\({ a \in \overline{\mathbb R^{n}} }\).\end{icloze}
    Тогда
    \[
        \lim_{k \to \infty} x^{k} = a \implies \begin{icloze}{1}\lim_{i \to \infty} x^{k_i} = a.\end{icloze}
    \]
\end{note}

\begin{note}{f3fc5ed889ed42c3ab9b94ad0dc2c4b6}
    В чём основная идея доказательства теоремы о пределе подпоследовательности в \({ \mathbb R^{n} }\)?

    \begin{cloze}{1}
        Свести к пределу подпоследовательности числовой последовательности.
    \end{cloze}
\end{note}

\begin{note}{2cc6bf0ec4b744219216b8f7b9131c13}
    Пусть \({ E \subset \mathbb R^{n} }\).
    Множество \({ E }\) называется \begin{icloze}{2}ограниченным,\end{icloze} если \begin{icloze}{1}\({ E \subset V_{\delta}(0) }\) для некоторого конечного \({ \delta > 0 }\).\end{icloze}

    \begin{center}
        \tiny
        (в терминах окрестностей)
    \end{center}
\end{note}

\begin{note}{3fdaf39b9f6849d3a7c1c0b77e0b2100}
    Пусть \({ E \subset \mathbb R^{n} }\).
    Множество \({ E }\) называется \begin{icloze}{2}ограниченным,\end{icloze} если \begin{icloze}{1}множество \({ \left\{ \left\lVert x \right\rVert \mid x \in E \right\} }\) ограничено.\end{icloze}

    \begin{center}
        \tiny
        (в терминах норм)
    \end{center}
\end{note}

\begin{note}{ef35bf7807bf49a4953be0fa863fa66a}
    Пусть \({ f : E \subset \mathbb R^{n} \to \mathbb R^{m} }\). Отображение \({ f }\) называется \begin{icloze}{2}ограниченным,\end{icloze} если \begin{icloze}{1}множество его значений ограничено.\end{icloze}
\end{note}

\begin{note}{c361e11e60c0426c9096ce58059852fe}
    Пусть \({ E \subset \mathbb R^{n} }\). Тогда \begin{icloze}{2}\({ E }\) ограничено\end{icloze} \begin{icloze}{3}\({ \iff }\)\end{icloze} \begin{icloze}{1}\({ \underset{x \in E}{\sup} \left\lVert x \right\rVert < +\infty }\).\end{icloze}

    \begin{center}
        \tiny
        (в терминах \({ \sup }\))
    \end{center}
\end{note}

\begin{note}{55039713f61943458ec789dbb6b145d8}
    Пусть \({ E \subset \mathbb R^{n} }\),\: \({ j \in [1 : n] }\).
    \begin{icloze}{1}Множество \({ \left\{ x_j \mid x \in E \right\} }\)\end{icloze} называется \begin{icloze}{2}проекцией \({ E }\) на \({ j }\)-ю ось.\end{icloze}
\end{note}

\begin{note}{558a15d4b2d8461a85f9900f08c2c24e}
    Пусть \({ E \subset \mathbb R^{n} }\),\: \({ j \in [1 : n] }\).
    \begin{icloze}{1}Проекция \({ E }\) на \({ j }\)-ю ось\end{icloze} обозначается \begin{icloze}{2}\({ E_j }\).\end{icloze}
\end{note}

\begin{note}{5f2ada96a96d40f288abf00365fd2a8f}
    Пусть \({ E \subset \mathbb R^{n} }\).
    Как ограниченность \({ E }\) связана с проекциями \({ E }\) на координатные оси?

    \begin{cloze}{1}
        Ограниченность \({ E }\) эквивалентна ограниченности каждой из его проекций \({ E_j }\).
    \end{cloze}
\end{note}

\begin{note}{f854719bde1446feb7b76aecad168192}
    Пусть \({ E \subset \mathbb R^{n} }\).
    Тогда ограниченность \({ E }\) эквивалентна ограниченности каждой из его проекций \({ E_j }\).
    В чём основная идея доказательства?

    \begin{cloze}{1}
        Непосредственно следует из оценки
        \[
            \left\lvert x_j \right\rvert \leqslant \left\lVert x \right\rVert \leqslant \sqrt{n} \cdot \underset{i}{\max} \left\lvert x_i \right\rvert.
        \]
    \end{cloze}
\end{note}

\begin{note}{3869f37cb0f24c2184567f8561a31e1f}
    Пусть \({ f : E \subset \mathbb R^{n} \to \mathbb R^{m} }\).
    Тогда \({ f }\) ограничено тогда и только тогда, когда \begin{icloze}{1}каждая из его координатных функций ограничена\end{icloze}.

    \begin{center}
        \tiny
        (в терминах координатных функций)
    \end{center}
\end{note}

\begin{note}{b7487016c4894d77a55418ebce7c8fc2}
    Пусть \({ f : E \subset \mathbb R^{n} \to \mathbb R^{m} }\).
    Тогда \({ f }\) ограничено тогда и только тогда, когда каждая из его координатных функций ограничена.
    В чём основная идея доказательства?

    \begin{cloze}{1}
        Множество \({ f(E) }\) ограничено \({ \iff }\) каждая из его проекций \({ f(E)_j }\) ограничена, но \({ f(E)_j = f_j(E) }\).
    \end{cloze}
\end{note}

\begin{note}{229fec40da6c48ceac33727eebf79e42}
    Пусть \({ \left\{ x^{k} \right\}_{k = 1}^{\infty} \subset \mathbb R^{n} }\).
    Тогда
    \begin{center}
        \({ \left\{ x^{k} \right\} }\) ограничена \({ \iff }\) \begin{icloze}{1}\({ \left\{ x_i^{k} \right\} }\) ограничена \({ \forall i }\).\end{icloze}
    \end{center}

    \begin{center}
        \tiny
        (в терминах координатных функций)
    \end{center}
\end{note}

\begin{note}{8166d65b01864fd88374f97ce75c3eda}
    Пусть \({ \left\{ x^{k} \right\}_{k = 1}^{\infty} \subset \mathbb R^{n} }\).
    Тогда
    \begin{center}
        \({ \left\{ x^{k} \right\} }\) ограничена \({ \iff }\) \({ \left\{ x_i^{k} \right\} }\) ограничена \({ \forall i }\).
    \end{center}
    В чём основная идея доказательства?

    \begin{cloze}{1}
        Частный случай аналогичной теоремы для произвольных отображений.
    \end{cloze}
\end{note}

\begin{note}{fec8d0a7bf5b47ecb36926b61a1818fb}
    Пусть \({ \left\{ x^{k} \right\}_{k = 1}^{\infty} \subset \mathbb R^{n} }\).
    Тогда если \({ \left\{ x^{k} \right\} }\) сходится, то она \begin{icloze}{1}ограничена\end{icloze}.
\end{note}

\begin{note}{7ac0f2188c6d4b72996dba94ed321553}
    Пусть \({ \left\{ x^{k} \right\}_{k = 1}^{\infty} \subset \mathbb R^{n} }\).
    Тогда если \({ \left\{ x^{k} \right\} }\) сходится, то она ограничена.
    В чём основная идея доказательства?

    \begin{cloze}{1}
        Покоординатная сходимость и аналогичная теорема для числовых последовательностей.
    \end{cloze}
\end{note}

\begin{note}{5f6e1b3d2fac4b1d8da33e308f6c785b}
    В чём основная идея доказательства принципа выбора Боль\-ца\-но-Вейерштрасса для последовательностей в \({ \mathbb R^{n} }\)?

    \begin{cloze}{1}
        Последовательно выбирать подпоследовательности, поочерёдно получая сходимость координатных последовательностей.
    \end{cloze}
\end{note}

\begin{note}{3e127d0b24f142e9beec4653c5ae676b}
    Пусть \({ \left\{ x^{k} \right\}_{k = 1}^{\infty} \subset \mathbb R^{n} }\).
    Тогда если \({ \left\{ x^{k} \right\} }\) \begin{icloze}{2}не ограничена,\end{icloze} то у неё есть подпоследовательность, стремящаяся к \begin{icloze}{1}\({ \infty }\).\end{icloze}
\end{note}

\begin{note}{f4a0b38770684084b9485851c73d7764}
    Пусть \({ \left\{ x^{k} \right\}_{k = 1}^{\infty} \subset \mathbb R^{n} }\).
    Тогда если \({ \left\{ x^{k} \right\} }\) не ограничена, то у неё есть подпоследовательность, стремящаяся к \({ \infty }\).
    В чём основная идея доказательства?

    \begin{cloze}{1}
        Последовательность \({ \left\{ \left\lVert x^{k} \right\rVert \right\} }\) неограниченна.
    \end{cloze}
\end{note}

\section{Лекция 14.05.22 (2)}
\begin{note}{8f1408b5072a4c4db503662e57f44fae}
    Чем определение последовательности Коши в \({ \mathbb R^{n} }\) отличается от такового для последовательностей в  \({ \mathbb R }\)?

    \begin{cloze}{1}
        Вместо модуля используются евклидова норма.
    \end{cloze}
\end{note}

\begin{note}{a55fb30e99b8448bbf4daad28b535da6}
    Выполняется ли критерий сходимости Больцано-Коши для последовательностей в \({ \mathbb R^{n} }\)?

    \begin{cloze}{1}
        Да, выполняется.
    \end{cloze}
\end{note}

\begin{note}{15ca58d336a843058f80b5e7f575bcd0}
    В чём основная идея доказательства критерия сходимости Больцано-Коши для последовательностей в \({ \mathbb R^{n} }\) (необходимость)?

    \begin{cloze}{1}
        Критерий Больцано-Коши для координатных последовательностей и неравенство
        \[
            \left\lvert x_k \right\rvert \leqslant \left\lVert x \right\rVert \leqslant \sqrt{n} \cdot \underset{i}{\max} \left\lvert x_i \right\rvert.
        \]
    \end{cloze}
\end{note}

\begin{note}{48fb62af875d46c5b05b401c40d098c4}
    Чем определение предельных точек для подмножеств \({ \mathbb R^{n} }\) отличается от такового для подмножеств \({ \mathbb R }\)?

    \begin{cloze}{1}
        Ничем.
    \end{cloze}
\end{note}

\begin{note}{a669a63c0542458f9252161374cec7f3}
    Пусть \({ E \subset \mathbb R^{n} }\),\: \begin{icloze}{4}\({ a \in \overline{\mathbb R^{n}} }\).\end{icloze} \begin{icloze}{2}Точка \({ a }\) является предельной для \({ E }\)\end{icloze} \begin{icloze}{3}тогда и только тогда, когда\end{icloze}
    \begin{icloze}{1}
        \[
            \exists \{ x^{k} \}_{k = 1}^{\infty} \subset E \setminus \left\{ a \right\} \quad x^{k} \to a.
        \]
    \end{icloze}

    \begin{center}
        \tiny (в терминах последовательностей)
    \end{center}
\end{note}

\begin{note}{15109511ff924c4baaf5512ee750e154}
    Пусть \({ E \subset \mathbb R^{n} }\),\: \({ a \in \overline{\mathbb R^{n}} }\). Точка \({ a }\) является предельной для \({ E }\) тогда и только тогда, когда
    \[
        \exists \{ x^{k} \}_{k = 1}^{\infty} \subset E \setminus \left\{ a \right\} \quad x^{k} \to a.
    \]
    Какие два случая рассматриваются в доказательстве?

    \begin{cloze}{1}
        1. \({ a \neq \infty }\); \quad 2.  \({ a = \infty }\).
    \end{cloze}
\end{note}

\begin{note}{75f06294c4514596b330d9f62883ed49}
    Пусть \({ E \subset \mathbb R^{n} }\),\: \({ a \in \overline{\mathbb R^{n}} }\). Точка \({ a }\) является предельной для \({ E }\) тогда и только тогда, когда
    \[
        \exists \{ x^{k} \}_{k = 1}^{\infty} \subset E \setminus \left\{ a \right\} \quad x^{k} \to a.
    \]
    В чём основная идея доказательства для \({ a \neq \infty }\) (необходимость)?

    \begin{cloze}{1}
        Выбирать \({ x^{k} }\) по \({ \varepsilon = \frac{1}{k} }\).
    \end{cloze}
\end{note}

\begin{note}{c78fe01a6f214aa38a3343be15a6543f}
    Пусть \({ E \subset \mathbb R^{n} }\),\: \({ a \in \overline{\mathbb R^{n}} }\). Точка \({ a }\) является предельной для \({ E }\) тогда и только тогда, когда
    \[
        \exists \{ x^{k} \}_{k = 1}^{\infty} \subset E \setminus \left\{ a \right\} \quad x^{k} \to a.
    \]
    В чём основная идея доказательства для \({ a \neq \infty }\) (достаточность)?

    \begin{cloze}{1}
        Из определения предела \({ \forall \varepsilon \quad \exists k \quad x^{k} \in \dot V_{\varepsilon}(a) }\).
    \end{cloze}
\end{note}

\begin{note}{22db3f3ad0394102a1d98af06cc40fbd}
    Пусть \({ E \subset \mathbb R^{n} }\),\: \begin{icloze}{3}\({ a \in E }\).\end{icloze}
    Точка \({ a }\) называется \begin{icloze}{2}внутренней для \({ E }\),\end{icloze} если \begin{icloze}{1}у неё есть окрестность, содержащаяся в \({ E }\).\end{icloze}
\end{note}

\begin{note}{18e311d437f341e9a873d377706cf0cb}
    Пусть \({ E \subset \mathbb R^{n} }\).
    Множество \({ E }\) называется \begin{icloze}{2}открытым в \({ \mathbb R^{n} }\),\end{icloze} если \begin{icloze}{1}любая его точка является внутренней для \({ E }\).\end{icloze}
\end{note}

\begin{note}{b8e86e0736db4159a59c29fffcb30ac8}
    Пусть \({ E \subset \mathbb R^{n} }\).
    Множество \({ E }\) называется \begin{icloze}{2}замкнутым в \({ \mathbb R^{n} }\),\end{icloze} если \begin{icloze}{1}его дополнение \({ \mathbb R^{n} \setminus E }\) открыто в \({ \mathbb R^{n} }\).\end{icloze}
\end{note}

\begin{note}{6c14a78f8f304062aa09cf96ee030f48}
    Пусть \({ E \subset \mathbb R^{n} }\).
    Тогда \begin{icloze}{2}\({ E }\) замкнуто в \({ \mathbb R^{n} }\)\end{icloze} \begin{icloze}{3}тогда и только тогда, когда\end{icloze} \begin{icloze}{1}\({ E }\) содержит все свои предельные точки в \({ \mathbb R^{n} }\).\end{icloze}

    \begin{center}
        \tiny (в терминах предельных точек)
    \end{center}
\end{note}

\begin{note}{4313f6dffa4d4be288c13720b79f5e31}
    Пусть \({ E \subset \mathbb R^{n} }\).
    Тогда \({ E }\) замкнуто в \({ \mathbb R^{n} }\) тогда и только тогда, когда \({ E }\) содержит все свои предельные точки в \({ \mathbb R^{n} }\).
    В чём основная идея доказательства (необходимость)?

    \begin{cloze}{1}
        От обратного и противоречие с открытостью \({ \mathbb R^{n} \setminus E }\).
    \end{cloze}
\end{note}

\begin{note}{338d88f6cd9b49a5b6e1fc14c46c07af}
    Пусть \({ E \subset \mathbb R^{n} }\).
    Тогда \({ E }\) замкнуто в \({ \mathbb R^{n} }\) тогда и только тогда, когда \({ E }\) содержит все свои предельные точки в \({ \mathbb R^{n} }\).
    В чём основная идея доказательства (достаточность)?

    \begin{cloze}{1}
        В \({ \mathbb R^{n} \setminus E }\) нет предельных точек \({ E }\).
    \end{cloze}
\end{note}

\begin{note}{78034d8289654f2d989fd7d3cde60d2b}
    Пусть \({ E \subset \mathbb R^{n} }\).
    Тогда \begin{icloze}{2}\({ E }\) замкнуто в \({ \mathbb R^{n} }\)\end{icloze} \begin{icloze}{3}тогда и только тогда, когда\end{icloze}
    \begin{icloze}{1}
        \[
            \lim_{k \to \infty} x^{k} \in E
        \]
        для любой сходящейся последовательности \({ \{ x^{k} \}_{k = 1}^{\infty} \subset E }\).
    \end{icloze}

    \begin{center}
        \tiny (в терминах последовательностей)
    \end{center}
\end{note}

\begin{note}{c4a98576816047bd8e45761c09ffdc70}
    Пусть \({ E \subset \mathbb R^{n} }\).
    Тогда \({ E }\) замкнуто в \({ \mathbb R^{n} }\) тогда и только тогда, когда
    \[
        \lim_{k \to \infty} x^{k} \in E
    \]
    для любой сходящейся последовательности \({ \{ x^{k} \}_{k = 1}^{\infty} \subset E }\).
    В чём основная идея доказательства (необходимость)?

    \begin{cloze}{1}
        От обратного и противоречие с открытостью \({ \mathbb R^{n} \setminus E }\).
    \end{cloze}
\end{note}

\begin{note}{2ad826deaebf4422a6d0864598a81e82}
    Пусть \({ E \subset \mathbb R^{n} }\).
    Тогда \({ E }\) замкнуто в \({ \mathbb R^{n} }\) тогда и только тогда, когда
    \[
        \lim_{k \to \infty} x^{k} \in E
    \]
    для любой сходящейся последовательности \({ \{ x^{k} \}_{k = 1}^{\infty} \subset E }\).
    В чём основная идея доказательства (достаточность)?

    \begin{cloze}{1}
        \({ E }\) содержит любую свою предельную точку.
    \end{cloze}
\end{note}

\begin{note}{0f1ca1a114fe4768af83b2987f80c626}
    Пусть \begin{icloze}{3}\({ \left\{ E_\alpha \right\}_{\alpha \in A} }\) --- семейство открытых подмножеств \({ \mathbb R^{n} }\).\end{icloze}
    Тогда \begin{icloze}{2}\({ \displaystyle \bigcup_{\alpha \in A}^{} E_\alpha }\)\end{icloze} \begin{icloze}{1}открыто в \({ \mathbb R^{n} }\).\end{icloze}
\end{note}

\begin{note}{3592d4aa6f1d49a088ea09f8f4850695}
    Пусть \begin{icloze}{4}\({ \left\{ E_\alpha \right\}_{\alpha \in A} }\) --- семейство открытых подмножеств \({ \mathbb R^{n} }\).\end{icloze}
    Тогда если \begin{icloze}{3}\({ \left\lvert A \right\rvert < \aleph_0 }\),\end{icloze} то \begin{icloze}{2}\({ \displaystyle \bigcap_{\alpha \in A}^{} E_\alpha }\)\end{icloze} \begin{icloze}{1}открыто в \({ \mathbb R^{n} }\).\end{icloze}
\end{note}

\begin{note}{d5e0544892f344dcbd519b58b207d8b6}
    Пусть \begin{icloze}{3}\({ \left\{ E_\alpha \right\}_{\alpha \in A} }\) --- семейство закрытых подмножеств \({ \mathbb R^{n} }\).\end{icloze}
    Тогда \begin{icloze}{2}\({ \displaystyle \bigcap_{\alpha \in A}^{} E_\alpha }\)\end{icloze} \begin{icloze}{1}замкнуто в \({ \mathbb R^{n} }\).\end{icloze}
\end{note}

\begin{note}{203fb961ea984b4195210820a687183d}
    Пусть \begin{icloze}{4}\({ \left\{ E_\alpha \right\}_{\alpha \in A} }\) --- семейство закрытых подмножеств \({ \mathbb R^{n} }\).\end{icloze}
    Тогда если \begin{icloze}{3}\({ \left\lvert A \right\rvert < \aleph_0 }\),\end{icloze} то \begin{icloze}{2}\({ \displaystyle \bigcup_{\alpha \in A}^{} E_\alpha }\)\end{icloze} \begin{icloze}{1}замкнуто в \({ \mathbb R^{n} }\).\end{icloze}
\end{note}

\begin{note}{cdbfd70790af41228a9ac6941706aed5}
    Пусть \({ E, F \subset \mathbb R^{n} }\). Тогда если \begin{icloze}{4}\({ E }\)\end{icloze} \begin{icloze}{3}открыто\end{icloze}, а \begin{icloze}{4}\({ F }\)\end{icloze} \begin{icloze}{3}замкнуто\end{icloze}, то \begin{icloze}{2}\({ E \setminus F }\)\end{icloze} \begin{icloze}{1}открыто.\end{icloze}
\end{note}

\begin{note}{11768bb518f845028aa77d77032444bf}
    Пусть \({ E, F \subset \mathbb R^{n} }\). Тогда если \begin{icloze}{4}\({ E }\)\end{icloze} \begin{icloze}{3}замкнуто\end{icloze}, а \begin{icloze}{4}\({ F }\)\end{icloze} \begin{icloze}{3}открыто\end{icloze}, то \begin{icloze}{2}\({ E \setminus F }\)\end{icloze} \begin{icloze}{1}замкнуто.\end{icloze}
\end{note}

\begin{note}{db6839cb255845099043c9415f55b4af}
    Открыто или замкнуто в \({ \mathbb R^{n} }\) множество \({ \mathbb R^{n} }\)?

    \begin{cloze}{1}
        И открыто, и замкнуто.
    \end{cloze}
\end{note}

\begin{note}{7f1672e6cd054b6ea16112670e15fe3d}
    Открыто или замкнуто в \({ \mathbb R^{n} }\) множество \({ \emptyset }\)?

    \begin{cloze}{1}
        И открыто, и замкнуто.
    \end{cloze}
\end{note}

\begin{note}{d23104a8197346939669f31522a74633}
    Открыто или замкнуто в \({ \mathbb R^{n} }\) произвольное конечное подмножество \({ \mathbb R^{n} }\)?

    \begin{cloze}{1}
        Замкнуто.
    \end{cloze}
\end{note}

\begin{note}{685cbdd3894a49c79e96e14be1a73996}
    Пусть \begin{icloze}{4}\({ a \in \mathbb R^{n} }\),\end{icloze}\: \begin{icloze}{3}\({ \delta > 0 }\).\end{icloze}
    \[
        \begin{icloze}{2}B_\delta(a)\end{icloze} \overset{\text{def}}= \begin{icloze}{1}\left\{ a \in \mathbb R^{n} \mid \left\lVert x - a \right\rVert < \delta \right\}.\end{icloze}
    \]
\end{note}

\begin{note}{dc873817f70d4d40b1f3c824ad89ad17}
    Пусть \begin{icloze}{4}\({ a \in \mathbb R^{n} }\),\end{icloze}\: \begin{icloze}{3}\({ \delta > 0 }\).\end{icloze}
    \[
        \begin{icloze}{2}\overline{B_\delta(a)}\end{icloze} \overset{\text{def}}= \begin{icloze}{1}\left\{ a \in \mathbb R^{n} \mid \left\lVert x - a \right\rVert \leqslant \delta \right\}.\end{icloze}
    \]
\end{note}

\begin{note}{e9a3af0fa0004802a9b0d082df09338e}
    Пусть \begin{icloze}{4}\({ a \in \mathbb R^{n} }\),\end{icloze}\: \begin{icloze}{3}\({ \delta > 0 }\).\end{icloze}
    \[
        \begin{icloze}{2}S_\delta(a)\end{icloze} \overset{\text{def}}= \begin{icloze}{1}\left\{ x \in \mathbb R^{n} \mid \left\lVert x - a \right\rVert = \delta \right\}.\end{icloze}
    \]
\end{note}

\begin{note}{f8f71a74217449c29d5e66761ce0fcb8}
    Пусть \({ a \in \mathbb R^{n} }\),\: \({ \delta > 0 }\).
    Открыто или замкнуто в \({ \mathbb R^{n} }\) множество \({ B_\delta(a) }\)?

    \begin{cloze}{1}
        Открыто.
    \end{cloze}
\end{note}

\begin{note}{537577c3bdfa4bde831b90a8d0f0615f}
    Пусть \({ a \in \mathbb R^{n} }\),\: \({ \delta > 0 }\).
    Открыто или замкнуто в \({ \mathbb R^{n} }\) множество \({ \overline{B_\delta(a)} }\)?

    \begin{cloze}{1}
        Замкнуто.
    \end{cloze}
\end{note}

\begin{note}{3230aeffdb12417ca0fdf94fb913dd39}
    Пусть \({ a \in \mathbb R^{n} }\),\: \({ \delta > 0 }\).
    Открыто или замкнуто в \({ \mathbb R^{n} }\) множество \({ S_\delta(a) }\)?

    \begin{cloze}{1}
        Замкнуто.
    \end{cloze}
\end{note}

\section{Лекция 21.05.22 (1)}
\begin{note}{59c721ad4aeb41949fb3434169a89aeb}
    Пусть \({ E \subset \mathbb R^{n} }\).
    \begin{icloze}{2}Замыканием множества \({ E }\)\end{icloze} называется \begin{icloze}{1}наименьшее по включению замкнутое множество, содержащее \({ E }\).\end{icloze}
\end{note}

\begin{note}{a94d9d6d06b6493c81f3b1e768b81375}
    Пусть \({ E \subset \mathbb R^{n} }\).
    \begin{icloze}{1}Замыкание множества \({ E }\)\end{icloze} обозначается
    \begin{icloze}{2}
        \begin{center}
            \({ \overline{E} }\) или \({ \operatorname{Cl} E }\).
        \end{center}
    \end{icloze}
\end{note}

\begin{note}{31d05cf282b9466e8a9a000950fede4b}
    Пусть \({ E \subset \mathbb R^{n} }\).
    Тогда
    \[
        \begin{icloze}{2}\overline{E}\end{icloze} = \begin{icloze}{1}\bigcap \big\{ F \subset \mathbb R^{n} \mid \text{\({ F }\) --- замкнуто},\: E \subset F \big\}.\end{icloze}
    \]
\end{note}

\begin{note}{69f36de3475242618fbb0eb4ae74b6dd}
    Пусть \({ E \subset \mathbb R^{n} }\),\: \begin{icloze}{4}\({ a \in \mathbb R^{n} }\).\end{icloze}
    Тогда
    \begin{center}
        \begin{icloze}{2}\({ a \in \overline{E} }\) \end{icloze}
        \begin{icloze}{3}\({ \iff }\)\end{icloze}
        \begin{icloze}{1}\({ a \in E }\) или \({ a }\) --- предельная точка \({ E }\).\end{icloze}
    \end{center}
\end{note}

\begin{note}{4bdd2826972c4b32bc4fdfbfcf85e6d3}
    Пусть \({ E \subset \mathbb R^{n} }\),\: \begin{icloze}{4}\({ a \in \mathbb R^{n} }\).\end{icloze}
    Тогда
    \begin{center}
        \begin{icloze}{2}\({ a \in \overline{E} }\) \end{icloze}
        \begin{icloze}{3}\({ \iff }\)\end{icloze}
        \begin{icloze}{1}\({ \exists \{ x^{k} \}_{k = 1}^{\infty} \subset E \quad x^{k} \to a }\).\end{icloze}
    \end{center}

    \begin{center}
        \tiny (в терминах последовательностей)
    \end{center}
\end{note}

\begin{note}{61bd6874e98f44a0882b56e7a750b9a0}
    Пусть \({ E \subset \mathbb R^{n} }\).
    \begin{icloze}{2}Внутренностью \({ E }\)\end{icloze} называется \begin{icloze}{1}множество всех внутренних точек \({ E }\).\end{icloze}
\end{note}

\begin{note}{139034b5d2bc422097d150cb9da30c33}
    Пусть \({ E \subset \mathbb R^{n} }\).
    \begin{icloze}{2}Внутренность \({ E }\)\end{icloze} обозначается
    \begin{icloze}{1}
        \begin{center}
            \({ \operatorname{Int} E }\) или \({ \mathring{E} }\).
        \end{center}
    \end{icloze}
\end{note}

\begin{note}{b8de5908fb8d44bf92fcd25520547f67}
    Пусть \({ E \subset \mathbb R^{n} }\).
    Открыто или закрыто в \({ \mathbb R^{n} }\) множество \({ \operatorname{Int} E }\)?

    \begin{cloze}{1}
        Открыто.
    \end{cloze}
\end{note}

\begin{note}{799638c4bc6f4c50bb965b8803d41eab}
    Пусть \({ E, G \subset \mathbb R^{n} }\).
    Тогда если \begin{icloze}{3}\({ G }\) открыто в \({ \mathbb R^{n} }\)\end{icloze} и \begin{icloze}{2}\({ G \subset E }\),\end{icloze} то \begin{icloze}{1}\({ G \subset \operatorname{Int} E }\).\end{icloze}
\end{note}

\begin{note}{cd90c5fff5534fb981914ccc37bd7ae9}
    Пусть \({ E \subset \mathbb R^{n} }\).
    \begin{icloze}{2}Внутренность \({ E }\)\end{icloze} --- это \begin{icloze}{1}наибольшее по включению открытое множество, содержащееся в \({ E }\).\end{icloze}

    \begin{center}
        \tiny (в терминах включения)
    \end{center}
\end{note}

\begin{note}{a1e8aa28d96d49799f0c4a4040f2dead}
    Пусть \({ E \subset \mathbb R^{n} }\).
    Тогда
    \[
        \begin{icloze}{2}\mathbb R^{n} \setminus \overline{E}\end{icloze} = \begin{icloze}{1}\operatorname{Int} (R^{n} \setminus E).\end{icloze}
    \]
\end{note}

\begin{note}{fb2a725e8e4b4c2fb0fa050078699cf5}
    Пусть \({ E \subset \mathbb R^{n} }\).
    Тогда
    \[
        \begin{icloze}{2}\mathbb R^{n} \setminus \operatorname{Int} E\end{icloze} = \begin{icloze}{1}\overline{(\mathbb R^{n} \setminus E)}.\end{icloze}
    \]
\end{note}

\begin{note}{fe2c858bf3c04acfa21126cdd9904aef}
    Пусть \({ E \subset \mathbb R^{n} }\).
    \begin{icloze}{2}Границей \({ E }\)\end{icloze} называется \begin{icloze}{1}
        множество
        \[
            \overline{E} \setminus \operatorname{Int} E.
        \]
    \end{icloze}
\end{note}

\begin{note}{0a35e47f857f43ec9bbff01116e04eb0}
    Пусть \({ E \subset \mathbb R^{n} }\).
    \begin{icloze}{1}Граница множества \({ E }\)\end{icloze} обозначается \begin{icloze}{2}\({ \partial E }\).\end{icloze}
\end{note}

\begin{note}{88b2a73094fb4a239e5b513a46f3a622}
    Пусть \({ E \subset \mathbb R^{n} }\).
    Открыто или замкнуто множество \({ \partial E }\)?

    \begin{cloze}{1}
        Замкнуто.
    \end{cloze}
\end{note}

\begin{note}{e5f44e996a0c40ec85bea1411325179d}
    Пусть \({ a \in \mathbb R^{n} }\),\: \({ \delta > 0 }\).
    Тогда \({ \operatorname{Int} B_\delta (a) = \begin{icloze}{1}B_\delta (a)\end{icloze} }\).
\end{note}

\begin{note}{351b99e8bd1248d0b7d36a78fbdd1807}
    Пусть \({ a \in \mathbb R^{n} }\),\: \({ \delta > 0 }\).
    Тогда \({ \operatorname{Int} \overline{B_\delta (a)} = \begin{icloze}{1}B_\delta(a)\end{icloze} }\).
\end{note}

\begin{note}{5134e609e6a542f895fa5477372047a1}
    Пусть \({ a \in \mathbb R^{n} }\),\: \({ \delta > 0 }\).
    Тогда \({ \partial B_\delta(a) = \begin{icloze}{1}S_\delta(a)\end{icloze} }\).
\end{note}

\begin{note}{6d51fdba058545838e2402ff10db74ea}
    Пусть \({ a \in \mathbb R^{n} }\),\: \({ \delta > 0 }\).
    Тогда \({ \partial(\overline{B_\delta(a)}) = \begin{icloze}{1}S_\delta(a)\end{icloze} }\).
\end{note}

\begin{note}{6091740d09a14c4da2f8435a9157b10d}
    Пусть \begin{icloze}{5}\({ E \subset \mathbb R^{n} }\),\end{icloze} \begin{icloze}{3}\({ F \subset E }\).\end{icloze}
    Точка \({ a \in \begin{icloze}{4}F\end{icloze} }\) называется \begin{icloze}{2}внутренней для \({ F }\) в \({ E }\),\end{icloze} если
    \begin{icloze}{1}
        \[
            \exists \delta > 0 \quad (V_{\delta}(a) \cap E) \subset F.
        \]
    \end{icloze}
\end{note}

\begin{note}{d9c779744732420dbf907539811051e7}
    Пусть \({ E \subset \mathbb R^{n} }\), \begin{icloze}{3}\({ F \subset E }\).\end{icloze}
    Множество \({ F }\) называется \begin{icloze}{2}открытым в \({ E }\),\end{icloze} если \begin{icloze}{1}любая его точка является внутренней для \({ F }\) в \({ E }\).\end{icloze}
\end{note}

\begin{note}{a6191d3dd3e94990aa8ee40f4b7e6d53}
    Пусть \({ E \subset \mathbb R^{n} }\), \begin{icloze}{3}\({ F \subset E }\).\end{icloze}
    Множество \({ F }\) называется \begin{icloze}{2}замкнутым в \({ E }\),\end{icloze} если \begin{icloze}{1}\({ E \setminus F }\) открыто в \({ E }\).\end{icloze}
\end{note}

\begin{note}{d0303c87476a424c80b3fb7522768304}
    Пусть \({ F \subset E \subset \mathbb R^{n} }\).
    Тогда \begin{icloze}{2}\({ F }\) открыто в \({ E }\)\end{icloze} \begin{icloze}{3}\({ \iff }\)\end{icloze}
    \begin{icloze}{1}
        существует открытое множество \({ G \subset \mathbb R^{n} }\), для которого
        \[
            F = E \cap G.
        \]
    \end{icloze}

    \begin{center}
        \tiny (в терминах открытости/замкнутости в \({ \mathbb R^{n} }\))
    \end{center}
\end{note}

\begin{note}{33888ff338f94ea59f1f245c0f5ae61b}
    Пусть \({ F \subset E \subset \mathbb R^{n} }\).
    Тогда \begin{icloze}{2}\({ F }\) замкнуто в \({ E }\)\end{icloze} \begin{icloze}{3}\({ \iff }\)\end{icloze} существует
    \begin{icloze}{1}
        замкнутое множество \({ G \subset \mathbb R^{n} }\), для которого
        \[
            F = E \cap G.
        \]
    \end{icloze}

    \begin{center}
        \tiny (в терминах открытости/замкнутости в \({ \mathbb R^{n} }\))
    \end{center}
\end{note}

\begin{note}{c75d8ade85014bf2a2e7c13cc3299418}
    Пусть \({ F \subset E \subset \mathbb R^{n} }\).
    Если \begin{icloze}{4}\({ E }\) открыто в \({ \mathbb R^{n} }\),\end{icloze} то
    \begin{center}
        \begin{icloze}{2}\({ F }\) открыто в \({ E }\)\end{icloze} \begin{icloze}{3}\({ \iff }\)\end{icloze} \begin{icloze}{1}\({ F }\) открыто в \({ \mathbb R^{n} }\).\end{icloze}
    \end{center}
\end{note}

\begin{note}{8607c43c4d9145ce8b739f277bc467f5}
    Пусть \({ E \subset \mathbb R^{n} }\).
    \begin{icloze}{1}Семейство \({ \Omega }\) подмножеств \({ \mathbb R^{n} }\) такое, что \({ E \subset \bigcup_{A \in \Omega}^{} A }\),\end{icloze} называется \begin{icloze}{2}покрытием множества \({ E }\).\end{icloze}
\end{note}

\begin{note}{89d8d5966eba46478e132c99f8a87207}
    Пусть \({ E \subset \mathbb R^{n} }\).
    Если \({ \Omega }\) --- это покрытие \({ E }\), то говорят, что \({ \Omega }\) \begin{icloze}{1}покрывает \({ E }\).\end{icloze}
\end{note}

\begin{note}{69ac5bea935247e8ba609ec7be95b65f}
    Пусть \({ E \subset \mathbb R^{n} }\),\: \({ \Omega }\) --- покрытие \({ E }\).
    \begin{icloze}{1}Подсемейство \({ \Omega }\), которое также покрывает \({ E }\),\end{icloze} называется \begin{icloze}{2}подпокрытием \({ \Omega }\).\end{icloze}
\end{note}

\begin{note}{8e8ddb4f754b447b863761db2945f748}
    Пусть \({ E \subset \mathbb R^{n} }\),\: \({ \Omega }\) --- покрытие \({ E }\).
    Покрытие \({ \Omega }\) называется \begin{icloze}{2}открытым,\end{icloze} если \begin{icloze}{1}любое множество \({ A \in \Omega }\) открыто.\end{icloze}
\end{note}

\begin{note}{33da2e11009a4fe9a561efdb024b9f9e}
    Пусть \({ E \subset \mathbb R^{n} }\).
    Множество \({ E }\) называется \begin{icloze}{2}компактным (или компактом),\end{icloze} если \begin{icloze}{1}из любого открытого покрытия \({ E }\) можно выбрать конечное подпокрытие.\end{icloze}
\end{note}

\begin{note}{cdb5bd07b1d340dd9b11c009a8d0fa29}
    Пусть \begin{icloze}{3}\({ E \subset \mathbb R^{n} }\) компактно.\end{icloze}
    Тогда любое \begin{icloze}{1}замкнутое\end{icloze} подмножество \({ E }\) \begin{icloze}{2}также компактно.\end{icloze}
\end{note}

\begin{note}{6efd282fcad744ba898c1ed2ad57dcb2}
    Пусть \({ E \subset \mathbb R^{n} }\).
    Тогда
    \begin{center}
        \begin{icloze}{2}\({ E }\) компактно\end{icloze} \begin{icloze}{3}\({ \iff }\)\end{icloze} \begin{icloze}{1}\({ E }\) замкнуто и ограничено.\end{icloze}
    \end{center}

    \begin{center}
        \tiny <<Теорема \begin{icloze}{4}Гейне-Бореля\end{icloze}>>
    \end{center}
\end{note}

\begin{note}{d8a0467fa82d4e8db762642af19e20b2}
    Пусть \({ E \subset \mathbb R^{n} }\).
    Тогда
    \begin{center}
        \begin{icloze}{2}\({ E }\) компактно \end{icloze}
        \begin{icloze}{3}\({ \iff }\)\end{icloze}
        \begin{icloze}{1}\({ \displaystyle \forall \{ x^{k} \}_{k = 1}^{\infty} \subset E \quad \exists \{ x^{k_i} \} \quad \lim_{i \to \infty} x^{k_i} \in E }\).\end{icloze}
    \end{center}

    \begin{center}
        \tiny (в терминах последовательностей)
    \end{center}
\end{note}

\section{Лекция 21.05.22 (2)}
\begin{note}{eab6566f65bf4003bec24142b9b187e5}
    \begin{icloze}{2}Отображение \({ E \subset R \to \mathbb R^{m} }\)\end{icloze} называется \begin{icloze}{1}вектор-фун\-кци\-ей.\end{icloze}
\end{note}

\begin{note}{ff7b125dc2ee4647b62cf48f1c0edc4c}
    Чем определение предела \({ E \subset \mathbb R^{n} \to \mathbb R^{m} }\) по Коши принципиально отличается от такового для функций \({ E \subset \mathbb R \to \mathbb R }\)?

    \begin{cloze}{1}
        Ничем.
    \end{cloze}
\end{note}

\begin{note}{21646334c02042498ac8c3104a6f0b24}
    Чем определение предела отображения \({ E \subset \mathbb R^{n} \to \mathbb R^{m} }\) по Гейне отличается от такового для функций \({ E \subset \mathbb R \to \mathbb R }\)?

    \begin{cloze}{1}
        Ничем.
    \end{cloze}
\end{note}

\begin{note}{911558d747cb439f8cd881a764ff8700}
    Пусть \({ f : E \subset \mathbb R^{n} \to \mathbb R^{m} }\),\: \begin{icloze}{2}\({ a, A, B \in \overline{\mathbb R^{m}} }\).\end{icloze}
    Тогда
    \[
        f(x) \underset{x \to a}\longrightarrow A \quad \land \quad f(x) \underset{x \to a}\longrightarrow B \implies \begin{icloze}{1}A = B.\end{icloze}
    \]
\end{note}

\begin{note}{67a1a0e4aa664174a6c43da05087c80d}
    Пусть \({ f : E \subset \mathbb R^{n} \to \mathbb R^{m} }\),\: \begin{icloze}{5}\({ a \in \overline{\mathbb R^{m}} }\),\end{icloze}\: \begin{icloze}{4}\({ A \in \mathbb R^{m} }\).\end{icloze}
    Тогда
    \[
        \begin{icloze}{2}f(x) \underset{x \to a}\longrightarrow A\end{icloze}
        \begin{icloze}{3}\iff\end{icloze}
        \begin{icloze}{1}\forall i \quad f_i(x) \underset{x \to a}\longrightarrow A_i.\end{icloze}
    \]

    \begin{center}
        \tiny (в терминах координатных функций)
    \end{center}
\end{note}

\begin{note}{897f1778b44c472ea1484dc061603ea0}
    Пусть \({ f, g : E \subset \mathbb R^{n} \to \mathbb R^{m} }\),\: \({ a }\) --- предельная точка \({ E }\),
    \[
        f(x) \underset{x \to a}\longrightarrow A \in \mathbb R^{m}, \quad g(x) \underset{x \to a}\longrightarrow B \in \mathbb R^{m}.
    \]
    Тогда если \begin{icloze}{1}\({ \top }\),\end{icloze} то
    \[
        (f(x) + g(x)) \underset{x \to a}\longrightarrow A + B.
    \]
\end{note}

\begin{note}{339c716c8963455ba9dc06fb9cac6365}
    Пусть \({ f : E \subset  \mathbb R^{n} \to \mathbb R^{m} }\),\: \({ \lambda : E \to \mathbb R }\),\: \({ a }\) --- предельная точка \({ E }\),
    \[
        f(x) \underset{x \to a}\longrightarrow A \in \mathbb R^{n}, \quad \lambda(x) \underset{x \to a}\longrightarrow \alpha \in \mathbb R.
    \]
    Тогда если \begin{icloze}{1}\({ \top }\),\end{icloze} то
    \[
        \lambda(x) f(x) \underset{x \to a}\longrightarrow \alpha A.
    \]
\end{note}

\begin{note}{61daf1572b2540f3809ed6f770b68fe2}
    Пусть \({ f, g : E \subset \mathbb R^{n} \to \mathbb R^{m} }\),\: \({ a }\) --- предельная точка \({ E }\),
    \[
        f(x) \underset{x \to a}\longrightarrow A \in \mathbb R^{m}, \quad g(x) \underset{x \to a}\longrightarrow B \in \mathbb R^{m}.
    \]
    Тогда если \begin{icloze}{1}\({ \top }\),\end{icloze} то
    \[
        f(x) \cdot g(x) \underset{x \to a}\longrightarrow A \cdot B.
    \]
\end{note}

\begin{note}{a80a77deaa7f4a91a4c9e4898c2d54a9}
    Пусть \({ f, g : E \subset \mathbb R^{n} \to \mathbb R^{m} }\),\: \({ a }\) --- предельная точка \({ E }\),
    \[
        f(x) \underset{x \to a}\longrightarrow A \in \mathbb R^{m}, \quad g(x) \underset{x \to a}\longrightarrow B \in \mathbb R^{m}.
    \]
    Тогда если \begin{icloze}{1}\({ m = 1 }\) и \({ B \neq 0 }\),\end{icloze} то
    \[
        \frac{f}{g}(x) \underset{x \to a}\longrightarrow \frac{A}{B}.
    \]
\end{note}

\begin{note}{d5a6f58c526346a3868af995d0553564}
    Пусть \begin{icloze}{4}\({ g : F \subset \mathbb R^{m} \to \mathbb R^{l} }\),\: \({ f : E \subset \mathbb R^{n} \to F }\),\end{icloze}\: \begin{icloze}{3}\({ a \in \overline{\mathbb R^{n}} }\) и \({ b \in F }\) --- предельные точки \({ E }\) и \({ F }\) соответственно.\end{icloze} Тогда если
    \begin{icloze}{2}
        \[
            f(x) \underset{x \to a}\longrightarrow b, \quad g(x) \underset{x \to b}\longrightarrow g(b),
        \]
    \end{icloze}
    то
    \begin{icloze}{1}
        \[
            (g \circ f)(x) \underset{x \to a}\longrightarrow g(b)
        \]
    \end{icloze}
\end{note}

\begin{note}{1fe3bd0f7cca46f3926bbb096e417e7c}
    Чем критерий Больцано-Коши для отображений \({ E \subset \mathbb R^{n} \to \mathbb R^{m} }\) отличается от такового для функций \({ E \subset \mathbb R \to \mathbb R }\)?

    \begin{cloze}{1}
        Ничем.
    \end{cloze}
\end{note}

\end{document}
