\documentclass[11pt, a5paper]{article}
\usepackage[width=10cm, top=0.5cm, bottom=2cm]{geometry}

\usepackage[T1,T2A]{fontenc}
\usepackage[utf8]{inputenc}
\usepackage[english,russian]{babel}
\usepackage{libertine}

\usepackage{amsmath}
\usepackage{amssymb}
\usepackage{amsthm}
\usepackage{mathrsfs}
\usepackage{framed}
\usepackage{xcolor}

\setlength{\parindent}{0pt}

% Force \pagebreak for every section
\let\oldsection\section
\renewcommand\section{\pagebreak\oldsection}

\renewcommand{\thesection}{}
\renewcommand{\thesubsection}{Note \arabic{subsection}}
\renewcommand{\thesubsubsection}{}
\renewcommand{\theparagraph}{}

\newenvironment{note}[1]{\goodbreak\par\subsection{\hfill \color{lightgray}\tiny #1}}{}
\newenvironment{cloze}[2][\ldots]{\begin{leftbar}}{\end{leftbar}}
\newenvironment{icloze}[2][\ldots]{%
  \ignorespaces\text{\tiny \color{lightgray}\{\{c#2::}\hspace{0pt}%
}{%
  \hspace{0pt}\text{\tiny\color{lightgray}\}\}}\unskip%
}


\begin{document}
\section{Лекция 07.02.22}
\begin{note}{662fbe59ca984f5b820ad1041f1eb840}
    Пусть \( f(x) : D \subset \mathbb R  \to \mathbb R, a \in D. \)
    \begin{icloze}{2}Многочлен \( p(x) \) степени \( n \)  такой, что \[
        \begin{gathered}
            f(x) = p(x) + o((x - a)^{n}), \\
            f(a) = p(a),
        \end{gathered}
    \]\end{icloze}
    называется \begin{icloze}{1}многочленом Тейлора функции \( f \) порядка \( n \) в точке \( a. \)\end{icloze}
\end{note}

\begin{note}{738279ec323b45e29a170a4e41b4bce0}
    Если многочлен Тейлора функции \( f \) порядка \( n \) в точке \( a \) существует, то \begin{icloze}{1}он единственен.\end{icloze}
\end{note}

\begin{note}{8f605243b193465799ba06e1576d171e}
    В чём ключевая идея доказательства единственности многочлена Тейлора?

    \begin{cloze}{1}
        Пусть коэффициент \( r_m  \) при \( (x - a)^{m} \) --- первый ненулевой коэффициент в многочлене \( p - q \).
        Тогда \[
            \frac{p - q}{(x - a)^{m} } \underset{x \to a}\longrightarrow r_m,
        \] но при этом \[
            \frac{p - q}{(x - a)^{m} } = o((x - a)^{n - m}) \underset{x \to a}\longrightarrow 0 \implies r_m = 0.
        \]
    \end{cloze}
\end{note}

\begin{note}{f4110a9b63c640be96d810d835d0d1fd}
    \begin{icloze}{2}Многочлен Тейлора функции \( f \) порядка \( n \) в точке \( a \)\end{icloze} обозначается \begin{icloze}{1}\( T_{a, n} f.  \)\end{icloze}
\end{note}

\begin{note}{1b7244a616994615a1d41bbc85768a3f}
    \subsubsection{<<\begin{icloze}{3}Формула Тейлора для многочленов\end{icloze}>>}

    Пусть \( p \) --- \begin{icloze}{2}многочлен степени не более \( n \).\end{icloze} Тогда \begin{icloze}{1}
        \[
            p(x) = \sum_{k=0}^{n} \frac{p^{(k)} (a)}{k! } (x - a)^{k}.
        \]
    \end{icloze}
\end{note}

\begin{note}{97c12315facb454e987cb94fae99be75}
\[
    \left. f(x) \right|_{x = a} \overset{\text{def}}= \begin{icloze}{1}f(a).\end{icloze}
\]
\end{note}

\begin{note}{cf7e5ab30b564c139557fd0a940f8204}
    \[
        \left. \left((x - a)^{k} \right)^{(n)}  \right|_{x = a} =
        \begin{icloze}{1}
            \begin{cases}
                0, & n \neq k, \\
                n!, & n = k.
            \end{cases}
        \end{icloze}
    \]
\end{note}

\begin{note}{9b6c61f4867142bea860ca4d00c07174}
    В чем основная идея доказательства истинности формулы Тейлора для многочленов?

    \begin{cloze}{1}
        Записать \( p(x) \) с неопределенными коэффициентами и вычислить \( p^{(k)} (a) \) для \( k = 0, 1, 2, \ldots, n \).
    \end{cloze}
\end{note}

\begin{note}{7597b782ce5f4e92998cc6445ce6f40e}
    \subsubsection{<<\begin{icloze}{3}Свойство n раз дифференцируемой функции\end{icloze}>>}

    Пусть \begin{icloze}{2}\( f : D \subset \mathbb R \to \mathbb R, a \in D \) и \[
        f(a) = f'(a) = \cdots = f^{(n)} (a) = 0.
    \]\end{icloze}
    Тогда \begin{icloze}{1}\( f(x) = o((x - a)^{n} ),  x \to a \).\end{icloze}
\end{note}

\begin{note}{22aa07051d4c4e0ebb08ce0114be5429}
    \subsubsection{<<Определение \( o \)-малого в терминах \( \varepsilon, \delta. \)>>}

    Пусть \( f, g : D \subset \mathbb R \to \mathbb R \), \( a \) --- предельная точка \( D \). Тогда
    \begin{multline*}
        f(x) = o(g(x)), \quad x \to a \overset{\hspace{-1pt}\text{\tiny def}}\iff \\
        \begin{icloze}{1}
            \forall \varepsilon > 0 \quad \exists \delta > 0 \quad \forall x \in D \cap \dot V_{\delta}(a) \quad |f(x)| \leqslant \varepsilon |g(x)|.
        \end{icloze}
    \end{multline*}
\end{note}

\begin{note}{b7ddf1bbcdf84c769dd7b409e5be494d}
    Какой метод используется в доказательстве свойства \( n \)-раз дифференцируемой функции?

    \begin{cloze}{1}
        Индукция по \( n. \)
    \end{cloze}
\end{note}

\begin{note}{f04179797fd64614827341d425616341}
    Какова основная идея в доказательстве свойства \( n \)-раз дифференцируемой функции (базовый случай)?

    \begin{cloze}{1}
        Подставить \( f(a) = f'(a) = 0 \) в определение дифференцируемости.
    \end{cloze}
\end{note}

\begin{note}{7a10e93958724ee6b93bc1637a13773f}
    Каков первый шаг в доказательстве свойства \( n \)-раз дифференцируемой функции (индукционный переход)?

    \begin{cloze}{1}
        Заметить, что из индукционного предположения
        \[
            f'(x) = o((x - a)^{n} )
        \]
        и расписать это равенство в терминах \( \varepsilon, \delta. \)
    \end{cloze}
\end{note}

\begin{note}{b863b13c8a8b45c09c6444b48e5c0b75}
    Какие ограничения накладываются на \( \delta \) в доказательстве свойства \( n \)-раз дифференцируемой функции (индукционный переход)?

    \begin{cloze}{1}
        \( V_{\delta} (a) \cap D \) есть невырожденный промежуток.
    \end{cloze}
\end{note}

\begin{note}{2506d5781f234e13a94358880699831a}
    Почему в доказательстве свойства \( n \)-раз дифференцируемой функции (индукционный переход) мы можем сказать, что
    \( \exists \delta > 0 \) такой,  что \( V_{\delta} (a) \cap D \) есть невырожденный отрезок?

    \begin{cloze}{1}
        По определению дифференцируемости функции.
    \end{cloze}
\end{note}

\begin{note}{73ed2cdbb8b444ce991d587d9ed279ed}
    В чем ключевая идея доказательства свойства \( n \)-раз дифференцируемой функции (индукционный переход)?

    \begin{cloze}{1}
        Выразить \( f(x) = f'(c) \cdot (x - a) \) по симметричной формуле конечных приращений и показать, что \( |f'(c)| < \varepsilon |x - a|^{n} \).
    \end{cloze}
\end{note}

\begin{note}{a08796d96ad841bd91a8e7daaab1857d}
    Откуда следует, что \( |f'(c)| < \varepsilon|x - a|^{n}  \) в доказательстве свойства \( n \)-раз дифференцируемой функции (индукционный переход)?

    \begin{cloze}{1}
        \[
            |c - a| < \delta \implies |f'(c)| < \varepsilon|c - a|^{n} < \varepsilon|x - a|^{n}
        \]
    \end{cloze}
\end{note}

\begin{note}{957fd9747bd84545bd6b1cca723d72ba}
    Пусть \( f : D \subset \mathbb R \to \mathbb R, a \in D, n \in \mathbb N \), \begin{icloze}{2}\(f(a) = 0, \)
    \[
        f'(x) = o((x - a)^{n} ), \quad x \to a.
    \]\end{icloze}

    Тогда \( f(x) = \begin{icloze}{1}o((x - a)^{n + 1}), \quad x \to a.\end{icloze} \)
\end{note}

\begin{note}{99a8f041e1a34dba923a682c6500c46b}
    \subsubsection{<<\begin{icloze}{3}Формула Тейлора-Пеано\end{icloze}>>}

    Пусть \begin{icloze}{2}\( f : D \subset R \to \mathbb R \) и \( f \) \( n \) раз дифференцируема в точке \( a. \)  \end{icloze}
    Тогда \begin{icloze}{1}\[
        f(x) = \sum_{k=0}^{n} \frac{f^{(k)} (a)}{k!} (x - a)^{k} + o((x - a)^{n} ).
    \]\end{icloze}
\end{note}

\section{Лекция 11.02.22}
\begin{note}{3bf65c72c3374838aecaa626de8a3a4d}
    Каков первый шаг в доказательстве истинности формулы Тейлора-Пеано?

    \begin{cloze}{1}
        Обозначить через \( p(x) \) многочлен в формуле:
        \[
            f(x) = \underbrace{\sum_{k=0}^{n} \frac{f^{(k)} (a)}{k!} (x - a)^{k}}_{p(x)}  + o((x - a)^{n} ).
        \]
    \end{cloze}
\end{note}

\begin{note}{6f41684761ec41308bf9f95619ec1849}
    Чему для \( k \leqslant n \) равна \( p^{(k)} (a) \) в доказательстве истинности формулы Тейлора-Пеано?

    \begin{cloze}{1}
        \[
            p^{(k)} (a) = f^{(k)} (a).
        \]
    \end{cloze}
\end{note}

\begin{note}{72455c0671414c80aca4c9ef2ba63d44}
    В чем основная идея доказательства истинности формулы Тейлора-Пеано?

    \begin{cloze}{1}
        По свойству \( n \) раз дифференцируемой функции \( f(x) - p(x) = o((x - a)^{n} )\).
    \end{cloze}
\end{note}

\begin{note}{db6e4a55afed4c5d95a38869cf9d2e00}
    Что позволяет применить свойство \( n \) раз дифференцируемой функции в доказательстве формулы Тейлора-Пеано?

    \begin{cloze}{1}
        \[
            \forall k \left. \leqslant n \quad \left(f(x) - p(x)\right)^{(k)} \right|_{x = a} = 0
        \]
    \end{cloze}
\end{note}

\begin{note}{8c823210f5c94ab99024c3e8c3d6778a}
    \[
        \begin{icloze}{2}\Delta _{a, b}\end{icloze}
        \overset{\text{def}}=
        \begin{icloze}{1}\begin{cases}
            [a, b], & a \leqslant b, \\
            [b, a], & a \geqslant b.
        \end{cases}\end{icloze}
    \]
\end{note}

\begin{note}{9755fb6343494fa9b0034b4542e518d3}
    \[
        \begin{icloze}{2}\widetilde \Delta _{a, b}\end{icloze}
        \overset{\text{def}}=
        \begin{icloze}{1}\begin{cases}
            (a, b), & a < b, \\
            (b, a), & a > b.
        \end{cases}\end{icloze}
    \]
\end{note}

\begin{note}{dbb25fcd6e834aa2ae54ec6ddc0c6787}
    \[
        \begin{icloze}{2}R_{a, n} f \end{icloze}
        \overset{\text{def}}=
        \begin{icloze}{1}f - T_{a, n} f\end{icloze}
    \]
\end{note}

\begin{note}{0d92b12a18f34554a0251578aa811b7f}
    \subsubsection{<<\begin{icloze}{3}Формула Тейлора-Лагранжа\end{icloze}>>}

    Пусть \( f : D \subset \mathbb R \to \mathbb R \), \quad \( a, x \in \mathbb R, a \neq x \), \quad \begin{icloze}{2}\( f \in C^{n} (\Delta _{a, x} ) \), \( f^{(n)} \) дифференцируема на \( \widetilde \Delta _{a, x}  \).\end{icloze} Тогда
    \begin{icloze}{1}
        найдется \( c \in \widetilde \Delta _{a, x}  \), для которой
        \[
            f(x) = T_{a, n} f (x) + \frac{f^{(n + 1)} (c)}{(n + 1)! }  (x - a)^{n + 1}.
        \]
    \end{icloze}
\end{note}

\begin{note}{f9314b4b0e184f52826c8f740c873e21}
    При \( n = 0 \) формула Тейлора-Лагранжа эквивалентна \begin{icloze}{1}теореме Лагранжа\end{icloze}.
\end{note}

\begin{note}{5fe508cfd3c445c4b15093e8d2c8c504}
    В чем основная идея доказательства истинности формулы Тейлора-Лагранжа?

    \begin{cloze}{1}
        Вычислить производную функции \( F(t) = R_{t, n} f(x) \) и найти точку \( c \) по теореме Коши.
    \end{cloze}
\end{note}

\begin{note}{e1a329fbc3ef4c5981773d8baad7d3b1}
    Для каких \( t \) определяется функция \( F(t) \) в доказательстве истинности формулы Тейлора-Лагранжа?

    \begin{cloze}{1}
        \[
            t \in \Delta _{a, x}.
        \]
    \end{cloze}
\end{note}

\begin{note}{a4f7e43161cc4c9fb58ac7a250610c50}
    Для каких \( t \) вычисляется \( F'(t) \) в доказательстве истинности формулы Тейлора-Лагранжа?

    \begin{cloze}{1}
        \[
            t \in \widetilde \Delta _{a, x}.
        \]
    \end{cloze}
\end{note}

\begin{note}{73e4df5e1b074010a95ee5dbe0458338}
    К каким функциям применяется теорема Коши в доказательстве истинности формулы Тейлора-Лагранжа?

    \begin{cloze}{1}
        К \( F(t) \) и \( \varphi(t) = (x - t)^{n + 1}  \).
    \end{cloze}
\end{note}

\begin{note}{b1d63dae062e4a438ceb891f94a33e96}
    К каким точкам применяется теорема Коши в доказательстве истинности формулы Тейлора-Лагранжа?

    \begin{cloze}{1}
        К границам отрезка \( \Delta _{a, x}  \).
    \end{cloze}
\end{note}

\begin{note}{b8f3f99b66794d59b6fa546eb06d7fb3}
    Какое неявное условие позволяет применить теорему коши к функциям \( F(t) \) и \( \varphi(t) \) с точках \( a \) и \( x \)?

    \begin{cloze}{1}
        \[
            F(x) = 0, \quad \varphi(x) = 0.
        \]
    \end{cloze}
\end{note}

\begin{note}{e425a1ef13124799b6b391e3884f86f1}
    По формуле Тейлора-Пиано при \( x \to 0 \)
    \[
        \begin{icloze}{2}e^{x}\end{icloze} = \begin{icloze}{1}\sum_{k=0}^{n} \frac{x^{k} }{k! } + o(x^{n} ).\end{icloze}
    \]
\end{note}

\begin{note}{70a13102af174271b95762b24e6b1169}
    По формуле Тейлора-Пиано при \( x \to 0 \)
    \[
        \begin{icloze}{2}\sin x\end{icloze} = \begin{icloze}{1}\sum_{k=0}^{n} (-1)^{k} \frac{x^{2k + 1} }{(2k + 1)! } + o\!\left(x^{2n + 2}\right). \end{icloze}
    \]
\end{note}

\begin{note}{9c528f645b0741ef90f268989f7701eb}
    По формуле Тейлора-Пиано при \( x \to 0 \)
    \[
        \begin{icloze}{2}\cos x\end{icloze} = \begin{icloze}{1}\sum_{k=0}^{n} (-1)^{k} \frac{x^{2k} }{(2k)! } + o\!\left(x^{2n + 1}\right). \end{icloze}
    \]
\end{note}

\begin{note}{90ff22c33f67493fae3fa800e93905f4}
    По формуле Тейлора-Пиано при \( x \to 0 \)
    \[
        \begin{icloze}{2}\ln (1 + x)\end{icloze} = \begin{icloze}{1}\sum_{k=1}^{n} (-1)^{k - 1} \frac{x^{k} }{k } + o\!\left(x^{n}\right). \end{icloze}
    \]
\end{note}

\begin{note}{aaf8ef38d3bb409baf7c7fcc1df14f48}
    \begin{icloze}{3}Обобщённый биномиальный коэффициент\end{icloze} задаётся формулой
    \[
        C_\alpha^k = \begin{icloze}{1}\frac{\alpha(\alpha - 1) \cdots (\alpha - k + 1)}{k! }\end{icloze}, \quad \alpha \in \begin{icloze}{2}\mathbb R\end{icloze}.
    \]
\end{note}

\begin{note}{5ed01e7f4e8e4b22adf1929f60e4d4f5}
    По формуле Тейлора-Пиано при \( x \to 0 \)
    \[
        \begin{icloze}{2}(1 + x)^{\alpha} \end{icloze} = \begin{icloze}{1}\sum_{k=0}^{n} C_\alpha^k x^{k} + o\!\left(x^{n}\right). \end{icloze}
    \]
\end{note}

\begin{note}{eb36b5f5a2b04e44b4d5b13d2278ff40}
    Формулу Тейлора-Пеано для \( (1 + x)^{\alpha} \) называют \begin{icloze}{1}биномиальным разложением\end{icloze}.
\end{note}

\begin{note}{c766c427b7e44be8a2e40e872ec7dd2b}
    \[
        C_{-1}^{k} = \begin{icloze}{1}(-1)^{k}.\end{icloze}
    \]
\end{note}

\begin{note}{82717b22134b4f66b014c17df3ba337c}
    По формуле Тейлора-Пиано при \( x \to 0 \)
    \[
        \begin{icloze}{2}(1 + x)^{-1} \end{icloze} = \begin{icloze}{1}\sum_{k=0}^{n} (-1)^{k} x^{k} + o\!\left(x^{n}\right). \end{icloze}
    \]
\end{note}

\begin{note}{7d3d35d9fcb344458f0d82ed7b2d940f}
    Пусть \begin{icloze}{3}функция \( f \) удовлетворяет условиям для разложения по формуле Тейлора-Лагранжа.\end{icloze}
    Тогда если
    \begin{icloze}{2}
        \[
            \forall t \in \widetilde \Delta _{a, x} \quad |f^{(n + 1)} (t)| \leqslant M,
        \]
    \end{icloze}
    то
    \begin{icloze}{1}
        \[
            |R_{a, n} f(x)| \leqslant \frac{M|x - a|^{n + 1} }{(n + 1)! }.
        \]
    \end{icloze}
\end{note}

\section{Семинар 17.02.22}
\begin{note}{05fb49aabf444b3daf73947c33bf8f10}
    \[
        \int x^{n} \: dx = \begin{icloze}{1}\frac{x^{n+1} }{n + 1} + C\end{icloze}, \quad (\begin{icloze}{2}n \neq -1\end{icloze}).
    \]
\end{note}

\begin{note}{3eae90c7fe9944e6a9d07784205f0d1d}
    \[
        \int \begin{icloze}{2}\frac{1}{x}\end{icloze} \: dx = \begin{icloze}{1}\ln |x| + C\end{icloze}.
    \]
\end{note}

\begin{note}{af533d11b4c2421baaad26c4fca61b2a}
    \[
        \int \begin{icloze}{2}\frac{1}{1 - x^2 }\end{icloze} \: dx = \begin{icloze}{1}\frac{1}{2} \ln\left|\frac{1 + x}{1 - x}\right| + C\end{icloze}.
    \]
\end{note}

\begin{note}{8939b90686dc43ae81c37c01fa728294}
    \[
        \int \begin{icloze}{2}\frac{1}{\sqrt{x^2 \pm 1} }\end{icloze} \: dx = \begin{icloze}{1}\ln |x + \sqrt{x^2 \pm 1}| + C\end{icloze}.
    \]
\end{note}

\begin{note}{709b5fa5f404426ea7b67b17dc16f830}
    \[
        \int a^{x} \: dx = \begin{icloze}{1}\frac{a^{x} }{\ln a} + C\end{icloze}.
    \]
\end{note}

\section{Лекция 18.02.22}
\begin{note}{b55d92bf361d4e31b5e60975656b3fb4}
    Пусть \begin{icloze}{4}\( f \in C\langle A, B \rangle  \) и дифференцируема на \( (A, B) \).\end{icloze} Тогда
    \begin{itemize}
        \item {}\begin{icloze}{2}\( f \!\nearrow \) на \( \langle A, B \rangle  \)\end{icloze}
            \begin{icloze}{3}\( \iff  \)\end{icloze}
            \begin{icloze}{1}\( f'(x) \geqslant 0 \quad \forall x \in (A, B) \).\end{icloze}
    \end{itemize}
\end{note}

\begin{note}{eb69e8bd92104c0ab3b235de95941521}
    Каков основной шаг в доказательстве критерия возрастания функции на промежутке (необходимость)?

    \begin{cloze}{1}
        Показать, что произвольное разностное отношение неотрицательно.
    \end{cloze}
\end{note}

\begin{note}{7d9850f850c2465aa217f34c4dbd1a66}
    Каков основной шаг в доказательстве критерия возрастания функции на промежутке (достаточность)?

    \begin{cloze}{1}
        Выразить для \( a < b \) разность \( f(b) - f(a) \) через формулу конечных приращений.
    \end{cloze}
\end{note}

\begin{note}{63e919dff3ba4ea282cb06d25b445300}
    Пусть \begin{icloze}{4}\( f \in C\langle A, B \rangle  \) и дифференцируема на \( (A, B) \).\end{icloze} Тогда
    \begin{itemize}
        \item {}\begin{icloze}{2}\( f \!\nearrow\!\!\nearrow \) на \( \langle A, B \rangle  \)\end{icloze}
            \begin{icloze}{3}\( \impliedby  \)\end{icloze}
            \begin{icloze}{1}\( f'(x) > 0 \quad \forall x \in (A, B) \).\end{icloze}
    \end{itemize}
\end{note}

\begin{note}{0e1b8bb37eca4c29af2ca084fcedc196}
    Каков основной шаг в доказательстве достаточного условия строгого возрастания функции на промежутке?

    \begin{cloze}{1}
        Выразить для \( a < b \) разность \( f(b) - f(a) \) через формулу конечных приращений.
    \end{cloze}
\end{note}

\begin{note}{2e3edf0757ba4f72bbdbb5b66dca690d}
    Пусть \begin{icloze}{4}\( f \in C\langle A, B \rangle  \) и дифференцируема на \( (A, B) \).\end{icloze} Тогда
    \begin{itemize}
        \item {}\begin{icloze}{2}\( f \) постоянна на \( \langle A, B \rangle  \)\end{icloze}
            \begin{icloze}{3}\( \iff  \)\end{icloze}
            \begin{icloze}{1}\( f'(x) = 0 \quad \forall x \in (A, B) \).\end{icloze}
    \end{itemize}
\end{note}

\begin{note}{b036d705ddbe49b6814f53a6ad2b93f9}
    Каков основной шаг в доказательстве критерия постоянства функции на промежутке (достаточность)?

    \begin{cloze}{1}
        Выразить для произвольных \( a \) и \( b \) разность \( f(b) - f(a) \) через формулу конечных приращений.
    \end{cloze}
\end{note}

\begin{note}{2dfd421d331745a0a8b2da63493d1b4f}
    Пусть \begin{icloze}{3}\( f, g \in C[A, B\rangle \) и дифференцируемы на \( (A,  B) \).\end{icloze} Тогда
    Если \begin{icloze}{2}\( f(A) = g(A) \) и
        \[
            f'(x) > g'(x) \quad \forall x \in (A, B),
        \]
    \end{icloze}    то
    \begin{icloze}{1}\[
        f(x) > g(x) \quad \forall x \in (A, B\rangle.
    \]\end{icloze}
\end{note}

\begin{note}{e2c4b9fb4f4147a3bf25e2ab97a3e24f}
    Пусть \begin{icloze}{3}\( f, g \in C \langle A, B] \) и дифференцируемы на \( (A,  B) \).\end{icloze} Тогда
    Если \begin{icloze}{2}\( f(B) = g(B) \) и
        \[
            f'(x) < g'(x) \quad \forall x \in (A, B),
        \]
    \end{icloze}    то
    \begin{icloze}{1}\[
        f(x) > g(x) \quad \forall x \in \langle A, B).
    \]\end{icloze}
\end{note}

\begin{note}{0f2a5e13f0a2495388e631ac0b4776aa}
    Пусть \( f : D \subset \mathbb R \to \mathbb R, a \in D \). Тогда точка \( a \) называется \begin{icloze}{2}точкой максимума функции \( f \),\end{icloze} если
    \begin{icloze}{1}\[
                         \exists \delta > 0 \quad \forall x \in \dot V_{\delta} (a) \cap D \quad f(x) \leqslant f(a).
                     \]\end{icloze}
\end{note}

\begin{note}{a89063cdc4a34df7aa891ad50a98d0a8}
    Пусть \( f : D \subset \mathbb R \to \mathbb R, a \in D \). Тогда точка \( a \) называется \begin{icloze}{2}точкой строгого максимума функции \( f \),\end{icloze} если
    \begin{icloze}{1}\[
        \exists \delta > 0 \quad \forall x \in \dot V_{\delta} (a) \cap D \quad f(x) < f(a).
    \]\end{icloze}
\end{note}

\begin{note}{0c2db077ea274453a5c14d982fe1c571}
    Пусть \( f : D \subset \mathbb R \to \mathbb R, a \in D \). Тогда точка \( a \) называется \begin{icloze}{2}точкой минимума функции \( f \),\end{icloze} если
    \begin{icloze}{1}\[
        \exists \delta > 0 \quad \forall x \in \dot V_{\delta} (a) \cap D \quad f(x) \geqslant f(a).
    \]\end{icloze}
\end{note}

\begin{note}{3bc6223309d34118a582302414c9632e}
    Пусть \( f : D \subset \mathbb R \to \mathbb R, a \in D \). Тогда точка \( a \) называется \begin{icloze}{2}точкой строгого минимума функции \( f \),\end{icloze} если
    \begin{icloze}{1}\[
        \exists \delta > 0 \quad \forall x \in \dot V_{\delta} (a) \cap D \quad f(x) > f(a).
    \]\end{icloze}
\end{note}

\begin{note}{a1e964e24fc6456ca0a297c008405c34}
    Если \begin{icloze}{2}точка \( a \) является точкой минимума или максимума функции \( f \),\end{icloze} то \( a \) называется \begin{icloze}{1}точкой экстремума \( f \).\end{icloze}
\end{note}

\begin{note}{98f3cebf02ca464ab3cf9e94355caaa2}
    \subsubsection{<<\begin{icloze}{3}Необходимое условие экстремума\end{icloze}>>}

    Пусть \begin{icloze}{2}\( f : \langle A, B \rangle \to \mathbb R, a \in (A, B) \), \( f \) дифференцируема в точке \( a \).\end{icloze}
    Тогда \begin{icloze}{1}если \( a \) является точкой экстремума \( f \), то \( f'(a) = 0 \). \end{icloze}
\end{note}

\begin{note}{acfe3357868e41809070b12ea6034081}
    Каков основной шаг в доказательстве необходимого условия экстремума?

    \begin{cloze}{1}
        Применить теорему Ферма к \( f|_{[a - \delta, a + \delta]}  \) для \( \delta \) из определения экстремума.
    \end{cloze}
\end{note}

\begin{note}{96502706cad4449ab9ac44074765a384}
    Точка \( a \) называется \begin{icloze}{1}стационарной точкой функции \( f \),\end{icloze} если
    \begin{icloze}{2}\[
        f'(a) = 0.
    \]\end{icloze}
\end{note}

\begin{note}{99ca6c71ff484416941c4e10086ca6ea}
    Пусть \( f : \langle A, B \rangle \to \mathbb R \). Тогда
    \begin{icloze}{1}точка \( a \in (A, B) \)\end{icloze} называется \begin{icloze}{2}критической
    точкой,\end{icloze} если \begin{icloze}{1}либо \( a \) стационарна для \( f \), либо \(
    f \) не дифференцируема в точке \( a \).\end{icloze}
\end{note}

\begin{note}{40f1ebf761e14f5ba885b2276d64dae7}
    Пусть \( f : \langle A, B \rangle \to \mathbb R  \). Тогда все
    \begin{icloze}{2}точки экстремума \( f \), принадлежащие \( (A, B) \),\end{icloze}
    лежат в \begin{icloze}{1}множестве её критических точек.\end{icloze}
\end{note}

\begin{note}{e8adcc7d8b474840907e72b38014fcdc}
    Пусть \( f \in C[a, b] \). Тогда
    \[
        \begin{icloze}{3}\max f([a, b])\end{icloze} = \begin{icloze}{1}\max \left\{ f(a), f(b), \max f(C) \right\},\end{icloze}
    \]
    где \( C \) --- \begin{icloze}{2}множество критических точек \( f \).\end{icloze}
\end{note}

\begin{note}{909932c22cec4a5fb5d8cfb506e7dbfb}
    \subsubsection{<<\begin{icloze}{4}Достаточное условие экстремума в терминах \( f' \)\end{icloze}>>}

    Пусть \begin{icloze}{3}\( f : \langle A, B \rangle \to \mathbb R \), \( a \in (A, B)  \), \( f \) непрерывна в точке \( a \) и дифференцируема на \( \dot V_{\delta} (a) \), \( \delta > 0 \).\end{icloze}
    Если
    \begin{icloze}{1}\[
        \operatorname{sgn} f'(x) = \operatorname{sgn} (a - x) \quad \forall x \in \dot V_{\delta} (a),
    \]\end{icloze}
    то \begin{icloze}{2}\( a \) --- точка строго максимума \( f \).\end{icloze}
\end{note}

\begin{note}{1b1674e5941040ee87e83073a1a0d57b}
    \subsubsection{<<Достаточное условие экстремума в терминах \( f' \)>>}

    Пусть \( f : \langle A, B \rangle \to \mathbb R \), \( a \in (A, B)  \), \( f \) непрерывна в точке \( a \) и дифференцируема на \( \dot V_{\delta} (a) \), \( \delta > 0 \).
    Если
    \begin{icloze}{1}\[
        \operatorname{sgn} f'(x) = \operatorname{sgn} (x - a) \quad \forall x \in \dot V_{\delta} (a),
    \]\end{icloze}
    то \begin{icloze}{2}\( a \) --- точка строго минимума \( f \).\end{icloze}
\end{note}

\section{Лекция 21.02.22}
\begin{note}{4d119e495cf043019ed8ee01f9a7957a}
    \subsubsection{<<\begin{icloze}{4}Достаточное условие экстремума в терминах \( f'' \) \end{icloze}>>}
    Пусть \begin{icloze}{3}\( f : \langle A, B \rangle \to \mathbb R, a \in (A, B) \), \( f'' \) определена в точке \( a \), \( f'(a) = 0 \).\end{icloze}
    Тогда если \begin{icloze}{1}\( f''(a) > 0 \),\end{icloze} то \begin{icloze}{2}\( a \) --- точка строгого минимума \( f \).\end{icloze}
\end{note}

\begin{note}{f8b71055f7eb427f8226b47df9ed1e05}
    \subsubsection{<<Достаточное условие экстремума в терминах \( f'' \) >>}
    Пусть \( f : \langle A, B \rangle \to \mathbb R, a \in (A, B) \), \( f'' \) определена в точке \( a \), \( f'(a) = 0 \).
    Тогда если \begin{icloze}{1}\( f''(a) < 0 \),\end{icloze} то \begin{icloze}{2}\( a \) --- точка строгого максимума \( f \).\end{icloze}
\end{note}

\begin{note}{5e0ea19ce2b043c693e2cbc7752fcaf1}
    Каков первый шаг в доказательстве достаточного условия экстремума в терминах \( f'' \)?

    \begin{cloze}{1}
        Выразить \( f(x) - f(a) \)  по формуле Тейлора-Пиано с
        \[
            o((x - a)^2 ).
        \]
    \end{cloze}
\end{note}

\begin{note}{3124302c512c44bfac961f48e231e1cc}
    В чем основная идея доказательства достаточного условия экстремума в терминах \( f'' \)?

    \begin{cloze}{1}
        Вынести в формуле Тейлора-Пиано \( \frac{f''(a)}{2} (x - a)^2  \) за скобки, далее по теореме о стабилизации функции.
    \end{cloze}
\end{note}

\begin{note}{bb068aa42bfe43deb084eaa739cd08c6}
    \subsubsection{<<\begin{icloze}{5}Связь экстремума со старшими производными\end{icloze}>>}

    Пусть \begin{icloze}{4}\( f : \langle A, B \rangle \to \mathbb R, a \in (A, B) \),\end{icloze}
    \begin{icloze}{3}\[
                         \begin{gathered}
                             f'(a) = f''(a) = \cdots = f^{(n - 1)}(a) = 0, \\
                             f^{(n)}(a) \neq 0.
                         \end{gathered}
                     \]\end{icloze}
    Тогда если \begin{icloze}{2}\( n \) нечётно,\end{icloze} то \begin{icloze}{1}\( f \) не имеет экстремума в точке \( a \).\end{icloze}
\end{note}

\begin{note}{b8ec49e21174443588a98b2e5c8cc032}
    \subsubsection{<<Связь экстремума со старшими производными>>}

    Пусть \( f : \langle A, B \rangle \to \mathbb R, a \in (A, B) \),
    \[
        \begin{gathered}
            f'(a) = f''(a) = \cdots = f^{(n - 1)}(a) = 0, \\
            f^{(n)}(a) \neq 0.
        \end{gathered}
    \]
    Тогда если \begin{icloze}{2}\( n \) чётно,\end{icloze} \begin{icloze}{1}то достаточное условие аналогично достаточному условию в терминах \( f'' \).\end{icloze}
\end{note}

\begin{note}{d2426d6723fd4c20966bd4397dce3eb3}
    \subsubsection{<<\begin{icloze}{3}Теорема Дарбу\end{icloze}>>}

    Пусть \begin{icloze}{2}\( f \) дифференцируема на \( \langle A, B \rangle  \), \( a, b \in \langle A, B \rangle \),
    \[
        f'(a) < 0, \quad f'(b) > 0.
    \]\end{icloze}
    Тогда \begin{icloze}{1}\( \exists c \in (a, b) \quad f'(c) = 0 \).\end{icloze}
\end{note}

\begin{note}{43152412fd6f41e984fc4a4e96521633}
    В чем основная идея доказательства теоремы Дарбу?

    \begin{cloze}{1}
        По теореме Вейерштрасса существует точка минимума \( c \), далее по теореме Ферма.
    \end{cloze}
\end{note}

\begin{note}{b0b7d5c649bf4839bde1e90102df6405}
    Что позволяет применить теорему Ферма в доказательстве теоремы Дарбу?

    \begin{cloze}{1}
        \( c \) --- внутренняя точка отрезка \( [a, b] \).
    \end{cloze}
\end{note}

\begin{note}{d480b573cf054a67a6bf5596881b0afb}
    Как в доказательстве теорему Дарбу показать, что \( c \) не лежит на границе \( [a, b] \)?

    \begin{cloze}{1}
        Расписать \( f'(a) \) через правосторонний предел и показать, что \( a \) --- не локальный минимум. Аналогично для \( b \).
    \end{cloze}
\end{note}

\begin{note}{bc1402d472ba422ea18b051e2a0615c4}
    Пусть \begin{icloze}{3}\( f \) дифференцируема на \( \langle A, B \rangle  \).\end{icloze} Если
    \begin{icloze}{2}\[
        f'(x) \neq 0 \quad \forall x \in \langle A, B \rangle,
    \]\end{icloze}
    то \begin{icloze}{1}\( f \) строго монотонна на \( \langle A, B \rangle  \).\end{icloze}
\end{note}

\begin{note}{e29cdd0f22c346cab64fe288db3fbdb8}
    В чем основная идея доказательства следствия о монотонности функции с ненулевой производной?

    \begin{cloze}{1}
        Доказать от противного, что \( f' \) не меняет знак на \( \langle A, B \rangle  \).
        Далее по достаточному условию строгой монотонности.
    \end{cloze}
\end{note}

\begin{note}{9fc77ac828a342f885c48ee472c09734}
    \subsubsection{<<\begin{icloze}{2}Следствие из теоремы Дарбу \\
    \phantom{<<} \quad о сохранении промежутка.\end{icloze}>>}

    \begin{icloze}{1}Пусть \( f \) дифференцируема на \( \langle A, B \rangle  \). Тогда \( f'(\langle A, B \rangle ) \) --- промежуток.\end{icloze}
\end{note}

\begin{note}{56d20a83493a46d1ac834fec9f4ebdef}
    В чем основная идея доказательства следствия из теоремы Дарбу о сохранении промежутка?

    \begin{cloze}{1}
        Показать, что для любых \( a, b \in \langle A, B \rangle  \)
        \[
            [f'(a), f'(b)] \subset f'(\langle A, B \rangle ).
        \]
    \end{cloze}
\end{note}

\begin{note}{0cd99b9f1fae4d1aadfac35788f440c6}
    Какое упрощение принимается (для определённости) для точек \( a, b \in \langle A, B \rangle  \) в доказательстве следствия из теоремы Дарбу о сохранении промежутка?

    \begin{cloze}{1}
        \[
            f'(a) \leqslant f'(b).
        \]
    \end{cloze}
\end{note}

\begin{note}{9ee92cbcb63b46e78fe63b31bbf7f924}
    Как в доказательстве следствия из теоремы Дарбу о сохранении промежутка показать, что
    \[
        \forall y \in (f'(a), f'(b)) \quad y \in f(\langle A, B \rangle )?
    \]

    \begin{cloze}{1}
        Применить теорему Дарбу к функции
        \[
            F(x) = f(x) - y \cdot x
        \]
        в точках \( a \) и \( b \).
    \end{cloze}
\end{note}

\begin{note}{3c1144d31e264164b099479d41f9abe3}
    \subsubsection{<<\begin{icloze}{2}Следствие из теоремы Дарбу \\
    \phantom{<<} \quad о скачках производной.\end{icloze}>>}

    \begin{icloze}{1}Пусть \( f \) дифференцируема на \( \langle A, B \rangle  \). Тогда функция \( f' \) не имеет скачков на \( \langle A, B \rangle  \).\end{icloze}
\end{note}

\begin{note}{f94b4bdf90b14fa0a4256a492cf742a5}
    В чем основная идея доказательства следствия из теоремы Дарбу о скачках производной?

    TODO
\end{note}

\begin{note}{027449ca442a449786b58ca872e4aff2}
    \begin{icloze}{3}Функция \( f : \langle A, B \rangle \to \mathbb R  \)\end{icloze} называется \begin{icloze}{2}выпуклой на \( \langle A, B \rangle  \),\end{icloze} если
    \begin{icloze}{1}\[
        \begin{gathered}
            \forall a, b \in \langle A, B \rangle, \lambda \in (0,1) \\
            f(\lambda a + (1 - \lambda)b) \leqslant \lambda f(a) + (1 - \lambda)f(b).
        \end{gathered}
    \]\end{icloze}
\end{note}

\begin{note}{0073407c9c4f473cb4759784548208bd}
    \begin{icloze}{3}Функция \( f : \langle A, B \rangle \to \mathbb R  \)\end{icloze} называется \begin{icloze}{2}строго выпуклой на \( \langle A, B \rangle  \),\end{icloze} если
    \begin{icloze}{1}\[
        \begin{gathered}
            \forall a, b \in \langle A, B \rangle, \lambda \in (0,1) \\
            f(\lambda a + (1 - \lambda)b) < \lambda f(a) + (1 - \lambda)f(b).
        \end{gathered}
    \]\end{icloze}
\end{note}

\begin{note}{a0e64a51b1ac405c9e5806d135c272da}
    \subsubsection{<<\begin{icloze}{4}Критерий строгой выпуклости \( f \) на \( \langle A, B \rangle  \)\end{icloze}>>}

    Пусть \( f : \langle A, B \rangle \to \mathbb R  \). Тогда \begin{icloze}{3}равносильны\end{icloze} следующие утверждения.
    \begin{itemize}
        \item \begin{icloze}{1}\( f \) строго выпукла на \( \langle A, B \rangle  \).\end{icloze}
        \item \begin{icloze}{2}\( \forall a, b, c \in \langle A, B \rangle, a < c < b \) справедливо неравенство
                \[
                    \frac{f(c) - f(a)}{c - a} < \frac{f(b) - f(c)}{b - c}.
                \]
            \end{icloze}
    \end{itemize}
\end{note}

\begin{note}{8969555424f24b2c8347358586f381e8}
    \subsubsection{<<\begin{icloze}{4}Лемма о трёх хордах\end{icloze}>>}

    Пусть \( f : \langle A, B \rangle \to \mathbb R  \). Тогда \begin{icloze}{3}равносильны\end{icloze} следующие утверждения.
    \begin{itemize}
        \item \begin{icloze}{1}\( f \) строго выпукла на \( \langle A, B \rangle  \).\end{icloze}
        \item \begin{icloze}{2}\( \forall a, b, c \in \langle A, B \rangle, a < c < b \) справедливы неравенства
                \[
                    \frac{f(c) - f(a)}{c - a} < \frac{f(b) - f(a)}{b - a}  < \frac{f(b) - f(c)}{b - c}.
                \]
            \end{icloze}
    \end{itemize}
\end{note}

\section{Лекция 25.02.22}
\begin{note}{0abcc31a29c74496883c555de61b5af7}
    Пусть \begin{icloze}{3}\( f : \langle A, B \rangle \to \mathbb R \), \( a \in \langle A, B \rangle  \)
    \[
        F(x) := \frac{f(x) - f(a)}{x - a}.
    \]\end{icloze}

    Тогда если \begin{icloze}{2}\( f \) выпукла на \( \langle A, B \rangle  \),\end{icloze} то
    \begin{icloze}{1}
        \begin{center}
            \( F \!\!\nearrow \) на \( \langle A, B \rangle \setminus \left\{ a \right\} \).
        \end{center}
    \end{icloze}
\end{note}

\begin{note}{6658c8d28bde461584886f85aacf4977}
    Пусть \begin{icloze}{3}\( f : \langle A, B \rangle \to \mathbb R \), \( a \in \langle A, B \rangle  \)
    \[
        F(x) := \frac{f(x) - f(a)}{x - a}.
    \]\end{icloze}

    Тогда если \begin{icloze}{2}\( f \) строго выпукла на \( \langle A, B \rangle  \),\end{icloze} то
    \begin{icloze}{1}
        \begin{center}
            \( F \!\!\nearrow\!\nearrow  \) на \( \langle A, B \rangle \setminus \left\{ a \right\} \).
        \end{center}
    \end{icloze}
\end{note}

\begin{note}{0bb5876454d448878db0853372d90fe7}
    Пусть \begin{icloze}{3}\( f \) выпукла на \( \langle A, B \rangle  \),\end{icloze} \begin{icloze}{2}\( a \in \langle A, B )  \).\end{icloze} Тогда
    \begin{icloze}{1}
        \[
            \exists f'_+(a) \in [-\infty, +\infty ).
        \]
    \end{icloze}
\end{note}

\begin{note}{960c7add5b8c4ab4b798301f26f12648}
    Пусть \begin{icloze}{3}\( f \) выпукла на \( \langle A, B \rangle  \),\end{icloze} \begin{icloze}{2}\( a \in (A, B \rangle  \).\end{icloze} Тогда
    \begin{icloze}{1}
        \[
            \exists f'_-(a) \in (-\infty, +\infty ].
        \]
    \end{icloze}
\end{note}

\begin{note}{2e664465fdc5410ca8b72059cfe627bc}
    Пусть \begin{icloze}{3}\( f \) выпукла на \( \langle A, B \rangle  \),\end{icloze} \begin{icloze}{2}\( a \in (A, B) \).\end{icloze} Тогда \begin{icloze}{1}\( f'_+ (a) \) и \( f'_- (a) \) конечны и \( f'_- (a) \leqslant f'_+ (a) \).\end{icloze}
\end{note}

\begin{note}{eb64f07db3d3434197d40b0980a78e66}
    Если функция \( f \) выпукла на \( \langle A, B \rangle  \), то она \begin{icloze}{1}непрерывна на \( (A, B) \).\end{icloze}
\end{note}

\begin{note}{9f16939e7619449e9fe1d75a7aae2e87}
    Пусть \begin{icloze}{3}\( f : \langle A, B \rangle \to \mathbb R  \), \( a \in \langle A, B \rangle  \).\end{icloze}
    \begin{icloze}{2}Прямая \( y = g(x) \)\end{icloze} называется \begin{icloze}{1}опорной для функции в точке \( a \),\end{icloze} если
    \begin{icloze}{2}
        она проходит через точку \( (a, f(a)) \) и
        \[
            \begin{gathered}
                f(x) \geqslant g(x) \quad \forall x \in \langle A, B \rangle.
            \end{gathered}
        \]
    \end{icloze}
\end{note}

\begin{note}{7b835ae738654ba5a0921df5133181e7}
    Пусть \( f : \langle A, B \rangle \to \mathbb R  \), \( a \in \langle A, B \rangle  \).
    \begin{icloze}{2}Прямая \( y = g(x) \)\end{icloze} называется \begin{icloze}{1}строго опорной для функции в точке \( a \),\end{icloze} если
    \begin{icloze}{2}
        она проходит через точку \( (a, f(a)) \) и
        \[
            \begin{gathered}
                f(x) > g(x) \quad \forall x \in \langle A, B \rangle \setminus \left\{ a \right\}.
            \end{gathered}
        \]
    \end{icloze}
\end{note}

\begin{note}{fedf029d618e48ddabe81280b131b72b}
    Пусть \begin{icloze}{5}\( f : \langle A, B \rangle \to \mathbb R  \), \( f \) выпукла на \( \langle A, B \rangle  \), \( a \in (A, B) \),\end{icloze} прямая \( \ell \) задаётся
    \begin{icloze}{4}
        уравнением
        \[
            y = f(a) + k(x - a).
        \]
    \end{icloze}

    Тогда прямая \( \ell  \) является \begin{icloze}{1}опорной для функции \( f \)
    в точке \( a \)\end{icloze} \begin{icloze}{3}тогда и только тогда, когда\end{icloze}
    \begin{icloze}{2}
        \( k \in [f'_-(a), f'_+(a)] \).
    \end{icloze}
\end{note}

\begin{note}{8ceccffa4cbe4c8d8330451f4f53876c}
    Пусть \begin{icloze}{4}\( f : \langle A, B \rangle \to \mathbb R  \), \( f \) строго выпукла на \( \langle A, B \rangle  \), \( a \in (A, B) \),\end{icloze} прямая \( \ell \) задаётся уравнением
    \[
        y = f(a) + k(x - a).
    \]

    Тогда прямая \( \ell  \) является \begin{icloze}{1}строго опорной для функции \( f \) в точке \( a \)\end{icloze} \begin{icloze}{3}тогда и только тогда, когда\end{icloze} \begin{icloze}{2}\( k \in [f'_-(a), f'_+(a)]\).\end{icloze}
\end{note}

\section{04.03.22}
\begin{note}{acc9492d0b4f4c4a8e6b1688ee26ed5e}
    В чем геометрический смысл \( T_{a, 1} f(x) \)?

    \begin{cloze}{1}
        График \( T_{a, 1} f(x) \) --- это касательная к функции \( f \) в точке \( a \).
    \end{cloze}
\end{note}

\begin{note}{570272578ee74dd988ea80f9e95cbc6f}
    \subsubsection{<<Связь выпуклости функции с её касательными>>}

    Пусть \begin{icloze}{4}\( f : \langle A, B \rangle \to \mathbb R \), \( f \) дифференцируема на \( (A, B) \).\end{icloze}
    Тогда \begin{icloze}{2}функция \( f \) выпукла на \( \langle A, B \rangle \)\end{icloze} \begin{icloze}{3}тогда и только тогда,\end{icloze} когда
    \begin{icloze}{1}
        \[
            \begin{gathered}
                \forall a \in (A, B), \quad x \in \langle A, B \rangle \\
                f(x) \geqslant T_{a, 1} f(x).
            \end{gathered}
        \]
    \end{icloze}
\end{note}

\begin{note}{32700c2a93204435b3f66db20ea03bf7}
    \subsubsection{<<Связь выпуклости функции с её касательными>>}

    Пусть \begin{icloze}{4}\( f : \langle A, B \rangle \to \mathbb R \), \( f \) дифференцируема на \( (A, B) \).\end{icloze}
    Тогда \begin{icloze}{2}функция \( f \) строго выпукла на \( \langle A, B \rangle \)\end{icloze} \begin{icloze}{3}тогда и только тогда, когда\end{icloze}
    \begin{icloze}{1}
        \[
            \begin{gathered}
                \forall a \in (A, B), x \in \langle A, B \rangle \setminus \left\{ a \right\} \\
                f(x) > T_{a, 1} f(x).
            \end{gathered}
        \]
    \end{icloze}
\end{note}

\begin{note}{76ff105d143e49dea8fe8db2b74ee9ff}
    В чем основная идея доказательства теоремы о связи выпуклости функции с её касательными?

    \begin{cloze}{1}
        \( f \) дифференцируема в любой точке \( (A, B) \) \( \implies \) касательная совпадает с опорной прямой.
    \end{cloze}
\end{note}

\begin{note}{3b6d6467bd5144febe2b52fd934c971a}
    Пусть \begin{icloze}{3}\( f : (A, +\infty) \to \mathbb R \) имеет при \( x \to +\infty \) асимптоту \( y = kx + b \).\end{icloze}
    Тогда если \begin{icloze}{2}\( f \) выпукла на \( (A, +\infty) \),\end{icloze} то
    \begin{icloze}{1}
        \[
            f(x) \geqslant kx + b \quad \forall x \in (A, +\infty).
        \]
    \end{icloze}
\end{note}

\begin{note}{e766cccf8cdf4765b58203bef6244390}
    Пусть \begin{icloze}{3}\( f : (A, +\infty) \to \mathbb R \) имеет при \( x \to +\infty \) асимптоту \( y = kx + b \).\end{icloze}
    Тогда если \begin{icloze}{2}\( f \) строго выпукла на \( (A, +\infty) \),\end{icloze} то
    \begin{icloze}{1}
        \[
            f(x) > kx + b \quad \forall x \in (A, +\infty).
        \]
    \end{icloze}
\end{note}

\begin{note}{94e7cdb6145142c3bb7cc8115035e5ad}
    \subsubsection{<<Связь выпуклости функции с \( f' \)>>}

    Пусть \begin{icloze}{4}\( f \in C\langle A, B \rangle \), \( f \) дифференцируема на \( (A, B) \).\end{icloze}
    Тогда \begin{icloze}{2}\( f \) выпукла на \( \langle A, B \rangle \)\end{icloze} \begin{icloze}{3}тогда и только тогда, когда\end{icloze}
    \begin{icloze}{1}
        \begin{center}
            \( f' \!\!\nearrow \) на \( (A, B) \).
        \end{center}
    \end{icloze}
\end{note}

\begin{note}{cfdb1a58f41247169b530e3bc3f5b061}
    \subsubsection{<<Связь выпуклости функции с \( f' \)>>}
    Пусть \begin{icloze}{4}\( f \in C\langle A, B \rangle \), \( f \) дифференцируема на \( (A, B) \).\end{icloze}
    Тогда \begin{icloze}{2}\( f \) строго выпукла на \( \langle A, B \rangle \)\end{icloze} \begin{icloze}{3}тогда и только тогда, когда\end{icloze}
    \begin{icloze}{1}
        \begin{center}
            \( f' \!\!\nearrow\!\nearrow \) на \( (A, B) \).
        \end{center}
    \end{icloze}
\end{note}

\begin{note}{1db6c044058c49e68328ad272c648da8}
    \subsubsection{<<Связь выпуклости функции с \( f'' \)>>}
    Пусть \begin{icloze}{4}\( f \in C\langle A, B \rangle \), \( f \) дважды дифференцируема на \( (A, B) \).\end{icloze}
    Тогда \begin{icloze}{2}\( f \) выпукла на \( \langle A, B \rangle \)\end{icloze} \begin{icloze}{3}тогда и только тогда, когда\end{icloze}
    \begin{icloze}{1}
        \[
            f''(x) \geqslant 0 \quad \forall x \in (A, B).
        \]
    \end{icloze}
\end{note}

\begin{note}{d78c1dfaebde4a2e89fdccfb43309163}
    \subsubsection{<<Связь выпуклости функции с \( f'' \)>>}
    Пусть \begin{icloze}{4}\( f \in C\langle A, B \rangle \), \( f \) дважды дифференцируема на \( (A, B) \).\end{icloze}
    Тогда \begin{icloze}{2}\( f \) строго выпукла на \( \langle A, B \rangle \),\end{icloze} \begin{icloze}{3}если\end{icloze}
    \begin{icloze}{1}
        \[
            f''(x) > 0 \quad \forall x \in (A, B).
        \]
    \end{icloze}
\end{note}

\begin{note}{399c82ffb7094f2e8e4a74da8023fc60}
    Пусть \begin{icloze}{3}\( f :  \langle A, B \rangle \to \mathbb R, a \in (A, B) \).\end{icloze}
    Точка \( a \) называется \begin{icloze}{2}точкой перегиба функции \( f \),\end{icloze} если
    \begin{icloze}{1}
        \begin{itemize}
            \item \( \exists \delta > 0 \) такое, что \( V_\delta (a) \subset (A, B) \) и \( f \) имеет разный характер выпуклости на \( (a - \delta, a] \) и \( [a, a + \delta) \);
            \item \( f \) непрерывна в точке \( a \);
            \item \( \exists f'(a) \in \overline{\mathbb R} \).
        \end{itemize}
    \end{icloze}
\end{note}

\begin{note}{9aa5847a39ac46e8ad8dbee41e14a904}
    Пусть \( f : \langle A, B \rangle \to \mathbb R, a \in (A, B) \), \( f \) дважды дифференцируема на \( a \).
    Если \begin{icloze}{2}\( a \) является точкой перегиба \( f \),\end{icloze} то \begin{icloze}{1}\( f''(a) = 0 \).\end{icloze}
\end{note}

\begin{note}{aca76c8bcbef4e38ad13dd619d48d19d}
    Является ли нулевая вторая производная достаточным условием перегиба?

    \begin{cloze}{1}
        Нет, это только необходимое условие.
    \end{cloze}
\end{note}

\begin{note}{c3615f4ec8d84748bde8c518c9e98375}
    Пусть \begin{icloze}{3}\( f : \langle A, B \rangle \to \mathbb R, a \in (A, B) \), \( f \) непрерывна в точке \( a \) и имеет в ней производную из \( \overline{\mathbb R} \).\end{icloze}
    Тогда если \begin{icloze}{1}\( \exists \delta > 0 \) такое, что \( f \) дважды дифференцируема на \( \dot V_{\delta}(a) \) и
    \begin{itemize}
        \item либо \quad \( \operatorname{sgn} f''(x) = \operatorname{sgn} (a - x) \quad \forall x \in \dot V_{\delta}(a), \)
        \item либо \quad \( \operatorname{sgn} f''(x) = \operatorname{sgn} (x - a) \quad \forall x \in \dot V_{\delta}(a), \)
    \end{itemize}\end{icloze}
    то \begin{icloze}{2}\( a \) --- точка перегиба \( f \).\end{icloze}
\end{note}

\section{Семинар 03.03.22}
\begin{note}{655ebf6da8c1489f84fdaeea82dcc793}
    \[
        \int \begin{icloze}{2}\ln x\end{icloze}\: dx = \begin{icloze}{1}x \ln x - x\end{icloze} + C
    \]
\end{note}

\begin{note}{310668af95114f9fbe87673be333fec8}
    \[
        \int \begin{icloze}{2}\frac{1}{\sin x}\end{icloze}\: dx = \begin{icloze}{1}\ln\left| \tan \frac{x}{2} \right|\end{icloze} + C
    \]
\end{note}

\begin{note}{898276fe3ef943c49921748d594000c8}
    \[
        \int \begin{icloze}{2}\frac{1}{\cos x}\end{icloze}\: dx = \begin{icloze}{1}\ln\left| \frac{1 + \tan \frac{x}{2}}{1 - \tan \frac{x}{2}} \right|\end{icloze} + C
    \]
\end{note}

\begin{note}{ce3022e62a4f4a6ea2d13195a9f94d31}
    \[
        \int \begin{icloze}{2}\frac{1}{x^2 + a^2}\end{icloze}\: dx
        = \begin{icloze}{1}\frac{1}{a} \arctan \frac{x}{a}\end{icloze} + C
        \quad (\begin{icloze}{3}a > 0\end{icloze})
    \]
\end{note}

\begin{note}{8661888336db411a89fed337ad926a76}
    \[
        \int \begin{icloze}{2}\frac{A}{x + a}\end{icloze}\: dx
        = \begin{icloze}{1}A \ln\left| x + a \right|\end{icloze} + C
    \]
\end{note}

\begin{note}{2cd6c699811f4760be34715a24b0081f}
    \[
        \int \begin{icloze}{2}\frac{1}{nx + a}\end{icloze}\: dx
        = \begin{icloze}{1}\frac{1}{n} \ln\left| x + \frac{a}{n} \right|\end{icloze} + C
        \quad (\begin{icloze}{3}n \neq 0\end{icloze})
    \]
\end{note}

\begin{note}{b7b778e748574ee8b52225ae5669cbe6}
    \[
        \int \begin{icloze}{2}\frac{A}{(x + a)^{k}}\end{icloze}\: dx
        = \begin{icloze}{1}\frac{A}{(1 - k)(x - a)^{k - 1}}\end{icloze} + C
        \quad (\begin{icloze}{3}k \neq 1\end{icloze})
    \]
\end{note}

\begin{note}{72b0aaea0b254078bbfcc47745885653}
    \begin{multline*}
        \int \begin{icloze}{4}\frac{Mx + N}{x^2 + px + q}\end{icloze}\: dx = \\
        \begin{icloze}{1}\frac{M}{2} \ln\left| x^2 + px + q \right|\end{icloze}
        + \begin{icloze}{2}\frac{2N - pM}{2a} \arctan \frac{2x + p}{2a}\end{icloze} \\
        + C,
    \end{multline*}
    где \( \displaystyle a^2 := \begin{icloze}{3}\frac{4q - p^2}{4} > 0\end{icloze} \).
\end{note}

\begin{note}{c7fcc3d1ab9443d2855e310bfb0beee8}
    \begin{multline*}
        \int \begin{icloze}{3}\frac{Mx + N}{\left( x^2 + px + q \right)^{k}}\end{icloze}\: dx = \\
        \int \begin{icloze}{1}\frac{N - M \frac{p}{2}}{\left( t^2 + a^2 \right)^{k}}\end{icloze}\: dt +
        \begin{icloze}{1}\int \frac{Mt}{\left( t^2 + a^2 \right)^{k}}\end{icloze}\: dt + C,
    \end{multline*}
    где \( \displaystyle \quad t := \begin{icloze}{2}x + \frac{p}{2},\end{icloze} \quad a^2 := \begin{icloze}{2}\frac{4q - p^2}{4} > 0\end{icloze} \).
\end{note}

\begin{note}{a3d0cc7201b74c4c9fab9590e7a6c0b2}
    \begin{align*}
        I_k &=: \int \begin{icloze}{3}\frac{1}{\left( t^2 + a^2 \right)^{k}}\end{icloze}\: dt \quad (\begin{icloze}{4}k > 1, a \neq 0\end{icloze}) \\
        I_k &= \begin{icloze}{1}\frac{2k - 3}{2(k - 1)a^2} I_{k - 1}\end{icloze} + \begin{icloze}{2}\frac{t}{2(k - 1)a^2(t^2 + a^2)^{k - 1}}\end{icloze}
    \end{align*}
\end{note}

\begin{note}{972b3ecb92a94f62b12e46795945593d}
    \[
        \int \begin{icloze}{2}\frac{Mt}{\left( t^2 + a^2 \right)^{k}}\end{icloze}\: dt = \begin{icloze}{1}\frac{M}{2(1 - k)(t^2 + a^2)^{k - 1}}\end{icloze} + C
    \]
\end{note}

\section{Лекция 07.03.22}
\begin{note}{8d4e84ad6e1a4cdc91020e2f61878f24}
    Пусть \begin{icloze}{3}\( f : \langle A, B \rangle \to \mathbb R \).\end{icloze}
    \begin{icloze}{1}Функция \( F : \langle A, B \rangle \to \mathbb R \)\end{icloze} наызвается \begin{icloze}{2}первообразной функции \( f \),\end{icloze} если
    \begin{icloze}{1}F дифференцируема на \( \langle A, B \rangle \) и
    \[
        F'(x) = f(x) \quad \forall x \in \langle A, B \rangle.
    \]\end{icloze}
\end{note}

\begin{note}{5436ab9b46cf488eb5fa6c2353bd3616}
    \begin{icloze}{1}Множество всех первообразных функции \( f \) на промежутке \( \langle A, B \rangle \)\end{icloze} обозначается \begin{icloze}{2}\( \mathscr P_f(\langle A, B \rangle) \).\end{icloze}
\end{note}

\begin{note}{ec64c5e7734140f888511699374deaec}
    Пусть \begin{icloze}{4}\( f, F, G : \langle A, B \rangle \to \mathbb R \), \( F \in \mathscr P_f(\langle A, B \rangle) \).\end{icloze}
    Тогда
    \[
        \begin{icloze}{2}G \in \mathscr P_f (\langle A, B \rangle) \end{icloze}
        \begin{icloze}{3}\iff\end{icloze}
        \begin{icloze}{1}\exists c \in \mathbb R \quad G(x) = F(x) + c.\end{icloze}
    \]
\end{note}

\begin{note}{e9bbf7b29a8d40b48aad130674b03cc9}
    Пусть \( f, F, G : \langle A, B \rangle \to \mathbb R \), \( F \in \mathscr P_f(\langle A, B \rangle) \).
    Тогда
    \[
        G \in \mathscr P_f (\langle A, B \rangle) \implies
        \exists c \in \mathbb R \quad G(x) = F(x) + c.
    \]

    В чем основная идея доказательства?

    \begin{cloze}{1}
        \( (F(x) - G(x))' \equiv 0 \) на \( \langle A, B \rangle \) \( \implies \) \( F(x) - G(x) \) постоянна на \( \langle A, B \rangle \).
    \end{cloze}
\end{note}

\begin{note}{64bcacf18cb94a4e9b96e551eff15e5b}
    Пусть \( f, F, G : \langle A, B \rangle \to \mathbb R \), \( F \in \mathscr P_f(\langle A, B \rangle) \).
    Тогда
    \[
        G \in \mathscr P_f (\langle A, B \rangle) \impliedby
        \exists c \in \mathbb R \quad G(x) = F(x) + c.
    \]

    В чем основная идея доказательства?

    \begin{cloze}{1}
        Тривиально следует из определения первообразной.
    \end{cloze}
\end{note}

\begin{note}{b196b146568446a2b31a62a77bcddd45}
    Пусть \begin{icloze}{3}\( f : \langle A, B \rangle \to \mathbb R, \quad F \in \mathscr P_f (\langle A, B \rangle) \).\end{icloze}
    \begin{icloze}{1}Множество функций
                     \[
                         \left\{ F(x) + c \mid c \in \mathbb R \right\}
                     \]\end{icloze}
    называется \begin{icloze}{2}неопределённым интегралом \( f \) на \( \langle A, B \rangle \).\end{icloze}
\end{note}

\begin{note}{98516b869bc740b9bacfcc5244a89cb0}
    Пусть \begin{icloze}{3}\( f : \langle A, B \rangle \to \mathbb R \).\end{icloze}
    \begin{icloze}{1}Неопределённый интеграл функции \( f \)  на \( \langle A, B \rangle \)\end{icloze} обозначается
    \begin{icloze}{2}
        \[
            \int f(x)\: dx.
        \]
    \end{icloze}
\end{note}

\begin{note}{7581f732c1c44de4bc99eae39e01f4ea}
    Корректна ли запись
    \[
        \int f(x)\: dx = F(x) + C \quad?
    \]

    \begin{cloze}{1}
        Строго говоря нет, поскольку формально интеграл является множеством, а не функцией, но такая запись удобна на практике.
    \end{cloze}
\end{note}

\begin{note}{ad021cd0f9bd4d9ca316d3574a3b67a4}
    Пусть \( f : \langle A, B \rangle \to \mathbb R \) и \( f \) имеет первообразную на \( \langle A, B \rangle \).
    \[
        \left( \int f(x)\: dx \right)' \overset{\text{def}}= \begin{icloze}{1}f(x).\end{icloze}
    \]
\end{note}

\begin{note}{a2f17fea47484277b1a9d9349fbea7ff}
    Пусть \( f, g : \langle A, B \rangle \to \mathbb R \), \quad \( F \in \mathscr P_f (\langle A, B \rangle), G \in \mathscr P_g (\langle A, B \rangle) \).
    \[
        \int f(x)\: dx + \int g(x)\: dx \overset{\text{def}}= \begin{icloze}{1}\Big\{ F(x) + H(x) + C \mid C \in \mathbb R \Big\}.\end{icloze}
    \]
\end{note}

\begin{note}{7d5f8b97d72747df93959cee3fb0bae9}
    Пусть \( f : \langle A, B \rangle \to \mathbb R \) и \( f \) имеет первообразную на \( \langle A, B \rangle \), \( \lambda \in \mathbb R \).
    \[
        \int f(x)\: dx \overset{\text{def}}= \begin{icloze}{1}\Big\{ \lambda f(x) + C \mid C \in \mathbb R \Big\}.\end{icloze}
    \]
\end{note}

\begin{note}{3fb6e723afb54981be16c06cf2bfb210}
    Из \begin{icloze}{3}теоремы Дарбу\end{icloze} следует, что
    если \begin{icloze}{2}\( f \) имеет первообразную на промежутке \( \langle A, B \rangle \),\end{icloze}
    то \begin{icloze}{1}\( f \) не имеет скачков на \( \langle A, B \rangle \)\end{icloze}
\end{note}

\begin{note}{3c586c7317d247a3be4f7b50373a0d46}
    Является ли непрерывность функции \( f \) необходимым условием для существования у неё первообразной?

    \begin{cloze}{1}
        Нет, поскольку \( f \) может иметь точки разрыва второго рода.
    \end{cloze}
\end{note}

\begin{note}{ca1243ec222b4440903a1f5a22a53b16}
    \subsubsection{<<Достаточное условие существования \\\phantom{<<}первообразной>>}
    \begin{icloze}{1}Если \( f \) непрерывна на \( \langle A, B \rangle \), то \( f \) имеет первообразную на \( \langle A, B \rangle \).\end{icloze}
\end{note}

\section{Лекция 11.03.22}
\begin{note}{8d01db3371424aba95e1092ffa2cd4dc}
    Пусть \begin{icloze}{3}\( f : E \subset \mathbb R \to \mathbb R \).\end{icloze} \begin{icloze}{1}Функция \( F : E \to \mathbb R \)\end{icloze} называется \begin{icloze}{2}первообразной \( f \) на множестве \( E \),\end{icloze} если \begin{icloze}{1}\( F \) дифференцируема на \( E \) и \( F'(x) = f(x) \) для любого \( x \in E \).\end{icloze}
\end{note}

\begin{note}{a36222511f224d049fc0a1fc0c465aa5}
    Интеграл \( \int f(x)\: dx \) называется \begin{icloze}{2}берущимся,\end{icloze} если \begin{icloze}{1}функция \( f \) имеет элементарную первообразную.\end{icloze}
\end{note}

\begin{note}{937d08196fed4fea9d424dfd802f1c82}
    Пусть \( f, g : \langle A, B \rangle \to \mathbb R \) имеют на  \( \langle A, B \rangle \) первообразную. Тогда для любых \( \alpha, \beta \in \mathbb R \setminus \left\{ 0 \right\} \)
    \[
        \int \left( \alpha f(x) + \beta g(x) \right)\: dx = \begin{icloze}{1}\alpha \int f(x)\: dx + \beta \int g(x)\: dx.\end{icloze}
    \]
\end{note}

\begin{note}{2f7dd89b9a244dacbf41650571c4f13c}
    Как доказать свойство линейности неопределённого интеграла?

    \begin{cloze}{1}
        По определению интеграла и первообразной.
    \end{cloze}
\end{note}

\begin{note}{26b34c9a101f488aaed5ddee4ddd43d2}
    \subsubsection{<<Теорема о замене переменной \\
    \phantom{<<}в неопределённом интеграле>>}

    Пусть \begin{icloze}{2}\( f : \langle A, B \rangle \to \mathbb R \), \( F \in \mathscr P_f (\langle A, B \rangle) \), \( \varphi : \langle C, D \rangle \to \langle A, B \rangle \) и \( \varphi \) дифференцируема на \( \langle C, D \rangle \).\end{icloze}
    Тогда
    \begin{icloze}{1}
        \[
            \int f(\varphi(x)) \cdot \varphi'(x)\: dx = F(\varphi(x)) + C.
        \]
    \end{icloze}
\end{note}

\begin{note}{2f7dd89b9a244dacbf41650571c4f13c}
    Как доказать теорему о замене переменной в неопределённом интеграле?

    \begin{cloze}{1}
        По определению интеграла и первообразной.
    \end{cloze}
\end{note}

\begin{note}{cf45cd81236549efb89f81fcce13349f}
    Пусть \begin{icloze}{3}\( f : \langle A, B \rangle \to \mathbb R \), \( \varphi : \langle C, D \rangle \to \langle A, B \rangle \) и \( \varphi \) дифференцируема на \( \langle A, B \rangle \) и обратима.\end{icloze} Тогда если \begin{icloze}{1}\( G \) --- первообразная функции \( \left( f \circ \varphi \right) \cdot \varphi' \),\end{icloze} то
    \[
        \begin{icloze}{2}\int f(x)\: dx\end{icloze} = \begin{icloze}{1}G(\varphi^{-1} (x)) + C.\end{icloze}
    \]
\end{note}

\begin{note}{f1d541a0c135409c8aef89920ad254e8}
    \subsubsection{<<Формула интегрирования по частям>>}
    Пусть \begin{icloze}{2}\( f, g \in C^{1} \left( \langle A, B \rangle \right) \).\end{icloze} Тогда
    \begin{icloze}{1}
        \[
            \int f(x) g'(x)\: dx = f(x) g(x) -  \int g(x) f'(x)\: dx.
        \]
    \end{icloze}
\end{note}


\begin{note}{e2df459e1699495f980cdddacc633f6f}
    В чем основная идея доказательства основной формулы интегрирования по частям?

    \begin{cloze}{1}
        \[
            (uv)'  = u' v + u v' \implies uv = \int v u'\: dx + \int u v'\: dx.
        \]
    \end{cloze}
\end{note}
\end{document}
