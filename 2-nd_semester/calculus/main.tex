\documentclass[11pt, a5paper]{article}
\usepackage[width=10cm, top=0.5cm, bottom=2cm]{geometry}

\usepackage[T1,T2A]{fontenc}
\usepackage[utf8]{inputenc}
\usepackage[english,russian]{babel}
\usepackage{libertine}

\usepackage{amsmath}
\usepackage{amssymb}
\usepackage{amsthm}
\usepackage{mathrsfs}
\usepackage{framed}
\usepackage{xcolor}

\setlength{\parindent}{0pt}

\renewcommand{\thesection}{}
\renewcommand{\thesubsection}{Note \arabic{subsection}}
\renewcommand{\thesubsubsection}{}
\renewcommand{\theparagraph}{}

\newenvironment{note}[1]{\goodbreak\par\subsection{\hfill \color{lightgray}\tiny #1}}{}
\newenvironment{cloze}[2][\ldots]{\begin{leftbar}}{\end{leftbar}}
\newenvironment{icloze}[2][\ldots]{%
  \ignorespaces\text{\tiny \color{lightgray} \{\{c#2:: }%
}{%
  \text{\tiny \color{lightgray}\}\}}\unskip%
}

\begin{document}
\section{Лекция 07.02.22}
\begin{note}{662fbe59ca984f5b820ad1041f1eb840}
    Пусть \( f(x) : D \subset \mathbb R  \to \mathbb R, a \in D. \)
    \begin{icloze}{2}Многочлен \( p(x) \) степени \( n \)  такой, что \[
        \begin{gathered}
            f(x) = p(x) + o((x - a)^{n}), \\
            f(a) = p(a),
        \end{gathered}
    \]\end{icloze}
    называется \begin{icloze}{1}многочленом Тейлора функции \( f \) порядка \( n \) в точке \( a. \)\end{icloze}
\end{note}

\begin{note}{738279ec323b45e29a170a4e41b4bce0}
    Если многочлен Тейлора функции \( f \) порядка \( n \) в точке \( a \) существует, то \begin{icloze}{1}он единственен.\end{icloze}
\end{note}

\begin{note}{8f605243b193465799ba06e1576d171e}
    В чём ключевая идея доказательства единственности многочлена Тейлора?

    \begin{cloze}{1}
        Пусть коэффициент \( r_m  \) при \( (x - a)^{m} \) --- первый ненулевой коэффициент в многочлене \( p - q \).
        Тогда \[
            \frac{p - q}{(x - a)^{m} } \underset{x \to a}\longrightarrow r_m,
        \] но при этом \[
            \frac{p - q}{(x - a)^{m} } = o((x - a)^{n - m}) \underset{x \to a}\longrightarrow 0 \implies r_m = 0.
        \]
    \end{cloze}
\end{note}

\begin{note}{f4110a9b63c640be96d810d835d0d1fd}
    \begin{icloze}{2}Многочлен Тейлора функции \( f \) порядка \( n \) в точке \( a \)\end{icloze} обозначается \begin{icloze}{1}\( T_{a, n} f.  \)\end{icloze}
\end{note}

\begin{note}{97c12315facb454e987cb94fae99be75}
\[
    \left. f(x) \right|_{x = a} \overset{\text{def}}= \begin{icloze}{1}f(a).\end{icloze}
\]
\end{note}

\begin{note}{cf7e5ab30b564c139557fd0a940f8204}
    \[
        \left. \left((x - a)^{k} \right)^{(n)}  \right|_{x = a} =
        \begin{icloze}{1}
            \begin{cases}
                0, & n \neq k, \\
                n!, & n = k.
            \end{cases}
        \end{icloze}
    \]
\end{note}

\begin{note}{7597b782ce5f4e92998cc6445ce6f40e}
    \subsubsection{<<\begin{icloze}{3}Свойство n раз дифференцируемой функции\end{icloze}>>}

    Пусть \begin{icloze}{2}\( f : D \subset R \to \mathbb R, a \in D, n \in \mathbb N \) и \[
        f(a) = f'(a) = \cdots = f^{(n)} (a) = 0.
    \]\end{icloze}
    Тогда \begin{icloze}{1}\( f(x) = o((x - a)^{n} ),  x \to a \).\end{icloze}
\end{note}

\begin{note}{22aa07051d4c4e0ebb08ce0114be5429}
    \subsubsection{<<Определение \( o \)-малого в терминах \( \varepsilon, \delta. \)>>}

    Пусть \( f, g : D \subset \mathbb R \to \mathbb R \), \( a \) --- предельная точка \( D \). Тогда
    \begin{multline*}
        f(x) = o(g(x)), \quad x \to a \overset{\hspace{-1pt}\text{\tiny def}}\iff \\
        \begin{icloze}{1}
            \forall \varepsilon > 0 \quad \exists \delta > 0 \quad \forall x \in D \cap \dot V_{\delta}(a) \quad |f(x)| \leqslant \varepsilon |g(x)|.
        \end{icloze}
    \end{multline*}
\end{note}

\begin{note}{b7ddf1bbcdf84c769dd7b409e5be494d}
    Какой метод используется в доказательстве свойства \( n \)-раз дифференцируемой функции?

    \begin{cloze}{1}
        Индукция по \( n. \)
    \end{cloze}
\end{note}

\begin{note}{f04179797fd64614827341d425616341}
    Какова основная идея в доказательстве свойства \( n \)-раз дифференцируемой функции (базовый случай)?

    \begin{cloze}{1}
        Подставить \( f(a) = f'(a) = 0 \) в определение дифференцируемости.
    \end{cloze}
\end{note}

\begin{note}{7a10e93958724ee6b93bc1637a13773f}
    Каков первый шаг в доказательстве свойства \( n \)-раз дифференцируемой функции (индукционный переход)?

    \begin{cloze}{1}
        Заметить, что из индукционного предположения
        \[
            f'(x) = o((x - a)^{n} )
        \]
        и расписать это равенство в терминах \( \varepsilon, \delta. \)
    \end{cloze}
\end{note}

\begin{note}{b863b13c8a8b45c09c6444b48e5c0b75}
    Какие ограничения накладываются на \( \delta \) в доказательстве свойства \( n \)-раз дифференцируемой функции (индукционный переход)?

    \begin{cloze}{1}
        \( \dot V_{\delta} (a) \cap D \) есть невырожденный промежуток.
    \end{cloze}
\end{note}

\begin{note}{2506d5781f234e13a94358880699831a}
    Почему в доказательстве свойства \( n \)-раз дифференцируемой функции (индукционный переход) мы можем сказать, что
    \( \exists \delta > 0 \) такой,  что \( \dot V_{\delta} (a) \cap D \) есть невырожденный отрезок?

    \begin{cloze}{1}
        По определению дифференцируемости функции.
    \end{cloze}
\end{note}

\begin{note}{73ed2cdbb8b444ce991d587d9ed279ed}
    В чем ключевая идея доказательства свойства \( n \)-раз дифференцируемой функции (индукционный переход)?

    \begin{cloze}{1}
        Выразить \( f(x) = f'(c) \cdot (x - a) \) по симметричной формуле конечных приращений и показать, что \( |f'(c)| < \varepsilon |x - a|^{n} \).
    \end{cloze}
\end{note}

\begin{note}{a08796d96ad841bd91a8e7daaab1857d}
    Откуда следует, что \( f'(c) < \varepsilon|x - a|^{n}  \) в доказательстве свойства \( n \)-раз дифференцируемой функции (индукционный переход)?

    \begin{cloze}{1}
        \[
            c \in \dot V_{\delta} (a) \implies f'(c) < \varepsilon|c - a|^{n} < \varepsilon|x - a|^{n}
        \]
    \end{cloze}
\end{note}

\begin{note}{957fd9747bd84545bd6b1cca723d72ba}
    Пусть \( f : D \subset \mathbb R \to \mathbb R, a \in D, n \in \mathbb N \), \begin{icloze}{2}\(f(a) = 0, \)
    \[
        f'(x) = o((x - a)^{n} ), \quad x \to a.
    \]\end{icloze}

    Тогда \( f(x) = \begin{icloze}{1}o((x - a)^{n + 1}), \quad x \to a.\end{icloze} \)
\end{note}

\begin{note}{99a8f041e1a34dba923a682c6500c46b}
    \subsubsection{<<\begin{icloze}{3}Формула Тейлора-Пеано\end{icloze}>>}

    Пусть \begin{icloze}{2}\( f : D \subset R \to \mathbb R, a \in D, n \in \mathbb N \) и \( f \) \( n \) раз дифференцируема в точке \( a. \)  \end{icloze}
    Тогда \begin{icloze}{1}\[
        T_{a, n} f = \sum_{k=0}^{n} \frac{f^{(k)} (a)}{k!} (x - a)^{k}.
    \]\end{icloze}
\end{note}
\end{document}
