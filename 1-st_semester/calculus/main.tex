\documentclass[12pt, a5paper]{article}
\usepackage[width=10cm, top=0.5cm, bottom=2cm]{geometry}

\usepackage[T1,T2A]{fontenc}
\usepackage[utf8]{inputenc}
\usepackage[english,russian]{babel}
\usepackage{libertine}

\usepackage{amsmath}
\usepackage{amssymb}
\usepackage{amsthm}
\usepackage{mathrsfs}

\setlength{\parindent}{0pt}

\renewcommand{\thesection}{}
\renewcommand{\thesubsection}{}
\renewcommand{\thesubsubsection}{}
\renewcommand{\theparagraph}{}

\newenvironment{note}[1]{\par\noindent\rule{\linewidth}{0.1mm}}{}
\newenvironment{cloze}[2][\ldots]{}{}

\begin{document}
    \begin{note}{61211f8d77d34c9fa47e8872540de683}
        Каков первый шаг в доказательстве любого из законов де Моргана?

        \begin{cloze}{1}
            Рассмотреть произвольный элемент \( a, \) принадлежащий левой (или
            правой) части соответствующего равенства.
        \end{cloze}
    \end{note}


    \begin{note}{7e46fe30f7624833823e79c0fedc16df}
        Какова основная идея доказательства любого из законов де Моргана?

        \begin{cloze}{1}
            Надо показать, что условие принадлежности произвольного элемента \( a \)
            левой части совпадают с таковыми для правой части.
        \end{cloze}
    \end{note}

    \begin{note}{010c7f55d37742fea697ee54e1b20715}
        Как показать, что произвольное бесконечное множество \( A \) содерждит
        счётное подмножество?

        \begin{cloze}{1}
            Выбрать
            \begin{itemize}
                \item \( a_1 \) из \( A, \)
                \item \( a_2 \) из \( A \setminus \{ a_1 \}, \)
                \item \( a_3 \) из \( A \setminus \{ a_1, a_2 \}, \) \\
                    \dots
            \end{itemize}
            Получим счётное множество \( \{ a_1, a_2, a_3, \ldots \} \subset A.
            \)
        \end{cloze}
    \end{note}

    \begin{note}{61cad32098d341eb8086313887d6cd8c}
        Как показать, что любое подмножество \( B \) счётного подмножества \( A
        \) счётно?

        \begin{cloze}{1}
            Пронумеровать элементы множества \( B \) в порядке их появления в
            последовательности \( \{ a_1, a_2, a_3, \ldots  \} \) элементов
            множества \( A. \)
        \end{cloze}
    \end{note}

    \begin{note}{3639a29f97084a048aae918aefdb9100}
        Пусть \( A \) --- счётное множество, \( B \in A. \) Что можно сказать о
        множестве \( B? \)

        \begin{cloze}{1}
            \( B \) не более чем счётно.
        \end{cloze}
    \end{note}

    \begin{note}{bad29a5101fe46c3bd91ed4d7f33015b}
        Как показать, что не более чем счётное объединение не более чем счётных
        множеств не более чем счётное?

        \begin{cloze}{1}
            Расположить элементы множеств по строкам в бесконечную таблицу и
            пронумеровать их в порядке их появления на ``побочных'' диагоналях.
        \end{cloze}
    \end{note}

    \begin{note}{23eae0cde4e049379eab7d391cd31769}
        Как показать, что множество \( \mathbb Q  \) счетно?

        \begin{cloze}{1}
            Представить его как объединение не более чем счетного семейства не
            более чем счётных множеств \( \{ \mathbb Q _i \}_{i \in \mathbb N }
            , \) где \[
                \mathbb Q _q := \left\{ \frac{p}{q} \mid p \in \mathbb Z  \right\}.
            \]
        \end{cloze}
    \end{note}
\end{document}
