\documentclass[11pt, a5paper]{article}
\usepackage[width=10cm, top=0.5cm, bottom=2cm]{geometry}

\usepackage[T1,T2A]{fontenc}
\usepackage[utf8]{inputenc}
\usepackage[english,russian]{babel}
\usepackage{libertine}

\usepackage{amsmath}
\usepackage{amssymb}
\usepackage{amsthm}
\usepackage{mathrsfs}
\usepackage{framed}
\usepackage{xcolor}

\setlength{\parindent}{0pt}

\renewcommand{\thesection}{}
\renewcommand{\thesubsection}{}
\renewcommand{\thesubsubsection}{Note \arabic{subsubsection}}
\renewcommand{\theparagraph}{}

\newenvironment{note}[1]{\goodbreak\par\subsubsection{\hfill \color{lightgray}\tiny #1}}{}
\newenvironment{cloze}[2][\ldots]{\begin{leftbar}}{\end{leftbar}}
\newenvironment{icloze}[2][\ldots]{\ignorespaces}{\unskip}

\begin{document}
    \begin{note}{61211f8d77d34c9fa47e8872540de683}
        Каков первый шаг в доказательстве любого из законов де Моргана?

        \begin{cloze}{1}
            Рассмотреть произвольный элемент \( a, \) принадлежащий левой (или
            правой) части соответствующего равенства.
        \end{cloze}
    \end{note}


    \begin{note}{7e46fe30f7624833823e79c0fedc16df}
        Какова основная идея доказательства любого из законов де Моргана?

        \begin{cloze}{1}
            Надо показать, что условие принадлежности произвольного элемента \( a \)
            левой части совпадают с таковыми для правой части.
        \end{cloze}
    \end{note}

    \begin{note}{010c7f55d37742fea697ee54e1b20715}
        Как показать, что произвольное бесконечное множество \( A \) содержит
        счётное подмножество?

        \begin{cloze}{1}
            Выбрать
            \begin{itemize}
                \item \( a_1 \) из \( A, \)
                \item \( a_2 \) из \( A \setminus \{ a_1 \}, \)
                \item \( a_3 \) из \( A \setminus \{ a_1, a_2 \}, \) \\
                    \dots
            \end{itemize}
            Получим счётное множество \( \{ a_1, a_2, a_3, \ldots \} \subset A.
            \)
        \end{cloze}
    \end{note}

    \begin{note}{61cad32098d341eb8086313887d6cd8c}
        Как показать, что любое бесконечное подмножество \( B \) счётного
        подмножества \( A \) счётно?

        \begin{cloze}{1}
            Пронумеровать элементы множества \( B \) в порядке их появления в
            последовательности \( \{ a_1, a_2, a_3, \ldots  \} \) элементов
            множества \( A. \)
        \end{cloze}
    \end{note}

    \begin{note}{3639a29f97084a048aae918aefdb9100}
        Пусть \( A \) --- счётное множество, \( B \in A. \) Что можно сказать о
        множестве \( B? \)

        \begin{cloze}{1}
            \( B \) не более чем счётно.
        \end{cloze}
    \end{note}

    \begin{note}{bad29a5101fe46c3bd91ed4d7f33015b}
        Как показать, что не более чем счётное объединение не более чем счётных
        множеств не более чем счётно?

        \begin{cloze}{1}
            Расположить элементы множеств по строкам в таблицу и пронумеровать
            их в порядке их появления на ``побочных'' диагоналях.
        \end{cloze}
    \end{note}

    \begin{note}{23eae0cde4e049379eab7d391cd31769}
        Как показать, что множество \( \mathbb Q  \) счетно?

        \begin{cloze}{1}
            Представить его как объединение не более чем счетного семейства не
            более чем счётных множеств \( \{ \mathbb Q _i \}_{i \in \mathbb N }
            , \) где \[
                \mathbb Q _q := \left\{ \frac{p}{q} \mid p \in \mathbb Z  \right\}.
            \]
        \end{cloze}
    \end{note}

    \begin{note}{fa0bde6f987c45f9b12f1e7a19f5ed7f}
        Пусть \( [a,  b] \) --- невырожденный отрезок. Как можно задать биекцию
        \( \varphi : [a,  b] \to [0, 1] \)?

        \begin{cloze}{1}
            \[
                \varphi(x) = \frac{(x - a)^{n}}{(b - a)^{n}}, \quad n \in
                \mathbb N.
            \]
        \end{cloze}
    \end{note}

    \begin{note}{d7dc9d0004e9406e8bedb136412f6d07}
        Как доказать, что для любого бесконечного множества \( A \) и его конечного
        подмножества \( B \) (пусть \( |B| = m \)) \[
            A \setminus B \sim  A?
        \]

        \begin{cloze}{1}
            Рассмотрим произвольную последовательность \[
                \{ x_n \}_{n = 1} ^{\infty }
            \] несовпадающих элементов множества \( A \) такую, что первые её \( m \)
            элементов --- это все элементы множества \( B. \) Обозначим теперь \[
                \varphi(x) = \begin{cases}
                    x_{k + m}, & x = x_k, \\
                    x, & x \not\in \{ x_n \}_{n = 1}^{\infty }.
                \end{cases}
            \] Тогда \( \varphi : A \to A \setminus B \) --- биекция, а значит
            \( A \setminus B \sim A. \)
        \end{cloze}
    \end{note}

    \begin{note}{75dda33bf56f4c7dae2140052f8d6f52}
        Как доказать, что \( [0, 1] \sim \mathbb R  \)?

        \begin{cloze}{1}
            \begin{itemize}
                \item \( [0, 1] \sim (0, 1),  \) поскольку \( (0, 1) = [0, 1]
                    \setminus \{ 0, 1 \},  \)
                \item \( (0, 1) \sim \mathbb R, \)  поскольку \( \cot (\pi
                    x)|_{(0, 1)} \) --- биекция.
            \end{itemize}
            Получаем \( [0, 1] \sim (0, 1) \sim \mathbb R \implies [0, 1] \sim
            \mathbb R. \)
        \end{cloze}
    \end{note}

    \begin{note}{c8ec225de29d4338add7adcea48cc2a2}
        Приведите пример системы вложенных отрезков в множестве \( \mathbb Q  \)
        для которой не выполняется аксиома Кантора.

        \begin{cloze}{1}
            Можно рассмотреть последовательность вложенных отрезков \[
                \{ [1;2], [1{,}4;1{,}5], [1{,}41;1{,}42],
                    [1{,}414;1{,}415],\ldots \}
            \] концы которых --- все более и более точные десятичные приближения
            иррационального числа \( \sqrt{2}. \)
        \end{cloze}
    \end{note}

    \begin{note}{e12a9ee541074f3f83c0236b906973d1}
        \[
            A \setminus (A \setminus B) = \begin{icloze}{1}A \cap B.\end{icloze}
        \]
    \end{note}

    \begin{note}{59b526f72a4343fb936d1fa20561c886}
        Как доказать, что \; \( C_{n + 1}^{k + 1} = C_n^k + C_n^{k + 1} \)?
        \begin{cloze}{1}
            \begin{align*}
                C_{n + 1}^{k + 1} &= \frac{(n + 1)!}{(k + 1)!(n - k)!} \\
                &= C_n^k \cdot \left( \frac{n + 1}{k + 1}  \right) \\
                &= C_n^k \cdot \left( 1 + \frac{n - k}{k + 1}  \right) \\
                &= C_n^k + \frac{n! (n - k)}{k!(n - k)!(k + 1)} \\
                &= C_n^k + \frac{n!}{(k + 1)!(n - (k + 1))!} \\
                &= C_n^k + C_n^{k + 1}.
            \end{align*}
        \end{cloze}
    \end{note}

    \begin{note}{a20a03ca2ebe4b5d85249845f15f1561}
        Как доказать, что во всяком конечном подмножестве \(\mathbb{R}\) есть
        наибольший элемент?

        \begin{cloze}{1}
            По индукции:
            \begin{itemize}
                \item максимум множества из одного элемента есть сам этот
                    элемент;
                \item максимум множества из \( n > 1 \) элемента есть либо
                    максимум каких-либо \( n - 1 \) его элементов, либо значение
                    оставшегося \( n \)-ого элемента.
            \end{itemize}
        \end{cloze}
    \end{note}

    \begin{note}{02fab2f581504672bc9dc06a5dfa4166}
        Как доказать, что во всяком непустом ограниченном сверху множестве \( A
        \subset \mathbb Z \) есть наибольший элемент?

        \begin{cloze}{1}
            Выберем произвольный \( x_0 \in A \) и обозначим \[
                \begin{gathered}
                    A_0 = \{ x \mid x \in A \land x \geqslant x_0 \},  \\
                    A_1 = A \setminus A_0.
                \end{gathered}
            \]
            Тогда \( A_0 \) --- конечное подмножество \( \mathbb R, \) а значит
            существует \( \max A_0. \) При этом для любого \( x \in A \) имеем
            два случая:
            \begin{enumerate}
                \item если \( x \in A_0,  \) то \( x \leqslant \max A_0 \) по
                    определению максимума;
                \item если \( x \not\in A_0,  \) то по построению \( A_0 \)
                    имеем \[ \forall \hat{x} \in A_0 \quad x < \hat{x}, \] а
                    значит \( x < \max A_0. \)
            \end{enumerate}

            В любом случае имеем \( x \leqslant \max A_0, \) так что \( \max A_0
            = \max A \) по определению.
        \end{cloze}
    \end{note}

    \begin{note}{c7dd2e717d9c47199cf723b912cf4e34}
        Как доказать, что во всяком интервале есть хотя бы одно рациональное
        число?

        \begin{cloze}{1}
            Пусть \( (a, b) \) --- интервал в \( \mathbb R. \) Тогда по аксиоме
            Архимеда \[
               \exists n \in \mathbb N \quad n > \frac{1}{b - a} \implies b - a
               > \frac{1}{n}.
            \] Нетрудно показать, что \[
                \dfrac{\lfloor na \rfloor + 1}{n} \in (a, b) \cap \mathbb Q .
            \]
        \end{cloze}
    \end{note}

    \begin{note}{9428e0290086401db17d784b26f66839}
        Если \( \forall \varepsilon > 0 \quad |b - a| < \varepsilon, \) то
        \begin{icloze}{1}\( a = b. \)\end{icloze}
    \end{note}

    \begin{note}{3eaaa1f0f8624d8db6fa6824b7394a4a}
        Как доказать, что если \( \forall \varepsilon > 0 \quad |b - a| <
        \varepsilon, \) то \( a = b \)?

        \begin{cloze}{1}
            Допустим, что \( a \neq b. \) Тогда для \( \varepsilon = |b - a| > 0
            \) не выполняется \( \varepsilon < |b - a| \implies \) противоречие
            \( \implies a = b. \)
        \end{cloze}
    \end{note}

    \begin{note}{d8e380894b49477d9b690778e94ae82c}
        Как доказать, что у любой последовательности может быть не более одного
        предела?

        \begin{cloze}{1}
            Из определения предела \[
                \forall \varepsilon > 0 \quad \exists N \in \mathbb N \quad \forall n > N \quad
                \begin{cases}
                    |x_n - a| < \frac{\varepsilon}{2}, \\
                    |x_n - b| < \frac{\varepsilon}{2}.
                \end{cases}
            \]
            Но тогда \[
                \begin{aligned}
                    |a - b| &= |a - x_n + x_n - b| \leqslant |a - x_n| + |b - x_n| < \\
                            &< \frac{\varepsilon}{2} +
                            \frac{\varepsilon}{2} = \varepsilon.
                \end{aligned}
            \]
            Получаем, что \(
                \forall \varepsilon > 0 \quad |b - a| <
                \varepsilon \implies a = b.
            \)
        \end{cloze}
    \end{note}

    \begin{note}{1e1fe886e2334eb18a97c2ab2cfadc49}
        Как доказать, что любая сходящаяся последовательность ограничена сверху?

        \begin{cloze}{1}
            Пусть \( x_n \to a \in \mathbb R. \) Возьмём \( \varepsilon > 0; \)
            тогда по определению предела \[
                \exists N \in \mathbb N \quad \forall n > N \quad a -
                1 < x_n < a + 1,
            \] т.е. множество \( A := \{ x_n \mid n > N \} \) ограниченно сверху
            значением \( a + 1, \) но тогда все множество \( A \) ограниченно
            сверху значением \[
                \max \{ x_1, x_2, \ldots, x_N, a - 1 \}.
            \]
        \end{cloze}
    \end{note}

    \begin{note}{4712e0fed7a44e8bac81bae5c16d1810}
        Если \( \forall \varepsilon > 0 \quad a - b < \varepsilon, \) то
        \begin{icloze}{1}\( a \leqslant b. \)\end{icloze}
    \end{note}

    \begin{note}{b0c8d698cc304040a59c01dc5852ed8b}
        Как доказать, что если \( \forall \varepsilon > 0 \quad a - b <
        \varepsilon, \) то \( a \leqslant b \)?

        \begin{cloze}{1}
            Допустим, что \( a > b. \) Тогда для \( \varepsilon = a - b > 0 \)
            не выполняется \( a - b < \varepsilon \implies  \) противоречие \(
            \implies a \leqslant b. \)
        \end{cloze}
    \end{note}

    \begin{note}{f492d02ab40942c3bb31727a84b97f36}
        Как доказать теорему о предельном переходе в неравенстве для
        последовательностей?

        \begin{cloze}{1}
            Пусть \( \quad x_n \to a, \quad y_n \to b, \quad \forall n\; x_n
            \leqslant y_n. \)

            Тогда из определения предела \[
                \begin{gathered}
                    \forall \varepsilon > 0 \quad \exists N \in \mathbb N \quad
                    \forall n > N \quad \begin{cases}
                        a - \frac{\varepsilon}{2} < x_n, \\
                        y_n < b + \frac{\varepsilon}{2}
                    \end{cases} \\
                    \begin{aligned}
                        &\implies a - \frac{\varepsilon}{2} < x_n \leqslant y_n < b + \frac{\varepsilon}{2} \\
                        &\implies a - b < \varepsilon.
                    \end{aligned}
                \end{gathered}
            \]
            Получаем, что \( \forall \varepsilon > 0 \quad a - b < \varepsilon
            \implies a \leqslant b. \)
        \end{cloze}
    \end{note}

    \begin{note}{6d56b828715344a4981b505177237b3d}
        Как доказать теорему о сжатой последовательности?

        \begin{cloze}{1}
            Пусть \( \quad x_n \to a, \quad z_n \to a, \quad
            \forall n\; x_n \leqslant y_n \leqslant z_n. \)

            Тогда из определения предела \[
                \begin{gathered}
                    \forall \varepsilon > 0 \quad \exists N \in \mathbb N \quad
                    \forall n > N \quad \begin{cases}
                        a - \varepsilon < x_n, \\
                        z_n < a + \varepsilon
                    \end{cases} \\
                    \begin{aligned}
                        &\implies a - \varepsilon < x_n \leqslant y_n \leqslant
                        x_n < a + \varepsilon \\
                        &\implies a - \varepsilon < y_n < a + \varepsilon \\
                        &\overset{\text{\tiny def}}\implies y_n \to a.
                    \end{aligned}
                \end{gathered}
            \]
        \end{cloze}
    \end{note}

    \begin{note}{120b6a6635764d8eb06a8353c96ff71c}
        Как доказать, что \( \forall \{ x_n \} \quad x \to a \iff x - a \to 0 \)?

        \begin{cloze}{1}
            Из определения предела \begin{multline*}
                x \to a \overset{\text{\tiny def}}\iff  \\
                \forall \varepsilon > 0 \quad \exists N \in \mathbb N \quad
                \forall n > N \quad |x - a| < \varepsilon \\
                \overset{\text{\tiny def}}\iff x - a \to 0.
            \end{multline*}
        \end{cloze}
    \end{note}

    \begin{note}{ee8b4fecca4148bcb931714d65a0f370}
        Как доказать, что произведение бесконечно малой последовательности на
        ограниченную есть бесконечно малая?

        \begin{cloze}{1}
            Пусть \( x_n \to 0, \quad \forall n\; |y_n| \leqslant M \in \mathbb R _+. \)

            Тогда из определения предела \begin{multline*}
                \forall \varepsilon > 0 \quad \exists N \in \mathbb N \quad
                \forall n > N \quad |x_n| < \frac{\varepsilon}{M} \\
                \implies |x_n y_n| < \varepsilon \overset{\text{\tiny def}}\implies  x_n y_n \to 0.
            \end{multline*}
        \end{cloze}
    \end{note}
\end{document}
