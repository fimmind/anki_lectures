%! TeX root = ./main.tex
\documentclass[11pt, a5paper]{article}
\usepackage[width=10cm, top=0.5cm, bottom=2cm]{geometry}

\usepackage[T1,T2A]{fontenc}
\usepackage[utf8]{inputenc}
\usepackage[english,russian]{babel}
\usepackage{libertine}

\usepackage{amsmath}
\usepackage{amssymb}
\usepackage{amsthm}
\usepackage{mathrsfs}
\usepackage{framed}
\usepackage{xcolor}

\setlength{\parindent}{0pt}

% Force \pagebreak for every section
\let\oldsection\section
\renewcommand\section{\pagebreak\oldsection}

\renewcommand{\thesection}{}
\renewcommand{\thesubsection}{Note \arabic{subsection}}
\renewcommand{\thesubsubsection}{}
\renewcommand{\theparagraph}{}

\newenvironment{note}[1]{\goodbreak\par\subsection{\hfill \color{lightgray}\tiny #1}}{}
\newenvironment{cloze}[2][\ldots]{\begin{leftbar}}{\end{leftbar}}
\newenvironment{icloze}[2][\ldots]{%
  \ignorespaces\text{\tiny \color{lightgray}\{\{c#2::}\hspace{0pt}%
}{%
  \hspace{0pt}\text{\tiny\color{lightgray}\}\}}\unskip%
}


\begin{document}
\section{Лекция 09.11.22}
\begin{note}{e60e9a5778ca41aaabad4a4a7e6c4fea}
    Какие есть основные виды дифференциальных уравнений?

    \begin{cloze}{1}
        Обыкновенные; в частных производных.
    \end{cloze}
\end{note}

\begin{note}{5e9fddb7111e4705aa4a145bb98b111f}
    \begin{icloze}{2}Обыкновенные дифференциальные уравнения\end{icloze} --- это \begin{icloze}{1}уравнения относительно функции одной переменной и её производных.\end{icloze}
\end{note}

\begin{note}{304a53cb8187428aaba248e942576ea2}
    ``\begin{icloze}{2}Обыкновенные дифференциальные уравнения\end{icloze}'' сокращается как ``\begin{icloze}{1}ОДУ.\end{icloze}''
\end{note}

\begin{note}{27aba707c8db4ff0aaa02aa522cb353d}
    \begin{icloze}{2}Уравнения в частных производных\end{icloze} --- это \begin{icloze}{1}уравнения относительно функции нескольких переменных и её частных производных.\end{icloze}
\end{note}

\begin{note}{54188b1d277440558390f93807cf9e7e}
    \begin{icloze}{2}Уравнения в частных производных\end{icloze} в русскоязычной среде так же называют \begin{icloze}{1}уравнениями математической физики.\end{icloze}
\end{note}

\begin{note}{6ed94a06c0164651bcacf8ee9c9f96fb}
    ``\begin{icloze}{2}Уравнения в частных производных\end{icloze}'' сокращается как ``\begin{icloze}{1}УрЧП.\end{icloze}''
\end{note}

\begin{note}{0fd7b116352242aba166bb75e3487f7e}
    \begin{icloze}{2}Порядком\end{icloze} дифференциального уравнения называется \begin{icloze}{1}порядок старшей производной, в него входящей.\end{icloze}
\end{note}

\begin{note}{69a06ae8b8d1413f84c38cc0ab521158}
    Является ли
    \[
        F(x, y) = 0,\: y = y(x)
    \]
    дифференциальным уравнением?

    \begin{cloze}{1}
        Нет, потому что нет производных.
    \end{cloze}
\end{note}

\begin{note}{f8ab33c8a60a4901a4c26ccff8a6fd1a}
    Множество \({ G \subset \mathbb R^n }\) называется \begin{icloze}{2}областью,\end{icloze} если \begin{icloze}{1}оно открыто и связно.\end{icloze}
\end{note}

\begin{note}{1425377052ae4b228fc834d5b4f63182}
    ОДУ первого порядка называется \begin{icloze}{2}разрешённым относительно производной,\end{icloze} если оно имеет вид
    \begin{icloze}{1}
        \[
            \frac{dy}{dx} = f(x, y),
        \]
    \end{icloze}
    где \({ f }\) --- \begin{icloze}{2}функция на области в \({ \mathbb R^2 }\).\end{icloze}
\end{note}

\begin{note}{aa5c740235f848c79fb3bbc39d4a3160}
    Функция \({ y }\) называется \begin{icloze}{2}решением ОДУ на множестве \({ X }\),\end{icloze} если \begin{icloze}{1}в любой точке \({ X }\) её подстановка её значений в ОДУ имеет смысл и приводит к верному равенству.\end{icloze}
\end{note}

\begin{note}{8515eff1b5844e0cba265de7445cf1a0}
    Пусть \({ y }\) --- \begin{icloze}{3}решение ОДУ.\end{icloze}
    \begin{icloze}{1}График \({ y }\)\end{icloze} называется \begin{icloze}{2}интегральной кривой этого уравнения.\end{icloze}
\end{note}

\begin{note}{01533944715c4477a02b8f21087f96d2}
    Сколько решений может иметь произвольное ОДУ?

    \begin{cloze}{1}
        Сколь угодно много.
    \end{cloze}
\end{note}

\begin{note}{76c92fefc3414e8da3c06f458e9e80ee}
    В чём состоит задача Коши для ОДУ первого порядка?

    \begin{cloze}{1}
        Найти решение, отвечающее начальным условиям.
    \end{cloze}
\end{note}

\begin{note}{94e8aba670fa45eab4fae8fee8d23568}
    Что есть ``начальные условия'' из формулировки задачи Коши для ОДУ первого порядка?

    \begin{cloze}{1}
        \({ y(x_0) = y_0 }\) для фиксированных \({ x_0, y_0 }\).
    \end{cloze}
\end{note}

\begin{note}{0277eb8d5c00466ca6bb797bd58c8279}
    Как называются значения \({ (x_0, y_0) }\) в задаче Коши для ОДУ первого порядка?

    \begin{cloze}{1}
        Начальные данные.
    \end{cloze}
\end{note}

\begin{note}{aae5b8ef39d14aab9d038ebe894b7b99}
    Какие значения могут принимать начальные данные в задаче Коши для ОДУ первого порядка?

    \begin{cloze}{1}
        Любые, для которых ОДУ имеет смысл.
    \end{cloze}
\end{note}

\begin{note}{4ecfb902661d484682379f7f0b7b2567}
    На каком множестве нужно найти решение задачи Коши с начальными данными \({ (x_0, y_0) }\)?

    \begin{cloze}{1}
        Интервал, включающий \({ x_0 }\).
    \end{cloze}
\end{note}

\begin{note}{c6f2ec8c4ae142118b3840ddb828a96b}
    Как называется теорема о существовании решения задачи Коши для ОДУ первого порядка, разрешённого относительно производной?

    \begin{cloze}{1}
        Теорема Пеано.
    \end{cloze}
\end{note}

\begin{note}{3623dc3931814e958fe21c89ae0b0c6e}
    Какое уравнение рассматривается в теореме Пеано?

    \begin{cloze}{1}
        ОДУ первого порядка, разрешённое относительно производной.
    \end{cloze}
\end{note}

\begin{note}{5abd317c6b0b4b02ab50a72f0ea7d792}
    При каком условии можно что-либо заключить из теоремы Пеано?

    \begin{cloze}{1}
        Функция, задающая разрешённое ОДУ, непрерывна на области.
    \end{cloze}
\end{note}

\begin{note}{dfdb355bf4d84b1d89e76ebd9e215e4a}
    Что можно заключить из теоремы Пеано?

    \begin{cloze}{1}
        Для любой точки существует решение задачи Коши с этими начальными данными.
    \end{cloze}
\end{note}

\begin{note}{3290d174cc33491e924cbd8ec7137a8a}
    Каков геометрический смысл теоремы Пеано?

    \begin{cloze}{1}
        Через любую точку области проходит интегральная кривая.
    \end{cloze}
\end{note}

\begin{note}{59c04413608e4901a682dfb9a1d29161}
    Что называют точкой единственности для уравнения
    \[
        \frac{dy}{dx} = f(x, y)\,?
    \]

    \begin{cloze}{1}
        Точка, для которой любые два решения задачи Коши совпадают (в какой-то окрестности).
    \end{cloze}
\end{note}

\begin{note}{7dbfe4396c5545578771b6066869de5a}
    В каком именно смысле совпадают любые два решения соответствующей задачи Коши в определении точки единственности уравнения \({ \frac{dy}{dx} = f(x, y) }\)?

    \begin{cloze}{1}
        Они равны на некоторой \({ V_{\delta}(x_0) }\).
    \end{cloze}
\end{note}

\begin{note}{c197734ceb654182bc3221591296e2f6}
    Пусть \({ f }\) --- функция на области в \({ \mathbb R^2 }\),
    \[
        \frac{dy}{dx} = f(x, y)\,.
    \]
    Тогда если \({ f }\) и \({ \frac{\partial f}{\partial y} }\) непрерывны, то \begin{icloze}{1}любая точка области является точкой единственности.\end{icloze}
\end{note}

\begin{note}{33930ae47c60424a9e1c3f6f5482f236}
    Пусть \({ f }\) --- функция на области в \({ \mathbb R^2 }\).
    При каком условии задача Коши для \({ \frac{dy}{dx} = f(x, y) }\) однозначно разрешима в любой точке?

    \begin{cloze}{1}
        \({ f }\) и \({ \frac{\partial f}{\partial y} }\) непрерывны.
    \end{cloze}
\end{note}

\begin{note}{785b85c103cb446090843629aba6f668}
    Пусть \({ f }\) --- функция на области в \({ \mathbb R^2 }\).
    Что называют особым решением уравнения \({ \frac{dy}{dx} = f(x, y) }\)?

    \begin{cloze}{1}
        Решение, любая точка графика которого не является точкой единственности (внутри интервала).
    \end{cloze}
\end{note}

\begin{note}{37a7853a3e8d43aa9d89f5bb427f5ebc}
    Пусть \({ f }\) --- функция на области в \({ \mathbb R^2 }\).
    Как называется решение уравнения \({ \frac{dy}{dx} = f(x, y) }\), любая точка графика которого не является точкой единственности?

    \begin{cloze}{1}
        Особое решение.
    \end{cloze}
\end{note}

\begin{note}{8c95e3912938472e90263540e560fb33}
    Пусть \({ f }\) --- функция на области в \({ \mathbb R^2 }\).
    Что называют общим решением уравнения \({ \frac{dy}{dx} = f(x, y) }\)?

    \begin{cloze}{1}
        Параметризованная совокупность решений, содержащая решение задачи Коши для любой точки области.
    \end{cloze}
\end{note}

\begin{note}{94b869be37d9483ea16e55b821468d6a}
    Пусть \({ f }\) --- функция на области в \({ \mathbb R^2 }\).
    Как задаётся общее решение уравнения \({ \frac{dy}{dx} = f(x, y) }\)?

    \begin{cloze}{1}
        Отображение \({ \Phi(x, c) }\), где \({ c }\) --- параметр, \({ x }\) --- переменная.
    \end{cloze}
\end{note}

\begin{note}{845626784e214307bef975cf77f33ece}
    Пусть \({ f }\) --- функция на области в \({ \mathbb R^2 }\).
    Что называют частным решением уравнения \({ \frac{dy}{dx} = f(x, y) }\)?

    \begin{cloze}{1}
        Одно из решений, входящих в некоторое общее решение.
    \end{cloze}
\end{note}

\begin{note}{e4f00b810c5346e7883e4848222bac62}
    \begin{icloze}{2}Векторное поле\end{icloze} --- это \begin{icloze}{1}отображение из линейного пространства в себя.\end{icloze}
\end{note}

\begin{note}{c910d42c89d94c0e911e1b327e5b0a36}
    Для каких ОДУ имеет смысл понятие поля направлений?

    \begin{cloze}{1}
        ОДУ первого порядка, разрешённое относительно производной.
    \end{cloze}
\end{note}

\begin{note}{bf5a5b553b8b4f74905e400cce49df6c}
    Пусть \({ f }\) --- функция на области в \({ \mathbb R^2 }\).
    Что называют полем направлений уравнения \({ \frac{dy}{dx} = f(x, y) }\)?

    \begin{cloze}{1}
        Векторное поле нормализованных векторов, задающих направления касательных к интегральным кривым.
    \end{cloze}
\end{note}

\begin{note}{7f47ec07174947169cc1a095d3b5dbd0}
    Пусть \({ f }\) --- функция на области в \({ \mathbb R^2 }\).
    Как строится визуальное представление поля направлений уравнения \({ \frac{dy}{dx} = f(x, y) }\)?

    \begin{cloze}{1}
        Через каждую точки сетки проводится соответствующе наклонённый отрезок.
    \end{cloze}
\end{note}

\begin{note}{a9509cb3d319496ca4f8ce25b5b988b2}
    Пусть \({ f }\) --- функция на области в \({ \mathbb R^2 }\).
    Гладкая кривая является \begin{icloze}{2}интегральной кривой\end{icloze} уравнения \({ \frac{dy}{dx} = f(x, y) }\) \begin{icloze}{3}тогда и только тогда, когда\end{icloze}
    \begin{icloze}{1}в любой точке она касается соответствующего элемента поля направлений.\end{icloze}

    \begin{center}
        \tiny
        (в терминах поля направлений)
    \end{center}
\end{note}

\begin{note}{1895f43f66e747cc9ce6e8a4dd317258}
    Пусть \({ f }\) --- функция на области в \({ \mathbb R^2 }\).
    Что называется изоклиной уравнения \({ \frac{dy}{dx} = f(x, y) }\)?

    \begin{cloze}{1}
        Кривая, во всех точках которой значение поля направлений одинаково.
    \end{cloze}
\end{note}

\begin{note}{efd8be00720a4e449757b00e4434d684}
    Пусть \({ f }\) --- функция на области в \({ \mathbb R^2 }\).
    Каким уравнением задаётся произвольная изоклина уравнения \({ \frac{dy}{dx} = f(x, y) }\)?

    \begin{cloze}{1}
        \({ f(x, y) = c }\) для \({ c \in \mathbb R }\).
    \end{cloze}
\end{note}

\section{Лекция 16.11.22}
\begin{note}{84332aa7764648b3b6b73355b0fb7064}
    Пусть \begin{icloze}{4}\({ G }\) --- область в \({ \mathbb R^2 }\).\end{icloze}
    Тогда выражение вида
    \[
        \begin{icloze}{2}m(x, y) \cdot dx + n(x, y) \cdot dy = 0\,,\end{icloze} \quad \text{где}\ \begin{icloze}{3}m, n : G \to \mathbb R\,,\end{icloze}
    \]
    называется \begin{icloze}{1}ОДУ первого порядка в симметричной форме.\end{icloze}
\end{note}

\begin{note}{0a8383be5d30458db9ab476e9ea34e84}
    Что называется решением ОДУ первого порядка в симметричной форме?

    \begin{cloze}{1}
        Решение любого из порождённых ОДУ, разрешённых относительно производной.
    \end{cloze}
\end{note}

\begin{note}{89d64566d77b4dc7b52af28eab7dae35}
    Какие два уравнения порождает ОДУ первого порядка в симметричной форме?

    \begin{cloze}{1}
        \[
            \frac{dy}{dx} = \cdots \quad \text{и} \quad \frac{dx}{dy} = \cdots
        \]
    \end{cloze}
\end{note}

\begin{note}{0981a7837c7644328e3f6ccbd939ec02}
    Всегда ли ОДУ первого порядка в симметричной форме порождает два уравнения?

    \begin{cloze}{1}
        Нет.
    \end{cloze}
\end{note}

\begin{note}{0bf3be66da26454f981de93a9f57fdb4}
    Пусть дано ОДУ в симметричной форме
    \[
        m(x, y) dx + n(x, y) dy = 0\,.
    \]
    \begin{icloze}{2}Точку, в которой и \({ m }\), и \({ n }\) обращаются в \({ 0 }\),\end{icloze} называют \begin{icloze}{1}особой точкой этого уравнения.\end{icloze}
\end{note}

\begin{note}{4fb058ed5261465fbcf62958e5b93441}
    Как ставится задача Коши для ОДУ первого порядка в симметричной форме?

    \begin{cloze}{1}
        Найти решение для любой из порождённых задач Коши для ОДУ, разрешённых относительно производной.
    \end{cloze}
\end{note}

\begin{note}{0d205d054d3843c19a806774476762c0}
    Пусть дано ОДУ в симметричной форме
    \[
        m(x, y) dx + n(x, y) dy = 0\,.
    \]
    Что называется неособой точкой этого уравнения?

    \begin{cloze}{1}
        Точка, в которой либо \({ m }\), либо \({ n }\) не обращается в \({ 0 }\).
    \end{cloze}
\end{note}

\begin{note}{61ddfc362e5649ab9020c85dd47f4454}
    При каком условии ОДУ в симметричной форме гарантированно имеет решение задачи Коши?

    \begin{cloze}{1}
        Если точка не является особой и функции при \({ dx }\) и \({ dy }\) непрерывны.
    \end{cloze}
\end{note}

\begin{note}{0ad4cca0576a409cbb3871385860853c}
    Что можно сказать про ОДУ в симметричной форме
    \[
        m(x, y) dx + n(x, y) dy = 0\,,
    \]
    если \({ m, n }\) непрерывны?

    \begin{cloze}{1}
        Оно имеет решение задачи Коши для любой неособой точки.
    \end{cloze}
\end{note}

\begin{note}{3d3a825102e94d268fceedc112aee02b}
    В чём основная идея доказательства теоремы о достаточного условия существования решения задачи Коши для ОДУ первого порядка в симметричной форме?

    \begin{cloze}{1}
        Теорема Пеано для порождённого ОДУ, разрешённого относительно производной.
    \end{cloze}
\end{note}

\begin{note}{84670525aec14b12ab1c60bc7c6cb90f}
    Какое есть ``тонкое'' место в доказательстве достаточного условия существования решения задачи Коши для ОДУ первого порядка в симметричной форме?

    \begin{cloze}{1}
        Применить теорему о стабилизации непрерывной функции для неравенства нулю в какой-то окрестности.
    \end{cloze}
\end{note}

\begin{note}{e22e81451b75462bb23f42da429d0822}
    Какое есть дополнение к достаточному условию существования решения задачи Коши для ОДУ первого порядка в симметричной форме?

    \begin{cloze}{1}
        Если обе определяющие функции не обращаются в ноль, то решение ``двойственно.''
    \end{cloze}
\end{note}

\begin{note}{c5937908ffc3421b88d12e29cd18618a}
    В каком смысле двойственно решение задачи Коши в дополнении к достаточному условию существования решения задачи Коши для ОДУ первого порядка в симметричной форме?

    \begin{cloze}{1}
        Обратная функция существует (на интервале) и является решением второй из порождённых задач Коши.
    \end{cloze}
\end{note}

\begin{note}{c77c237173dd4909bef0d7fbb475945a}
    Пусть для \({ m, n \in C(G) }\) дано ОДУ в симметричной форме
    \[
        m(x, y) dx + n(x, y) dy = 0\,,
    \]
    и \({ y = \varphi(x) }\) --- решение задачи Коши в т. \({ (x_0, y_0) }\).
    Почему \({ \varphi }\) обратима (на интервале)?

    \begin{cloze}{1}
        \({ \varphi' \neq 0  }\) и следствие из теоремы Дарбу о строгой монотонности.
    \end{cloze}
\end{note}

\begin{note}{1408fcbdb21643b29a526f5c9c698052}
    Что называется точкой единственности для ОДУ первого порядка в симметричной форме?

    \begin{cloze}{1}
        Точка, являющаяся точкой единственности для обеих из порождённых задач Коши.
    \end{cloze}
\end{note}

\begin{note}{f037d477228b4d1a983be67a181f7fc6}
    Каково условие для единственности решения задачи Коши для ОДУ первого порядка в симметричной форме?

    \begin{cloze}{1}
        Функции при \({ dx }\) и \({ dy }\) являются гладкими и не равны нулю в точке.
    \end{cloze}
\end{note}

\begin{note}{b4d211a3083b4bfca552c183c7cd0ce8}
    Почему в теореме о единственности решения задачи Коши для ОДУ первого порядка в симметричной форме мы требуем неравенство нулю функций при \({ dx }\) и \({ dy }\)?

    \begin{cloze}{1}
        Иначе какая-то из задач Коши не имеет смысла.
    \end{cloze}
\end{note}

\begin{note}{91cb20330134439695809b01acd7909f}
    В каком смысле единственно решение задачи Коши для ОДУ первого порядка в симметричной форме в теореме о единственность?

    \begin{cloze}{1}
        Точка является точкой единственности.
    \end{cloze}
\end{note}

\begin{note}{c34ebe00830e4594b9f2df519066c9f8}
    В чём ключевая идея доказательства теоремы о единственности решения задачи Коши для ОДУ первого порядка в симметричной форме?

    \begin{cloze}{1}
        Теорема о единственности для ОДУ, разрешённого относительно производной.
    \end{cloze}
\end{note}

\begin{note}{83eb2e7882ac49fd950bc3c5d734f88c}
    Если \({ (x_0, y_0) }\) является точкой единственности для ОДУ первого порядка, то через неё проходит \begin{icloze}{1}единственная интегральная кривая.\end{icloze}
\end{note}

\begin{note}{931e9e5942144396af4bc98f8d917c45}
    В каком смысле единственна интегральная кривая, проходящая через точку единственности для ОДУ первого порядка?

    \begin{cloze}{1}
        Любые две интегральные кривые совпадают на какой-то окрестности.
    \end{cloze}
\end{note}

\begin{note}{67befaeaa43b4d1daf13c54d0f3c9a18}
    Чем в первую очередь является интеграл ОДУ первого порядка в симметричной форме?

    \begin{cloze}{1}
        Гладкая функция на всей области.
    \end{cloze}
\end{note}

\begin{note}{6cd34528c11649468f156e7160361bc4}
    Какому соотношению по определению должен удовлетворять интеграл \({ u }\) ОДУ в симметричной форме
    \[
        m(x, y) dx + n(x, y) dy = 0\,?
    \]

    \begin{cloze}{1}
        \[
            \begin{vmatrix}
                \frac{\partial u}{\partial x} & \frac{\partial u}{\partial y} \\
                m & n
            \end{vmatrix}
            \equiv 0.
        \]
    \end{cloze}
\end{note}

\begin{note}{4e43bac3e13f49e890052b539cb99a0a}
    Интеграл \({ u }\) ОДУ в симметричной форме
    \[
        m(x, y) dx + n(x, y) dy = 0
    \]
    называется \begin{icloze}{2}гладким,\end{icloze} если \begin{icloze}{1}\({ \nabla u \neq 0 }\) на всей области.\end{icloze}
\end{note}

\begin{note}{d57e85cf8770406ca4be0153888980ff}
    Каково основное свойство гладкого интеграла \({ u }\) ОДУ первого порядка в симметричной форме?

    \begin{cloze}{1}
        Решение уравнения \({ u(x, y) = u(x_0, y_0) }\) есть решение задачи Коши в \({ (x_0, y_0) }\).
    \end{cloze}
\end{note}

\begin{note}{750fae6a0be8412885d8d65ccd8b532f}
    Пусть \({ u }\) --- гладкий интеграл ОДУ первого порядка в симметричной форме.
    Для каких точек решение уравнения
    \[
        u(x, y) = u(x_0, y_0)
    \]
    есть решение задачи Коши в \({ (x_0, y_0) }\)?

    \begin{cloze}{1}
        Для неособых.
    \end{cloze}
\end{note}

\begin{note}{3ec100db172a42a4821a9c2a59df0cdc}
    В контексте ОДУ первого порядка в симметричной форме, для какого объекта \({ u }\) решение уравнения
    \[
        u(x, y) = u(x_0, y_0)
    \]
    гарантированно является решением задачи Коши в неособой точке \({ (x_0, y_0) }\)?

    \begin{cloze}{1}
        Гладкий интеграл.
    \end{cloze}
\end{note}

\begin{note}{2d035fe05f234a678fb45f0a77098ee9}
    В чём ключевая идея доказательства теоремы о представлении решения задачи Коши для ОДУ первого порядка в симметричной форме через его интеграл?

    \begin{cloze}{1}
        Теорема о неявной функции для \({ u(x, y) - u(x_0, y_0) = 0 }\).
    \end{cloze}
\end{note}

\begin{note}{8ee64b4f58d64f3e9c544c3728edc66c}
    Что называется общим интегралом ОДУ первого порядка в симметричной форме?

    \begin{cloze}{1}
        Выражение \({ u(x, y) = c }\), где \({ u }\) --- интеграл уравнения.
    \end{cloze}
\end{note}

\begin{note}{4b9745c311764d9aa57c35f678ca0b4b}
    Пусть \({ u }\) --- интеграл ОДУ первого порядка в симметричной форме.
    Как называется выражение
    \[
        u(x, y) = c\,?
    \]

    \begin{cloze}{1}
        Общий интеграл.
    \end{cloze}
\end{note}

\begin{note}{d24cae635b5149fca8646f3627b9bf90}
    Пусть \({ u }\) --- интеграл ОДУ первого порядка в симметричной форме.
    Что можно сказать о произвольном решении \({ y(x) }\) этого ОДУ?

    \begin{cloze}{1}
        \({ u(x, y(x)) \equiv const }\) на всём интервале.
    \end{cloze}
\end{note}

\begin{note}{37ad7fd23ba04b2b8bf1e5946420d716}
    Каким свойством должен должен обладать интеграл \({ u }\) ОДУ первого порядка в симметричной форме, чтобы для любого решения \({ y(x) }\) было верно
    \[
        u(x, y(x)) \equiv const\,?
    \]

    \begin{cloze}{1}
        Никаким.
    \end{cloze}
\end{note}

\section{Лекция 23.11.22}
\begin{note}{3ae25bc9f84c408eb1dfcf3dc3b35ed1}
    Какое выражение называется ОДУ с разделёнными переменными?

    \begin{cloze}{1}
        \({ m(x) dx + n(y) dy = 0 }\).
    \end{cloze}
\end{note}

\begin{note}{d625dac9c8dd43e0a5fa91bdeec1f8da}
    На каких множествах определены функции-коэффициенты, задающие ОДУ с разделёнными переменными?

    \begin{cloze}{1}
        Некоторые два интервала.
    \end{cloze}
\end{note}

\begin{note}{ec328e8cd19d4c57a27c32e263138ce7}
    На какой области рассматриваются ОДУ с разделёнными переменными?

    \begin{cloze}{1}
        Декартово произведение областей определения функций-коэффициентов.
    \end{cloze}
\end{note}

\begin{note}{5cac08cbed25499493dc1b18d0df3ce1}
    Какими должны быть функции-коэффициенты, задающие ОДУ с разделёнными переменными?

    \begin{cloze}{1}
        Непрерывными.
    \end{cloze}
\end{note}

\begin{note}{afa181edb7c84bdea0ac73d5a39b338e}
    Чем примечательны ОДУ с разделёнными переменными?

    \begin{cloze}{1}
        У него есть общий интеграл в известной форме.
    \end{cloze}
\end{note}

\begin{note}{8422d34d16d74daba0b88c018148285d}
    Как выражается интеграл \({ u(x, y) }\) ОДУ с разделёнными переменными
    \[
        m(x) dx + n(y) dy = 0\,?
    \]

    \begin{cloze}{1}
        \[
            \int_{x_0}^{x} m + \int_{y_0}^{y} n\,.
        \]
    \end{cloze}
\end{note}

\begin{note}{8dbb27c86a01499b83be44a189e22c09}
    При каком условии интеграл ОДУ с разделёнными переменными является гладким?

    \begin{cloze}{1}
        Если нет особых точек.
    \end{cloze}
\end{note}

\begin{note}{af655cb8e5aa41348f1b128abe72c8f2}
    Какие значения могут принимать параметры \({ (x_0, y_0) }\) из общего вида интеграла ОДУ с разделёнными переменными?

    \begin{cloze}{1}
        Любая точка из области определения.
    \end{cloze}
\end{note}

\begin{note}{a7ee1153931c46d59a1a71e028b00b0a}
    Как на практике задаётся общий интеграл ОДУ с разделёнными переменными?

    \begin{cloze}{1}
        С неопределёнными интегралами.
    \end{cloze}
\end{note}

\begin{note}{dd91a079474849748c001328b0709b66}
    Что называется ОДУ с разделяющимися переменными?

    \begin{cloze}{1}
        ОДУ в симметричной форме, у которого коэффициенты представляются как произведения функций одной переменной.
    \end{cloze}
\end{note}

\begin{note}{4405ce3a7fc64f66a33baddd2945d5e2}
    Какими должны быть функции-коэффициенты, задающие ОДУ с разделяющимися переменными?

    \begin{cloze}{1}
        Непрерывными.
    \end{cloze}
\end{note}

\begin{note}{7007d702730d4b4bb53ea6b9fc990bfa}
    Какие ОДУ называются уравнениями в полных дифференциалах?

    \begin{cloze}{1}
        Оду в симметричной форме, представимые в виде \({ du = 0 }\), где \({ u }\) --- гладкая функция на области.
    \end{cloze}
\end{note}

\begin{note}{a18c66c2343a4d1ab182426f9393edac}
    Представимость \({ m(x, y) dx + n(x, y) dy = 0 }\) в виде \({ du = 0 }\) в определении ОДУ в полных дифференциалах означает, что
    \begin{icloze}{1}
        \[
            \nabla u = (m, n).
        \]
    \end{icloze}
\end{note}

\begin{note}{5c173b6af64a42a48e1a9e79cf27b712}
    Чем примечательны ОДУ в полных дифференциалах?

    \begin{cloze}{1}
        У них есть общий интеграл в известной форме.
    \end{cloze}
\end{note}

\begin{note}{ce9b4078bbd64be0893565bf2b878c9d}
    Как выражается интеграл ОДУ в полых дифференциалах?

    \begin{cloze}{1}
        Функция \({ u }\) из определения и есть интеграл.
    \end{cloze}
\end{note}

\begin{note}{d545e5481d8648c084c87047ded49bc4}
    Как определить, является ли уравнение ОДУ в симметричной форме
    \[
        m(x, y) dx + n(x, y) dy = 0
    \]
    ОДУ в полных дифференциалах?

    \begin{cloze}{1}
        Является \({ \iff m'_y = n'_x }\).
    \end{cloze}
\end{note}

\begin{note}{385f908cec6342339c9a708dc9e734a5}
    Что можно сказать о ОДУ в симметричной форме
    \[
        m(x, y) dx + n(x, y) dy = 0\,,
    \]
    если \({ m'_y \equiv n'_x }\)?

    \begin{cloze}{1}
        Это ОДУ в полных дифференциалах.
    \end{cloze}
\end{note}

\begin{note}{2d3ee1ddf95c48f0b570c8fcbf997605}
    Какими должны быть функции-коэффициенты ОДУ в симметричной форме в критерии ОДУ в полных дифференциалах?

    \begin{cloze}{1}
        Гладкими (т.е. из \({ C^{1} }\).)
    \end{cloze}
\end{note}

\begin{note}{e5db36f5f9f8404e8d347d5f20461e67}
    В чём ключевая идея доказательства критерия ОДУ в полных дифференциалах (необходимость)?

    \begin{cloze}{1}
        Достаточное условие равенства вторых производных.
    \end{cloze}
\end{note}

\begin{note}{188e7ec06df04d23872b628d4ae87856}
    В чём ключевая идея доказательства критерия ОДУ в полных дифференциалах (достаточность)?

    \begin{cloze}{1}
        Явно предоставить интеграл \({ u }\).
    \end{cloze}
\end{note}

\begin{note}{de914195b18342b8bf511d5f61cfb004}
    Как выражается интеграл ОДУ в полных дифференциалах
    \[
        m(x, y) dx + n(x, y) dy = 0\,?
    \]

    \begin{cloze}{1}
        \[
            \int_{x_0}^{x} m(\,\cdot\,, y ) + \int_{y_0}^{y} n(x_0, \,\cdot\,)\,.
        \]
        (или, аналогично, \({ y_0 }\) и \({ x }\))
    \end{cloze}
\end{note}

\begin{note}{e0c076b820de4d93b46b1b2098c87e2c}
    Какие значения могут принимать параметры \({ (x_0, y_0) }\) из общего вида интеграла ОДУ в полных дифференциалах?

    \begin{cloze}{1}
        Любая точка из области определения.
    \end{cloze}
\end{note}

\begin{note}{e55223eb41114fe092beeffc38509358}
    При каком условии можно дифференцировать по параметру под знаком определённого интеграла?

    \begin{cloze}{1}
        Функция и её производная по параметру непрерывны.
    \end{cloze}
\end{note}

\begin{note}{766a804de4ef42379f53963b92bd5472}
    Пусть \({ E \subseteq \mathbb R^2 }\),\: \({ f, f'_y \in C(E) }\).
    Тогда
    \[
        \begin{icloze}{2}\frac{\partial}{\partial y} \int_{a}^{b} f(x, y)\: dx\end{icloze} = \begin{icloze}{1}\int_{a}^{b} \frac{\partial f}{\partial y} (x, y)\: dx\,.\end{icloze}
    \]
\end{note}

\section{Семинар 10.11.22}
\begin{note}{53d901a2d9e34a6f8caeebc01a48fbeb}
    Как решаются ОДУ с разделяющимися переменными?

    \begin{cloze}{1}
        Собрать все с \({ x }\) и все с \({ y }\) с разных сторон и проинтегрировать.
    \end{cloze}
\end{note}

\begin{note}{6ec2b60aaf164b1ab14b191fd853fc08}
    Какое есть ``тонкое'' место при решении ОДУ с разделяющимися переменными?

    \begin{cloze}{1}
        Отдельно рассматривать случай равенства нулю при делении.
    \end{cloze}
\end{note}

\begin{note}{56d9aa3e00934312bd51c5b2873b2733}
    Пусть \({ f : \mathbb R^{n} \to \mathbb R }\).
    Функция \({ f }\) называется \begin{icloze}{2}однородной степени \({ q }\),\end{icloze} если
    \begin{icloze}{1}
        \[
            f(\lambda x) = \lambda^{q} f(x) \quad \forall x \in \mathbb R^{n}.
        \]
    \end{icloze}
\end{note}

\begin{note}{99f9700eabbf4ae6aa184065479b2aab}
    Какие ОДУ называются однородными?

    \begin{cloze}{1}
        Зависит от типа ОДУ.
    \end{cloze}
\end{note}

\begin{note}{1a1f83b72fc544d0bbc883e1d551a4ab}
    Какие ОДУ первого порядка в симметричной форме называются однородными?

    \begin{cloze}{1}
        Обе определяющие функции являются однородными одной степени.
    \end{cloze}
\end{note}

\begin{note}{f1642f7a316a46b7873c4f05c9aabc18}
    Какие ОДУ первого порядка, разрешённые относительно производной, называются однородными?

    \begin{cloze}{1}
        Определяющая функция является однородной нулевой степени.
    \end{cloze}
\end{note}

\begin{note}{575d0a91af124785806cfd53d6872fe0}
    Как можно показать, что ОДУ первого порядка, разрешённое относительно производной является однородным?

    \begin{cloze}{1}
        Представить правую часть как функцию от \({ \frac{y}{x} }\).
    \end{cloze}
\end{note}

\begin{note}{8211f8127d8b4dc38252a59d920e70b8}
    Как решаются однородные ОДУ?

    \begin{cloze}{1}
        \({ y = tx }\) и выразить \({ dy }\).
    \end{cloze}
\end{note}

\begin{note}{1a7488f29ba54697ac88b71b8158f763}
    Почему для однородных ОДУ работает подстановка \({ y = tx }\)?

    \begin{cloze}{1}
        Из однородности выносится и сокращается \({ x }\).
    \end{cloze}
\end{note}

\section{Семинар 24.11.22}
\begin{note}{ace60f48b6414173867b0ab70d5e367f}
    Как решаются ОДУ в симметричной форме с линейными коэффициентами?

    \begin{cloze}{1}
        Заменой переменных сводятся к однородным.
    \end{cloze}
\end{note}

\begin{note}{245fd749caee40409a18db2183e4ba4f}
    \begin{icloze}{2}
        Уравнения вида
        \[
            y' + a(x)y = b(x)
        \]
    \end{icloze}
    называются \begin{icloze}{1}линейными ОДУ первого порядка.\end{icloze}
\end{note}

\begin{note}{c436a0567cc849e38bfb6e9f64e5a7be}
    Как называется метод решения линейных ОДУ первого порядка?

    \begin{cloze}{1}
        Метод Бернулли.
    \end{cloze}
\end{note}

\begin{note}{23dafe9e59384eadab89ccf1265da08a}
    Какие ОДУ решаются методом Бернулли?

    \begin{cloze}{1}
        Линейные ОДУ первого порядка.
    \end{cloze}
\end{note}

\begin{note}{59ace38917f5481e968e4f918be1aacd}
     В чём основная идея метода Бернулли для решения линейных ОДУ первого порядка?

     \begin{cloze}{1}
         Представить функцию как произведение, и удачно подобрать один из множителей.
     \end{cloze}
\end{note}

\begin{note}{62d24394b1b342568f532607f56260fb}
    Как подбирается второй множитель при решении линейных ОДУ первого порядка методом Бернулли?

    \begin{cloze}{1}
        Подставить произведение в изначальное уравнение и ``занулить'' два слагаемых в левой части.
    \end{cloze}
\end{note}

\section{Семинар 01.12.22}
\begin{note}{820af7a707d14597ae03a86364f7f109}
    Для каких ОДУ первого порядка применим метод введения параметра?

    \begin{cloze}{1}
        Для которых можно выразить \({ y }\) через \({ x, y' }\).
    \end{cloze}
\end{note}

\begin{note}{0f79aa4b959d44ec81cd48e536c7548c}
    В чём состоит метод введения параметра для решения ОДУ первого порядка?

    \begin{cloze}{1}
        Ввести параметр \({ p = y' }\) и взять полный дифференциал \({ y }\) и его выражения (через \({ x, y' }\).)
    \end{cloze}
\end{note}

\section{Семинар 08.12.22}
\begin{note}{db6bdb752fda41a0adb8fa99386aacf2}
    Как понизить степень ОДУ первого порядка, не содержащего искомую функцию?

    \begin{cloze}{1}
        Взять низшую из производных за неизвестную.
    \end{cloze}
\end{note}

\begin{note}{259ae7c54d674879b528902f8acfdfd7}
    Как понизить степень ОДУ первого порядка, не содержащего независимую переменную?

    \begin{cloze}{1}
        Взять искомую функцию за независимое переменное, а производную --- за искомую функцию.
    \end{cloze}
\end{note}

\begin{note}{65478a36706440d488ab1a196c9f2a2f}
    Как понизить степень ОДУ первого порядка, однородного относительно \({ y }\) и его производных?

    \begin{cloze}{1}
        Постановкой \({ y' = yt }\).
    \end{cloze}
\end{note}

\section{Лекция 30.12.22}
\begin{note}{1bbbb42e2ec34ff190e8fe2fa258a219}
    Для каких ОДУ вводят понятие интегрирующего множителя?

    \begin{cloze}{1}
        ОДУ первого порядка в симметричной форме.
    \end{cloze}
\end{note}

\begin{note}{4be57b13e2ff4fc0812c48c168caf8d8}
    Что есть интегрирующий множитель ОДУ в симметричной форме
    \[
        m dx + n dy = 0\,?
    \]

    \begin{cloze}{1}
        Гладкая ненулевая функция, умножение на которую приводит к уравнению в полных дифференциалах.
    \end{cloze}
\end{note}

\begin{note}{607c3ff75892464ea473b71badf112c3}
    Как в общем случае находится интегрирующий множитель ОДУ в симметричной форме?

    \begin{cloze}{1}
        Общего алгоритма нет.
    \end{cloze}
\end{note}

\begin{note}{0ec5420a438e44be9c69d3e0a06c84c5}
    Какой частный случай рассматривается в теореме о существовании интегрирующего множителя?

    \begin{cloze}{1}
        Множитель --- функция от гладкой функции на области.
    \end{cloze}
\end{note}

\begin{note}{4b053d6afd1c429f90535aafc57a1beb}
    Каким типом условия является теорема о существовании интегрирующего множителя?

    \begin{cloze}{1}
        Критерий.
    \end{cloze}
\end{note}

\begin{note}{dccef2b32f784a0ba2b1c390f25df44e}
    Какое условие рассматривается в теореме о существовании интегрирующего множителя?

    \begin{cloze}{1}
        Определённое выражение зависит только от функции-аргумента множителя.
    \end{cloze}
\end{note}

\begin{note}{e92e49ddfb2a440897504bd8660493fb}
    Какое выражение рассматривается в теореме о существовании интегрирующего множителя для ОДУ в симметричной форме
    \[
        m dx + n dy = 0\,?
    \]

    \begin{cloze}{1}
        \[
            \frac{m \frac{\partial w}{\partial y} - n \frac{\partial w}{\partial x}}{\frac{\partial n}{\partial x} - \frac{\partial m}{\partial y}}\,,
        \]
        где \({ w }\) --- функция-аргумент.
    \end{cloze}
\end{note}

\begin{note}{b32b28f5868247118234555ca71d17ed}
    Пусть \({ w \in C^1 }\) и дано ОДУ в симметричной форме
    \[
        m dx + n dy = 0\,.
    \]
    Что можно сказать, если \({ \frac{m \frac{\partial w}{\partial y} - n \frac{\partial w}{\partial x}}{\frac{\partial n}{\partial x} - \frac{\partial m}{\partial y}} }\) зависит только от \({ w }\)?

    \begin{cloze}{1}
        Существует интегрирующий множитель вида \({ \mu \circ w }\).
    \end{cloze}
\end{note}

\begin{note}{c3227395f1404ab295f4401b9c8401f2}
    В чём основная идея доказательства теоремы о существовании интегрирующего множителя?

    \begin{cloze}{1}
        Критерий ОДУ в полных дифференциалах.
    \end{cloze}
\end{note}

\begin{note}{4e0f81875d544375856a688e78b0c9af}
    Какими должны быть функции-коэффициенты линейного ОДУ первого прядка?

    \begin{cloze}{1}
        Непрерывными.
    \end{cloze}
\end{note}

\section{Лекция 07.12.22}
\begin{note}{da7b414e99214fe49b3937bbc5dda688}
    В чём основная идея доказательства теоремы Пеано?

    \begin{cloze}{1}
        Построить последовательность приближённых решений для эквивалентного интегрального уравнения.
    \end{cloze}
\end{note}

\begin{note}{f915e26dcad547a99a1887f67b12cc3b}
    Какому интегральному уравнению эквивалентна задача Коши для ОДУ \({ y' = f(x, y) }\)?

    \begin{cloze}{1}
        \[
            y(x) = y_0 + \int_{x_0}^{x} f(x, y(\,\cdot\,))\,.
        \]
    \end{cloze}
\end{note}

\begin{note}{e97baee3a1da423dad9aeaab712b5c7d}
    Какой задаче эквивалентно интегральное уравнение
    \[
        y(x) = y_0 + \int_{x_0}^{x} f(x, y(\,\cdot\,))\,?
    \]

    \begin{cloze}{1}
        Задача Коши для \({ y' = f(x, y) }\).
    \end{cloze}
\end{note}

\section{Лекция 14.12.22}
\begin{note}{1dbe2f8bdd994378839dddcaf69ffebf}
    Что называется нормальной системой ОДУ?

    \begin{cloze}{1}
        Система ОДУ \({ \frac{dy_i}{dx} = f_i(x, y_1, \ldots, y_m) }\) для \({ i = 1, \ldots, m }\).
    \end{cloze}
\end{note}

\begin{note}{5acbc74ff5274cebae6e7b060adfd7fc}
    Какой должны быть функции \({ f_i }\) из определения нормальной системы ОДУ?

    \begin{cloze}{1}
        Любые функции на некоторой общей области.
    \end{cloze}
\end{note}

\begin{note}{62d1d346e7d84fa480c253bc7ffe8feb}
    Любое дифференциальное уравнение, разрешённое относительно старшей производной, сводится к \begin{icloze}{1}нормальной системе ОДУ специальной вида.\end{icloze}
\end{note}

\begin{note}{96ea219edca14258bd7a4fce45d39800}
    Какие ОДУ высших порядков гарантированно сводятся к нормальной системе ОДУ?

    \begin{cloze}{1}
        Разрешённые относительно старшей производной.
    \end{cloze}
\end{note}

\begin{note}{85cb7f9eac2b41d8ac0ff8551a26e0b6}
    К какой нормальной системе сводится ОДУ
    \[
        \frac{d^{m}y}{dx^{m}} = f(x, y, y', \ldots, y^{(m-1)})\,?
    \]

    \begin{cloze}{1}
        \[
            \begin{cases}
                y_0 = y, \\
                y_{i+1} = y'_i, \\
                y_m = f(\ldots).
            \end{cases}
        \]
    \end{cloze}
\end{note}

\begin{note}{0ad55ab69a854bcbbb1a172e5f3ea76c}
    Пусть \({ x \in \mathbb R^{n} }\).
    \begin{icloze}{2}
        Величина
        \[
            \max_i \left\lvert x_i \right\rvert
        \]
    \end{icloze}
    называется \begin{icloze}{1}максимум-нормой вектора \({ x }\).\end{icloze}
\end{note}

\begin{note}{5148d0bcd40f4a1caf2a3cb03b8d1fe2}
    Пусть \({ x \in \mathbb R^{n} }\).
    \begin{icloze}{2}Максимум-норма вектора \({ x }\)\end{icloze} обозначается \begin{icloze}{1}\({ \left\lvert x \right\rvert }\).\end{icloze}
\end{note}

\begin{note}{898c3fb2a0e949ec92dfe51da414f60f}
    Пусть \({ x \in \mathbb R^{n} }\).
    Каким соотношением связаны \({ \left\lVert x \right\rVert }\) и \({ \left\lvert x \right\rvert }\)?
    (Оценка снизу для \({ \left\lvert x \right\rvert }\).)

    \begin{cloze}{1}
        \[
            \left\lVert x \right\rVert \leqslant \sqrt{n} \left\lvert x \right\rvert\,.
        \]
    \end{cloze}
\end{note}

\begin{note}{7104b4d248e24eb68bd2b7ba38bad530}
    Пусть \({ x \in \mathbb R^{n} }\).
    Каким соотношением связаны \({ \left\lVert x \right\rVert }\) и \({ \left\lvert x \right\rvert }\)?
    (Оценка сверху для \({ \left\lvert x \right\rvert }\).)

    \begin{cloze}{1}
        \[
            \quad \left\lvert x \right\rvert \leqslant \left\lVert x \right\rVert\,.
        \]
    \end{cloze}
\end{note}

\begin{note}{489adc55335c42919fe3f46965f5a6ae}
    Пусть \({ A = (a_{ij}) \in \mathbb R^{n \times m} }\).
    \begin{icloze}{2}
        Величина
        \[
            \max_{i,j} \left\lvert a_{ij} \right\rvert
        \]
    \end{icloze}
    называется \begin{icloze}{1}максимум-нормой матрицы \({ A }\).\end{icloze}
\end{note}

\begin{note}{f810fa6058094303a4bd1bb3cedaaf8e}
    Пусть \({ A = (a_{ij}) \in \mathbb R^{n \times m} }\).
    \begin{icloze}{2}Максимум-норма матрицы \({ A }\)\end{icloze} обозначается \begin{icloze}{1}\({ \left\lVert A \right\rVert }\).\end{icloze}
\end{note}

\begin{note}{b352b66d7dd14b1a9b2ba2e66d888d5c}
    Пусть \({ A \in \mathbb R^{n \times m}, B \in \mathbb R^{m \times l} }\).
    Что можно сказать о \({ \left\lVert A \cdot B \right\rVert }\)?

    \begin{cloze}{1}
        \[
            \left\lVert A \cdot B \right\rVert \leqslant m \left\lVert A \right\rVert \left\lVert B \right\rVert\,.
        \]
    \end{cloze}
\end{note}

\begin{note}{335e8560b4c44a4c806181b559981011}
    Пусть \({ A \in \mathbb R^{n \times m}, x \in \mathbb R^{m} }\).
    Что можно сказать о \({ \left\lvert Ax \right\rvert }\)?

    \begin{cloze}{1}
        \[
            \left\lvert x \right\rvert \leqslant m \left\lVert A \right\rVert \cdot \left\lvert x \right\rvert\,.
        \]
    \end{cloze}
\end{note}

\begin{note}{b4cbd2c25e924ec78b6044c556b86bf3}
    Пусть \({ f : [a, b] \to \mathbb R^{n} }\).
    Вектор-функция \({ f }\) называется \begin{icloze}{2}интегрируемой,\end{icloze} если \begin{icloze}{1}все \({ f_i }\) интегрируемы.\end{icloze}
\end{note}

\begin{note}{04c57fb9b7804a999fdec80fd902491c}
    Пусть \({ f : [a, b] \to \mathbb R^{n} }\).
    Тогда
    \[
        \begin{icloze}{2}\int_{a}^{b} f\end{icloze} \overset{\text{def}}= \begin{icloze}{1}\begin{pmatrix}
            \int_{a}^{b} f_i
        \end{pmatrix} \in \mathbb R^{n}.\end{icloze}
    \]
\end{note}

\begin{note}{ee80a65e53c644b7a5ad9908f5e49528}
    Пусть \({ f : [a, b] \to \mathbb R^{n} }\) интегрируема и \begin{icloze}{2}\({ \left\lvert f \right\rvert \leqslant M }\).\end{icloze}
    Тогда
    \[
        \left\lvert \int_{a}^{b} f \right\rvert \leqslant \begin{icloze}{1}M(b - a).\end{icloze}
    \]
\end{note}

\begin{note}{b957e9e00eee4363903fc87643ce7fd5}
    Как в векторном виде записывается нормальная система ОДУ?

    \begin{cloze}{1}
        \[
            \frac{dy}{dx} = f(x, y)\,,
        \]
        где \({ y }\) --- вектор-функция, \({ f }\) --- отображение на области.
    \end{cloze}
\end{note}

\begin{note}{ee6d475bfc6d40e5af0793bfd09da0e8}
    Что фактически является решением нормальной системы ОДУ в векторном виде?

    \begin{cloze}{1}
        Вектор-функция на некотором интервале.
    \end{cloze}
\end{note}

\begin{note}{231b400e13e949e9a728a321099f9010}
    Как ставится задача Коши для нормальной системы ОДУ?

    \begin{cloze}{1}
        Так же как для ОДУ первого порядка, но с векторными начальными данными.
    \end{cloze}
\end{note}

\begin{note}{c7fa040588b644e0872aab7d789419ec}
    Что устанавливает условие Липшица (интуитивно)?

    \begin{cloze}{1}
        Отображение увеличивает расстояние между точками не более чем в \({ L }\) раз.
    \end{cloze}
\end{note}

\begin{note}{f828a753435f412b8a1544ceaffdf050}
    Как выглядит условие Липшица для \({ f : G \subseteq \mathbb R^{n} \to \mathbb R^{m} }\)?

    \begin{cloze}{1}
        \begin{gather*}
            \left\lVert f(x) - f(y) \right\rVert \leqslant L\left\lVert x - y \right\rVert \\
            \forall x, y \in G\,.
        \end{gather*}
    \end{cloze}
\end{note}

\begin{note}{ccc2b3d677b44cc2a1105918d53fa9db}
    Пусть \({ f : G \subseteq \mathbb R^{n} \to \mathbb R^{m} }\).
    Как называется условие
    \begin{gather*}
        \left\lVert f(x) - f(y) \right\rVert \leqslant L\left\lVert x - y \right\rVert \\
        \forall x, y \in G\,?
    \end{gather*}

    \begin{cloze}{1}
        Условие Липшица.
    \end{cloze}
\end{note}

\begin{note}{4b7cc86c9503491290ec6d101a4cb723}
    \begin{icloze}{2}Значение \({ L }\)\end{icloze} из определения \begin{icloze}{2}условия Липшица\end{icloze} называют \begin{icloze}{1}константой Липшица.\end{icloze}
\end{note}

\begin{note}{d28025b79af146e8a00c926de2bd80b8}
    Множество \begin{icloze}{2}всех липшицевых отображений\end{icloze} на множестве \({ H }\) обозначается \begin{icloze}{1}\({ \operatorname{Lip}(H) }\).\end{icloze}
\end{note}

\begin{note}{f57d54d7b53747ae9e26af0c12ea83f0}
    Пусть \({ f : G \subseteq \mathbb R^{n} \to \mathbb R^{m} }\).
    Говорят, что \begin{icloze}{2}\({ f }\) удовлетворяет условию Липшица по набору переменных,\end{icloze} если
    \begin{icloze}{1}
        \({ f }\) является липшицевой при любом фиксированном значении остальных переменных.
    \end{icloze}
\end{note}

\begin{note}{8f3c6f4b5a514cf5a5bd4fc268c0ca17}
    Множество \begin{icloze}{2}всех функций, удовлетворяющих условию Липшица по переменной \({ x }\),\end{icloze} на множестве \({ H }\) обозначается \begin{icloze}{1}\({ \operatorname{Lip}_x (H) }\).\end{icloze}
\end{note}

\begin{note}{8dd4d146c7f6410d81e59d0d9cf70e79}
    Говорят, что \({ f : G \subseteq \mathbb R^{n} \to \mathbb R^{m} }\) \begin{icloze}{2}удовлетворяет условию Липшица локально в точке \({ x_0 \in G }\),\end{icloze} если
    \begin{icloze}{1}
        внутри \({ G }\) найдётся окрестность \({ x_0 }\), на которой \({ f }\) является липшицевой.
    \end{icloze}
\end{note}

\begin{note}{915f61d4e6dd41e3ae1cc164fdeb321b}
    Говорят, что \({ f : G \subseteq \mathbb R^{n} \to \mathbb R^{m} }\) \begin{icloze}{2}удовлетворяет условию Липшица локально на \({ G }\),\end{icloze} если
    \begin{icloze}{1}
        \({ f }\) локально липшицева в любой точке \({ G }\).
    \end{icloze}
\end{note}

\begin{note}{51c47fa7fd634814985cc777d605fbff}
    Множество \begin{icloze}{2}всех локально липшицевых отображений\end{icloze} на множестве \({ H }\) обозначается \begin{icloze}{1}\({ \operatorname{Lip}_{loc}(H) }\).\end{icloze}
\end{note}

\begin{note}{021e656ff17a413980caea28d688c37f}
    Множество \begin{icloze}{2}всех локально липшицевых по переменной \({ x }\) отображений\end{icloze} на множестве \({ H }\) обозначается \begin{icloze}{1}\({ \operatorname{Lip}_{x, loc}(H) }\).\end{icloze}
\end{note}

\begin{note}{4b7a29c8468344d09b177c34482c7a14}
    Пусть \({ f : G \subseteq \mathbb R^{n} \to \mathbb R^{m} }\).
    В чём состоит достаточное условие локальной липшицевости \({ f }\) на \({ G }\)?

    \begin{cloze}{1}
        \({ f \in C^{1}(G) }\).
    \end{cloze}
\end{note}

\begin{note}{9ce16ab1d1de4fee86933cb29ce5d26e}
    Пусть \({ f : G \subseteq \mathbb R^{n} \to \mathbb R^{m} }\).
    В чём состоит достаточное условие локальной липшицевости \({ f }\) по переменным \({ x_{i_1}, \ldots, x_{i_k} }\) на \({ G }\)?

    \begin{cloze}{1}
        Частная производная по любой из этих переменных непрерывна на \({ G }\).
    \end{cloze}
\end{note}

\begin{note}{8ed16a99d0bc4e2a94bec19c9886a11b}
    В чём основная идея доказательства достаточного условия локальной липшицевости для отображений
    \[
        f : G \subseteq \mathbb R^{n} \to \mathbb R^{m}\,?
    \]

    \begin{cloze}{1}
        Теорема Вейерштрасса для частных производных и формула конечных приращений.
    \end{cloze}
\end{note}

\begin{note}{272334ddc9d341549659ef7e3d32a75f}
    Пусть \({ f : G \subseteq \mathbb R^{n} \to \mathbb R^{m} }\).
    В чём состоит достаточное условие липшицевости \({ f }\) на \({ G }\)?

    \begin{cloze}{1}
        \({ G }\) выпукло и все частные производные ограничены.
    \end{cloze}
\end{note}

\begin{note}{935bbf65c0ef43c8a2615048d6af6b1d}
    Пусть \({ f : G \subseteq \mathbb R^{n} \to \mathbb R^{m} }\).
    В чём состоит достаточное условие липшицевости \({ f }\) по переменным \({ x_{i_1}, \ldots, x_{i_k} }\) на \({ G }\)?

    \begin{cloze}{1}
        \({ G }\) выпукло и частные производные по этим переменным ограничены.
    \end{cloze}
\end{note}

\begin{note}{35012253e1c4488eaa030f235a65f273}
    В чём основная идея доказательства достаточного условия липшицевости для отображений
    \[
        f : G \subseteq \mathbb R^{n} \to \mathbb R^{m}\,?
    \]

    \begin{cloze}{1}
        Формула конченых приращений.
    \end{cloze}
\end{note}

\end{document}
