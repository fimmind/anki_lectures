%! TeX root = ./main.tex
\documentclass[11pt, a5paper]{article}
\usepackage[width=10cm, top=0.5cm, bottom=2cm]{geometry}

\usepackage[T1,T2A]{fontenc}
\usepackage[utf8]{inputenc}
\usepackage[english,russian]{babel}
\usepackage{libertine}

\usepackage{amsmath}
\usepackage{amssymb}
\usepackage{amsthm}
\usepackage{mathrsfs}
\usepackage{framed}
\usepackage{xcolor}

\setlength{\parindent}{0pt}

% Force \pagebreak for every section
\let\oldsection\section
\renewcommand\section{\pagebreak\oldsection}

\renewcommand{\thesection}{}
\renewcommand{\thesubsection}{Note \arabic{subsection}}
\renewcommand{\thesubsubsection}{}
\renewcommand{\theparagraph}{}

\newenvironment{note}[1]{\goodbreak\par\subsection{\hfill \color{lightgray}\tiny #1}}{}
\newenvironment{cloze}[2][\ldots]{\begin{leftbar}}{\end{leftbar}}
\newenvironment{icloze}[2][\ldots]{%
  \text{\tiny \color{lightgray}\{\{c#2::}\hspace{0pt}\ignorespaces%
}{%
  \unskip\hspace{0pt}\text{\tiny\color{lightgray}\}\}}%
}


\begin{document}
\section{Лекция 09.11.22}
\begin{note}{e60e9a5778ca41aaabad4a4a7e6c4fea}
    Какие есть основные виды дифференциальных уравнений?

    \begin{cloze}{1}
        Обыкновенные; в частных производных.
    \end{cloze}
\end{note}

\begin{note}{5e9fddb7111e4705aa4a145bb98b111f}
    \begin{icloze}{2}Обыкновенные дифференциальные уравнения\end{icloze} --- это \begin{icloze}{1}уравнения относительно функции одной переменной и её производных.\end{icloze}
\end{note}

\begin{note}{304a53cb8187428aaba248e942576ea2}
    ``\begin{icloze}{2}Обыкновенные дифференциальные уравнения\end{icloze}'' сокращается как ``\begin{icloze}{1}ОДУ.\end{icloze}''
\end{note}

\begin{note}{27aba707c8db4ff0aaa02aa522cb353d}
    \begin{icloze}{2}Уравнения в частных производных\end{icloze} --- это \begin{icloze}{1}уравнения относительно функции нескольких переменных и её частных производных.\end{icloze}
\end{note}

\begin{note}{54188b1d277440558390f93807cf9e7e}
    \begin{icloze}{2}Уравнения в частных производных\end{icloze} в русскоязычной среде так же называют \begin{icloze}{1}уравнениями математической физики.\end{icloze}
\end{note}

\begin{note}{6ed94a06c0164651bcacf8ee9c9f96fb}
    ``\begin{icloze}{2}Уравнения в частных производных\end{icloze}'' сокращается как ``\begin{icloze}{1}УрЧП.\end{icloze}''
\end{note}

\begin{note}{0fd7b116352242aba166bb75e3487f7e}
    \begin{icloze}{2}Порядком\end{icloze} дифференциального уравнения называется \begin{icloze}{1}порядок старшей производной, в него входящей.\end{icloze}
\end{note}

\begin{note}{69a06ae8b8d1413f84c38cc0ab521158}
    Является ли
    \[
        F(x, y) = 0,\: y = y(x)
    \]
    дифференциальным уравнением?

    \begin{cloze}{1}
        Нет, потому что нет производных.
    \end{cloze}
\end{note}

\begin{note}{f8ab33c8a60a4901a4c26ccff8a6fd1a}
    Множество \({ G \subset \mathbb R^n }\) называется \begin{icloze}{2}областью,\end{icloze} если \begin{icloze}{1}оно открыто и связно.\end{icloze}
\end{note}

\begin{note}{1425377052ae4b228fc834d5b4f63182}
    ОДУ первого порядка называется \begin{icloze}{2}разрешённым относительно производной,\end{icloze} если оно имеет вид
    \begin{icloze}{1}
        \[
            \frac{dy}{dx} = f(x, y),
        \]
    \end{icloze}
    где \({ f }\) --- \begin{icloze}{2}функция на области в \({ \mathbb R^2 }\).\end{icloze}
\end{note}

\begin{note}{aa5c740235f848c79fb3bbc39d4a3160}
    Функция \({ y }\) называется \begin{icloze}{2}решением ОДУ на множестве \({ X }\),\end{icloze} если \begin{icloze}{1}в любой точке \({ X }\) её подстановка её значений в ОДУ имеет смысл и приводит к верному равенству.\end{icloze}
\end{note}

\begin{note}{8515eff1b5844e0cba265de7445cf1a0}
    Пусть \({ y }\) --- \begin{icloze}{3}решение ОДУ.\end{icloze}
    \begin{icloze}{1}График \({ y }\)\end{icloze} называется \begin{icloze}{2}интегральной кривой этого уравнения.\end{icloze}
\end{note}

\begin{note}{01533944715c4477a02b8f21087f96d2}
    Сколько решений может иметь произвольное ОДУ?

    \begin{cloze}{1}
        Сколь угодно много.
    \end{cloze}
\end{note}

\begin{note}{76c92fefc3414e8da3c06f458e9e80ee}
    В чём состоит задача Коши для ОДУ первого порядка?

    \begin{cloze}{1}
        Найти решение, отвечающее начальным условиям.
    \end{cloze}
\end{note}

\begin{note}{94e8aba670fa45eab4fae8fee8d23568}
    Что есть ``начальные условия'' из формулировки задачи Коши для ОДУ первого порядка?

    \begin{cloze}{1}
        \({ y(x_0) = y_0 }\) для фиксированных \({ x_0, y_0 }\).
    \end{cloze}
\end{note}

\begin{note}{0277eb8d5c00466ca6bb797bd58c8279}
    Как называются значения \({ (x_0, y_0) }\) в задаче Коши для ОДУ первого порядка?

    \begin{cloze}{1}
        Начальные данные.
    \end{cloze}
\end{note}

\begin{note}{aae5b8ef39d14aab9d038ebe894b7b99}
    Какие значения могут принимать начальные данные в задаче Коши для ОДУ первого порядка?

    \begin{cloze}{1}
        Любые, для которых ОДУ имеет смысл.
    \end{cloze}
\end{note}

\begin{note}{4ecfb902661d484682379f7f0b7b2567}
    На каком множестве нужно найти решение задачи Коши с начальными данными \({ (x_0, y_0) }\)?

    \begin{cloze}{1}
        Интервал, включающий \({ x_0 }\).
    \end{cloze}
\end{note}

\begin{note}{c6f2ec8c4ae142118b3840ddb828a96b}
    Как называется теорема о существовании решения задачи Коши для ОДУ первого порядка, разрешённого относительно производной?

    \begin{cloze}{1}
        Теорема Пеано.
    \end{cloze}
\end{note}

\begin{note}{3623dc3931814e958fe21c89ae0b0c6e}
    Какое уравнение рассматривается в теореме Пеано?

    \begin{cloze}{1}
        ОДУ первого порядка, разрешённое относительно производной.
    \end{cloze}
\end{note}

\begin{note}{5abd317c6b0b4b02ab50a72f0ea7d792}
    При каком условии можно что-либо заключить из теоремы Пеано?

    \begin{cloze}{1}
        Функция, задающая разрешённое ОДУ, непрерывна на области.
    \end{cloze}
\end{note}

\begin{note}{dfdb355bf4d84b1d89e76ebd9e215e4a}
    Что можно заключить из теоремы Пеано?

    \begin{cloze}{1}
        Для любой точки существует решение задачи Коши с этими начальными данными.
    \end{cloze}
\end{note}

\begin{note}{3290d174cc33491e924cbd8ec7137a8a}
    Каков геометрический смысл теоремы Пеано?

    \begin{cloze}{1}
        Через любую точку области проходит интегральная кривая.
    \end{cloze}
\end{note}

\begin{note}{59c04413608e4901a682dfb9a1d29161}
    Что называют точкой единственности для уравнения
    \[
        \frac{dy}{dx} = f(x, y)\,?
    \]

    \begin{cloze}{1}
        Точка, для которой любые два решения задачи Коши совпадают (в какой-то окрестности).
    \end{cloze}
\end{note}

\begin{note}{7dbfe4396c5545578771b6066869de5a}
    В каком именно смысле совпадают любые два решения соответствующей задачи Коши в определении точки единственности уравнения \({ \frac{dy}{dx} = f(x, y) }\)?

    \begin{cloze}{1}
        Они равны на некоторой \({ V_{\delta}(x_0) }\).
    \end{cloze}
\end{note}

\begin{note}{c197734ceb654182bc3221591296e2f6}
    Пусть \({ f }\) --- функция на области в \({ \mathbb R^2 }\),
    \[
        \frac{dy}{dx} = f(x, y)\,.
    \]
    Тогда если \({ f }\) и \({ \frac{\partial f}{\partial y} }\) непрерывны, то \begin{icloze}{1}любая точка области является точкой единственности.\end{icloze}
\end{note}

\begin{note}{33930ae47c60424a9e1c3f6f5482f236}
    Пусть \({ f }\) --- функция на области в \({ \mathbb R^2 }\).
    При каком условии задача Коши для \({ \frac{dy}{dx} = f(x, y) }\) однозначно разрешима в любой точке?

    \begin{cloze}{1}
        \({ f }\) и \({ \frac{\partial f}{\partial y} }\) непрерывны.
    \end{cloze}
\end{note}

\begin{note}{785b85c103cb446090843629aba6f668}
    Пусть \({ f }\) --- функция на области в \({ \mathbb R^2 }\).
    Что называют особым решением уравнения \({ \frac{dy}{dx} = f(x, y) }\)?

    \begin{cloze}{1}
        Решение, любая точка графика которого не является точкой единственности (внутри интервала).
    \end{cloze}
\end{note}

\begin{note}{37a7853a3e8d43aa9d89f5bb427f5ebc}
    Пусть \({ f }\) --- функция на области в \({ \mathbb R^2 }\).
    Как называется решение уравнения \({ \frac{dy}{dx} = f(x, y) }\), любая точка графика которого не является точкой единственности?

    \begin{cloze}{1}
        Особое решение.
    \end{cloze}
\end{note}

\begin{note}{8c95e3912938472e90263540e560fb33}
    Пусть \({ f }\) --- функция на области в \({ \mathbb R^2 }\).
    Что называют общим решением уравнения \({ \frac{dy}{dx} = f(x, y) }\)?

    \begin{cloze}{1}
        Параметризованная совокупность решений, содержащая решение задачи Коши для любой точки области.
    \end{cloze}
\end{note}

\begin{note}{94b869be37d9483ea16e55b821468d6a}
    Пусть \({ f }\) --- функция на области в \({ \mathbb R^2 }\).
    Как задаётся общее решение уравнения \({ \frac{dy}{dx} = f(x, y) }\)?

    \begin{cloze}{1}
        Отображение \({ \Phi(x, c) }\), где \({ c }\) --- параметр, \({ x }\) --- переменная.
    \end{cloze}
\end{note}

\begin{note}{845626784e214307bef975cf77f33ece}
    Пусть \({ f }\) --- функция на области в \({ \mathbb R^2 }\).
    Что называют частным решением уравнения \({ \frac{dy}{dx} = f(x, y) }\)?

    \begin{cloze}{1}
        Одно из решений, входящих в некоторое общее решение.
    \end{cloze}
\end{note}

\begin{note}{e4f00b810c5346e7883e4848222bac62}
    \begin{icloze}{2}Векторное поле\end{icloze} --- это \begin{icloze}{1}отображение из линейного пространства в себя.\end{icloze}
\end{note}

\begin{note}{c910d42c89d94c0e911e1b327e5b0a36}
    Для каких ОДУ имеет смысл понятие поля направлений?

    \begin{cloze}{1}
        ОДУ первого порядка, разрешённое относительно производной.
    \end{cloze}
\end{note}

\begin{note}{bf5a5b553b8b4f74905e400cce49df6c}
    Пусть \({ f }\) --- функция на области в \({ \mathbb R^2 }\).
    Что называют полем направлений уравнения \({ \frac{dy}{dx} = f(x, y) }\)?

    \begin{cloze}{1}
        Векторное поле нормализованных векторов, задающих направления касательных к интегральным кривым.
    \end{cloze}
\end{note}

\begin{note}{7f47ec07174947169cc1a095d3b5dbd0}
    Пусть \({ f }\) --- функция на области в \({ \mathbb R^2 }\).
    Как строится визуальное представление поля направлений уравнения \({ \frac{dy}{dx} = f(x, y) }\)?

    \begin{cloze}{1}
        Через каждую точки сетки проводится соответствующе наклонённый отрезок.
    \end{cloze}
\end{note}

\begin{note}{a9509cb3d319496ca4f8ce25b5b988b2}
    Пусть \({ f }\) --- функция на области в \({ \mathbb R^2 }\).
    Гладкая кривая является \begin{icloze}{2}интегральной кривой\end{icloze} уравнения \({ \frac{dy}{dx} = f(x, y) }\) \begin{icloze}{3}тогда и только тогда, когда\end{icloze}
    \begin{icloze}{1}в любой точке она касается соответствующего элемента поля направлений.\end{icloze}

    \begin{center}
        \tiny
        (в терминах поля направлений)
    \end{center}
\end{note}

\begin{note}{1895f43f66e747cc9ce6e8a4dd317258}
    Пусть \({ f }\) --- функция на области в \({ \mathbb R^2 }\).
    Что называется изоклиной уравнения \({ \frac{dy}{dx} = f(x, y) }\)?

    \begin{cloze}{1}
        Кривая, во всех точках которой значение поля направлений одинаково.
    \end{cloze}
\end{note}

\begin{note}{efd8be00720a4e449757b00e4434d684}
    Пусть \({ f }\) --- функция на области в \({ \mathbb R^2 }\).
    Каким уравнением задаётся произвольная изоклина уравнения \({ \frac{dy}{dx} = f(x, y) }\)?

    \begin{cloze}{1}
        \({ f(x, y) = c }\) для \({ c \in \mathbb R }\).
    \end{cloze}
\end{note}

\section{Лекция 16.11.22}
\begin{note}{84332aa7764648b3b6b73355b0fb7064}
    Пусть \begin{icloze}{4}\({ G }\) --- область в \({ \mathbb R^2 }\).\end{icloze}
    Тогда выражение вида
    \[
        \begin{icloze}{2}m(x, y) \cdot dx + n(x, y) \cdot dy = 0\,,\end{icloze} \quad \text{где}\ \begin{icloze}{3}m, n : G \to \mathbb R\,,\end{icloze}
    \]
    называется \begin{icloze}{1}ОДУ первого порядка в симметричной форме.\end{icloze}
\end{note}

\begin{note}{0a8383be5d30458db9ab476e9ea34e84}
    Что называется решением ОДУ первого порядка в симметричной форме?

    \begin{cloze}{1}
        Решение любого из порождённых ОДУ, разрешённых относительно производной.
    \end{cloze}
\end{note}

\begin{note}{89d64566d77b4dc7b52af28eab7dae35}
    Какие два уравнения порождает ОДУ первого порядка в симметричной форме?

    \begin{cloze}{1}
        \[
            \frac{dy}{dx} = \cdots \quad \text{и} \quad \frac{dx}{dy} = \cdots
        \]
    \end{cloze}
\end{note}

\begin{note}{0981a7837c7644328e3f6ccbd939ec02}
    Всегда ли ОДУ первого порядка в симметричной форме порождает два уравнения?

    \begin{cloze}{1}
        Нет.
    \end{cloze}
\end{note}

\begin{note}{0bf3be66da26454f981de93a9f57fdb4}
    Пусть дано ОДУ в симметричной форме
    \[
        m(x, y) dx + n(x, y) dy = 0\,.
    \]
    \begin{icloze}{2}Точку, в которой и \({ m }\), и \({ n }\) обращаются в \({ 0 }\),\end{icloze} называют \begin{icloze}{1}особой точкой этого уравнения.\end{icloze}
\end{note}

\begin{note}{4fb058ed5261465fbcf62958e5b93441}
    Как ставится задача Коши для ОДУ первого порядка в симметричной форме?

    \begin{cloze}{1}
        Найти решение для любой из порождённых задач Коши для ОДУ, разрешённых относительно производной.
    \end{cloze}
\end{note}

\begin{note}{0d205d054d3843c19a806774476762c0}
    Пусть дано ОДУ в симметричной форме
    \[
        m(x, y) dx + n(x, y) dy = 0\,.
    \]
    Что называется неособой точкой этого уравнения?

    \begin{cloze}{1}
        Точка, в которой либо \({ m }\), либо \({ n }\) не обращается в \({ 0 }\).
    \end{cloze}
\end{note}

\begin{note}{61ddfc362e5649ab9020c85dd47f4454}
    При каком условии ОДУ в симметричной форме гарантированно имеет решение задачи Коши?

    \begin{cloze}{1}
        Если точка не является особой и функции при \({ dx }\) и \({ dy }\) непрерывны.
    \end{cloze}
\end{note}

\begin{note}{0ad4cca0576a409cbb3871385860853c}
    Что можно сказать про ОДУ в симметричной форме
    \[
        m(x, y) dx + n(x, y) dy = 0\,,
    \]
    если \({ m, n }\) непрерывны?

    \begin{cloze}{1}
        Оно имеет решение задачи Коши для любой неособой точки.
    \end{cloze}
\end{note}

\begin{note}{3d3a825102e94d268fceedc112aee02b}
    В чём основная идея доказательства теоремы о достаточного условия существования решения задачи Коши для ОДУ первого порядка в симметричной форме?

    \begin{cloze}{1}
        Теорема Пеано для порождённого ОДУ, разрешённого относительно производной.
    \end{cloze}
\end{note}

\begin{note}{84670525aec14b12ab1c60bc7c6cb90f}
    Какое есть ``тонкое'' место в доказательстве достаточного условия существования решения задачи Коши для ОДУ первого порядка в симметричной форме?

    \begin{cloze}{1}
        Применить теорему о стабилизации непрерывной функции для неравенства нулю в какой-то окрестности.
    \end{cloze}
\end{note}

\begin{note}{e22e81451b75462bb23f42da429d0822}
    Какое есть дополнение к достаточному условию существования решения задачи Коши для ОДУ первого порядка в симметричной форме?

    \begin{cloze}{1}
        Если обе определяющие функции не обращаются в ноль, то решение ``двойственно.''
    \end{cloze}
\end{note}

\begin{note}{c5937908ffc3421b88d12e29cd18618a}
    В каком смысле двойственно решение задачи Коши в дополнении к достаточному условию существования решения задачи Коши для ОДУ первого порядка в симметричной форме?

    \begin{cloze}{1}
        Обратная функция существует (на интервале) и является решением второй из порождённых задач Коши.
    \end{cloze}
\end{note}

\begin{note}{c77c237173dd4909bef0d7fbb475945a}
    Пусть для \({ m, n \in C(G) }\) дано ОДУ в симметричной форме
    \[
        m(x, y) dx + n(x, y) dy = 0\,,
    \]
    и \({ y = \varphi(x) }\) --- решение задачи Коши в т. \({ (x_0, y_0) }\).
    Почему \({ \varphi }\) обратима (на интервале)?

    \begin{cloze}{1}
        \({ \varphi' \neq 0  }\) и следствие из теоремы Дарбу о строгой монотонности.
    \end{cloze}
\end{note}

\begin{note}{1408fcbdb21643b29a526f5c9c698052}
    Что называется точкой единственности для ОДУ первого порядка в симметричной форме?

    \begin{cloze}{1}
        Точка, являющаяся точкой единственности для обеих из порождённых задач Коши.
    \end{cloze}
\end{note}

\begin{note}{f037d477228b4d1a983be67a181f7fc6}
    Каково условие для единственности решения задачи Коши для ОДУ первого порядка в симметричной форме?

    \begin{cloze}{1}
        Функции при \({ dx }\) и \({ dy }\) являются гладкими и не равны нулю в точке.
    \end{cloze}
\end{note}

\begin{note}{b4d211a3083b4bfca552c183c7cd0ce8}
    Почему в теореме о единственности решения задачи Коши для ОДУ первого порядка в симметричной форме мы требуем неравенство нулю функций при \({ dx }\) и \({ dy }\)?

    \begin{cloze}{1}
        Иначе какая-то из задач Коши не имеет смысла.
    \end{cloze}
\end{note}

\begin{note}{91cb20330134439695809b01acd7909f}
    В каком смысле единственно решение задачи Коши для ОДУ первого порядка в симметричной форме в теореме о единственность?

    \begin{cloze}{1}
        Точка является точкой единственности.
    \end{cloze}
\end{note}

\begin{note}{c34ebe00830e4594b9f2df519066c9f8}
    В чём ключевая идея доказательства теоремы о единственности решения задачи Коши для ОДУ первого порядка в симметричной форме?

    \begin{cloze}{1}
        Теорема о единственности для ОДУ, разрешённого относительно производной.
    \end{cloze}
\end{note}

\begin{note}{83eb2e7882ac49fd950bc3c5d734f88c}
    Если \({ (x_0, y_0) }\) является точкой единственности для ОДУ первого порядка, то через неё проходит \begin{icloze}{1}единственная интегральная кривая.\end{icloze}
\end{note}

\begin{note}{931e9e5942144396af4bc98f8d917c45}
    В каком смысле единственна интегральная кривая, проходящая через точку единственности для ОДУ первого порядка?

    \begin{cloze}{1}
        Любые две интегральные кривые совпадают на какой-то окрестности.
    \end{cloze}
\end{note}

\begin{note}{67befaeaa43b4d1daf13c54d0f3c9a18}
    Чем в первую очередь является интеграл ОДУ первого порядка в симметричной форме?

    \begin{cloze}{1}
        Гладкая функция на всей области.
    \end{cloze}
\end{note}

\begin{note}{6cd34528c11649468f156e7160361bc4}
    Какому соотношению по определению должен удовлетворять интеграл \({ u }\) ОДУ в симметричной форме
    \[
        m(x, y) dx + n(x, y) dy = 0\,?
    \]

    \begin{cloze}{1}
        \[
            \begin{vmatrix}
                \frac{\partial u}{\partial x} & \frac{\partial u}{\partial y} \\
                m & n
            \end{vmatrix}
            \equiv 0.
        \]
    \end{cloze}
\end{note}

\begin{note}{4e43bac3e13f49e890052b539cb99a0a}
    Интеграл \({ u }\) ОДУ в симметричной форме
    \[
        m(x, y) dx + n(x, y) dy = 0
    \]
    называется \begin{icloze}{2}гладким,\end{icloze} если \begin{icloze}{1}\({ \nabla u \neq 0 }\) на всей области.\end{icloze}
\end{note}

\begin{note}{d57e85cf8770406ca4be0153888980ff}
    Каково основное свойство гладкого интеграла \({ u }\) ОДУ первого порядка в симметричной форме?

    \begin{cloze}{1}
        Решение уравнения \({ u(x, y) = u(x_0, y_0) }\) есть решение задачи Коши в \({ (x_0, y_0) }\).
    \end{cloze}
\end{note}

\begin{note}{750fae6a0be8412885d8d65ccd8b532f}
    Пусть \({ u }\) --- гладкий интеграл ОДУ первого порядка в симметричной форме.
    Для каких точек решение уравнения
    \[
        u(x, y) = u(x_0, y_0)
    \]
    есть решение задачи Коши в \({ (x_0, y_0) }\)?

    \begin{cloze}{1}
        Для неособых.
    \end{cloze}
\end{note}

\begin{note}{3ec100db172a42a4821a9c2a59df0cdc}
    В контексте ОДУ первого порядка в симметричной форме, для какого объекта \({ u }\) решение уравнения
    \[
        u(x, y) = u(x_0, y_0)
    \]
    гарантированно является решением задачи Коши в неособой точке \({ (x_0, y_0) }\)?

    \begin{cloze}{1}
        Гладкий интеграл.
    \end{cloze}
\end{note}

\begin{note}{2d035fe05f234a678fb45f0a77098ee9}
    В чём ключевая идея доказательства теоремы о представлении решения задачи Коши для ОДУ первого порядка в симметричной форме через его интеграл?

    \begin{cloze}{1}
        Теорема о неявной функции для \({ u(x, y) - u(x_0, y_0) = 0 }\).
    \end{cloze}
\end{note}

\begin{note}{8ee64b4f58d64f3e9c544c3728edc66c}
    Что называется общим интегралом ОДУ первого порядка в симметричной форме?

    \begin{cloze}{1}
        Выражение \({ u(x, y) = c }\), где \({ u }\) --- интеграл уравнения.
    \end{cloze}
\end{note}

\begin{note}{4b9745c311764d9aa57c35f678ca0b4b}
    Пусть \({ u }\) --- интеграл ОДУ первого порядка в симметричной форме.
    Как называется выражение
    \[
        u(x, y) = c\,?
    \]

    \begin{cloze}{1}
        Общий интеграл.
    \end{cloze}
\end{note}

\begin{note}{d24cae635b5149fca8646f3627b9bf90}
    Пусть \({ u }\) --- интеграл ОДУ первого порядка в симметричной форме.
    Что можно сказать о произвольном решении \({ y(x) }\) этого ОДУ?

    \begin{cloze}{1}
        \({ u(x, y(x)) = const }\) на всём интервале.
    \end{cloze}
\end{note}

\end{document}
