%! TeX root = ./main.tex
\documentclass[11pt, a5paper]{article}
\usepackage[width=10cm, top=0.5cm, bottom=2cm]{geometry}

\usepackage[T1,T2A]{fontenc}
\usepackage[utf8]{inputenc}
\usepackage[english,russian]{babel}
\usepackage{libertine}

\usepackage{amsmath}
\usepackage{amssymb}
\usepackage{amsthm}
\usepackage{mathrsfs}
\usepackage{framed}
\usepackage{xcolor}

\setlength{\parindent}{0pt}

% Force \pagebreak for every section
\let\oldsection\section
\renewcommand\section{\pagebreak\oldsection}

\renewcommand{\thesection}{}
\renewcommand{\thesubsection}{Note \arabic{subsection}}
\renewcommand{\thesubsubsection}{}
\renewcommand{\theparagraph}{}

\newenvironment{note}[1]{\goodbreak\par\subsection{\hfill \color{lightgray}\tiny #1}}{}
\newenvironment{cloze}[2][\ldots]{\begin{leftbar}}{\end{leftbar}}
\newenvironment{icloze}[2][\ldots]{%
  \text{\tiny \color{lightgray}\{\{c#2::}\hspace{0pt}\ignorespaces%
}{%
  \unskip\hspace{0pt}\text{\tiny\color{lightgray}\}\}}%
}


\begin{document}
\section{Лекция 09.11.22}
\begin{note}{e60e9a5778ca41aaabad4a4a7e6c4fea}
    Какие есть основные виды дифференциальных уравнений?

    \begin{cloze}{1}
        Обыкновенные; в частных производных.
    \end{cloze}
\end{note}

\begin{note}{5e9fddb7111e4705aa4a145bb98b111f}
    \begin{icloze}{2}Обыкновенные дифференциальные уравнения\end{icloze} --- это \begin{icloze}{1}уравнения относительно функции одной переменной и её производных.\end{icloze}
\end{note}

\begin{note}{304a53cb8187428aaba248e942576ea2}
    ``\begin{icloze}{2}Обыкновенные дифференциальные уравнения\end{icloze}'' сокращается как ``\begin{icloze}{1}ОДУ.\end{icloze}''
\end{note}

\begin{note}{27aba707c8db4ff0aaa02aa522cb353d}
    \begin{icloze}{2}Уравнения в частных производных\end{icloze} --- это \begin{icloze}{1}уравнения относительно функции нескольких переменных и её частных производных.\end{icloze}
\end{note}

\begin{note}{54188b1d277440558390f93807cf9e7e}
    \begin{icloze}{2}Уравнения в частных производных\end{icloze} в русскоязычной среде так же называют \begin{icloze}{1}уравнениями математической физики.\end{icloze}
\end{note}

\begin{note}{6ed94a06c0164651bcacf8ee9c9f96fb}
    ``\begin{icloze}{2}Уравнения в частных производных\end{icloze}'' сокращается как ``\begin{icloze}{1}УрЧП.\end{icloze}''
\end{note}

\begin{note}{0fd7b116352242aba166bb75e3487f7e}
    \begin{icloze}{2}Порядком\end{icloze} дифференциального уравнения называется \begin{icloze}{1}порядок старшей производной, в него входящей.\end{icloze}
\end{note}

\begin{note}{69a06ae8b8d1413f84c38cc0ab521158}
    Является ли
    \[
        F(x, y) = 0,\: y = y(x)
    \]
    дифференциальным уравнением?

    \begin{cloze}{1}
        Нет, потому что нет производных.
    \end{cloze}
\end{note}

\begin{note}{f8ab33c8a60a4901a4c26ccff8a6fd1a}
    Множество \({ G \subset \mathbb R^n }\) называется \begin{icloze}{2}областью,\end{icloze} если \begin{icloze}{1}оно открыто и связно.\end{icloze}
\end{note}

\begin{note}{1425377052ae4b228fc834d5b4f63182}
    ОДУ первого порядка называется \begin{icloze}{2}разрешённым относительно производной,\end{icloze} если оно имеет вид
    \begin{icloze}{1}
        \[
            \frac{dy}{dx} = f(x, y),
        \]
    \end{icloze}
    где \({ f }\) --- \begin{icloze}{2}функция на области в \({ \mathbb R^2 }\).\end{icloze}
\end{note}

\begin{note}{aa5c740235f848c79fb3bbc39d4a3160}
    Функция \({ y }\) называется \begin{icloze}{2}решением ОДУ на множестве \({ X }\),\end{icloze} если \begin{icloze}{1}в любой точке \({ X }\) её подстановка её значений в ОДУ имеет смысл и приводит к верному равенству.\end{icloze}
\end{note}

\begin{note}{8515eff1b5844e0cba265de7445cf1a0}
    Пусть \({ y }\) --- \begin{icloze}{3}решение ОДУ.\end{icloze}
    \begin{icloze}{1}График \({ y }\)\end{icloze} называется \begin{icloze}{2}интегральной кривой этого уравнения.\end{icloze}
\end{note}

\begin{note}{01533944715c4477a02b8f21087f96d2}
    Сколько решений может иметь произвольное ОДУ?

    \begin{cloze}{1}
        Сколь угодно много.
    \end{cloze}
\end{note}

\begin{note}{76c92fefc3414e8da3c06f458e9e80ee}
    В чём состоит задача Коши для ОДУ первого порядка?

    \begin{cloze}{1}
        Найти решение, отвечающее начальным условиям.
    \end{cloze}
\end{note}

\begin{note}{94e8aba670fa45eab4fae8fee8d23568}
    Что есть ``начальные условия'' из формулировки задачи Коши для ОДУ первого порядка?

    \begin{cloze}{1}
        \({ y(x_0) = y_0 }\) для фиксированных \({ x_0, y_0 }\).
    \end{cloze}
\end{note}

\begin{note}{0277eb8d5c00466ca6bb797bd58c8279}
    Как называются значения \({ (x_0, y_0) }\) в задаче Коши для ОДУ первого порядка?

    \begin{cloze}{1}
        Начальные данные.
    \end{cloze}
\end{note}

\begin{note}{aae5b8ef39d14aab9d038ebe894b7b99}
    Какие значения могут принимать начальные данные в задаче Коши для ОДУ первого порядка?

    \begin{cloze}{1}
        Любые, для которых ОДУ имеет смысл.
    \end{cloze}
\end{note}

\begin{note}{4ecfb902661d484682379f7f0b7b2567}
    На каком множестве нужно найти решение задачи Коши с начальными данными \({ (x_0, y_0) }\)?

    \begin{cloze}{1}
        Интервал, включающий \({ x_0 }\).
    \end{cloze}
\end{note}


\end{document}
