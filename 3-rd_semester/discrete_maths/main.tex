%! TeX root = ./main.tex
\documentclass[11pt, a5paper]{article}
\usepackage[width=10cm, top=0.5cm, bottom=2cm]{geometry}

\usepackage[T1,T2A]{fontenc}
\usepackage[utf8]{inputenc}
\usepackage[english,russian]{babel}
\usepackage{libertine}

\usepackage{amsmath}
\usepackage{amssymb}
\usepackage{amsthm}
\usepackage{mathrsfs}
\usepackage{framed}
\usepackage{xcolor}

\setlength{\parindent}{0pt}

% Force \pagebreak for every section
\let\oldsection\section
\renewcommand\section{\pagebreak\oldsection}

\renewcommand{\thesection}{}
\renewcommand{\thesubsection}{Note \arabic{subsection}}
\renewcommand{\thesubsubsection}{}
\renewcommand{\theparagraph}{}

\newenvironment{note}[1]{\goodbreak\par\subsection{\hfill \color{lightgray}\tiny #1}}{}
\newenvironment{cloze}[2][\ldots]{\begin{leftbar}}{\end{leftbar}}
\newenvironment{icloze}[2][\ldots]{%
  \ignorespaces\text{\tiny \color{lightgray}\{\{c#2::}\hspace{0pt}%
}{%
  \hspace{0pt}\text{\tiny\color{lightgray}\}\}}\unskip%
}


\begin{document}
\section{Интуитивная теория множеств}
\begin{note}{6c0ed6eb23d8405e911650386a84b770}
    Под \begin{icloze}{2}множеством\end{icloze} понимается \begin{icloze}{1}некоторая, вполне определённая совокупность объектов.\end{icloze}
\end{note}

\begin{note}{5f9814dbb38246348e00ffce1554e94a}
    Два основных способа задания множеств.

    \begin{cloze}{1}
        Перечисление, характеристическое правило.
    \end{cloze}
\end{note}

\begin{note}{325300814df34c129e29e55cd92829be}
    \begin{icloze}{2}Пустое множество\end{icloze} есть \begin{icloze}{1}множество, которое не содержит элементов.\end{icloze}
\end{note}

\begin{note}{f4cb071a174b4cd29c7ac0c7cd405265}
    \begin{icloze}{2}Пустое\end{icloze} множество обозначается \begin{icloze}{1}\({ \emptyset }\) или \({ \left\{  \right\} }\).\end{icloze}
\end{note}

\begin{note}{ee3c092ea6f8412982372151ed6a3ef8}
    Пусть \({ A }\) --- множество.
    \begin{icloze}{1}Само множество \({ A }\) и пустое множество\end{icloze} называют \begin{icloze}{2}несобственными подмножествами\end{icloze} множества \({ A }\).
\end{note}

\begin{note}{d2d19259b6054a569cee5d5a0b24b0fe}
    Пусть \({ A }\) --- множество.
    \begin{icloze}{1}Все подмножества \({ A }\), кроме \({ \emptyset }\) и \({ A }\),\end{icloze} называют \begin{icloze}{2}собственными подмножествами\end{icloze} множества \({ A }\)
\end{note}

\begin{note}{02ebf0e734664103a97df0f5c597b8c7}
    Пусть \({ A }\) --- множество.
    \begin{icloze}{2}Множество всех подмножеств множества \({ A }\)\end{icloze} называется \begin{icloze}{1}булеаном\end{icloze} множества \({ A }\).
\end{note}

\begin{note}{ac2c9531b8ad48eabb9e76bac3fdffaa}
    Пусть \({ A }\) --- множество.
    \begin{icloze}{2}Булеан\end{icloze} множества \({ A }\) обозначается \begin{icloze}{1}\({ \mathcal P(A) }\).\end{icloze}
\end{note}

\begin{note}{2355b9e8f18a44148a0a3fd9f08c2034}
    \begin{icloze}{2}Универсальное множество\end{icloze} есть \begin{icloze}{1}множество такое, что все рассматриваемые множества являются его подмножествами.\end{icloze}
\end{note}

\begin{note}{446b3cd12ece46568e02af4ed65f3155}
    \begin{icloze}{2}Универсальное\end{icloze} множество обычно обозначается \begin{icloze}{1}\({ U }\) или \({ I }\).\end{icloze}
\end{note}

\begin{note}{c7621865085b4ac5a4b2b24efb11cf87}
    Приоритет операций над множествами: \begin{icloze}{1}\({ \overline{\ \cdot\ }, \cap, \cup, \ldots }\)\end{icloze}
\end{note}

\begin{note}{6b9f3c8671f2472e9e3b9a20aeb66aa5}
    Пусть \({ A }\) и \({ B }\) --- множества.
    Для удобства часто используется сокращение
    \[
        \begin{icloze}{2}AB\end{icloze} \coloneqq \begin{icloze}{1}A \cap B.\end{icloze}
    \]
\end{note}

\begin{note}{dc6fc558021f401696123dddc6c61abe}
    Пусть \({ A }\) и \({ B }\) --- множества.
    \begin{icloze}{2}Симметрической разностью\end{icloze} множеств \({ A }\) и \({ B }\) называется множество
    \begin{icloze}{1}
        \[
            (A \cup B) \setminus (A \cap B).
        \]
    \end{icloze}
\end{note}

\begin{note}{1c0cfd677111482c8d16fb1c43f9f802}
    Пусть \({ A }\) и \({ B }\) --- множества.
    \begin{icloze}{2}Симметрическая разность\end{icloze} множеств \({ A }\) и \({ B }\) обозначается \begin{icloze}{1}\({ A \bigtriangleup B }\).\end{icloze}
\end{note}

\begin{note}{658fb28e676a412082702daf0103e08e}
    Пусть \({ A }\) --- множество.
    \begin{icloze}{2}Дополнение \({ A }\)\end{icloze} обозначается \begin{icloze}{1}\({ \overline{A} }\).\end{icloze}
\end{note}

\begin{note}{13a0dc7af20b45a4b8d8785debbb106a}
    Три первых свойства свойства операций объединения и пересечения множеств.

    \begin{cloze}{1}
        Коммутативность, ассоциативность, дистрибутивность.
    \end{cloze}
\end{note}

\begin{note}{0ab39012eaa94abcb901e5c26354d65b}
    Пусть \({ A }\) --- множество.
    \[
        A \cap A = \begin{icloze}{1}A.\end{icloze}
    \]
\end{note}

\begin{note}{99349135847f4ab7a28f76b06715594e}
    Пусть \({ A }\) --- множество.
    \[
        A \cup A = \begin{icloze}{1}A.\end{icloze}
    \]
\end{note}

\begin{note}{02876f67e1514f6d92d1e32ce2a5673f}
    Пусть \({ A }\) --- множество.
    \[
        A \cup \overline{A} = \begin{icloze}{1}U.\end{icloze}
    \]
\end{note}

\begin{note}{3303d884a57c4c979ab67f664325626a}
    Пусть \({ A }\) --- множество.
    \[
        A \cap \overline{A} = \begin{icloze}{1}\emptyset.\end{icloze}
    \]
\end{note}

\begin{note}{c6b6114579204c8e99c5bfbc80ac53b9}
    Пусть \({ A }\) --- множество.
    \[
        A \cup \emptyset = \begin{icloze}{1}A.\end{icloze}
    \]
\end{note}

\begin{note}{35fbc385403041a7a92f1a9980d5643f}
    Пусть \({ A }\) --- множество.
    \[
        A \cap \emptyset = \begin{icloze}{1}\emptyset.\end{icloze}
    \]
\end{note}

\begin{note}{bf06afa6211c4b10bd2ecffa833b05a2}
    Пусть \({ A }\) --- множество.
    \[
        A \cup U = \begin{icloze}{1}U.\end{icloze}
    \]
\end{note}

\begin{note}{b5e4ab6a90eb4de38aa91aa27c7c4847}
    Пусть \({ A }\) --- множество.
    \[
        A \cap U = \begin{icloze}{1}A.\end{icloze}
    \]
\end{note}

\begin{note}{4e1167b5fa7748e68b1a4b9a80eaacb3}
    Пусть \({ A }\) и \({ B }\) --- множества.
    \[
        A \begin{icloze}{2}\cup\end{icloze} (A \begin{icloze}{3}\cap\end{icloze} B) = \begin{icloze}{1}A.\end{icloze}
    \]

    \begin{center}
        \tiny
        <<\begin{icloze}{4}Закон поглощения\end{icloze}>>
    \end{center}
\end{note}

\begin{note}{478752160fb94508a605ed54a8601340}
    Пусть \({ A }\) и \({ B }\) --- множества.
    \[
        A \begin{icloze}{2}\cap\end{icloze} (A \begin{icloze}{3}\cup\end{icloze} B) = \begin{icloze}{1}A.\end{icloze}
    \]

    \begin{center}
        \tiny
        <<\begin{icloze}{4}Закон поглощения\end{icloze}>>
    \end{center}
\end{note}

\begin{note}{84569bc3ab574cb78e9bbc9f21dc6bd6}
    Пусть \({ A }\) и \({ B }\) --- множества.
    \[
        A \cap (B \cup \overline{A}) = \begin{icloze}{1}A \cap B.\end{icloze}
    \]
\end{note}

\begin{note}{8c46cf622a9840ba818604b1ddcbd74f}
    Пусть \({ A }\) и \({ B }\) --- множества.
    \[
        A \cup (B \cap \overline{A}) = \begin{icloze}{1}A \cup B.\end{icloze}
    \]
\end{note}

\begin{note}{f391250023de4aefa419991a4de9c8ab}
    Пусть \({ A }\) и \({ B }\) --- множества.
    \[
        (A \cup B) \begin{icloze}{2}\cap\end{icloze} (A \cup \overline{B}) = \begin{icloze}{1}A.\end{icloze}
    \]

    \begin{center}
        \tiny
        <<\begin{icloze}{3}Закон расщепления\end{icloze}>>
    \end{center}
\end{note}

\begin{note}{29ec5d118d8849bea46146efcbbc4473}
    Пусть \({ A }\) и \({ B }\) --- множества.
    \[
        (A \cap B) \begin{icloze}{2}\cup\end{icloze} (A \cap \overline{B}) = \begin{icloze}{1}A.\end{icloze}
    \]

    \begin{center}
        \tiny
        <<\begin{icloze}{3}Закон расщепления\end{icloze}>>
    \end{center}
\end{note}

\begin{note}{cfe43c6f8ac74a43a3f82ea5e01fee7d}
    Пусть \({ A }\) --- множество.
    \[
        \overline{\overline{A}} = \begin{icloze}{1}A.\end{icloze}
    \]
\end{note}

\begin{note}{edcde29726c04401a88af2ef23f3c264}
    Пусть \({ A }\) и \({ B }\) --- множества.
    \[
        A \setminus B = \begin{icloze}{1}A \cap \overline{B}.\end{icloze}
    \]
\end{note}

\begin{note}{aed19cd8fa0d4ee3abf314b502af697d}
    Пусть \({ A, B }\) и \({ X }\) --- множества.
    \[
        \begin{icloze}{2}X \cup A \subseteq B\end{icloze} \begin{icloze}{3}\iff\end{icloze} \begin{icloze}{1}X \subseteq B \text{ и } A \subseteq B.\end{icloze}
    \]

    \begin{center}
        \tiny
        (при решений уравнений относительно \({ X }\))
    \end{center}
\end{note}

\begin{note}{ec3c6b3684844799a206353c8668876c}
    Пусть \({ A, B }\) и \({ X }\) --- множества.
    \[
        \begin{icloze}{2}A \subseteq X \cap B\end{icloze} \begin{icloze}{3}\iff\end{icloze} \begin{icloze}{1}A \subseteq X \text{ и } A \subseteq B.\end{icloze}
    \]

    \begin{center}
        \tiny
        (при решений уравнений относительно \({ X }\))
    \end{center}
\end{note}

\begin{note}{5f70eba8ee804221a8e31f858c0b43ec}
    Пусть \({ A, B }\) и \({ X }\) --- множества.
    \[
        \begin{icloze}{2}X \cap A \subseteq B\end{icloze} \begin{icloze}{3}\iff\end{icloze} \begin{icloze}{1}X \subseteq \overline{A} \cup B.\end{icloze}
    \]

    \begin{center}
        \tiny
        (при решений уравнений относительно \({ X }\))
    \end{center}
\end{note}

\begin{note}{72ac0b5d9c1746c79264bb9bd3a0b5f2}
    Пусть \({ A, B }\) и \({ X }\) --- множества.
    \[
        \begin{icloze}{2}A \subseteq X \cup B\end{icloze} \begin{icloze}{3}\iff\end{icloze} \begin{icloze}{1}A \cap \overline{B} \subseteq X.\end{icloze}
    \]

    \begin{center}
        \tiny
        (при решений уравнений относительно \({ X }\))
    \end{center}
\end{note}

\begin{note}{9d92e00aafb44695841b52ab137664da}
    Пусть \({ A, B, C, D }\) и \({ X }\) --- множества.
    \[
        \begin{cases}
            A \subseteq X \subseteq B \\
            C \subseteq X \subseteq D
        \end{cases}
        \iff
        \begin{icloze}{2}A \cup C\end{icloze} \subseteq X \subseteq \begin{icloze}{1}B \cap D.\end{icloze}
    \]

    \begin{center}
        \tiny
        (при решений уравнений относительно \({ X }\))
    \end{center}
\end{note}

\begin{note}{ee9afcd63b43416d954d357d1dc689bb}
    В чём основная идея общего алгоритма для решения систем уравнений со множествами?

    \begin{cloze}{1}
        Привести систему к виду \({ A X \cup B \overline{X} = \emptyset }\), где \({ A }\) и \({ B }\) не зависят от \({ X }\).
    \end{cloze}
\end{note}

\begin{note}{f443d4e12a8745178ba97fdf0f1d8772}
    Пусть \({ A }\) и \({ B }\) --- множества.
    \[
        \begin{icloze}{3}A = B\end{icloze} \begin{icloze}{4}\iff\end{icloze} \begin{icloze}{1}A \bigtriangleup B\end{icloze} = \begin{icloze}{2}\emptyset.\end{icloze}
    \]
\end{note}

\begin{note}{06c3d3d8c5614af3b760a31c9b94fdc8}
    Пусть \({ A }\) и \({ B }\) --- множества.
    \[
        A \cup B = \emptyset \begin{icloze}{2}\iff\end{icloze} \begin{icloze}{1}A = \emptyset \text{ и } B = \emptyset.\end{icloze}
    \]
\end{note}

\begin{note}{73259212f85a4411b131299cc49d90dc}
    Пусть \({ A }\) и \({ X }\) --- множества.
    \[
        AX = \emptyset \iff \begin{icloze}{1}X \subseteq \overline{A}.\end{icloze}
    \]

    \begin{center}
        \tiny
        (при решений уравнений относительно \({ X }\))
    \end{center}
\end{note}

\begin{note}{c02302f80f0143d0bb7cdc18b8929288}
    Пусть \({ B }\) и \({ X }\) --- множества.
    \[
        B \overline{X} = \emptyset \iff \begin{icloze}{1}B \subseteq X.\end{icloze}
    \]

    \begin{center}
        \tiny
        (при решений уравнений относительно \({ X }\))
    \end{center}
\end{note}

\begin{note}{96e46cd4122448b3a6c8a8543d793a05}
    Пусть \({ A }\) и \({ B }\) --- множества.
    При каком условии система
    \[
        \begin{cases}
            AX = \emptyset, \\
            B \overline{X} = \emptyset
        \end{cases}
    \]
    имеет решение?

    \begin{cloze}{1}
        \({ B \subseteq \overline{A} }\).
    \end{cloze}
\end{note}

\begin{note}{5e8c77b24b74411e9c9d6769ee278443}
    Пусть \({ A }\) и \({ B }\) --- множества.
    Каково решение системы
    \[
        \begin{cases}
            AX = \emptyset, \\
            B \overline{X} = \emptyset.
        \end{cases}
    \]

    \begin{cloze}{1}
        \[
            B \subseteq X \subseteq \overline{A}.
        \]
    \end{cloze}
\end{note}

\begin{note}{f1c5541c7c884dba936d4374ff51af88}
    Пусть \({ A }\) и \({ B }\) --- множества. Как в уравнении \({ AX \cup B \overline{X} \cup C = \emptyset }\) избавиться от <<свободного>> множества \({ C }\)?

    \begin{cloze}{1}
        \({ C = \emptyset }\) --- условие совместности системы.
    \end{cloze}
\end{note}

\begin{note}{86475fdea01944fba56365048d57b02d}
    Пусть \({ A }\) и \({ B }\) --- множества.
    \[
        (A \times B) \cap (B \times A) = \begin{icloze}{2}\emptyset\end{icloze} \quad \begin{icloze}{3}\iff\end{icloze} \quad \begin{icloze}{1}A \cap B = \emptyset.\end{icloze}
    \]
\end{note}

\begin{note}{8ca45754929648bda3ca5496c7cba70f}
    Операция \begin{icloze}{3}декартового произведения\end{icloze} \begin{icloze}{2}дистрибутивна\end{icloze} относительно \begin{icloze}{1}операций \({ \cap }\), \({ \cup }\), \({ \setminus }\), \({ \bigtriangleup }\).\end{icloze}
\end{note}

\begin{note}{ad330727e2cb4c27970e8cb8fdcdeb23}
    Пусть \({ A, B }\) и \({ C }\) --- множества.
    Равны ли множества \({ (A \times B) \times C }\) и \({ A \times (B \times C) }\)?

    \begin{cloze}{1}
        Их отождествляют и считают равными.
    \end{cloze}
\end{note}

\begin{note}{06a0896de5284f44bac5ddff2170cbb1}
    Пусть \({ A }\) и \({ B }\) --- множества. Для \begin{icloze}{2}конечных\end{icloze} множеств,
    \[
        \left\lvert A \times B \right\rvert = \begin{icloze}{1}\left\lvert A \right\rvert \cdot \left\lvert B \right\rvert.\end{icloze}
    \]
\end{note}

\section{Бинарные отношения}
\begin{note}{cfc293ce41644e75b3df5d21a2bf036d}
    Пусть \({ A }\) и \({ B }\) --- множества.
    \begin{icloze}{2}Бинарным отношением\end{icloze} на множествах \({ A }\) и \({ B }\) называется \begin{icloze}{1}некоторое подмножество \({ A \times B }\).\end{icloze}
\end{note}

\begin{note}{3ba559fe73cf4c90b5b919ce1a45881a}
    Четыре способа задания бинарных отношений.

    \begin{cloze}{1}
        Перечисление, правило, матрица, граф.
    \end{cloze}
\end{note}

\begin{note}{c0ee3ac94a454d748e625d9e8c854763}
    Пусть \({ R \subseteq A \times B }\) --- бинарное отношение.
    \[
        aRb \overset{\hspace{-1pt}\text{\tiny def}}\iff \begin{icloze}{1}(a, b) \in R.\end{icloze}
    \]
\end{note}

\begin{note}{cef6486539a64268a1827f863aa7b9e1}
    Пусть \({ R \subseteq A \times B }\) --- бинарное отношение.
    \begin{icloze}{2}Обратным отношением к \({ R }\)\end{icloze} называется
    \begin{icloze}{1}
        множество
        \[
            \left\{ (b, a) \mid aRb \right\}.
        \]
    \end{icloze}
\end{note}

\begin{note}{5e2c602b70a3473684a8ea79d93c7d68}
    Пусть \({ R \subseteq A \times B }\) --- бинарное отношение.
    \begin{icloze}{2}Обратное отношение к \({ R }\)\end{icloze} обозначается \begin{icloze}{1}\({ R^{-1} }\).\end{icloze}
\end{note}

\begin{note}{d6e34168370e44feafa7891c93b2df04}
    Пусть \({ R \subseteq A \times B }\) --- бинарное отношение.
    Тогда
    \[
        R^{-1} \subseteq \begin{icloze}{1}B \times A\end{icloze}.
    \]
\end{note}

\begin{note}{86c9a04a7ac14724b416780fec688449}
    Пусть \({ R \subseteq A \times B }\) --- бинарное отношение.
    \[
        (R^{-1})^{-1} = \begin{icloze}{1}R.\end{icloze}
    \]
\end{note}

\begin{note}{e91e90545919488bb2c2ebe373b9e615}
    Пусть \({ R \subseteq A \times B }\) --- бинарное отношение.
    \begin{icloze}{2}Областью определения \({ R }\)\end{icloze} называется
    \begin{icloze}{1}
        множество
        \[
            \left\{ x \mid \exists y : xRy \right\}.
        \]
    \end{icloze}
\end{note}

\begin{note}{08e952c62da84566a99743eb4c6c48a5}
    Пусть \({ R \subseteq A \times B }\) --- бинарное отношение.
    \begin{icloze}{2}Область определения \({ R }\)\end{icloze} обозначается \begin{icloze}{1}\({ D(R) }\),\end{icloze} \begin{icloze}{1}\({ \delta_{R} }\)\end{icloze} или \begin{icloze}{1}\({ \operatorname{dom} R }\).\end{icloze}
\end{note}

\begin{note}{13e35bd817d9438690104754dc4d016d}
    Пусть \({ R \subseteq A \times B }\) --- бинарное отношение.
    \begin{icloze}{2}Областью значений \({ R }\)\end{icloze} называется
    \begin{icloze}{1}
        множество
        \[
            \left\{ y \mid \exists x : xRy \right\}.
        \]
    \end{icloze}
\end{note}

\begin{note}{051cc32e89b94beebd49875c952f6b5b}
    Пусть \({ R \subseteq A \times B }\) --- бинарное отношение.
    \begin{icloze}{2}Область значений \({ R }\)\end{icloze} обозначается \begin{icloze}{1}\({ E(R) }\),\end{icloze} \begin{icloze}{1}\({ \rho_{R} }\)\end{icloze} или \begin{icloze}{1}\({ \operatorname{im} R }\).\end{icloze}
\end{note}

\begin{note}{c0426f6bec33477e9bc759610c4d426b}
    Пусть \({ R \subseteq A \times B }\) и \({ S \subseteq B \times C }\) --- бинарные отношения.
    \begin{icloze}{2}Композицией \({ R }\) и \({ S }\)\end{icloze} называется
    \begin{icloze}{1}
        множество
        \[
            \left\{ (a, c) \mid \exists b : aRb \text{ и } bSc \right\}.
        \]
    \end{icloze}
\end{note}

\begin{note}{41418613a7934da4ab810abfcdf24e1a}
    Пусть \({ R \subseteq A \times B }\) и \({ S \subseteq B \times C }\) --- бинарные отношения.
    \begin{icloze}{2}композиция \({ R }\) и \({ S }\)\end{icloze} обозначается
    \begin{icloze}{1}
        \[
            R \circ S.
        \]
    \end{icloze}
\end{note}

\begin{note}{78bbe389ea094b0aad40c370c5092937}
    Является ли операция композиции бинарных отношений коммутативной?

    \begin{cloze}{1}
        Нет.
    \end{cloze}
\end{note}

\begin{note}{63f83037312e4f29a81de945fb387d06}
    Является ли операция композиции бинарных отношений ассоциативной?

    \begin{cloze}{1}
        Да.
    \end{cloze}
\end{note}

\begin{note}{1530beb1e1c24540a8be6f534775cca0}
    Пусть \({ R \subseteq A \times B }\) и \({ S \subseteq B \times C }\) --- бинарные отношения.
    \[
        (R \circ S)^{-1} = \begin{icloze}{1}S^{-1} \circ R^{-1}.\end{icloze}
    \]
\end{note}

\begin{note}{10fae1eae25a48a2998a9be7d6af2e4d}
    Пусть \({ R \subseteq \begin{icloze}{3}A \times A\end{icloze} }\).
    Отношение \({ R }\) называется \begin{icloze}{2}несимметричным,\end{icloze} если \begin{icloze}{1}оно не симметрично, не асимметрично и не антисимметрично.\end{icloze}
\end{note}

\begin{note}{8e02e778a9a5426fa89340cd47a6a0c5}
    Пусть \({ R \subseteq \begin{icloze}{3}A \times A\end{icloze} }\) --- бинарное отношение.
    Отношение \({ R }\) называется \begin{icloze}{2}интранзитивным,\end{icloze} если
    \begin{icloze}{1}
        \[
            aRb \text{ и } bRc \implies \overline{aRc}.
        \]
    \end{icloze}
\end{note}

\begin{note}{fdbc65b9ca3c4a5c8c62ece25b744e92}
    Пусть \({ R \subseteq \begin{icloze}{3}A \times A\end{icloze} }\) --- бинарное отношение.
    Отношение \({ R }\) называется \begin{icloze}{2}нетранзитивным,\end{icloze} если \begin{icloze}{1}оно не транзитивно и не интранзитивно.\end{icloze}
\end{note}

\begin{note}{f3fcca348ef844da9d3cf01b1e27fe1f}
    Матрица \({ A }\) называется \begin{icloze}{2}бинарной,\end{icloze} если \begin{icloze}{1}все её элементы принадлежат множеству \({ \left\{ 0, 1 \right\} }\).\end{icloze}
\end{note}

\begin{note}{25d02bbd94644780a0346254f22a07df}
    Пусть \({ R \subseteq A \times B }\) --- бинарное отношение, \begin{icloze}{3}\({ A }\) и \({ B }\) конечны.\end{icloze}
    \begin{icloze}{2}Матрицей отношения \({ R }\)\end{icloze} называется
    \begin{icloze}{1}
        бинарная матрица
        \[
            \Big( a_i R b_j \Big) \sim \left\lvert A \right\rvert \times \left\lvert B \right\rvert.
        \]
    \end{icloze}
\end{note}

\begin{note}{ce9cf9f0367d40f9bbdd914eb95eb396}
    Пусть \({ R \subseteq A \times B }\) --- бинарное отношение, \({ A }\) и \({ B }\) конечны.
    \begin{icloze}{2}Матрица отношения \({ R }\)\end{icloze} обозначается \begin{icloze}{1}\({ \left\lVert R \right\rVert }\).\end{icloze}
\end{note}

\begin{note}{1f23045998c647aca7a97bcf2a5b5d31}
    Пусть \({ R \subseteq A \times B }\) --- бинарное отношение, \begin{icloze}{3}\({ x \in A }\).\end{icloze}
    \begin{icloze}{1}Множество \({ \left\{ b \mid xRb \right\} }\)\end{icloze} называется \begin{icloze}{2}образом элемента \({ x }\) при отношении \({ R }\).\end{icloze}
\end{note}

\begin{note}{65b799e6a5bc4b01bff56d2146031199}
    Пусть \({ R \subseteq A \times B }\) --- бинарное отношение, \({ x \in A }\).
    \begin{icloze}{2}Образ элемента \({ x }\) при отношении \({ R }\)\end{icloze} обозначается \begin{icloze}{1}\({ R(x) }\).\end{icloze}
\end{note}

\begin{note}{477523df314842d1ad7c5a4d978f2f7a}
    Пусть \({ R \subseteq A \times B }\) --- бинарное отношение, \begin{icloze}{3}\({ x \in B }\).\end{icloze}
    \begin{icloze}{1}Множество \({ \left\{ a \mid aRx \right\} }\)\end{icloze} называется \begin{icloze}{2}прообразом элемента \({ x }\) при отношении \({ R }\).\end{icloze}
\end{note}

\begin{note}{5150bf45802b4bad925b14f51a0d2f24}
    Пусть \({ R \subseteq A \times B }\) --- бинарное отношение, \({ x \in B }\).
    \begin{icloze}{2}Прообраз элемента \({ x }\) при отношении \({ R }\)\end{icloze} обозначается \begin{icloze}{1}\({ R^{-1}(x) }\).\end{icloze}
\end{note}

\begin{note}{3348d69b0cf149a8a70f5ec94b05b306}
    Пусть \({ R \subseteq A \times B }\) --- бинарное отношение, \begin{icloze}{2}\({ X \subseteq A }\).\end{icloze}
    \[
        \begin{icloze}{3}R(X)\end{icloze} \overset{\text{def}}= \begin{icloze}{1}\bigcup_{x \in X} R(x).\end{icloze}
    \]
\end{note}

\begin{note}{5c26a7f17db242d7b8db989512093cc6}
    Пусть \({ R \subseteq A \times B }\) --- бинарное отношение, \begin{icloze}{2}\({ X \subseteq B }\).\end{icloze}
    \[
        \begin{icloze}{3}R^{-1}(X)\end{icloze} \overset{\text{def}}= \begin{icloze}{1}\bigcup_{x \in X} R^{-1}(x).\end{icloze}
    \]
\end{note}

\begin{note}{5b5ba1073a2e479f8b8eca3f6c2c7329}
    Пусть \({ A }\) множество. \begin{icloze}{1}Отношение \({ \left\{ (x, x) \mid x \in A \right\} }\)\end{icloze} называется \begin{icloze}{2}тождественным отношением на \({ A }\).\end{icloze}
\end{note}

\begin{note}{c1e1caa30e724485b938627008bc28d0}
    Пусть \({ A }\) множество. \begin{icloze}{2}Тождественное отношение на \({ A }\)\end{icloze} обозначается \begin{icloze}{1}\({ E }\).\end{icloze}
\end{note}

\begin{note}{1dc3c3c6dff84c6f8ba496ed57840291}
    Пусть \({ R \subseteq A \times B }\) --- бинарное отношение.
    Тогда \({ R }\) \begin{icloze}{2}рефлексивно\end{icloze} тогда и только тогда, когда
    \begin{icloze}{1}
        \[
            E \subseteq R.
        \]
    \end{icloze}

    \begin{center}
        \tiny
        <<В терминах множеств>>
    \end{center}
\end{note}

\begin{note}{ebabe4fca55b4c1c89734c5895e06ff8}
    Пусть \({ R \subseteq A \times B }\) --- бинарное отношение.
    Тогда \({ R }\) \begin{icloze}{2}антирефлексивно\end{icloze} тогда и только тогда, когда
    \begin{icloze}{1}
        \[
            R \cap E = \emptyset.
        \]
    \end{icloze}

    \begin{center}
        \tiny
        <<В терминах множеств>>
    \end{center}
\end{note}

\begin{note}{0b173912f3f54d539053ec72781173bf}
    Пусть \({ R \subseteq A \times B }\) --- бинарное отношение.
    Тогда \({ R }\) \begin{icloze}{2}симметрично\end{icloze} тогда и только тогда, когда
    \begin{icloze}{1}
        \[
            R = R^{-1}.
        \]
    \end{icloze}

    \begin{center}
        \tiny
        <<В терминах множеств>>
    \end{center}
\end{note}

\begin{note}{1d0d52561f0b48f8a96ed987369af728}
    Пусть \({ R \subseteq A \times B }\) --- бинарное отношение.
    Тогда \({ R }\) \begin{icloze}{2}антисимметрично\end{icloze} тогда и только тогда, когда
    \begin{icloze}{1}
        \[
            R \cap R^{-1} \subseteq E.
        \]
    \end{icloze}

    \begin{center}
        \tiny
        <<В терминах множеств>>
    \end{center}
\end{note}

\begin{note}{92c95593c51a4ac08d44f6be1cf69e5e}
    Пусть \({ R \subseteq A \times B }\) --- бинарное отношение.
    Тогда \({ R }\) \begin{icloze}{2}асимметрично\end{icloze} тогда и только тогда, когда
    \begin{icloze}{1}
        \[
            R \cap R^{-1} = \emptyset.
        \]
    \end{icloze}

    \begin{center}
        \tiny
        <<В терминах множеств>>
    \end{center}
\end{note}

\begin{note}{3fb92e0a21764e979cf2dce095e95aea}
    Пусть \({ R \subseteq A \times B }\) --- бинарное отношение.
    Тогда \({ R }\) \begin{icloze}{2}транзитивно\end{icloze} тогда и только тогда, когда
    \begin{icloze}{1}
        \[
            R \circ R \subseteq R.
        \]
    \end{icloze}

    \begin{center}
        \tiny
        <<В терминах множеств>>
    \end{center}
\end{note}

\begin{note}{045dab85eeaa4728b61896649dc1ba75}
    Пусть \({ A, B \in \mathbb R^{n \times m} }\).
    Тогда
    \[
        \begin{icloze}{2}A \leqslant B\end{icloze} \overset{\hspace{-1pt}\text{\tiny def}}\iff \begin{icloze}{1}a_{ij} \leqslant b_{ij} \quad \forall i, j.\end{icloze}
    \]
\end{note}

\begin{note}{3b1e7f3609054643ae820caaeae6db2a}
    Пусть \({ A, B \in \mathbb R^{n \times m} }\).
    Тогда
    \[
        \begin{icloze}{2}A < B\end{icloze} \overset{\hspace{-1pt}\text{\tiny def}}\iff \begin{icloze}{1}A \leqslant B \text{ и } A \neq B.\end{icloze}
    \]
\end{note}

\begin{note}{cfdc6aac0b1d4a87b2bec698ca44ce30}
    Пусть \({ A, B \in \mathbb R^{n \times m} }\).
    Матрицы \({ A }\) и \({ B }\) называют \begin{icloze}{2}несравнимыми,\end{icloze} если \begin{icloze}{1}не выполняется ни \({ A \leqslant B }\), ни \({ B \leqslant A }\).\end{icloze}
\end{note}

\begin{note}{303fa2bd38f446e59e6690ebc8c9c824}
    Бинарную операцию \begin{icloze}{2}<<или>>\end{icloze} так же называют логистическим \begin{icloze}{1}сложением.\end{icloze}
\end{note}

\begin{note}{46107ba23b0a4fcdaaa341d70b37861c}
    Бинарную операцию \begin{icloze}{2}<<и>>\end{icloze} так же называют логистическим \begin{icloze}{1}умножением.\end{icloze}
\end{note}

\begin{note}{1c8c345204344de49b5b669a648c128d}
    \begin{icloze}{2}Операция поэлементного умножения матриц\end{icloze} называется \begin{icloze}{1}произведением Адамара.\end{icloze}
\end{note}

\begin{note}{5510b762349a41cc87225739c6fe6dc0}
    Пусть \({ A, B \in \mathbb R^{n \times m} }\).
    \begin{icloze}{2}Произведение Адамара матриц \({ A }\) и \({ B }\)\end{icloze} обозначается \begin{icloze}{1}\({ A \circ B }\)\end{icloze} или \begin{icloze}{1}\({ A \odot B }\).\end{icloze}
\end{note}

\begin{note}{5054e224483f4cc28f2739f6fad9f517}
    Пусть \({ R, S \subseteq A \times B }\) --- бинарные отношение.
    \[
        \begin{icloze}{2}\left\lVert R \cap S \right\rVert\end{icloze} = \begin{icloze}{1}\left\lVert R \right\rVert \odot \left\lVert S \right\rVert.\end{icloze}
    \]
\end{note}

\begin{note}{93467a16ee87438cbc954b8b71d23aa4}
    Пусть \({ R, S \subseteq A \times B }\) --- бинарные отношение.
    \[
        \begin{icloze}{2}\left\lVert R \cup S \right\rVert\end{icloze} = \begin{icloze}{1}\left\lVert R \right\rVert + \left\lVert S \right\rVert \quad \text{\tiny(с логистическим сложением).}\end{icloze}
    \]
\end{note}

\begin{note}{1c75356f6fe44393ae1e2c195bed3c1e}
    Пусть \({ R \subseteq A \times B }\) и \({ S \subseteq B \times C }\) --- бинарные отношения.
    \[
        \begin{icloze}{2}\left\lVert R \circ S \right\rVert\end{icloze} = \begin{icloze}{1}\left\lVert R \right\rVert \cdot \left\lVert S \right\rVert \quad \text{\tiny(с логистическим сложением).}\end{icloze}
    \]
\end{note}

\begin{note}{525cce9b6e944f94911754eec1fc824b}
    Пусть \({ R, S \subseteq A \times B }\) --- бинарные отношение.
    \[
        \begin{icloze}{2}R \subseteq S\end{icloze} \begin{icloze}{3}\iff\end{icloze} \begin{icloze}{1}\left\lVert R \right\rVert \leqslant \left\lVert S \right\rVert.\end{icloze}
    \]

    \begin{center}
        \tiny
        (в терминах матриц)
    \end{center}
\end{note}

\begin{note}{1b250fddc61e44539e477e7d17458d9b}
    Пусть \({ R \subseteq A \times B }\) --- бинарное отношение.
    Тогда \({ R }\) \begin{icloze}{2}транзитивно\end{icloze} тогда и только тогда, когда
    \begin{icloze}{1}
        \[
            \left\lVert R \right\rVert^2 \leqslant \left\lVert R \right\rVert \quad \text{\tiny(с логистическим сложением).}
        \]
    \end{icloze}

    \begin{center}
        \tiny
        <<В терминах матриц>>
    \end{center}
\end{note}

\begin{note}{bb1c4dba55ad47adbacfd250e1f39101}
    Пусть \({ R \subseteq A \times A }\) --- отношение эквивалентности.
    \begin{icloze}{2}Множество классов эквивалентности \({ R }\)\end{icloze} обозначается \begin{icloze}{1}\({ [A]_{R} }\).\end{icloze}
\end{note}

\begin{note}{c54eb7123d974c8aba9972163019b4ac}
    Пусть \({ R \subseteq A \times A }\) --- отношение эквивалентности, \({ a \in A }\).
    \begin{icloze}{2}Класс эквивалентности, порождённый \({ a }\),\end{icloze} обозначается \begin{icloze}{1}\({ [a] }\).\end{icloze}
\end{note}

\begin{note}{b21c1b2e3c504807a89717a4205b3fdf}
    Пусть \({ A }\) --- множество.
    \begin{icloze}{2}Разбиение множества \({ A }\)\end{icloze} обозначается \begin{icloze}{1}\({ \langle A \rangle }\).\end{icloze}
\end{note}

\begin{note}{3d8bf9b65a4b4898be5460faaaecab86}
    Пусть \({ R \subseteq A \times A }\) --- бинарное отношение.
    \begin{icloze}{2}Транзитивным замыканием \({ R }\)\end{icloze} называют \begin{icloze}{1}наименьшее транзитивное отношение на \({ A }\), включающее \({ R }\).\end{icloze}
\end{note}

\begin{note}{08c79ddd7572454f9ecc2f3580a39674}
    Пусть \({ R \subseteq A \times A }\) --- бинарное отношение.
    Если \begin{icloze}{2}\({ R }\) транзитивно,\end{icloze} то транзитивное замыкание \({ R }\) есть \begin{icloze}{1}само \({ R }\).\end{icloze}
\end{note}

\begin{note}{e7b56866ed8e4192a45f157195f949e4}
    Пусть \({ R \subseteq A \times A }\) --- бинарное отношение.
    \begin{icloze}{3}Транзитивным сокращением \({ R }\)\end{icloze} называется \begin{icloze}{2}минимальное отношение \({ R' }\) на \({ A }\)\end{icloze} такое, что \begin{icloze}{1}транзитивное замыкание \({ R' }\) совпадает с транзитивным замыканием \({ R }\).\end{icloze}
\end{note}

\begin{note}{6de64f5b2a734fa0a32f7312d9f29346}
    \begin{icloze}{3}Диаграмма Хáссе\end{icloze} --- это вид диаграмм, используемый для представления \begin{icloze}{1}конечного частично упорядоченного множества\end{icloze} в виде \begin{icloze}{2}графа его транзитивного сокращения.\end{icloze}
\end{note}

\section{Элементы комбинаторики}
\begin{note}{48bfce03d7414c5ab08f51dd7162fe63}
    \hyphenation{вы-бор-кой}
    \begin{icloze}{2}\({ r }\)-элементный набор из \({ n }\)-элементного множества\end{icloze} называется \begin{icloze}{1}выборкой объёма \({ r }\) из \({ n }\) элементов.\end{icloze}
\end{note}

\begin{note}{9c40042b9af64db3823fd0fc687379f5}
    \hyphenation{вы-бор-кой}
    \begin{icloze}{2}Выборку объёма \({ r }\) из \({ n }\) элементов\end{icloze} так же называют \begin{icloze}{1}\({ (n, r) }\)-выборкой.\end{icloze}
\end{note}

\begin{note}{7b9c414597ef428981257c73511e44d2}
    \begin{icloze}{2}\({ (n, r) }\)-выборка, в которой элементы могут повторяться,\end{icloze} называется \begin{icloze}{1}\({ (n, r) }\)-выборкой с повторениями.\end{icloze}
\end{note}

\begin{note}{6afeb348dbbf4ce7a258ad26ba469c48}
    \hyphenation{вы-бор-кой}
    \begin{icloze}{2}\({ (n, r) }\)-выборка, в которой элементы попарно различны,\end{icloze} называется \begin{icloze}{1}\({ (n, r) }\)-выборкой без повторений.\end{icloze}
\end{note}

\begin{note}{ef4dbbc893164d0db276530cb20c94c7}
    \begin{icloze}{2}Упорядоченная \({ (n, r) }\)-выборка\end{icloze} называется \begin{icloze}{1}\({ (n, r) }\)-пе\-ре\-ста\-нов\-кой.\end{icloze}
\end{note}

\begin{note}{514e05b8ce994556a7d4f31540bfee43}
    Число \begin{icloze}{1}\({ (n, r) }\)-перестановок без повторений\end{icloze} обозначается \begin{icloze}{2}
        \[
            P(n, r).
        \]
    \end{icloze}
\end{note}

\begin{note}{400452c068e84e42a0865821bd703a7b}
    Число \begin{icloze}{2}\({ (n, r) }\)-перестановок с повторениями\end{icloze} обозначается \begin{icloze}{1}
        \[
            \widehat{P}(n, r).
        \]
    \end{icloze}
\end{note}

\begin{note}{376b9f21513d43118cf832e0ad6f8ef0}
    \begin{icloze}{2}Неупорядоченная \({ (n, r) }\)-выборка\end{icloze} называется \begin{icloze}{1}\({ (n, r) }\)-со\-че\-та\-ни\-ем.\end{icloze}
\end{note}

\begin{note}{6470ab31727449d8a82512cafaea2837}
    Число \begin{icloze}{2}\({ (n, r) }\)-сочетаний без повторений\end{icloze} обозначается \begin{icloze}{1}
        \[
            C(n, r)
        \]
    \end{icloze}
\end{note}

\begin{note}{ca7c36f0138749fb90d8876a44c92a24}
    Число \begin{icloze}{2}\({ (n, r) }\)-сочетаний с повторениями\end{icloze} обозначается \begin{icloze}{1}
        \[
            \widehat{C}(n, r)
        \]
    \end{icloze}
\end{note}

\begin{note}{59712aabfb56413995a990d0c381fbee}
    Пусть \({ n \in \mathbb R }\), \({ r \in \mathbb N }\).
    \[
        \begin{icloze}{2}(n)_{r}\end{icloze} \overset{\text{def}}= \begin{icloze}{1}n(n - 1) \cdots (n - r + 1).\end{icloze}
    \]
\end{note}

\begin{note}{a10b62e86e38446c85f4bb8c5807d6c2}
    Биномиальный коэффициент из \({ n }\) по \({ r }\) обозначается
    \[
        \begin{icloze}{1}C_n^r\end{icloze} \quad \text{или} \quad \begin{icloze}{1}\binom{n}{r}.\end{icloze}
    \]
\end{note}

\begin{note}{3221712b5dda4ebe9e522f4508804522}
    \[
        \binom{n}{r} \overset{\text{def}}= \begin{icloze}{1}\frac{(n)_{r}}{r!}\end{icloze}
    \]
\end{note}

\begin{note}{e1ac7a181662466fa92a7768e3bb6899}
    \[
        P(n, r) = \begin{icloze}{1}(n)_{r}\end{icloze}
    \]
\end{note}

\begin{note}{4722cda874c44899a9bc36727640274a}
    \[
        \widehat{P}(n, r) = \begin{icloze}{1}n^{r}.\end{icloze}
    \]
\end{note}

\begin{note}{bdb9dd6722f644019fedc6c94810b129}
    \[
        C(n, r) = \begin{icloze}{1}\binom{n}{r}.\end{icloze}
    \]
\end{note}

\begin{note}{a46501a9e6f54ccbb15eb513c9b73039}
    \[
        \widehat{C}(n, r) = \begin{icloze}{1}\binom{n+r-1}{n-1}.\end{icloze}
    \]
\end{note}

\section{Алгебра логики}
\begin{note}{d782cd98cdeb44098d0d1c31a6912d8d}
    Кратко булев набор \({ (\alpha_1, \alpha_2, \ldots, \alpha_n) }\) обозначается \begin{icloze}{1}\({ \widetilde \alpha ^{n} }\)\end{icloze} или \begin{icloze}{1}\({ \widetilde \alpha }\).\end{icloze}
\end{note}

\begin{note}{b4eabdff1da9430caa51dea1f7973ebb}
    \begin{icloze}{2}Множество всех двоичных наборов длины \({ n }\)\end{icloze} называют \begin{icloze}{1}\({ n }\)-мерным булевым кубом.\end{icloze}
\end{note}

\begin{note}{ae679a4256e04958a9d0d03ce4b174a2}
    \begin{icloze}{2}\({ n }\)-мерный булев куб\end{icloze} обозначается \begin{icloze}{1}\({ B^{n} }\)\end{icloze} или \begin{icloze}{1}\({ E_2^{n} }\).\end{icloze}
\end{note}

\begin{note}{c06059c4d3014e8f8d7762ecb2e7fb4e}
    Пусть \({ \widetilde \alpha, \widetilde \beta \in B^{n} }\).
    \begin{icloze}{2}Расстоянием Хэмминга между \({ \widetilde \alpha^{n} }\) и \({ \widetilde \beta^{n} }\)\end{icloze} называется
    \begin{icloze}{1}число координат, в которых наборы \({ \widetilde \alpha }\) и \({ \widetilde \beta }\) различны.\end{icloze}
\end{note}

\begin{note}{4b632fb9309f40b5b4286d3acbc0f06d}
    Пусть \({ \widetilde \alpha, \widetilde \beta \in B^{n} }\).
    \begin{icloze}{2}Расстояние Хэмминга между \({ \widetilde \alpha }\) и \({ \widetilde \beta }\)\end{icloze} обозначается \begin{icloze}{1}\({ \rho(\widetilde \alpha, \widetilde \beta) }\).\end{icloze}
\end{note}

\begin{note}{37fb9e894285459ea68b457c8ce5d2d3}
    Булевы наборы \({ \widetilde \alpha^{n} }\) и \({ \widetilde \beta^{n} }\) называются \begin{icloze}{2}соседними,\end{icloze} если
    \begin{icloze}{1}
        \[
            \rho(\widetilde \alpha, \widetilde \beta) = 1.
        \]
    \end{icloze}
\end{note}

\begin{note}{96ae235391374f19821a21b56b6d8b98}
    Булевы наборы \({ \widetilde \alpha^{n} }\) и \({ \widetilde \beta^{n} }\) называются \begin{icloze}{2}противоположными,\end{icloze} если
    \begin{icloze}{1}
        \[
            \rho(\widetilde \alpha, \widetilde \beta) = n.
        \]
    \end{icloze}
\end{note}

\begin{note}{463c6576cc6b4f2cb06d5c75b36c3db1}
    \begin{icloze}{2}Множество всех булевых функций, зависящих от переменных \({ x_1, \ldots, x_n }\)\end{icloze} будем обозначать через
    \begin{icloze}{1}
        \[
            P_2(X^{n}).
        \]
    \end{icloze}
\end{note}

\begin{note}{d4d14bf7854d445e80552d7c67d03d4f}
    \[
        \left\lVert P_2(X^{n}) \right\rVert = \begin{icloze}{1}2^{2^{n}}.\end{icloze}
    \]
\end{note}

\begin{note}{152863b499e64f64b2374c749fbde8ad}
    \begin{icloze}{1}Константы \({ 0, 1 }\)\end{icloze} являются \begin{icloze}{2}нульместными\end{icloze} булевыми функциями.
\end{note}

\begin{note}{d06f0289418d4a41b071c552a48bff09}
    Пусть \({ f(\widetilde x^{n}) }\) --- булева функция.
    Тогда
    \begin{icloze}{2}
        переменные
        \[
            x_1, x_2, \ldots, x_2
        \]
    \end{icloze}
    называются \begin{icloze}{1}аргументами функции \({ f }\).\end{icloze}
\end{note}

\begin{note}{a0b4e4b264844402aed619c4fd37ca24}
    Пусть \({ f(\widetilde x^{n}) }\) --- булева функция.
    Что есть \({ T(f) }\)?

    \begin{cloze}{1}
        Таблица, в которой слева --- значения аргументов, справа --- значения функции.
    \end{cloze}
\end{note}

\begin{note}{bf9b50dc17e84201a627f5c871ea6c31}
    Пусть \({ f(\widetilde x^{n}) }\) --- булева функция.
    \begin{icloze}{2}Таблица \({ T(f) }\)\end{icloze} называется \begin{icloze}{1}таблицей истинности \({ f }\).\end{icloze}
\end{note}

\begin{note}{1122c8a455b64cf789b394df752b821e}
    Пусть \({ f(\widetilde x^{n}) }\) --- булева функция.
    Что есть \({ \Pi_{k,n-k}(f) }\)?

    \begin{cloze}{1}
        Таблица, в которой слева --- значения \({ k }\) аргументов, сверху --- значения остальных аргументов, на пересечении --- значение функции.
    \end{cloze}
\end{note}

\begin{note}{903d8fda79124586a7240910bb5d8a70}
    Пусть \({ f(\widetilde x^{n}) }\) --- булева функция.
    В каком порядке идут значения аргументов в таблице \({ \Pi_{k,n-k} }\)?

    \begin{cloze}{1}
        Слева направо, сверху вниз.
    \end{cloze}
\end{note}

\begin{note}{53de4b3157f34550908c2cc6c8752a37}
    Пусть \({ f(\widetilde x^{n}) }\) --- булева функция.
    Что есть \({ N_{f} }\)?

    \begin{cloze}{1}
        Множество наборов \({ \widetilde \alpha }\), для которых \({ f(\widetilde \alpha) = 1 }\).
    \end{cloze}
\end{note}

\begin{note}{ec43c333201b45f28e9a03f6a2828c27}
    Как булева функция задаётся в виде вектора значений?

    \begin{cloze}{1}
        Значения функции в лексикографическом порядке следования наборов аргументов.
    \end{cloze}
\end{note}

\begin{note}{6b8ed8864e3f468bb75dec0ad534e2b6}
    Как строится разложение булевой функции по нескольким переменным?

    \begin{cloze}{1}
        Аналогично разложению по одной построить дизъюнкцию по всем возможным значениям этих переменных.
    \end{cloze}
\end{note}

\begin{note}{3915fbd594044a7782a5ecfb1b697c23}
    Пусть \({ f }\) --- булева функция.
    Переменная \({ x_i }\) называется \begin{icloze}{2}фиктивной,\end{icloze} если \begin{icloze}{1}\({ f }\) выражается формулой, не содержащей \({ x_i }\).\end{icloze}
\end{note}

\begin{note}{ca616b103ccb43dbbb362f5dcf0fa7f6}
    Пусть \({ f }\) --- булева функция.
    Переменная \({ x_i }\) называется \begin{icloze}{2}существенной,\end{icloze} если \begin{icloze}{1}она не является фиктивной.\end{icloze}
\end{note}

\begin{note}{1a853308fb53438c836240f876a13076}
    Пусть \({ f }\) --- булева функция.
    Как показать, что \({ x_i }\) является существенной переменной?

    \begin{cloze}{1}
        Найти два набора аргументов, отличающихся только значением \({ x_i }\), для которых отличается значение \({ f }\).
    \end{cloze}
\end{note}

\begin{note}{b171f4d5373543008b3c7165b170294d}
    Пусть \({ x_i \in X^{n} }\) и \({ \sigma \in B }\).
    Тогда
    \[
        \begin{icloze}{2}x_i^{\sigma}\end{icloze} \overset{\text{def}}=
        \begin{icloze}{1}
            \begin{cases}
                x_i, & \sigma = 1, \\
                \bar x_i, & \sigma = 0.
            \end{cases}
        \end{icloze}
    \]
\end{note}

\begin{note}{3ff8ee76361d4f8aa2dc6817fb8e23c5}
    Пусть \({ x_i \in X^{n} }\) и \({ \sigma \in B }\).
    Выражение \({ x_i^{\sigma} }\) называется \begin{icloze}{1}буквой.\end{icloze}
\end{note}

\begin{note}{434c03fe1f394ebf9ed47f45f7432dcc}
    Что называется конъюнкцией над множеством переменных?

    \begin{cloze}{1}
        Конъюнкция набора букв из этих переменных.
    \end{cloze}
\end{note}

\begin{note}{285409914ff4434dba7d33024928ee2c}
    Что называется дизъюнкцией над множеством переменных?

    \begin{cloze}{1}
        Дизъюнкция набора букв из этих переменных.
    \end{cloze}
\end{note}

\begin{note}{08a4fb8d8f014dfab25d4e1bb8a7f2e3}
    Конъюнкция (или дизъюнкция) над множеством переменных называется \begin{icloze}{2}элементарной,\end{icloze} если она \begin{icloze}{1}не содержит двух одинаковых переменных.\end{icloze}
\end{note}

\begin{note}{69bac32971cb4720860b4dd4a9d453cb}
    \begin{icloze}{2}Число букв\end{icloze} в элементарной конъюнкции (или дизъюнкции) называется \begin{icloze}{1}её рангом.\end{icloze}
\end{note}

\begin{note}{085748c48aca434f80ee81ccc4326795}
    Что считают элементарной конъюнкцией нулевого ранга?

    \begin{cloze}{1}
        Константу \({ 1 }\).
    \end{cloze}
\end{note}

\begin{note}{54299e611ec9467c80fdf0edb992ffda}
    Что считают элементарной дизъюнкцией нулевого ранга?

    \begin{cloze}{1}
        Константу \({ 0 }\).
    \end{cloze}
\end{note}

\begin{note}{7daec66acbc147049ae871ee63127e67}
    \begin{icloze}{1}Дизъюнкция попарно различных элементарных конъюнкций\end{icloze} называется \begin{icloze}{2}дизъюнктивной нормальной формой.\end{icloze}
\end{note}

\begin{note}{9da6ac44d6864608adffeff2f2653b9e}
    \begin{icloze}{1}Конъюнкция попарно различных элементарных дизъюнкций\end{icloze} называется \begin{icloze}{2}конъюнктивной нормальной формой.\end{icloze}
\end{note}

\begin{note}{4b63a16430be4e91a44d687409d1a59f}
    Дизъюнктивная нормальная форма называется \begin{icloze}{2}совершенной,\end{icloze} если \begin{icloze}{1}все её элементарные конъюнкции имеют максимальный ранг.\end{icloze}
\end{note}

\begin{note}{7f6e14160f284a26b1685ab037b732a0}
    Конъюнктивная нормальная форма называется \begin{icloze}{2}совершенной,\end{icloze} если \begin{icloze}{1}все её элементарные дизъюнкции имеют максимальный ранг.\end{icloze}
\end{note}

\begin{note}{5f9836f8024544a29c0d75faa28110d8}
    Что для элементарной конъюнкции означает обладание максимальным рангом?

    \begin{cloze}{1}
        Она содержит все переменные.
    \end{cloze}
\end{note}

\begin{note}{5c8cca9d57834de1a77e826c87fcff5f}
    Что называется минимальным термом набора переменных?

    \begin{cloze}{1}
        Элементарная конъюнкция, содержащая все эти переменные.
    \end{cloze}
\end{note}

\begin{note}{487bd0adfeec4611abb3e25da2b5c9d7}
    \begin{icloze}{2}Минимальный терм\end{icloze} сокращается как \begin{icloze}{1}минтерм.\end{icloze}
\end{note}

\begin{note}{4c650846a2ca498cb6f414f00874d0c6}
    В чём основная особенность любого минтерма?

    \begin{cloze}{1}
        Он истинен только ровно на одном наборе аргументов.
    \end{cloze}
\end{note}

\begin{note}{6353970c4054404c9fe9936241a3f8d2}
    В чём основная особенность любой элементарной дизъюнкции максимального ранга?

    \begin{cloze}{1}
        Она ложна ровно на одном наборе аргументов.
    \end{cloze}
\end{note}

\begin{note}{291d8c80606b4957b2615d312164f8da}
    Какая элементарная дизъюнкция обращается в ноль ровно на одном наборе аргументов?

    \begin{cloze}{1}
        Элементарная дизъюнкция максимального ранга.
    \end{cloze}
\end{note}

\begin{note}{76f3e85c81f14d4fa7c9d4699fd71df7}
    Элементарная \begin{icloze}{1}дизъюнкция\end{icloze} максимального ранга обращается в \begin{icloze}{2}ноль\end{icloze} ровно на одном наборе аргументов.
\end{note}

\begin{note}{c5ad27ca1aaa4a5f98e04fa49b506bd4}
    Элементарная \begin{icloze}{1}конъюнкция\end{icloze} максимального ранга обращается в \begin{icloze}{2}единицу\end{icloze} ровно на одном наборе аргументов.
\end{note}

\begin{note}{bbfcc84390d84f24abd0aee0aa1dfc13}
    Пусть \({ f(\widetilde x^{n}) }\) --- булева функция.
    Тогда
    \[
        f(\widetilde x^{n}) = \begin{icloze}{1}x_i \cdot f|_{x_i = 1} \lor \bar x_i \cdot f|_{x_i = 0}.\end{icloze}
    \]

    \begin{center}
        \tiny
        <<\begin{icloze}{2}Теорема разложения для д.н.ф.\end{icloze}>>
    \end{center}
\end{note}

\begin{note}{45d478dc3b0b46789b3b2af3d10d0c35}
    Каждая булева функция \begin{icloze}{3}единственным образом\end{icloze} представляется в виде \begin{icloze}{1}совершенной\end{icloze} дизъюнктивной (или конъюнктивной) нормальной формы \begin{icloze}{2}своих аргументов.\end{icloze}
\end{note}

\begin{note}{3115335543484943b10f4139847114fa}
    Каждая булева функция единственным образом представляется в виде совершенной дизъюнктивной нормальной формы своих аргументов.
    В чём ключевая идея доказательства?

    \begin{cloze}{1}
        С.д.н.ф. состоит из минтермов, отвечающих единицам в векторе значений.
    \end{cloze}
\end{note}

\begin{note}{ad0588753e0342ea901fb670f955419b}
    Как строится с.д.н.ф. по таблице значений булевой функции?

    \begin{cloze}{1}
        Для каждой единицы строится соответствующий минтерм.
   \end{cloze}
\end{note}

\begin{note}{86b0370b79784d5cb1f61f8d43ea62ca}
    Как строится с.к.н.ф. по таблице значений булевой функции?

    \begin{cloze}{1}
        Для каждого нуля строится соответствующая элементарная дизъюнкция.
   \end{cloze}
\end{note}

\begin{note}{d5f6af188d8a41b2a3a644065c3ffa60}
    Логическая операция \begin{icloze}{2}<<не-и>>\end{icloze} так же называется \begin{icloze}{1}штрихом Шеффера.\end{icloze}
\end{note}

\begin{note}{5cb3b093852847ad992d8f92afed8778}
    В булевой алгебре, \begin{icloze}{2}штрих Шеффера\end{icloze} обозначается \begin{icloze}{1}\({ x_1 \mid x_2 }\).\end{icloze}
\end{note}

\begin{note}{53029bbedc0b455e801cee6510a98bf5}
    В чём смысл штриха Шеффера?

    \begin{cloze}{1}
        Аргументы не могут быть истинными одновременно.
    \end{cloze}
\end{note}

\begin{note}{77f305f399f947bd90023950db2c7072}
    Пусть \({ \widetilde x \in B^2 }\).
    Как читается выражение <<\({ x_1 \mid x_2 }\)>>?

    \begin{cloze}{1}
        \({ x_1 }\) и \({ x_2 }\) не совместны.
    \end{cloze}
\end{note}

\begin{note}{ec1d6d17cbb045b8abb57d404f0e5314}
    Логическая операция \begin{icloze}{2}<<не-или>>\end{icloze} так же называется \begin{icloze}{1}стрелкой Пирса.\end{icloze}
\end{note}

\begin{note}{5bc24294b87949fe93f07f9db9363cc5}
    В булевой алгебре, \begin{icloze}{2}стрелка Пирса\end{icloze} обозначается \begin{icloze}{1}\({ x_1 \downarrow x_2 }\).\end{icloze}
\end{note}

\begin{note}{8c3f9b67e62f4865a81c11d4cf2258c8}
    В чём смысл стрелки Пирса?

    \begin{cloze}{1}
        Оба аргумента ложны.
    \end{cloze}
\end{note}

\begin{note}{44e41888826f48bb87b5d52cfffdff82}
    Пусть \({ \widetilde x \in B^2 }\).
    Как читается выражение <<\({ x_1 \downarrow x_2 }\)>>?

    \begin{cloze}{1}
        Ни \({ x_1 }\), ни \({ x_2 }\).
    \end{cloze}
\end{note}

\begin{note}{811fe08112ea4d4184e59731f1744d49}
    В алгебре логики, \begin{icloze}{2}сложение по модулю \({ 2 }\)\end{icloze} обозначается \begin{icloze}{1}\({ \oplus }\).\end{icloze}
\end{note}

\begin{note}{e2f54e00eb6649a2924ef3362262984c}
    В алгебре логики, \({ \overline{a \sim b} = \begin{icloze}{1}a \oplus b\end{icloze} }\).
\end{note}

\begin{note}{c2e62b1e91d7494ba8a272bc7d96ae90}
    В алгебре логики, \({ \overline{a \oplus b} = \begin{icloze}{1}a \sim b\end{icloze} }\).
\end{note}

\begin{note}{87a39833e41348ee966b516f50698494}
    \begin{icloze}{2}Многочлен над полем \({ \mathbb Z_2 }\)\end{icloze} называется \begin{icloze}{1}многочленом Жегалкина.\end{icloze}
\end{note}

\begin{note}{6bda2a19f81f481ea8880a737a3ff3fa}
    Любая булева функция \begin{icloze}{2}единственным образом\end{icloze} представляется в виде многочлена \begin{icloze}{1}Жегалкина.\end{icloze}
\end{note}

\begin{note}{59a8d454f30342909e9db0c643c83e4a}
    \begin{icloze}{2}Длиной\end{icloze} полинома Жегалкина называется \begin{icloze}{1}число слагаемых в его записи.\end{icloze}
\end{note}

\begin{note}{58b6e8c19b7e4e02aa6fd2ff9accd7af}
    Какие есть основные способы построения полинома Жегалкина для данной булевой функции?

    \begin{cloze}{1}
        Метод неопределённых коэффициентов; преобразование вектора значений; алгоритм Паскаля.
    \end{cloze}
\end{note}

\begin{note}{90f0eb74c2594473a47dc084cdceeee0}
    В каком порядке стоит брать наборы аргументов при построении полинома Жегалкина методом неопределённых коэффициентов?

    \begin{cloze}{1}
        Лексикографическом, начиная с нуля.
    \end{cloze}
\end{note}

\begin{note}{22dd666ef1ac46b6a3db9f23cd06b74e}
    В чём состоит метод преобразования вектора значений для построения полинома Жегалкина?

    \begin{cloze}{1}
        Последовательно преобразовывать блоками все большей величины.
    \end{cloze}
\end{note}

\begin{note}{3fd377445ff746d4bf5d09dfa3611128}
    Какое преобразование используется при построении полинома Жегалкина методом преобразования вектора значений?

    \begin{cloze}{1}
        Делим блок пополам и прибавляем левую часть к правой.
    \end{cloze}
\end{note}

\begin{note}{81aaa6e743d74b498974ea1716a6fb32}
    Какое сложение используется при построении полинома Жегалкина методом преобразования вектора значений?

    \begin{cloze}{1}
        По модулю двух.
    \end{cloze}
\end{note}

\begin{note}{44b6a657f85a489eb1e9d33ec8995171}
    Какого размера блоки берутся при построении полинома Жегалкина методом преобразования вектора значений?

    \begin{cloze}{1}
        Начиная с двух и на каждом следующем шаге в два раза больше.
    \end{cloze}
\end{note}

\begin{note}{a9bd9cf64cea4452bfa49092ad3436d8}
    В чём состоит метод Паскаля построения полинома Жегалкина?

    \begin{cloze}{1}
        Построить треугольник аналогичный треугольнику Паскаля.
    \end{cloze}
\end{note}

\begin{note}{94806c084ac540ef804d5444799594f0}
    Как строиться треугольник при построении полинома Жегалкина методом Паскаля?

    \begin{cloze}{1}
        На каждом шаге складываются по модулю двух соседние элементы вектора.
    \end{cloze}
\end{note}

\begin{note}{b1d7be08723b44eaa0f476e513b31931}
    Где в построенном треугольнике находятся искомые коэффициенты при построении полинома Жегалкина методом Паскаля?

    \begin{cloze}{1}
        Левые элементы сверху вниз.
    \end{cloze}
\end{note}

\begin{note}{b69324ff24e643199e1b880f783ff767}
    В каком порядке берутся коэффициенты полинома Жегалкина при представлении его в виде вектора коэффициентов?

    \begin{cloze}{1}
        В лексикографическом порядке следования булевых масок вхождения элементов в группы.
    \end{cloze}
\end{note}

\begin{note}{bfd91e076ac840e49fdb06ac940c1e2b}
    Как дизъюнкция выражается полиномом Жегалкина?

    \begin{cloze}{1}
        \({ x \oplus y \oplus xy }\).
    \end{cloze}
\end{note}

\begin{note}{61a8295f62404faa91e19a69262e7a15}
    Пусть \({ f(\widetilde x^{n}) }\) --- булева функция.
    \begin{icloze}{3}Импликантой функции \({ f }\)\end{icloze} называется такая \begin{icloze}{2}элементарная конъюнкция \({ k }\) её аргументов,\end{icloze} что
    \begin{icloze}{1}
        \[
            k \lor f(\widetilde x^{n}) = f(\widetilde x^{n}).
        \]
    \end{icloze}
\end{note}

\begin{note}{a3e3fca754924589b36ea95c9edd20fc}
    Что означает условие \({ k \lor f(\widetilde x^{n}) = f(\widetilde x^{n}) }\) из определения импликанты \({ k }\) булевой функции?

    \begin{cloze}{1}
        \({ k \to f(\widetilde x^{n}) }\).
    \end{cloze}
\end{note}

\begin{note}{8f3f29bfc6d54b9ca09a2fd1fdd6e58b}
    Импликанта булевой функции называется \begin{icloze}{2}простой,\end{icloze} если \begin{icloze}{1}после отбрасывания любой из букв она перестаёт быть импликантой.\end{icloze}
\end{note}

\begin{note}{4a3577b789674da58ba03a52b5710ea3}
    Пусть \({ f(\widetilde x^{n}) }\) --- булева функция.
    \begin{icloze}{2}Дизъюнкция всех простых импликант функции \({ f }\)\end{icloze} называется \begin{icloze}{1}сокращённой д.н.ф. функции \({ f }\).\end{icloze}
\end{note}

\begin{note}{c5d072cd661d45e4b1027aadbe203045}
    Конъюнкции, входящие в д.н.ф., называются \begin{icloze}{1}её слагаемыми.\end{icloze}
\end{note}

\begin{note}{1acef456faac42fd8420cc99d8890e5b}
    \begin{icloze}{2}Число слагаемых\end{icloze} д.н.ф. называется \begin{icloze}{1}её длиной.\end{icloze}
\end{note}

\begin{note}{c1dcc531f8c44588bfc596aae2905e6b}
    \begin{icloze}{2}Сумма рангов всех слагаемых\end{icloze} д.н.ф. называется \begin{icloze}{1}её сложностью.\end{icloze}
\end{note}

\begin{note}{8f3adf33c7d542cb89e2c7f19554d84c}
    Д.н.ф. называется \begin{icloze}{3}минимальной,\end{icloze} если она имеет наименьшую \begin{icloze}{1}сложность\end{icloze} среди \begin{icloze}{2}всех д.н.ф., эквивалентных ей.\end{icloze}
\end{note}

\begin{note}{1dd146ddf117426cacc4a76fa773d84d}
    Д.н.ф. называется \begin{icloze}{3}кратчайшей,\end{icloze} если она имеет наименьшую \begin{icloze}{1}длину\end{icloze} среди \begin{icloze}{2}всех д.н.ф., эквивалентных ей.\end{icloze}
\end{note}

\begin{note}{e0998ae22e13457aa7bedbe548e0ce57}
    Д.н.ф. называется \begin{icloze}{2}тупиковой,\end{icloze} если \begin{icloze}{1}отбрасывание любых её слагаемого или буквы приводит к д.н.ф., не эквивалентной исходной.\end{icloze}
\end{note}

\begin{note}{3c3cb501fe524d508cee8a396e246de5}
    Любая минимальная д.н.ф. является \begin{icloze}{1}тупиковой.\end{icloze}
\end{note}

\begin{note}{41b825d38ba944ac959b52db25423ee8}
    Любая кратчайшая д.н.ф. является \begin{icloze}{1}тупиковой.\end{icloze}
\end{note}

\begin{note}{381196254f5b4ff88f7cb1dbe58cdf33}
    \begin{icloze}{1}Двоичный код,\end{icloze} в котором \begin{icloze}{2}соседние значения представляются соседними булевыми наборами\end{icloze}, называется \begin{icloze}{3}кодом Грея.\end{icloze}
\end{note}

\begin{note}{9fb6eca8a3984a83ac41b14b95e52ea7}
    Какое табличное представление функции используется для построения её сокращений д.н.ф.?

    \begin{cloze}{1}
        Карта Карно.
    \end{cloze}
\end{note}

\begin{note}{0c27efc060274a43bd146a48ba517e7d}
    Чем, в первую очередь, является карта Карно?

    \begin{cloze}{1}
        Табличное представление булевой функции.
    \end{cloze}
\end{note}

\begin{note}{404849406d374f84b0027730b5a93b8c}
    Чем карта Карно отличается от обычного табличного представления?

    \begin{cloze}{1}
        Наборы значений аргументов расположены в коде Грея.
    \end{cloze}
\end{note}

\begin{note}{f49b1a9385c14feebbe5a75682af3073}
    Как называется табличное представление булевой функции, в котором наборы значений переменных на каждой из сторон прямоугольника расположены в коде Грея?

    \begin{cloze}{1}
        Карта Карно.
    \end{cloze}
\end{note}

\begin{note}{6f39d187479f4790bb4e1a151406918b}
    Как по другому называют карты Карно?

    \begin{cloze}{1}
        Диаграммы Вейча.
    \end{cloze}
\end{note}

\begin{note}{7ec72a3172754a4c9b75c2c12151406d}
    Для каких функций имеет смысл использование карты Карно для построения сокращённой д.н.ф.?

    \begin{cloze}{1}
        Не более четырёх аргументов.
    \end{cloze}
\end{note}

\begin{note}{3367571a83104bc8aa72c23b88036a5c}
    Что на карте Карно представляет импликанты булевой функции малого числа аргументов?

    \begin{cloze}{1}
        Прямоугольники размера \({ 2^{k} }\), покрывающие только единицы.
    \end{cloze}
\end{note}

\begin{note}{73244fdc1053472da90d49b391d90c61}
    Какие из соответствующих прямоугольников на карте Карно отвечают \textbf{простым} импликантам булевой функции малого числа аргументов?

    \begin{cloze}{1}
        Максимальные по включению.
    \end{cloze}
\end{note}

\begin{note}{3d6c9e0393a241a7a81798b38ba03769}
    Как по данному прямоугольнику на карте Карно строится импликанта функции?

    \begin{cloze}{1}
        Берутся переменные, сохраняющие значение внутри прямоугольника, с соответствующими знаками отрицания.
    \end{cloze}
\end{note}

\begin{note}{dcec278578344605aeec4341ee1fbccf}
    Сколько переменных входит в импликанту, отвечающую прямоугольнику размера \({ 2^{k} }\) на карте Карно?

    \begin{cloze}{1}
        \({ [\text{число переменных}] - k }\)
    \end{cloze}
\end{note}

\begin{note}{12df0cc8ac6448aeaabca21d69e8ac58}
    Как строится сокращённая д.н.ф. по карте Карно?

    \begin{cloze}{1}
        На карте находятся все прямоугольники, отвечающие простым импликантам, и из них строиться сокращённая д.н.ф.
    \end{cloze}
\end{note}

\begin{note}{84f5a9f698ca42048f6b5e5a6c52a18d}
    Как по карте Карно построить все тупиковые д.н.ф. функции?

    \begin{cloze}{1}
        Построить к.н.ф., где каждый множитель --- это дизъюнкция имён импликант, покрывающих данную единицу, и привести её к д.н.ф.
    \end{cloze}
\end{note}

\begin{note}{ec5b60a019984beeb30acf754aa9bf67}
    При построении всех возможных тупиковых д.н.ф. функции по карте Карно, что отвечает искомым д.н.ф. в построенной д.н.ф. из импликант?

    \begin{cloze}{1}
        Каждое отдельное слагаемое.
    \end{cloze}
\end{note}

\begin{note}{8213d7c755db4307b3e66b4f49906322}
    Что используется вместо карты Карно, когда булева функция зависит от большого числа аргументов?

    \begin{cloze}{1}
        Алгоритм Квайна.
    \end{cloze}
\end{note}

\begin{note}{72dd709a9ecb4ffdaa0fffd682758ab6}
    В чём, в общих чертах, состоит алгоритм Квайна?

    \begin{cloze}{1}
        Пошагово применять операции склеивания и поглощения ко всем парам слагаемых.
    \end{cloze}
\end{note}

\begin{note}{6dc04a146f6c4bc786292e7eeaeeabf1}
    С какого представления функции начинается алгоритм Квайна?

    \begin{cloze}{1}
        Совершенная д.н.ф.
    \end{cloze}
\end{note}

\begin{note}{793a3497f4734adf9c710e04dd9c6905}
    До какого момента применяются операции склеивания в алгоритме Квайна?

    \begin{cloze}{1}
        Пока это возможно.
    \end{cloze}
\end{note}

\begin{note}{4e9b5e650af741f5b2d73141d0add6f6}
    В каком порядке применяются операции склеивания и поглощения в алгоритме Квайна?

    \begin{cloze}{1}
        Сначала склеивание ко всем возможным парам, потом так же поглощение и так далее.
    \end{cloze}
\end{note}

\begin{note}{1cc8d30b9006404d862ba491dd696589}
    Как применяется операция склеивания в алгоритме Квайна?

    \begin{cloze}{1}
        Без удаления склеиваемых слагаемых.
    \end{cloze}
\end{note}

\begin{note}{b1ebc3b297184053aaabf561ed08c9d0}
    Что есть результат выполнения алгоритма Квайна?

    \begin{cloze}{1}
        Сокращённая д.н.ф.
    \end{cloze}
\end{note}

\begin{note}{93c3aff28286465d8f1c04e2309d4857}
    Что перечисляется в левой части таблицы Квайна?

    \begin{cloze}{1}
        Простые импликанты булевой функции.
    \end{cloze}
\end{note}

\begin{note}{195e9ac7ffd840a9bd372abb29260386}
    Что перечисляется в верхней части таблицы Квайна?

    \begin{cloze}{1}
        Значения, на которых функция обращается в единицу.
    \end{cloze}
\end{note}

\begin{note}{f44386bc3fe1460b9c4b4d9245cc3b46}
    Что стоит на пересечениях в таблице Квайна?

    \begin{cloze}{1}
        Значения импликант на наборах.
    \end{cloze}
\end{note}

\begin{note}{bd6c452a75a54f8fac85d08a227081e9}
    Как по таблице Квайна строится кратчайшая д.н.ф. функции?

    \begin{cloze}{1}
        Выбрать набор строк минимальной длины, ``покрывающий'' каждый из столбцов.
    \end{cloze}
\end{note}

\begin{note}{2429e5dfb85944919fb847dec0c29a8a}
    В чём основная практическая ценность минимальных и кратчайших д.н.ф.?

    \begin{cloze}{1}
        Они потенциально упрощают схемы.
   \end{cloze}
\end{note}

\begin{note}{72c248c379834f5b86701f9b7d0bc07b}
    Пусть \({ K }\) --- некоторое множество функций алгебры логики.
    \begin{icloze}{2}Замыканием множества \({ K }\)\end{icloze} называется \begin{icloze}{1}множество всех функций, являющихся суперпозициями функций из \({ K }\).\end{icloze}
\end{note}

\begin{note}{3c14f05d74b3432087dc8d02b5e32d2f}
    Пусть \({ K }\) --- некоторое множество функций алгебры логики.
    \begin{icloze}{2}Замыкание множества \({ K }\)\end{icloze} обозначается \begin{icloze}{1}\({ \left[ K \right] }\).\end{icloze}
\end{note}

\begin{note}{a9c3b7f892cc4f0ba7e429a2ae9bfe25}
    Пусть \({ K }\) --- некоторое множество функций алгебры логики.
    Множество \({ K }\) называется \begin{icloze}{2}замкнутым,\end{icloze} если \begin{icloze}{1}\({ [K] = K }\).\end{icloze}
\end{note}

\begin{note}{3887b6c690454ba19288df70564724d1}
    Пусть \({ K }\) --- некоторое множество функций алгебры логики, \({ P \subseteq K }\).
    Множество \({ P }\) называется \begin{icloze}{2}полным в \({ K }\),\end{icloze} если \begin{icloze}{1}\({ \left[ P \right] = K }\).\end{icloze}
\end{note}

\begin{note}{493f75cfa6a04dee935fd6d480889c06}
    Пусть \({ K }\) --- некоторое множество функций алгебры логики, \({ P \subseteq K }\).
    Если \begin{icloze}{2}\({ K }\) замкнуто, а \({ P }\) полно в \({ K }\),\end{icloze} то \({ P }\) называется \begin{icloze}{1}базисом класса \({ K }\).\end{icloze}
\end{note}

\begin{note}{8715e50cadda45ca9047f6e0eff2e87b}
    Что можно сказать о пресечении двух замкнутых множеств функций алгебры логики?

    \begin{cloze}{1}
        Оно замкнуто.
    \end{cloze}
\end{note}

\begin{note}{b1c5e8a567764fb48b6febf6215f31d4}
    Что можно сказать об объединении двух замкнутых множеств функций алгебры логики?

    \begin{cloze}{1}
        В общем случае ничего.
    \end{cloze}
\end{note}

\begin{note}{7fd05ce54904435eb18294179811b1f7}
    Чем примечательно множество булевых функций \({ \left\{ x, \bar x \right\} }\)?

    \begin{cloze}{1}
        Оно замкнуто.
    \end{cloze}
\end{note}

\begin{note}{da842f2d910a4a338420a0c994752d6f}
    Чем примечательно множество булевых функций \({ \left\{ 0, 1 \right\} }\)?

    \begin{cloze}{1}
        Оно замкнуто.
    \end{cloze}
\end{note}

\begin{note}{0a6c832056ee4d2b96e0566585d86be0}
    Две формулы алгебры логики называются \begin{icloze}{2}конгруэнтными,\end{icloze} если \begin{icloze}{1}одна из них может быть получена из другой заменой переменных.\end{icloze}
\end{note}

\begin{note}{3655826e951542a3bb98d89da369c6bb}
    Пусть \({ f }\) --- формула алгебры логики.
    \begin{icloze}{2}Множество всех формул, конгруэнтных \({ f }\),\end{icloze} обозначается \begin{icloze}{1}\({ \left\{ f \right\} }\).\end{icloze}
\end{note}

\begin{note}{d4e78c956bef4290883dfc89f3533b13}
    Класс булевых функций, \begin{icloze}{2}сохраняющих \({ 0 }\) (или \({ 1 }\))\end{icloze}, обозначается \begin{icloze}{1}\({ T_0 }\) (или \({ T_1 }\).)\end{icloze}
\end{note}

\begin{note}{964153dc496c462396cb9600c28b120d}
    Класс \begin{icloze}{2}монотонных\end{icloze} булевых функций обозначается \begin{icloze}{1}\({ M }\).\end{icloze}
\end{note}

\begin{note}{fdac4e7fcde642d3a96938c08b180ae5}
    Класс \begin{icloze}{2}линейных\end{icloze} булевых функций обозначается \begin{icloze}{1}\({ L }\).\end{icloze}
\end{note}

\begin{note}{5d2bf26b265b4bcbb2ebe7ceaf6f6d72}
    Класс \begin{icloze}{2}самодвойственных\end{icloze} булевых функций обозначается \begin{icloze}{1}\({ S }\).\end{icloze}
\end{note}

\begin{note}{e76e73941c6348b09e32f473c6449dda}
    Булева \({ f(\widetilde x^{n}) }\) называется \begin{icloze}{2}линейной по аргументу \({ x_n }\),\end{icloze} если
    \begin{icloze}{1}
        \[
            f = x_n \oplus \varphi(\widetilde x^{n-1}).
        \]
    \end{icloze}
\end{note}

\begin{note}{70b10874d9274fc58691cfa4137a2174}
    Как вектору значений булевой функции \({ f }\) определяется линейность \({ f }\) по аргументу \({ x }\)?

    \begin{cloze}{1}
        Любые два набора, соседние по \({ x }\), отвечают различным значениям \({ f }\).
    \end{cloze}
\end{note}

\begin{note}{1f25f83507a342ebbb345d6f11c3009e}
    Что можно сказать о булевой функции, если она линейна по каждому из своих аргументов?

    \begin{cloze}{1}
        Функция линейна и существенно зависит от всех переменных.
    \end{cloze}
\end{note}

\begin{note}{c1af37cf021b4a39b021704d1b1fcae1}
    Булева функция \begin{icloze}{2}является линейной\end{icloze} \begin{icloze}{3}тогда и только тогда, когда\end{icloze} \begin{icloze}{1}она линейна по каждой существенной переменной.\end{icloze}
\end{note}

\begin{note}{356ba7d44d8b49398fdbc67b91eb9f69}
    Как по вектору значений булевой функции \({ f }\) определить, является ли \({ f }\) линейной?

    \begin{cloze}{1}
        Проверить линейность по каждой существенной переменной.
    \end{cloze}
\end{note}

\section{Булевы алгебры}
\begin{note}{20f3d9e66cae48b4872cde6c9ed57d3a}
    Что есть верхняя граница в контексте произвольного частично упорядоченного множества?

    \begin{cloze}{1}
        Элемент \({ \geqslant }\) любому элементу множества.
    \end{cloze}
\end{note}

\begin{note}{43fa1e0fee2c4a718e26e0c7f9c37c46}
    Что есть нижняя граница в контексте произвольного частично упорядоченного множества?

    \begin{cloze}{1}
        Элемент \({ \leqslant }\) любому элементу множества.
    \end{cloze}
\end{note}

\begin{note}{0ac55444db6e4fed8fa19d402da0fde0}
    Что есть супремум в контексте произвольного частично упорядоченного множества?

    \begin{cloze}{1}
        Наименьшая из верхних границ.
    \end{cloze}
\end{note}

\begin{note}{7f565979844841de8441229417e3e1c8}
    Что есть инфимум в контексте произвольного частично упорядоченного множества?

    \begin{cloze}{1}
        Наибольшая из нижних границ.
    \end{cloze}
\end{note}

\begin{note}{b6e17bbaad124d3ebd6e98b8381a867f}
    Какое множество рассматривается в определении решётки?

    \begin{cloze}{1}
        Частично упорядоченное.
    \end{cloze}
\end{note}

\begin{note}{99e77ba59fe64260b79e25d9f28cad08}
    Какое частично упорядоченное множество называется решёткой?

    \begin{cloze}{1}
        Любое двухэлементное подмножество имеет \({ \sup }\) и \({ \inf }\).
    \end{cloze}
\end{note}

\begin{note}{37a3e12e05de4cdfab62d0dea07344d1}
    Пусть \({ (X, \leqslant) }\) --- решётка, \({ a, b \in X }\).
    Тогда
    \[
        \begin{icloze}{2}a \lor b\end{icloze} \overset{\text{def}}= \begin{icloze}{1}\sup \left\{ a, b \right\}.\end{icloze}
    \]
\end{note}

\begin{note}{c9da1f844e3f4bd9bbe116283730ceeb}
    Пусть \({ (X, \leqslant) }\) --- решётка, \({ a, b \in X }\).
    Тогда
    \[
        \begin{icloze}{2}a \land b\end{icloze} \overset{\text{def}}= \begin{icloze}{1}\inf \left\{ a, b \right\}.\end{icloze}
    \]
\end{note}

\begin{note}{95e0104c61f1411a8b36c0dccc9a0c7d}
    Решётку так же можно определить как универсальную алгебру с операциями \begin{icloze}{1}\({ \land }\) и \({ \lor }\).\end{icloze}
\end{note}

\begin{note}{647880d6a2d84657b6ed1d688cf8a568}
    Какие аксиомы должны выполняться в определении решётки как универсальной алгебры?

    \begin{cloze}{1}
        Идемпотентность, коммутативность, ассоциативность, поглощение.
    \end{cloze}
\end{note}

\begin{note}{a9dc77aa008449b2a79f45c128715ce9}
    В определении решётки \begin{icloze}{1}свойство идемпотентности\end{icloze} на самом деле выводится из \begin{icloze}{2}свойства поглощения.\end{icloze}
\end{note}

\begin{note}{a70eaa64be014d1f8d32db2187ac39b0}
    \begin{icloze}{1}
        \[
            a \land a = a \quad \text{и} \quad a \lor a = a.
        \]
    \end{icloze}

    \begin{center}
        \tiny
        <<\begin{icloze}{2}Идемпотентность\end{icloze}>> (из определения решётки)
    \end{center}
\end{note}

\begin{note}{3d74bd246a8448588f6d80ab75840d4e}
    Пусть \({ (X, \leqslant) }\) --- решётка,\: \({ a, b \in X }\).
    Тогда
    \[
        \begin{icloze}{2}a \leq b\end{icloze} \iff a \land b = \begin{icloze}{1}a\end{icloze}.
    \]
\end{note}

\begin{note}{900809a351c54ff4a39993d52ae1e388}
    Пусть \({ (X, \leqslant) }\) --- решётка,\: \({ a, b \in X }\).
    Тогда
    \[
        \begin{icloze}{2}a \leq b\end{icloze} \iff a \lor b = \begin{icloze}{1}b\end{icloze}.
    \]
\end{note}

\begin{note}{33c337ecec8e4e5e8457ad2312f7a0ce}
    Решётка называется \begin{icloze}{2}дистрибутивной,\end{icloze} если \begin{icloze}{1}в ней \({ \land }\) и \({ \lor }\) обоюдно дистрибутивны.\end{icloze}
\end{note}

\begin{note}{e6915cd9a90b49cdbba81443ba0a14ab}
    \begin{icloze}{2}Нулём\end{icloze} \begin{icloze}{3}частично\end{icloze} упорядоченного множества называется \begin{icloze}{1}его наименьший элемент.\end{icloze}
\end{note}

\begin{note}{12dd64fc8d0648eca210d44a0b7e5ae5}
    Ноль частично упорядоченного множества обозначается \begin{icloze}{1}\({ \mathbf{0} }\).\end{icloze}
\end{note}

\begin{note}{730dc345dc804811be6c1f82b9350a94}
    \begin{icloze}{2}Единицей\end{icloze} \begin{icloze}{3}частично\end{icloze} упорядоченного множества называется \begin{icloze}{1}его наибольший элемент.\end{icloze}
\end{note}

\begin{note}{543c12d0c8214d79ac9fca56bb2ec02e}
    Единица частично упорядоченного множества обозначается \begin{icloze}{1}\({ \mathbf{1} }\).\end{icloze}
\end{note}

\begin{note}{b4bef54da8aa41d1863424cc97398a77}
    Пусть \({ A }\) --- множество.
    Тогда ноль \({ (\mathcal P(A), \subseteq) }\) --- это \begin{icloze}{1}\({ \emptyset }\).\end{icloze}
\end{note}

\begin{note}{8c05502489844ebb8fa500e583edd7d1}
    Пусть \({ A }\) --- множество.
    Тогда единица \({ (\mathcal P(A), \subseteq) }\) --- это \begin{icloze}{1}\({ A }\).\end{icloze}
\end{note}

\begin{note}{94ed508a73f945fc8a6b4c1803cef774}
    Пусть \({ (X, \leqslant) }\) --- решётка,\: \({ x, y \in X }\).
    Элементы \({ x }\) и \({ y }\) называются \begin{icloze}{2}дизъюнктивными,\end{icloze} если
    \begin{icloze}{1}
        \[
            x \land y = \mathbf{0}.
        \]
    \end{icloze}
\end{note}

\begin{note}{ef9dbbf66ec64b659e16cdb4bb830f5e}
    Пусть \({ (X, \leqslant) }\) --- решётка,\: \({ x, y \in X }\).
    Элемент \({ y }\) называется \begin{icloze}{2}дополнением \({ x }\),\end{icloze} если
    \begin{icloze}{1}
        \[
            x \land y = \mathbf{0} \quad \text{и} \quad x \lor y = \mathbf{1}.
        \]
    \end{icloze}
\end{note}

\begin{note}{eb16dfe04f534fcd95dfce6b0eab9636}
    Для каких решёток имеет смысл понятие дополнения?

    \begin{cloze}{1}
        Для решёток с нулём и единицей.
    \end{cloze}
\end{note}

\begin{note}{2b49c7f09eeb4fa09c205e6ef5416e6b}
    Для начала, булева алгебра --- это \begin{icloze}{1}решётка.\end{icloze}
\end{note}

\begin{note}{e823cfd66b9a4b7aba2bcfeb6ae82e0c}
    Какую решётку называют булевой алгеброй?

    \begin{cloze}{1}
        Дистрибутивную; с нулём и единицей; каждый элемент имеет дополнение.
    \end{cloze}
\end{note}

\begin{note}{520a3543220b49dea92ab712105e5b01}
    Как называют дистрибутивную решётку с нулём и единицей, каждый элемент которой имеет дополнение?

    \begin{cloze}{1}
        Булева алгебра.
    \end{cloze}
\end{note}

\begin{note}{55a7f39372324f8692538470e13cebb7}
    В определении \begin{icloze}{3}булевой алгебры\end{icloze} \begin{icloze}{2}свойство поглощения\end{icloze} можно заменить на \begin{icloze}{1}закон тождественности.\end{icloze}
\end{note}

\begin{note}{7c150397485f4e128ed7a0aa7d34f6cc}
    \begin{icloze}{1}
        \[
            a \land 1 = a \quad \text{ и } \quad a \lor 0 = a.
        \]
    \end{icloze}

    \begin{center}
        \tiny
        <<\begin{icloze}{2}Закон тожественности\end{icloze}>>

        (из определения булевой алгебры)
    \end{center}
\end{note}

\begin{note}{5486cf3db94f4129bf68eed1e3194939}
    Каждый элемент булевой алгебры имеет \begin{icloze}{1}единственное дополнение.\end{icloze}
\end{note}

\begin{note}{eb156d1ea036435a89006401851c38d6}
    Пусть \({ (X, \leqslant) }\) --- булева алгебра,\: \({ x \in X }\).
    \begin{icloze}{2}Дополнение \({ x }\)\end{icloze} обозначается \begin{icloze}{1}\({ \overline{x} }\).\end{icloze}
\end{note}

\begin{note}{2b88d7541f5641d5ac8623f8a78c3384}
    Каждый элемент булевой алгебры имеет единственное дополнение.
    В чём ключевая идея доказательства?

    \begin{cloze}{1}
        Умножить \({ \overline{x} }\) на \({ x \lor x^* }\), где \({ x^* }\) --- второе дополнение.
    \end{cloze}
\end{note}

\begin{note}{6baeed8e7301491ea3dfca0b6fd7750d}
    Для булевых алгебр верны \begin{icloze}{1}все основные законы\end{icloze} алгебры логики.
\end{note}

\begin{note}{4faa85785e62481db5f40d0026177f6e}
    В чём ключевая идея доказательства законов Де-Моргана для булевых алгебр?

    \begin{cloze}{1}
        Показать, что правая часть является дополнением по определению.
    \end{cloze}
\end{note}

\begin{note}{d7959d4251d34780bf02493d717f2c6a}
    Пусть \({ (X, \leqslant) }\) --- булева алгебра,\: \({ f, g : X^{n} \to X }\).
    Функции \({ f }\) и \({ g }\) называются \begin{icloze}{2}взаимно двойственными,\end{icloze} если
    \begin{icloze}{1}
        \[
            f(x_1, \ldots, x_n) = \overline{g(\overline{x_1}, \ldots, \overline{x_n})}.
        \]
    \end{icloze}
\end{note}

\begin{note}{13cfa6ef0b5649518e5f73d626669239}
    Пусть \({ (X, \leqslant) }\) --- булева алгебра, \({ f : X^{n} \to X }\).
    \begin{icloze}{2}Функция, двойственная к \({ f }\),\end{icloze} обозначается \begin{icloze}{1}\({ f^* }\).\end{icloze}
\end{note}

\begin{note}{dd63d9bebef6405bad499b8ac612b3d2}
    Пусть \({ (X, \leqslant) }\) --- булева алгебра, \({ f : X^{n} \to X }\).
    \[
        (f^*)^* = \begin{icloze}{1}f.\end{icloze}
    \]
\end{note}

\begin{note}{f46b68627d9a4a49ae7759f86e1198b1}
    Пусть \({ (X, \leqslant) }\) --- булева алгебра, \({ f : X^{n} \to X }\).
    Функция \({ f }\) называется \begin{icloze}{1}самодвойственной,\end{icloze} если \begin{icloze}{2}\({ f^* = f }\).\end{icloze}
\end{note}

\begin{note}{f94e35db4a9047a7bee032a5cae511d0}
    Пусть \({ f : B^{n} \to B }\).
    Если
    \[
        f = (\alpha_1, \ldots, \alpha_{2^{n}}),
    \]
    то
    \[
        f^* = \begin{icloze}{1}(\bar \alpha_{2^{n}}, \ldots, \bar \alpha_1).\end{icloze}
    \]
\end{note}

\begin{note}{cdf74506714e476ca5dfe2e59883d032}
    Пусть \({ (X, \leqslant) }\) --- булева алгебра. \({ \mathbf{0}^* = \begin{icloze}{1}\mathbf{1}\end{icloze} }\).
\end{note}

\begin{note}{7a361ce9cbe84f77b236a38d96ee29a6}
    Пусть \({ (X, \leqslant) }\) --- булева алгебра. \({ \mathbf{1}^* = \begin{icloze}{1}\mathbf{0}\end{icloze} }\).
\end{note}

\begin{note}{a83c28cd41a247e0905cb9b46bc30c39}
    Пусть \({ (X, \leqslant) }\) --- булева алгебра. \({ \land^* = \begin{icloze}{1}\lor\end{icloze} }\).
\end{note}

\begin{note}{32ec500412874f4cb81569a5483da5c9}
    Пусть \({ (X, \leqslant) }\) --- булева алгебра. \({ \lor^* = \begin{icloze}{1}\land\end{icloze} }\).
\end{note}

\begin{note}{1ed5cffea17343ef9953a3bed3081904}
    В алгебре логики, \({ \oplus^* = \begin{icloze}{1}\operatorname{\sim}\end{icloze} }\).
\end{note}

\begin{note}{c7ec94fd1e224455bb8338bc967651b5}
    В алгебре логики, \({ \operatorname{\sim}^* = \begin{icloze}{1}\oplus\end{icloze} }\).
\end{note}

\begin{note}{6d5ed061caee4a21b5e098a197019176}
    В алгебре логики, \({ \mid^* = \begin{icloze}{1}\downarrow\end{icloze} }\).
\end{note}

\begin{note}{d99e4bd6874d47aeb0185f58610c8ab8}
    В алгебре логики, \({ \downarrow^* = \begin{icloze}{1}\mid\end{icloze} }\).
\end{note}

\begin{note}{abf23bd834914587b24eb8d0cf857c59}
    Пусть \({ f : B \to B }\) и \({ f(x) = x }\).
    Тогда \({ f^* = \begin{icloze}{1}f\end{icloze} }\).
\end{note}

\end{document}
