%! TeX root = ./main.tex
\documentclass[11pt, a5paper]{article}
\usepackage[width=10cm, top=0.5cm, bottom=2cm]{geometry}

\usepackage[T1,T2A]{fontenc}
\usepackage[utf8]{inputenc}
\usepackage[english,russian]{babel}
\usepackage{libertine}

\usepackage{amsmath}
\usepackage{amssymb}
\usepackage{amsthm}
\usepackage{mathrsfs}
\usepackage{framed}
\usepackage{xcolor}

\setlength{\parindent}{0pt}

% Force \pagebreak for every section
\let\oldsection\section
\renewcommand\section{\pagebreak\oldsection}

\renewcommand{\thesection}{}
\renewcommand{\thesubsection}{Note \arabic{subsection}}
\renewcommand{\thesubsubsection}{}
\renewcommand{\theparagraph}{}

\newenvironment{note}[1]{\goodbreak\par\subsection{\hfill \color{lightgray}\tiny #1}}{}
\newenvironment{cloze}[2][\ldots]{\begin{leftbar}}{\end{leftbar}}
\newenvironment{icloze}[2][\ldots]{%
  \ignorespaces\text{\tiny \color{lightgray}\{\{c#2::}\hspace{0pt}%
}{%
  \hspace{0pt}\text{\tiny\color{lightgray}\}\}}\unskip%
}


\begin{document}
\section{Интуитивная теория множеств}
\begin{note}{6c0ed6eb23d8405e911650386a84b770}
    Под \begin{icloze}{2}множеством\end{icloze} понимается \begin{icloze}{1}некоторая, вполне определённая совокупность объектов.\end{icloze}
\end{note}

\begin{note}{5f9814dbb38246348e00ffce1554e94a}
    Два основных способа задания множеств.

    \begin{cloze}{1}
        Перечисление, характеристическое правило.
    \end{cloze}
\end{note}

\begin{note}{325300814df34c129e29e55cd92829be}
    \begin{icloze}{2}Пустое множество\end{icloze} есть \begin{icloze}{1}множество, которое не содержит элементов.\end{icloze}
\end{note}

\begin{note}{f4cb071a174b4cd29c7ac0c7cd405265}
    \begin{icloze}{2}Пустое\end{icloze} множество обозначается \begin{icloze}{1}\({ \emptyset }\) или \({ \left\{  \right\} }\).\end{icloze}
\end{note}

\begin{note}{ee3c092ea6f8412982372151ed6a3ef8}
    Пусть \({ A }\) --- множество.
    \begin{icloze}{1}Само множество \({ A }\) и пустое множество\end{icloze} называют \begin{icloze}{2}несобственными подмножествами\end{icloze} множества \({ A }\).
\end{note}

\begin{note}{d2d19259b6054a569cee5d5a0b24b0fe}
    Пусть \({ A }\) --- множество.
    \begin{icloze}{1}Все подмножества \({ A }\), кроме \({ \emptyset }\) и \({ A }\),\end{icloze} называют \begin{icloze}{2}собственными подмножествами\end{icloze} множества \({ A }\)
\end{note}

\begin{note}{02ebf0e734664103a97df0f5c597b8c7}
    Пусть \({ A }\) --- множество.
    \begin{icloze}{2}Множество всех подмножеств множества \({ A }\)\end{icloze} называется \begin{icloze}{1}булеаном\end{icloze} множества \({ A }\).
\end{note}

\begin{note}{ac2c9531b8ad48eabb9e76bac3fdffaa}
    Пусть \({ A }\) --- множество.
    \begin{icloze}{2}Булеан\end{icloze} множества \({ A }\) обозначается \begin{icloze}{1}\({ \mathcal P(A) }\).\end{icloze}
\end{note}

\begin{note}{2355b9e8f18a44148a0a3fd9f08c2034}
    \begin{icloze}{2}Универсальное множество\end{icloze} есть \begin{icloze}{1}множество такое, что все рассматриваемые множества являются его подмножествами.\end{icloze}
\end{note}

\begin{note}{446b3cd12ece46568e02af4ed65f3155}
    \begin{icloze}{2}Универсальное\end{icloze} множество обычно обозначается \begin{icloze}{1}\({ U }\) или \({ I }\).\end{icloze}
\end{note}

\begin{note}{dc6fc558021f401696123dddc6c61abe}
    Пусть \({ A }\) и \({ B }\) --- множества.
    \begin{icloze}{2}Симметрической разностью\end{icloze} множеств \({ A }\) и \({ B }\) называется множество
    \begin{icloze}{1}
        \[
            (A \cup B) \setminus (A \cap B).
        \]
    \end{icloze}
\end{note}

\begin{note}{1c0cfd677111482c8d16fb1c43f9f802}
    Пусть \({ A }\) и \({ B }\) --- множества.
    \begin{icloze}{2}Симметрическая разность\end{icloze} множеств \({ A }\) и \({ B }\) обозначается \begin{icloze}{1}\({ A \bigtriangleup B }\).\end{icloze}
\end{note}

\begin{note}{658fb28e676a412082702daf0103e08e}
    Пусть \({ A }\) --- множество.
    \begin{icloze}{2}Дополнение \({ A }\)\end{icloze} обозначается \begin{icloze}{1}\({ \overline{A} }\).\end{icloze}
\end{note}

\begin{note}{13a0dc7af20b45a4b8d8785debbb106a}
    Три первых свойства свойства операций объединения и пересечения множеств.

    \begin{cloze}{1}
        Коммутативность, ассоциативность, дистрибутивность.
    \end{cloze}
\end{note}

\begin{note}{0ab39012eaa94abcb901e5c26354d65b}
    Пусть \({ A }\) --- множество.
    \[
        A \cap A = \begin{icloze}{1}A.\end{icloze}
    \]
\end{note}

\begin{note}{99349135847f4ab7a28f76b06715594e}
    Пусть \({ A }\) --- множество.
    \[
        A \cup A = \begin{icloze}{1}A.\end{icloze}
    \]
\end{note}

\begin{note}{02876f67e1514f6d92d1e32ce2a5673f}
    Пусть \({ A }\) --- множество.
    \[
        A \cup \overline{A} = \begin{icloze}{1}U.\end{icloze}
    \]
\end{note}

\begin{note}{3303d884a57c4c979ab67f664325626a}
    Пусть \({ A }\) --- множество.
    \[
        A \cap \overline{A} = \begin{icloze}{1}\emptyset.\end{icloze}
    \]
\end{note}

\begin{note}{c6b6114579204c8e99c5bfbc80ac53b9}
    Пусть \({ A }\) --- множество.
    \[
        A \cup \emptyset = \begin{icloze}{1}A.\end{icloze}
    \]
\end{note}

\begin{note}{35fbc385403041a7a92f1a9980d5643f}
    Пусть \({ A }\) --- множество.
    \[
        A \cap \emptyset = \begin{icloze}{1}\emptyset.\end{icloze}
    \]
\end{note}

\begin{note}{bf06afa6211c4b10bd2ecffa833b05a2}
    Пусть \({ A }\) --- множество.
    \[
        A \cup U = \begin{icloze}{1}U.\end{icloze}
    \]
\end{note}

\begin{note}{b5e4ab6a90eb4de38aa91aa27c7c4847}
    Пусть \({ A }\) --- множество.
    \[
        A \cap U = \begin{icloze}{1}A.\end{icloze}
    \]
\end{note}

\begin{note}{4e1167b5fa7748e68b1a4b9a80eaacb3}
    Пусть \({ A }\) и \({ B }\) --- множества.
    \[
        A \begin{icloze}{2}\cup\end{icloze} (A \begin{icloze}{3}\cap\end{icloze} B) = \begin{icloze}{1}A.\end{icloze}
    \]

    \begin{center}
        \tiny
        <<\begin{icloze}{4}Закон поглощения\end{icloze}>>
    \end{center}
\end{note}

\begin{note}{478752160fb94508a605ed54a8601340}
    Пусть \({ A }\) и \({ B }\) --- множества.
    \[
        A \begin{icloze}{2}\cap\end{icloze} (A \begin{icloze}{3}\cup\end{icloze} B) = \begin{icloze}{1}A.\end{icloze}
    \]

    \begin{center}
        \tiny
        <<\begin{icloze}{4}Закон поглощения\end{icloze}>>
    \end{center}
\end{note}

\begin{note}{84569bc3ab574cb78e9bbc9f21dc6bd6}
    Пусть \({ A }\) и \({ B }\) --- множества.
    \[
        A \cap (B \cup \overline{A}) = \begin{icloze}{1}A \cap B.\end{icloze}
    \]
\end{note}

\begin{note}{8c46cf622a9840ba818604b1ddcbd74f}
    Пусть \({ A }\) и \({ B }\) --- множества.
    \[
        A \cup (B \cap \overline{A}) = \begin{icloze}{1}A \cup B.\end{icloze}
    \]
\end{note}

\begin{note}{f391250023de4aefa419991a4de9c8ab}
    Пусть \({ A }\) и \({ B }\) --- множества.
    \[
        (A \cup B) \begin{icloze}{2}\cap\end{icloze} (A \cup \overline{B}) = \begin{icloze}{1}A.\end{icloze}
    \]

    \begin{center}
        \tiny
        <<\begin{icloze}{3}Закон расщепления\end{icloze}>>
    \end{center}
\end{note}

\begin{note}{29ec5d118d8849bea46146efcbbc4473}
    Пусть \({ A }\) и \({ B }\) --- множества.
    \[
        (A \cap B) \begin{icloze}{2}\cup\end{icloze} (A \cap \overline{B}) = \begin{icloze}{1}A.\end{icloze}
    \]

    \begin{center}
        \tiny
        <<\begin{icloze}{3}Закон расщепления\end{icloze}>>
    \end{center}
\end{note}

\begin{note}{cfe43c6f8ac74a43a3f82ea5e01fee7d}
    Пусть \({ A }\) --- множество.
    \[
        \overline{\overline{A}} = \begin{icloze}{1}A.\end{icloze}
    \]
\end{note}

\begin{note}{edcde29726c04401a88af2ef23f3c264}
    Пусть \({ A }\) и \({ B }\) --- множества.
    \[
        A \setminus B = \begin{icloze}{1}A \cap \overline{B}.\end{icloze}
    \]
\end{note}

\begin{note}{86475fdea01944fba56365048d57b02d}
    Пусть \({ A }\) и \({ B }\) --- множества.
    \[
        (A \times B) \cap (B \times A) = \begin{icloze}{2}\emptyset\end{icloze} \quad \begin{icloze}{3}\iff\end{icloze} \quad \begin{icloze}{1}A \cap B = \emptyset.\end{icloze}
    \]
\end{note}

\begin{note}{8ca45754929648bda3ca5496c7cba70f}
    Операция \begin{icloze}{3}декартового произведения\end{icloze} \begin{icloze}{2}дистрибутивна\end{icloze} относительно \begin{icloze}{1}операций \({ \cap }\), \({ \cup }\), \({ \setminus }\), \({ \bigtriangleup }\).\end{icloze}
\end{note}

\begin{note}{ad330727e2cb4c27970e8cb8fdcdeb23}
    Пусть \({ A, B }\) и \({ C }\) --- множества.
    Равны ли множества \({ (A \times B) \times C }\) и \({ A \times (B \times C) }\)?

    \begin{cloze}{1}
        Их отождествляют и считают равными.
    \end{cloze}
\end{note}

\begin{note}{06a0896de5284f44bac5ddff2170cbb1}
    Пусть \({ A }\) и \({ B }\) --- множества. Для \begin{icloze}{2}конечных\end{icloze} множеств,
    \[
        \left\lvert A \times B \right\rvert = \begin{icloze}{1}\left\lvert A \right\rvert \cdot \left\lvert B \right\rvert.\end{icloze}
    \]
\end{note}

\begin{note}{cfc293ce41644e75b3df5d21a2bf036d}
    Пусть \({ A }\) и \({ B }\) --- множества.
    \begin{icloze}{2}Бинарным отношением\end{icloze} на множествах \({ A }\) и \({ B }\) называется \begin{icloze}{1}некоторое подмножество \({ A \times B }\).\end{icloze}
\end{note}

\begin{note}{3ba559fe73cf4c90b5b919ce1a45881a}
    Четыре способа задания бинарных отношений.

    \begin{cloze}{1}
        Перечисление, правило, матрица, граф.
    \end{cloze}
\end{note}

\end{document}
