%! TeX root = ./main.tex
\documentclass[11pt, a5paper]{article}
\usepackage[width=10cm, top=0.5cm, bottom=2cm]{geometry}

\usepackage[T1,T2A]{fontenc}
\usepackage[utf8]{inputenc}
\usepackage[english,russian]{babel}
\usepackage{libertine}

\usepackage{amsmath}
\usepackage{amssymb}
\usepackage{amsthm}
\usepackage{mathrsfs}
\usepackage{framed}
\usepackage{xcolor}

\setlength{\parindent}{0pt}

% Force \pagebreak for every section
\let\oldsection\section
\renewcommand\section{\pagebreak\oldsection}

\renewcommand{\thesection}{}
\renewcommand{\thesubsection}{Note \arabic{subsection}}
\renewcommand{\thesubsubsection}{}
\renewcommand{\theparagraph}{}

\newenvironment{note}[1]{\goodbreak\par\subsection{\hfill \color{lightgray}\tiny #1}}{}
\newenvironment{cloze}[2][\ldots]{\begin{leftbar}}{\end{leftbar}}
\newenvironment{icloze}[2][\ldots]{%
  \ignorespaces\text{\tiny \color{lightgray}\{\{c#2::}\hspace{0pt}%
}{%
  \hspace{0pt}\text{\tiny\color{lightgray}\}\}}\unskip%
}


\begin{document}
\section{Лекция 07.09.22}
\begin{note}{1afcb80707524feb886d294c984a52dc}
    \begin{icloze}{2}Абсолютное значение\end{icloze} мультииндекса \({ \alpha \in \mathbb Z_+^{n} }\) так же называют \begin{icloze}{1}порядком \({ \alpha }\).\end{icloze}
\end{note}

\begin{note}{18494d24db8b401ab85e8094eb880381}
    \begin{icloze}{3}Многочленом \({ n }\) переменных со значениями в \({ \mathbb R^{m} }\)\end{icloze} называется \begin{icloze}{4}отображение\end{icloze} вида
    \[
        x \mapsto \begin{icloze}{1}\sum_{\alpha} c_{\alpha} x^{\alpha},\end{icloze}
    \]
    где \begin{icloze}{2}\({ \left\{ c_{\alpha} \right\} \subset \mathbb R^{m} }\) --- конечное семейство\end{icloze} и \begin{icloze}{5}\({ \alpha \in \mathbb Z_+^{n} }\).\end{icloze}
\end{note}

\begin{note}{3ac8ca4a2feb446b91d36973c81be6c9}
    Пусть \({ p : x \mapsto \sum c_{\alpha}x^{\alpha} }\) --- многочлен.
    Если \begin{icloze}{2}\({ p \not\equiv 0 }\),\end{icloze} то \begin{icloze}{3}степенью\end{icloze} многочлена \({ p }\) называется
    \begin{icloze}{1}
        число
        \[
            \max \left\{ \left\lvert \alpha \right\rvert : c_{\alpha} \neq 0 \right\}.
        \]
    \end{icloze}
\end{note}

\begin{note}{b531da86b4704f8a98fa60c7e92fed4f}
    Пусть \({ p : x \mapsto \sum c_{\alpha}x^{\alpha} }\) --- многочлен.
    Если \begin{icloze}{2}\({ p \equiv 0 }\),\end{icloze} то \begin{icloze}{3}степень\end{icloze} многочлена \({ p }\) полагают равной \begin{icloze}{1}\({ -\infty }\).\end{icloze}
\end{note}

\begin{note}{a810b4eb7a9c412e956ede41dfa9bf20}
    Пусть \({ p : x \mapsto \sum c_{\alpha}x^{\alpha} }\) --- многочлен.
    \begin{icloze}{2}Степень\end{icloze} многочлена \({ p }\) обозначается
    \begin{icloze}{1}
        \[
            \deg p.
        \]
    \end{icloze}
\end{note}

\begin{note}{208b23c3a625454aa756b911bec91ab0}
    Пусть \({ p : x \mapsto \sum c_{\alpha}x^{\alpha} }\) --- многочлен.
    Многочлен \({ p }\) называется \begin{icloze}{2}однородным,\end{icloze} если
    \begin{icloze}{1}
        для всех \({ c_{\alpha} \neq 0 }\)
        \[
            \left\lvert \alpha \right\rvert = \deg p.
        \]
    \end{icloze}
\end{note}

\begin{note}{544930fdea1c4e5d80a0df01959e347d}
    Пусть \({ f : E \subset \mathbb R^{n} \to \mathbb R^{m} }\), \begin{icloze}{4}\({ a \in \operatorname{Int} E }\),\end{icloze} \begin{icloze}{5}\({ s \in \mathbb Z_+ }\).\end{icloze}
    \begin{icloze}{2}Многочлен \({ p }\) степени не выше \({ s }\),\end{icloze} для которого
    \begin{icloze}{1}
        \[
            p(a) = f(a) \quad \text{и} \quad f(x) = p(x) + o(\left\lVert x - a \right\rVert^{s}), \quad x \to a,
        \]
    \end{icloze}
    называется \begin{icloze}{3}многочленом Тейлора \({ f }\) порядка \({ s }\) в точке \({ a }\).\end{icloze}
\end{note}

\begin{note}{eb19d56da526470cb6e9080b543d4274}
    Пусть \({ f : E \subset \mathbb R^{n} \to \mathbb R^{m} }\), \({ a \in \operatorname{Int} E }\), \({ s \in \mathbb Z_+ }\).
    \begin{icloze}{2}Многочлен Тейлора \({ f }\) порядка \({ s }\) в точке \({ a }\)\end{icloze} обозначается
    \begin{icloze}{1}
        \[
            T_{a,s}f.
        \]
    \end{icloze}
\end{note}

\begin{note}{933573807d4c48759570240ceab80b99}
    Пусть \({ f : E \subset \mathbb R^{n} \to \mathbb R^{m} }\), \({ a \in \operatorname{Int} E }\), \({ s \in \mathbb Z_+ }\).
    Если \({ T_{a,s}f }\) существует, то он \begin{icloze}{1}единственный.\end{icloze}
\end{note}

\begin{note}{58c9f6950530458f9675a1dbdf0ada74}
    Пусть \({ f : E \subset \mathbb R^{n} \to \mathbb R^{m} }\), \({ a \in \operatorname{Int} E }\), \({ s \in \mathbb Z_+ }\).
    Если \({ T_{a,s}f }\) существует, то он единственный.
    В чём ключевая идея доказательства?

    \begin{cloze}{1}
        Разность двух многочленов есть \({ o(\left\lVert x - a \right\rVert^{s}) }\), \({ x \to a }\).
    \end{cloze}
\end{note}

\begin{note}{279f256b32fa4f1597e48070542d1328}
    Пусть \({ p }\) --- многочлен \begin{icloze}{2}степени не выше \({ s }\),\end{icloze} \begin{icloze}{3}\({ a \in \mathbb R^{n} }\).\end{icloze}
    Тогда если
    \[
        p(x) = o(\left\lVert x - a \right\rVert^{s}), \quad x \to a,
    \]
    то \begin{icloze}{1}\({ p \equiv 0 }\).\end{icloze}
\end{note}

\begin{note}{f753051ba1584ae3809a7b03ecf311f7}
    Пусть \({ p }\) --- многочлен степени не выше \({ s }\), \({ a \in \mathbb R^{n} }\).
    Тогда если \({ p(x) = o(\left\lVert x - a \right\rVert^{s}) }\) при \({ x \to a }\), то \({ p \equiv 0 }\).
    Каков первый шаг в доказательстве?

    \begin{cloze}{1}
        Рассмотреть два случая: \({ a = 0 }\) и \({ a \neq 0 }\).
    \end{cloze}
\end{note}

\begin{note}{dbbbdf10a7154a108f480966e50f47f4}
    Пусть \({ p }\) --- многочлен степени не выше \({ s }\), \({ a \in \mathbb R^{n} }\).
    Тогда если \({ p(x) = o(\left\lVert x \right\rVert^{s}) }\) при \({ x \to 0 }\), то \({ p \equiv 0 }\).
    В чём ключевая идея доказательства?

    \begin{cloze}{1}
        Разбить \({ p }\) на однородные компоненты и рассмотреть \({ p(tx) }\) как многочлен переменной \({ t }\).
    \end{cloze}
\end{note}

\begin{note}{f5bb46b7a1ed4834958c83c4ad14592b}
    Пусть \({ p }\) --- многочлен степени не выше \({ s }\), \({ a \in \mathbb R^{n} }\).
    Тогда если \({ p(x) = o(\left\lVert x \right\rVert^{s}) }\) при \({ x \to 0 }\), то \({ p \equiv 0 }\).
    Как представляется многочлен \({ p(tx) }\) в доказательстве?

    \begin{cloze}{1}
        \[
            p(tx) = \sum_{k} p_k(x) \cdot t^{k},
        \]
        где \({ p_{k} }\) --- однородный многочлен степени \({ k }\).
    \end{cloze}
\end{note}

\begin{note}{8d6ee9673c3342a08abd86f26262f4d4}
    Пусть \({ p }\) --- многочлен степени не выше \({ s }\), \({ a \in \mathbb R^{n} }\).
    Тогда если \({ p(x) = o(\left\lVert x \right\rVert^{s}) }\) при \({ x \to 0 }\), то \({ p \equiv 0 }\).
    В доказательстве, что нужно показать про многочлен \({ p(tx) }\)?

    \begin{cloze}{1}
        \[
            p(tx) = o(\lvert t \rvert^{s}) \quad \text{при}\ x \to 0.
        \]
    \end{cloze}
\end{note}

\begin{note}{d1c7e28676534f9c9c27d38c3b300a28}
    Пусть \({ p }\) --- многочлен степени не выше \({ s }\), \({ a \in \mathbb R^{n} }\).
    Тогда если \({ p(x) = o(\left\lVert x \right\rVert^{s}) }\) при \({ x \to 0 }\), то \({ p \equiv 0 }\).
    В доказательстве мы получили, что \({ \sum_{k} p_k(x) \cdot t^{k} = o(\lvert t \rvert^{s}) }\) при \({ t \to 0 }\).
    Что дальше?

    \begin{cloze}{1}
        Применить аналогичную теорему к координатным функциям.
    \end{cloze}
\end{note}

\begin{note}{833c8cc496364d6fa95263abe312262d}
    Пусть \({ p }\) --- многочлен степени не выше \({ s }\), \({ a \in \mathbb R^{n} }\).
    Тогда если \({ p(x) = o(\left\lVert x - a \right\rVert^{s}) }\) при \({ x \to a }\), то \({ p \equiv 0 }\).
    В чём ключевая идея доказательства (случай \({ a \neq 0 }\))?

    \begin{cloze}{1}
        \({ p(a + h) = o(\left\lVert h \right\rVert^{s}) }\) при \({ h \to 0 }\).
    \end{cloze}
\end{note}

\begin{note}{6fd36ee18228464ca25e12817347ca1c}
    Пусть \({ f : E \subset \mathbb R^{n} \to \mathbb R^{m} }\), \({ a \in \operatorname{Int} E }\).
    \({ T_{a,0}f }\) существует \begin{icloze}{2}тогда и только тогда, когда\end{icloze} \begin{icloze}{1}\({ f }\) непрерывна в точке \({ a }\).\end{icloze}
\end{note}

\begin{note}{803bd99b5a65458a8e290e6262c9de9d}
    Пусть \({ f : E \subset \mathbb R^{n} \to \mathbb R^{m} }\), \({ a \in \operatorname{Int} E }\).
    \({ T_{a,1}f }\) существует \begin{icloze}{2}тогда и только тогда, когда\end{icloze} \begin{icloze}{1}\({ f }\) дифференцируемо в точке \({ a }\).\end{icloze}
\end{note}

\begin{note}{2f87c61fe7f54f968db50ac94e832bae}
    Пусть \({ f : E \subset \mathbb R^{n} \to \mathbb R^{m} }\) \begin{icloze}{3}\({ s }\) раз дифференцируемо в точке \({ a }\).\end{icloze}
    Тогда если \begin{icloze}{2}\({ f }\) и все его частные производные порядка не выше \({ s }\) равны \({ 0 }\) в точке \({ a }\),\end{icloze} то
    \begin{icloze}{1}
        \[
            f(a + h) = o(\left\lVert h \right\rVert^{s}) \quad \text{при}\ h \to 0.
        \]
    \end{icloze}
\end{note}

\begin{note}{1058f8e8aff94db385633d84438c4915}
    Пусть \({ f : E \subset \mathbb R^{n} \to \mathbb R^{m} }\) \({ s }\) раз дифференцируемо в точке \({ a }\).
    Тогда если \({ f }\) и все его частные производные порядка не выше \({ s }\) равны \({ 0 }\) в точке \({ a }\),
    то \({ f(a + h) = o(\left\lVert h \right\rVert^{s}) }\) при \({ h \to 0 }\).
    Каков первый шаг в доказательстве?

    \begin{cloze}{1}
        Рассмотреть два случая: \({ m = 1 }\) и \({ m > 1 }\).
    \end{cloze}
\end{note}

\begin{note}{5c30a0ff84484046b766901eef5af420}
    Пусть \({ f : E \subset \mathbb R^{n} \to \mathbb R^{m} }\) \({ s }\) раз дифференцируемо в точке \({ a }\).
    Тогда если \({ f }\) и все его частные производные порядка не выше \({ s }\) равны \({ 0 }\) в точке \({ a }\),
    то \({ f(a + h) = o(\left\lVert h \right\rVert^{s}) }\) при \({ h \to 0 }\).
    В чём ключевая идея доказательства (случай \({ m > 1 }\))?

    \begin{cloze}{1}
        Следует из случая \({ m = 1 }\) для координатных функций.
    \end{cloze}
\end{note}

\begin{note}{82636304ac3c484eb96726ccdc702d46}
    Пусть \({ f : E \subset \mathbb R^{n} \to \mathbb R^{m} }\) \({ s }\) раз дифференцируемо в точке \({ a }\).
    Тогда если \({ f }\) и все его частные производные порядка не выше \({ s }\) равны \({ 0 }\) в точке \({ a }\),
    то \({ f(a + h) = o(\left\lVert h \right\rVert^{s}) }\) при \({ h \to 0 }\).
    В чём ключевая идея доказательства (случай \({ m = 1 }\))?

    \begin{cloze}{1}
        Индукция по \({ s }\) начиная с \({ s = 1 }\).
    \end{cloze}
\end{note}

\begin{note}{e2a21df035814a499b4289ae94f9ce3b}
    Пусть \({ f : E \subset \mathbb R^{n} \to \mathbb R^{m} }\) \({ s }\) раз дифференцируемо в точке \({ a }\).
    Тогда если \({ f }\) и все его частные производные порядка не выше \({ s }\) равны \({ 0 }\) в точке \({ a }\),
    то \({ f(a + h) = o(\left\lVert h \right\rVert^{s}) }\) при \({ h \to 0 }\).
    В чём ключевая идея доказательства (случай \({ m = 1 }\), база индукции)?

    \begin{cloze}{1}
        Выразить \({ f(a + h) }\) через дифференциал, а его через производные.
    \end{cloze}
\end{note}

\begin{note}{8470c9b044c34960816bc23c9dacd863}
    Пусть \({ f : E \subset \mathbb R^{n} \to \mathbb R^{m} }\) \({ s }\) раз дифференцируемо в точке \({ a }\).
    Тогда если \({ f }\) и все его частные производные порядка не выше \({ s }\) равны \({ 0 }\) в точке \({ a }\),
    то \({ f(a + h) = o(\left\lVert h \right\rVert^{s}) }\) при \({ h \to 0 }\).
    В чём ключевая идея доказательства (случай \({ m = 1 }\), индукционный переход)?

    \begin{cloze}{1}
        Индукционное предположение для первых частных производных и формула конечных приращений.
    \end{cloze}
\end{note}

\begin{note}{440fc6e6c8f44ad2a31f2846aff7b4a1}
    Пусть \({ f : E \subset \mathbb R^{n} \to \mathbb R^{m} }\) \begin{icloze}{3}\({ s }\) раз дифференцируемо в точке \({ a }\).\end{icloze}
    Тогда
    \[
        \begin{icloze}{2}T_{a,s}f(x)\end{icloze} = \begin{icloze}{1}\sum_{\left\lvert \alpha \right\rvert \leqslant s} \frac{1}{\alpha!} \frac{\partial^{\alpha} f}{\partial x^{\alpha}}(a) (x - a)^{\alpha}.\end{icloze}
    \]

    \begin{center}
        \tiny
        <<\begin{icloze}{4}Формула Тейлора-Пеано\end{icloze}>>
    \end{center}
\end{note}

\begin{note}{2aeb576d5fa547d7bda0b219d979ee26}
    Пусть \({ f : E \subset \mathbb R^{n} \to \mathbb R^{m} }\) \({ s }\) раз дифференцируемо в точке \({ a }\).
    Тогда
    \[
        T_{a,s}f(x) = \sum_{\left\lvert \alpha \right\rvert \leqslant s} \frac{1}{\alpha!} \frac{\partial^{\alpha} f}{\partial x^{\alpha}}(a) (x - a)^{\alpha}.
    \]
    В чём ключевая идея доказательства?

    \begin{cloze}{1}
        \[
            \frac{\partial^{\alpha} (f - p)}{\partial x^{\alpha}} (a) = 0 \quad \text{для}\ \left\lvert \alpha \right\rvert \leqslant s.
        \]
    \end{cloze}
\end{note}

\begin{note}{e0459301f4f34ae58524dc3c38939440}
    Пусть \({ f : E \subset \mathbb R^{n} \to \mathbb R^{m} }\) \begin{icloze}{3}\({ s }\) раз дифференцируемо в точке \({ a }\).\end{icloze}
    Тогда
    \[
        \begin{icloze}{2}T_{a,s}f(x)\end{icloze} = \begin{icloze}{1}\sum_{k = 0}^{s} \frac{d_{a}^{k}f(x - a)}{k!}.\end{icloze}
    \]

    \begin{center}
        \tiny
        (в терминах дифференциалов)
    \end{center}
\end{note}

\begin{note}{caac2b23fb274c30b7cb6175b0f99c2f}
    Пусть \({ p : \mathbb R^{n} \to \mathbb R^{m} }\) --- \begin{icloze}{2}многочлен степени не выше \({ s }\),\end{icloze} \({ a \in \mathbb R^{n} }\).
    Тогда
    \[
        R_{a,s}p(x) = \begin{icloze}{1}0.\end{icloze}
    \]
\end{note}

\begin{note}{4be29edc8e4d451e828ecd8e46049315}
    Пусть \({ a, b \in \mathbb R^{n} }\).
    \[
        \begin{icloze}{2}\widetilde\Delta_{a,b}\end{icloze} \overset{\text{def}}= \begin{icloze}{1}\Delta_{a,b} \setminus \left\{ a, b \right\}.\end{icloze}
    \]
\end{note}

\begin{note}{e167a0bd9f704b6c9c7939124e1af308}
    Пусть \({ f : E \subset \mathbb R^{n} \to \begin{icloze}{4}\mathbb R\end{icloze} }\) \begin{icloze}{6}дифференцируемо \({ s + 1 }\) раз на \({ E }\),\end{icloze} \begin{icloze}{5}\({ a \neq x }\) и \({ \Delta_{a,x} \subset E }\).\end{icloze}
    Тогда \({ \exists c \in \begin{icloze}{2}\widetilde \Delta_{a,x}\end{icloze} }\) для которой
    \[
        \begin{icloze}{3}R_{a,s}f(x)\end{icloze} = \begin{icloze}{1}\frac{d_{c}^{s + 1}f(x - a)}{(s + 1)!}.\end{icloze}
    \]

    \begin{center}
        \tiny
        <<\begin{icloze}{7}Формула Тейлора-Лагранжа\end{icloze}>>
    \end{center}
\end{note}

\begin{note}{41ca37ac01bb45e0a61e5ef62d8970de}
    Пусть \({ f : E \subset \mathbb R^{n} \to \mathbb R }\) дифференцируемо \({ s + 1 }\) раз на \({ E }\), \({ a \neq x }\) и \({ \Delta_{a,x} \subset E }\).
    Тогда \({ \exists c \in \widetilde \Delta_{a,x} }\) для которой
    \[
        R_{a,s}f(x) = \frac{d_{c}^{s + 1}f(x - a)}{(s + 1)!}.
    \]
    В чём ключевая идея доказательства?

    \begin{cloze}{1}
        Одномерная формула Тейлора-Лагранжа для функции
        \[
            t \mapsto f(a + th).
        \]
    \end{cloze}
\end{note}

\begin{note}{f5319b0cf6d14e598448bc9db8842901}
    Пусть \({ f : E \subset \mathbb R^{n} \to \mathbb R }\) дифференцируемо \({ s + 1 }\) раз на \({ E }\), \({ a \neq x }\) и \({ \Delta_{a,x} \subset E }\).
    Тогда
    \[
        \begin{icloze}{2}\left\lvert R_{a,s}f(x) \right\rvert\end{icloze} \leqslant \begin{icloze}{1}\underset{c \in \widetilde\Delta_{a, x}}{\sup} \frac{\left\lvert d_{c}^{s + 1}f(x - a) \right\rvert}{(s + 1)!}.\end{icloze}
    \]
\end{note}

\section{Лекция 14.09.22}
\begin{note}{5bfe3eea62cf4923be0b768ada48f104}
    Пусть \({ f : E \subset \mathbb R^{n} \to \mathbb R }\) \begin{icloze}{5}дифференцируема на \({ E }\),\end{icloze} \begin{icloze}{4}\({ a \neq b }\) и \({ \Delta_{a,b} \subset E }\).\end{icloze}
    Тогда \({ \exists c \in \begin{icloze}{2}\widetilde\Delta_{a,b}\end{icloze} }\), для которой
    \begin{icloze}{1}
        \[
            f(b) - f(a) = d_{c}f(b - a).
        \]
    \end{icloze}

    \begin{center}
        \tiny
        <<\begin{icloze}{3}Теорема о среднем\end{icloze}>>
    \end{center}
\end{note}

\begin{note}{43e52abb706d4b2490ad6248608ac691}
    В чём ключевая идея доказательства теоремы о среднем для функций \({ n }\) вещественных переменных?

    \begin{cloze}{1}
        Формула Тейлора-Лагранжа для многочлена Тейлора степени \({ 0 }\).
    \end{cloze}
\end{note}

\begin{note}{4772fe28cbb6493b9863306e7b371ceb}
    Верна ли формула Тейлора-Лагранжа для \textbf{отображений} нескольких переменных?

    \begin{cloze}{1}
        Нет, только для функций.
    \end{cloze}
\end{note}

\begin{note}{e3d62c0a50b24f098a41c202d267ca75}
    Пример отображения \({ \mathbb R \to \mathbb R^2 }\), для которого не верна формула Тейлора-Лагранжа.

    \begin{cloze}{1}
        \({ x \mapsto (\cos x, \sin x) }\),\: \({ a = 0 }\),\: \({ b = 2\pi }\).
    \end{cloze}
\end{note}

\begin{note}{26f3582d133f4c2894e1a02890184d65}
    Пусть \({ f : E \subset \mathbb R^{n} \to \begin{icloze}{5}\mathbb R^{m}\end{icloze} }\) дифференцируемо \({ s+1 }\) раз на \({ E }\), \({ a \neq x }\) и \({ \Delta_{a,x} \subset E }\).
    Тогда
    \[
        \begin{icloze}{2}\left\lVert R_{a,s}f(x) \right\rVert\end{icloze} \begin{icloze}{3}\leqslant\end{icloze} \begin{icloze}{1}\frac{1}{(s + 1)!} \cdot \sup_{c \in \widetilde \Delta_{a,x}} \left\lVert d_{c}^{s + 1}f(x - a) \right\rVert.\end{icloze}
    \]

    \begin{center}
        \tiny
        <<\begin{icloze}{4}Формула Тейлора-Лагранжа для отображений\end{icloze}>>
    \end{center}
\end{note}

\begin{note}{a15c2119bc5c43b38389896cea1098b3}
    Пусть \({ f : E \subset \begin{icloze}{5}\mathbb R^{n}\end{icloze} \to \begin{icloze}{5}\mathbb R^{m}\end{icloze} }\) \begin{icloze}{4}непрерывно дифференцируемо \({ s+1 }\) раз на \({ E }\),\end{icloze} \({ a \neq x }\) и \({ \Delta_{a, x} \subset E }\).
    Тогда
    \[
        \begin{icloze}{3}\left\lVert R_{a,s}f(x) \right\rVert\end{icloze} \leqslant \begin{icloze}{2}\frac{M}{(s + 1)!}(\sqrt{n}\left\lVert x - a \right\rVert)^{s+1},\end{icloze}
    \]
    где
    \[
        M = \begin{icloze}{1}\max_{\left\lvert \alpha \right\rvert = s+1} \sup_{c \in \widetilde \Delta_{a,x}} \left\lVert \partial^{\alpha}f(c) \right\rVert\end{icloze} < \begin{icloze}{6}+\infty.\end{icloze}
    \]

    \begin{center}
        \tiny
        (\({ \alpha \in \mathbb Z_{+}^{n} }\))
    \end{center}
\end{note}

\begin{note}{785a0606f4ba4f8982e917a94bd795a4}
    Будем записывать элементы \({ \mathbb R^{n+m} }\) в виде \begin{icloze}{1}\({ (x, y) }\),\end{icloze} где \({ x \in \begin{icloze}{2}\mathbb R^{n}\end{icloze} }\), \({ y \in \begin{icloze}{2}\mathbb R^{m}\end{icloze} }\).
\end{note}

\begin{note}{0dc937ef2a384d4187a878487bec114b}
    Пусть \({ f : E \subset \begin{icloze}{4}\mathbb R^{n+m}\end{icloze} \to \begin{icloze}{4}\mathbb R^{l}\end{icloze} }\) и существует \({ \psi : \begin{icloze}{3}\mathbb R^{n}\end{icloze} \to \begin{icloze}{3}\mathbb R^{m}\end{icloze} }\) такая, что
    \begin{icloze}{2}
        \[
            f(x, y) = 0 \iff y = \psi(x),
        \]
    \end{icloze}
    то \({ \psi }\) называют \begin{icloze}{1}неявным отображением, порождённым уравнением \({ f(x, y) = 0 }\).\end{icloze}
\end{note}

\begin{note}{142e9f4e8b2b46f2adbaafa498f6208b}
    Пусть \({ f : E \subset \mathbb R^{n + m} \to \mathbb R^{m} }\) дифференцируемо в точке \({ a }\).
    Тогда в контексте теоремы о неявном отображении, порождённом уравнением \({ f(x, y) = 0 }\),
    \[
        f'_{x}(a) \coloneqq
        \begin{icloze}{1}
            \begin{bmatrix}
                \frac{\partial f_i}{\partial x_j}(a)
            \end{bmatrix}
        \end{icloze}
        \sim \begin{icloze}{2}m \times n.\end{icloze}
    \]
\end{note}

\begin{note}{c673a237b83846babd512f854f96bb58}
    Пусть \({ f : E \subset \mathbb R^{n + m} \to \mathbb R^{m} }\) дифференцируемо в точке \({ a }\).
    Тогда в контексте теоремы о неявном отображении, порождённом уравнением \({ f(x, y) = 0 }\),
    \[
        f'_{y}(a) \coloneqq
        \begin{icloze}{1}
            \begin{bmatrix}
                \frac{\partial f_i}{\partial y_j}(a)
            \end{bmatrix}
        \end{icloze}
        \sim \begin{icloze}{2}m \times m.\end{icloze}
    \]
\end{note}

\begin{note}{0d54b2d03b084afebb7c5cc072280a39}
    Пусть \({ f : E \subset \begin{icloze}{4}\mathbb R^{n+m}\end{icloze} \to \begin{icloze}{4}\mathbb R^{m}\end{icloze} }\), \begin{icloze}{5}\({ f \in C^{s}(E) }\).\end{icloze}
    Тогда если
    \begin{icloze}{1}
        \[
            f(x_{0}, y_{0}) = 0 \quad \text{и} \quad \det f'_{y}(x^{0}, y^{0}) \neq 0,
        \]
    \end{icloze}
    то существуют такие \begin{icloze}{2}\({ \delta > 0 }\) и \({ \psi \in C^{s}(V_{\delta}(x^0)) }\),\end{icloze} что
    \[
        \begin{gathered}
            \begin{icloze}{3}f(x, y) = 0 \iff y = \psi(x)\end{icloze} \\
            \forall (x, y) \in \begin{icloze}{6}V_{\delta}(x_0, y_0).\end{icloze}
        \end{gathered}
    \]

    \begin{center}
        \tiny
        <<\begin{icloze}{7}Теорема о неявном отображении\end{icloze}>>
    \end{center}
\end{note}

\begin{note}{5891b9dc6bf94339a65fafd5858d8186}
    Отображение \({ \psi }\), введённое в теореме о неявном отображении называется \begin{icloze}{1}неявным отображением, порождённым уравнением \({ f(x, y) = 0 }\) в окрестности точки \({ (x^{0}, y^{0}) }\).\end{icloze}
\end{note}

\begin{note}{d2223a858b21419e921d5878682e3338}
    В чём основная идея доказательства теоремы о неявном отображении (интуитивно)?

    \begin{cloze}{1}
        \({ d_{a}f(x - x^{0}, y - y^{0}) = o(\left\lVert h \right\rVert) \implies d_{a}f(\ldots) = 0 }\).
    \end{cloze}
\end{note}

\begin{note}{dd7bb23f22a042dab6dacf623aca0455}
    Пусть \({ f : E \subset \mathbb R^{n+m} \to \mathbb R^{m} }\),\: \({ (x^{0}, y^{0}) \in E }\).
    Если
    \begin{icloze}{2}
        в окрестности точки \({ (x^0, y^0) }\) уравнение \({ f(x, y) = 0 }\) порождает неявную функцию,
    \end{icloze}
    то \({ f }\) называется \begin{icloze}{1}локально разрешимым в точке \({ (x^0, y^0) }\).\end{icloze}
\end{note}

\begin{note}{0e4cd0362fdd4f31a10ea88deaf016fc}
    Пусть \({ f : E \subset \mathbb R^{n+m} \to \mathbb R^{m} }\).
    Если \begin{icloze}{2}\({ f }\) локально разрешимо в любой точке \({ E }\),\end{icloze} то \({ f }\) называется \begin{icloze}{1}локально разрешимым на \({ E }\).\end{icloze}
\end{note}

\section{Лекция 21.09.22}
\begin{note}{434affe9da3e443d839f4a7d6af1b180}
    Чем определение экстремума для функций \({ n }\) вещественных переменных отличается от такового для одномерных функций?

    \begin{cloze}{1}
        Ничем.
    \end{cloze}
\end{note}

\begin{note}{2efcc79ba51544848aacbd7da7dfa4b1}
    В определении экстремума функции \({ n }\) вещественных переменных, для каких \({ x }\) требуется выполнение соответствующего неравенства?

    \begin{cloze}{1}
        \({ \forall x \in \dot V_{\delta}(a) \cap D(f) }\).
    \end{cloze}
\end{note}

\begin{note}{30ab720865e54af19fcc351cfa78d165}
    Пусть \({ f : E \subset \mathbb R^{n} \to \mathbb R^{m} }\) \begin{icloze}{3}дифференцируема в точке \({ a }\).\end{icloze}
    Точка \({ a }\) называется \begin{icloze}{2}седловой,\end{icloze} если \begin{icloze}{1}\({ d_{a}f \equiv 0 }\) и \({ a }\) --- не экстремум.\end{icloze}
\end{note}

\begin{note}{27a6b8777b9b4bb9919dcfa31d963432}
    \begin{icloze}{2}Тождественно нулевой оператор\end{icloze} обозначается \begin{icloze}{1}\({ \mathbb O }\).\end{icloze}
\end{note}

\begin{note}{43fd5e0b1a8b410a887c991151fac7e0}
    Пусть \({ f : E \subset \begin{icloze}{4}\mathbb R^{n}\end{icloze} \to \begin{icloze}{4}\mathbb R\end{icloze} }\) \begin{icloze}{3}дифференцируема в точке \({ a }\).\end{icloze}
    Если \begin{icloze}{2}\({ a }\) является точкой экстремума,\end{icloze} то \begin{icloze}{1}\({ d_{a}f \equiv 0 }\).\end{icloze}
\end{note}

\begin{note}{834aecf3894642098d0ddc0256091117}
    Пусть \({ f : E \subset \mathbb R^{n} \to \mathbb R }\) дифференцируема в точке \({ a }\).
    Если \({ a }\) является точкой экстремума, то \({ d_{a}f \equiv 0 }\).

    \begin{center}
        \tiny
        <<\begin{icloze}{1}Необходимое условие экстремума\end{icloze}>>
    \end{center}
\end{note}

\begin{note}{e5cfde8512814dbbaed4676461636b0d}
    Пусть \({ f : E \subset \mathbb R^{n} \to \mathbb R }\) \begin{icloze}{3}дифференцируема в точке \({ a }\).\end{icloze}
    Точку \({ a }\) называют \begin{icloze}{2}стационарной для \({ f }\),\end{icloze} если \begin{icloze}{1}\({ d_{a}f \equiv 0 }\).\end{icloze}
\end{note}

\begin{note}{c658e89677264e7bb7317334e026304e}
    В чём ключевая идея доказательства необходимого условия экстремума для функций \({ n }\) вещественных переменных?

    \begin{cloze}{1}
        Рассмотреть функции \({ t \mapsto f(a + te^{k}) }\).
    \end{cloze}
\end{note}

\begin{note}{02b9316ee1f84262ae08eebecbe71c2b}
    В доказательстве необходимого условия экстремума для функций \({ n }\) вещественных переменных, мы положили
    \[
        F_k(t) = f(a + te^{k}).
    \]
    Что нужно показать про функцию \({ F_k }\)?

    \begin{cloze}{1}
        \({ 0 }\) --- точка экстремума \({ F_k }\) и рассмотреть \({ F_k'(0) }\).
    \end{cloze}
\end{note}

\begin{note}{ec6b1873ffe44cc48c87520ef5bf2f10}
    Какие случаи не охватываются необходимым условием экстремума?

    \begin{cloze}{1}
        Когда дифференциал сохраняет знак, но не является положительно или отрицательно определённым.
    \end{cloze}
\end{note}

\begin{note}{091dad1735594dfa8923e46c5a461ab0}
    Пусть \({ f : E \subset \mathbb R^{n} \to \mathbb R }\) \begin{icloze}{4}дважды дифференцируема в точке \({ a }\),\end{icloze} \begin{icloze}{3}\({ d_{a}f \equiv 0 }\).\end{icloze}
    Тогда если \begin{icloze}{2}\({ d_{a}^2f }\) положительно определён,\end{icloze} то \begin{icloze}{1}\({ a }\) --- точка строгого минимума \({ f }\).\end{icloze}
\end{note}

\begin{note}{a87f168ae18d4f65b7595866b9407e8d}
    Пусть \({ f : E \subset \mathbb R^{n} \to \mathbb R }\) дважды дифференцируема в точке \({ a }\), \({ d_{a}f \equiv 0 }\).
    Тогда если \begin{icloze}{2}\({ d_{a}^2f }\) отрицательно определён,\end{icloze} то \begin{icloze}{1}\({ a }\) --- точка строгого максимума \({ f }\).\end{icloze}
\end{note}

\begin{note}{4643a84f76b84c9bbc34e220ee2196c3}
    Пусть \({ f : E \subset \mathbb R^{n} \to \mathbb R }\) дважды дифференцируема в точке \({ a }\), \({ d_{a}f \equiv 0 }\).
    Тогда если \begin{icloze}{2}\({ d_{a}^2f }\) принимает как положительные, так и отрицательные значения,\end{icloze} то \begin{icloze}{1}\({ f }\) не имеет экстремума в точке \({ a }\).\end{icloze}
\end{note}

\begin{note}{41fccd97c432427e893b7c1e3b850374}
    Пусть \({ f : E \subset \mathbb R^{n} \to \begin{icloze}{4}\mathbb R\end{icloze} }\) \begin{icloze}{3}дважды дифференцируема в точке \({ a }\).\end{icloze}
    \begin{icloze}{2}Матрица квадратичной формы \({ d_{a}^2f }\)\end{icloze} называется \begin{icloze}{1}матрицей Гессе \({ f }\) в точке \({ a }\).\end{icloze}
\end{note}

\begin{note}{36db8ae7809a4dcbb6f350f51ca0a54c}
    Пусть \({ f : E \subset \mathbb R^{n} \to \mathbb R }\) дважды дифференцируема в точке \({ a }\).
    \begin{icloze}{2}Матрица Гессе \({ f }\) в точке \({ a }\)\end{icloze} обозначается \begin{icloze}{1}\({ H(f) }\).\end{icloze}
\end{note}

\begin{note}{5ffaf8568eaa4a8fbb3815c215d20c25}
    Пусть \({ f : E \subset \mathbb R^{n} \to \mathbb R }\) дважды дифференцируема в точке \({ a }\).
    \begin{icloze}{2}Матрица Гессе \({ f }\) в точке \({ a }\)\end{icloze} имеет вид
    \[
        \begin{icloze}{1}
            \begin{bmatrix}
                \frac{\partial^2 f}{\partial x_i \partial x_j}(a)
            \end{bmatrix}
            \sim n \times n.
        \end{icloze}
    \]
\end{note}

\begin{note}{56bf1e23f53c4cae8d62101295cb0a9b}
    Пусть \({ f : E \subset \begin{icloze}{3}\mathbb R^{2}\end{icloze} \to \begin{icloze}{3}\mathbb R\end{icloze} }\) дважды дифференцируема в точке \({ a }\), \begin{icloze}{4}\({ d_{a}f \equiv 0 }\).\end{icloze}
    Тогда если \begin{icloze}{2}\({ \det H(f) < 0 }\),\end{icloze} то \begin{icloze}{1}\({ f }\) не имеет экстремума в точке \({ a }\).\end{icloze}

    \begin{center}
        \tiny
        (в терминах \({ H(f) }\))
    \end{center}
\end{note}

\begin{note}{b78c95a2236a4edf964b8e2a29238dae}
    Пусть \({ f : E \subset \mathbb R^{2} \to \mathbb R }\) дважды дифференцируема в точке \({ a }\), \({ d_{a}f \equiv 0 }\).
    Тогда если \({ \det H(f) < 0 }\), то \({ f }\) не имеет экстремума в точке \({ a }\).
    В чём ключевая идея доказательства?

    \begin{cloze}{1}
        Зафиксировать одну компоненту приращения \({ d_{a}^2f }\) и показать, что он принимает как положительные, так и отрицательные значения.
    \end{cloze}
\end{note}

\begin{note}{535062161d6044bdb41631a3ae479223}
    \begin{icloze}{2}Дополнительные равенства, которым должны удовлетворять точки из области определения \({ f }\)\end{icloze} в определении понятия условного экстремума называются \begin{icloze}{1}уравнениями связи.\end{icloze}
\end{note}

\begin{note}{7dc4fe4c435946a091efc32d6a120ec1}
    Пусть \({ f : E \subset \mathbb R^{n} \to \mathbb R }\), \begin{icloze}{4}\({ \Phi : E \to \mathbb R^{m} }\),\end{icloze} \({ a \in E }\), \begin{icloze}{3}\({ m < n }\).\end{icloze}
    Если
    \begin{icloze}{1}
        \({ a }\) является точкой экстремума сужения \({ f }\) на множество
        \[
            \left\{ x \in E \mid \Phi(x) = \Phi(a) \right\},
        \]
    \end{icloze}
    то \({ a }\) называется \begin{icloze}{2}точкой условного экстремума \({ f }\), подчинённого уравнениям связи \({ \Phi(x) = \Phi(a) }\).\end{icloze}
\end{note}

\begin{note}{efdc5b1690a340499b26b776ada24e98}
    Точку условного экстремума так же называют \begin{icloze}{1}точкой относительного экстремума.\end{icloze}
\end{note}

\begin{note}{9f8b54ca8ee64e32987bdb713c25e8d7}
    В определении условного экстремума не умаляя общности считают, что
    \[
        \begin{icloze}{2}\Phi(a)\end{icloze} = \begin{icloze}{1}0.\end{icloze}
    \]
\end{note}

\begin{note}{7866e63fde9849e0aaf93125f6796f8c}
    Почему в определении условного экстремума можно не умаляя общности считать, что \({ \Phi(a) = 0 }\)?

    \begin{cloze}{1}
        Всегда можно рассмотреть \({ \Phi^* : x \mapsto \Phi(x) - \Phi(a) }\).
    \end{cloze}
\end{note}

\begin{note}{fce85d7d2d244b4daaa0334b43719a92}
    В определении условного экстремума не умаляя общности считают, что \begin{icloze}{2}\({ \Phi'_y(a) }\)\end{icloze} \begin{icloze}{1}невырождена.\end{icloze}
\end{note}

\begin{note}{b53cabc0b0a24f6ead9453c041ad560d}
    Почему в определении условного экстремума можно не умаляя общности считать, что \({ \Phi'_y(a) }\) невырождена?

    \begin{cloze}{1}
        Иначе какое-то из условий следует из остальные.
    \end{cloze}
\end{note}

\section{Лекция 28.09.22}
\begin{note}{ffe18c64640a4c0c9009a7e054fc1af5}
    Пусть \({ E \subset \mathbb R^{n} }\), \begin{icloze}{6}\({ m < n }\),\end{icloze} \({ f \in \begin{icloze}{4}C^{1}(E, \mathbb R)\end{icloze} }\), \({ \Phi \in \begin{icloze}{4}C^{1}(E, \mathbb R^{m})\end{icloze} }\), \({ a \in E }\), \begin{icloze}{5}\({ \operatorname{rk} \Phi'(a) = m }\).\end{icloze}
    Тогда если \begin{icloze}{3}\({ a }\) --- условный экстремум \({ f }\), подчинённый \({ \Phi(x) = 0 }\),\end{icloze} то \begin{icloze}{2}\({ \exists \lambda_1, \ldots \lambda_m \in \mathbb R }\),\end{icloze} для которых
    \begin{icloze}{1}
        \[
            \nabla f(a) = \sum_{k} \lambda_k \cdot \nabla \Phi_k(a).
        \]
    \end{icloze}
\end{note}

\begin{note}{af9f952d67cd4db8a2bc302521d5f590}
    Пусть \({ E \subset \mathbb R^{n} }\), \({ m < n }\), \({ f \in C^{1}(E, \mathbb R) }\), \({ \Phi \in C^{1}(E, \mathbb R^{m}) }\), \({ a \in E }\), \({ \operatorname{rk} \Phi'(a) = m }\).
    Тогда если \({ a }\) --- условный экстремум \({ f }\), подчинённый \({ \Phi(x) = 0 }\), то \({ \exists \lambda_1, \ldots \lambda_m \in \mathbb R }\), для которых
    \[
        \nabla f(a) = \sum_{k} \lambda_k \cdot \nabla \Phi_k(a).
    \]

    \begin{center}
        \tiny
        <<\begin{icloze}{1}Необходимое условие относительного экстремума\end{icloze}>>
    \end{center}
\end{note}

\begin{note}{82213d3b2284465e9b85ecf7bcfe686b}
    \begin{icloze}{2}Коэффициенты \({ \lambda_1, \ldots, \lambda_m }\)\end{icloze} из теоремы о \begin{icloze}{3}необходимом условии относительного экстремума\end{icloze} называются \begin{icloze}{1}множителями Лагранжа \({ f }\) в точке \({ a }\).\end{icloze}
\end{note}

\begin{note}{832a0c560cf747c492a100c673d806cc}
    В чём ключевая идея доказательства необходимого условия относительного экстремума?

    \begin{cloze}{1}
        Теорема о неявной функции для \({ \Phi(x, y) = 0 }\).
    \end{cloze}
\end{note}

\begin{note}{9434998277aa4d2d946b7fcdfecf36f8}
    В доказательстве необходимого условия относительного экстремума мы построили неявную функцию \({ \psi }\) из уравнения \({ \Phi(x, y) = 0 }\).
    Что дальше?

    \begin{cloze}{1}
        Рассмотреть \({ f(x, \psi(x)) - \lambda \Phi(x, \psi(x)) }\).
    \end{cloze}
\end{note}

\begin{note}{feb9eccc46494af296a7c929c6b4fd57}
    В доказательстве необходимого условия относительного экстремума
    \[
        f(x, \psi(x))'_{x} \Big|_{\begin{icloze}{2}x^{0}\end{icloze}} = \begin{icloze}{1}0.\end{icloze}
    \]
\end{note}

\begin{note}{038381f4bf64457da1a04c86b8ab0eb5}
    Почему в доказательстве необходимого условия относительного экстремума
    \[
        f(x, \psi(x))'_{x} \Big|_{x^{0}} = 0 \quad \text{?}
    \]

    \begin{cloze}{1}
        \({ x_{0} }\) --- т. экстремума \({ f(x, \psi(x)) }\).
    \end{cloze}
\end{note}

\begin{note}{63892565775440a9afc699c3c9ed5419}
    В доказательстве необходимого условия относительного экстремума
    \[
        \Phi(x, \psi(x))'_{x} \Big|_{\begin{icloze}{2}x^{0}\end{icloze}} = \begin{icloze}{1}0.\end{icloze}
    \]
\end{note}

\begin{note}{b333ed4af4a44349b84e8aed99b46a18}
    Почему в доказательстве необходимого условия относительного экстремума
    \[
        \Phi(x, \psi(x))'_{x} \Big|_{x^{0}} = 0 \quad \text{?}
    \]

    \begin{cloze}{1}
        \({ \Phi(x, \psi(x)) = 0 }\) в окрестности \({ x^{0} }\).
    \end{cloze}
\end{note}

\begin{note}{082928b830ed48038502597330cc2d6b}
    Чему равно \({ \lambda }\) из теоремы о необходимом условии относительного экстремума?

    \begin{cloze}{1}
        \({ \lambda^{T} = f'_{y}(a) \cdot \left( \Phi'_{y}(a) \right)^{-1} }\).
    \end{cloze}
\end{note}

\begin{note}{c403eaa3be2d4d1dac52b422443e29b2}
    В условиях теоремы о необходимом условии относительного экстремума отображение
    \[
        \begin{icloze}{3}(x, \lambda)\end{icloze} \mapsto \begin{icloze}{2}f(x) - \sum_{k=1}^{m} \lambda_k \Phi_k(x)\end{icloze}
    \]
    называется \begin{icloze}{1}функцией Лагранжа.\end{icloze}
\end{note}

\begin{note}{5568e5ccac324a0cb09e98b4ba65503b}
    В условиях теоремы о необходимом условии относительного экстремума \begin{icloze}{2}функция Лагранжа\end{icloze} обозначается \begin{icloze}{1}\({ L(x, \lambda) }\).\end{icloze}
\end{note}

\begin{note}{113bd38c503e4af990386a637b1aa830}
    Необходимое условие относительного экстремума в терминах функции Лагранжа примет вид
    \begin{icloze}{1}
        \[
            \exists \lambda \in \mathbb R^{m} \quad \nabla L(a, \lambda) = 0.
        \]
    \end{icloze}
\end{note}

\begin{note}{10647bd11e764a13b935dd15418841c6}
    Пусть \({ N \triangleleft \mathbb R^{n} }\), \({ f }\) --- квадратичная форма в \({ \mathbb R^{n} }\).
    \({ f }\) называется \begin{icloze}{2}положительно определённой на \({ N }\),\end{icloze} если \begin{icloze}{1}\({ f|_{N} }\) положительна определена.\end{icloze}
\end{note}

\begin{note}{24a7c3a60b0144b78d3bdf59396178e9}
    Пусть \({ N \triangleleft \mathbb R^{n} }\), \({ f }\) --- квадратичная форма в \({ \mathbb R^{n} }\).
    \({ f }\) называется \begin{icloze}{2}отрицательно определённой на \({ N }\),\end{icloze} если \begin{icloze}{1}\({ f|_{N} }\) отрицательно определена.\end{icloze}
\end{note}

\begin{note}{6af1663b3c2f432eb8d779d75937d630}
    Пусть в условиях теоремы \begin{icloze}{6}о необходимом условии относительного экстремума\end{icloze} \({ f \in \begin{icloze}{5}C^{2}(E)\end{icloze} }\) и \({ g \in \begin{icloze}{5}C^{2}(E)\end{icloze} }\)
    Положим \({ L(x) = \begin{icloze}{4}L(x, \lambda)\end{icloze} }\), где \({ \lambda }\) --- \begin{icloze}{3}множители Лагранжа.\end{icloze}
    Тогда если \begin{icloze}{1}\({ d_{a}^2L }\) положительно определён на \({ \ker d_{a}\Phi }\),\end{icloze} то \begin{icloze}{2}\({ a }\) --- точка условного минимума, подчинённая \({ \Phi(x) = \Phi(a) }\).\end{icloze}
\end{note}

\begin{note}{1abdb12e039f42bb991a4e08218e2bc0}
    Пусть в условиях теоремы о необходимом условии относительного экстремума \({ f \in C^{2}(E) }\) и \({ g \in C^{2}(E) }\)
    Положим \({ L(x) = L(x, \lambda) }\), где \({ \lambda }\) --- множители Лагранжа.
    Тогда если \begin{icloze}{1}\({ d_{a}^2L }\) отрицательно определён на \({ \ker d_{a}\Phi }\),\end{icloze} то \begin{icloze}{2}\({ a }\) --- точка условного максимума, подчинённая \({ \Phi(x) = \Phi(a) }\).\end{icloze}
\end{note}

\begin{note}{18275ac96f4145228cd295901a3c4aeb}
    Пусть в условиях теоремы о необходимом условии относительного экстремума \({ f \in C^{2}(E) }\) и \({ g \in C^{2}(E) }\)
    Положим \({ L(x) = L(x, \lambda) }\), где \({ \lambda }\) --- множители Лагранжа.
    Тогда если \begin{icloze}{1}\({ d_{a}^2 L }\) принимает на \({ \ker d_{a}\Phi }\) как положительные, так и отрицательные значения,\end{icloze} то \begin{icloze}{2}\({ f }\) не имеет в точке \({ a }\) условного экстремума, подчинённого \({ \Phi(x) = \Phi(a) }\).\end{icloze}
\end{note}

\begin{note}{b5b66b7db90b4d1bacf06b32b1446fcb}
    Пусть \begin{icloze}{3}\({ \left\{ a_k \right\}_{k=1}^{\infty} }\) --- вещественная последовательность.\end{icloze}
    \begin{icloze}{2}Рядом\end{icloze} называется
    \begin{icloze}{1}
        формальное выражение вида
        \[
            \sum_{k=1}^{\infty} a_k = a_1 + a_2 + a_3 + \cdots.
        \]
    \end{icloze}
\end{note}

\begin{note}{a5592c04ea8d40cd952f2bf6391ed438}
    Пусть \({ \sum_{k=1}^{\infty} a_k }\) --- вещественный ряд.
    \begin{icloze}{2}Элемент \({ a_k }\)\end{icloze} называется \begin{icloze}{1}общим членом ряда.\end{icloze}
\end{note}

\begin{note}{ff62bd61f29a475eb73a358793f8a46a}
    Пусть \({ \sum_{k=1}^{\infty} a_k }\) --- вещественный ряд, \begin{icloze}{3}\({ n \in \mathbb N }\).\end{icloze}
    \begin{icloze}{2}
        Сумму вида
        \[
            a_1 + a_2 + \cdots + a_n
        \]
    \end{icloze}
    называют \begin{icloze}{1}частичной суммой ряда.\end{icloze}
\end{note}

\begin{note}{515ccf52bf2e4d89a5a2c282984860f2}
    Пусть \({ \sum_{k=1}^{\infty} a_k }\) --- вещественный ряд, \({ n \in \mathbb N }\).
    \begin{icloze}{1}
        Сумму
        \[
            a_1 + a_3 + \cdots + a_n
        \]
    \end{icloze}
    часто обозначают \begin{icloze}{2}\({ S_n }\).\end{icloze}
\end{note}

\begin{note}{ad2ee0b419444746bd51026b07fa19d6}
    Пусть \({ \sum_{k=1}^{\infty} a_k }\) --- вещественный ряд.
    \begin{icloze}{1}
        Величина
        \[
            \lim_{n \to \infty} S_n
        \]
    \end{icloze}
    называется \begin{icloze}{2}суммой ряда \({ \sum_{k=1}^{\infty} a_k }\).\end{icloze}
\end{note}

\begin{note}{6505e18ad4754436aa6c22aae25686bd}
    Пусть \({ \sum_{k=1}^{\infty} a_k }\) --- вещественный ряд.
    Если \begin{icloze}{2}величина \({ A }\) есть сумма ряда \({ \sum_{k=1}^{\infty} a_k }\),\end{icloze} то пишут
    \begin{icloze}{1}
        \[
            \sum_{k=1}^{\infty} a_k = A.
        \]
    \end{icloze}
\end{note}

\begin{note}{a3c5b50440ad4896bb8af7492d946bc5}
    Пусть \({ \sum_{k=1}^{\infty} a_k }\) --- вещественный ряд.
    Говорят, что ряд \begin{icloze}{2}сходится,\end{icloze} если \begin{icloze}{1}его сумма существует и конечна.\end{icloze}
\end{note}

\begin{note}{fb03f6a351984ebf8a7077cc5c2126e4}
    Пусть \({ \sum_{k=1}^{\infty} a_k }\) --- вещественный ряд.
    Говорят, что ряд \begin{icloze}{2}расходится,\end{icloze} если \begin{icloze}{1}он не сходится.\end{icloze}
\end{note}

\section{Семинар 05.10.22}
\begin{note}{80570e9f813648cfa0315b3dd7e327a3}
    В условиях теоремы о достаточном условии относительного экстремума, как определяется знакоопределённость \({ d^2_{a}L }\) на \({ \ker d_{a} \Phi }\)?

    \begin{cloze}{1}
        Через миноры матрицы
        \[
            \begin{bmatrix}
                0 & \Phi' \\
                (\Phi')^{T} & H(L)
            \end{bmatrix}.
        \]
    \end{cloze}
\end{note}

\begin{note}{66548b31237c4009a91c2423d6836660}
    В условиях теоремы о достаточном условии относительного экстремума, какие миноры матрицы
    \[
        \begin{bmatrix}
            0 & \Phi' \\
            (\Phi')^{T} & H(L)
        \end{bmatrix}.
    \]
    рассматриваются для определения знакоопределённости \({ d^2_{a}L }\) на \({ \ker d_{a} \Phi }\)?

    \begin{cloze}{1}
        Начиная с \({ \Delta_{2m + 1} }\). \quad (\({ m }\) --- количество уравнений связи.)
    \end{cloze}
\end{note}

\begin{note}{ae3633661bf248a389c6fc301e701039}
    В условиях теоремы о достаточном условии относительного экстремума, какими должны быть рассматриваемые миноры матрицы
    \[
        \begin{bmatrix}
            0 & \Phi' \\
            (\Phi')^{T} & H(L)
        \end{bmatrix}.
    \]
    чтобы \({ a }\) была точкой условного минимума?

    \begin{cloze}{1}
        Все имеют знак \({ (-1)^{m} }\).
    \end{cloze}
\end{note}

\begin{note}{8d4f2a4f84f54fc4921ae524773400b4}
    В условиях теоремы о достаточном условии относительного экстремума, какими должны быть рассматриваемые миноры матрицы
    \[
        \begin{bmatrix}
            0 & \Phi' \\
            (\Phi')^{T} & H(L)
        \end{bmatrix}.
    \]
    чтобы \({ a }\) была точкой условного максимума?

    \begin{cloze}{1}
        Первый имеет знак \({ (-1)^{m + 1} }\) и далее чередуются.
    \end{cloze}
\end{note}

\section{Лекция 26.10.22}
\begin{note}{f86d2dd9e8f64a59a96d4d321a706cc4}
    Как определяется переместительное свойство ряда (интуитивно)?

    \begin{cloze}{1}
        Перестановка элементов не влияет на сумму.
    \end{cloze}
\end{note}

\begin{note}{eda4110756264d5885252e3c548ad9ac}
    Применимо ли переместительное свойство к расходящимся рядам?

    \begin{cloze}{1}
        Да.
    \end{cloze}
\end{note}

\begin{note}{f0d78b12e0654f259e4b5390870df0e0}
    Что означает преместительное свойство в контексте расходящихся рядов?

    \begin{cloze}{1}
        Любая перестановка тоже расходится.
    \end{cloze}
\end{note}

\begin{note}{2bd23513ad354d76a4f5dfcc5b23cf38}
    Пусть \({ f, g : E \to \mathbb R }\).
    \begin{icloze}{1}
        Величина
        \[
            \sup_{x \in E} \left\lvert f(x) - g(x) \right\rvert
        \]
    \end{icloze}
    называется \begin{icloze}{2}Чебышёвским уклонением.\end{icloze}
\end{note}

\begin{note}{806916674138455a8d97f6dc173173b2}
    Пусть \({ f, g : E \to \mathbb R }\).
    \begin{icloze}{2}Чебышёвское уклонение\end{icloze} обозначается \begin{icloze}{1}\({ \rho(f, g) }\).\end{icloze}
\end{note}

\section{Лекция 09.11.22}
\begin{note}{20bbdbfc6a3e47d5a68ff0fe6148f498}
    Пусть \begin{icloze}{3}\({ f }\) и все \({ f_n }\) интегрируемы на \({ [a, b] }\)\end{icloze} и \begin{icloze}{2}\({ f_n \rightrightarrows f }\).\end{icloze}
    Тогда
    \begin{icloze}{1}
        \[
            \lim \int_{a}^{b} f_n = \int_{a}^{b} f.
        \]
    \end{icloze}
\end{note}

\begin{note}{8ac3641f138f4e04a7f112015f0cc4df}
    Пусть \({ f }\) и все \({ f_n }\) интегрируемы на \({ [a, b] }\) и \({ f_n \rightrightarrows f }\).
    Тогда
    \[
        \lim \int_{a}^{b} f_n = \int_{a}^{b} f.
    \]

    \begin{center}
        \tiny
        <<\begin{icloze}{1}Предельный переход под знаком интеграла\end{icloze}>>
    \end{center}
\end{note}

\begin{note}{9ab163a208184d589048e4103dbf7978}
    Пусть все \({ f_n }\) и \({ f }\) интегрируемы и \({ f_n \to f }\).
    При каком условии можно выполнять предельный переход под знаком интеграла \({ \int_{a}^{b} f_n }\)?

    \begin{cloze}{1}
        Если сходимость равномерна.
    \end{cloze}
\end{note}

\begin{note}{73d904ec0c8444e989c74d892653ef06}
    Пример, показывающий, что поточечной сходимости не достаточно для выполнения предельного перехода под знаком интеграла.

    \begin{cloze}{1}
        \[
            f_n(x) = \begin{cases}
                n, & x \in (0, \frac{1}{n}), \\
                0, & \text{иначе}.
            \end{cases}
            \quad \text{на}\ [0, 1]
        \]
    \end{cloze}
\end{note}

\begin{note}{4a61840b6b7e43d88e86a8f43109394f}
    В чём основная идея доказательства теоремы о предельном переходе под знаком интеграла?

    \begin{cloze}{1}
        Определение равномерной сходимости и оценка сверху для модуля разности интегралов.
    \end{cloze}
\end{note}

\section{Лекция 16.11.22}
\begin{note}{cbbd0644365c41c8bad0a7182149b40d}
    Пусть \begin{icloze}{3}все \({ f_n }\) и \({ \sum f_n }\) интегрируемы на \({ [a, b] }\).\end{icloze}
    Если \begin{icloze}{2}\({ \sum f_n }\) сходится равномерно на \({ [a, b] }\),\end{icloze} то
    \begin{icloze}{1}
        \[
            \int_{a}^{b} \left( \sum_{n=1}^{\infty} f_n \right) = \sum_{n=1}^{\infty} \int_{a}^{b} f_n.
        \]
    \end{icloze}
\end{note}

\section{Лекция 23.11.22}
\begin{note}{81b3a95a5a394f6792bcb6ecfeba6cc8}
    Степенной ряд можно \begin{icloze}{2}интегрировать почленно\end{icloze} по любому отрезку внутри \begin{icloze}{1}множества сходимости.\end{icloze}
\end{note}

\begin{note}{a4e98cb1efd6427b8caee76f4a590979}
    Степенной ряд можно интегрировать почленно по любому отрезку внутри множества сходимости.
    В чём ключевая идея доказательства?

    \begin{cloze}{1}
        Степенной ряд сходится равномерно на любом таком отрезке.
    \end{cloze}
\end{note}

\section{Лекция 30.11.22}
\begin{note}{e71a98c1f27443538a4b0775e3034bd8}
    Какие ряды рассматриваются в теореме о радиусах степенных рядов?

    \begin{cloze}{1}
        Степенной ряд и ряды производных и первообразных его членов.
    \end{cloze}
\end{note}

\begin{note}{d1689e2077044a749292b08131dd2c51}
    Что утверждается в теореме о радиусах степенных рядов?

    \begin{cloze}{1}
        Рассматриваемые ряды имеют одинаковые радиусы сходимости.
    \end{cloze}
\end{note}

\section{Лекция 07.12.22}
\begin{note}{2e2030ff369747f6a18c94defaa57ba9}
    Пусть \({ f : (A, B) \to \mathbb R, a \in (A, B) }\).
    Функция \({ f }\) называется \begin{icloze}{2}аналитической в точке \({ a }\),\end{icloze} если \begin{icloze}{1}на окрестности точки \({ a }\) она представляется степенным рядом с центром в точке \({ a }\).\end{icloze}
\end{note}

\begin{note}{24d660592f734550a7586675e3b2e307}
    Пусть \({ f : (A, B) \to \mathbb R }\).
    Функция \({ f }\) называется \begin{icloze}{2}аналитической на промежутке \({ (A, B) }\),\end{icloze} если \begin{icloze}{1}она аналитична в каждой точке \({ (A, B) }\).\end{icloze}
\end{note}

\begin{note}{124c24286bb94e2281c6386440f38d2c}
    В чём основная задача анализа Фурье?

    \begin{cloze}{1}
        Разложение периодических функций в ряд по косинусам и синусам.
    \end{cloze}
\end{note}

\begin{note}{f659146465f64b918c69ce46c8ae72e5}
    Задачей \begin{icloze}{2}анализа Фурье\end{icloze} является поиск разложения вида
    \[
        f(x) = \begin{icloze}{1}\frac{a_0}{2} + \sum_{n=1}^{\infty} (a_n \cos nx + b_n \sin nx)\,.\end{icloze}
    \]
\end{note}

\begin{note}{c6879c7d93d34aadac9c7b7f69bdc5eb}
    Разложение вида
    \[
        f(x) = \frac{a_0}{2} + \sum_{n=1}^{\infty} (a_n \cos nx + b_n \sin nx)\,.
    \]
    применимо к функциям \begin{icloze}{1}с периодом \({ 2\pi }\).\end{icloze}
\end{note}

\begin{note}{2af4cc16ab5c4331ba6fdc4d8e33606d}
    Пусть функция \({ f }\) имеет период \({ T }\).
    Как привезти её к функции с периодом \({ 2\pi }\)?

    \begin{cloze}{1}
        Рассмотреть
        \[
            x \mapsto f\left( \frac{x}{2\pi} \cdot T \right)\,.
        \]
    \end{cloze}
\end{note}

\begin{note}{dfca446221d84d72a637ce570b31bbce}
    Чем примечательно интегрирование периодических функций?

    \begin{cloze}{1}
        Значение интеграла по промежутку длиной в период не зависит от сдвига промежутка.
    \end{cloze}
\end{note}

\begin{note}{512ee88e6cec4fa193d82ca1d7252a90}
    Как показать, что система
    \[
        1, \cos x, \sin x, \cos 2x, \sin 2x, \ldots
    \]
    ортогональна?

    \begin{cloze}{1}
        Явно вычислить скалярные произведения, используя формулы для произведений \({ \sin }\) и \({ \cos }\).
    \end{cloze}
\end{note}

\begin{note}{cf31d7904d0f4319829798700280b559}
    В контексте какого скалярного произведения система
    \[
        1, \cos x, \sin x, \cos 2x, \sin 2x, \ldots
    \]
    является ортогональной?

    \begin{cloze}{1}
        \({ \displaystyle (f, g) = \int_{0}^{2\pi} f g }\)\,.
    \end{cloze}
\end{note}

\begin{note}{e680e2296ca447878109aa1bd98e0564}
    \[
        \int_{0}^{2\pi} \sin mx = \begin{icloze}{1}0,\end{icloze} \quad (n > 0).
    \]
\end{note}

\begin{note}{a263ac35453841f1a469b0b5826b0799}
    \[
        \int_{0}^{2\pi} \cos nx = \begin{icloze}{1}0,\end{icloze} \quad (n > 0).
    \]
\end{note}

\begin{note}{5c3bed81f2b94a069fc8c60688cd3739}
    \[
        \int_{0}^{2\pi} \sin mx \cdot \cos nx\: dx = \begin{icloze}{1}0,\end{icloze} \quad (\begin{icloze}{2}\forall n, m\end{icloze})
    \]
\end{note}

\begin{note}{4cb5eceb345d4acbb7db16116de48c3b}
    \[
        \int_{0}^{2\pi} \sin mx \sin nx\: dx = \begin{icloze}{1}0,\end{icloze} \quad (\begin{icloze}{2}n \neq m\end{icloze}).
    \]
\end{note}

\begin{note}{198569868a6b4e77a0e1891b73c4c088}
    \[
        \int_{0}^{2\pi} \sin^2 nx \: dx = \begin{icloze}{1}\pi.\end{icloze}
    \]
\end{note}

\begin{note}{2b2d5635b2a14ee18b7cf8824f1bbc56}
    \[
        \int_{0}^{2\pi} \cos mx \cos nx\: dx = \begin{icloze}{1}0,\end{icloze} \quad (\begin{icloze}{2}n \neq m\end{icloze}).
    \]
\end{note}

\begin{note}{69df715bc9ba414bae72641bbf74d1f5}
    \[
        \int_{0}^{2\pi} \cos^2 nx \: dx = \begin{icloze}{1}\pi.\end{icloze}
    \]
\end{note}

\begin{note}{8a035b960ad54a9aa060b304bf8462ee}
    Пусть функция \({ f }\) имеет разложение вида
    \[
        f(x) = \frac{a_0}{2} + \sum_{n=1}^{\infty} (a_n \cos nx + b_n \sin nx)\,.
    \]
    Как можно найти значения коэффициента \({ a_k }\)?

    \begin{cloze}{1}
        Проинтегрировать \({ f(x) \cos kx }\) на \({ [0, 2\pi] }\).
    \end{cloze}
\end{note}

\begin{note}{ebe1bbbdc4ce44c5941e1f2e887651d9}
    Пусть функция \({ f }\) имеет разложение вида
    \[
        f(x) = \frac{a_0}{2} + \sum_{n=1}^{\infty} (a_n \cos nx + b_n \sin nx)\,.
    \]
    Тогда
    \[
        a_k = \begin{icloze}{1}\frac{1}{\pi} \int_{0}^{2\pi} f(x) \cos kx\: dx\,.\end{icloze}
    \]
\end{note}

\begin{note}{8a5ae715c85e4758b5c5fcc97e93bcd9}
    Пусть функция \({ f }\) имеет разложение вида
    \[
        f(x) = \frac{a_0}{2} + \sum_{n=1}^{\infty} (a_n \cos nx + b_n \sin nx)\,.
    \]
    Как можно найти значения коэффициента \({ b_k }\)?

    \begin{cloze}{1}
        Проинтегрировать \({ f(x) \sin kx }\) на \({ [0, 2\pi] }\).
    \end{cloze}
\end{note}

\begin{note}{cfcfaee0ac304ebe95c11b2b5c016c7d}
    Пусть функция \({ f }\) имеет разложение вида
    \[
        f(x) = \frac{a_0}{2} + \sum_{n=1}^{\infty} (a_n \cos nx + b_n \sin nx)\,.
    \]
    Тогда
    \[
        b_k = \begin{icloze}{1}\frac{1}{\pi} \int_{0}^{2\pi} f(x) \sin kx\: dx\,.\end{icloze}
    \]
\end{note}

\begin{note}{67542462c1d046c7b986155c804058a9}
    Пусть \({ f }\) периодична с периодом \({ 2\pi }\).
    Коэффициенты,
    \begin{align*}
        a_k &= \frac{1}{\pi} \int_{0}^{2\pi} f(x) \cos kx\: dx\,, \\
        b_k &= \frac{1}{\pi} \int_{0}^{2\pi} f(x) \sin kx\: dx
    \end{align*}
    называются \begin{icloze}{1}коэффициентами Фурье функции \({ f }\).\end{icloze}
\end{note}

\begin{note}{7045668d58db45ad9892a6b3813128fd}
    Пусть \({ f }\) периодична с периодом \({ 2\pi }\).
    \begin{icloze}{2}Разложение \({ f }\) по системе \({ \left\{ 1, \cos nx, \sin nx \right\} }\) с коэффициентами Фурье\end{icloze} называется \begin{icloze}{1}рядом Фурье функции \({ f }\).\end{icloze}
\end{note}

\begin{note}{e548ae34226a44dbb7bff902ff24ce38}
    Коэффициенты перед \begin{icloze}{2}\({ \cos nx }\)\end{icloze} в ряде Фурье периодической функций обычно обозначаются \begin{icloze}{1}\({ a_n }\).\end{icloze}
\end{note}

\begin{note}{485d606663f741c4aa0456d205bff990}
    Коэффициенты перед \begin{icloze}{2}\({ \sin nx }\)\end{icloze} в ряде Фурье периодической функций обычно обозначаются \begin{icloze}{1}\({ b_n }\).\end{icloze}
\end{note}

\begin{note}{fdee5342607e4a57933450d0d3fe819e}
    Как строится ряд Фурье для функции, определённой на конечном интервале?

    \begin{cloze}{1}
        Сначала функция периодически продолжается на всю числовую ось.
    \end{cloze}
\end{note}

\section{Лекция 14.12.22}
\begin{note}{fe7d28e98d744f1981b58bcd9b88eb53}
    Функция \({ f : D \subset \mathbb R \to \mathbb R }\) называется \begin{icloze}{2}кусочно гладкой на множестве \({ D }\),\end{icloze} если \begin{icloze}{1}её производная кусочно-непрерывна на \({ D }\).\end{icloze}
\end{note}

\begin{note}{a11d984e49654f1fbe979c15242cc24b}
    Для удобства будем считать, что периодическая функция является \begin{icloze}{2}кусочно-не\-пре\-рыв\-ной,\end{icloze} если \begin{icloze}{1}она кусочно-не\-пре\-рыв\-на на каком-либо отрезке длинной в период.\end{icloze}
\end{note}

\begin{note}{49958c86df924ee8b4f132d6cb5e5ab9}
    Какой вопрос рассматривается в теореме Дирихле (о рядах Фурье)?

    \begin{cloze}{1}
        Сходимость ряда Фурье периодической функции.
    \end{cloze}
\end{note}

\begin{note}{9d03d5873804424896da3f6d23a32213}
    Каким видом условия является теорема Дирихле (о рядах Фурье)?

    \begin{cloze}{1}
        Достаточным.
    \end{cloze}
\end{note}

\begin{note}{1658d6c012f44632ac5d5b1a42ff3bbf}
    Какая функция рассматривается в теореме Дирихле (о рядах Фурье)?

    \begin{cloze}{1}
        Кусочно-гладкая на отрезке длинной в период.
    \end{cloze}
\end{note}

\begin{note}{5f323e2ec77b4cfa81fe10bc0ef46eb2}
    Что мы заключаем из теоремы Дирихле (о рядах Фурье)?

    \begin{cloze}{1}
        Ряд Фурье функции сходится в любой точке к среднему значению односторонних переделов.
    \end{cloze}
\end{note}

\begin{note}{18b17e6720aa4d3f85dbe0d61d54dc23}
    Что мы заключаем из теоремы Дирихле (о рядах Фурье) для точек непрерывности функции?

    \begin{cloze}{1}
        Ряд Фурье сходится к функции.
    \end{cloze}
\end{note}

\begin{note}{a04607043611474398e000b5085c0486}
    Как ведёт себя ряд Фурье кусочно-гладкой периодической функции?

    \begin{cloze}{1}
        Сходится в любой точке (но не обязательно к самой функции.)
    \end{cloze}
\end{note}

\begin{note}{6fc46361ef9349e1985098dcd7af4b4d}
    Как ведёт себя ряд Фурье кусочно-гладкой периодической функции в точках её непрерывности?

    \begin{cloze}{1}
        Сходится к значению функции.
    \end{cloze}
\end{note}

\begin{note}{54ef07badb244f39a259e3237d5855c8}
    Как ведёт себя ряд Фурье кусочно-гладкой периодической функции в точках её разрыва?

    \begin{cloze}{1}
        Сходится к среднему значению односторонних пределов.
    \end{cloze}
\end{note}

\begin{note}{b5963613427a41178042d681a5259f4f}
    Как выглядит ряд Фурье функции \({ f(x) = x }\) на \({ [-\pi, \pi] }\)?

    \begin{cloze}{1}
        \[
            \sum_{n=1}^{\infty} \frac{2(-1)^{n+1}}{n} \sin nx\,.
        \]
    \end{cloze}
\end{note}

\begin{note}{e8a7d9c8925a418188d9ee3362f79705}
    Как с использованием рядов Фурье можно найти значение ряда
    \[
        1 - \frac{1}{3} + \frac{1}{5} - \frac{1}{7} + \cdots\,?
    \]

    \begin{cloze}{1}
        Использовать ряд Фурье \({ f(x) = x }\) в точке \({ \frac{\pi}{2} }\).
    \end{cloze}
\end{note}

\begin{note}{730f19be667548e9b24c9ce33ea43784}
    Для каких периодических функций ряд Фурье обязан сходится абсолютно?

    \begin{cloze}{1}
        Для непрерывных и кусочно-гладких.
    \end{cloze}
\end{note}

\begin{note}{bbe026a519824222b2c98c639678857a}
    Для каких периодических функций ряд Фурье обязан сходится равномерно?

    \begin{cloze}{1}
        Для непрерывных и кусочно-гладких.
    \end{cloze}
\end{note}

\begin{note}{5c566e83fa0340c58eac557785785279}
    Что можно сказать о ряде Фурье непрерывной кусочно гладкой периодической функции?

    \begin{cloze}{1}
        Он сходится к ней абсолютно и равномерно.
    \end{cloze}
\end{note}

\begin{note}{aa0ff21a64b54ffd97cd4c26310596a9}
    Пусть периодическая функция \({ f }\) имеет непрерывные производные порядка вплоть до \({ N - 1 }\) и кусочно-непрерывную производную порядка \({ N }\).
    Что можно сказать о её ряде Фурье?

    \begin{cloze}{1}
        \[
            \left\lvert f(x) - s_k(x) \right\rvert < \frac{\delta_k}{k^{N - 1/2}}\,, \quad \text{где}\ \delta_k \to 0\,.
        \]
    \end{cloze}
\end{note}

\begin{note}{24eb28a514eb4fc2811401b3ee9164d8}
    Для каких периодических функций верна оценка
    \[
        \left\lvert f(x) - s_k(x) \right\rvert < \frac{\delta_k}{k^{N - 1/2}}\,, \quad \text{где}\ \delta_k \to 0\,,
    \]
    для остатка ряда Фурье?

    \begin{cloze}{1}
        Имеющих непрерывные производные порядка вплоть до \({ N - 1 }\) и кусочно-непрерывную производную порядка \({ N }\).
    \end{cloze}
\end{note}

\begin{note}{4c6aad88a3934ce89326f70fa8fdbeb0}
    Пусть \({ f }\) периодична и \({ f^{(k)} }\) кусочно-гладка.
    Что можно сказать о коэффициентах ряда Фурье \({ f }\)?

    \begin{cloze}{1}
        \[
            a_n, b_n = o(1 / n^{k+1})\,, \quad n \to \infty\,.
        \]
    \end{cloze}
\end{note}

\begin{note}{9b59e17ad69d491e8297660c69a7f64c}
    Для каких периодических функций верна оценка
    \[
        a_n, b_n = o(1 / n^k)\,, \quad n \to \infty\,,
    \]
    для коэффициентов ряда Фурье?

    \begin{cloze}{1}
        Имеющих кусочно-гладкую производную порядка \({ k-1 }\).
    \end{cloze}
\end{note}

\begin{note}{15a14a3394e24021a7a78a7cb85cf781}
    Пусть \({ f }\) периодична, \({ f^{(k-1)} }\) непрерывна и \({ f^{(k)} }\) кусочно-гладка.
    Что можно сказать о коэффициентах ряда Фурье \({ f }\)?

    \begin{cloze}{1}
        \[
            a_n, b_n = O(1 / n^{k+1})\,, \quad n \to \infty\,,
        \]
    \end{cloze}
\end{note}

\begin{note}{e9cbe9cc647b4cf7b15c0476a89bb310}
    Для каких периодических функций верна оценка
    \[
        a_n, b_n = O(1 /n^k)\,, \quad n \to \infty\,,
    \]
    для коэффициентов ряда Фурье?

    \begin{cloze}{1}
        Таких что \({ f^{(k-2)} }\) непрерывна и \({ f^{(k-1)} }\) кусочно-гладка.
    \end{cloze}
\end{note}

\begin{note}{2b5908be228748108c55d8da7d970a5d}
    Пусть \({ f }\) --- квадратично интегрируемая, \({ 2\pi }\)-периодическая функция.
    Для какого выражения даётся верхняя оценка в неравенстве Бесселя, в контексте ряда Фурье \({ f }\)?

    \begin{cloze}{1}
        \[
            \frac{a_0^2}{2} + \sum_{n=1}^{m}  (a_n^2 + b_n^2)\,.
        \]
    \end{cloze}
\end{note}

\begin{note}{966d28a6d438455b9f7dc0d5a608af56}
    Пусть \({ f }\) --- квадратично интегрируемая, \({ 2\pi }\)-периодическая функция.
    Какое выражение выступает в качестве верхней оценки в неравенстве Бесселя, в контексте ряда Фурье \({ f }\)?

    \begin{cloze}{1}
        \[
            \frac{1}{\pi} \int_{0}^{2\pi} f^2\,.
        \]
    \end{cloze}
\end{note}

\begin{note}{c1eb80029ccc472590e25f079f413198}
    Пусть \({ f }\) --- квадратично интегрируемая, \({ 2\pi }\)-периодическая функция.
    Допускает ли неравенство Бесселя, в контексте ряда Фурье \({ f }\), равенство левой и правой его частей?

    \begin{cloze}{1}
        Да.
    \end{cloze}
\end{note}

\begin{note}{51cba508f22a4be288d572f18d428542}
    Для каких периодических функций выполняется неравенство Бесселя для коэффициентов ряда Фурье?

    \begin{cloze}{1}
        Квадратично-интегрируемых.
    \end{cloze}
\end{note}

\begin{note}{a8c7cbed087d4a4eabdd85ccfc7ce2ec}
    Пусть \({ f }\) --- квадратично интегрируемая, \({ 2\pi }\)-периодическая функция.
    Как называется неравенство
    \[
        \frac{a_0^2}{2} + \sum_{n=1}^{m}  (a_n^2 + b_n^2) \leqslant \frac{1}{\pi} \int_{0}^{2\pi} f^2
    \]
    для коэффициентов ряда Фурье \({ f }\)?

    \begin{cloze}{1}
        Неравенство Бесселя.
    \end{cloze}
\end{note}

\begin{note}{7b93ebc1526c495c84cb4ac052cd0641}
    Что моментально следует из неравенства Бесселя в контексте рядов Фурье?

    \begin{cloze}{1}
        Ряд из сумм квадратов коэффициентов сходится.
    \end{cloze}
\end{note}

\begin{note}{303c44053e5f4bcebada6128be2e8a82}
    Пусть \({ f }\) --- квадратично интегрируемая, \({ 2\pi }\)-периодическая функция.
    Что устанавливает равенство Парсеваля, в контексте ряда Фурье \({ f }\)?

    \begin{cloze}{1}
        Неравенство Бесселя в пределе обращается в равенство.
    \end{cloze}
\end{note}

\begin{note}{a64141f971434ca0880e5256d1e3e5f0}
    Для каких периодических функций выполняется равенство Парсеваля для коэффициентов ряда Фурье?

    \begin{cloze}{1}
        Квадратично-интегрируемых.
    \end{cloze}
\end{note}

\begin{note}{15bdb891041c4f00a17f6c19d4244acc}
    Пусть \({ f }\) --- квадратично интегрируемая, \({ 2\pi }\)-периодическая функция.
    Как называется равенство
    \[
        \frac{a_0^2}{2} + \sum_{n=1}^{\infty} (a_n^2 + b_n^2) = \frac{1}{\pi} \int_{0}^{2\pi} f^2
    \]
    для коэффициентов ряда Фурье \({ f }\)?

    \begin{cloze}{1}
        Равенство Парсеваля.
    \end{cloze}
\end{note}

\section{Семинар 23.12.22}
\begin{note}{7c999c871dc44972b63eb58569ebb0a1}
    Как радикальный признак Коши применяется к комплексным рядам?

    \begin{cloze}{1}
        Так же, как к вещественным.
    \end{cloze}
\end{note}

\begin{note}{4f946c4839c64b6ea3c62676a3dbe503}
    Как признак Даламбера применяется к комплексным рядам?

    \begin{cloze}{1}
        Так же, как к вещественным.
    \end{cloze}
\end{note}

\begin{note}{7ca2abd7e18c49fa9d428134d358aae9}
    Как формула Коши-Адамара применяется к комплексным степенным рядам?

    \begin{cloze}{1}
        Так же, как к вещественным.
    \end{cloze}
\end{note}

\begin{note}{5ae400877c8c4b039cfebcabc92d3b07}
    Сходится ли ряд \({ \sum \frac{\left\lvert \cos 3n \right\rvert}{\sqrt{n}} }\)?

    \begin{cloze}{1}
        Расходится.
    \end{cloze}
\end{note}

\begin{note}{f5e47353454b4e4fafc3b0c27a0e95ee}
    Как показать расходимость ряд \({ \sum \frac{\left\lvert \cos 3n \right\rvert}{\sqrt{n}} }\)?

    \begin{cloze}{1}
        Оценить снизу через квадрат косинуса + формула понижения степени.
    \end{cloze}
\end{note}

\begin{note}{52af15d5e4b34a538fec7a76ab2870b6}
    Почему ряд \({ \sum \frac{1 + \cos 6n}{2 \sqrt{n}} }\) расходится?

    \begin{cloze}{1}
        Он представляется как сумма расходящегося и сходящегося рядов.
    \end{cloze}
\end{note}

\section{Лекция 21.12.22}
\begin{note}{f075132ed82c4c6a87d2bacaeb61a7a6}
    Как определяется частичная сумма ряда \({ \sum_{n \in \mathbb Z} a_n }\)?

    \begin{cloze}{1}
        Сумма от \({ a_{-k} }\) до \({ a_k }\).
    \end{cloze}
\end{note}

\begin{note}{52518a8f21a94b5ca40cce0ed48b6420}
    Как ряд Фурье запаивается в экспоненциальной форме?

    \begin{cloze}{1}
        \[
            \sum_{n \in \mathbb Z} c_n e^{i nx}.
        \]
    \end{cloze}
\end{note}

\begin{note}{5cb814e06e514bbab9de89636a2056de}
    Как ряд Фурье в тригонометрической записи привести к экспоненциальной форме?

    \begin{cloze}{1}
        Через формулы Эйлера.
    \end{cloze}
\end{note}

\begin{note}{2a543da90ce94d149fc00d00159cf586}
    Пусть дан ряд Фурье
    \[
        \frac{a_0}{2} + \sum_{n=1}^{\infty} (a_n \cos nx + b_n \sin nx) = \sum_{n \in \mathbb Z} c_n e^{i nx}\,.
    \]
    Как \({ c_n }\) выражается через \({ a_n, b_n }\) для \({ n > 0 }\)?

    \begin{cloze}{1}
        \({ \frac{1}{2}(a_n - ib_n) }\).
    \end{cloze}
\end{note}

\begin{note}{8f1f4605f0d643f3a81edcb52c36b5bf}
    Пусть дан ряд Фурье
    \[
        \frac{a_0}{2} + \sum_{n=1}^{\infty} (a_n \cos nx + b_n \sin nx) = \sum_{n \in \mathbb Z} c_n e^{i nx}\,.
    \]
    Как \({ c_{-n} }\) выражается через \({ a_n, b_n }\) для \({ n > 0 }\)?

    \begin{cloze}{1}
        \({ \frac{1}{2}(a_n + ib_n) }\).
    \end{cloze}
\end{note}

\begin{note}{17d3e4c6c13d4ff68957bcc03f88748d}
    Пусть дан ряд Фурье
    \[
        \frac{a_0}{2} + \sum_{n=1}^{\infty} (a_n \cos nx + b_n \sin nx) = \sum_{n \in \mathbb Z} c_n e^{i nx}\,.
    \]
    Как \({ c_0 }\) выражается через изначальные коэффициенты?

    \begin{cloze}{1}
        \({ \frac{a_0}{2} }\).
    \end{cloze}
\end{note}

\begin{note}{613d457df89a4ad68ab32c124f308600}
    Пусть \({ f }\) периодична с периодом \({ 2\pi }\).
    Как выводится интегральная формула для коэффициентов ряда Фурье \({ f }\)?

    \begin{cloze}{1}
        Через их связь с коэффициентами в тригонометрической записи.
    \end{cloze}
\end{note}

\begin{note}{d3624b2bd96a4696b561dcae634c1da1}
    Пусть \({ f }\) периодична с периодом \({ 2\pi }\).
    Как определяются коэффициенты Фурье \({ f }\) в тригонометрической форме?

    \begin{cloze}{1}
        \[
            c_n = \frac{1}{2\pi} \int_{-\pi}^{\pi} f(x) e^{-i nx}\: dx\,.
        \]
    \end{cloze}
\end{note}

\end{document}
