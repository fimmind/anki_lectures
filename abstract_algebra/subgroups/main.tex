%! TeX root = ./main.tex
\documentclass[11pt, a5paper]{article}
\usepackage[width=10cm, top=0.5cm, bottom=2cm]{geometry}

\usepackage[T1,T2A]{fontenc}
\usepackage[utf8]{inputenc}
\usepackage[english,russian]{babel}
\usepackage{libertine}

\usepackage{amsmath}
\usepackage{amssymb}
\usepackage{amsthm}
\usepackage{mathrsfs}
\usepackage{framed}
\usepackage{xcolor}

\setlength{\parindent}{0pt}

% Force \pagebreak for every section
\let\oldsection\section
\renewcommand\section{\pagebreak\oldsection}

\renewcommand{\thesection}{}
\renewcommand{\thesubsection}{Note \arabic{subsection}}
\renewcommand{\thesubsubsection}{}
\renewcommand{\theparagraph}{}

\newenvironment{note}[1]{\goodbreak\par\subsection{\hfill \color{lightgray}\tiny #1}}{}
\newenvironment{cloze}[2][\ldots]{\begin{leftbar}}{\end{leftbar}}
\newenvironment{icloze}[2][\ldots]{%
  \text{\tiny \color{lightgray}\{\{c#2::}\hspace{0pt}\ignorespaces%
}{%
  \unskip\hspace{0pt}\text{\tiny\color{lightgray}\}\}}%
}


\begin{document}
\section{Definition and Examples}
\begin{note}{9080791fc8754b0bb88c381c10acbdfc}
    Let \({ G }\) be a group.
    If \begin{icloze}{2}\({ H }\) is a subgroup of \({ G }\)\end{icloze} we shall write
    \begin{icloze}{1}
        \[
            H \leq G.
        \]
    \end{icloze}
\end{note}

\begin{note}{66e7f23728af4c9d8839d172e59d716a}
    Let \({ G }\) be a group and \({ H \leq G }\).
    We shall denote the operation for \({ H }\) by \begin{icloze}{1}the same symbol as the operation for \({ G }\).\end{icloze}
\end{note}

\begin{note}{e76ada2ee6da4b5fb71966e9f7ce3ded}
    Let \({ G }\) be a group.
    If \begin{icloze}{2}\({ H \leq G }\) and \({ H \neq G }\)\end{icloze} we shall write \begin{icloze}{1}\({ H < G }\).\end{icloze}
\end{note}

\begin{note}{1d28c11c52c84bd0b639505598bb1dce}
    If \({ H }\) is a subgroup of \({ G }\) then any equation in the subgroup \({ H }\) may also be viewed as \begin{icloze}{1}an equation in the group \({ G }\).\end{icloze}
\end{note}

\begin{note}{8f5b765961884460823141645b5ea08b}
    Let \({ G }\) be a group and \({ H \leq G }\).
    What is the identity of \({ H }\)?

    \begin{cloze}{1}
        The identity of \({ G }\).
    \end{cloze}
\end{note}

\begin{note}{7c122a5400f64eba9a76438c1ff296ee}
    Let \({ G }\) be a group and \({ H \leq G }\).
    The identity of \({ H }\) is the identity of \({ G }\).
    What is the key idea in the proof?

    \begin{cloze}{1}
        The identity is unique and it is the identity of \({ G }\).
    \end{cloze}
\end{note}

\begin{note}{83cba804764b43e2baf282ffee513694}
    Let \({ G }\) be a group.
    What is the minimal subgroup of \({ G }\)?

    \begin{cloze}{1}
        The singleton \({ \left\{ 1 \right\} }\).
    \end{cloze}
\end{note}

\begin{note}{d40df9f46a6b43ecb3fea9b5b37e5b1c}
    Let \({ G }\) be a group.
    What is the element that any subgroup of \({ G }\) must contain?

    \begin{cloze}{1}
        The identity of \({ G }\).
    \end{cloze}
\end{note}

\begin{note}{bc95ad6358814213a6fff2eb0cfd544b}
    Let \({ G }\) be a group and \({ H \leq G }\).
    What is the inverse of an element \({ x }\) in \({ H }\)?

    \begin{cloze}{1}
        The inverse of \({ x }\) in \({ G }\).
    \end{cloze}
\end{note}

\begin{note}{be9f1756cf3449e8a6718069fd4aedf5}
    Let \({ G }\) be a group and \({ H \leq G }\).
    Why is the notation \({ x^{-1} }\) unambiguous?

    \begin{cloze}{1}
        The inverse in \({ H }\) is the same as the inverse in \({ G }\).
    \end{cloze}
\end{note}

\begin{note}{8aabd93df8a5437eb3e50c3e0d438381}
    Let \({ G }\) be a group.
    \begin{icloze}{2}The subgroup \({ \left\{ 1 \right\} }\) of \({ G }\)\end{icloze} is called \begin{icloze}{1}the trivial subgroup.\end{icloze}
\end{note}

\begin{note}{eb859714e1f34f4db3dc35755f562945}
    Let \({ G }\) be a group.
    \begin{icloze}{2}The trivial subgroup\end{icloze} is denoted by \begin{icloze}{1}\({ 1 }\).\end{icloze}
\end{note}

\begin{note}{74ab3ceb9a36420fb53dc2b3a22eb5f2}
    Which subgroups does any group have?

    \begin{cloze}{1}
        The trivial subgroup and the group itself.
    \end{cloze}
\end{note}

\begin{note}{5683ff4198a74e9d988f501c925d85ad}
    If \({ H }\) is a subgroup of \({ G }\) and \({ K }\) is a subgroup of \({ H }\), then \begin{icloze}{1}\({ K }\) is a subgroup of \({ G }\).\end{icloze}
\end{note}

\begin{note}{50a07cbb14aa4bed8866efcbedb0be4d}
    Which object is considered in the Subgroup Criterion?

    \begin{cloze}{1}
        Any subset of a group.
    \end{cloze}
\end{note}

\begin{note}{840038893a3642a18c3e43c4e89aed17}
    What are the conditions of the Subgroup Criterion?

    \begin{cloze}{1}
        The subset is nonempty and closed under \({ (x, y) \mapsto x \cdot y^{-1} }\).
    \end{cloze}
\end{note}

\begin{note}{71291d04ca2941fca2fc08759d8fd302}
    What is the special case considered in the Subgroup Criterion?

    \begin{cloze}{1}
        The subset is finite.
    \end{cloze}
\end{note}

\begin{note}{a1e69be09e78402d989b3805b3dfc54f}
    What are the conditions of the Subgroup Criterion for a finite subset?

    \begin{cloze}{1}
        The subset is nonempty and closed under the operation.
    \end{cloze}
\end{note}

\begin{note}{5bcd55a73e184bcd9bcc32f1ee47da2e}
    What is the key idea in the proof of the Subgroup Criterion for a finite subset?

    \begin{cloze}{1}
        Any element's inverse is it's \({ n }\)-th power.
    \end{cloze}
\end{note}

\end{document}
