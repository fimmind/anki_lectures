%! TeX root = ./main.tex
\documentclass[11pt, a5paper]{article}
\usepackage[width=10cm, top=0.5cm, bottom=2cm]{geometry}

\usepackage[T1,T2A]{fontenc}
\usepackage[utf8]{inputenc}
\usepackage[english,russian]{babel}
\usepackage{libertine}

\usepackage{amsmath}
\usepackage{amssymb}
\usepackage{amsthm}
\usepackage{mathrsfs}
\usepackage{framed}
\usepackage{xcolor}

\setlength{\parindent}{0pt}

% Force \pagebreak for every section
\let\oldsection\section
\renewcommand\section{\pagebreak\oldsection}

\renewcommand{\thesection}{}
\renewcommand{\thesubsection}{Note \arabic{subsection}}
\renewcommand{\thesubsubsection}{}
\renewcommand{\theparagraph}{}

\newenvironment{note}[1]{\goodbreak\par\subsection{\hfill \color{lightgray}\tiny #1}}{}
\newenvironment{cloze}[2][\ldots]{\begin{leftbar}}{\end{leftbar}}
\newenvironment{icloze}[2][\ldots]{%
  \ignorespaces\text{\tiny \color{lightgray}\{\{c#2::}\hspace{0pt}%
}{%
  \hspace{0pt}\text{\tiny\color{lightgray}\}\}}\unskip%
}


\begin{document}
\section{Definition and Examples}
\begin{note}{9080791fc8754b0bb88c381c10acbdfc}
    Let \({ G }\) be a group.
    If \begin{icloze}{2}\({ H }\) is a subgroup of \({ G }\)\end{icloze} we shall write
    \begin{icloze}{1}
        \[
            H \leq G.
        \]
    \end{icloze}
\end{note}

\begin{note}{66e7f23728af4c9d8839d172e59d716a}
    Let \({ G }\) be a group and \({ H \leq G }\).
    We shall denote the operation for \({ H }\) by \begin{icloze}{1}the same symbol as the operation for \({ G }\).\end{icloze}
\end{note}

\begin{note}{e76ada2ee6da4b5fb71966e9f7ce3ded}
    Let \({ G }\) be a group.
    If \begin{icloze}{2}\({ H \leq G }\) and \({ H \neq G }\)\end{icloze} we shall write \begin{icloze}{1}\({ H < G }\).\end{icloze}
\end{note}

\begin{note}{1d28c11c52c84bd0b639505598bb1dce}
    If \({ H }\) is a subgroup of \({ G }\) then any equation in the subgroup \({ H }\) may also be viewed as \begin{icloze}{1}an equation in the group \({ G }\).\end{icloze}
\end{note}

\begin{note}{8f5b765961884460823141645b5ea08b}
    Let \({ G }\) be a group and \({ H \leq G }\).
    What is the identity of \({ H }\)?

    \begin{cloze}{1}
        The identity of \({ G }\).
    \end{cloze}
\end{note}

\begin{note}{7c122a5400f64eba9a76438c1ff296ee}
    Let \({ G }\) be a group and \({ H \leq G }\).
    The identity of \({ H }\) is the identity of \({ G }\).
    What is the key idea in the proof?

    \begin{cloze}{1}
        The identity is unique and it is the identity of \({ G }\).
    \end{cloze}
\end{note}

\begin{note}{83cba804764b43e2baf282ffee513694}
    Let \({ G }\) be a group.
    What is the minimal subgroup of \({ G }\)?

    \begin{cloze}{1}
        The singleton \({ \left\{ 1 \right\} }\).
    \end{cloze}
\end{note}

\begin{note}{d40df9f46a6b43ecb3fea9b5b37e5b1c}
    Let \({ G }\) be a group.
    What is the element that any subgroup of \({ G }\) must contain?

    \begin{cloze}{1}
        The identity of \({ G }\).
    \end{cloze}
\end{note}

\begin{note}{bc95ad6358814213a6fff2eb0cfd544b}
    Let \({ G }\) be a group and \({ H \leq G }\).
    What is the inverse of an element \({ x }\) in \({ H }\)?

    \begin{cloze}{1}
        The inverse of \({ x }\) in \({ G }\).
    \end{cloze}
\end{note}

\begin{note}{be9f1756cf3449e8a6718069fd4aedf5}
    Let \({ G }\) be a group and \({ H \leq G }\).
    Why is the notation \({ x^{-1} }\) unambiguous?

    \begin{cloze}{1}
        The inverse in \({ H }\) is the same as the inverse in \({ G }\).
    \end{cloze}
\end{note}

\begin{note}{8aabd93df8a5437eb3e50c3e0d438381}
    Let \({ G }\) be a group.
    \begin{icloze}{2}The subgroup \({ \left\{ 1 \right\} }\) of \({ G }\)\end{icloze} is called \begin{icloze}{1}the trivial subgroup.\end{icloze}
\end{note}

\begin{note}{eb859714e1f34f4db3dc35755f562945}
    Let \({ G }\) be a group.
    \begin{icloze}{2}The trivial subgroup\end{icloze} is denoted by \begin{icloze}{1}\({ 1 }\).\end{icloze}
\end{note}

\begin{note}{74ab3ceb9a36420fb53dc2b3a22eb5f2}
    Which subgroups does any group have?

    \begin{cloze}{1}
        The trivial subgroup and the group itself.
    \end{cloze}
\end{note}

\begin{note}{5683ff4198a74e9d988f501c925d85ad}
    If \({ H }\) is a subgroup of \({ G }\) and \({ K }\) is a subgroup of \({ H }\), then \begin{icloze}{1}\({ K }\) is a subgroup of \({ G }\).\end{icloze}
\end{note}

\begin{note}{50a07cbb14aa4bed8866efcbedb0be4d}
    Which object is considered in the Subgroup Criterion?

    \begin{cloze}{1}
        Any subset of a group.
    \end{cloze}
\end{note}

\begin{note}{840038893a3642a18c3e43c4e89aed17}
    What are the conditions of the Subgroup Criterion?

    \begin{cloze}{1}
        The subset is nonempty and closed under \({ (x, y) \mapsto x \cdot y^{-1} }\).
    \end{cloze}
\end{note}

\begin{note}{71291d04ca2941fca2fc08759d8fd302}
    What is the special case considered in the Subgroup Criterion?

    \begin{cloze}{1}
        The subset is finite.
    \end{cloze}
\end{note}

\begin{note}{a1e69be09e78402d989b3805b3dfc54f}
    What are the conditions of the Subgroup Criterion for a finite subset?

    \begin{cloze}{1}
        The subset is nonempty and closed under the operation.
    \end{cloze}
\end{note}

\begin{note}{5bcd55a73e184bcd9bcc32f1ee47da2e}
    What is the key idea in the proof of the Subgroup Criterion for a finite subset?

    \begin{cloze}{1}
        Any element's inverse is it's \({ n }\)-th power.
    \end{cloze}
\end{note}

\begin{note}{0e1ccaae016c4900ac96b733fb9e1764}
    Why is the set of \({ 2 }\)-cycles in \({ S_n }\) not a subgroup of \({ S_n }\)?

    \begin{cloze}{1}
        It does not contain the identity.
    \end{cloze}
\end{note}

\begin{note}{587390d0450f4681a66bcbc8c0d5889c}
    Why is the set of reflection in \({ D_{2n} }\) not a subgroup of \({ D_{2n} }\)?

    \begin{cloze}{1}
        It does not contain the identity.
    \end{cloze}
\end{note}

\begin{note}{fc87d2283cb546708502ce325e326258}
    Why is the set of reflection in \({ D_{2n} }\) together with \({ 1 }\) not a subgroup of \({ D_{2n} }\)?

    \begin{cloze}{1}
        Two distinct reflections induce a rotation.
    \end{cloze}
\end{note}

\begin{note}{24b90e714649459ba38e6b40f07f6b2a}
    Is \({ \left\{ 1, r^2, s, sr^2 \right\} }\) a subgroup of \({ D_8 }\)?

    \begin{cloze}{1}
        Yes.
    \end{cloze}
\end{note}

\begin{note}{eac99978715a4ec894b296f8e1ee52f3}
    Is \({ \left\{ 1, r, s, sr \right\} }\) a subgroup of \({ D_8 }\)?

    \begin{cloze}{1}
        No.
    \end{cloze}
\end{note}

\begin{note}{64ea968bdce94647b6fb2c351a60f2a2}
    Is \({ \left\{ 1, r^2, sr, sr^3 \right\} }\) a subgroup of \({ D_8 }\)?

    \begin{cloze}{1}
        Yes.
    \end{cloze}
\end{note}

\begin{note}{678b87f890ac4d8da5be6a78cb619358}
    Is \({ \left\{ 1, r, r^2 \right\} }\) a subgroup of \({ D_8 }\)?

    \begin{cloze}{1}
        No.
    \end{cloze}
\end{note}

\begin{note}{e036f3cc7667461b98e50e94ff3a8c80}
    Is \({ \left\{ 1, r, r^2, r^3 \right\} }\) a subgroup of \({ D_8 }\)?

    \begin{cloze}{1}
        Yes.
    \end{cloze}
\end{note}

\begin{note}{209944ca7a524af3be44b398de974c2d}
    Give an example of a group and its infinite subset that is closed under the operations, but is not a subgroup of the original group.

    \begin{cloze}{1}
        Positive integers under addition.
    \end{cloze}
\end{note}

\begin{note}{547363a46106478187c20c5cbb868461}
    For what groups is the notion of the torsion subgroup introduced?

    \begin{cloze}{1}
        For abelian groups.
    \end{cloze}
\end{note}

\begin{note}{d29b9ffdb46c4c909fbfb2a438abb0a0}
    What is the torsion subgroup of an abelian group?

    \begin{cloze}{1}
        The set of all the elements of a finite order.
    \end{cloze}
\end{note}

\begin{note}{b2a854579339471d8ae41776f1661f29}
    Let \({ G }\) be an abelian group. What is the name of the set
    \[
        \left\{ g \in G : \left\lvert g \right\rvert < \infty \right\}\,?
    \]

    \begin{cloze}{1}
        The torsion subgroup of \({ G }\).
    \end{cloze}
\end{note}

\begin{note}{2a685e6476b94b9eac539a17441574ef}
    Why is the notion of the torsion subgroup introduced only for abelian groups?

    \begin{cloze}{1}
        For non-abelian groups the set is not guaranteed to form a subgroup.
    \end{cloze}
\end{note}

\begin{note}{22a771a961c3498f88a030fabf778797}
    Give an example of a non-abelian group, who's ``torsion subgroup'' is not actually a subgroup.

    \begin{cloze}{1}
        \({ GL_3(\mathbb R) }\)
    \end{cloze}
\end{note}

\begin{note}{f4edb9436c094103b0b9b82019185296}
    Give an example of two elements \({ a, b }\) in \({ GL_3(\mathbb R) }\) such that
    \[
        \left\lvert a \right\rvert, \left\lvert b \right\rvert < \infty \quad \text{and} \quad \left\lvert ab \right\rvert = \infty.
    \]

    \begin{cloze}{1}
        \[
            \begin{bmatrix}
                1 & 1  & 0 \\
                0 & -1 & 0 \\
                0 & 0  & 1
            \end{bmatrix}\,,
            \quad
            \begin{bmatrix}
                1 & 0  & 0 \\
                0 & -1 & 1 \\
                0 & 0  & 1
            \end{bmatrix}\,.
        \]
    \end{cloze}
\end{note}

\begin{note}{f2c41a74f8a04bb892b471915e533055}
    What is the torsion subgroup of \({ \mathbb Z \times (\mathbb Z / n\mathbb Z) }\)?

    \begin{cloze}{1}
        The set of elements who's first component is \({ 0 }\).
    \end{cloze}
\end{note}

\begin{note}{4df6b5997d30483fb469565c89630322}
    When is the union of two subgroups also a subgroup?

    \begin{cloze}{1}
        If and only if one of the subgroups is a subset of the other.
    \end{cloze}
\end{note}

\begin{note}{b523dcaec101461e902f34191451e112}
    When is the union of an infinite number of subgroups also a subgroup?

    \begin{cloze}{1}
        It depends.
    \end{cloze}
\end{note}

\begin{note}{60129b39ceab4468915a6d2237915c1a}
    Let \({ H }\) and \({ K }\) be subgroups of \({ G }\) and \({ H \subseteq K }\).
    What do we know about \({ H \cup K }\)?

    \begin{cloze}{1}
        It is a subgroup of \({ G }\).
    \end{cloze}
\end{note}

\begin{note}{791301f78ecf4800a13e3a0299c57028}
    Let \({ H }\) and \({ K }\) be subgroups of \({ G }\).
    If \({ H \cup K }\) is a subgroup of \({ G }\), then \({ H \subseteq K }\) or \({ K \subseteq H }\).
    What is the key idea in the proof?

    \begin{cloze}{1}
        By contradiction.
    \end{cloze}
\end{note}

\begin{note}{cc8decdf60194667b3b27ff0941c9fc0}
    What is the special linear group?

    \begin{cloze}{1}
        The set of square matrices who's determinant is \({ 1 }\).
    \end{cloze}
\end{note}

\begin{note}{1e420dd97e1942b3b7bc70d71fc0953e}
    \begin{icloze}{2}The special linear group of \({ n \times n }\) matrices over a field \({ F }\)\end{icloze} is denoted \begin{icloze}{1}\({ SL_n(F) }\).\end{icloze}
\end{note}

\begin{note}{941aeeb281da4b009ffdc95864eddb3b}
    When is the intersection of two subgroups also a subgroup?

    \begin{cloze}{1}
        Always.
    \end{cloze}
\end{note}

\begin{note}{887cf7600d994fcd9662e35fc9719c62}
    When is the intersection of an infinite number of subgroups also a subgroup?

    \begin{cloze}{1}
        Always.
    \end{cloze}
\end{note}

\begin{note}{3bdd7a0f0e044c6b9c3c1811d4478f10}
    Let \({ H_1 \leq H_2 \leq \cdots }\) be an ascending chain of subgroups of \({ G }\).
    Then \begin{icloze}{1}\({ \bigcup_{i=1}^{\infty} H_i }\) is a subgroup of \({ G }\).\end{icloze}
\end{note}

\section{Centralizers and Normalizers, Stabilizers and Kernels}
\begin{note}{93de251693e74655a5752529379e7081}
    For what do we define centralizers in groups?

    \begin{cloze}{1}
        For nonempty subsets of the group.
    \end{cloze}
\end{note}

\begin{note}{b46233e067ea4c24b38af57081ef1db3}
    Let \({ G }\) be a group and \({ A }\) be a nonempty subset of \({ G }\).
    \begin{icloze}{1}
        The set
        \[
            \left\{ g \in G \mid gag^{-1} = a \text{ for all } a \in A \right\}
        \]
    \end{icloze}
    is called \begin{icloze}{2}the centralizer of \({ A }\) in \({ G }\).\end{icloze}
\end{note}

\begin{note}{3c2adb104b55494a8a248b4e6cf72980}
    Let \({ G }\) be a group and \({ A }\) be a nonempty subset of \({ G }\).
    \begin{icloze}{2}The centralizer of \({ A }\) in \({ G }\)\end{icloze} is denoted
    \begin{icloze}{1}
        \[
            C_G(A)\,.
        \]
    \end{icloze}
\end{note}

\begin{note}{aeea9d02d1a8429ab94927313c1e2194}
    How can centralizers be redefined in terms of commutativity?

    \begin{cloze}{1}
        As the set of all the elements that commute with every element of the subset.
    \end{cloze}
\end{note}

\begin{note}{588fd51b4281485c87a74faa9ddbf8f5}
    Let \({ G }\) be a group and \({ A }\) be a nonempty subset of \({ G }\).
    The centralizer of \({ A }\) in \({ G }\) forms \begin{icloze}{1}a subgroup of \({ G }\).\end{icloze}
\end{note}

\begin{note}{18e60ffefbe647a6aa9b7a9feeb58ef1}
    Let \({ G }\) be a group and \({ A }\) be a nonempty subset of \({ G }\).
    When is the centralizer of \({ A }\) in \({ G }\) a subgroup of \({ G }\)?

    \begin{cloze}{1}
        Always.
    \end{cloze}
\end{note}

\begin{note}{23eee6bafc20447987eaab729108324e}
    Let \({ G }\) be a group and \({ A }\) be a nonempty subset of \({ G }\).
    In the special case when \({ A = \left\{ a \right\} }\) we shall write \begin{icloze}{1}simply \({ C_{G}(a) }\)\end{icloze} instead of \begin{icloze}{2}\({ C_{G}(\left\{ a \right\}) }\).\end{icloze}
\end{note}

\begin{note}{92e1f52031224232bf8ac69f4014862c}
    Let \({ G }\) be a group and \({ a \in G }\).
    Then
    \[
        \begin{icloze}{1}\langle a \rangle\end{icloze} \subseteq C_G(a)\,.
    \]
\end{note}

\begin{note}{656f69d38b7e4f41882f7feda5410dde}
    \[
        C_{Q_8}(i) = \begin{icloze}{1}\left\{ 1, -1, i, -i \right\}\,.\end{icloze}
    \]
\end{note}

\begin{note}{1fd69e94ef324e30a0054ea4860105e4}
    \[
        C_{Q_8}(1) = \begin{icloze}{1}Q_8\,.\end{icloze}
    \]
\end{note}

\begin{note}{1ec04d8a609e442690da0ee9332a9647}
    For what do we define centers in groups?

    \begin{cloze}{1}
        For the group itself.
    \end{cloze}
\end{note}

\begin{note}{936431cf3df24996965ce022800fa1bc}
    Let \({ G }\) be a group.
    \begin{icloze}{2}The set of elements of \({ G }\) commuting with all elements of \({ G }\)\end{icloze} is called \begin{icloze}{1}the center of \({ G }\).\end{icloze}
\end{note}

\begin{note}{4fdf7ab9640d453a9eb90b77b45f35b2}
    Let \({ G }\) be a group.
    \begin{icloze}{2}The center of \({ G }\)\end{icloze} is denoted \begin{icloze}{1}\({ Z(G) }\).\end{icloze}
\end{note}

\begin{note}{b7d3c0377db64825b9428611799a938c}
    Let \({ G }\) be a group.
    The center of \({ G }\) forms \begin{icloze}{1}a subgroup of \({ G }\).\end{icloze}
\end{note}

\begin{note}{3c3e0bd81b194850bbeed0d6688646ea}
    Let \({ G }\) be a group.
    When is the center of \({ G }\) a subgroup of \({ G }\)?

    \begin{cloze}{1}
        Always.
    \end{cloze}
\end{note}

\begin{note}{10b56b05fab34d6a91d494a9c515f2a4}
    Let \({ G }\) be a group.
    \begin{icloze}{2}The center of \({ G }\)\end{icloze} is the centralizer of \begin{icloze}{1}\({ G }\) in \({ G }\).\end{icloze}
\end{note}

\begin{note}{4c598c0713ef4d649fe3629dfcd8a0c7}
    For what do we define normalizers in groups?

    \begin{cloze}{1}
        For nonempty subsets.
    \end{cloze}
\end{note}

\begin{note}{ea05e39de520479892867fd132778337}
    Let \({ G }\) be a group and \({ A }\) be a nonempty subset of \({ G }\).
    \begin{icloze}{2}
        The set
        \[
            \left\{ g \in G \mid gAg^{-1} = A \right\}
        \]
    \end{icloze}
    is called \begin{icloze}{1}the normalizer of \({ A }\) in \({ G }\).\end{icloze}
\end{note}

\begin{note}{c4b4c4424c6549c6b9afccb2945d17ee}
    Let \({ G }\) be a group and \({ A }\) be a nonempty subset of \({ G }\).
    \begin{icloze}{2}The normalizer of \({ A }\) in \({ G }\)\end{icloze} is denoted
    \begin{icloze}{1}
        \[
            N_G(A)\,.
        \]
    \end{icloze}
\end{note}

\begin{note}{f831383807f34538b90b130d417dfb95}
    Let \({ G }\) be a group and \({ A }\) be a nonempty subset of \({ G }\).
    The normalizer of \({ A }\) in \({ G }\) forms \begin{icloze}{1}a subgroup of \({ G }\).\end{icloze}
\end{note}

\begin{note}{b536e255fed54b02a1036b9baf6a7dc6}
    Let \({ G }\) be a group and \({ A }\) be a nonempty subset of \({ G }\).
    When is the normalizer of \({ A }\) in \({ G }\) a subgroup of \({ G }\)?

    \begin{cloze}{1}
        Always.
    \end{cloze}
\end{note}

\end{document}
