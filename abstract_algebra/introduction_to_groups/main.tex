\documentclass[11pt, a5paper]{article}
\usepackage[width=10cm, top=0.5cm, bottom=2cm]{geometry}

\usepackage[T1,T2A]{fontenc}
\usepackage[utf8]{inputenc}
\usepackage[english,russian]{babel}
\usepackage{libertine}

\usepackage{amsmath}
\usepackage{amssymb}
\usepackage{amsthm}
\usepackage{mathrsfs}
\usepackage{framed}
\usepackage{xcolor}

\setlength{\parindent}{0pt}

% Force \pagebreak for every section
\let\oldsection\section
\renewcommand\section{\pagebreak\oldsection}

\renewcommand{\thesection}{}
\renewcommand{\thesubsection}{Note \arabic{subsection}}
\renewcommand{\thesubsubsection}{}
\renewcommand{\theparagraph}{}

\newenvironment{note}[1]{\goodbreak\par\subsection{\hfill \color{lightgray}\tiny #1}}{}
\newenvironment{cloze}[2][\ldots]{\begin{leftbar}}{\end{leftbar}}
\newenvironment{icloze}[2][\ldots]{%
  \text{\tiny \color{lightgray}\{\{c#2::}\hspace{0pt}\ignorespaces%
}{%
  \unskip\hspace{0pt}\text{\tiny\color{lightgray}\}\}}%
}


\begin{document}
\section{Basic Axioms and Examples}
\begin{note}{cca4f1927b2c4eeaa3123dbcf0680bc0}
    Given a set \({ G }\), \begin{icloze}{2}a binary operation \({ \star }\) on \({ G }\)\end{icloze} is
    \begin{icloze}{1}
        a function
        \[
            \star : G \times G \to G.
        \]
    \end{icloze}
\end{note}

\begin{note}{7732d25ebb1e40dd9696c1c921803c17}
    Given a binary operation \({ \star }\) on a set \({ G }\),
    for any \({ a, b \in G }\) we shall write \begin{icloze}{2}\({ a \star b }\)\end{icloze} for \begin{icloze}{1}\({ \star(a, b) }\).\end{icloze}
\end{note}

\begin{note}{4fc60827250f4af4ab6a669ac7632568}
    A binary operation \({ \star }\) on a set \({ G }\) is \begin{icloze}{2}associative\end{icloze} if
    \begin{icloze}{1}
        for all \({ a, b, c \in G }\) we have
        \[
            a \star (b \star c) = (a \star b) \star c.
        \]
    \end{icloze}
\end{note}

\begin{note}{192d8d86f22349cabcd9f4229fc45290}
    If \({ \star }\) is a binary operation on a set \({ G }\) we say elements \({ a }\) and \({ b }\) of \({ G }\) \begin{icloze}{1}commute\end{icloze} if
    \begin{icloze}{2}
        \[
            a \star b = b \star a.
        \]
    \end{icloze}
\end{note}

\begin{note}{e5cbf512d6a54c91950c65450a07a501}
    A binary operation \({ \star }\) on a set \({ G }\) is \begin{icloze}{2}commutative\end{icloze} if
    \begin{icloze}{1}
        for all \({ a, b \in G }\) we have
        \[
            a \star b = b \star a.
        \]
    \end{icloze}
\end{note}

\begin{note}{36b096eebd7f4264ab071a5fa4eefe13}
    Suppose that \({ \star }\) is a binary operation on a set \({ G }\) and \({ H \subseteq G }\).
    If \begin{icloze}{2}the restriction of \({ \star }\) to \({ H }\) is a binary operation on \({ H }\),\end{icloze} then \({ H }\) is said to be \begin{icloze}{1}closed under \({ \star }\).\end{icloze}
\end{note}

\begin{note}{644b1cd8fa014885ad295ae5c089e5a7}
    \begin{icloze}{3}A group\end{icloze} is \begin{icloze}{2}an ordered pair \({ (G, \star) }\)\end{icloze} where \begin{icloze}{1}\({ G }\) is a set and \({ \star }\) is a binary operation on \({ G }\)\end{icloze} satisfying \begin{icloze}{4}the group axioms.\end{icloze}
\end{note}

\begin{note}{5de4e717b4814adf8aed4f8d9a93322c}
    How many axiom are there in the definition of a group \({ (G, \star) }\)?

    \begin{cloze}{1}
        Three.
    \end{cloze}
\end{note}

\begin{note}{2de690f5008a4b8c8691e36308e44295}
    What is the first axiom from the definition of a group \({ (G, \star) }\)?

    \begin{cloze}{1}
        \({ \star }\) is associative.
    \end{cloze}
\end{note}

\begin{note}{4fcc137e66a048459cc73d6735e4ccea}
    Given a binary operation \({ \star }\) on a set \({ G }\), \begin{icloze}{3}an element \({ e \in G }\)\end{icloze} is called \begin{icloze}{2}an identity of \({ G }\)\end{icloze} if
    \begin{icloze}{1}
        for all \({ a \in G }\) we have
        \[
            a \star e = e \star a = a.
        \]
    \end{icloze}
\end{note}

\begin{note}{a3cd125f152f432082757242096a76ef}
    What is the second axiom from the definition of a group \({ (G, \star) }\)?

    \begin{cloze}{1}
        There exists an identity of \({ G }\).
    \end{cloze}
\end{note}

\begin{note}{5d438f0c3fb24b1a97507e81f868846e}
    Given a binary operation \({ \star }\) on a set \({ G }\) and \({ a \in G }\), \begin{icloze}{3}an element \({ \tilde a \in G }\)\end{icloze} is called \begin{icloze}{2}an inverse of \({ a }\)\end{icloze} if
    \begin{icloze}{1}
        \[
            a \star \tilde a = \tilde a \star a = e.
        \]
    \end{icloze}
\end{note}

\begin{note}{d840b7b910d740f3bea231c74feba51c}
    Given a binary operation \({ \star }\) on a set \({ G }\) and \({ a \in G }\), \begin{icloze}{2}an inverse of \({ a }\)\end{icloze} is usually denoted \begin{icloze}{1}\({ a^{-1} }\).\end{icloze}
\end{note}

\begin{note}{c4c56a11c6f746b3ae287ee386b4e12b}
    What is the third axiom from the definition of a group \({ (G, \star) }\)?

    \begin{cloze}{1}
        For all \({ a \in G }\) there exists \({ a^{-1} }\).
    \end{cloze}
\end{note}

\begin{note}{be05e23d350d4f49a65602b65045f888}
    A group \({ (G, \star) }\) is called \begin{icloze}{2}abelian\end{icloze} if \begin{icloze}{1}\({ \star }\) is commutative.\end{icloze}
\end{note}

\begin{note}{978f23382d594a28a3de168b7f661c30}
    We shall say \({ G }\) is \begin{icloze}{2}a group under \({ \star }\)\end{icloze} if \begin{icloze}{1}\({ (G, \star) }\) is a group.\end{icloze}
\end{note}

\begin{note}{497f01593d7f4ffabb546b455788b354}
    We shall say a set \({ G }\) is \begin{icloze}{2}a group\end{icloze} if \begin{icloze}{1}\({ G }\) is a group under an operation that is clear from the context.\end{icloze}
\end{note}

\begin{note}{61ea2504ca474fe4aae902eb1965576c}
    \({ \mathbb Z, \mathbb Q, \mathbb R }\) and \({ \mathbb C }\) are \begin{icloze}{2}groups\end{icloze} under \begin{icloze}{1}\({ + }\).\end{icloze}
\end{note}

\begin{note}{84b6a231d3934ab3b4f63226549a9589}
    \({ \mathbb Q - \left\{ 0 \right\},\: \mathbb R - \left\{ 0 \right\},\: \mathbb C - \left\{ 0 \right\} }\) are \begin{icloze}{2}groups\end{icloze} under \begin{icloze}{1}\({ \times }\).\end{icloze}
\end{note}

\begin{note}{3051cd354f5040e2bdf0809e005635ed}
    \({ \mathbb Q^{+}, \mathbb R^{+} }\) are \begin{icloze}{2}groups\end{icloze} under \begin{icloze}{1}\({ \times }\).\end{icloze}
\end{note}

\begin{note}{21f924e833cd4e0bbae5f4588dff47b5}
    Is \({ \mathbb Z - \left\{ 0 \right\} }\) a group under \({ \times }\)?

    \begin{cloze}{1}
        No. (There is no inverse.)
    \end{cloze}
\end{note}

\begin{note}{edec2a960f6d43dbb5e19283c28db7bd}
    Let \({ V }\) be a vector space. Then \({ V }\) is \begin{icloze}{2}a group\end{icloze} under \begin{icloze}{1}\({ + }\).\end{icloze}
\end{note}

\begin{note}{47a03e2c688244b1b3a5126fd04a21c7}
    Let \({ n \in \mathbb Z^{+} }\). Then \begin{icloze}{3}\({ \mathbb Z / n\mathbb Z }\)\end{icloze} is \begin{icloze}{2}a group\end{icloze} under \begin{icloze}{1}addition\end{icloze} of residue classes.
\end{note}

\begin{note}{f6a5a40cfee6495dae0d36f7b3288cb2}
    Let \({ n \in \mathbb Z^{+} }\). Then \begin{icloze}{3}\({ \left( \mathbb Z / n\mathbb Z \right)^{\times} }\)\end{icloze} is \begin{icloze}{2}a group\end{icloze} under \begin{icloze}{1}multiplication\end{icloze} of residue classes.
\end{note}

\begin{note}{3e94ca73ca344269bb98d94a22204fd9}
    If \({ (A, \star) }\) and \({ (B, \diamond) }\) are \begin{icloze}{4}groups,\end{icloze} then the group \begin{icloze}{2}\({ A \times B }\),\end{icloze} whose operation is
    \begin{icloze}{1}
        defined componentwise:
        \[
            (a, b)(c, d) = (a \star c, b \diamond d),
        \]
    \end{icloze}
    is called \begin{icloze}{3}the direct product of the two groups.\end{icloze}
\end{note}

\begin{note}{e23d8e577b3948af9b0cadd5df7c9141}
    If \({ (G, \star) }\) is a group, then \begin{icloze}{2}the identity of \({ G }\)\end{icloze} is \begin{icloze}{1}unique.\end{icloze}
\end{note}

\begin{note}{5b5391986e9b49ea9c5f9f73813e9594}
    If \({ (G, \star) }\) is a group, then the identity of \({ G }\) is unique.
    What is the key idea in the proof?

    \begin{cloze}{1}
        Consider the product of two arbitrary identities.
    \end{cloze}
\end{note}

\begin{note}{0989a259fae446c48bb0f6c40394efd0}
    If \({ (G, \star) }\) is a group, then for every \({ a \in G }\), \begin{icloze}{2}\({ a^{-1} }\)\end{icloze} is \begin{icloze}{1}uniquely determined.\end{icloze}
\end{note}

\begin{note}{f0b0a651592c466ba8067beb3b1570b8}
    If \({ (G, \star) }\) is a group, then for every \({ a \in G }\), \({ a^{-1} }\) is uniquely determined.
    What is the key idea in the proof?

    \begin{cloze}{1}
        Multiply an inverse on the right by \({ a \star a^{-1} }\).
    \end{cloze}
\end{note}

\begin{note}{4a6a6806d8874839bb7956d76e384333}
    If \({ (G, \star) }\) is a group and \({ a \in G }\), then
    \[
        (a^{-1})^{-1} = \begin{icloze}{1}a.\end{icloze}
    \]
\end{note}

\begin{note}{9ab0e972d6a24baea99f1577ebf03423}
    If \({ (G, \star) }\) is a group and \({ a, b \in G }\), then
    \[
        \begin{icloze}{2}(a \star b)^{-1}\end{icloze} = \begin{icloze}{1}(b^{-1}) \star (a^{-1}).\end{icloze}
    \]
\end{note}

\begin{note}{69b3db6e70ad4629aa55a855b8df8096}
    If \({ (G, \star) }\) is a group and \({ a_1, \ldots, a_n \in G }\), then
    the value of
    \[
        a_1 \star \cdots \star a_n
    \]
    is \begin{icloze}{2}independent\end{icloze} of \begin{icloze}{1}how the expression is bracketed.\end{icloze}

    \begin{center}
        \tiny
        <<\begin{icloze}{3}The generalized associative law\end{icloze}>>
    \end{center}
\end{note}

\begin{note}{05cc8fd523084650adb46704dde222a7}
    What is the key idea in the proof of the generalized associative law for a group \({ (G, \star) }\)?

    \begin{cloze}{1}
        By induction.
    \end{cloze}
\end{note}

\begin{note}{9ca193d1531c4c49b296732d7ff12fb5}
    Henceforth our abstract groups \({ G }\), \({ H }\), \textit{etc.} will always be written with the operation as \begin{icloze}{1}\({ \cdot }\).\end{icloze}
\end{note}

\begin{note}{7d06acac21c14a628ad1ccb470fe6398}
    Henceforth for an abstract group \({ G }\) (operation \({ \cdot }\))
    an expression \begin{icloze}{2}\({ a \cdot b }\)\end{icloze} will always be written as \begin{icloze}{1}\({ ab }\).\end{icloze}
\end{note}

\begin{note}{0994e6080f3042ad81bc90d1ced0b747}
    Henceforth for an abstract group \({ G }\) (operation \({ \cdot }\))
    we denote \begin{icloze}{2}the identity of \({ G }\)\end{icloze} by \begin{icloze}{1}\({ 1 }\).\end{icloze}
\end{note}

\begin{note}{361c99f13a9b4304868fcdb350b45dbf}
    For any group \({ G }\) and \({ x \in G }\) and \begin{icloze}{3}\({ n \in \mathbb Z^{+} }\)\end{icloze}
    we shall denote by \begin{icloze}{2}\({ x^{n} }\)\end{icloze}
    \begin{icloze}{1}
        the product
        \[
            \underbrace{x x \cdots x}_{\text{\({ n }\) terms}}.
        \]
    \end{icloze}
\end{note}

\begin{note}{5b7f3c41cf0147e2bffc3929ed9ec480}
    For any group \({ G }\) and \({ x \in G }\) and \begin{icloze}{3}\({ n \in \mathbb Z^{+} }\)\end{icloze}
    we shall denote by \begin{icloze}{2}\({ x^{-n} }\)\end{icloze}
    \begin{icloze}{1}
        the product
        \[
            \underbrace{x^{-1} x^{-1} \cdots x^{-1}}_{\text{\({ n }\) terms}}.
        \]
    \end{icloze}
\end{note}

\begin{note}{a7a44229ee0f4a44b11d1410dc0fab0f}
    For any group \({ G }\) and \begin{icloze}{3}\({ x \in G }\),\end{icloze} let \({ x^{\begin{icloze}{2}0\end{icloze}} \overset{\text{def}}= \begin{icloze}{1}1, \text{the identity of \({ G }\)}\end{icloze} }\).
\end{note}

\begin{note}{ed8673154d544c7b86ac358facc79101}
    For \({ G }\) a group and \({ x \in G }\)
    define \begin{icloze}{2}the order of \({ x }\)\end{icloze} to be
    \begin{icloze}{1}
        the smallest positive integer \({ n }\) such that
        \[
            x^{n} = 1.
        \]
    \end{icloze}
\end{note}

\begin{note}{8c334a6360be4bee8fae7f712ab2c4ee}
    For \({ G }\) a group and \({ x \in G }\), if \begin{icloze}{2}no positive power of \({ x }\) is the identity,\end{icloze} \begin{icloze}{3}the order of \({ x }\)\end{icloze} is defined to be \begin{icloze}{1}infinity.\end{icloze}
\end{note}

\begin{note}{ba4143a322564f8383f6e7d91ca32a75}
    For \({ G }\) a group and \({ x \in G }\), denote \begin{icloze}{2}the order of \({ x }\)\end{icloze} by \begin{icloze}{1}\({ \left\lvert x \right\rvert }\).\end{icloze}
\end{note}

\begin{note}{d7fee5bcbdbd47bcb6f4a2ba086fa2ed}
    For \({ G }\) a group and \({ x \in G }\), if \begin{icloze}{2}the order of \({ x }\) is an integer \({ n }\),\end{icloze} \({ x }\) is said to be \begin{icloze}{1}of order \({ n }\).\end{icloze}
\end{note}

\begin{note}{db12c606699d40e89d499d554bd52b28}
    For \({ G }\) a group and \({ x \in G }\), if \begin{icloze}{2}the order of \({ x }\) is infinite,\end{icloze} \({ x }\) is said to be \begin{icloze}{1}of infinite order.\end{icloze}
\end{note}

\begin{note}{2e514c62ce4e48eb9c6bd3b5de1d7c44}
    An element of a group has order \({ 1 }\) \begin{icloze}{2}if and only if\end{icloze} \begin{icloze}{1}it is the identity.\end{icloze}
\end{note}

\begin{note}{babeb7cf1b394be6a4f8d86e1a099cda}
    Let \({ G = \left\{ g_1, g_2, \ldots, g_n \right\} }\) be \begin{icloze}{4}a finite group\end{icloze} with \begin{icloze}{3}\({ g_1 = 1 }\).\end{icloze}
    The \begin{icloze}{2}multiplication table\end{icloze} or \begin{icloze}{2}group table\end{icloze} of \({ G }\) is
    \begin{icloze}{1}
        the matrix
        \[
            \Big[ g_i g_j \Big] \sim n \times n.
        \]
    \end{icloze}
\end{note}

\end{document}
