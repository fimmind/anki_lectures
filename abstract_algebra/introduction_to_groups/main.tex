%! TeX root = ./main.tex
\documentclass[11pt, a5paper]{article}
\usepackage[width=10cm, top=0.5cm, bottom=2cm]{geometry}

\usepackage[T1,T2A]{fontenc}
\usepackage[utf8]{inputenc}
\usepackage[english,russian]{babel}
\usepackage{libertine}

\usepackage{amsmath}
\usepackage{amssymb}
\usepackage{amsthm}
\usepackage{mathrsfs}
\usepackage{framed}
\usepackage{xcolor}

\setlength{\parindent}{0pt}

% Force \pagebreak for every section
\let\oldsection\section
\renewcommand\section{\pagebreak\oldsection}

\renewcommand{\thesection}{}
\renewcommand{\thesubsection}{Note \arabic{subsection}}
\renewcommand{\thesubsubsection}{}
\renewcommand{\theparagraph}{}

\newenvironment{note}[1]{\goodbreak\par\subsection{\hfill \color{lightgray}\tiny #1}}{}
\newenvironment{cloze}[2][\ldots]{\begin{leftbar}}{\end{leftbar}}
\newenvironment{icloze}[2][\ldots]{%
  \ignorespaces\text{\tiny \color{lightgray}\{\{c#2::}\hspace{0pt}%
}{%
  \hspace{0pt}\text{\tiny\color{lightgray}\}\}}\unskip%
}


\begin{document}
\section{Basic Axioms and Examples}
\begin{note}{cca4f1927b2c4eeaa3123dbcf0680bc0}
    Given a set \({ G }\), \begin{icloze}{2}a binary operation \({ \star }\) on \({ G }\)\end{icloze} is
    \begin{icloze}{1}
        a function
        \[
            \star : G \times G \to G.
        \]
    \end{icloze}
\end{note}

\begin{note}{7732d25ebb1e40dd9696c1c921803c17}
    Given a binary operation \({ \star }\) on a set \({ G }\),
    for any \({ a, b \in G }\) we shall write \begin{icloze}{2}\({ a \star b }\)\end{icloze} for \begin{icloze}{1}\({ \star(a, b) }\).\end{icloze}
\end{note}

\begin{note}{4fc60827250f4af4ab6a669ac7632568}
    A binary operation \({ \star }\) on a set \({ G }\) is \begin{icloze}{2}associative\end{icloze} if
    \begin{icloze}{1}
        for all \({ a, b, c \in G }\) we have
        \[
            a \star (b \star c) = (a \star b) \star c.
        \]
    \end{icloze}
\end{note}

\begin{note}{192d8d86f22349cabcd9f4229fc45290}
    If \({ \star }\) is a binary operation on a set \({ G }\) we say elements \({ a }\) and \({ b }\) of \({ G }\) \begin{icloze}{1}commute\end{icloze} if
    \begin{icloze}{2}
        \[
            a \star b = b \star a.
        \]
    \end{icloze}
\end{note}

\begin{note}{e5cbf512d6a54c91950c65450a07a501}
    A binary operation \({ \star }\) on a set \({ G }\) is \begin{icloze}{2}commutative\end{icloze} if
    \begin{icloze}{1}
        for all \({ a, b \in G }\) we have
        \[
            a \star b = b \star a.
        \]
    \end{icloze}
\end{note}

\begin{note}{36b096eebd7f4264ab071a5fa4eefe13}
    Suppose that \({ \star }\) is a binary operation on a set \({ G }\) and \({ H \subseteq G }\).
    If \begin{icloze}{2}the restriction of \({ \star }\) to \({ H }\) is a binary operation on \({ H }\),\end{icloze} then \({ H }\) is said to be \begin{icloze}{1}closed under \({ \star }\).\end{icloze}
\end{note}

\begin{note}{644b1cd8fa014885ad295ae5c089e5a7}
    \begin{icloze}{3}A group\end{icloze} is \begin{icloze}{2}an ordered pair \({ (G, \star) }\)\end{icloze} where \begin{icloze}{1}\({ G }\) is a set and \({ \star }\) is a binary operation on \({ G }\)\end{icloze} satisfying \begin{icloze}{4}the group axioms.\end{icloze}
\end{note}

\begin{note}{5de4e717b4814adf8aed4f8d9a93322c}
    How many axiom are there in the definition of a group \({ (G, \star) }\)?

    \begin{cloze}{1}
        Three.
    \end{cloze}
\end{note}

\begin{note}{2de690f5008a4b8c8691e36308e44295}
    What is the first axiom from the definition of a group \({ (G, \star) }\)?

    \begin{cloze}{1}
        \({ \star }\) is associative.
    \end{cloze}
\end{note}

\begin{note}{4fcc137e66a048459cc73d6735e4ccea}
    Given a binary operation \({ \star }\) on a set \({ G }\), \begin{icloze}{3}an element \({ e \in G }\)\end{icloze} is called \begin{icloze}{2}an identity of \({ G }\)\end{icloze} if
    \begin{icloze}{1}
        for all \({ a \in G }\) we have
        \[
            a \star e = e \star a = a.
        \]
    \end{icloze}
\end{note}

\begin{note}{a3cd125f152f432082757242096a76ef}
    What is the second axiom from the definition of a group \({ (G, \star) }\)?

    \begin{cloze}{1}
        There exists an identity of \({ G }\).
    \end{cloze}
\end{note}

\begin{note}{5d438f0c3fb24b1a97507e81f868846e}
    Given a binary operation \({ \star }\) on a set \({ G }\) and \({ a \in G }\), \begin{icloze}{3}an element \({ \tilde a \in G }\)\end{icloze} is called \begin{icloze}{2}an inverse of \({ a }\)\end{icloze} if
    \begin{icloze}{1}
        \[
            a \star \tilde a = \tilde a \star a = e.
        \]
    \end{icloze}
\end{note}

\begin{note}{d840b7b910d740f3bea231c74feba51c}
    Given a binary operation \({ \star }\) on a set \({ G }\) and \({ a \in G }\), \begin{icloze}{2}an inverse of \({ a }\)\end{icloze} is usually denoted \begin{icloze}{1}\({ a^{-1} }\).\end{icloze}
\end{note}

\begin{note}{c4c56a11c6f746b3ae287ee386b4e12b}
    What is the third axiom from the definition of a group \({ (G, \star) }\)?

    \begin{cloze}{1}
        For all \({ a \in G }\) there exists \({ a^{-1} }\).
    \end{cloze}
\end{note}

\begin{note}{be05e23d350d4f49a65602b65045f888}
    A group \({ (G, \star) }\) is called \begin{icloze}{2}abelian\end{icloze} if \begin{icloze}{1}\({ \star }\) is commutative.\end{icloze}
\end{note}

\begin{note}{978f23382d594a28a3de168b7f661c30}
    We shall say \({ G }\) is \begin{icloze}{2}a group under \({ \star }\)\end{icloze} if \begin{icloze}{1}\({ (G, \star) }\) is a group.\end{icloze}
\end{note}

\begin{note}{497f01593d7f4ffabb546b455788b354}
    We shall say a set \({ G }\) is \begin{icloze}{2}a group\end{icloze} if \begin{icloze}{1}\({ G }\) is a group under an operation that is clear from the context.\end{icloze}
\end{note}

\begin{note}{61ea2504ca474fe4aae902eb1965576c}
    \({ \mathbb Z, \mathbb Q, \mathbb R }\) and \({ \mathbb C }\) are \begin{icloze}{2}groups\end{icloze} under \begin{icloze}{1}\({ + }\).\end{icloze}
\end{note}

\begin{note}{84b6a231d3934ab3b4f63226549a9589}
    \({ \mathbb Q - \left\{ 0 \right\},\: \mathbb R - \left\{ 0 \right\},\: \mathbb C - \left\{ 0 \right\} }\) are \begin{icloze}{2}groups\end{icloze} under \begin{icloze}{1}\({ \times }\).\end{icloze}
\end{note}

\begin{note}{3051cd354f5040e2bdf0809e005635ed}
    \({ \mathbb Q^{+}, \mathbb R^{+} }\) are \begin{icloze}{2}groups\end{icloze} under \begin{icloze}{1}\({ \times }\).\end{icloze}
\end{note}

\begin{note}{21f924e833cd4e0bbae5f4588dff47b5}
    Is \({ \mathbb Z - \left\{ 0 \right\} }\) a group under \({ \times }\)?

    \begin{cloze}{1}
        No. (There is no inverse.)
    \end{cloze}
\end{note}

\begin{note}{edec2a960f6d43dbb5e19283c28db7bd}
    Let \({ V }\) be a vector space. Then \({ V }\) is \begin{icloze}{2}a group\end{icloze} under \begin{icloze}{1}\({ + }\).\end{icloze}
\end{note}

\begin{note}{47a03e2c688244b1b3a5126fd04a21c7}
    Let \({ n \in \mathbb Z^{+} }\). Then \begin{icloze}{3}\({ \mathbb Z / n\mathbb Z }\)\end{icloze} is \begin{icloze}{2}a group\end{icloze} under \begin{icloze}{1}addition\end{icloze} of residue classes.
\end{note}

\begin{note}{f6a5a40cfee6495dae0d36f7b3288cb2}
    Let \({ n \in \mathbb Z^{+} }\). Then \begin{icloze}{3}\({ \left( \mathbb Z / n\mathbb Z \right)^{\times} }\)\end{icloze} is \begin{icloze}{2}a group\end{icloze} under \begin{icloze}{1}multiplication\end{icloze} of residue classes.
\end{note}

\begin{note}{3e94ca73ca344269bb98d94a22204fd9}
    If \({ (A, \star) }\) and \({ (B, \diamond) }\) are \begin{icloze}{4}groups,\end{icloze} then the group \begin{icloze}{2}\({ A \times B }\),\end{icloze} whose operation is
    \begin{icloze}{1}
        defined componentwise:
        \[
            (a, b)(c, d) = (a \star c, b \diamond d),
        \]
    \end{icloze}
    is called \begin{icloze}{3}the direct product of the two groups.\end{icloze}
\end{note}

\begin{note}{e23d8e577b3948af9b0cadd5df7c9141}
    If \({ (G, \star) }\) is a group, then \begin{icloze}{2}the identity of \({ G }\)\end{icloze} is \begin{icloze}{1}unique.\end{icloze}
\end{note}

\begin{note}{5b5391986e9b49ea9c5f9f73813e9594}
    If \({ (G, \star) }\) is a group, then the identity of \({ G }\) is unique.
    What is the key idea in the proof?

    \begin{cloze}{1}
        Consider the product of two arbitrary identities.
    \end{cloze}
\end{note}

\begin{note}{0989a259fae446c48bb0f6c40394efd0}
    If \({ (G, \star) }\) is a group, then for every \({ a \in G }\), \begin{icloze}{2}\({ a^{-1} }\)\end{icloze} is \begin{icloze}{1}uniquely determined.\end{icloze}
\end{note}

\begin{note}{f0b0a651592c466ba8067beb3b1570b8}
    If \({ (G, \star) }\) is a group, then for every \({ a \in G }\), \({ a^{-1} }\) is uniquely determined.
    What is the key idea in the proof?

    \begin{cloze}{1}
        Multiply an inverse on the right by \({ a \star a^{-1} }\).
    \end{cloze}
\end{note}

\begin{note}{4a6a6806d8874839bb7956d76e384333}
    If \({ (G, \star) }\) is a group and \({ a \in G }\), then
    \[
        (a^{-1})^{-1} = \begin{icloze}{1}a.\end{icloze}
    \]
\end{note}

\begin{note}{9ab0e972d6a24baea99f1577ebf03423}
    If \({ (G, \star) }\) is a group and \({ a, b \in G }\), then
    \[
        \begin{icloze}{2}(a \star b)^{-1}\end{icloze} = \begin{icloze}{1}(b^{-1}) \star (a^{-1}).\end{icloze}
    \]
\end{note}

\begin{note}{69b3db6e70ad4629aa55a855b8df8096}
    If \({ (G, \star) }\) is a group and \({ a_1, \ldots, a_n \in G }\), then
    the value of
    \[
        a_1 \star \cdots \star a_n
    \]
    is \begin{icloze}{2}independent\end{icloze} of \begin{icloze}{1}how the expression is bracketed.\end{icloze}

    \begin{center}
        \tiny
        <<\begin{icloze}{3}The generalized associative law\end{icloze}>>
    \end{center}
\end{note}

\begin{note}{05cc8fd523084650adb46704dde222a7}
    What is the key idea in the proof of the generalized associative law for a group \({ (G, \star) }\)?

    \begin{cloze}{1}
        By induction.
    \end{cloze}
\end{note}

\begin{note}{9ca193d1531c4c49b296732d7ff12fb5}
    Henceforth our abstract groups \({ G }\), \({ H }\), \textit{etc.} will always be written with the operation as \begin{icloze}{1}\({ \cdot }\).\end{icloze}
\end{note}

\begin{note}{7d06acac21c14a628ad1ccb470fe6398}
    Henceforth for an abstract group \({ G }\) (operation \({ \cdot }\))
    an expression \begin{icloze}{2}\({ a \cdot b }\)\end{icloze} will always be written as \begin{icloze}{1}\({ ab }\).\end{icloze}
\end{note}

\begin{note}{0994e6080f3042ad81bc90d1ced0b747}
    Henceforth for an abstract group \({ G }\) (operation \({ \cdot }\))
    we denote \begin{icloze}{2}the identity of \({ G }\)\end{icloze} by \begin{icloze}{1}\({ 1 }\).\end{icloze}
\end{note}

\begin{note}{361c99f13a9b4304868fcdb350b45dbf}
    For any group \({ G }\) and \({ x \in G }\) and \begin{icloze}{3}\({ n \in \mathbb Z^{+} }\)\end{icloze}
    we shall denote by \begin{icloze}{2}\({ x^{n} }\)\end{icloze}
    \begin{icloze}{1}
        the product
        \[
            \underbrace{x x \cdots x}_{\text{\({ n }\) terms}}.
        \]
    \end{icloze}
\end{note}

\begin{note}{5b7f3c41cf0147e2bffc3929ed9ec480}
    For any group \({ G }\) and \({ x \in G }\) and \begin{icloze}{3}\({ n \in \mathbb Z^{+} }\)\end{icloze}
    we shall denote by \begin{icloze}{2}\({ x^{-n} }\)\end{icloze}
    \begin{icloze}{1}
        the product
        \[
            \underbrace{x^{-1} x^{-1} \cdots x^{-1}}_{\text{\({ n }\) terms}}.
        \]
    \end{icloze}
\end{note}

\begin{note}{a7a44229ee0f4a44b11d1410dc0fab0f}
    For any group \({ G }\) and \begin{icloze}{3}\({ x \in G }\),\end{icloze} let \({ x^{\begin{icloze}{2}0\end{icloze}} \overset{\text{def}}= \begin{icloze}{1}1, \text{the identity of \({ G }\)}\end{icloze} }\).
\end{note}

\begin{note}{b1be1b97f53c45fa9451eaa7112ca406}
    Let \({ G }\) be a group and let \({ a, u, v \in G }\).
    Then \({ au = av }\) \begin{icloze}{2}if and only if\end{icloze} \begin{icloze}{1}\({ u = v }\).\end{icloze}

    \begin{center}
        \tiny
        <<\begin{icloze}{3}Cancellation rule\end{icloze}>>
    \end{center}
\end{note}

\begin{note}{ed8673154d544c7b86ac358facc79101}
    For \({ G }\) a group and \({ x \in G }\)
    define \begin{icloze}{2}the order of \({ x }\)\end{icloze} to be
    \begin{icloze}{1}
        the smallest positive integer \({ n }\) such that
        \[
            x^{n} = 1.
        \]
    \end{icloze}
\end{note}

\begin{note}{8c334a6360be4bee8fae7f712ab2c4ee}
    For \({ G }\) a group and \({ x \in G }\), if \begin{icloze}{2}no positive power of \({ x }\) is the identity,\end{icloze} \begin{icloze}{3}the order of \({ x }\)\end{icloze} is defined to be \begin{icloze}{1}infinity.\end{icloze}
\end{note}

\begin{note}{ba4143a322564f8383f6e7d91ca32a75}
    For \({ G }\) a group and \({ x \in G }\), denote \begin{icloze}{2}the order of \({ x }\)\end{icloze} by \begin{icloze}{1}\({ \left\lvert x \right\rvert }\).\end{icloze}
\end{note}

\begin{note}{d7fee5bcbdbd47bcb6f4a2ba086fa2ed}
    For \({ G }\) a group and \({ x \in G }\), if \begin{icloze}{2}the order of \({ x }\) is an integer \({ n }\),\end{icloze} \({ x }\) is said to be \begin{icloze}{1}of order \({ n }\).\end{icloze}
\end{note}

\begin{note}{db12c606699d40e89d499d554bd52b28}
    For \({ G }\) a group and \({ x \in G }\), if \begin{icloze}{2}the order of \({ x }\) is infinite,\end{icloze} \({ x }\) is said to be \begin{icloze}{1}of infinite order.\end{icloze}
\end{note}

\begin{note}{2e514c62ce4e48eb9c6bd3b5de1d7c44}
    An element of a group has order \({ 1 }\) \begin{icloze}{2}if and only if\end{icloze} \begin{icloze}{1}it is the identity.\end{icloze}
\end{note}

\begin{note}{babeb7cf1b394be6a4f8d86e1a099cda}
    Let \({ G = \left\{ g_1, g_2, \ldots, g_n \right\} }\) be \begin{icloze}{4}a finite group\end{icloze} with \begin{icloze}{3}\({ g_1 = 1 }\).\end{icloze}
    The \begin{icloze}{2}multiplication table\end{icloze} or \begin{icloze}{2}group table\end{icloze} of \({ G }\) is
    \begin{icloze}{1}
        the matrix
        \[
            \Big[ g_i g_j \Big] \sim n \times n.
        \]
    \end{icloze}
\end{note}

\begin{note}{f245736b42b44f178f9a1d661bc4a5c7}
    Let \({ G = \begin{icloze}{3}\left\{ x \in \mathbb R \mid x \in [0, 1) \right\}\end{icloze} }\) and for \({ x, y \in G }\) let \({ x \star y }\) be \begin{icloze}{2}the fractional part of \({ x + y }\).\end{icloze}
    Then the group \begin{icloze}{4}\({ (G, \star) }\)\end{icloze} is called \begin{icloze}{1}the real numbers mod \({ 1 }\).\end{icloze}
\end{note}

\begin{note}{3664191737c844f38816547b7acd64c1}
    Let \({ G = \left\{ z \in \begin{icloze}{3}\mathbb C\end{icloze} \mid \begin{icloze}{2}z^{n} = 1\ \text{for some}\ n \in \mathbb Z^{+}\end{icloze} \right\} }\).
    Then the group \({ (G, +) }\) is called \begin{icloze}{1}the group of roots of unity in \({ \mathbb C }\).\end{icloze}
\end{note}

\begin{note}{85c981d4f1564164bb547096829d245b}
    A finite group is \begin{icloze}{3}abelian\end{icloze} \begin{icloze}{2}if and only if\end{icloze} its group table is \begin{icloze}{1}a symmetric matrix.\end{icloze}
\end{note}

\begin{note}{859ef5188ad14b35b58dc9428333e5ad}
    Let \({ G }\) a group and \({ x \in G }\) and \({ a, b \in \begin{icloze}{2}\mathbb Z\end{icloze} }\). Then \({ x^{a + b} = \begin{icloze}{1}x^{a}x^{b}\end{icloze} }\).
\end{note}

\begin{note}{0c6c419e61fc48139ff6afd4a8e28be6}
    Let \({ G }\) a group and \({ x \in G }\). Then \({ \lvert x^{-1} \rvert = \begin{icloze}{1}\lvert x \rvert\end{icloze} }\).
\end{note}

\begin{note}{f221410dc76e4c7881175d62226ecdf4}
    Let \({ G }\) a group and \({ x, g \in G }\).
    Then \({ \lvert g^{-1} x g \rvert = \begin{icloze}{1}\lvert x \rvert\end{icloze} }\).
\end{note}

\begin{note}{9951f3d62ec841df9c6f8cfc07f3c04f}
    Let \({ G }\) a group and \({ a, b \in G }\).
    Then \({ \lvert ba \rvert = \begin{icloze}{1}\lvert ab \rvert\end{icloze} }\).
\end{note}

\begin{note}{b30a94de99fe4c2f91b5417fdbb3e99d}
    Let \({ G }\) a group, \({ x \in G }\), \({ \lvert x \rvert = n < \infty }\) and \({ s \in \mathbb Z }\).
    Then \begin{icloze}{3}\({ x^{s} = 1 }\)\end{icloze} \begin{icloze}{2}if and only if\end{icloze} \begin{icloze}{1}\({ n \mid s }\).\end{icloze}
\end{note}

\begin{note}{5a0539b7021242e2a9a5769a1c156889}
    Let \({ G }\) a group, \({ x \in G }\), \({ \lvert x \rvert = \begin{icloze}{3}n < \infty\end{icloze} }\) and \({ s \in \begin{icloze}{4}\mathbb Z\end{icloze} }\).
    Then
    \[
        \begin{icloze}{2}\lvert x^{s} \rvert\end{icloze} = \begin{icloze}{1}\frac{n}{(n, s)}.\end{icloze}
    \]
\end{note}

\begin{note}{7ec585d338e941d7b465cf11d0f42550}
    Let \({ G }\) a group, \({ x \in G }\), \({ \lvert x \rvert = n < \infty }\) and \({ s \in \mathbb Z }\).
    Then \({ \lvert x^{s} \rvert = \frac{n}{(n, s)} }\).
    What is the key idea in the proof?

    \begin{cloze}{1}
        \({ (x^{s})^{k} = 1 }\) if and only if \({ n \mid sk }\).
    \end{cloze}
\end{note}

\begin{note}{00c58492691e442b9a8c0a5ba21a0c7f}
    Let \({ G }\) a group, \({ x \in G }\). If \({ x^2 = 1 }\) then
    \[
        x^{-1} = \begin{icloze}{1}x.\end{icloze}
    \]
\end{note}

\begin{note}{89199067c84244c094af347afff31c8a}
    Let \({ G }\) a group. If \begin{icloze}{3}\({ a }\) and \({ b }\) are commuting elements of \({ G }\)\end{icloze} then \({ \begin{icloze}{2}(ab)^{n}\end{icloze} = \begin{icloze}{1}a^{n}b^{n}\end{icloze} }\).
\end{note}

\begin{note}{87374145922242e3a5bc43fa952448dc}
    Let \({ G }\) a group. If \({ x^2 = 1 }\) for all \({ x \in G }\) then \({ G }\) is \begin{icloze}{1}abelian.\end{icloze}
\end{note}

\begin{note}{cdf7b8b7731c4e619920d66f7520b423}
    Let \({ G }\) a group. If \({ x^2 = 1 }\) for all \({ x \in G }\) then \({ G }\) is abelian.
    What is the key idea in the proof?

    \begin{cloze}{1}
        \({ 1 = (ab)^2 }\) and multiply by \({ a }\) on the left and by \({ b }\) on the right.
    \end{cloze}
\end{note}

\begin{note}{c48695948e6a4cf69846a629c6b45cb5}
    Let \({ (G, \star) }\) be a group and \begin{icloze}{4}\({ H \subseteq G }\).\end{icloze}
    If \begin{icloze}{2}\({ H }\) is a group under the operation \({ \star }\) restricted to \({ H }\)\end{icloze} then \begin{icloze}{3}\({ H }\)\end{icloze} is called \begin{icloze}{1}a subgroup of \({ G }\).\end{icloze}
\end{note}

\begin{note}{7d6238e012914817a45ec81f6024cf10}
    Let \({ (G, \star) }\) and \({ H \subseteq G }\).
    We shall say \begin{icloze}{2}\({ H }\) is closed under inverses\end{icloze} if \begin{icloze}{1}for all \({ h \in H }\) we have \({ h^{-1} \in H }\).\end{icloze}
\end{note}

\begin{note}{b7191557c1c0477ba2b75dfd1485197d}
    Let \({ (G, \star) }\) be a group and \begin{icloze}{4}\({ H \subseteq G }\) be nonempty.\end{icloze}
    If \({ H }\) is \begin{icloze}{3}closed\end{icloze} under \begin{icloze}{2}\({ \star }\) and inverses,\end{icloze} then \begin{icloze}{1}\({ H }\) is a subgroup of \({ G }\).\end{icloze}
\end{note}

\begin{note}{39703ef9887d48e9b763bea0c6519b19}
    Let \({ G }\) a group and \begin{icloze}{3}\({ x \in G }\).\end{icloze} Then \begin{icloze}{2}the subgroup \({ \left\{ x^{n} \mid n \in \mathbb Z \right\} }\) of \({ G }\)\end{icloze} is called \begin{icloze}{1}the cyclic subgroup of \({ G }\) generated by \({ x }\).\end{icloze}
\end{note}

\begin{note}{7d1e317bc7c64c538d09a6fd6c2e2011}
    Let \({ A }\) and \({ B }\) be groups. Then \({ A \times B }\) is \begin{icloze}{3}abelian\end{icloze} \begin{icloze}{2}if and only if\end{icloze} \begin{icloze}{1}both \({ A }\) and \({ B }\) are abelian.\end{icloze}
\end{note}

\begin{note}{c048d6c9ce83411c94e040e5991b3524}
    Let \({ A }\) and \({ B }\) be groups, \({ (a, b) \in A \times B }\).
    Then the order of \({ (a, b) }\) is \begin{icloze}{1}the least common multiple of \({ \left\lvert a \right\rvert }\) and \({ \left\lvert b \right\rvert }\).\end{icloze}
\end{note}

\begin{note}{de71dcf7adc64bcd9b53502c90a0cefa}
    Let \({ A }\) and \({ B }\) be groups, \({ (a, b) \in A \times B }\).
    Then
    \[
        (a, b)^{k} = \begin{icloze}{1}(a^{k}, b^{k})\end{icloze}
    \]
    for all \({ k \in \begin{icloze}{2}\mathbb Z\end{icloze} }\).
\end{note}

\begin{note}{e672cc6907124507a4fd998675844d02}
    Let \({ A }\) and \({ B }\) be groups, \({ (a, b) \in A \times B }\).
    Then the order of \({ (a, b) }\) is the least common multiple of \({ \left\lvert a \right\rvert }\) and \({ \left\lvert b \right\rvert }\).
    What is the key idea in the proof?

    \begin{cloze}{1}
        \({ (a, b)^{k} = (a^{k}, b^{k}) }\).
    \end{cloze}
\end{note}

\begin{note}{e6f9e981e45d4f55a3aafa3eb6d77ef1}
    Any finite group of \begin{icloze}{2}even\end{icloze} order contains an element of order \begin{icloze}{1}\({ 2 }\).\end{icloze}
\end{note}

\begin{note}{d13862b410194166829309d8ea4880a6}
    Any finite group of even order contains an element of order \({ 2 }\).
    What is the key idea in the proof?

    \begin{cloze}{1}
        Show that the set \({ \left\{ g \in G \mid g \neq g^{-1} \right\} }\) has an even number of elements.
    \end{cloze}
\end{note}

\begin{note}{86e37f9ff199460995332631a61f9a00}
    Let \({ G }\) a group, \({ x \in G }\) and \({ \lvert x \rvert = n < \infty }\).
    Then the elements
    \begin{icloze}{2}
        \[
            1, x, x^2, \ldots, x^{n-1}
        \]
    \end{icloze}
    \begin{icloze}{1}are distinct.\end{icloze}
\end{note}

\begin{note}{1bf4e9f92f854544bf96f1364e0064ed}
    Let \({ G }\) a group, \({ x \in G }\) and \({ \lvert x \rvert < \infty }\).
    Then \({ \lvert x \rvert \begin{icloze}{2}\leq\end{icloze} \begin{icloze}{1}\left\lvert G \right\rvert\end{icloze} }\).
\end{note}

\begin{note}{5f4f77e21f2b4052979906547275dfd9}
    Let \({ G }\) a group, \({ x \in G }\) and \({ \lvert x \rvert < \infty }\).
    Then \({ \lvert x \rvert \leq \left\lvert G \right\rvert }\).
    What is the key idea in the proof?

    \begin{cloze}{1}
        The elements \({ 1, x, \ldots, x^{n - 1} }\) are the only powers of \({ x }\).
    \end{cloze}
\end{note}

\begin{note}{4f07acc87f6949e092c057cb5a580c77}
    Let \({ G }\) a group, \({ x \in G }\) and \({ \lvert x \rvert = \infty }\).
    Then the elements
    \begin{icloze}{2}
        \[
            x^{n}, n \in \mathbb Z
        \]
    \end{icloze}
    \begin{icloze}{1}are distinct.\end{icloze}
\end{note}

\section{Dihedral Groups}
\begin{note}{c895ff9d20ae4a2286bb783680b3cee8}
    \begin{icloze}{1}A symmetry\end{icloze} of a regular \({ n }\)-gon is \begin{icloze}{2}any rigid motion of the \({ n }\)-gon\end{icloze} which can be effected by \begin{icloze}{3}taking a copy of the \({ n }\)-gon, moving this copy in any fashion in \({ 3 }\)-space\end{icloze} and then \begin{icloze}{4}placing the copy back on the original \({ n }\)-gon so it exactly covers it.\end{icloze}
\end{note}

\begin{note}{3a08bb223d9241bbb5cd4dae15a4a23d}
    Each symmetry of a regular \({ n }\)-gon can be described uniquely by \begin{icloze}{1}the corresponding permutation of \({ \left\{ 1, 2, \ldots, n \right\} }\),\end{icloze} representing \begin{icloze}{2}the permutation of the vertices.\end{icloze}
\end{note}

\begin{note}{c0d6e6d3d60b45058b7957002e045102}
   Given \({ n \in \mathbb Z^{+} }\) and \({ n \geq 3 }\),
   \begin{icloze}{2}the group of symmetries of a regular \({ n }\)-gon\end{icloze} is called \begin{icloze}{1}the dihedral group of order \({ 2n }\).\end{icloze}
\end{note}

\begin{note}{7a77331a22e144ceaf6ca7c1b475a99a}
    Given \begin{icloze}{3}\({ n \in \mathbb Z^{+} }\) and \({ n \geq 3 }\),\end{icloze}
    \begin{icloze}{1}the dihedral group of order \({ 2n }\)\end{icloze} is denoted \begin{icloze}{2}\({ D_{2n} }\).\end{icloze}
\end{note}

\begin{note}{a81873dbe4e6432f93bb1d8c3c5978f1}
   Given \({ n \in \mathbb Z^{+} }\), \({ n \geq 3 }\) and \({ s, t \in D_{2n} }\), \begin{icloze}{2}the product \({ st }\)\end{icloze} is defined to be \begin{icloze}{1}the symmetry obtained by first applying \({ t }\) then \({ s }\) to the \({ n }\)-gon.\end{icloze}
\end{note}

\begin{note}{1457d2279c1d432a9d371d8797d9b621}
   Given \({ n \in \mathbb Z^{+} }\) and \({ n \geq 3 }\),
   \[
       \lvert D_{2n} \rvert = \begin{icloze}{1}2n.\end{icloze}
   \]
\end{note}

\begin{note}{af77787eb9d94bae94d7df59b0415212}
   Given \({ n \in \mathbb Z^{+} }\) and \({ n \geq 3 }\), \({ \lvert D_{2n} \rvert = 2n. }\)
   What is the key idea in the proof?

   \begin{cloze}{1}
       Every symmetry is uniquely determined by how it affects some two adjacent vertices.
   \end{cloze}
\end{note}

\begin{note}{1a3443407b8641c5adc691e47eef2f1e}
    For convenience, the regular \({ n }\)-gon viewed in \({ D_{2n} }\) is fixed \begin{icloze}{1}centered at the origin.\end{icloze}
\end{note}

\begin{note}{b85d695a3fff47fbafbff07b7a341cd0}
    For convenience, the vertices of the regular \({ n }\)-gon viewed in \({ D_{2n} }\) are labeled \begin{icloze}{1}consecutively from \({ 1 }\) to \({ n }\) in a clockwise manner.\end{icloze}
\end{note}

\begin{note}{8005824717e34f4a8e154dfa84d25f17}
    In the context of the \({ D_{2n} }\) group, let \begin{icloze}{2}\({ r }\)\end{icloze} be \begin{icloze}{1}the rotation clockwise about the origin through \({ 2\pi / n }\) radian.\end{icloze}
\end{note}

\begin{note}{8439aae412044be9bf8f6c59334cd570}
    In the context of the \({ D_{2n} }\) group, let \begin{icloze}{2}\({ s }\)\end{icloze} be \begin{icloze}{1}the reflection about the line of symmetry through vertex \({ 1 }\) and the origin.\end{icloze}
\end{note}

\begin{note}{d46303ae65e74f2e8f610b873f4e559b}
    In the context of the \({ D_{2n} }\) group,
    is it possible that \({ s = r^{i} }\) for some \({ i }\)?

    \begin{cloze}{1}
        No.
    \end{cloze}
\end{note}

\begin{note}{5aaf131bef484c89b455ce9f5b4a2eae}
    In the context of the \({ D_{2n} }\) group,
    is it possible that \({ s r^{i} = s r ^{j} }\) for some \({ i \not \equiv j \pmod n }\)?

    \begin{cloze}{1}
        No.
    \end{cloze}
\end{note}

\begin{note}{79c14b3dba52416f934c9d820acb0be7}
    Each element of \({ D_{2n} }\) can be written \begin{icloze}{2}uniquely\end{icloze} in the form \begin{icloze}{1}\({ s^{k}r^{i} }\) for some \({ k = 0 }\) or \({ 1 }\) and \({ 0 \leq i \leq n - 1 }\).\end{icloze}
\end{note}

\begin{note}{f3a7147f62d84c53b1eec4f7da081eba}
    In the context of the \({ D_{2n} }\) group,
    \[
        r^{i}s = \begin{icloze}{1}s r^{-i},\end{icloze}\ \text{for \begin{icloze}{2}all \({ 0 \leq i \leq n }\)\end{icloze}}.
    \]
\end{note}

\begin{note}{2600f25fd1ec408b8e47e341dc6cdb64}
    In the context of the \({ D_{2n} }\) group,
    \[
        r^{i}s = s r^{-i},\ \text{for all \({ 0 \leq i \leq n }\)}.
    \]
    What is the key idea in the proof?

    \begin{cloze}{1}
        \({ rs = s r^{-1} }\) and by induction.
    \end{cloze}
\end{note}

\begin{note}{f56559b6eae841cea409f8438221c1b2}
    \begin{icloze}{3}A subset \({ S }\) of elements\end{icloze} of a group \({ G }\) with the property that \begin{icloze}{1}every element of \({ G }\) can be written as a (finite) product of elements of \({ S }\) and their inverses\end{icloze} is called \begin{icloze}{2}a set of generators of \({ G }\).\end{icloze}
\end{note}

\begin{note}{d72e348121f94214980378f08a5e45a3}
    If \({ S }\) is \begin{icloze}{2}a set of generators\end{icloze} of a group \({ G }\), we shall write
    \begin{icloze}{1}
        \[
            G = \langle S \rangle.
        \]
    \end{icloze}
\end{note}

\begin{note}{7bc8f288fd5d45e0a89eb59abdc95810}
    If \({ S }\) is \begin{icloze}{2}a set of generators\end{icloze} of a group \({ G }\), we shall say \({ G }\) is \begin{icloze}{1}generated by \({ S }\).\end{icloze}
\end{note}

\begin{note}{cb072c7b416e4fbf8e3cf95f496f4083}
    In terms of generators, the group \({ D_{2n} = \begin{icloze}{1}\langle r, s \rangle\end{icloze} }\).
\end{note}

\begin{note}{4fd6980a252a486980db01306accceef}
    In a \begin{icloze}{2}finite\end{icloze} group \({ G }\) a set \({ S }\) generates \({ G }\) if every element of \({ G }\) is \begin{icloze}{1}a finite product of elements of \({ S }\).\end{icloze}
\end{note}

\begin{note}{90b6154b7aaa48398ddeeb91083d71ac}
    In the \({ D_{2n}  }\) group, the relations \({ r^{n} = 1 }\), \({ s^2 = 1 }\) and \({ rs = s r^{-1} }\) have the additional property that \begin{icloze}{1}any any other relation between elements of the group may be derived from these three.\end{icloze}
\end{note}

\begin{note}{4681975ee07c4032a3ced2de0ccfa631}
    In the \({ D_{2n}  }\) group, the relations \({ r^{n} = 1 }\), \({ s^2 = 1 }\) and \({ rs = s r^{-1} }\) have the additional property that any any other relation between elements of the group may be derived from these three.
    What is the key idea in the proof?

    \begin{cloze}{1}
        We can determine exactly when two group elements are equal by using only these three relations.
    \end{cloze}
\end{note}

\begin{note}{b0bcc70704c64cccba8d832a4540749b}
    Let \({ G }\) be a group. \begin{icloze}{1}Any equations in \({ G }\) that the generators satisfy\end{icloze} are called \begin{icloze}{2}relations in \({ G }\).\end{icloze}
\end{note}

\begin{note}{b8f9a5669c634d39ac14a6115f6b142d}
    Let \({ G }\) be a group. If \begin{icloze}{4}\({ G }\) is generated by a subset \({ S }\)\end{icloze} and \begin{icloze}{3}there is some collection of relations such that any relation among the elements of \({ S }\) can be deduced from these,\end{icloze} we shall call \begin{icloze}{2}these generators and relations\end{icloze} \begin{icloze}{1}a presentation of \({ G }\).\end{icloze}
\end{note}

\begin{note}{265ab5c6f292430c8ecc6ab97f25c8a8}
    Let \({ G }\) be a group. If \begin{icloze}{4}a subset \({ S }\)\end{icloze} and \begin{icloze}{3}a collection of relations \({ R_1, \ldots, R_{m} }\)\end{icloze} form \begin{icloze}{2}a presentation of \({ G }\),\end{icloze} we shall write
    \begin{icloze}{1}
        \[
            G = \langle S \mid R_1, \ldots, R_{m} \rangle.
        \]
    \end{icloze}
\end{note}

\begin{note}{b8acfa74c7df4502a3b76c59342afbac}
    One presentation of \begin{icloze}{3}the dihedral group \({ D_{2n} }\)\end{icloze} is
    \[
        \begin{icloze}{3}D_{2n}\end{icloze} = \langle \begin{icloze}{2}r, s\end{icloze} \mid \begin{icloze}{1}r^{n} = s^2 = 1,\: rs = s r^{-1}\end{icloze} \rangle.
    \]
\end{note}

\begin{note}{06f27cb1fae140be909a72d4d52162a8}
    If \({ n = 2k }\) is even and \({ n \geq 4 }\) then \begin{icloze}{3}\({ r^{k} }\)\end{icloze} is the only \begin{icloze}{2}nonidentity\end{icloze} element of \({ D_{2n} }\) which \begin{icloze}{1}commutes with all elements of \({ D_{2n} }\).\end{icloze}
\end{note}

\section{Symmetric Groups}
\begin{note}{35103f7401374322997a41574a878c47}
    Given a set \({ \Omega }\), the set of \begin{icloze}{1}all bijections from \({ \Omega }\) to itself\end{icloze} is denoted \begin{icloze}{2}\({ S_{\Omega} }\).\end{icloze}
\end{note}

\begin{note}{e190a5ea3cc542a09f5d22434fd383e7}
    Let \({ \Omega }\) be a \begin{icloze}{4}nonempty\end{icloze} set. Then the group \({ (\begin{icloze}{2}S_{\Omega}\end{icloze}, \begin{icloze}{3}\circ\end{icloze}) }\) is called \begin{icloze}{1}the symmetric group on the set \({ \Omega }\).\end{icloze}
\end{note}

\begin{note}{41b5292492ba47c2b0c6733f4d86e86e}
    Let \({ n \in \mathbb Z^{+} }\).
    \begin{icloze}{2}The symmetric group on the set \({ \left\{ 1, 2, \ldots, n \right\} }\)\end{icloze} is called \begin{icloze}{1}the symmetric group of degree \({ n }\).\end{icloze}
\end{note}

\begin{note}{b7fae74ed3df4e71b785bd65d2e5e42b}
    Let \({ n \in \mathbb Z^{+} }\).
    \begin{icloze}{2}The symmetric group of degree \({ n }\)\end{icloze} is denoted \begin{icloze}{1}\({ S_n }\).\end{icloze}
\end{note}

\begin{note}{40a80f61353b460c9200ea835050fc6d}
    Let \({ n \in \mathbb Z^{+} }\). Then
    \[
        \left\lvert S_n \right\rvert = \begin{icloze}{1}n!.\end{icloze}
    \]
\end{note}

\begin{note}{971a7ee9395248c3ad6b53fc7e57223c}
    \begin{icloze}{3}A cycle\end{icloze} is \begin{icloze}{1}a string of integers\end{icloze} which represents the element of \({ S_n }\) which \begin{icloze}{2}cyclically permutes these integers (and fixes all other integers).\end{icloze}
\end{note}

\begin{note}{0d979e8f5d444cd2b20479e738e3b244}
    \begin{icloze}{2}The cycle \({ (a_1\: a_2\: \ldots\: a_m) }\)\end{icloze} in \({ S_n }\) is
    \begin{icloze}{1}
        the permutation
        \[
            a_i \mapsto a_{i+1} \qquad a_m \mapsto a_1.
        \]
    \end{icloze}
\end{note}

\begin{note}{1ecd3b35c9e34999a9cd7481ad891e0d}
    \begin{icloze}{2}The length\end{icloze} of a cycle in \({ S_n }\) is \begin{icloze}{1}the number of integers that appear in it.\end{icloze}
\end{note}

\begin{note}{d53a85a6dc624da7b74414131a5c9b0b}
    \begin{icloze}{2}A cycle of length \({ t }\)\end{icloze} in \({ S_n }\) is called \begin{icloze}{1}a \({ t }\)-cycle.\end{icloze}
\end{note}

\begin{note}{69e31cdb5b8644e790c3368b3b37f9fc}
    Two cycles in \({ S_n }\) are called \begin{icloze}{2}disjoint\end{icloze} if \begin{icloze}{1}they have no numbers in common.\end{icloze}
\end{note}

\begin{note}{d1990b072a9244dca0f3ae3ea60a5bf0}
    Let \({ \sigma \in S_n }\). \begin{icloze}{1}The representation of \({ \sigma }\) as the products of pairwise disjoint cycles\end{icloze} is called \begin{icloze}{2}the cycle decomposition of \({ \sigma }\).\end{icloze}
\end{note}

\begin{note}{562293a2603f483ab79fd4e9cbd6d36e}
    \begin{icloze}{2}The identity permutation\end{icloze} of \({ S_n }\) will be written as \begin{icloze}{1}\({ 1 }\).\end{icloze}
\end{note}

\begin{note}{2e6d113b0fdf478f9cacb6d733a989c3}
    \({ S_n }\) is a \begin{icloze}{2}non-abelian\end{icloze} group for \begin{icloze}{1}all \({ n \geq 3 }\).\end{icloze}
\end{note}

\begin{note}{3ff49b43f20f4b3390f3e555c41492a1}
    \begin{icloze}{2}Disjoint\end{icloze} cycles in \({ S_n }\) \begin{icloze}{1}commute.\end{icloze}
\end{note}

\begin{note}{6c2a6cc5ad27457db08cdcca242f5353}
    \begin{icloze}{3}The cycle decomposition\end{icloze} of each permutation in \({ S_n }\) is the \begin{icloze}{1}unique\end{icloze} way of expressing a permutation as \begin{icloze}{2}a product of disjoint cycles (up to rearrangement.)\end{icloze}
\end{note}

\begin{note}{f1f1d98922ba4affa25f8a1500989973}
    \begin{icloze}{3}The order\end{icloze} of a permutation in \({ S_n }\) is the \begin{icloze}{2}l.c.m.\ \end{icloze} of \begin{icloze}{1}the lengths of the cycles in its cycle decomposition.\end{icloze}
\end{note}

\begin{note}{e539f64a563146f68a90f700cee3c2a6}
    Let \({ \sigma }\) be a \({ k }\)-cycle in \({ S_n }\).
    Then
    \[
        \left\lvert \sigma \right\rvert = \begin{icloze}{1}k.\end{icloze}
    \]
\end{note}

\begin{note}{0f104518132a45bf947b3861439a4677}
    Let \({ \sigma }\) be a \({ k }\)-cycle in \({ S_n }\).
    For which positive integers \({ i }\) is \({ \sigma^{i} }\) also a \({ k }\)-cycle?

    \begin{cloze}{1}
        For \({ i }\) relatively prime to \({ k }\).
    \end{cloze}
\end{note}

\begin{note}{2c81a75df4394dd8945d760a2dd538a3}
    Let \({ \sigma }\) be a \({ k }\)-cycle in \({ S_n }\).
    What is special about the cyclic decomposition of \({ \sigma^{i} }\) for an arbitrary \({ i \in \mathbb Z^{+} }\)?

    \begin{cloze}{1}
        All of the disjoint cycles have the same length and are ``evenly spaced.''
    \end{cloze}
\end{note}

\begin{note}{a6f4e4da4d104be3b33ee481ae4a34fc}
    Let \({ p }\) be \begin{icloze}{3}a prime.\end{icloze} An element has order \begin{icloze}{2}\({ p }\)\end{icloze} in \({ S_n }\) if and only if \begin{icloze}{4}its cycle decomposition\end{icloze} is \begin{icloze}{1}a product of commuting \({ p }\)-cycles.\end{icloze}
\end{note}

\begin{note}{1cebfe78ebcb423e82d93932c114de3b}
    \[
        \begin{icloze}{4}S_3\end{icloze} = \langle a, b \mid \begin{icloze}{1}a^2 = b^2 = 1,\end{icloze}\: \begin{icloze}{2}aba = bab\end{icloze} \rangle,
    \]
    where \({ a = \begin{icloze}{3}(1\: 2)\end{icloze} }\), \({ b = \begin{icloze}{3}(2\: 3)\end{icloze} }\).
\end{note}

\section{Matrix Groups}
\begin{note}{d24745203ab949c39f465e4e32838554}
    First, a field is \begin{icloze}{2}a set \({ F }\)\end{icloze} together with \begin{icloze}{1}two binary operations \({ + }\) and \({ \cdot }\) on \({ F }\).\end{icloze}
\end{note}

\begin{note}{f3c03aa477f94f80bb772b9ea31136f9}
    How are the properties of \({ + }\) summarised in the definition of a field \({ F }\)?

    \begin{cloze}{1}
        \({ (F, +) }\) is an abelian group.
    \end{cloze}
\end{note}

\begin{note}{dace43278a624b1a8496c81bfef9be5b}
    How are the properties of \({ \cdot }\) summarised in the definition of a field \({ F }\)?

    \begin{cloze}{1}
        \({ (F - \left\{ 0 \right\}, \cdot) }\) is an abelian group.
    \end{cloze}
\end{note}

\begin{note}{bc1d11a398434735a6129ece90078d0b}
    In the definition of a field \({ F }\), what does \({ 0 }\) refer to?

    \begin{cloze}{1}
        The identity of \({ F }\) with respect to \({ + }\).
    \end{cloze}
\end{note}

\begin{note}{a12c4072b03f4f4385856eb63232dc6e}
    What is the key property that relates \({ + }\) and \({ \cdot }\) in the definition of a field \({ F }\)?

    \begin{cloze}{1}
        The distributive law.
    \end{cloze}
\end{note}

\begin{note}{e20af7d49d0f467694122bf0aad50c59}
    The distributive law from the definition of a field \({ F }\) states that
    \begin{icloze}{1}
        \[
            a \cdot (b + c) = (a \cdot b) + (a \cdot c), \quad \text{for all}\ a, b, c \in F.
        \]
    \end{icloze}
\end{note}

\begin{note}{34af1e3acbfe4b2e8485bd9ddf265c4a}
    For any field \({ F }\) let \({ \begin{icloze}{2}F^{\times}\end{icloze} = \begin{icloze}{1}F - \left\{ 0 \right\}\end{icloze} }\).
\end{note}

\begin{note}{18535f79dcff4a67b32868a9f60b8f7e}
    Given \begin{icloze}{3}a prime \({ p }\),\end{icloze} we shall denote \begin{icloze}{1}the field \({ \mathbb Z / p\mathbb Z }\)\end{icloze} as \begin{icloze}{2}\({ \mathbb F_{p} }\)\end{icloze} to emphasize that \begin{icloze}{4}it is a field.\end{icloze}
\end{note}

\begin{note}{53ab9021abdc4aab9077b752f52df492}
    Given \begin{icloze}{3}\({ n \in \mathbb Z^{+} }\)\end{icloze} and \begin{icloze}{4}a field \({ F }\),\end{icloze} \begin{icloze}{1}the general linear group of degree \({ n }\)\end{icloze} is denoted
    \begin{icloze}{2}
        \[
            GL_n(F).
        \]
    \end{icloze}
\end{note}

\begin{note}{26483d8a0f29448db95af373c3315d8f}
    What are the elements of \({ GL_n(F) }\)?

    \begin{cloze}{1}
        All \({ n \times n }\) matrices whose entries come from a field \({ F }\) and whose determinant is nonzero.
    \end{cloze}
\end{note}

\begin{note}{62ea6e7bafed4c0fa4621e7fcb7a612d}
    What is the operation of \({ GL_n(F) }\)?

    \begin{cloze}{1}
        Matrix multiplication.
    \end{cloze}
\end{note}

\begin{note}{ecd39cec434e4ca58d71efd99f59f652}
    Let \({ F }\) be a field.
    If \begin{icloze}{2}\({ \left\lvert F \right\rvert < \infty }\),\end{icloze} then
    \[
        \begin{icloze}{3}\left\lvert F \right\rvert\end{icloze} = \begin{icloze}{1}p^{m} \quad \text{for some prime}\ p\ \text{and an integer}\ m.\end{icloze}
    \]
\end{note}

\begin{note}{4262c7c840fb4996904e2a18838f8766}
    Let \({ F }\) be a field.
    If \begin{icloze}{4}\({ \left\lvert F \right\rvert = q < \infty }\),\end{icloze} then
    \[
        \begin{icloze}{3}\left\lvert GL_n(F) \right\rvert\end{icloze} = \prod_{k = \begin{icloze}{2}0\end{icloze}}^{\begin{icloze}{2}n - 1\end{icloze}} \begin{icloze}{1}(q^{n} - q^{k}).\end{icloze}
    \]
\end{note}

\begin{note}{81873bd4066c43069f33d507d19a4f29}
    Let \({ F }\) be \begin{icloze}{4}a field.\end{icloze} The subgroup of \begin{icloze}{1}all the unit upper triangular matrices\end{icloze} in \begin{icloze}{2}\({ GL_3(F) }\)\end{icloze} is called \begin{icloze}{3}the Heisenberg group over \({ F }\).\end{icloze}
\end{note}

\begin{note}{ab984848eb1941a4ba184e6bc8efece7}
    Let \({ F }\) be a field. \begin{icloze}{1}The Heisenberg group over \({ F }\)\end{icloze} is denoted \begin{icloze}{2}\({ H(F) }\).\end{icloze}
\end{note}

\begin{note}{dfc23a9c6de94344a15a98b09a255950}
    Let \({ F }\) be a field. Then
    \[
        \left\lvert H(F) \right\rvert = \begin{icloze}{1}\left\lvert F \right\rvert^3.\end{icloze}
    \]
\end{note}

\begin{note}{d23fde00126540ae999cadfb9f0101f4}
    Let \({ x \in H(\mathbb R) }\). If \begin{icloze}{2}\({ x \neq 1 }\),\end{icloze} then
    \[
        \left\lvert x \right\rvert = \begin{icloze}{1}\infty.\end{icloze}
    \]
\end{note}

\section{The Quaternion Group}

\begin{note}{239ae951128a45148a7a0b069837604e}
    \begin{icloze}{2}The quaternion group\end{icloze} is denoted \begin{icloze}{1}\({ Q_{8} }\).\end{icloze}
\end{note}

\begin{note}{555bc9c4e6e549a8954f57e369620c61}
    \[
        Q_8 = \left\{ \begin{icloze}{1}1, -1, i, -i, j, -j, k, -k\end{icloze} \right\}.
    \]
\end{note}

\begin{note}{6e7c0944844b48afbf9531e5d1fa28c7}
    \[
        \left\lvert Q_8 \right\rvert = \begin{icloze}{1}8.\end{icloze}
    \]
\end{note}

\begin{note}{4227539ce42c4875a7adfbb3ea6f82b3}
    In \({ Q_8 }\), \quad
    \({ 1 \cdot x = \begin{icloze}{1}x \cdot 1\end{icloze} = \begin{icloze}{1}x\end{icloze} }\), \quad for all \({ x }\).
\end{note}

\begin{note}{fe801d8cadf94adc8af7a8f10209ed3b}
    In \({ Q_8 }\), \quad
    \({ (-1) \cdot x = \begin{icloze}{1}x \cdot (-1)\end{icloze} = \begin{icloze}{1}-x\end{icloze} }\), \quad for all \({ x }\).
\end{note}

\begin{note}{3b0e08d205ac4eb59c5eb69b79b45201}
    In \({ Q_8 }\), \quad
    \({ (-1) \cdot (-1) = \begin{icloze}{1}1\end{icloze} }\).
\end{note}

\begin{note}{c779211682144018898526c5025c9048}
    In \({ Q_8 }\), \quad
    \({ i \cdot i = \begin{icloze}{1}-1\end{icloze} }\).
\end{note}

\begin{note}{796029c2d52b43d6944db20f10fdd8af}
    In \({ Q_8 }\), \quad
    \({ j \cdot j = \begin{icloze}{1}-1\end{icloze} }\).
\end{note}

\begin{note}{ec5bd7ed074a42d6ba94f6bd9da19392}
    In \({ Q_8 }\), \quad
    \({ k \cdot k = \begin{icloze}{1}-1\end{icloze} }\).
\end{note}

\begin{note}{e38c1bab9ae74d96b1ad4505d46d0417}
    In \({ Q_8 }\), \quad
    \({ i \cdot j = \begin{icloze}{1}k\end{icloze} }\).
\end{note}

\begin{note}{15482b24d8c34da580643a244f8080dc}
    In \({ Q_8 }\), \quad
    \({ j \cdot k = \begin{icloze}{1}i\end{icloze} }\).
\end{note}

\begin{note}{654c41a13a224d9f941e791b4ce340e1}
    In \({ Q_8 }\), \quad
    \({ k \cdot i = \begin{icloze}{1}j\end{icloze} }\).
\end{note}

\begin{note}{9be42c8a75764261a0ec49fd8d867350}
    In \({ Q_8 }\), \quad
    \({ j \cdot i = \begin{icloze}{1}-k\end{icloze} }\).
\end{note}

\begin{note}{de5cd3f588fd4c59a098434677b65581}
    In \({ Q_8 }\), \quad
    \({ k \cdot j = \begin{icloze}{1}-i\end{icloze} }\).
\end{note}

\begin{note}{b0e3385e5ab240d88ab165e2347bedc9}
    In \({ Q_8 }\), \quad
    \({ i \cdot k = \begin{icloze}{1}-j\end{icloze} }\).
\end{note}

\begin{note}{34b82771d5ed4619b3024d89b76ed248}
    \[
        Q_8 = \langle \begin{icloze}{1}i, j\end{icloze} \rangle
    \]
\end{note}

\begin{note}{e8cdf713f64b4e39b536f9b52b4ee716}
    \[
        \begin{icloze}{3}Q_8\end{icloze} = \langle i, j \mid \begin{icloze}{1}i^2 = j^2,\end{icloze}\: \begin{icloze}{2}ij = ji^{-1}\end{icloze} \rangle.
    \]
\end{note}

\section{Homomorphisms and Isomorphism}
\begin{note}{96860b79fbf04dc09da2c397394f3238}
    Let \({ (G, \star) }\) and \({ (H, \diamond) }\) be groups.
    A map \({ \varphi : \begin{icloze}{3}G \to H\end{icloze} }\) such that
    \begin{icloze}{1}
        \[
            \varphi(x \star y) = \varphi(x) \diamond \varphi(y), \qquad \text{for all}\ x, y \in G
        \]
    \end{icloze}
    is called \begin{icloze}{2}a homomorphism.\end{icloze}
\end{note}

\begin{note}{e30cd56ca334447195645d566046ce38}
    Let \({ G }\) and \({ H }\) be groups.
    A map \({ \varphi : \begin{icloze}{3}G \to H\end{icloze} }\) is called \begin{icloze}{2}an isomorphism\end{icloze} if \begin{icloze}{1}it is a bijective homomorphism.\end{icloze}
\end{note}

\begin{note}{b2c556679c0e45a6b32ec8e8790a54c0}
    Let \({ G }\) and \({ H }\) be groups.
    \({ G }\) and \({ H }\) are said to be \begin{icloze}{2}isomorphic\end{icloze} or \begin{icloze}{2}of the same isomorphism type\end{icloze} if \begin{icloze}{1}there exists an isomorphism \({ G \to H }\).\end{icloze}
\end{note}

\begin{note}{1a95e054ea864182b2ea76e195bb7481}
    Let \({ G }\) and \({ H }\) be groups.
    If \begin{icloze}{2}\({ G }\) and \({ H }\) are isomorphic\end{icloze} we shall write
    \begin{icloze}{1}
        \[
            G \cong H.
        \]
    \end{icloze}
\end{note}

\begin{note}{1cd3c49fc5504bbeb32520664e32242d}
    Let \({ \mathcal G }\) be a nonempty collection of groups.
    Then \({ \cong }\) is \begin{icloze}{1}an equivalence relation\end{icloze} on \({ \mathcal G }\).
\end{note}

\begin{note}{ee4d500cb1d74f94967ba13f8ba4504a}
    Let \({ \mathcal G }\) be a nonempty collection of groups.
    \begin{icloze}{2}The equivalence classes of \({ \cong }\)\end{icloze} are called \begin{icloze}{1}isomorphism classes.\end{icloze}
\end{note}

\begin{note}{284b06e23e8a4abba099fb983d10b5a1}
    \begin{icloze}{3}The map \({ x \mapsto e^{x} }\)\end{icloze} is \begin{icloze}{2}an isomorphism\end{icloze} from \begin{icloze}{1}\({ (\mathbb R, +) }\)\end{icloze} to \begin{icloze}{1}\({ (\mathbb R^{+}, \times) }\)\end{icloze}.
\end{note}

\begin{note}{f462694f01354b2dbe42e428795fe196}
    The isomorphism type of a symmetric group depends \begin{icloze}{2}only\end{icloze} on \begin{icloze}{1}the cardinality of the underlying set being permuted.\end{icloze}
\end{note}

\begin{note}{270ed61fd46c4d37815329b5c32c58cd}
    Let \({ \Delta }\) and \({ \Omega }\) be nonempty sets. Then \({ S_{\Delta} \begin{icloze}{3}\cong\end{icloze} S_{\Omega} }\) \begin{icloze}{2}if and only if\end{icloze}
    \begin{icloze}{1}
        \[
            \left\lvert \Delta \right\rvert = \left\lvert \Omega \right\rvert.
        \]
    \end{icloze}
\end{note}

\begin{note}{83592018138d4e7aa463202f134641aa}
    Let \({ \Delta }\) and \({ \Omega }\) be nonempty sets.
    Then \({ S_{\Delta} \cong S_{\Omega} }\) if \({ \left\lvert \Delta \right\rvert = \left\lvert \Omega \right\rvert }\).
    What is the key idea in the proof?

    \begin{cloze}{1}
        Build an isomorphism using a bijection \({ \Delta \to \Omega }\).
    \end{cloze}
\end{note}

\begin{note}{412d560d91f1457582455d66f228bfb4}
    Let \({ \Delta }\) and \({ \Omega }\) be nonempty sets.
    Then \({ S_{\Delta} \cong S_{\Omega} }\) if \({ \left\lvert \Delta \right\rvert = \left\lvert \Omega \right\rvert }\).
    In the proof, how do we define the isomorphism \({ \varphi }\)?

    \begin{cloze}{1}
        \({ \varphi(\sigma) = \theta \circ \sigma \circ \theta^{-1} }\), where \({ \theta : \Delta \to \Omega }\) is a bijection.
    \end{cloze}
\end{note}

\begin{note}{5fea8ecc491b4cdc9c34f69f48157b76}
    Theorems that \begin{icloze}{2}determine what properties of a structure specify its isomorphism type\end{icloze} are referred to as \begin{icloze}{1}classification theorems.\end{icloze}
\end{note}

\begin{note}{9b1e1641c6d64bb6af49a60adc2cc8aa}
    If \({ \varphi : G \to H }\) is a group isomorphism, then
    \[
        \left\lvert G \right\rvert = \begin{icloze}{1}\left\lvert H \right\rvert.\end{icloze}
    \]
\end{note}

\begin{note}{3c156770eb5c46be965e82efae228841}
    If \({ \varphi : G \to H }\) is a group isomorphism, then
    \begin{center}
        \({ H }\) is abelian \begin{icloze}{2}if and only if\end{icloze} \begin{icloze}{1}\({ G }\) is abelian.\end{icloze}
    \end{center}
\end{note}

\begin{note}{b2ede5339fce42f3abd2efb2ddce810f}
    If \({ \varphi : G \to H }\) is a group isomorphism and \({ x \in G }\), then
    \[
        \left\lvert \varphi(x) \right\rvert = \begin{icloze}{1}\left\lvert x \right\rvert.\end{icloze}
    \]
\end{note}

\begin{note}{419a5ad390f344458c0e7691b738bc35}
    Let \({ G }\) and \({ H }\) be groups, \begin{icloze}{6}\({ \left\lvert G \right\rvert < \infty }\),\end{icloze} and
    \[
        \begin{gathered}
            G = \begin{icloze}{3}\langle s_1, \ldots, s_m \mid R_1, \ldots, R_k \rangle,\end{icloze} \\
            \begin{icloze}{4}\left\{ r_1, \ldots, r_m \right\} \subseteq H.\end{icloze}
        \end{gathered}
    \]
    If \begin{icloze}{1}any relation \({ R_j }\) is satisfied in \({ H }\) when each \({ s_i }\) is replaced by \({ r_i }\),\end{icloze} then there is a \begin{icloze}{5}unique\end{icloze}
    \begin{icloze}{2}
        homomorphism
        \[
            \varphi : G \to H, \quad s_i \mapsto r_i.
        \]
    \end{icloze}
\end{note}

\begin{note}{01c1c1207426404aa19d1999da72740e}
    Let \({ G }\) and \({ H }\) be groups, \({ \left\lvert G \right\rvert < \infty }\), and
    \[
        \begin{gathered}
            G = \langle s_1, \ldots, s_m \mid R_1, \ldots, R_k \rangle, \\
            \left\{ r_1, \ldots, r_m \right\} \subseteq H, \\
            \varphi : s_i \mapsto r_i\ \text{be a homomorphism}.
        \end{gathered}
    \]
    If \begin{icloze}{2}\({ H }\) is generated by \({ \left\{ r_1, \ldots, r_m \right\} }\),\end{icloze} then \begin{icloze}{1}\({ \varphi }\) is surjective.\end{icloze}
\end{note}

\begin{note}{8f56d68fe62346659ff46ae2b1f31651}
    Let \({ G }\) and \({ H }\) be groups, \({ \left\lvert G \right\rvert < \infty }\), and
    \[
        \begin{gathered}
            G = \langle s_1, \ldots, s_m \mid R_1, \ldots, R_k \rangle, \\
            \left\{ r_1, \ldots, r_m \right\} \subseteq H, \\
            \varphi : s_i \mapsto r_i\ \text{be a homomorphism}.
        \end{gathered}
    \]
    If \begin{icloze}{3}\({ H }\) is generated by \({ \left\{ r_1, \ldots, r_m \right\} }\)\end{icloze} and \begin{icloze}{2}\({ \left\lvert H \right\rvert = \left\lvert G \right\rvert }\),\end{icloze} then \begin{icloze}{1}\({ \varphi }\) is an isomorphism.\end{icloze}
\end{note}

\begin{note}{67b60cd5209c46fbbb3f98e3f10b05e1}
    \[
        D_{\begin{icloze}{2}6\end{icloze}} \cong S_{\begin{icloze}{1}3\end{icloze}}.
    \]
\end{note}

\begin{note}{1f9aa544bf9a4eb4a414531f8156f4cd}
    \({ D_{6} \cong S_{3} }\). What is the key idea in the proof?

    \begin{cloze}{1}
        Build a homomorphism using the presentation of \({ D_6 }\).
    \end{cloze}
\end{note}

\begin{note}{a042a5ebe7b449ff86ce5a24db9bb191}
    \({ D_{6} \cong S_{3} }\). Which generator of \({ D_6 }\) are used in the proof?

    \begin{cloze}{1}
        \({ r }\) and \({ s }\).
    \end{cloze}
\end{note}

\begin{note}{1daa5ca4598f44fcbd36ee24a840501e}
    \({ D_{6} \cong S_{3} }\). Which generator of \({ S_3 }\) are used in the proof?

    \begin{cloze}{1}
        \({ a = (1\: 2) }\) and \({ b = (1\: 2\: 3) }\).
    \end{cloze}
\end{note}

\begin{note}{d9a1d23815594891aa15dd743f29acfc}
    Let \({ \varphi : G \to H }\) be a group homomorphism.
    Then
    \[
        \varphi(x^{n}) = \begin{icloze}{1}\varphi(x)^{n}\end{icloze} \quad \text{for all } n \in \begin{icloze}{2}\mathbb Z.\end{icloze}
    \]
\end{note}

\begin{note}{2ac7af235abd475ea3f54b2b250b4400}
    Let \({ \varphi : G \to H }\) be a group homomorphism.
    Then
    \[
        \varphi(x^{-1}) = \begin{icloze}{1}\varphi(x)^{-1}.\end{icloze}
    \]
\end{note}

\begin{note}{2dc3b42cb1a242a49fce2ea25c98de72}
    Let \({ \varphi : G \to H }\) be a group homomorphism.
    Then
    \[
        \varphi(1) = \begin{icloze}{1}1.\end{icloze}
    \]
\end{note}

\begin{note}{71cc11a4e7454bd4ab3e4eebc22c9eb1}
    Let \({ \varphi : G \to H }\) be a group \begin{icloze}{3}isomorphism.\end{icloze}
    Then
    \[
        \varphi(x) = 1 \begin{icloze}{2}\iff\end{icloze} \begin{icloze}{1}x = 1.\end{icloze}
    \]
\end{note}

\begin{note}{f48d2a825d984e4b823201e3b6b9d3de}
    Let \({ \varphi : G \to H }\) be a group homomorphism.
    Is it unconditionally true that \({ \left\lvert \varphi(x) \right\rvert = \left\lvert x \right\rvert }\) for all \({ x \in G }\)?

    \begin{cloze}{1}
        No.
    \end{cloze}
\end{note}

\begin{note}{844e79dfacbc4157acfd4aebdbaa4947}
    Let \({ \varphi : G \to H }\) be a group homomorphism and \({ G }\) be abelian.
    Then
    \begin{center}
        \begin{icloze}{1}\({ \varphi }\) is surjective\end{icloze} \begin{icloze}{3}\({ \implies }\)\end{icloze} \begin{icloze}{2}\({ H }\) is abelian.\end{icloze}
    \end{center}
\end{note}

\begin{note}{aec0f7a10a044411b844a5a08d3303d9}
    Are the groups \({ (\mathbb R, +) }\) and \({ (\mathbb R^{+}, \times) }\) isomorphic?

    \begin{cloze}{1}
        Yes.
    \end{cloze}
\end{note}

\begin{note}{57b13746efbe42fda870594eda8f6965}
    Are \({ D_6 }\) and \({ S_3 }\) isomorphic?

    \begin{cloze}{1}
        Yes.
    \end{cloze}
\end{note}

\begin{note}{4a3c6aee4f084dfea611c081da1c5925}
    Are the multiplicative groups \({ \mathbb R^{\times} }\) and \({ \mathbb C^{\times} }\) isomorphic?

    \begin{cloze}{1}
        No.
    \end{cloze}
\end{note}

\begin{note}{b33c67212d1444d3a8679e7c31a39c66}
    The multiplicative groups \({ \mathbb R^{\times} }\) and \({ \mathbb C^{\times} }\) are not isomorphic.
    What is the key idea in the proof?

    \begin{cloze}{1}
        \({ \mathbb C^{\times} }\) has elements of order \({ n > 2 }\).
    \end{cloze}
\end{note}

\begin{note}{ded0e803e7e34c5f913f8a715f6186b0}
    Are the additive groups \({ \mathbb R }\) and \({ \mathbb Q }\) isomorphic?

    \begin{cloze}{1}
        No.
    \end{cloze}
\end{note}

\begin{note}{e7bf7586388f460993d9976f09ab36d5}
    The additive groups \({ \mathbb R }\) and \({ \mathbb Q }\) are not isomorphic.
    What is the key idea in the proof?

    \begin{cloze}{1}
        \({ \left\lvert \mathbb R \right\rvert \neq \left\lvert \mathbb Q \right\rvert }\).
    \end{cloze}
\end{note}

\begin{note}{5de7b924c5d74c24bebd18083881ebd6}
    Are the additive groups \({ \mathbb Z }\) and \({ \mathbb Q }\) isomorphic?

    \begin{cloze}{1}
        No.
    \end{cloze}
\end{note}

\begin{note}{033cd7915d6349bead591259dd238b2d}
    The additive groups \({ \mathbb Z }\) and \({ \mathbb Q }\) are not isomorphic.
    What is the key idea in the proof?

    \begin{cloze}{1}
        \({ \mathbb Q }\) has elements of order \({ n > 1 }\).
    \end{cloze}
\end{note}

\begin{note}{42380e58738b40099ecc62e995bc6049}
    Are \({ D_8 }\) and \({ Q_8 }\) isomorphic?

    \begin{cloze}{1}
        No.
    \end{cloze}
\end{note}

\begin{note}{c920ffc0d7ba4faeb1cfc6748bbb0df5}
    \({ D_8 }\) and \({ Q_8 }\) are not isomorphic.
    What is the key idea in the proof?

    \begin{cloze}{1}
        \({ D_8 }\) has for elements of order \({ 2 }\), namely \({ s, sr, sr^2, sr^3 }\).
    \end{cloze}
\end{note}

\begin{note}{7f2f3fa158c7439f8857975eadb00fc5}
    Are \({ D_{24} }\) and \({ S_4 }\) isomorphic?

    \begin{cloze}{1}
        No.
    \end{cloze}
\end{note}

\begin{note}{e883c893667e4f09b6f83f082c41281a}
    \({ D_{24} }\) and \({ S_4 }\) are not isomorphic.
    What is the key idea in the proof?

    \begin{cloze}{1}
        \({ D_{24} }\) has an element of order \({ 12 }\).
    \end{cloze}
\end{note}

\begin{note}{d5bb3b5b1d3c440caaeab945defc239d}
    Let \({ A }\) and \({ B }\) be groups.
    Are \({ A \times B }\) and \({ B \times A }\) isomorphic?

    \begin{cloze}{1}
        Yes.
    \end{cloze}
\end{note}

\begin{note}{c47557cf76554d059f5755de18a4df00}
    Let \({ A }\), \({ B }\) and \({ C }\) be groups.
    Are \({ (A \times B) \times C }\) and \({ A \times (B \times C) }\) isomorphic?

    \begin{cloze}{1}
        Yes.
    \end{cloze}
\end{note}

\begin{note}{0e9033cc45ee4e97917feb313e9598ac}
    Let \({ \varphi : G \to H }\) be a group homomorphism.
    Then \begin{icloze}{3}the image\end{icloze} of \({ \varphi }\) is \begin{icloze}{1}a subgroup\end{icloze} of \begin{icloze}{2}\({ H }\).\end{icloze}
\end{note}

\begin{note}{2771feb8afd448049fd587b560e40ec9}
    Let \({ \varphi : G \to H }\) be a group homomorphism.
    Then
    \[
        \begin{icloze}{2}\varphi\ \text{is injective}\end{icloze}
        \begin{icloze}{5}\iff\end{icloze}
        \begin{icloze}{3}G\end{icloze} \begin{icloze}{4}\cong\end{icloze} \begin{icloze}{1}\varphi(G).\end{icloze}
    \]
\end{note}

\begin{note}{12077d31c47349a5bceaecda0fde9ea1}
    Let \({ \varphi : G \to H }\) be a group homomorphism.
    If \({ \varphi }\) is injective then \({ G \cong \varphi(G) }\).
    What is the key idea in the proof?

    \begin{cloze}{1}
        \({ \varphi }\) is surjective when its codomain is restricted to \({ \varphi(G) }\).
    \end{cloze}
\end{note}

\begin{note}{2a8327bf4f684de0b39f3817aad067b9}
    Let \({ \varphi : G \to H }\) be a group homomorphism.
    \begin{icloze}{2}The kernel of \({ \varphi }\)\end{icloze} is \begin{icloze}{1}the fiber of \({ \varphi }\) over \({ 1 }\).\end{icloze}
\end{note}

\begin{note}{b3a98b9d64f74a20acda1441ecdc9369}
    Let \({ \varphi : G \to H }\) be a group homomorphism.
    Then \begin{icloze}{3}the kernel\end{icloze} of \({ \varphi }\) is \begin{icloze}{1}a subgroup\end{icloze} of \begin{icloze}{2}\({ G }\).\end{icloze}
\end{note}

\begin{note}{f41a2cf31300472c964735adcca416bd}
    Let \({ \varphi : G \to H }\) be a group homomorphism.
    Then \begin{icloze}{3}\({ \varphi }\) is injective\end{icloze} \begin{icloze}{4}if and only if\end{icloze} \begin{icloze}{2}the kernel of \({ \varphi }\)\end{icloze} \begin{icloze}{1}is the identity subgroup of \({ G }\).\end{icloze}
\end{note}

\begin{note}{086cd029adec4c3185796e6c095f117c}
    Let \({ G }\) be a group. Then the map \({ g \mapsto g^{-1} }\) is a homomorphism \begin{icloze}{2}if and only if\end{icloze} \begin{icloze}{1}\({ G }\) is abelian.\end{icloze}
\end{note}

\begin{note}{f748fe4d306c4cfe8ab2701c9737cac4}
    Let \({ G }\) be a group. Then the map \({ g \mapsto g^2 }\) is a homomorphism \begin{icloze}{2}if and only if\end{icloze} \begin{icloze}{1}\({ G }\) is abelian.\end{icloze}
\end{note}

\begin{note}{a34cd518f90a4b80aa8fa09ec58f79b5}
    Is any surjective homomorphism necessarily an isomorphism?

    \begin{cloze}{1}
        No.
    \end{cloze}
\end{note}

\begin{note}{be5f48d2e20e41dbbab81bc5b7d6334c}
    Let \({ G }\) be a group.
    \begin{icloze}{2}An isomorphism from \({ G }\) onto \({ G }\)\end{icloze} is called \begin{icloze}{1}an automorphism of \({ G }\).\end{icloze}
\end{note}

\begin{note}{175a31f669ee492bb4698ccb5a311115}
    Let \({ G }\) be a group.
    \begin{icloze}{2}The set of all automorphisms of \({ G }\)\end{icloze} is denoted
    \begin{icloze}{1}
        \[
            \operatorname{Aut}(G).
        \]
    \end{icloze}
\end{note}

\begin{note}{bf4133e84db44fd4a19013814259b4f1}
    Let \({ G }\) be a group.
    Then \({ \operatorname{Aut}(G) }\) is \begin{icloze}{2}a group\end{icloze} under \begin{icloze}{1}function composition.\end{icloze}
\end{note}

\begin{note}{3d4887e6cf02407b97125275d1ae4b0b}
    Let \({ G }\) be a group.
    \begin{icloze}{2}The group \({ (\operatorname{Aut}(G), \circ) }\)\end{icloze} is called \begin{icloze}{1}the automorphism group of \({ G }\).\end{icloze}
\end{note}

\begin{note}{29c4a6993b1c42d9b60563794a41dd58}
    Let \({ G }\) be a group.
    A homomorphism \({ \sigma }\) such that
    \begin{icloze}{1}
        \[
            \sigma(g) = g \text{ implies } g = 1
        \]
    \end{icloze}
    is called \begin{icloze}{2}fixed point free.\end{icloze}
\end{note}

\begin{note}{10d4efc454cd45189f531b1b8e225ebf}
    Let \({ G }\) be a \begin{icloze}{4}finite\end{icloze} group.
    If \({ \sigma }\) is a \begin{icloze}{3}fixed point free\end{icloze} \begin{icloze}{5}automorphism\end{icloze} of order \begin{icloze}{2}\({ 2 }\)\end{icloze} in \({ G }\), then
    \[
        \sigma : x \mapsto \begin{icloze}{1}x^{-1}.\end{icloze}
    \]
\end{note}

\begin{note}{48b10b987faa4173afd091d11bce69f8}
    Let \({ G }\) be a finite group.
    If \({ \sigma }\) is a fixed point free automorphism of order \({ 2 }\) in \({ G }\), then \({ \sigma : x \mapsto x^{-1} }\).
    What is the key idea in the proof?

    \begin{cloze}{1}
        Any \({ x \in G }\) may be expressed as \({ g^{-1} \sigma(g) }\) for some \({ g \in G }\).
    \end{cloze}
\end{note}

\begin{note}{559ae14dc2c142eba35c1b72238ad25e}
    Let \({ G }\) be a finite group.
    If \({ G }\) possesses a fixed point free automorphism of order \({ 2 }\), then \begin{icloze}{1}\({ G }\) is abelian.\end{icloze}
\end{note}

\begin{note}{47792a9d48b64045b05ea3dd789e8d12}
    Let \({ G }\) be a finite group.
    If \({ G }\) possesses a fixed point free automorphism of order \({ 2 }\), then \({ G }\) is abelian.
    What is the key idea in the proof?

    \begin{cloze}{1}
        The automorphism must be \({ x \mapsto x^{-1} }\).
    \end{cloze}
\end{note}

\begin{note}{ee7230fadcc9450e89d741cb94e940d7}
    Let \({ G = \langle x, y \rangle }\) be a finite group, \({ x \neq y }\) and \({ \left\lvert x \right\rvert = \left\lvert y \right\rvert = 2 }\).
    Then
    \[
        G \cong \begin{icloze}{1}D_{2n},\end{icloze}
    \]
    where \({ n = \begin{icloze}{2}\left\lvert xy \right\rvert\end{icloze} }\).
\end{note}

\section{Group Actions}
\begin{note}{47fb1f971a0e47ca900bca9d515b276f}
    Let \({ G }\) be \begin{icloze}{4}a group\end{icloze} and \({ A }\) be \begin{icloze}{3}a set.\end{icloze}
    First, \begin{icloze}{2}a group action of \({ G }\) on \({ A }\)\end{icloze} is \begin{icloze}{1}a map from \({ G \times A }\) to \({ A }\).\end{icloze}
\end{note}

\begin{note}{0949588890c9477795b23452373695d9}
    Let \({ G }\) be a group and \({ A }\) be a set.
    \begin{icloze}{2}The group action of \({ G }\) on \({ A }\)\end{icloze} is written
    \[
        (g, a) \mapsto \begin{icloze}{1}g \cdot a.\end{icloze}
    \]
\end{note}

\begin{note}{9f314e8d6fb04ba78bda7bde0c259aa5}
    Let \({ G }\) be a group and \({ A }\) be a set.
    How many properties are there in the definition of a group action of \({ G }\) on \({ A }\)?

    \begin{cloze}{1}
        Two.
    \end{cloze}
\end{note}

\begin{note}{2150fd542e2b4a4483ef8f54aafebf19}
    Let \({ G }\) be a group and \({ A }\) be a set.
    What is the first property from the definition of a group action of \({ G }\) on \({ A }\)?

    \begin{cloze}{1}
        \({ g \cdot (h \cdot a) = (gh) \cdot a }\) for all \({ g, h \in G }\), \({ a \in A }\).
    \end{cloze}
\end{note}

\begin{note}{c886b7ab9579413389c0a1a864a71f9b}
    Let \({ G }\) be a group and \({ A }\) be a set.
    What is the second property from the definition of a group action of \({ G }\) on \({ A }\)?

    \begin{cloze}{1}
        \({ 1 \cdot a = a }\) for all \({ a \in A }\).
    \end{cloze}
\end{note}

\begin{note}{923e161350c04b37925f4d8a5ec7a48f}
    Let \({ G }\) be a group. We shall say \({ G }\) \begin{icloze}{2}acts on a set \({ A }\)\end{icloze} if \begin{icloze}{1}a group action of \({ G }\) on \({ A }\) is given.\end{icloze}
\end{note}

\begin{note}{4a108c989e364bf1a4f08acf4b13afed}
    Let a group \({ G }\) act on a set \({ A }\).
    Given \({ g \in G }\) and \({ a \in A }\), the expression \begin{icloze}{2}\({ g \cdot a }\)\end{icloze} will usually be written \begin{icloze}{1}simply as \({ ga }\).\end{icloze}
\end{note}

\begin{note}{1e4797055dac4aaebaff4f3e7ca94dc9}
    Let a group \({ G }\) act on a set \({ A }\).
    Given \begin{icloze}{4}\({ g \in G }\),\end{icloze} the map \begin{icloze}{3}\({ \sigma_{g} }\)\end{icloze} is defined by
    \[
        \begin{icloze}{3}\sigma_{g}\end{icloze} : \begin{icloze}{2}A\end{icloze} \to \begin{icloze}{2}A\end{icloze}  \qquad a \mapsto \begin{icloze}{1}g \cdot a.\end{icloze}
    \]
\end{note}

\begin{note}{a7a8a2e8422c4eb1a55326e03fdfae3e}
    Let a group \({ G }\) act on a set \({ A }\).
    For each fixed \({ g \in G }\), \({ \sigma_{g} }\) is \begin{icloze}{1}a permutation\end{icloze} of \({ A }\).
\end{note}

\begin{note}{069a58b690ce40bcb6ac7e8b2bd10732}
    Let a group \({ G }\) act on a set \({ A }\).
    For each fixed \({ g \in G }\), \({ \sigma_{g} }\) is a permutation of \({ A }\).
    What is the key idea in the proof?

    \begin{cloze}{1}
        Find the \({ 2 }\)-sided inverse of \({ \sigma }\).
    \end{cloze}
\end{note}

\begin{note}{3d3a86bc4244429da7757d00efca180a}
    Let a group \({ G }\) act on a set \({ A }\).
    Given \({ g \in G }\),
    \[
        \begin{icloze}{2}\sigma_{g}^{-1}\end{icloze} = \begin{icloze}{1}\sigma_{g^{-1}}.\end{icloze}
    \]
\end{note}

\begin{note}{3c4e03fdddab413f84ca6812e5ffed8e}
    Let a group \({ G }\) act on a set \({ A }\).
    The map from \({ G }\) to \begin{icloze}{3}\({ S_{A} }\)\end{icloze} defined by
    \[
        g \mapsto \begin{icloze}{2}\sigma_{g}\end{icloze}
    \]
    is \begin{icloze}{1}a homomorphism.\end{icloze}
\end{note}

\begin{note}{a3383f0de3ab4bb48b322d1900ca9519}
    Intuitively, \begin{icloze}{3}a group action of \({ G }\) on a set \({ A }\)\end{icloze} just means that every element \({ g }\) in \({ G }\) acts as \begin{icloze}{1}a permutation\end{icloze} on \({ A }\) in a manner \begin{icloze}{2}consistent with the group operations in \({ G }\).\end{icloze}
\end{note}

\begin{note}{2bd735bbaeaf44a68e6820678cdfd0ef}
    Let a group \({ G }\) act on a set \({ A }\).
    \begin{icloze}{3}The homomorphism \({ g \mapsto \sigma_{g} }\)\end{icloze} from \begin{icloze}{4}\({ G }\)\end{icloze} to \begin{icloze}{4}\({ S_{A} }\)\end{icloze} is called \begin{icloze}{1}the permutation representation\end{icloze} \begin{icloze}{2}associated to the given action.\end{icloze}
\end{note}

\begin{note}{c1df0995320a40d2a6d9ee88309ffada}
    Let \({ G }\) be a group and \begin{icloze}{5}\({ A }\) a set.\end{icloze}
    Any \begin{icloze}{3}homomorphism\end{icloze}
    \begin{icloze}{1}
        \[
            \varphi : G \to S_{A}
        \]
    \end{icloze}
    corresponds to \begin{icloze}{2}a group action of \({ G }\) on \({ A }\).\end{icloze}
\end{note}

\begin{note}{b74b0d2663b04de589ca4386ed7cfef4}
    Let \({ G }\) be a group and \({ A }\) a set.
    To which group action does a homomorphism \({ \varphi : G \to S_{A}  }\) correspond?

    \begin{cloze}{1}
        \[
            g \cdot a = \varphi(g)(a).
        \]
    \end{cloze}
\end{note}

\begin{note}{5974a55856e34f02b8ea7ea3aa308ede}
    Let \({ G }\) be a group.
    More precisely, a group action of the form
    \begin{icloze}{2}
        \[
            G \times A \to A
        \]
    \end{icloze}
    should be named \begin{icloze}{1}a left action.\end{icloze}
\end{note}

\begin{note}{98196731647644bbb8b84f0dd0e9ba38}
    Let \({ G }\) be a group and \({ A }\) a nonempty set.
    The action of \({ G }\) on \({ A }\) defined by
    \begin{icloze}{1}
        \[
            ga = a, \quad \text{for all}\ g \in G, a \in A
        \]
    \end{icloze}
    is called \begin{icloze}{2}the trivial action.\end{icloze}
\end{note}

\begin{note}{e3f75516876c4c8396233ef839dc2834}
    Let \({ G }\) be a group acting on a set \({ A }\).
    We shall say \begin{icloze}{2}\({ G }\) acts trivially on \({ A }\)\end{icloze} if \begin{icloze}{1}the action is trivial.\end{icloze}
\end{note}

\begin{note}{e845e29f5f0145adb058744cfa19d71a}
    Let \({ G }\) be a group acting on a set \({ A }\).
    If \begin{icloze}{2}distinct elements of \({ G }\) induce distinct permutations of \({ A }\),\end{icloze} the action is said to be \begin{icloze}{1}faithful.\end{icloze}
\end{note}

\begin{note}{1d6ec9bd320440c7948bea6cc552bf8f}
    A group action \begin{icloze}{3}is faithful\end{icloze} \begin{icloze}{4}if and only if\end{icloze} \begin{icloze}{2}the associated permutation representation\end{icloze} \begin{icloze}{1}is injective.\end{icloze}
\end{note}

\begin{note}{3b17f64aea804c2ca27b2bc733a2a014}
    Let \({ G }\) be a group acting on a set \({ A }\).
    \begin{icloze}{2}The kernel of the action\end{icloze} is defined to be
    \begin{icloze}{1}
        \[
            \left\{ g \in G \mid \sigma_{g} = 1 \right\}.
        \]
    \end{icloze}
\end{note}

\begin{note}{d3de7470fa0e4b758296e348c93d3fdf}
    Let \({ V }\) be a vector space over a field \({ F }\).
    Then \begin{icloze}{2}the multiplicative group \({ F^{\times} }\)\end{icloze} \begin{icloze}{3}acts\end{icloze} on \begin{icloze}{1}the set \({ V }\).\end{icloze}
\end{note}

\begin{note}{6e14b840046548aa83a89f744766a642}
    For any nonempty set \({ A }\) the group \({ S_{A} }\) acts on \({ A }\) by \({ \sigma \cdot a = \begin{icloze}{1}\sigma(a)\end{icloze} }\), for all \({ \sigma \in S_{A}, a \in A }\).
\end{note}

\begin{note}{2b11c4a7f35b44008803b61ae22c4ec1}
    Let \({ A }\) be a nonempty set.
    The permutation representation associated with the action of \({ S_{A} }\) on \({ A }\) is \begin{icloze}{1}the identity map in \({ S_{A} }\).\end{icloze}
\end{note}

\begin{note}{ba1f69df223242b5af6517275f8203bc}
    Given \({ \alpha \in D_{2n} }\), let \begin{icloze}{2}\({ \sigma_{\alpha} }\)\end{icloze} denote \begin{icloze}{1}the permutation of \({ \left\{ 1, 2, \ldots, n \right\} }\), representing the permutation of the vertices.\end{icloze}
\end{note}

\begin{note}{85081cd4076348dfba43dc9e737ff07d}
    \({ D_{2n} }\) acts on \begin{icloze}{2}\({ \left\{ 1, 2, \ldots, n \right\} }\)\end{icloze} by \({ \alpha \cdot i = \begin{icloze}{1}\sigma_{\alpha}(i)\end{icloze} }\) for all \({ \alpha \in D_{2n} }\).
\end{note}

\begin{note}{4a12094119574b18b849c194030d0a13}
    Let \({ G }\) be a group.
    Given \({ g \in G }\), the map
    \begin{icloze}{2}
        \[
            a \mapsto ga \qquad G \to G
        \]
    \end{icloze}
    is called \begin{icloze}{1}left multiplication.\end{icloze}
\end{note}

\begin{note}{99ec6293e8ff4118b8f67a8ff7d4aebf}
    Let \({ G }\) be a group written additively.
    Given \({ g \in G }\), the map
    \begin{icloze}{2}
        \[
            a \mapsto g + a \qquad G \to G
        \]
    \end{icloze}
    is called \begin{icloze}{1}a left translation.\end{icloze}
\end{note}

\begin{note}{0e7c694af67e4cdb98b480b89d61373f}
    Let \({ G }\) be a group.
    The group action of \({ G }\) on \begin{icloze}{3}itself\end{icloze} defined by
    \begin{icloze}{1}
        \[
            g \cdot a = ga, \quad \text{for all}\ g, a \in G
        \]
    \end{icloze}
    is called \begin{icloze}{2}the left regular action of \({ G }\) on itself.\end{icloze}
\end{note}

\begin{note}{a97e2c6daab04e188a70813e7ec93207}
    Let \({ G }\) be a group acting on a set \({ A }\).
    The kernel of the action is \begin{icloze}{2}a subgroup\end{icloze} of \begin{icloze}{1}\({ G }\).\end{icloze}
\end{note}

\begin{note}{5a1dfb95d2d14756960572225eade8e2}
    Let \({ G }\) be a group acting on a set \({ A }\).
    Given \begin{icloze}{3}\({ a \in A }\),\end{icloze}
    \begin{icloze}{1}
        the set
        \[
            \left\{ g \in G \mid ga = a \right\}
        \]
    \end{icloze}
    is called \begin{icloze}{2}the stabilizer of \({ a }\) in \({ G }\).\end{icloze}
\end{note}

\begin{note}{40824336a8b0481394e4c7dbd2a5e12e}
    Let \({ G }\) be a group acting on a set \({ A }\) and \({ a \in A }\).
    The stabilizer of \({ a }\) in \({ G }\) is \begin{icloze}{2}a subgroup\end{icloze} of \begin{icloze}{1}\({ G }\).\end{icloze}
\end{note}

\begin{note}{de72c6c8aeb345c6b825e47ed14d68d3}
    \begin{icloze}{2}The kernel\end{icloze} of an action of the group \({ G }\) on a set \({ A }\) is the same as \begin{icloze}{3}the kernel\end{icloze} of \begin{icloze}{1}the associated permutation representation.\end{icloze}
\end{note}

\begin{note}{63892c14f75b4100b0a22a46fc813060}
    A group \({ G }\) acts \begin{icloze}{3}faithfully\end{icloze} on a set \({ A }\) \begin{icloze}{4}if and only if\end{icloze} \begin{icloze}{2}the kernel of the action\end{icloze} is \begin{icloze}{1}the identity subgroup of \({ G }\).\end{icloze}
\end{note}

\begin{note}{4460f313478a45de80ee8b6fea906453}
    Let \({ V }\) be a vector space over a field \({ F }\).
    Is the action of the multiplicative group \({ F^{\times} }\) on \({ V }\) faithful?

    \begin{cloze}{1}
        Yes.
    \end{cloze}
\end{note}

\begin{note}{4f4a58eb88264a478cb3baf175b6eb3f}
    Let \({ A }\) be a nonempty set and let \begin{icloze}{4}\({ k \in \mathbb Z^{+} }\) with \({ k \leq \left\lvert A \right\rvert }\).\end{icloze}
    \begin{icloze}{3}The symmetric group \({ S_{A} }\)\end{icloze} acts on \begin{icloze}{2}the set of all subsets of \({ A }\) of cardinality \({ k }\)\end{icloze} by
    \[
        \sigma \cdot \left\{ a_1, \ldots, a_k \right\} = \begin{icloze}{1}\left\{ \sigma a_1, \ldots, \sigma a_k \right\}.\end{icloze}
    \]
\end{note}

\begin{note}{8c10a40354e149a09e1606629ecca53d}
    Let \({ A }\) be a nonempty set and let \begin{icloze}{4}\({ k \in \mathbb Z^{+} }\) with \({ k \leq \left\lvert A \right\rvert }\).\end{icloze}
    \begin{icloze}{3}The symmetric group \({ S_{A} }\)\end{icloze} acts on \begin{icloze}{2}the set of all ordered \({ k }\)-tuples of elements of \({ A }\)\end{icloze} by
    \[
        \sigma \cdot (a_1, \ldots, a_k) = \begin{icloze}{1}(\sigma a_1, \ldots, \sigma a_k).\end{icloze}
    \]
\end{note}

\begin{note}{b4791a825b5d456bbbcec43de8d31d2d}
    Let \({ A }\) be a nonempty set.
    For which values of \({ k }\) the action of \({ S_A }\) on \({ k }\)-element subsets of \({ A }\) is faithful?

    \begin{cloze}{1}
        For \({ 1 \leq k \leq n - 1 }\).
    \end{cloze}
\end{note}

\begin{note}{431f8a3fcec64f729d93244133a3f071}
    Let \({ A }\) be a nonempty set.
    The action of \({ S_A }\) on \({ k }\)-element subsets of \({ A }\) is faithful if and only if \({ 1 \leq k \leq n - 1 }\).
    What is the key idea in the proof?

    \begin{cloze}{1}
        The kernel of the associated permutation representation is the identity group.
    \end{cloze}
\end{note}

\begin{note}{8b5ed16ce5324edd8c7e9442e9fbd350}
    Let \({ A }\) be a nonempty set.
    For which values of \({ k }\) the action of \({ S_A }\) on \({ k }\)-tuples of elements of \({ A }\) is faithful?

    \begin{cloze}{1}
        For any \({ k \in \mathbb Z^{+} }\).
    \end{cloze}
\end{note}

\begin{note}{40928e143c7a46fdae82b9d43c4c39c6}
    What is the kernel of the left regular action of a group on itself?

    \begin{cloze}{1}
        The identity subgroup.
    \end{cloze}
\end{note}

\begin{note}{8dfcac2e546647a09d97caa49e817fb9}
    Let \({ G }\) be a group. When does the map defined by
    \[
        g \cdot a = ag
    \]
    satisfy the axiom of a left group action of \({ G }\) on itself?

    \begin{cloze}{1}
        If and only if \({ G }\) is abelian.
    \end{cloze}
\end{note}

\begin{note}{c3b0b98b6bda48839417574435d9b837}
    Let \({ G }\) be a group. When does the map defined by
    \[
        g \cdot a = ag^{-1}
    \]
    satisfy the axiom of a left group action of \({ G }\) on itself?

    \begin{cloze}{1}
        Always.
    \end{cloze}
\end{note}

\begin{note}{06d71056aa1c4169854a102610a88aa0}
    Let \({ G }\) be a group. When does the map defined by
    \[
        g \cdot a = gag^{-1}
    \]
    satisfy the axiom of a left group action of \({ G }\) on itself?

    \begin{cloze}{1}
        Always
    \end{cloze}
\end{note}

\begin{note}{2581320a80be4725a9b0d50b9c490374}
    Let \({ G }\) be a group.
    The group action of \({ G }\) on \begin{icloze}{3}itself\end{icloze} defined by
    \begin{icloze}{1}
        \[
            g \cdot a = gag^{-1}, \quad \text{for all}\ g, a \in G
        \]
    \end{icloze}
    is called \begin{icloze}{2}left conjugation by \({ g }\).\end{icloze}
\end{note}

\begin{note}{2a661ecbd43640739905e3cdd87b1d0c}
    Let \({ G }\) be a group and \({ g \in G }\).
    Conjugation by \({ g }\) is \begin{icloze}{1}an automorphism\end{icloze} of \begin{icloze}{2}\({ G }\).\end{icloze}
\end{note}

\begin{note}{e7c43fbf0fce41629c685b5037106184}
    Let \({ G }\) be a group and \({ g \in G }\).
    How do you show that conjugation by \({ g }\) is a bijection?

    \begin{cloze}{1}
        Find the \({ 2 }\)-sided inverse.
    \end{cloze}
\end{note}

\begin{note}{deb3c83d964e41ceb7cf97aad04ecc30}
    Let \({ G }\) be a group and \({ g \in G }\) and \begin{icloze}{3}\({ A \subseteq G }\).\end{icloze}
    \[
        \begin{icloze}{2}gA\end{icloze} \overset{\text{def}}= \begin{icloze}{1}\left\{ ga \mid a \in A \right\}.\end{icloze}
    \]
\end{note}

\begin{note}{045b4413d1aa4eacb303dcbc6b2e2ae7}
    Let \({ G }\) be a group and \({ g \in G }\) and \begin{icloze}{3}\({ A \subseteq G }\).\end{icloze}
    \[
        \begin{icloze}{2}Ag\end{icloze} \overset{\text{def}}= \begin{icloze}{1}\left\{ ag \mid a \in A \right\}.\end{icloze}
    \]
\end{note}

\begin{note}{8b0d9b1f260c43d4861523f2d7621dda}
    Let \({ G }\) be a group and \({ g \in G }\) and \begin{icloze}{2}\({ A \subseteq G }\).\end{icloze}
    \[
        \left\lvert gAg^{-1} \right\rvert = \begin{icloze}{1}\left\lvert A \right\rvert.\end{icloze}
    \]
\end{note}

\begin{note}{6b31b050ff2943eb88be9957877f008a}
    Let \({ G }\) be a group and \({ g \in G }\) and \({ A \subseteq G }\).
    \({ \left\lvert gAg^{-1} \right\rvert = \left\lvert A \right\rvert }\).
    What is the key idea in the proof?

    \begin{cloze}{1}
        Conjugation by \({ g }\) is an automorphism and, thus, an injection.
    \end{cloze}
\end{note}

\begin{note}{ebb01bf770dd4afb8a2ebe1666e97128}
    Let \({ H }\) be a group acting on a set \({ A }\).
    The relation \({ \sim }\) defined by
    \[
        a \sim b \quad \iff \quad \begin{icloze}{2}a = h \cdot b\end{icloze} \quad \text{for}\ \begin{icloze}{3}h \in H.\end{icloze}
    \]
    is \begin{icloze}{1}an equivalence relation.\end{icloze}
\end{note}

\begin{note}{d02903a0f72944ec92fb741c94f25b33}
    Let \({ H }\) be a group acting on a set \({ A }\) and \begin{icloze}{4}\({ x \in A }\).\end{icloze}
    \begin{icloze}{3}The equivalence class of \({ x }\)\end{icloze} under the relation \({ \sim }\) defined by
    \[
        a \sim b \quad \iff \begin{icloze}{2}\quad a = h \cdot b \quad \text{for}\ h \in H.\end{icloze}
    \]
    is called \begin{icloze}{1}the orbit of \({ x }\) under the action of \({ H }\).\end{icloze}
\end{note}

\begin{note}{f4668a17736b4297bf9eb6b5d51b3504}
    Let \({ H }\) be a group acting on a set \({ A }\).
    The orbits under the action of \({ H }\) form \begin{icloze}{1}a partition of the set \({ A }\).\end{icloze}
\end{note}

\begin{note}{fa085d4e6fd54c97806a352dc7d0b791}
    Let \({ H }\) be \begin{icloze}{4}a subgroup of a finite group \({ G }\)\end{icloze} and let \({ H }\) act on \({ G }\) by \begin{icloze}{3}left multiplication.\end{icloze}
    Let \begin{icloze}{5}\({ x \in G }\)\end{icloze} and let \({ \mathcal O }\) be \begin{icloze}{2}the orbit of \({ x }\) under the action of \({ H }\).\end{icloze}
    Then
    \[
        \left\lvert \mathcal O \right\rvert = \begin{icloze}{1}\left\lvert H \right\rvert.\end{icloze}
    \]
\end{note}

\begin{note}{ae8a449fc9704c54807b10f7f0d473dd}
    Let \({ H }\) be a subgroup of a finite group \({ G }\) and let \({ H }\) act on \({ G }\) by left multiplication.
    All orbits under the action of \({ H }\) have  cardinality \({ \left\lvert H \right\rvert }\).
    What is the key idea in the proof?

    \begin{cloze}{1}
        The map \({ H \to \mathcal O,\: h \mapsto hx }\) is a bijection.
    \end{cloze}
\end{note}

\begin{note}{aed4b7c3f17d4ace8f7a20f82e0020a6}
    If \({ G }\) is a \begin{icloze}{3}finite\end{icloze} group and \({ H }\) is \begin{icloze}{2}a subgroup of \({ G }\)\end{icloze} then \begin{icloze}{1}\({ \left\lvert H \right\rvert }\) divides \({ \left\lvert G \right\rvert }\).\end{icloze}

    \begin{center}
        \tiny
        <<\begin{icloze}{4}Lagrange's Theorem\end{icloze}>>
    \end{center}
\end{note}

\begin{note}{d3bbb355ed9a4089ba2f0ba426b06df4}
    If \({ G }\) is a finite group and \({ H }\) is a subgroup of \({ G }\) then \({ \left\lvert H \right\rvert }\) divides \({ \left\lvert G \right\rvert }\).
    What is the key idea in the proof?

    \begin{cloze}{1}
        All orbits under the action of \({ H }\) by left multiplication have the same cardinality.
    \end{cloze}
\end{note}

\begin{note}{19a51b72c3b148b4a34462c74a6736eb}
    The group of rigid motions of a tetrahedron is \begin{icloze}{2}isomorphic\end{icloze} to \begin{icloze}{1}a subgroup\end{icloze} of \({ S_4 }\).
\end{note}

\begin{note}{1c21b0d9d8dc4aecb5a34b921fd2d954}
    The group of rigid motions of a tetrahedron is isomorphic to a subgroup of \({ S_4 }\).
    What is the key idea in the proof?

    \begin{cloze}{1}
        Distinct motions induce distinct permutations of vertices.
    \end{cloze}
\end{note}

\begin{note}{7d30497d371f434a89585a39f63f8d0d}
    The group of rigid motions of a cube is \begin{icloze}{2}isomorphic\end{icloze} to \begin{icloze}{1}\({ S_4 }\).\end{icloze}
\end{note}

\begin{note}{5c9601a05fce40d0be8211c48d7fd289}
    The group of rigid motions of a cube is isomorphic to \({ S_4 }\).

    \begin{cloze}{1}
        Every motion induces a unique permutation of the diagonals.
    \end{cloze}
\end{note}

\begin{note}{3c039fe65c9547babc4c107296a9b733}
    The group of rigid motions of an octahedron is \begin{icloze}{2}isomorphic\end{icloze} to \begin{icloze}{1}a subgroup\end{icloze} of \({ S_4 }\).
\end{note}

\begin{note}{beac07114df44718bc29340cd10111bf}
    The group of rigid motions of an octahedron is isomorphic to a subgroup of \({ S_4 }\).
    What is the key idea in the proof?

    \begin{cloze}{1}
        Every motion induces a unique permutation of the set of pairs of opposite faces.
    \end{cloze}
\end{note}

\end{document}
