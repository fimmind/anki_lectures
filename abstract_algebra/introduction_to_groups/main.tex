\documentclass[11pt, a5paper]{article}
\usepackage[width=10cm, top=0.5cm, bottom=2cm]{geometry}

\usepackage[T1,T2A]{fontenc}
\usepackage[utf8]{inputenc}
\usepackage[english,russian]{babel}
\usepackage{libertine}

\usepackage{amsmath}
\usepackage{amssymb}
\usepackage{amsthm}
\usepackage{mathrsfs}
\usepackage{framed}
\usepackage{xcolor}

\setlength{\parindent}{0pt}

% Force \pagebreak for every section
\let\oldsection\section
\renewcommand\section{\pagebreak\oldsection}

\renewcommand{\thesection}{}
\renewcommand{\thesubsection}{Note \arabic{subsection}}
\renewcommand{\thesubsubsection}{}
\renewcommand{\theparagraph}{}

\newenvironment{note}[1]{\goodbreak\par\subsection{\hfill \color{lightgray}\tiny #1}}{}
\newenvironment{cloze}[2][\ldots]{\begin{leftbar}}{\end{leftbar}}
\newenvironment{icloze}[2][\ldots]{%
  \ignorespaces\text{\tiny \color{lightgray}\{\{c#2::}\hspace{0pt}%
}{%
  \hspace{0pt}\text{\tiny\color{lightgray}\}\}}\unskip%
}


\begin{document}
\section{Basic Axioms and Examples}
\begin{note}{cca4f1927b2c4eeaa3123dbcf0680bc0}
    Given a set \({ G }\), \begin{icloze}{2}a binary operation \({ \star }\) on \({ G }\)\end{icloze} is
    \begin{icloze}{1}
        a function
        \[
            \star : G \times G \to G.
        \]
    \end{icloze}
\end{note}

\begin{note}{7732d25ebb1e40dd9696c1c921803c17}
    Given a binary operation \({ \star }\) on a set \({ G }\),
    for any \({ a, b \in G }\) we shall write \begin{icloze}{2}\({ a \star b }\)\end{icloze} for \begin{icloze}{1}\({ \star(a, b) }\).\end{icloze}
\end{note}

\begin{note}{4fc60827250f4af4ab6a669ac7632568}
    A binary operation \({ \star }\) on a set \({ G }\) is \begin{icloze}{2}associative\end{icloze} if
    \begin{icloze}{1}
        for all \({ a, b, c \in G }\) we have
        \[
            a \star (b \star c) = (a \star b) \star c.
        \]
    \end{icloze}
\end{note}

\begin{note}{192d8d86f22349cabcd9f4229fc45290}
    If \({ \star }\) is a binary operation on a set \({ G }\) we say elements \({ a }\) and \({ b }\) of \({ G }\) \begin{icloze}{1}commute\end{icloze} if
    \begin{icloze}{2}
        \[
            a \star b = b \star a.
        \]
    \end{icloze}
\end{note}

\begin{note}{e5cbf512d6a54c91950c65450a07a501}
    A binary operation \({ \star }\) on a set \({ G }\) is \begin{icloze}{2}commutative\end{icloze} if
    \begin{icloze}{1}
        for all \({ a, b \in G }\) we have
        \[
            a \star b = b \star a.
        \]
    \end{icloze}
\end{note}

\begin{note}{36b096eebd7f4264ab071a5fa4eefe13}
    Suppose that \({ \star }\) is a binary operation on a set \({ G }\) and \({ H \subseteq G }\).
    If \begin{icloze}{2}the restriction of \({ \star }\) to \({ H }\) is a binary operation on \({ H }\),\end{icloze} then \({ H }\) is said to be \begin{icloze}{1}closed under \({ \star }\).\end{icloze}
\end{note}

\begin{note}{644b1cd8fa014885ad295ae5c089e5a7}
    \begin{icloze}{3}A group\end{icloze} is \begin{icloze}{2}an ordered pair \({ (G, \star) }\)\end{icloze} where \begin{icloze}{1}\({ G }\) is a set and \({ \star }\) is a binary operation on \({ G }\)\end{icloze} satisfying \begin{icloze}{4}the group axioms.\end{icloze}
\end{note}

\begin{note}{5de4e717b4814adf8aed4f8d9a93322c}
    How many axiom are there in the definition of a group \({ (G, \star) }\)?

    \begin{cloze}{1}
        Three.
    \end{cloze}
\end{note}

\begin{note}{2de690f5008a4b8c8691e36308e44295}
    What is the first axiom from the definition of a group \({ (G, \star) }\)?

    \begin{cloze}{1}
        \({ \star }\) is associative.
    \end{cloze}
\end{note}

\begin{note}{4fcc137e66a048459cc73d6735e4ccea}
    Given a binary operation \({ \star }\) on a set \({ G }\), \begin{icloze}{3}an element \({ e \in G }\)\end{icloze} is called \begin{icloze}{2}an identity of \({ G }\)\end{icloze} if
    \begin{icloze}{1}
        for all \({ a \in G }\) we have
        \[
            a \star e = e \star a = a.
        \]
    \end{icloze}
\end{note}

\begin{note}{a3cd125f152f432082757242096a76ef}
    What is the second axiom from the definition of a group \({ (G, \star) }\)?

    \begin{cloze}{1}
        There exists an identity of \({ G }\).
    \end{cloze}
\end{note}

\begin{note}{5d438f0c3fb24b1a97507e81f868846e}
    Given a binary operation \({ \star }\) on a set \({ G }\) and \({ a \in G }\), \begin{icloze}{3}an element \({ \tilde a \in G }\)\end{icloze} is called \begin{icloze}{2}an inverse of \({ a }\)\end{icloze} if
    \begin{icloze}{1}
        \[
            a \star \tilde a = \tilde a \star a = e.
        \]
    \end{icloze}
\end{note}

\begin{note}{d840b7b910d740f3bea231c74feba51c}
    Given a binary operation \({ \star }\) on a set \({ G }\) and \({ a \in G }\), \begin{icloze}{2}an inverse of \({ a }\)\end{icloze} is usually denoted \begin{icloze}{1}\({ a^{-1} }\).\end{icloze}
\end{note}

\begin{note}{c4c56a11c6f746b3ae287ee386b4e12b}
    What is the third axiom from the definition of a group \({ (G, \star) }\)?

    \begin{cloze}{1}
        For all \({ a \in G }\) there exists \({ a^{-1} }\).
    \end{cloze}
\end{note}

\begin{note}{be05e23d350d4f49a65602b65045f888}
    A group \({ (G, \star) }\) is called \begin{icloze}{2}abelian\end{icloze} if \begin{icloze}{1}\({ \star }\) is commutative.\end{icloze}
\end{note}

\begin{note}{978f23382d594a28a3de168b7f661c30}
    We shall say \({ G }\) is \begin{icloze}{2}a group under \({ \star }\)\end{icloze} if \begin{icloze}{1}\({ (G, \star) }\) is a group.\end{icloze}
\end{note}

\begin{note}{497f01593d7f4ffabb546b455788b354}
    We shall say a set \({ G }\) is \begin{icloze}{2}a group\end{icloze} if \begin{icloze}{1}\({ G }\) is a group under an operation that is clear from the context.\end{icloze}
\end{note}

\begin{note}{61ea2504ca474fe4aae902eb1965576c}
    \({ \mathbb Z, \mathbb Q, \mathbb R }\) and \({ \mathbb C }\) are \begin{icloze}{2}groups\end{icloze} under \begin{icloze}{1}\({ + }\).\end{icloze}
\end{note}

\begin{note}{84b6a231d3934ab3b4f63226549a9589}
    \({ \mathbb Q - \left\{ 0 \right\},\: \mathbb R - \left\{ 0 \right\},\: \mathbb C - \left\{ 0 \right\} }\) are \begin{icloze}{2}groups\end{icloze} under \begin{icloze}{1}\({ \times }\).\end{icloze}
\end{note}

\begin{note}{3051cd354f5040e2bdf0809e005635ed}
    \({ \mathbb Q^{+}, \mathbb R^{+} }\) are \begin{icloze}{2}groups\end{icloze} under \begin{icloze}{1}\({ \times }\).\end{icloze}
\end{note}

\begin{note}{21f924e833cd4e0bbae5f4588dff47b5}
    Is \({ \mathbb Z - \left\{ 0 \right\} }\) a group under \({ \times }\)?

    \begin{cloze}{1}
        No. (There is no inverse.)
    \end{cloze}
\end{note}

\begin{note}{edec2a960f6d43dbb5e19283c28db7bd}
    Let \({ V }\) be a vector space. Then \({ V }\) is \begin{icloze}{2}a group\end{icloze} under \begin{icloze}{1}\({ + }\).\end{icloze}
\end{note}

\begin{note}{47a03e2c688244b1b3a5126fd04a21c7}
    Let \({ n \in \mathbb Z^{+} }\). Then \begin{icloze}{3}\({ \mathbb Z / n\mathbb Z }\)\end{icloze} is \begin{icloze}{2}a group\end{icloze} under \begin{icloze}{1}addition\end{icloze} of residue classes.
\end{note}

\begin{note}{f6a5a40cfee6495dae0d36f7b3288cb2}
    Let \({ n \in \mathbb Z^{+} }\). Then \begin{icloze}{3}\({ \left( \mathbb Z / n\mathbb Z \right)^{\times} }\)\end{icloze} is \begin{icloze}{2}a group\end{icloze} under \begin{icloze}{1}multiplication\end{icloze} of residue classes.
\end{note}

\begin{note}{3e94ca73ca344269bb98d94a22204fd9}
    If \({ (A, \star) }\) and \({ (B, \diamond) }\) are \begin{icloze}{4}groups,\end{icloze} then the group \begin{icloze}{2}\({ A \times B }\),\end{icloze} whose operation is
    \begin{icloze}{1}
        defined componentwise:
        \[
            (a, b)(c, d) = (a \star c, b \diamond d),
        \]
    \end{icloze}
    is called \begin{icloze}{3}the direct product of the two groups.\end{icloze}
\end{note}

\begin{note}{e23d8e577b3948af9b0cadd5df7c9141}
    If \({ (G, \star) }\) is a group, then \begin{icloze}{2}the identity of \({ G }\)\end{icloze} is \begin{icloze}{1}unique.\end{icloze}
\end{note}

\begin{note}{5b5391986e9b49ea9c5f9f73813e9594}
    If \({ (G, \star) }\) is a group, then the identity of \({ G }\) is unique.
    What is the key idea in the proof?

    \begin{cloze}{1}
        Consider the product of two arbitrary identities.
    \end{cloze}
\end{note}

\begin{note}{0989a259fae446c48bb0f6c40394efd0}
    If \({ (G, \star) }\) is a group, then for every \({ a \in G }\), \begin{icloze}{2}\({ a^{-1} }\)\end{icloze} is \begin{icloze}{1}uniquely determined.\end{icloze}
\end{note}

\begin{note}{f0b0a651592c466ba8067beb3b1570b8}
    If \({ (G, \star) }\) is a group, then for every \({ a \in G }\), \({ a^{-1} }\) is uniquely determined.
    What is the key idea in the proof?

    \begin{cloze}{1}
        Multiply an inverse on the right by \({ a \star a^{-1} }\).
    \end{cloze}
\end{note}

\begin{note}{4a6a6806d8874839bb7956d76e384333}
    If \({ (G, \star) }\) is a group and \({ a \in G }\), then
    \[
        (a^{-1})^{-1} = \begin{icloze}{1}a.\end{icloze}
    \]
\end{note}

\begin{note}{9ab0e972d6a24baea99f1577ebf03423}
    If \({ (G, \star) }\) is a group and \({ a, b \in G }\), then
    \[
        \begin{icloze}{2}(a \star b)^{-1}\end{icloze} = \begin{icloze}{1}(b^{-1}) \star (a^{-1}).\end{icloze}
    \]
\end{note}

\begin{note}{69b3db6e70ad4629aa55a855b8df8096}
    If \({ (G, \star) }\) is a group and \({ a_1, \ldots, a_n \in G }\), then
    the value of
    \[
        a_1 \star \cdots \star a_n
    \]
    is \begin{icloze}{2}independent\end{icloze} of \begin{icloze}{1}how the expression is bracketed.\end{icloze}

    \begin{center}
        \tiny
        <<\begin{icloze}{3}The generalized associative law\end{icloze}>>
    \end{center}
\end{note}

\begin{note}{05cc8fd523084650adb46704dde222a7}
    What is the key idea in the proof of the generalized associative law for a group \({ (G, \star) }\)?

    \begin{cloze}{1}
        By induction.
    \end{cloze}
\end{note}

\begin{note}{9ca193d1531c4c49b296732d7ff12fb5}
    Henceforth our abstract groups \({ G }\), \({ H }\), \textit{etc.} will always be written with the operation as \begin{icloze}{1}\({ \cdot }\).\end{icloze}
\end{note}

\begin{note}{7d06acac21c14a628ad1ccb470fe6398}
    Henceforth for an abstract group \({ G }\) (operation \({ \cdot }\))
    an expression \begin{icloze}{2}\({ a \cdot b }\)\end{icloze} will always be written as \begin{icloze}{1}\({ ab }\).\end{icloze}
\end{note}

\begin{note}{0994e6080f3042ad81bc90d1ced0b747}
    Henceforth for an abstract group \({ G }\) (operation \({ \cdot }\))
    we denote \begin{icloze}{2}the identity of \({ G }\)\end{icloze} by \begin{icloze}{1}\({ 1 }\).\end{icloze}
\end{note}

\begin{note}{361c99f13a9b4304868fcdb350b45dbf}
    For any group \({ G }\) and \({ x \in G }\) and \begin{icloze}{3}\({ n \in \mathbb Z^{+} }\)\end{icloze}
    we shall denote by \begin{icloze}{2}\({ x^{n} }\)\end{icloze}
    \begin{icloze}{1}
        the product
        \[
            \underbrace{x x \cdots x}_{\text{\({ n }\) terms}}.
        \]
    \end{icloze}
\end{note}

\begin{note}{5b7f3c41cf0147e2bffc3929ed9ec480}
    For any group \({ G }\) and \({ x \in G }\) and \begin{icloze}{3}\({ n \in \mathbb Z^{+} }\)\end{icloze}
    we shall denote by \begin{icloze}{2}\({ x^{-n} }\)\end{icloze}
    \begin{icloze}{1}
        the product
        \[
            \underbrace{x^{-1} x^{-1} \cdots x^{-1}}_{\text{\({ n }\) terms}}.
        \]
    \end{icloze}
\end{note}

\begin{note}{a7a44229ee0f4a44b11d1410dc0fab0f}
    For any group \({ G }\) and \begin{icloze}{3}\({ x \in G }\),\end{icloze} let \({ x^{\begin{icloze}{2}0\end{icloze}} \overset{\text{def}}= \begin{icloze}{1}1, \text{the identity of \({ G }\)}\end{icloze} }\).
\end{note}

\begin{note}{b1be1b97f53c45fa9451eaa7112ca406}
    Let \({ G }\) be a group and let \({ a, u, v \in G }\).
    Then \({ au = av }\) \begin{icloze}{2}if and only if\end{icloze} \begin{icloze}{1}\({ u = v }\).\end{icloze}

    \begin{center}
        \tiny
        <<\begin{icloze}{3}Cancellation rule\end{icloze}>>
    \end{center}
\end{note}

\begin{note}{ed8673154d544c7b86ac358facc79101}
    For \({ G }\) a group and \({ x \in G }\)
    define \begin{icloze}{2}the order of \({ x }\)\end{icloze} to be
    \begin{icloze}{1}
        the smallest positive integer \({ n }\) such that
        \[
            x^{n} = 1.
        \]
    \end{icloze}
\end{note}

\begin{note}{8c334a6360be4bee8fae7f712ab2c4ee}
    For \({ G }\) a group and \({ x \in G }\), if \begin{icloze}{2}no positive power of \({ x }\) is the identity,\end{icloze} \begin{icloze}{3}the order of \({ x }\)\end{icloze} is defined to be \begin{icloze}{1}infinity.\end{icloze}
\end{note}

\begin{note}{ba4143a322564f8383f6e7d91ca32a75}
    For \({ G }\) a group and \({ x \in G }\), denote \begin{icloze}{2}the order of \({ x }\)\end{icloze} by \begin{icloze}{1}\({ \left\lvert x \right\rvert }\).\end{icloze}
\end{note}

\begin{note}{d7fee5bcbdbd47bcb6f4a2ba086fa2ed}
    For \({ G }\) a group and \({ x \in G }\), if \begin{icloze}{2}the order of \({ x }\) is an integer \({ n }\),\end{icloze} \({ x }\) is said to be \begin{icloze}{1}of order \({ n }\).\end{icloze}
\end{note}

\begin{note}{db12c606699d40e89d499d554bd52b28}
    For \({ G }\) a group and \({ x \in G }\), if \begin{icloze}{2}the order of \({ x }\) is infinite,\end{icloze} \({ x }\) is said to be \begin{icloze}{1}of infinite order.\end{icloze}
\end{note}

\begin{note}{2e514c62ce4e48eb9c6bd3b5de1d7c44}
    An element of a group has order \({ 1 }\) \begin{icloze}{2}if and only if\end{icloze} \begin{icloze}{1}it is the identity.\end{icloze}
\end{note}

\begin{note}{babeb7cf1b394be6a4f8d86e1a099cda}
    Let \({ G = \left\{ g_1, g_2, \ldots, g_n \right\} }\) be \begin{icloze}{4}a finite group\end{icloze} with \begin{icloze}{3}\({ g_1 = 1 }\).\end{icloze}
    The \begin{icloze}{2}multiplication table\end{icloze} or \begin{icloze}{2}group table\end{icloze} of \({ G }\) is
    \begin{icloze}{1}
        the matrix
        \[
            \Big[ g_i g_j \Big] \sim n \times n.
        \]
    \end{icloze}
\end{note}

\begin{note}{f245736b42b44f178f9a1d661bc4a5c7}
    Let \({ G = \begin{icloze}{3}\left\{ x \in \mathbb R \mid x \in [0, 1) \right\}\end{icloze} }\) and for \({ x, y \in G }\) let \({ x \star y }\) be \begin{icloze}{2}the fractional part of \({ x + y }\).\end{icloze}
    Then the group \begin{icloze}{4}\({ (G, \star) }\)\end{icloze} is called \begin{icloze}{1}the real numbers mod \({ 1 }\).\end{icloze}
\end{note}

\begin{note}{3664191737c844f38816547b7acd64c1}
    Let \({ G = \left\{ z \in \begin{icloze}{3}\mathbb C\end{icloze} \mid \begin{icloze}{2}z^{n} = 1\ \text{for some}\ n \in \mathbb Z^{+}\end{icloze} \right\} }\).
    Then the group \({ (G, +) }\) is called \begin{icloze}{1}the group of roots of unity in \({ \mathbb C }\).\end{icloze}
\end{note}

\begin{note}{85c981d4f1564164bb547096829d245b}
    A finite group is \begin{icloze}{3}abelian\end{icloze} \begin{icloze}{2}if and only if\end{icloze} its group table is \begin{icloze}{1}a symmetric matrix.\end{icloze}
\end{note}

\begin{note}{859ef5188ad14b35b58dc9428333e5ad}
    Let \({ G }\) a group and \({ x \in G }\) and \({ a, b \in \begin{icloze}{2}\mathbb Z\end{icloze} }\). Then \({ x^{a + b} = \begin{icloze}{1}x^{a}x^{b}\end{icloze} }\).
\end{note}

\begin{note}{0c6c419e61fc48139ff6afd4a8e28be6}
    Let \({ G }\) a group and \({ x \in G }\). Then \({ \lvert x^{-1} \rvert = \begin{icloze}{1}\lvert x \rvert\end{icloze} }\).
\end{note}

\begin{note}{f221410dc76e4c7881175d62226ecdf4}
    Let \({ G }\) a group and \({ x, g \in G }\).
    Then \({ \lvert g^{-1} x g \rvert = \begin{icloze}{1}\lvert x \rvert\end{icloze} }\).
\end{note}

\begin{note}{9951f3d62ec841df9c6f8cfc07f3c04f}
    Let \({ G }\) a group and \({ a, b \in G }\).
    Then \({ \lvert ba \rvert = \begin{icloze}{1}\lvert ab \rvert\end{icloze} }\).
\end{note}

\begin{note}{b30a94de99fe4c2f91b5417fdbb3e99d}
    Let \({ G }\) a group, \({ x \in G }\), \({ \lvert x \rvert = n < \infty }\) and \({ s \in \mathbb Z }\).
    Then \begin{icloze}{3}\({ x^{s} = 1 }\)\end{icloze} \begin{icloze}{2}if and only if\end{icloze} \begin{icloze}{1}\({ n \mid s }\).\end{icloze}
\end{note}

\begin{note}{5a0539b7021242e2a9a5769a1c156889}
    Let \({ G }\) a group, \({ x \in G }\), \({ \lvert x \rvert = \begin{icloze}{3}n < \infty\end{icloze} }\) and \({ s \in \begin{icloze}{4}\mathbb Z\end{icloze} }\).
    Then
    \[
        \begin{icloze}{2}\lvert x^{s} \rvert\end{icloze} = \begin{icloze}{1}\frac{n}{(n, s)}.\end{icloze}
    \]
\end{note}

\begin{note}{7ec585d338e941d7b465cf11d0f42550}
    Let \({ G }\) a group, \({ x \in G }\), \({ \lvert x \rvert = n < \infty }\) and \({ s \in \mathbb Z }\).
    Then \({ \lvert x^{s} \rvert = \frac{n}{(n, s)} }\).
    What is the key idea in the proof?

    \begin{cloze}{1}
        \({ (x^{s})^{k} = 1 }\) if and only if \({ n \mid sk }\).
    \end{cloze}
\end{note}

\begin{note}{00c58492691e442b9a8c0a5ba21a0c7f}
    Let \({ G }\) a group, \({ a \in G }\). If \({ x^2 = 1 }\) for all \({ x \in G }\) then
    \[
        a^{-1} = \begin{icloze}{1}a.\end{icloze}
    \]
\end{note}

\begin{note}{89199067c84244c094af347afff31c8a}
    Let \({ G }\) a group. If \begin{icloze}{3}\({ a }\) and \({ b }\) are commuting elements of \({ G }\)\end{icloze} then \({ \begin{icloze}{2}(ab)^{n}\end{icloze} = \begin{icloze}{1}a^{n}b^{n}\end{icloze} }\).
\end{note}

\begin{note}{87374145922242e3a5bc43fa952448dc}
    Let \({ G }\) a group. If \({ x^2 = 1 }\) for all \({ x \in G }\) then \({ G }\) is \begin{icloze}{1}abelian.\end{icloze}
\end{note}

\begin{note}{cdf7b8b7731c4e619920d66f7520b423}
    Let \({ G }\) a group. If \({ x^2 = 1 }\) for all \({ x \in G }\) then \({ G }\) is abelian.
    What is the key idea in the proof?

    \begin{cloze}{1}
        \({ 1 = (ab)^2 }\) and multiply by \({ a }\) on the left and by \({ b }\) on the right.
    \end{cloze}
\end{note}

\begin{note}{c48695948e6a4cf69846a629c6b45cb5}
    Let \({ (G, \star) }\) be a group and \begin{icloze}{4}\({ H \subseteq G }\).\end{icloze}
    If \begin{icloze}{2}\({ H }\) is a group under the operation \({ \star }\) restricted to \({ H }\)\end{icloze} then \begin{icloze}{3}\({ H }\)\end{icloze} is called \begin{icloze}{1}a subgroup of \({ G }\).\end{icloze}
\end{note}

\begin{note}{39703ef9887d48e9b763bea0c6519b19}
    Let \({ G }\) a group and \begin{icloze}{3}\({ x \in G }\).\end{icloze} Then \begin{icloze}{2}the subgroup \({ \left\{ x^{n} \mid n \in \mathbb Z \right\} }\) of \({ G }\)\end{icloze} is called \begin{icloze}{1}the cyclic subgroup of \({ G }\) generated by \({ x }\).\end{icloze}
\end{note}

\begin{note}{7d1e317bc7c64c538d09a6fd6c2e2011}
    Let \({ A }\) and \({ B }\) be groups. Then \({ A \times B }\) is \begin{icloze}{3}abelian\end{icloze} \begin{icloze}{2}if and only if\end{icloze} \begin{icloze}{1}both \({ A }\) and \({ B }\) are abelian.\end{icloze}
\end{note}

\begin{note}{c048d6c9ce83411c94e040e5991b3524}
    Let \({ A }\) and \({ B }\) be groups, \({ (a, b) \in A \times B }\).
    Then the order of \({ (a, b) }\) is \begin{icloze}{1}the least common multiple of \({ \left\lvert a \right\rvert }\) and \({ \left\lvert b \right\rvert }\).\end{icloze}
\end{note}

\begin{note}{de71dcf7adc64bcd9b53502c90a0cefa}
    Let \({ A }\) and \({ B }\) be groups, \({ (a, b) \in A \times B }\).
    Then
    \[
        (a, b)^{k} = \begin{icloze}{1}(a^{k}, b^{k})\end{icloze}
    \]
    for all \({ k \in \begin{icloze}{2}\mathbb Z\end{icloze} }\).
\end{note}

\begin{note}{e672cc6907124507a4fd998675844d02}
    Let \({ A }\) and \({ B }\) be groups, \({ (a, b) \in A \times B }\).
    Then the order of \({ (a, b) }\) is the least common multiple of \({ \left\lvert a \right\rvert }\) and \({ \left\lvert b \right\rvert }\).
    What is the key idea in the proof?

    \begin{cloze}{1}
        \({ (a, b)^{k} = (a^{k}, b^{k}) }\).
    \end{cloze}
\end{note}

\begin{note}{e6f9e981e45d4f55a3aafa3eb6d77ef1}
    Any finite group of \begin{icloze}{2}even\end{icloze} order contains an element of order \begin{icloze}{1}\({ 2 }\).\end{icloze}
\end{note}

\begin{note}{d13862b410194166829309d8ea4880a6}
    Any finite group of even order contains an element of order \({ 2 }\).
    What is the key idea in the proof?

    \begin{cloze}{1}
        Show that the set \({ \left\{ g \in G \mid g \neq g^{-1} \right\} }\) has an even number of elements.
    \end{cloze}
\end{note}

\begin{note}{86e37f9ff199460995332631a61f9a00}
    Let \({ G }\) a group, \({ x \in G }\) and \({ \lvert x \rvert = n < \infty }\).
    Then the elements
    \begin{icloze}{2}
        \[
            1, x, x^2, \ldots, x^{n-1}
        \]
    \end{icloze}
    \begin{icloze}{1}are distinct.\end{icloze}
\end{note}

\begin{note}{1bf4e9f92f854544bf96f1364e0064ed}
    Let \({ G }\) a group, \({ x \in G }\) and \({ \lvert x \rvert < \infty }\).
    Then \({ \lvert x \rvert \begin{icloze}{2}\leq\end{icloze} \begin{icloze}{1}\left\lvert G \right\rvert\end{icloze} }\).
\end{note}

\begin{note}{5f4f77e21f2b4052979906547275dfd9}
    Let \({ G }\) a group, \({ x \in G }\) and \({ \lvert x \rvert < \infty }\).
    Then \({ \lvert x \rvert \leq \left\lvert G \right\rvert }\).
    What is the key idea in the proof?

    \begin{cloze}{1}
        The elements \({ 1, x, \ldots, x^{n - 1} }\) are the only powers of \({ x }\).
    \end{cloze}
\end{note}

\begin{note}{4f07acc87f6949e092c057cb5a580c77}
    Let \({ G }\) a group, \({ x \in G }\) and \({ \lvert x \rvert = \infty }\).
    Then the elements
    \begin{icloze}{2}
        \[
            x^{n}, n \in \mathbb Z
        \]
    \end{icloze}
    \begin{icloze}{1}are distinct.\end{icloze}
\end{note}

\section{Dihedral Groups}
\begin{note}{c895ff9d20ae4a2286bb783680b3cee8}
    \begin{icloze}{1}A symmetry\end{icloze} of a regular \({ n }\)-gon is \begin{icloze}{1}any rigid motion of the \({ n }\)-gon\end{icloze} which can be effected by \begin{icloze}{3}taking a copy of the \({ n }\)-gon, moving this copy in any fashion in \({ 3 }\)-space\end{icloze} and then \begin{icloze}{4}placing the copy back on the original \({ n }\)-gon so it exactly covers it.\end{icloze}
\end{note}

\begin{note}{3a08bb223d9241bbb5cd4dae15a4a23d}
    Each symmetry of a regular \({ n }\)-gon can be described uniquely by \begin{icloze}{1}the corresponding permutation of \({ \left\{ 1, 2, \ldots, n \right\} }\),\end{icloze} representing \begin{icloze}{2}the permutation of the vertices.\end{icloze}
\end{note}

\begin{note}{7a77331a22e144ceaf6ca7c1b475a99a}
    Given \begin{icloze}{3}\({ n \in \mathbb Z^{+} }\) and \({ n \geq 3 }\),\end{icloze} let \begin{icloze}{2}\({ D_{2n} }\)\end{icloze} be \begin{icloze}{1}the set of symmetries of a regular \({ n }\)-gon.\end{icloze}
\end{note}

\begin{note}{a81873dbe4e6432f93bb1d8c3c5978f1}
   Given \({ n \in \mathbb Z^{+} }\), \({ n \geq 3 }\) and \({ s, t \in D_{2n} }\), \begin{icloze}{2}the product \({ st }\)\end{icloze} is defined to be \begin{icloze}{1}the symmetry obtained by first applying \({ t }\) then \({ s }\) to the \({ n }\)-gon.\end{icloze}
\end{note}

\begin{note}{c0d6e6d3d60b45058b7957002e045102}
   Given \({ n \in \mathbb Z^{+} }\) and \({ n \geq 3 }\),
   \begin{icloze}{2}the group \({ D_{2n} }\)\end{icloze} is called \begin{icloze}{1}the dihedral group of order \({ 2n }\).\end{icloze}
\end{note}

\begin{note}{1457d2279c1d432a9d371d8797d9b621}
   Given \({ n \in \mathbb Z^{+} }\) and \({ n \geq 3 }\),
   \[
       \lvert D_{2n} \rvert = \begin{icloze}{1}2n.\end{icloze}
   \]
\end{note}

\begin{note}{af77787eb9d94bae94d7df59b0415212}
   Given \({ n \in \mathbb Z^{+} }\) and \({ n \geq 3 }\), \({ \lvert D_{2n} \rvert = 2n. }\)
   What is the key idea in the proof?

   \begin{cloze}{1}
       Every symmetry is uniquely determined by how it affects some two adjacent vertices.
   \end{cloze}
\end{note}

\begin{note}{1a3443407b8641c5adc691e47eef2f1e}
    For convenience, the regular \({ n }\)-gon viewed in \({ D_{2n} }\) is fixed \begin{icloze}{1}centered at the origin.\end{icloze}
\end{note}

\begin{note}{b85d695a3fff47fbafbff07b7a341cd0}
    For convenience, the vertices of the regular \({ n }\)-gon viewed in \({ D_{2n} }\) are labeled \begin{icloze}{1}consecutively from \({ 1 }\) to \({ n }\) in a clockwise manner.\end{icloze}
\end{note}

\begin{note}{8005824717e34f4a8e154dfa84d25f17}
    In the context of the \({ D_{2n} }\) group, let \begin{icloze}{2}\({ r }\)\end{icloze} be \begin{icloze}{1}the rotation clockwise about the origin through \({ 2\pi / n }\) radian.\end{icloze}
\end{note}

\begin{note}{8439aae412044be9bf8f6c59334cd570}
    In the context of the \({ D_{2n} }\) group, let \begin{icloze}{2}\({ s }\)\end{icloze} be \begin{icloze}{1}the reflection about the line of symmetry through vertex \({ 1 }\) and the origin.\end{icloze}
\end{note}

\begin{note}{d46303ae65e74f2e8f610b873f4e559b}
    In the context of the \({ D_{2n} }\) group,
    is it possible that \({ s = r^{i} }\) for some \({ i }\)?

    \begin{cloze}{1}
        No.
    \end{cloze}
\end{note}

\begin{note}{5aaf131bef484c89b455ce9f5b4a2eae}
    In the context of the \({ D_{2n} }\) group,
    is it possible that \({ s r^{i} = s r ^{j} }\) for some \({ i \not \equiv j \pmod n }\)?

    \begin{cloze}{1}
        No.
    \end{cloze}
\end{note}

\begin{note}{79c14b3dba52416f934c9d820acb0be7}
    Each element of \({ D_{2n} }\) can be written \begin{icloze}{2}uniquely\end{icloze} in the form \begin{icloze}{1}\({ s^{k}r^{i} }\) for some \({ k = 0 }\) or \({ 1 }\) and \({ 0 \leq i \leq n - 1 }\).\end{icloze}
\end{note}

\begin{note}{f3a7147f62d84c53b1eec4f7da081eba}
    In the context of the \({ D_{2n} }\) group,
    \[
        r^{i}s = \begin{icloze}{1}s r^{-i},\end{icloze}\ \text{for \begin{icloze}{2}all \({ 0 \leq i \leq n }\)\end{icloze}}.
    \]
\end{note}

\begin{note}{2600f25fd1ec408b8e47e341dc6cdb64}
    In the context of the \({ D_{2n} }\) group,
    \[
        r^{i}s = s r^{-i},\ \text{for all \({ 0 \leq i \leq n }\)}.
    \]
    What is the key idea in the proof?

    \begin{cloze}{1}
        \({ rs = s r^{-1} }\) and by induction.
    \end{cloze}
\end{note}

\begin{note}{f56559b6eae841cea409f8438221c1b2}
    \begin{icloze}{3}A subset \({ S }\) of elements\end{icloze} of a group \({ G }\) with the property that \begin{icloze}{1}every element of \({ G }\) can be written as a (finite) product of elements of \({ S }\) and their inverses\end{icloze} is called \begin{icloze}{2}a set of generators of \({ G }\).\end{icloze}
\end{note}

\begin{note}{d72e348121f94214980378f08a5e45a3}
    If \({ S }\) is \begin{icloze}{2}a set of generators\end{icloze} of a group \({ G }\), we shall write
    \begin{icloze}{1}
        \[
            G = \langle S \rangle.
        \]
    \end{icloze}
\end{note}

\begin{note}{7bc8f288fd5d45e0a89eb59abdc95810}
    If \({ S }\) is \begin{icloze}{2}a set of generators\end{icloze} of a group \({ G }\), we shall say \({ G }\) is \begin{icloze}{1}generated by \({ S }\).\end{icloze}
\end{note}

\begin{note}{cb072c7b416e4fbf8e3cf95f496f4083}
    In terms of generators, the group \({ D_{2n} = \begin{icloze}{1}\langle r, s \rangle\end{icloze} }\).
\end{note}

\begin{note}{4fd6980a252a486980db01306accceef}
    In a \begin{icloze}{2}finite\end{icloze} group \({ G }\) a set \({ S }\) generates \({ G }\) if every element of \({ G }\) is \begin{icloze}{1}a finite product of elements of \({ S }\).\end{icloze}
\end{note}

\begin{note}{90b6154b7aaa48398ddeeb91083d71ac}
    In the \({ D_{2n}  }\) group, the relations \({ r^{n} = 1 }\), \({ s^2 = 1 }\) and \({ rs = s r^{-1} }\) have the additional property that \begin{icloze}{1}any any other relation between elements of the group may be derived from these three.\end{icloze}
\end{note}

\begin{note}{4681975ee07c4032a3ced2de0ccfa631}
    In the \({ D_{2n}  }\) group, the relations \({ r^{n} = 1 }\), \({ s^2 = 1 }\) and \({ rs = s r^{-1} }\) have the additional property that any any other relation between elements of the group may be derived from these three.
    What is the key idea in the proof?

    \begin{cloze}{1}
        We can determine exactly when two group elements are equal by using only these three relations.
    \end{cloze}
\end{note}

\begin{note}{b0bcc70704c64cccba8d832a4540749b}
    \begin{icloze}{1}Any equations\end{icloze} in a general group \({ G }\) that \begin{icloze}{1}the generators satisfy\end{icloze} are called \begin{icloze}{2}relations in \({ G }\).\end{icloze}
\end{note}

\begin{note}{b8f9a5669c634d39ac14a6115f6b142d}
    Let \({ G }\) be a group. If \begin{icloze}{4}\({ G }\) is generated by a subset \({ S }\)\end{icloze} and \begin{icloze}{3}there is some collection of relations such that any relation among the elements of \({ S }\) can be deduced from these,\end{icloze} we shall call \begin{icloze}{2}these generators and relations\end{icloze} \begin{icloze}{1}a presentation of \({ G }\).\end{icloze}
\end{note}

\begin{note}{265ab5c6f292430c8ecc6ab97f25c8a8}
    Let \({ G }\) be a group. If \begin{icloze}{4}a subset \({ S }\)\end{icloze} and \begin{icloze}{3}a collection of relations \({ R_1, \ldots, R_{m} }\)\end{icloze} form \begin{icloze}{2}a presentation of \({ G }\),\end{icloze} we shall write
    \begin{icloze}{1}
        \[
            G = \langle S \mid R_1, \ldots, R_{m} \rangle.
        \]
    \end{icloze}
\end{note}

\begin{note}{b8acfa74c7df4502a3b76c59342afbac}
    One presentation for \begin{icloze}{3}the dihedral group \({ D_{2n} }\)\end{icloze} is
    \[
        \begin{icloze}{3}D_{2n}\end{icloze} = \langle \begin{icloze}{2}r, s\end{icloze} \mid \begin{icloze}{1}r^{n} = s^2 = 1,\: rs = s r^{-1}\end{icloze} \rangle.
    \]
\end{note}

\end{document}
