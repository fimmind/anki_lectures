\documentclass[11pt, a5paper]{article}
\usepackage[width=10cm, top=0.5cm, bottom=2cm]{geometry}

\usepackage[T1,T2A]{fontenc}
\usepackage[utf8]{inputenc}
\usepackage[english,russian]{babel}
\usepackage{libertine}

\usepackage{amsmath}
\usepackage{amssymb}
\usepackage{amsthm}
\usepackage{mathrsfs}
\usepackage{framed}
\usepackage{xcolor}

\setlength{\parindent}{0pt}

% Force \pagebreak for every section
\let\oldsection\section
\renewcommand\section{\pagebreak\oldsection}

\renewcommand{\thesection}{}
\renewcommand{\thesubsection}{Note \arabic{subsection}}
\renewcommand{\thesubsubsection}{}
\renewcommand{\theparagraph}{}

\newenvironment{note}[1]{\goodbreak\par\subsection{\hfill \color{lightgray}\tiny #1}}{}
\newenvironment{cloze}[2][\ldots]{\begin{leftbar}}{\end{leftbar}}
\newenvironment{icloze}[2][\ldots]{%
  \text{\tiny \color{lightgray}\{\{c#2::}\hspace{0pt}\ignorespaces%
}{%
  \unskip\hspace{0pt}\text{\tiny\color{lightgray}\}\}}%
}


\begin{document}
\section{Basics}
\begin{note}{21e64c2f0430467f8a36481045e172b3}
    \begin{icloze}{2}\({ \mathbb Z^{+} }\), \({ \mathbb Q^{+} }\) and \({ \mathbb R^{+} }\)\end{icloze} denote \begin{icloze}{1}the positive (nonzero) elements in \({ \mathbb Z }\), \({ \mathbb Q }\) and \({ \mathbb R }\), respectively.\end{icloze}
\end{note}

\begin{note}{ad32571de5d04cafb2a7d6a27aee4b14}
    Given a function \({ f : A \to B }\), \begin{icloze}{1}the set \({ B }\)\end{icloze} is called the codomain of \({ f }\).
\end{note}

\begin{note}{0cd492ad876a4dfbbb22c9210039fcc1}
    Given a function \({ f : A \to B }\) and \begin{icloze}{3}a \({ b \in B }\),\end{icloze} \begin{icloze}{2}the preimage of \({ \left\{ b \right\} }\) under \({ f }\)\end{icloze} is called \begin{icloze}{1}the fiber of \({ f }\) over \({ b }\).\end{icloze}
\end{note}

\begin{note}{b2a02c66209140a591b43dede69ffbf1}
    If \({ f : A \to B }\) and \({ g : B \to C }\), then the \begin{icloze}{1}composite map\end{icloze}
    \[
        g \circ f : A \to C
    \]
    is defined by
    \[
        (g \circ f)(a) = g(f(a)).
    \]
\end{note}

\begin{note}{1b2bf2fe79dc4063a151a960f45698d9}
    A function \({ f : A \to B }\) \begin{icloze}{3}has a left inverse\end{icloze} if there is a function \({ g : \begin{icloze}{2}B \to A\end{icloze} }\), such that
    \begin{icloze}{1}
        \[
            g \circ f = id_A.
        \]
    \end{icloze}
\end{note}

\begin{note}{d9a63bd7866e44ab83172cf9189e9b9a}
    A function \({ f : A \to B }\) \begin{icloze}{3}has a right inverse\end{icloze} if there is a function \({ g : \begin{icloze}{2}B \to A\end{icloze} }\), such that
    \begin{icloze}{1}
        \[
            f \circ g = id_B.
        \]
    \end{icloze}
\end{note}

\begin{note}{8098015b57774529a721663e39eabb18}
    A map \({ f }\) is \begin{icloze}{1}injective\end{icloze} if and only if \({ f }\) has a \begin{icloze}{2}left\end{icloze} inverse.
\end{note}

\begin{note}{205100b0fd6447a9bcc94e4d7711a606}
    A map \({ f }\) is \begin{icloze}{1}surjective\end{icloze} if and only if \({ f }\) has a \begin{icloze}{2}right\end{icloze} inverse.
\end{note}

\begin{note}{8e4ddf27550f4aa9a1daf0b67cd2f7e4}
    A \begin{icloze}{2}permutation\end{icloze} of a set \({ A }\) is \begin{icloze}{1}a bijection from \({ A }\) to itself.\end{icloze}
\end{note}

\begin{note}{1feef80fbcdd48618084ce93c88df83b}
    If \({ A \subseteq B  }\) and \({ f : B \to C }\), \begin{icloze}{2}the restriction of \({ f }\) to \({ A }\)\end{icloze} is denoted \begin{icloze}{1}\({ f|_{A} }\).\end{icloze}
\end{note}

\begin{note}{01a5a3b0e5f24f6782e689090b17c437}
    If \({ A \subseteq B }\) and \({ g : A \to C }\) and there is a function \({ f : B \to C }\) such that \begin{icloze}{2}\({ f|_{A} = g }\),\end{icloze} we shall say \({ f }\) is \begin{icloze}{1}an extension of \({ g }\) to \({ B }\).\end{icloze}
\end{note}

\begin{note}{6ca7e478954c4c898718ce116219822f}
    \begin{icloze}{2}A binary relation on a set \({ A }\)\end{icloze} is \begin{icloze}{1}a subset \({ R }\) of \({ A \times A }\).\end{icloze}
\end{note}

\begin{note}{50bb82cd97cb40bf8621065845545d18}
    Let \({ R }\) be a binary relation on a set \({ A }\).
    We write \begin{icloze}{2}\({ a \sim b }\)\end{icloze} if \begin{icloze}{1}\({ (a, b) \in R }\).\end{icloze}
\end{note}

\begin{note}{65287096376a47f399a0048c0d8092d0}
    A binary relation \({ R }\) on \({ A }\) is said to be \begin{icloze}{2}reflexive\end{icloze} if
    \begin{icloze}{1}
        \begin{center}
            \({ a \sim a }\), for all \({ a \in A }\).
        \end{center}
    \end{icloze}
\end{note}

\begin{note}{71b961a1f8f347dcbf7b9c7c8dee303c}
    A binary relation \({ R }\) on \({ A }\) is said to be \begin{icloze}{2}symmetric\end{icloze} if
    \begin{icloze}{1}
        \begin{center}
            \({ a \sim b }\) implies \({ b \sim a }\) for all \({ a, b \in A }\).
        \end{center}
    \end{icloze}
\end{note}

\begin{note}{40964931d9594b2997437cc9e3e150cc}
    A binary relation \({ R }\) on \({ A }\) is said to be \begin{icloze}{2}transitive\end{icloze} if
    \begin{icloze}{1}
        \begin{center}
            \({ a \sim b }\) and \({ b \sim c }\) implies \({ a \sim c }\) for all \({ a, b c \in A }\).
        \end{center}
    \end{icloze}
\end{note}

\begin{note}{54a959a8e36045c1aea2d838ce8998b8}
    A binary relation is \begin{icloze}{2}an equivalence relation\end{icloze} if \begin{icloze}{1}it is reflexive, symmetric and transitive.\end{icloze}
\end{note}

\begin{note}{7c28d643ddd74509b88cfe2f75e6d743}
    If \({ \sim }\) defines an \begin{icloze}{3}equivalence\end{icloze} relation on \({ A }\), then \begin{icloze}{2}the equivalence class\end{icloze} of \({ a \in A }\) is defined to be
    \begin{icloze}{1}
        \[
            \left\{ x \in A \mid x \sim a \right\}.
        \]
    \end{icloze}
\end{note}

\begin{note}{323fae73cd4b47ddb8c19cb515ffd4cf}
    If \({ C }\) is an equivalence class, \begin{icloze}{2}any element of \({ C }\)\end{icloze} is called \begin{icloze}{1}a representative of the class \({ C }\).\end{icloze}
\end{note}

\begin{note}{3a597e1d5c48420490d792b972a38fe6}
    \begin{icloze}{2}A partition of a set \({ A }\)\end{icloze} is \begin{icloze}{3}any collection \({ \left\{ A_i \mid i \in I \right\} }\) of nonempty subsets of \({ A }\)\end{icloze} such that \begin{icloze}{1}\({ A }\) is the disjoint union of all \({ A_i }\).\end{icloze}
\end{note}

\begin{note}{c2216701429649b7a262afdd5c85a72d}
    If \({ \sim }\) defines an equivalence relation on \({ A }\) then \begin{icloze}{2}the set of equivalence classes of \({ \sim }\)\end{icloze} form \begin{icloze}{1}a partition of \({ A }\).\end{icloze}
\end{note}

\section{Properties of the Integers}
\begin{note}{f535d29c343f494fa35bccefce9d6988}
    Let \({ a, b \in \mathbb Z }\).
    We write \begin{icloze}{2}\({ a \mid b }\)\end{icloze} if \begin{icloze}{1}\({ a }\) divides \({ b }\).\end{icloze}
\end{note}

\begin{note}{96293ae3b76348d8ba9f0b02c8b49a94}
    Let \({ a, b \in \mathbb Z }\) with \({ a \neq 0 }\).
    We write \begin{icloze}{2}\({ a \nmid b }\)\end{icloze} if \begin{icloze}{1}\({ a }\) does not divide \({ b }\).\end{icloze}
\end{note}

\begin{note}{533403fe830341a39cee216314b861e8}
    Let \({ a, b \in \begin{icloze}{3}\mathbb Z - \left\{ 0 \right\}\end{icloze} }\).
    \begin{icloze}{2}The greatest common divisor of \({ a }\) and \({ b }\)\end{icloze} is denoted by \begin{icloze}{1}\({ (a, b) }\).\end{icloze}
\end{note}

\begin{note}{20b204b896884b6b9d07ca3023b7cf4a}
    Let \({ a, b \in \begin{icloze}{3}\mathbb Z - \left\{ 0 \right\}\end{icloze} }\).
    If \begin{icloze}{2}\({ (a, b) = 1 }\),\end{icloze} we say that \({ a }\) and \({ b }\) are \begin{icloze}{1}relatively prime.\end{icloze}
\end{note}

\begin{note}{69adfe8820204997a5aa44c50b353a40}
    If \({ a, b \in \mathbb Z - \left\{ 0 \right\} }\), then there exists unique \({ q, r \in \mathbb Z }\) such that
    \begin{center}
        \({ a = q b + r }\) and \({ 0 \leqslant r < \left\lvert b \right\rvert }\),
    \end{center}
    where \({ q }\) is \begin{icloze}{1}the quotient\end{icloze} and \({ r }\) \begin{icloze}{1}the remainder.\end{icloze}

    \begin{center}
        \tiny
        <<\begin{icloze}{2}Division Algorithm\end{icloze}>>
    \end{center}
\end{note}

\begin{note}{6267be99c4884a09b1282d041ac05e18}
    If \({ a, b \in \mathbb Z - \left\{ 0 \right\} }\), then there exist \({ x, y \in \begin{icloze}{3}\mathbb Z\end{icloze} }\) such that
    \[
        \begin{icloze}{2}(a, b)\end{icloze} = \begin{icloze}{1}xa + yb.\end{icloze}
    \]
\end{note}

\begin{note}{e30ea564f2ce479391e71512867aea51}
    If \({ p }\) is prime and \({ p \mid ab }\), for some \({ a, b \in \mathbb Z }\), then
    \begin{icloze}{1}
        \begin{center}
            either \({ p \mid a }\) or \({ p \mid b }\).
        \end{center}
    \end{icloze}
\end{note}

\begin{note}{3cc931ead0ec4cddae21114d84f1de0c}
    \begin{icloze}{3}The Euler \({ \varphi }\)-function\end{icloze} is defined as follows: for \({ n \in \begin{icloze}{2}\mathbb Z^{+}\end{icloze} }\) let \({ \varphi(n) }\) be \begin{icloze}{1}the number of positive integers \({ a \leq n }\) with \({ a }\) relatively prime to \({ n }\).\end{icloze}
\end{note}

\begin{note}{03f37e11eb9d40d29ca92031ac27d9ed}
    Let \({ \varphi }\) stand for the Euler \({ \varphi }\)-function.
    If \({ p }\) is \begin{icloze}{3}prime\end{icloze} and \({ a \geq 1 }\), then
    \[
        \begin{icloze}{2}\varphi(p^{a})\end{icloze} = \begin{icloze}{1}p^{a} - p^{a - 1}.\end{icloze}
    \]
\end{note}

\begin{note}{7dc766a783c04a309951678711bd8317}
    Let \({ \varphi }\) stand for the Euler \({ \varphi }\)-function.
    Then
    \begin{center}
        \begin{icloze}{1}\({ \varphi(ab) = \varphi(a) \varphi(b) }\)\end{icloze} \quad if \begin{icloze}{2}\({ (a, b) = 1 }\).\end{icloze}
    \end{center}
\end{note}

\end{document}
